\documentclass[]{article}
\usepackage{lmodern}
\usepackage{amssymb,amsmath}
\usepackage{ifxetex,ifluatex}
\usepackage{fixltx2e} % provides \textsubscript
\ifnum 0\ifxetex 1\fi\ifluatex 1\fi=0 % if pdftex
  \usepackage[T1]{fontenc}
  \usepackage[utf8]{inputenc}
\else % if luatex or xelatex
  \ifxetex
    \usepackage{mathspec}
    \usepackage{xltxtra,xunicode}
  \else
    \usepackage{fontspec}
  \fi
  \defaultfontfeatures{Mapping=tex-text,Scale=MatchLowercase}
  \newcommand{\euro}{€}
\fi
% use upquote if available, for straight quotes in verbatim environments
\IfFileExists{upquote.sty}{\usepackage{upquote}}{}
% use microtype if available
\IfFileExists{microtype.sty}{\usepackage{microtype}}{}
\usepackage[margin=1in]{geometry}
\ifxetex
  \usepackage[setpagesize=false, % page size defined by xetex
              unicode=false, % unicode breaks when used with xetex
              xetex]{hyperref}
\else
  \usepackage[unicode=true]{hyperref}
\fi
\hypersetup{breaklinks=true,
            bookmarks=true,
            pdfauthor={},
            pdftitle={},
            colorlinks=true,
            citecolor=blue,
            urlcolor=blue,
            linkcolor=magenta,
            pdfborder={0 0 0}}
\urlstyle{same}  % don't use monospace font for urls
% Make links footnotes instead of hotlinks:
\renewcommand{\href}[2]{#2\footnote{\url{#1}}}
\setlength{\parindent}{0pt}
\setlength{\parskip}{6pt plus 2pt minus 1pt}
\setlength{\emergencystretch}{3em}  % prevent overfull lines
\setcounter{secnumdepth}{0}


\begin{document}

\% Streaming Huffman Compression in Haskell (Part 2: Binary and Searches) \%
Justin Le \% April 11, 2014

\emph{Originally posted on
\textbf{\href{https://blog.jle.im/entry/streaming-huffman-compression-in-haskell-part-2-binary.html}{in
Code}}.}

Continuing on this series of beginner/intermediate projects for newer Haskell
users, let's look back at our Huffman encoding project.

In our
\href{http://blog.jle.im/entry/streaming-huffman-compression-in-haskell-part-1-trees}{last
post} we went over two types of binary trees implemented as algebraic data
structures in Haskell, and also a scheme for assembling a Huffman encoding tree
using the State monad.

Now let's look at serializing and unserializing our prefix trees for easy
storage, and then at actually using them to encode and decode!

\section{Binary}

There are a couple of serialization libraries in Haskell; the dominant one is
\href{http://hackage.haskell.org/package/binary}{binary}, but
\href{http://hackage.haskell.org/package/cereal}{cereal} is also not uncommon.
The two diverge on several design points, and you can read up on them in the
documentation for \emph{cereal}. We'll be using \emph{binary} for the this
tutorial; among many reasons, for its easy integration with the \emph{pipes}
library we will be working with later.

\subsection{The Easy Way}

So let's make \texttt{PreTree} serialize/unserializable.

The easy way is to enable the \texttt{DeriveGeneric} language extension on GHC,
use \texttt{deriving\ (Generic)} when we define our \texttt{PreTree}, and then:

\textasciitilde{}\textasciitilde{}\textasciitilde{}haskell instance Binary a
=\textgreater{} Binary (PreTree a)
\textasciitilde{}\textasciitilde{}\textasciitilde{}

And...that's it! We just auto-generated functions to serialize and deserialize
our \texttt{PreTree}s (if what they contain is itself serializable).

In real life, we would do this. However, for the sake of learning, let's dig a
bit more into the \texttt{Binary} typeclass.

\subsection{\texorpdfstring{The \emph{Other} Easy Way}{The Other Easy Way}}

So the big crux of \emph{binary} is the \texttt{Binary} typeclass:

\textasciitilde{}\textasciitilde{}\textasciitilde{}haskell class Binary t where
put :: t -\textgreater{} Put get :: Get t
\textasciitilde{}\textasciitilde{}\textasciitilde{}

where \texttt{Put} and \texttt{Get} are sort of "instruction objects for
putting/getting binary". \texttt{Get} is a monad, and \texttt{Put} is a wrapped
\texttt{PutM}, which is a writer monad. (To be more specific, \texttt{Put} is
\texttt{PutM\ ()}, because the final action has no result and only "writes")

So \texttt{Binary} things are things that you can serialize (with the
intructions in \texttt{put}) and deserialize (with the instructions in
\texttt{get}).

Luckily, because of Haskell's great composition tools, assembling these
instructions by hand are easy peasy!

\subsubsection{Put}

Let's define our own custom \texttt{Put} for our \texttt{PreTree}s:

\textasciitilde{}\textasciitilde{}\textasciitilde{}haskell -\/- source:
https://github.com/mstksg/inCode/tree/master/code-samples/huffman/PreTree.hs\#L69-L76
-\/- interactive: https://www.fpcomplete.com/user/jle/huffman-encoding putPT ::
Binary a =\textgreater{} PreTree a -\textgreater{} Put putPT (PTLeaf x) = do put
True -\/- signify we have a leaf put x putPT (PTNode pt1 pt2) = do put False
-\/- signify we have a node put pt1 put pt2
\textasciitilde{}\textasciitilde{}\textasciitilde{}

This all should be fairly readable and self-explanatory.

\begin{itemize}
\item
  "To put a \texttt{PTLeaf\ x}, first put a flag saying you have a leaf, then
  put the value of \texttt{x}."
\item
  "To put a \texttt{PTNode\ pt1\ pt2}, first put a flag saying you have a node,
  then put both trees."
\end{itemize}

Due to how monads and pattern matching work, the whole thing is pretty
expressive, pleasant to read, and satisfying to write.

The only slightly annoying thing is that we subject ourselves to
\href{http://existentialtype.wordpress.com/2011/03/15/boolean-blindness/}{boolean
blindness} by using \texttt{True} or \texttt{False}; we have to keep track of
what means what. Alternatively, we can create our own binary data types,
\texttt{data\ PTType\ =\ IsNode\ \textbar{}\ IsLeaf}, and \texttt{put}
\emph{that}, instead...but in this case it might not be so bad to live with
boolean blindness for now.

\subsubsection{Get}

Now let's define our own custom \texttt{Get}:

\textasciitilde{}\textasciitilde{}\textasciitilde{}haskell -\/- source:
https://github.com/mstksg/inCode/tree/master/code-samples/huffman/PreTree.hs\#L79-L84
-\/- interactive: https://www.fpcomplete.com/user/jle/huffman-encoding getPT ::
Binary a =\textgreater{} Get (PreTree a) getPT = do isLeaf \textless{}- get if
isLeaf then PTLeaf \textless{}\$\textgreater{} get else PTNode
\textless{}\$\textgreater{} get \textless{}*\textgreater{} get
\textasciitilde{}\textasciitilde{}\textasciitilde{}

This also shouldn't be too bad!

\begin{itemize}
\tightlist
\item
  "Get" the boolean flag, to tell you if you have a leaf or a node.
\item
  If it's a leaf, then \texttt{get} the data inside the leaf, and wrap it in a
  \texttt{PTLeaf}.
\item
  If it's not, \texttt{get} the next two \texttt{PreTree\ a}'s, and put them
  both in a \texttt{PTNode}.
\end{itemize}

The neat thing here is that \texttt{get} is polymorphic in its return type. We
know that the first \texttt{get} expects a \texttt{Bool}, so it knows to parse a
\texttt{Bool}. We know that the second \texttt{get} expects an \texttt{a}, so it
knows to parse an \texttt{a}. We know that the final two \texttt{get}s both
expect \texttt{PreTree\ a}'s, so it nows what to parse for that too.

Hooray for type inference!

If you're not familiar with the
\texttt{f\ \textless{}\$\textgreater{}\ x\ \textless{}*\textgreater{}\ y} idiom,
you can consider it to be the same thing as \texttt{f\ x\ y}, except that
\texttt{x} and \texttt{y} are "inside" things:

\textasciitilde{}\textasciitilde{}\textasciitilde{}haskell ghci\textgreater{}
(+) 1 4 5 ghci\textgreater{} (+) \textless{}\$\textgreater{} Just 1
\textless{}*\textgreater{} Just 4 Just 5
\textasciitilde{}\textasciitilde{}\textasciitilde{}

Where \texttt{(\textless{}\$\textgreater{})} and
\texttt{(\textless{}*\textgreater{})} come from \texttt{Control.Applicative}. We
call this style "applicative style", in the biz.

\subsubsection{Wrapping it up}

And finally, to tie it all together:

\textasciitilde{}\textasciitilde{}\textasciitilde{}haskell -\/- source:
https://github.com/mstksg/inCode/tree/master/code-samples/huffman/PreTree.hs\#L36-L38
-\/- interactive: https://www.fpcomplete.com/user/jle/huffman-encoding instance
Binary a =\textgreater{} Binary (PreTree a) where put = putPT get = getPT
\textasciitilde{}\textasciitilde{}\textasciitilde{}

\subsection{Testing it out}

However way we decide to write our \texttt{Binary} instance, let's test it all
out.

\textasciitilde{}\textasciitilde{}\textasciitilde{}haskell ghci\textgreater{}
let (Just pt) = runBuildTree "hello world" ghci\textgreater{} let encoded =
encode pt ghci\textgreater{} :t encoded encoded :: ByteString -\/- a string of
bytes ghci\textgreater{} let decoded = decode encoded :: PreTree Char
ghci\textgreater{} decoded PTNode (PTNode (PTNode (PTLeaf 'h') (PTLeaf 'e') )
(PTNode (PTLeaf 'w') (PTLeaf 'r') ) ) (PTNode (PTLeaf 'l') (PTNode (PTNode
(PTLeaf 'd') (PTLeaf ' ') ) (PTLeaf 'o') ) ) ghci\textgreater{} decoded == t
True \textasciitilde{}\textasciitilde{}\textasciitilde{}

Neat! We can also write it to a file and re-read:

\textasciitilde{}\textasciitilde{}\textasciitilde{}haskell ghci\textgreater{}
encodeFile "test.dat" t ghci\textgreater{} t' \textless{}- decodeFile "test.dat"
:: IO (PreTree Char) ghci\textgreater{} t' PTNode (PTNode (PTNode (PTLeaf 'h')
(PTLeaf 'e') ) (PTNode (PTLeaf 'w') (PTLeaf 'r') ) ) (PTNode (PTLeaf 'l')
(PTNode (PTNode (PTLeaf 'd') (PTLeaf ' ') ) (PTLeaf 'o') ) ) ghci\textgreater{}
t' == t True \textasciitilde{}\textasciitilde{}\textasciitilde{}

And this looks like it works pretty well!

\section{Encoding}

Now that we've got that out of the way, let's work on actually encoding and
decoding.

So, basically, we encode a character in a huffman tree by path you take to reach
the character.

Let's represent this path as a list of \texttt{Direction}s:

\textasciitilde{}\textasciitilde{}\textasciitilde{}haskell -\/- source:
https://github.com/mstksg/inCode/tree/master/code-samples/huffman/PreTree.hs\#L25-L30
-\/- interactive: https://www.fpcomplete.com/user/jle/huffman-encoding data
Direction = DLeft \textbar{} DRight deriving (Show, Eq, Generic)

type Encoding = {[}Direction{]}
\textasciitilde{}\textasciitilde{}\textasciitilde{}

Eventually, an \texttt{Encoding} will be turned into a \texttt{ByteString}, with
\texttt{DLeft} representing the 0 bit and \texttt{DRight} representing the 1
bit. But we keep them as their own data types now because everyone hates
\href{http://existentialtype.wordpress.com/2011/03/15/boolean-blindness/}{boolean
blindness}. Instead of keeping a \texttt{True} or \texttt{False}, we keep data
types that actually carry semantic meaning :) And we can't do silly things like
use a boolean as a direction...what the heck? Why would you even want to do
that? How is "true" a direction?

\subsection{Direct search}

Here's a naive recursive direct (depth-first) search.

\textasciitilde{}\textasciitilde{}\textasciitilde{}haskell -\/- source:
https://github.com/mstksg/inCode/tree/master/code-samples/huffman/PreTree.hs\#L90-L96
-\/- interactive: https://www.fpcomplete.com/user/jle/huffman-encoding findPT ::
Eq a =\textgreater{} PreTree a -\textgreater{} a -\textgreater{} Maybe Encoding
findPT pt0 x = go pt0 {[}{]} where go (PTLeaf y ) enc \textbar{} x == y = Just
(reverse enc) \textbar{} otherwise = Nothing go (PTNode pt1 pt2) enc = go pt1
(DLeft : enc) \textless{}\textbar{}\textgreater{} go pt2 (DRight : enc)
\textasciitilde{}\textasciitilde{}\textasciitilde{}

The algorithm goes:

\begin{enumerate}
\item
  If you find a \texttt{PTLeaf}, if the data matches what you are looking for,
  return the current path in a \texttt{Just}. If not, this is a dead-end; return
  \texttt{Nothing}.
\item
  If you find a \texttt{PTNode}, search the left branch adding a \texttt{DLeft}
  to the current path, and the right branch adding a \texttt{DRight} to the
  current path. Use \texttt{(\textless{}\textbar{}\textgreater{})} to perform
  the search lazily (ie, stop after the first success).
\end{enumerate}

\textasciitilde{}\textasciitilde{}\textasciitilde{}haskell ghci\textgreater{}
let pt = runBuildTree "hello world" ghci\textgreater{} findPT pt 'e' Just
{[}DLeft, DLeft, DRight{]} ghci\textgreater{} findPT pt 'q' Nothing
\textasciitilde{}\textasciitilde{}\textasciitilde{}

While it is clearly horribly inefficient, it does serve as a nice clean example
of a depth-first search (which exits as soon as it finds the goal), and probably
a nice reference implementation for us to reference later.

Its inefficiency lies in many things -\/-\/- chiefly of those being the fact
that Huffman trees don't give you any real help as a search tree, and nothing
short of a full depth-first traversal would work. Also, you probably don't want
to do this every time you want to encode something; you'd want to have some sort
of memoing and cacheing, ideally.

\subsection{Pre-searching}

We can sort of "solve" both of these problems this by traversing through our
\texttt{PreTree} and adding an entry to a \texttt{Map} at every leaf. This fixes
our repetition problem by memoizing all of our results into a map...and it fixes
our search problem because \texttt{Map}s are an ordered binary search tree with
efficient O(log n) lookups.{[}\^{}rewrite{]}

There are many ways to do this; my favorite right now is to do it by doing
collapsing our tree into one giant map, using the Monoid instance of
\texttt{Map}.

Basically, we turn each of our leaves into little \texttt{Map}s, and then
"combine" them all, using \texttt{(\textless{}\textgreater{})}, which "combines"
or merges two \texttt{Map\ k\ v}'s, using their Monoid instance:

\textasciitilde{}\textasciitilde{}\textasciitilde{}haskell -\/- source:
https://github.com/mstksg/inCode/tree/master/code-samples/huffman/PreTree.hs\#L101-L106
-\/- interactive: https://www.fpcomplete.com/user/jle/huffman-encoding ptTable
:: Ord a =\textgreater{} PreTree a -\textgreater{} Map a Encoding ptTable pt =
go pt {[}{]} where go (PTLeaf x) enc = x \texttt{M.singleton} reverse enc go
(PTNode pt1 pt2) enc = go pt1 (DLeft : enc) \textless{}\textgreater{} go pt2
(DRight : enc) \textasciitilde{}\textasciitilde{}\textasciitilde{}

We do some sort of fancy depth-first "map" over all of the leaves, keeping track
of how deep we are. Then we combine it all as we go along with
\texttt{\textless{}\textgreater{}}.

Note how it is almost identical in structure to \texttt{findPT}:

\textasciitilde{}\textasciitilde{}\textasciitilde{}haskell -\/- source:
https://github.com/mstksg/inCode/tree/master/code-samples/huffman/PreTree.hs\#L90-L96
-\/- interactive: https://www.fpcomplete.com/user/jle/huffman-encoding findPT ::
Eq a =\textgreater{} PreTree a -\textgreater{} a -\textgreater{} Maybe Encoding
findPT pt0 x = go pt0 {[}{]} where go (PTLeaf y ) enc \textbar{} x == y = Just
(reverse enc) \textbar{} otherwise = Nothing go (PTNode pt1 pt2) enc = go pt1
(DLeft : enc) \textless{}\textbar{}\textgreater{} go pt2 (DRight : enc)
\textasciitilde{}\textasciitilde{}\textasciitilde{}

Except instead of doing a "short-circuit combination" with
\texttt{(\textless{}\textbar{}\textgreater{})}, we do a "full combination" with
\texttt{(\textless{}\textgreater{})}.

\subsection{Lookup, Act 2}

So now that we have our lookup table, our new lookup/find function is both
simple and performant:

\textasciitilde{}\textasciitilde{}\textasciitilde{}haskell -\/- source:
https://github.com/mstksg/inCode/tree/master/code-samples/huffman/PreTree.hs\#L110-L111
-\/- interactive: https://www.fpcomplete.com/user/jle/huffman-encoding
lookupPTTable :: Ord a =\textgreater{} Map a Encoding -\textgreater{} a
-\textgreater{} Maybe Encoding lookupPTTable = flip M.lookup
\textasciitilde{}\textasciitilde{}\textasciitilde{}

given, of course, that we generate our table first.

\textasciitilde{}\textasciitilde{}\textasciitilde{}haskell ghci\textgreater{}
let pt = runBuildTree "hello world" ghci\textgreater{} let tb = fmap ptTable pt
ghci\textgreater{} tb \textgreater{}\textgreater{}= \textbackslash{}tb'
-\textgreater{} lookupPTTable tb' 'e' Just {[}DLeft, DLeft, DRight{]}
ghci\textgreater{} tb \textgreater{}\textgreater{}= \textbackslash{}tb'
-\textgreater{} lookupPTTable tb' 'q' Nothing
\textasciitilde{}\textasciitilde{}\textasciitilde{}

(Here we use the Monad instance for Maybe, to extract the
\texttt{tb\textquotesingle{}} out of the \texttt{Just\ tb}. We "sequence" two
Maybe's together. For more information, check out my
\href{http://blog.jle.im/entry/practical-fun-with-monads-introducing-monadplus}{blog
post} on this exact topic)

\subsection{Encoding many}

Now, we'd like to be able to decode an entire stream of \texttt{a}'s, returning
a list of the encodings.

\textasciitilde{}\textasciitilde{}\textasciitilde{}haskell -\/- source:
https://github.com/mstksg/inCode/tree/master/code-samples/huffman/PreTree.hs\#L114-L117
-\/- interactive: https://www.fpcomplete.com/user/jle/huffman-encoding encodeAll
:: Ord a =\textgreater{} PreTree a -\textgreater{} {[}a{]} -\textgreater{} Maybe
Encoding encodeAll pt xs = concat \textless{}\$\textgreater{} sequence (map
(lookupPTTable tb) xs) where tb = ptTable pt
\textasciitilde{}\textasciitilde{}\textasciitilde{}

This is a bit dense! But I'm sure that you are up for it.

\begin{enumerate}
\item
  First, we build the lookup table and call it \texttt{tb}.
\item
  Then, we map \texttt{lookupPTTable\ tb} over our list \texttt{xs}, to get a
  list of type \texttt{{[}Maybe\ Encoding{]}}.
\item
  Then, we use \texttt{sequence}, which in our case is
  \texttt{{[}Maybe\ a{]}\ -\textgreater{}\ Maybe\ {[}a{]}}. It turns a list of
  Maybe's into a list inside a Maybe. Recall the semantics of the Maybe monad:
  If you ever encounter a \texttt{Nothing}, the \emph{whole thing} is a
  \texttt{Nothing}. So in this case, if \emph{any} of the inputs are not
  decodable, \emph{the entire thing is Nothing}.

  \textasciitilde{}\textasciitilde{}\textasciitilde{}haskell ghci\textgreater{}
  sequence {[}Just 5, Just 4{]} Just {[}5,4{]} ghci\textgreater{} sequence
  {[}Just 6, Nothing{]} Nothing
  \textasciitilde{}\textasciitilde{}\textasciitilde{}

  Note that the standard libraries provide a synonym for
  \texttt{sequence\ .\ map} -\/-\/- \texttt{mapM}. So we could have written it
  as \texttt{mapM\ (lookupPTTable\ t)\ xs}...but that is significantly less
  clear/immediately understandable.
\item
  Recall that our \texttt{sequence} left us with a
  \texttt{Maybe\ {[}Encoding{]}}...but we only want \texttt{Maybe\ Encoding}. So
  we can use \texttt{(\textless{}\$\textgreater{})} to \texttt{concat} all of
  the \texttt{Encoding}s inside the Maybe.
\end{enumerate}

\textasciitilde{}\textasciitilde{}\textasciitilde{}haskell ghci\textgreater{}
let pt = runBuildTree "hello world" -\/- :: Maybe (PreTree Char)
ghci\textgreater{} pt \textgreater{}\textgreater{}= \textbackslash{}pt'
-\textgreater{} encodeAll pt' "hello world" Just {[}DLeft, DLeft, DLeft, DLeft,
DLeft, DRight, DRight, DLeft, DRight, DLeft, DRight, DRight, DRight, DRight,
DRight, DLeft, DRight, DLeft, DRight, DLeft, DRight, DRight, DRight, DLeft,
DRight, DRight, DRight, DLeft, DRight, DRight, DLeft, DLeft{]}
ghci\textgreater{} pt \textgreater{}\textgreater{}= \textbackslash{}pt'
-\textgreater{} encodeAll pt' "hello worldq" Nothing
\textasciitilde{}\textasciitilde{}\textasciitilde{}

Welp, that's half the battle!

\section{Decoding}

For huffman trees, decoding is the much simpler process. Simply traverse down
the tree using the given encoding and return a value whenever you reach a leaf.

\textasciitilde{}\textasciitilde{}\textasciitilde{}haskell -\/- source:
https://github.com/mstksg/inCode/tree/master/code-samples/huffman/PreTree.hs\#L123-L128
-\/- interactive: https://www.fpcomplete.com/user/jle/huffman-encoding decodePT
:: PreTree a -\textgreater{} Encoding -\textgreater{} Maybe (a, Encoding)
decodePT (PTLeaf x) ds = Just (x, ds) decodePT (PTNode pt1 pt2) (d:ds) = case d
of DLeft -\textgreater{} decodePT pt1 ds DRight -\textgreater{} decodePT pt2 ds
decodePT (PTNode \_ \_) {[}{]} = Nothing
\textasciitilde{}\textasciitilde{}\textasciitilde{}

The logic should seem pretty familiar. The main algorithm involves going down
the tree, "following" the direction list. If you reach a leaf, then you have
found something (and return the directions you haven't followed yet). If you run
out of directions while on a node...something has gone wrong.

\textasciitilde{}\textasciitilde{}\textasciitilde{}haskell ghci\textgreater{} do
pt \textless{}- runBuildTree "hello world" \textbar{} enc \textless{}- encodeAll
pt "hello world" \textbar{} decodePT pt enc Just ('h', {[}DLeft, DLeft ...{]})
\textasciitilde{}\textasciitilde{}\textasciitilde{}

(Here we are using the Maybe monad, in order to "stitch together" three
possibly-failing operations in a row. We call \texttt{pt} and \texttt{enc} the
values "inside" the \texttt{Just\ pt} and \texttt{Just\ enc} returned by
\texttt{runBuildTree} and \texttt{encodeAll}; the whole thing fails if any of
the steps fail at any time. If you are not familiar with this,
\href{http://blog.jle.im/entry/practical-fun-with-monads-introducing-monadplus}{I
sort of literally wrote an entire blog post} on this subject :) )

\subsection{Decoding many}

We'd like to repeatedly iterate this until we have consumed our entire encoding.

Basically, starting with a list of encodings, we want to continually chop it up
and build a list from it.

This sounds a lot like the \texttt{Data.List} function \texttt{unfoldr}:

\textasciitilde{}\textasciitilde{}\textasciitilde{}haskell unfoldr :: (b
-\textgreater{} Maybe (a, b)) -\textgreater{} b -\textgreater{} {[}a{]}
\textasciitilde{}\textasciitilde{}\textasciitilde{}

\texttt{unfoldr} makes a list by applying your function repeatedly to a
"de-cumulator", carrying the state of the decumulator, and stopping when your
function returns \texttt{Nothing}. You can think of it as the "opposite" of
\texttt{foldr}.

Using \texttt{unfoldr}, we can write a \texttt{decodeAll}:

\textasciitilde{}\textasciitilde{}\textasciitilde{}haskell -\/- source:
https://github.com/mstksg/inCode/tree/master/code-samples/huffman/PreTree.hs\#L132-L133
-\/- interactive: https://www.fpcomplete.com/user/jle/huffman-encoding decodeAll
:: PreTree a -\textgreater{} Encoding -\textgreater{} {[}a{]} decodeAll pt =
unfoldr (decodePT pt) \textasciitilde{}\textasciitilde{}\textasciitilde{}

\textasciitilde{}\textasciitilde{}\textasciitilde{}haskell ghci\textgreater{} do
pt \textless{}- runBuildTree "hello world" \textbar{} enc \textless{}- encodeAll
pt "hello world" \textbar{} return (decodeAll pt enc)
\textasciitilde{}\textasciitilde{}\textasciitilde{}

Which works exactly as we'd like!

\section{Testing}

We can write a utility to test our encoding/decoding functions:

\textasciitilde{}\textasciitilde{}\textasciitilde{}haskell -\/- source:
https://github.com/mstksg/inCode/tree/master/code-samples/huffman/Huffman.hs\#L106-L113
-\/- interactive: https://www.fpcomplete.com/user/jle/huffman-encoding testTree
:: Ord a =\textgreater{} {[}a{]} -\textgreater{} {[}a{]} testTree {[}{]} =
{[}{]} -\/- handle the empty list testTree xs = decoded where Just decoded = do
pt \textless{}- runBuildTree xs enc \textless{}- encodeAll pt xs return
(decodeAll pt enc) \textasciitilde{}\textasciitilde{}\textasciitilde{}

(Again, refer to my
\href{http://blog.jle.im/entry/practical-fun-with-monads-introducing-monadplus}{MonadPlus}
article from earlier, if you are unfamiliar with working with the Maybe monad)

\texttt{testTree} should be an identity; that is,
\texttt{testTree\ xs\ ===\ xs}.

\textasciitilde{}\textasciitilde{}\textasciitilde{}haskell ghci\textgreater{}
testTree "hello world" "hello world" ghci\textgreater{} testTree "the quick
brown fox jumps over the lazy dog" "the quick brown fox jumps over the lazy dog"
\textasciitilde{}\textasciitilde{}\textasciitilde{}

Note the very unsafe irrefutable pattern match on \texttt{Just\ decoded}. We'll
fix this later!

\subsection{QuickCheck}

Now that we have a neat proposition, we can use \texttt{quickcheck} on it, from
the great \href{http://hackage.haskell.org/package/QuickCheck}{QuickCheck}
library. \texttt{quickcheck} will basically test our proposition
\texttt{testTree\ xs\ ==\ xs} by generating several random \texttt{xs}'s.

\textasciitilde{}\textasciitilde{}\textasciitilde{}haskell ghci\textgreater{}
import Test.QuickCheck ghci\textgreater{} :set -XScopedTypeVariables
ghci\textgreater{} quickCheck ((xs :: String) -\textgreater{} testTree xs == xs)
*** Failed! Falsifiable (after 3 tests and 2 shrinks): "a"
\textasciitilde{}\textasciitilde{}\textasciitilde{}

\subsubsection{Failure!}

Oh! We failed? And on such a simple case? What happened?

If we look at how \texttt{"a"} is encoded, it'll become apparent:

\textasciitilde{}\textasciitilde{}\textasciitilde{}haskell ghci\textgreater{}
let (Just pt) = runBuildTree "aaa" ghci\textgreater{} pt PTLeaf 'a'
ghci\textgreater{} findPT pt 'a' Just {[}{]} ghci\textgreater{} encodeAll pt
"aaaaaaaaaaa" Just {[}{]} \textasciitilde{}\textasciitilde{}\textasciitilde{}

Ah. Well, that's a problem. Basically, our input string has
\href{http://en.wikipedia.org/wiki/Entropy_(information_theory)}{"zero"
entropy}, according to typical measurements. So we cannot naively huffman encode
it.

\subsubsection{Success!}

There are a few ways to deal with this. The most "immediate" way would be to
realize that \texttt{decodeAll} is partial (that is, it does not terminate/is
undefined on some of its inputs), and will actually never terminate if the given
tree is a singleton tree. We can write a "safe" \texttt{decodeAll}:

\textasciitilde{}\textasciitilde{}\textasciitilde{}haskell -\/- source:
https://github.com/mstksg/inCode/tree/master/code-samples/huffman/PreTree.hs\#L137-L139
-\/- interactive: https://www.fpcomplete.com/user/jle/huffman-encoding
decodeAll' :: PreTree a -\textgreater{} Encoding -\textgreater{} Maybe {[}a{]}
decodeAll' (PTLeaf \emph{) } = Nothing decodeAll' pt enc = Just \$ unfoldr
(decodePT pt) enc \textasciitilde{}\textasciitilde{}\textasciitilde{}

In doing this, we don't exactly "fix" the problem...we only defer
responsibility. Now, whoever uses \texttt{decodeAll\textquotesingle{}} (like our
eventual encoding interface) is \emph{forced to handle the error} (by handing
the \texttt{Nothing} case). In this way, \emph{the type system enforces safety}.
Had we always used the unsafe \texttt{decodeAll}, then whoever uses it
eventually has to "manually remember" to handle the unterminating case, by
carefuly reading documentation or something. In this case, the type system is a
big, explicit reminder saying "hey, deal with the unterminating case."

We'll also a "safe" \texttt{testTree}:

\textasciitilde{}\textasciitilde{}\textasciitilde{}haskell -\/- source:
https://github.com/mstksg/inCode/tree/master/code-samples/huffman/Huffman.hs\#L117-L121
-\/- interactive: https://www.fpcomplete.com/user/jle/huffman-encoding testTree'
:: Ord a =\textgreater{} {[}a{]} -\textgreater{} Maybe {[}a{]} testTree' xs = do
pt \textless{}- runBuildTree xs enc \textless{}- encodeAll pt xs decodeAll' pt
enc \textasciitilde{}\textasciitilde{}\textasciitilde{}

So we can now quickcheck:

\textasciitilde{}\textasciitilde{}\textasciitilde{}haskell ghci\textgreater{}
quickCheck ((xs :: String) -\textgreater{} testTree' xs \texttt{elem}
{[}Nothing, Just xs{]}) +++ OK, passed 100 tests.
\textasciitilde{}\textasciitilde{}\textasciitilde{}

Hooray!

\subsection{Re: Testing}

All I'll admit that I didn't even anticipate the degenerate singleton tree case
until I decided to add a quickcheck section to this post. It just goes to show
that you should always test! And it also shows how easy it is to write tests in
quickcheck. One line could mean five unit tests, and you might even test
edge/corner cases that you might have never even thought about!

For example, we probably should have tested \texttt{lookupPTTable} against
\texttt{findPT}, our reference implementation :) We should have also tested our
binary encode/decode!

\section{Next Time}

We're almost there!

For our last section, we are going to be focusing on pulling it all together to
make a streaming compression/decompression interface that will be able to read a
file and encode/decode into a new file as it goes, in constant memory, using
pipes. After that, we will also be looking at how to profile code, applying some
optimization tricks we can do to get things just right, and other things to wrap
up.

\end{document}
