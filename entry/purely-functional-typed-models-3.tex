\documentclass[]{article}
\usepackage{lmodern}
\usepackage{amssymb,amsmath}
\usepackage{ifxetex,ifluatex}
\usepackage{fixltx2e} % provides \textsubscript
\ifnum 0\ifxetex 1\fi\ifluatex 1\fi=0 % if pdftex
  \usepackage[T1]{fontenc}
  \usepackage[utf8]{inputenc}
\else % if luatex or xelatex
  \ifxetex
    \usepackage{mathspec}
    \usepackage{xltxtra,xunicode}
  \else
    \usepackage{fontspec}
  \fi
  \defaultfontfeatures{Mapping=tex-text,Scale=MatchLowercase}
  \newcommand{\euro}{€}
\fi
% use upquote if available, for straight quotes in verbatim environments
\IfFileExists{upquote.sty}{\usepackage{upquote}}{}
% use microtype if available
\IfFileExists{microtype.sty}{\usepackage{microtype}}{}
\usepackage[margin=1in]{geometry}
\usepackage{color}
\usepackage{fancyvrb}
\newcommand{\VerbBar}{|}
\newcommand{\VERB}{\Verb[commandchars=\\\{\}]}
\DefineVerbatimEnvironment{Highlighting}{Verbatim}{commandchars=\\\{\}}
% Add ',fontsize=\small' for more characters per line
\newenvironment{Shaded}{}{}
\newcommand{\AlertTok}[1]{\textcolor[rgb]{1.00,0.00,0.00}{\textbf{#1}}}
\newcommand{\AnnotationTok}[1]{\textcolor[rgb]{0.38,0.63,0.69}{\textbf{\textit{#1}}}}
\newcommand{\AttributeTok}[1]{\textcolor[rgb]{0.49,0.56,0.16}{#1}}
\newcommand{\BaseNTok}[1]{\textcolor[rgb]{0.25,0.63,0.44}{#1}}
\newcommand{\BuiltInTok}[1]{#1}
\newcommand{\CharTok}[1]{\textcolor[rgb]{0.25,0.44,0.63}{#1}}
\newcommand{\CommentTok}[1]{\textcolor[rgb]{0.38,0.63,0.69}{\textit{#1}}}
\newcommand{\CommentVarTok}[1]{\textcolor[rgb]{0.38,0.63,0.69}{\textbf{\textit{#1}}}}
\newcommand{\ConstantTok}[1]{\textcolor[rgb]{0.53,0.00,0.00}{#1}}
\newcommand{\ControlFlowTok}[1]{\textcolor[rgb]{0.00,0.44,0.13}{\textbf{#1}}}
\newcommand{\DataTypeTok}[1]{\textcolor[rgb]{0.56,0.13,0.00}{#1}}
\newcommand{\DecValTok}[1]{\textcolor[rgb]{0.25,0.63,0.44}{#1}}
\newcommand{\DocumentationTok}[1]{\textcolor[rgb]{0.73,0.13,0.13}{\textit{#1}}}
\newcommand{\ErrorTok}[1]{\textcolor[rgb]{1.00,0.00,0.00}{\textbf{#1}}}
\newcommand{\ExtensionTok}[1]{#1}
\newcommand{\FloatTok}[1]{\textcolor[rgb]{0.25,0.63,0.44}{#1}}
\newcommand{\FunctionTok}[1]{\textcolor[rgb]{0.02,0.16,0.49}{#1}}
\newcommand{\ImportTok}[1]{#1}
\newcommand{\InformationTok}[1]{\textcolor[rgb]{0.38,0.63,0.69}{\textbf{\textit{#1}}}}
\newcommand{\KeywordTok}[1]{\textcolor[rgb]{0.00,0.44,0.13}{\textbf{#1}}}
\newcommand{\NormalTok}[1]{#1}
\newcommand{\OperatorTok}[1]{\textcolor[rgb]{0.40,0.40,0.40}{#1}}
\newcommand{\OtherTok}[1]{\textcolor[rgb]{0.00,0.44,0.13}{#1}}
\newcommand{\PreprocessorTok}[1]{\textcolor[rgb]{0.74,0.48,0.00}{#1}}
\newcommand{\RegionMarkerTok}[1]{#1}
\newcommand{\SpecialCharTok}[1]{\textcolor[rgb]{0.25,0.44,0.63}{#1}}
\newcommand{\SpecialStringTok}[1]{\textcolor[rgb]{0.73,0.40,0.53}{#1}}
\newcommand{\StringTok}[1]{\textcolor[rgb]{0.25,0.44,0.63}{#1}}
\newcommand{\VariableTok}[1]{\textcolor[rgb]{0.10,0.09,0.49}{#1}}
\newcommand{\VerbatimStringTok}[1]{\textcolor[rgb]{0.25,0.44,0.63}{#1}}
\newcommand{\WarningTok}[1]{\textcolor[rgb]{0.38,0.63,0.69}{\textbf{\textit{#1}}}}
\ifxetex
  \usepackage[setpagesize=false, % page size defined by xetex
              unicode=false, % unicode breaks when used with xetex
              xetex]{hyperref}
\else
  \usepackage[unicode=true]{hyperref}
\fi
\hypersetup{breaklinks=true,
            bookmarks=true,
            pdfauthor={Justin Le},
            pdftitle={A Purely Functional Typed Approach to Trainable Models (Part 3)},
            colorlinks=true,
            citecolor=blue,
            urlcolor=blue,
            linkcolor=magenta,
            pdfborder={0 0 0}}
\urlstyle{same}  % don't use monospace font for urls
% Make links footnotes instead of hotlinks:
\renewcommand{\href}[2]{#2\footnote{\url{#1}}}
\setlength{\parindent}{0pt}
\setlength{\parskip}{6pt plus 2pt minus 1pt}
\setlength{\emergencystretch}{3em}  % prevent overfull lines
\setcounter{secnumdepth}{0}

\title{A Purely Functional Typed Approach to Trainable Models (Part 3)}
\author{Justin Le}
\date{May 14, 2018}

\begin{document}
\maketitle

\emph{Originally posted on
\textbf{\href{https://blog.jle.im/entry/purely-functional-typed-models-3.html}{in
Code}}.}

Hi again! Today we're going to jump straight into tying together the functional
framework described in this series and see how it can give us some interesting
insight, as well as wrapping it up by talking about the scaffolding needed to
turn this all into a working system you can apply today.

The name of the game is a purely functional typed approach to writing trainable
models using differentiable programming. Be sure to check out
\href{https://blog.jle.im/entry/purely-functional-typed-models-1.html}{Part 1}
and \href{https://blog.jle.im/entry/purely-functional-typed-models-2.html}{Part
2} if you haven't, because this is a direct continuation.

My favorite part about this system really is how we have pretty much free reign
over how we can combine and manipulate our models, since they are just
functions. Combinators --- a word I'm going to be using to mean higher-order
functions that return functions --- tie everything together so well. Some models
we might have thought were standalone entities might just be derivable from
other models using basic functional combinators. And the best part is that
they're never \emph{necessary}; just \emph{helpful}.

Again, if you want to follow along, the source code for the written code in this
module is available
\href{https://github.com/mstksg/inCode/tree/master/code-samples/functional-models/model.hs}{on
github}.

\hypertarget{combinator-fun}{%
\section{Combinator Fun}\label{combinator-fun}}

\hypertarget{recurrence}{%
\subsection{Recurrence}\label{recurrence}}

Here's one example of how the freedom that ``normal functions'' gives you can
help reveal insight. While working through this approach, I stumbled upon an
interesting way of defining recurrent neural networks --- a lot of times, a
``recurrent neural network'' really just means that some function of the
\emph{previous} output is used as an ``extra input''.

This sounds like we can really write a recurrent model as a ``normal'' model,
and then use a combinator to feed it back into itself.

To say in types:

\begin{Shaded}
\begin{Highlighting}[]
\NormalTok{recurrently}
\OtherTok{    ::} \DataTypeTok{Model}\NormalTok{  p   (a }\FunctionTok{:&}\NormalTok{ b) b}
    \OtherTok{->} \DataTypeTok{ModelS}\NormalTok{ p b  a       b}
\end{Highlighting}
\end{Shaded}

A ``normal, non-stateful model'' taking an \texttt{a\ :\&\ b} and returning a
\texttt{b} can really be turned into a stateful model with state \texttt{b} (the
\emph{previous output}) and only taking in an \texttt{a} input.

This sort of combinator is a joy to write in Haskell because it's a ``follow the
types'' kinda deal --- you set up the function, and the compiler pretty much
writes it for you, because the types guide the entire implementation:

\begin{Shaded}
\begin{Highlighting}[]
\CommentTok{-- source: https://github.com/mstksg/inCode/tree/master/code-samples/functional-models/model.hs#L300-L306}

\NormalTok{recurrently}
\OtherTok{    ::}\NormalTok{ (}\DataTypeTok{Backprop}\NormalTok{ a, }\DataTypeTok{Backprop}\NormalTok{ b)}
    \OtherTok{=>} \DataTypeTok{Model}\NormalTok{  p   (a }\FunctionTok{:&}\NormalTok{ b) b}
    \OtherTok{->} \DataTypeTok{ModelS}\NormalTok{ p b  a       b}
\NormalTok{recurrently f p x yLast }\FunctionTok{=}\NormalTok{ (y, y)}
  \KeywordTok{where}
\NormalTok{    y }\FunctionTok{=}\NormalTok{ f p (x }\FunctionTok{:&&}\NormalTok{ yLast)}
\end{Highlighting}
\end{Shaded}

In general though, it'd be nice to have \emph{some function} of the previous
output be stored as the state. We can write this combinator as well, taking the
function that transforms the previous output into the stored state:

\begin{Shaded}
\begin{Highlighting}[]
\CommentTok{-- source: https://github.com/mstksg/inCode/tree/master/code-samples/functional-models/model.hs#L308-L315}

\NormalTok{recurrentlyWith}
\OtherTok{    ::}\NormalTok{ (}\DataTypeTok{Backprop}\NormalTok{ a, }\DataTypeTok{Backprop}\NormalTok{ b)}
    \OtherTok{=>}\NormalTok{ (forall z}\FunctionTok{.} \DataTypeTok{Reifies}\NormalTok{ z }\DataTypeTok{W} \OtherTok{=>} \DataTypeTok{BVar}\NormalTok{ z c }\OtherTok{->} \DataTypeTok{BVar}\NormalTok{ z b)}
    \OtherTok{->} \DataTypeTok{Model}\NormalTok{  p   (a }\FunctionTok{:&}\NormalTok{ b) c}
    \OtherTok{->} \DataTypeTok{ModelS}\NormalTok{ p b  a       c}
\NormalTok{recurrentlyWith store f p x yLast }\FunctionTok{=}\NormalTok{ (y, store y)}
  \KeywordTok{where}
\NormalTok{    y }\FunctionTok{=}\NormalTok{ f p (x }\FunctionTok{:&&}\NormalTok{ yLast)}
\end{Highlighting}
\end{Shaded}

Again, once we figure out the \emph{type} our combinator has\ldots{}the function
\emph{writes itself}. The joys of Haskell! I wouldn't dare try to write this in
a language without static types and type inference. But it's a real treat to
write this out in a language like Haskell.

\texttt{recurrentlyWith} takes a \texttt{c\ -\textgreater{}\ b} function and
turns a pure model taking an \texttt{a\ :\&\ b} into a stateful model with state
\texttt{b} taking in an \texttt{a}. The \texttt{c\ -\textgreater{}\ b} tells you
how to turn the previous output into the new state.

To me, \texttt{recurrentlyWith} captures the ``essence'' of what a recurrent
model or recurrent neural network is --- the network is allowed to ``see'' some
form of its previous output.

How is this useful? Well, we can use this to define a fully connected recurrent
neural network layer as simply a recurrent version of a normal fully connected
feed-forward layer.

We can redefine a pre-mapped version of \texttt{feedForward} which takes a tuple
of two vectors and concatenates them before doing anything:

\begin{Shaded}
\begin{Highlighting}[]
\CommentTok{-- | Concatenate two vectors}
\OtherTok{(#) ::} \DataTypeTok{BVar}\NormalTok{ z (}\DataTypeTok{R}\NormalTok{ i) }\OtherTok{->} \DataTypeTok{BVar}\NormalTok{ z (}\DataTypeTok{R}\NormalTok{ o) }\OtherTok{->} \DataTypeTok{BVar}\NormalTok{ z (}\DataTypeTok{R}\NormalTok{ (i }\FunctionTok{+}\NormalTok{ o))}
\CommentTok{-- source: https://github.com/mstksg/inCode/tree/master/code-samples/functional-models/model.hs#L317-L320}

\NormalTok{ffOnSplit}
\OtherTok{    ::}\NormalTok{ forall i o}\FunctionTok{.}\NormalTok{ (}\DataTypeTok{KnownNat}\NormalTok{ i, }\DataTypeTok{KnownNat}\NormalTok{ o)}
    \OtherTok{=>} \DataTypeTok{Model}\NormalTok{ _ (}\DataTypeTok{R}\NormalTok{ i }\FunctionTok{:&} \DataTypeTok{R}\NormalTok{ o) (}\DataTypeTok{R}\NormalTok{ o)}
\NormalTok{ffOnSplit p (rI }\FunctionTok{:&&}\NormalTok{ rO) }\FunctionTok{=}\NormalTok{ feedForward p (rI }\FunctionTok{#}\NormalTok{ rO)}
\end{Highlighting}
\end{Shaded}

\texttt{ffOnSplit} is a feed-forward layer taking an \texttt{R\ (i\ +\ o)},
except we pre-map it to take a tuple \texttt{R\ i\ :\&\ R\ o} instead. This
isn't anything special, just some plumbing.

Now our fully connected recurrent layer is just
\texttt{recurrentlyWith\ logistic\ ffOnSplit}:

\begin{Shaded}
\begin{Highlighting}[]
\NormalTok{fcrnn'}
\OtherTok{    ::}\NormalTok{ (}\DataTypeTok{KnownNat}\NormalTok{ i, }\DataTypeTok{KnownNat}\NormalTok{ o)}
    \OtherTok{=>} \DataTypeTok{ModelS}\NormalTok{ _ (}\DataTypeTok{R}\NormalTok{ o) (}\DataTypeTok{R}\NormalTok{ i) (}\DataTypeTok{R}\NormalTok{ o)}
\NormalTok{fcrnn' }\FunctionTok{=}\NormalTok{ recurrentlyWith logistic ffOnSplit}
\end{Highlighting}
\end{Shaded}

Basically just a recurrent version of \texttt{feedForward}! If we factor out
some of the manual uncurrying and pre-mapping, we get a nice functional
definition:

\begin{Shaded}
\begin{Highlighting}[]
\CommentTok{-- source: https://github.com/mstksg/inCode/tree/master/code-samples/functional-models/model.hs#L322-L325}

\NormalTok{fcrnn'}
\OtherTok{    ::}\NormalTok{ (}\DataTypeTok{KnownNat}\NormalTok{ i, }\DataTypeTok{KnownNat}\NormalTok{ o)}
    \OtherTok{=>} \DataTypeTok{ModelS}\NormalTok{ _ (}\DataTypeTok{R}\NormalTok{ o) (}\DataTypeTok{R}\NormalTok{ i) (}\DataTypeTok{R}\NormalTok{ o)}
\NormalTok{fcrnn' }\FunctionTok{=}\NormalTok{ recurrentlyWith logistic (\textbackslash{}p }\OtherTok{->}\NormalTok{ feedForward p }\FunctionTok{.}\NormalTok{ uncurryT (}\FunctionTok{#}\NormalTok{))}
\end{Highlighting}
\end{Shaded}

\hypertarget{lag}{%
\subsection{Lag}\label{lag}}

Another interesting result -- we can write a ``lagged'' combinator that takes a
model expecting a vector as an input, and turn it into a stateful model taking a
\emph{single} input, and feeding the original model that input and also a
history of the \texttt{n} most recent inputs.

If that sounds confusing, let's just try to state it out using types:

\begin{Shaded}
\begin{Highlighting}[]
\OtherTok{lagged ::} \DataTypeTok{Model}\NormalTok{  p       (}\DataTypeTok{R}\NormalTok{ (n }\FunctionTok{+} \DecValTok{1}\NormalTok{)) b}
       \OtherTok{->} \DataTypeTok{ModelS}\NormalTok{ p (}\DataTypeTok{R}\NormalTok{ n) }\DataTypeTok{Double}\NormalTok{      b}
\end{Highlighting}
\end{Shaded}

The result is a \texttt{ModelS\ p\ (R\ n)\ Double\ b}; the state is the
\texttt{n} most recent inputs, and it feeds that in at every step and keeps it
updated. Let's write it using \texttt{headTail} and \texttt{\&}, which splits a
vector and adds an item to the end, respectively.

\begin{Shaded}
\begin{Highlighting}[]
\CommentTok{-- source: https://github.com/mstksg/inCode/tree/master/code-samples/functional-models/model.hs#L327-L335}

\NormalTok{lagged}
\OtherTok{    ::}\NormalTok{ (}\DataTypeTok{KnownNat}\NormalTok{ n, }\DecValTok{1} \FunctionTok{<=}\NormalTok{ n)}
    \OtherTok{=>} \DataTypeTok{Model}\NormalTok{  p       (}\DataTypeTok{R}\NormalTok{ (n }\FunctionTok{+} \DecValTok{1}\NormalTok{)) b}
    \OtherTok{->} \DataTypeTok{ModelS}\NormalTok{ p (}\DataTypeTok{R}\NormalTok{ n) }\DataTypeTok{Double}\NormalTok{      b}
\NormalTok{lagged f p x xLasts }\FunctionTok{=}\NormalTok{ (y, xLasts')}
  \KeywordTok{where}
\NormalTok{    fullLasts    }\FunctionTok{=}\NormalTok{ xLasts }\FunctionTok{&}\NormalTok{ x}
\NormalTok{    y            }\FunctionTok{=}\NormalTok{ f p fullLasts}
\NormalTok{    (_, xLasts') }\FunctionTok{=}\NormalTok{ headTail fullLasts}
\end{Highlighting}
\end{Shaded}

What can we do with this? Well\ldots{} we can write a general autoregressive
model AR(p) of \emph{any} degree, simply by lagging a fully connected ANN layer:

\begin{Shaded}
\begin{Highlighting}[]
\CommentTok{-- source: https://github.com/mstksg/inCode/tree/master/code-samples/functional-models/model.hs#L337-L339}

\OtherTok{ar ::}\NormalTok{ (}\DataTypeTok{KnownNat}\NormalTok{ n, }\DecValTok{1} \FunctionTok{<=}\NormalTok{ n)}
   \OtherTok{=>} \DataTypeTok{ModelS}\NormalTok{ _ (}\DataTypeTok{R}\NormalTok{ n) }\DataTypeTok{Double} \DataTypeTok{Double}
\NormalTok{ar }\FunctionTok{=}\NormalTok{ lagged (\textbackslash{}p }\OtherTok{->}\NormalTok{ fst }\FunctionTok{.}\NormalTok{ headTail }\FunctionTok{.}\NormalTok{ feedForward }\FunctionTok{@}\NormalTok{_ }\FunctionTok{@}\DecValTok{1}\NormalTok{ p)}
\end{Highlighting}
\end{Shaded}

(using \texttt{fst\ .\ headTail} to extract the first \texttt{Double} from an
\texttt{R\ 1})

And that's it! Our original AR(2) \texttt{ar2} is just \texttt{ar\ @2} \ldots{}
and we can write can write an AR(10) model by just using \texttt{ar\ @10}, and
AR(20) model with \texttt{ar\ @20}, etc.

\begin{Shaded}
\begin{Highlighting}[]
\CommentTok{-- source: https://github.com/mstksg/inCode/tree/master/code-samples/functional-models/model.hs#L341-L342}

\OtherTok{ar2' ::} \DataTypeTok{ModelS}\NormalTok{ _ (}\DataTypeTok{R} \DecValTok{2}\NormalTok{) }\DataTypeTok{Double} \DataTypeTok{Double}
\NormalTok{ar2' }\FunctionTok{=}\NormalTok{ ar }\FunctionTok{@}\DecValTok{2}
\end{Highlighting}
\end{Shaded}

Who would have thought that an autoregressive model is just a fully connected
neural network layer with lag?

Take a fully connected ANN layer and add recurrence --- you get a fully
connected RNN layer. Take a fully connected ANN layer and add lag --- you get an
autoregressive model from statistics!

There are many more such combinators possible! Combinators like
\texttt{recurrentlyWith} and \texttt{lagged} just scratch the surface. Best of
all, they help reveal to us that seemingly exotic things really are just simple
applications of combinators from other basic things.

\hypertarget{fun-with-explicit-types}{%
\section{Fun with explicit types}\label{fun-with-explicit-types}}

One of the advantages of the statically typed functional approach is that it
forces you to keep track of parameter types as a part of your model
manipulation. You can explicitly keep track of them, or let the compiler do it
for you (and have the information ready when you need it). In what we have been
doing so far, we have been letting the compiler have the fun. But we can get
some interesting results with explicit manipulation of types, as well.

For example, an \href{https://en.wikipedia.org/wiki/Autoencoder}{autoencoder} is
a type of model that composes a function that ``compresses'' information with a
function that ``decompresses'' it; training an autoencoder involves training the
composition of those two functions to produce the identity function.

We can represent a simple autoencoder:

\begin{Shaded}
\begin{Highlighting}[]
\OtherTok{encoder ::} \DataTypeTok{Model}\NormalTok{ q (}\DataTypeTok{R} \DecValTok{100}\NormalTok{) (}\DataTypeTok{R} \DecValTok{5}\NormalTok{)}
\OtherTok{decoder ::} \DataTypeTok{Model}\NormalTok{ p (}\DataTypeTok{R} \DecValTok{5}\NormalTok{)   (}\DataTypeTok{R} \DecValTok{100}\NormalTok{)}

\OtherTok{autoencoder ::} \DataTypeTok{Model}\NormalTok{ (p }\FunctionTok{:&}\NormalTok{ q) (}\DataTypeTok{R} \DecValTok{100}\NormalTok{) (}\DataTypeTok{R} \DecValTok{100}\NormalTok{)}
\NormalTok{autoencoder }\FunctionTok{=}\NormalTok{ decoder }\FunctionTok{<~}\NormalTok{ encoder}
\end{Highlighting}
\end{Shaded}

\texttt{autoencoder} now ``encodes'' a 100-dimensional space into a
5-dimensional one.

We can train \texttt{autoencoder} on our data set, but keep the ``trained
parameters'' separate:

\begin{Shaded}
\begin{Highlighting}[]
\NormalTok{ghci}\FunctionTok{>}\NormalTok{ decParam }\FunctionTok{:&}\NormalTok{ encParam }\OtherTok{<-}\NormalTok{ trainModelIO autoencoder }\FunctionTok{$}\NormalTok{ map (\textbackslash{}x }\OtherTok{->}\NormalTok{ (x,x)) samps}
\end{Highlighting}
\end{Shaded}

Now \texttt{decParam} and \texttt{encParam} make \texttt{autoencoder} an
identity function. But, we can just use \texttt{encParam} with \texttt{encoder}
to \emph{encode} data, and \texttt{decParam} with \texttt{decoder} to
\emph{decode} data!

\begin{Shaded}
\begin{Highlighting}[]
\NormalTok{evalBP2 encoder}\OtherTok{ encParam ::} \DataTypeTok{R} \DecValTok{100} \OtherTok{->} \DataTypeTok{R} \DecValTok{5}        \CommentTok{-- trained encoder}
\NormalTok{evalBP2 decoder}\OtherTok{ decParam ::} \DataTypeTok{R} \DecValTok{5}   \OtherTok{->} \DataTypeTok{R} \DecValTok{100}      \CommentTok{-- trained decoder}
\end{Highlighting}
\end{Shaded}

The types help by keeping track of what goes with what, so you don't have to;
the compiler helps you match up \texttt{encoder} with \texttt{encParam}, and can
even ``fill in the code'' for you if you leave in a typed hole!

\hypertarget{a-unified-representation}{%
\section{A Unified Representation}\label{a-unified-representation}}

This section now is a small aside for those familiar with more advanced Haskell
techniques like DataKinds and dependent types; if you aren't too comfortable
with these, feel free to skip to the next section! This stuff won't come up
again later.

If you're still reading, one ugly thing you might have noticed was that we had
to give different ``types'' for both our \texttt{Model} and \texttt{ModelS}, so
we cannot re-use useful functions on both. For example, \texttt{mapS} only works
on \texttt{ModelS}, but not \texttt{Model}.
\texttt{(\textless{}\textasciitilde{})} only works on \texttt{Model}s,
\texttt{(\textless{}*\textasciitilde{}*)} only works on two \texttt{ModelS}s,
and we had to define a different combinator
\texttt{(\textless{}*\textasciitilde{})}.

This is not a fundamental limitation! With \emph{DataKinds} and dependent types
we can unify these both under a common type. If we had:

\begin{Shaded}
\begin{Highlighting}[]
\KeywordTok{type} \DataTypeTok{Model}\NormalTok{ (}\OtherTok{p ::} \DataTypeTok{Type}\NormalTok{) (}\OtherTok{a ::} \DataTypeTok{Type}\NormalTok{) (}\OtherTok{b ::} \DataTypeTok{Type}\NormalTok{) }\FunctionTok{=}
\NormalTok{       forall z}\FunctionTok{.} \DataTypeTok{Reifies}\NormalTok{ z }\DataTypeTok{W}
    \OtherTok{=>} \DataTypeTok{BVar}\NormalTok{ z p}
    \OtherTok{->} \DataTypeTok{BVar}\NormalTok{ z a}
    \OtherTok{->} \DataTypeTok{BVar}\NormalTok{ z b}

\KeywordTok{type} \DataTypeTok{ModelS}\NormalTok{ (}\OtherTok{p ::} \DataTypeTok{Type}\NormalTok{) (}\OtherTok{s ::} \DataTypeTok{Type}\NormalTok{) (}\OtherTok{a ::} \DataTypeTok{Type}\NormalTok{) (}\OtherTok{b ::} \DataTypeTok{Type}\NormalTok{) }\FunctionTok{=}
\NormalTok{       forall z}\FunctionTok{.} \DataTypeTok{Reifies}\NormalTok{ z }\DataTypeTok{W}
    \OtherTok{=>} \DataTypeTok{BVar}\NormalTok{ z p}
    \OtherTok{->} \DataTypeTok{BVar}\NormalTok{ z a}
    \OtherTok{->} \DataTypeTok{BVar}\NormalTok{ z s}
    \OtherTok{->}\NormalTok{ (}\DataTypeTok{BVar}\NormalTok{ z b, }\DataTypeTok{BVar}\NormalTok{ z s)}
\end{Highlighting}
\end{Shaded}

We can unify them by making either the \texttt{p} or \texttt{s} be optional, a
\texttt{Maybe\ Type}, and using the \texttt{Option} type from
\emph{\href{https://hackage.haskell.org/package/type-combinators/docs/Data-Type-Option.html}{Data.Type.Option}},
from the
\emph{\href{https://hackage.haskell.org/package/type-combinators}{type-combinators}}
package:

\begin{Shaded}
\begin{Highlighting}[]
\KeywordTok{type} \DataTypeTok{Model'}\NormalTok{ (}\OtherTok{p ::} \DataTypeTok{Maybe} \DataTypeTok{Type}\NormalTok{) (}\OtherTok{s ::} \DataTypeTok{Maybe} \DataTypeTok{Type}\NormalTok{) (}\OtherTok{a ::} \DataTypeTok{Type}\NormalTok{) (}\OtherTok{b ::} \DataTypeTok{Type}\NormalTok{) }\FunctionTok{=}
\NormalTok{       forall z}\FunctionTok{.} \DataTypeTok{Reifies}\NormalTok{ z }\DataTypeTok{W}
    \OtherTok{=>} \DataTypeTok{Option}\NormalTok{ (}\DataTypeTok{BVar}\NormalTok{ z) p}
    \OtherTok{->} \DataTypeTok{BVar}\NormalTok{ z a}
    \OtherTok{->} \DataTypeTok{Option}\NormalTok{ (}\DataTypeTok{BVar}\NormalTok{ z) s}
    \OtherTok{->}\NormalTok{ (}\DataTypeTok{BVar}\NormalTok{ z b, }\DataTypeTok{Option}\NormalTok{ (}\DataTypeTok{BVar}\NormalTok{ z) s)}
\end{Highlighting}
\end{Shaded}

\texttt{Option\ f\ a} contains a value if \texttt{a} is
\texttt{\textquotesingle{}Just}, and does not if \texttt{a} is
\texttt{\textquotesingle{}Nothing}. More precisely, if \texttt{a} is
\texttt{\textquotesingle{}Just\ b}, it will contain an \texttt{f\ b}. So if
\texttt{p} is \texttt{\textquotesingle{}Just\ p\textquotesingle{}}, an
\texttt{Option\ (BVar\ z)\ p} will contain a
\texttt{BVar\ z\ p\textquotesingle{}}.

We can then re-define our previous types:

\begin{Shaded}
\begin{Highlighting}[]
\KeywordTok{type} \DataTypeTok{Model}\NormalTok{  p   }\FunctionTok{=} \DataTypeTok{Model'}\NormalTok{ (}\CharTok{'Just p) '}\DataTypeTok{Nothing}
\KeywordTok{type} \DataTypeTok{ModelS}\NormalTok{ p s }\FunctionTok{=} \DataTypeTok{Model'}\NormalTok{ (}\CharTok{'Just p) ('}\DataTypeTok{Just}\NormalTok{ s)}
\end{Highlighting}
\end{Shaded}

And now that we have unified everything under the same type, we can write
\texttt{mapS} that takes both stateful and non-stateful models, merge
\texttt{(\textless{}\textasciitilde{})},
\texttt{(\textless{}*\textasciitilde{}*)} and
\texttt{(\textless{}*\textasciitilde{})}, etc., thanks to the power of dependent
types.

As an added benefit, we also can unify parameterless functions too, which are
often useful for composition:

\begin{Shaded}
\begin{Highlighting}[]
\KeywordTok{type} \DataTypeTok{Func}\NormalTok{ a b }\FunctionTok{=}\NormalTok{ forall z}\FunctionTok{.} \DataTypeTok{Reifies}\NormalTok{ z }\DataTypeTok{W} \OtherTok{=>} \DataTypeTok{BVar}\NormalTok{ z a }\OtherTok{->} \DataTypeTok{BVar}\NormalTok{ z b}
\CommentTok{-- or}
\KeywordTok{type} \DataTypeTok{Func}     \FunctionTok{=} \DataTypeTok{Model'} \CharTok{'Nothing '}\DataTypeTok{Nothing}
\end{Highlighting}
\end{Shaded}

and we can use this with our unified \texttt{(\textless{}\textasciitilde{})}
etc. to implement functions like \texttt{mapS} for free.

Note that dependent types and DataKind shenanigans aren't necessary for any of
this to work --- it just has the possibility to make things even more seamless
and unified.

\hypertarget{a-practical-framework}{%
\section{A Practical Framework}\label{a-practical-framework}}

At the end of it all, I really think that we don't ever ``need'' a ``neural
network library'' or a ``neural network framework''. I don't want to be hemmed
into a specific opaque interface with a compositional API that requires me to
learn new rules of composition or application or clunky object methods.

To be able to utilize this all today, you really only need a few things.

\begin{itemize}
\item
  A handful of small primitive models expressed as normal functions (like
  \texttt{linReg}, \texttt{fullyConnected}, \texttt{convolution}, \texttt{lstm}
  etc.)

  The number of small primitives might be surprisingly small, given the
  combinators that we are able to write. However, basic fundamental primitives
  are important to be able to jump in and write any model you might need.
\item
  Some useful higher-order functions acting as utility combinators to common
  patterns of function composition, like \texttt{map},
  \texttt{\textless{}\textasciitilde{}}, etc.

  These are never \emph{required} --- just convenient, since the functional API
  is already fully featured as it is. They are all defined ``within the
  language'', in that you can always just implement them using normal function
  application and definition.

  Having these handy will make certain workflows simpler, and also help to
  de-duplicate common patterns that come up often.

  With these, models that seem seemingly very different can be defined in terms
  of simple combinator applications of other models, and that simple base models
  can be used to derive other models in surprising ways (like how a feed-forward
  layer can be turned into a recurrent layer or an autoregressive model)
\item
  A handy collection of (differentiable) \emph{loss functions}; in this post, we
  only used squared error, but in other situations there might be other useful
  ones like cross-entropy. Just having common loss functions (and combinators to
  manipulate loss functions) at hand is useful for quick prototyping.

  Loss functions can be combined with regularizing terms from parameters, if the
  regularization functions themselves are differentiable.
\item
  A handy collection of \emph{optimizers}, allowing you to take a loss function,
  a set of samples, and a model, and return the optimal parameters using
  performant optimizers.

  In this post we only used stochastic gradient descent, but other great
  optimizers out there are also worth having available, like momentum, adam,
  adagrad, etc.

  These optimizers should be easily usable with different data streams for
  observations.
\end{itemize}

That's really it, I feel! Just the models \emph{as functions}, the combinators,
and methods to evaluate and train those functions. No ``objects'' defining
layers as data (they're not data, they're functions!); just the full freedom of
expressing a model as any old function you want.\footnote{This is the basis
  behind my work-in-progress \href{https://github.com/mstksg/opto}{opto} and
  \href{https://github.com/mstksg/backprop-learn}{backprop-learn} libraries.}

\hypertarget{a-path-forward}{%
\section{A Path Forward}\label{a-path-forward}}

Thank you for making it to the end! I hope at this point you have been able to
gain some appreciation for differential programming in a purely functional
style, and see the sort of doors that this opens.

To tie it all together, I want to restate that a lot of things have to come
together to make this all practical and useful. And, without any one of these,
the whole thing would become clumsy.

\begin{enumerate}
\def\labelenumi{\arabic{enumi}.}
\item
  \textbf{Functional Programming}. Higher-order functions and combinators that
  take functions and return functions. Again, this allows us to draw from
  mathematical models directly, but also gives us full control over how we
  reshape, redefine, manipulate our models.

  We aren't forced to adhere to a limited API provided for our models; it all is
  just normal function application and higher-order functions --- something that
  functional programming is very, very good at dealing with. In addition,
  writing our models as ``just functions'' means we can re-use functional
  programming staples like \texttt{foldl} (left folds) and \texttt{mapAccumL}.

  Combinators are powerful --- we saw how many models were just
  ``combinator-applied'' versions of simpler models.

  Functional programming also forces us to consider state \emph{explicitly},
  instead of being an implicit part of the runtime. This makes combinators like
  \texttt{zeroState}, \texttt{unroll}, \texttt{recurrently}, and \texttt{lagged}
  possible. Because state is not a magic part of the system, it is something
  that we can \emph{explicitly talk about} and \emph{transform}, just as a
  first-class thing.
\item
  \textbf{Differentiable Programming}. This should go without saying that
  nothing here would work without our functions all being differentiable. This
  is what allows us to train our models using gradient descent.

  Again, I really don't know if this is best when supported at the
  language/compiler level or at the library level. For this exploration, it is
  done at the library level, and I really don't think it's too bad!

  In any case, I want to emphasize again that functional programming is a
  natural fit for differentiable programming, and the combination of them
  together is what makes this approach very powerful.
\item
  \textbf{Purely functional programming} is, again, what lets us draw the
  correspondence between mathematical models and the models we describe here.
  And, as seen in the last part, this constraint forces us to consider
  alternatives to implicit state, which ends up yielding very fruitful results.

  In impure languages, this is something that we have to always explicitly state
  as a property of our models. Purity is a \emph{benefit}, especially when
  reasoning with stateful models. Tying the state of our models with the
  implicit state functionality of a programming language's runtime system?
  Definitely a recipe for confusion and disaster.
\item
  \textbf{Strong expressive static type system} with type inference makes this
  all possible to work with at the practical level.

  I couldn't imagine doing any of this without the help of a compiler that keeps
  track of your types for you. Most of our combinators manipulate state types of
  functions, many of them manipulate parameter types, and almost all of them
  manipulate input and output types. Having a compiler that keeps track of this
  for you and lets you ask questions about them is essential. The compiler also
  \emph{helps you write your code} --- if you leave a ``typed hole'' in your
  code, the compiler will tell you all of the combinators or values available
  that can fit inside that hole, and it usually is exactly the one you need.

  And if you can state your desired model in terms of its types, sometimes the
  combinator applications and functions write themselves. They all act together
  as edges of puzzle pieces; and best of all, the compiler can tell you exactly
  what pieces you have available fit with what you have, automatically.
  Additionally, the process of thinking of types (within the language) can guide
  you in \emph{writing} new combinators.

  This method requires some complex types when you write non-trivial models;
  type inference frees you from the burden of keeping track of your parameter
  and state type, and has the compiler handle the work and the memory for you.
  And, at the end, when you have your finished model, your compiler will verify
  things like providing the right parameter to the right model, generating the
  correct parameter shape, etc.
\end{enumerate}

\hypertarget{comparisons}{%
\subsection{Comparisons}\label{comparisons}}

Almost all current neural network and deep learning frameworks implement the
full features that are described here. \emph{tensorflow} and related libraries
all provide a wrapper around essentially pure graph API. You can get started
with all of this right away in python with tools like
\href{https://github.com/HIPS/autograd}{autograd}.

What I'm really talking about isn't specifically about Haskell or
\emph{backprop}; it's more of a \emph{functional approach} to these sorts of
models. Currently right now, imperative API's dominate the field. Sometimes when
talking to friends, they can't imagine how a functional or pure API would make
sense.

The point of this series is to show that a functional and pure API with static
types isn't just possible, it's immensely beneficial:

\begin{itemize}
\item
  There is no need for an imperative API, even as a wrapper. Even imperative
  API's require an explicit assumption or promise of purity, anyway, that cannot
  be enforced --- so what's the point?
\item
  \emph{Layers as objects} (or as data) is not necessary. \emph{Layers as
  functions} is the more faithful and extensible way. Almost all frameworks
  (like \emph{\href{https://www.tensorflow.org/}{tensorflow}},
  \emph{\href{http://caffe.berkeleyvision.org/}{caffe}},
  \emph{\href{http://hackage.haskell.org/package/grenade-0.1.0}{grenade}}) fall
  into the this
  \href{https://docs.google.com/presentation/d/1UeKXVgRvvxg9OUdh_UiC5G71UMscNPlvArsWER41PsU/edit\#slide=id.gc2fcdcce7_216_264}{layer-as-data}
  mentality.

  For example, what if we wanted to turn a model \texttt{a\ -\textgreater{}\ b}
  (predicting b's from a's) into a model
  \texttt{{[}a{]}\ -\textgreater{}\ {[}b{]}} (predicting the contents of a list
  of b's from the contents of a list of a's)?

  In libraries like \emph{tensorflow} and \emph{caffe} and \emph{grenade}, you
  might have to:

  \begin{enumerate}
  \def\labelenumi{\arabic{enumi}.}
  \tightlist
  \item
    Create a new data structure
  \item
    Use the API of the layer data structure to implement a bunch of methods for
    your data structure
  \item
    Write a ``forward'' mode
  \item
    Write a ``backwards'' mode
  \item
    Define initializers for your data structure
  \item
    Write trainers/nudgers for your data structure
  \end{enumerate}

  But in this system where layers are functions, this is just:

\begin{Shaded}
\begin{Highlighting}[]
\OtherTok{overList ::} \DataTypeTok{Model}\NormalTok{ p a b }\OtherTok{->} \DataTypeTok{Model}\NormalTok{ p [a] [b]}
\NormalTok{overList f p }\FunctionTok{=}\NormalTok{ fmap (f p)}
\end{Highlighting}
\end{Shaded}

  There is some minor boilerplate to make the types line up (so it's actually
  \texttt{\textbackslash{}f\ -\textgreater{}\ collectVars\ .\ map\ f\ .\ sequenceVar}),
  but that's essentially what it is. No special data structure, no abstract API
  to work with\ldots{}just normal functions.
\item
  A functional and statically typed interface helps you, as a developer,
  \emph{explore options} in ways that an imperative or untyped approach cannot.
  Removing the barrier between the math and the code helps with your thinking.
  It also guides how you look at combinators and creating models from others.
  Functional approaches also mean you have to think of no implicit state
  interactions behind the hood.
\end{itemize}

In short, other similar frameworks might have some mix of of differentiable and
``functional'' programming, and some even with purity by contract. But it is
specifically the combination of \emph{all} of these (with static types) adds a
lot of value in how you create and use and discover models.

One thing I excluded from discussion here is performance. Performance is going
to be up to the system you use for differentiable programming, and so is not
something I can meaningfully talk about. My posts here are simply about
interface, and how they can help shape your thought when designing your own
models.

\hypertarget{signing-off}{%
\subsection{Signing off}\label{signing-off}}

In the end, this is all something that I'm still actively exploring. In a year
now, my opinions might be very different. However, I've reached a point where I
truly believe the future of differentiable programming and deep learning is
functional, pure, and typed. For me, however, functional, pure, and typed
differentiable programming is \emph{my present}. Its contributions to my
understanding of models and building new models is something that I take
advantage of every day in my own modeling and research. I hope it can be helpful
to you, as well!

\end{document}
