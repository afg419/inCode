\documentclass[]{article}
\usepackage{lmodern}
\usepackage{amssymb,amsmath}
\usepackage{ifxetex,ifluatex}
\usepackage{fixltx2e} % provides \textsubscript
\ifnum 0\ifxetex 1\fi\ifluatex 1\fi=0 % if pdftex
  \usepackage[T1]{fontenc}
  \usepackage[utf8]{inputenc}
\else % if luatex or xelatex
  \ifxetex
    \usepackage{mathspec}
    \usepackage{xltxtra,xunicode}
  \else
    \usepackage{fontspec}
  \fi
  \defaultfontfeatures{Mapping=tex-text,Scale=MatchLowercase}
  \newcommand{\euro}{€}
\fi
% use upquote if available, for straight quotes in verbatim environments
\IfFileExists{upquote.sty}{\usepackage{upquote}}{}
% use microtype if available
\IfFileExists{microtype.sty}{\usepackage{microtype}}{}
\usepackage[margin=1in]{geometry}
\usepackage{graphicx}
\makeatletter
\def\maxwidth{\ifdim\Gin@nat@width>\linewidth\linewidth\else\Gin@nat@width\fi}
\def\maxheight{\ifdim\Gin@nat@height>\textheight\textheight\else\Gin@nat@height\fi}
\makeatother
% Scale images if necessary, so that they will not overflow the page
% margins by default, and it is still possible to overwrite the defaults
% using explicit options in \includegraphics[width, height, ...]{}
\setkeys{Gin}{width=\maxwidth,height=\maxheight,keepaspectratio}
\ifxetex
  \usepackage[setpagesize=false, % page size defined by xetex
              unicode=false, % unicode breaks when used with xetex
              xetex]{hyperref}
\else
  \usepackage[unicode=true]{hyperref}
\fi
\hypersetup{breaklinks=true,
            bookmarks=true,
            pdfauthor={Justin Le},
            pdftitle={A Purely Functional Typed Approach to Trainable Models},
            colorlinks=true,
            citecolor=blue,
            urlcolor=blue,
            linkcolor=magenta,
            pdfborder={0 0 0}}
\urlstyle{same}  % don't use monospace font for urls
% Make links footnotes instead of hotlinks:
\renewcommand{\href}[2]{#2\footnote{\url{#1}}}
\setlength{\parindent}{0pt}
\setlength{\parskip}{6pt plus 2pt minus 1pt}
\setlength{\emergencystretch}{3em}  % prevent overfull lines
\setcounter{secnumdepth}{0}

\title{A Purely Functional Typed Approach to Trainable Models}
\author{Justin Le}

\begin{document}
\maketitle

\emph{Originally posted on
\textbf{\href{https://blog.jle.im/entry/purely-functional-typed-models.html}{in
Code}}.}

With the release of
\href{http://hackage.haskell.org/package/backprop}{backprop}, I've been
exploring the space of parameterized models of all sorts, from linear and
logistic regression and other statistical models to artificial neural networks,
feed-forward and recurrent (stateful). I wanted to see to what extent we can
really apply automatic differentiation and iterative gradient decent-based
training to all of these different models.

I'm starting to see a picture unifying all of these models, painted in the
language of purely typed functional programming. I'm already applying these to
models I'm using in real life and in my research, and I thought I'd take some
time to put my thoughts to writing in case anyone else finds these illuminating
or useful.

As a big picture, I really believe that a purely functional typed approach is
\emph{the} way to move forward in the future for models like artificial neural
networks -- and that one day, object-oriented and imperative approaches will
seem quaint.

I'm not the first person to attempt to build a conceptual framework for these
types of models in a purely functional typed sense --
\href{http://colah.github.io/posts/2015-09-NN-Types-FP/}{Christopher Olah's
famous post} comes to mind, and is definitely worth a read. However, Olah's post
is more of an abstract piece; the approach I am describing here can be applied
\emph{today}, to start building and \emph{discovering} effective models and
training them. And I have code! :)

The code in this post is written in Haskell, using the
\href{http://hackage.haskell.org/package/backprop}{backprop},
\href{http://hackage.haskell.org/package/hmatrix}{hmatrix} (with
\href{http://hackage.haskell.org/package/hmatrix-backprop}{hmatrix-backprop}),
and \href{http://hackage.haskell.org/package/vector-sized}{vector-sized}
libraries.

\hypertarget{essence-of-a-model}{%
\section{Essence of a Model}\label{essence-of-a-model}}

For the purpose of this post, a \emph{parameterized model} is a function from
some input ``question'' (predictor, independent variable) to some output
``answer'' (predictand, dependent variable)

Notationally, we might write it as a function:

{[} f\_p(x) =
y{]}(https://latex.codecogs.com/png.latex?\%0Af\_p\%28x\%29\%20\%3D\%20y\%0A "
f\_p(x) = y ``)

The important thing is that, for every choice of \emph{parameterization}
\includegraphics{https://latex.codecogs.com/png.latex?p}, we get a
\emph{different function}
\includegraphics{https://latex.codecogs.com/png.latex?f_p\%28x\%29}.

For example, you might want to write a model that, when given an email, outputs
whether or not that email is spam.

The parameterization \emph{p} is some piece of data that we tweak to produce a
different \includegraphics{https://latex.codecogs.com/png.latex?f_p\%28x\%29}.
So, ``training'' (or ``learning'', or ``estimating'') a model is a process of
picking the \includegraphics{https://latex.codecogs.com/png.latex?p} that gives
the ``correct'' function
\includegraphics{https://latex.codecogs.com/png.latex?f_p\%28x\%29} --- that is,
the function that accurately predicts spam or whatever thing you are trying to
predict.

For example, for linear regression, you are trying to ``fit'' your
\includegraphics{https://latex.codecogs.com/png.latex?\%28x\%2C\%20y\%29} data
points to some function
\includegraphics{https://latex.codecogs.com/png.latex?f\%28x\%29\%20\%3D\%20\%5Cbeta\%20\%2B\%20\%5Calpha\%20x}.
The \emph{parameters} are
\includegraphics{https://latex.codecogs.com/png.latex?\%5Calpha} and
\includegraphics{https://latex.codecogs.com/png.latex?\%5Cbeta}, the
\emph{input} is \includegraphics{https://latex.codecogs.com/png.latex?x}, and
the \emph{output} is
\includegraphics{https://latex.codecogs.com/png.latex?\%5Cbeta\%20\%2B\%20\%5Calpha\%20x}.

\hypertarget{optimizing-models-with-observations}{%
\subsection{Optimizing Models with
Observations}\label{optimizing-models-with-observations}}

Something interesting happens if we flip the script. What if, instead of
\includegraphics{https://latex.codecogs.com/png.latex?f_p\%28x\%29}, we talked
about \includegraphics{https://latex.codecogs.com/png.latex?f_x\%28p\%29}? That
is, we fix the input and vary the parameter, and see what type of outputs we get
for the same output while we vary the parameter?

If we have an ``expected output'' for our input, then one thing we can do is
look at \includegraphics{https://latex.codecogs.com/png.latex?f_p\%28x\%29} and
see when the result is close to
\includegraphics{https://latex.codecogs.com/png.latex?y_x} (the expected output
of our model when given
\includegraphics{https://latex.codecogs.com/png.latex?x}).

In fact, we can turn this into an optimization problem by trying to pick
\includegraphics{https://latex.codecogs.com/png.latex?p} that minimizes the
difference between
\includegraphics{https://latex.codecogs.com/png.latex?f_x\%28p\%29} and
\includegraphics{https://latex.codecogs.com/png.latex?y_x}. We can say that our
model with parameter \includegraphics{https://latex.codecogs.com/png.latex?p}
predicts \includegraphics{https://latex.codecogs.com/png.latex?y_x} the best
when we minimize:

{[} (f\_x(p) -
y\_x)\^{}2{]}(https://latex.codecogs.com/png.latex?\%0A\%28f\_x\%28p\%29\%20-\%20y\_x\%29\%5E2\%0A
" (f\_x(p) - y\_x)\^{}2 ``)

If we minimize the squared error between the result of picking the parameter and
the expected result, we find the best parameters for that given input!

In general, picking the best parameter for the model involves picking the
\includegraphics{https://latex.codecogs.com/png.latex?p} that minimizes the
relationship

{[} \textbackslash{}text\{loss\}(y\_x,
f\_x(p)){]}(https://latex.codecogs.com/png.latex?\%0A\%5Ctext\%7Bloss\%7D\%28y\_x\%2C\%20f\_x\%28p\%29\%29\%0A
" \text{loss}(y\_x, f\_x(p)) ``)

Where
\includegraphics{https://latex.codecogs.com/png.latex?\%5Ctext\%7Bloss\%7D\%28target\%2C\%20result\%29\%20\%3A\%20\%5Cmathbb\%7BR\%7D}
gives a measure of ``how badly'' the model result differs from the expected
target. Common loss functions include squared error, cross-entropy, etc.

This gives us a supervised way to train any model: if we have enough
observations
(\includegraphics{https://latex.codecogs.com/png.latex?\%28x\%2C\%20y_x\%29}
pairs) we can just pick a
\includegraphics{https://latex.codecogs.com/png.latex?p} that does its best to
make the loss between all observations as small as possible.

\hypertarget{stochastic-gradient-descent}{%
\subsection{Stochastic Gradient Descent}\label{stochastic-gradient-descent}}

If our model is a \emph{differentiable function}, then we have a nice tool we
can use: \emph{stochastic gradient descent}.

That is, we can always calculate the \emph{gradient} of the loss function with
respect to our parameters. This gives us the direction we can ``nudge'' our
parameters to make the loss bigger or smaller.

That is, if we get the gradient of the loss with respect to
\includegraphics{https://latex.codecogs.com/png.latex?p}:

{[} \textbackslash{}nabla\_p \textbackslash{}text\{loss\}(f\_x(p),
y\_x){]}(https://latex.codecogs.com/png.latex?\%0A\%5Cnabla\_p\%20\%5Ctext\%7Bloss\%7D\%28f\_x\%28p\%29\%2C\%20y\_x\%29\%0A
" \nabla\_p \text{loss}(f\_x(p), y\_x) ``)

We now have a nice way to ``train'' our model:

\begin{enumerate}
\def\labelenumi{\arabic{enumi}.}
\tightlist
\item
  Start with an initial guess at the parameter
\item
  Look at a random
  \includegraphics{https://latex.codecogs.com/png.latex?\%28x\%2C\%20y_x\%29}
  observation pair.
\item
  Compute the gradient
  \includegraphics{https://latex.codecogs.com/png.latex?\%5Cnabla_p\%20\%5Ctext\%7Bloss\%7D\%28f_x\%28p\%29\%2C\%20y_x\%29}
  of our current \includegraphics{https://latex.codecogs.com/png.latex?p}, which
  tells us a direction we can ``nudge''
  \includegraphics{https://latex.codecogs.com/png.latex?p} in to make the loss
  smaller.
\item
  Nudge \includegraphics{https://latex.codecogs.com/png.latex?p} in that
  direction
\item
  Pick a new
  \includegraphics{https://latex.codecogs.com/png.latex?\%28x\%2C\%20y_x\%29}
  observation pair.
\end{enumerate}

With every new observation, we see how we can nudge the parameter to make the
model more accurate, and then we perform that nudge.

\hypertarget{gradient-descent}{%
\subsection{Gradient Descent}\label{gradient-descent}}

\end{document}
