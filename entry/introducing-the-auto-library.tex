\documentclass[]{article}
\usepackage{lmodern}
\usepackage{amssymb,amsmath}
\usepackage{ifxetex,ifluatex}
\usepackage{fixltx2e} % provides \textsubscript
\ifnum 0\ifxetex 1\fi\ifluatex 1\fi=0 % if pdftex
  \usepackage[T1]{fontenc}
  \usepackage[utf8]{inputenc}
\else % if luatex or xelatex
  \ifxetex
    \usepackage{mathspec}
    \usepackage{xltxtra,xunicode}
  \else
    \usepackage{fontspec}
  \fi
  \defaultfontfeatures{Mapping=tex-text,Scale=MatchLowercase}
  \newcommand{\euro}{€}
\fi
% use upquote if available, for straight quotes in verbatim environments
\IfFileExists{upquote.sty}{\usepackage{upquote}}{}
% use microtype if available
\IfFileExists{microtype.sty}{\usepackage{microtype}}{}
\usepackage[margin=1in]{geometry}
\usepackage{color}
\usepackage{fancyvrb}
\newcommand{\VerbBar}{|}
\newcommand{\VERB}{\Verb[commandchars=\\\{\}]}
\DefineVerbatimEnvironment{Highlighting}{Verbatim}{commandchars=\\\{\}}
% Add ',fontsize=\small' for more characters per line
\newenvironment{Shaded}{}{}
\newcommand{\AlertTok}[1]{\textcolor[rgb]{1.00,0.00,0.00}{\textbf{#1}}}
\newcommand{\AnnotationTok}[1]{\textcolor[rgb]{0.38,0.63,0.69}{\textbf{\textit{#1}}}}
\newcommand{\AttributeTok}[1]{\textcolor[rgb]{0.49,0.56,0.16}{#1}}
\newcommand{\BaseNTok}[1]{\textcolor[rgb]{0.25,0.63,0.44}{#1}}
\newcommand{\BuiltInTok}[1]{#1}
\newcommand{\CharTok}[1]{\textcolor[rgb]{0.25,0.44,0.63}{#1}}
\newcommand{\CommentTok}[1]{\textcolor[rgb]{0.38,0.63,0.69}{\textit{#1}}}
\newcommand{\CommentVarTok}[1]{\textcolor[rgb]{0.38,0.63,0.69}{\textbf{\textit{#1}}}}
\newcommand{\ConstantTok}[1]{\textcolor[rgb]{0.53,0.00,0.00}{#1}}
\newcommand{\ControlFlowTok}[1]{\textcolor[rgb]{0.00,0.44,0.13}{\textbf{#1}}}
\newcommand{\DataTypeTok}[1]{\textcolor[rgb]{0.56,0.13,0.00}{#1}}
\newcommand{\DecValTok}[1]{\textcolor[rgb]{0.25,0.63,0.44}{#1}}
\newcommand{\DocumentationTok}[1]{\textcolor[rgb]{0.73,0.13,0.13}{\textit{#1}}}
\newcommand{\ErrorTok}[1]{\textcolor[rgb]{1.00,0.00,0.00}{\textbf{#1}}}
\newcommand{\ExtensionTok}[1]{#1}
\newcommand{\FloatTok}[1]{\textcolor[rgb]{0.25,0.63,0.44}{#1}}
\newcommand{\FunctionTok}[1]{\textcolor[rgb]{0.02,0.16,0.49}{#1}}
\newcommand{\ImportTok}[1]{#1}
\newcommand{\InformationTok}[1]{\textcolor[rgb]{0.38,0.63,0.69}{\textbf{\textit{#1}}}}
\newcommand{\KeywordTok}[1]{\textcolor[rgb]{0.00,0.44,0.13}{\textbf{#1}}}
\newcommand{\NormalTok}[1]{#1}
\newcommand{\OperatorTok}[1]{\textcolor[rgb]{0.40,0.40,0.40}{#1}}
\newcommand{\OtherTok}[1]{\textcolor[rgb]{0.00,0.44,0.13}{#1}}
\newcommand{\PreprocessorTok}[1]{\textcolor[rgb]{0.74,0.48,0.00}{#1}}
\newcommand{\RegionMarkerTok}[1]{#1}
\newcommand{\SpecialCharTok}[1]{\textcolor[rgb]{0.25,0.44,0.63}{#1}}
\newcommand{\SpecialStringTok}[1]{\textcolor[rgb]{0.73,0.40,0.53}{#1}}
\newcommand{\StringTok}[1]{\textcolor[rgb]{0.25,0.44,0.63}{#1}}
\newcommand{\VariableTok}[1]{\textcolor[rgb]{0.10,0.09,0.49}{#1}}
\newcommand{\VerbatimStringTok}[1]{\textcolor[rgb]{0.25,0.44,0.63}{#1}}
\newcommand{\WarningTok}[1]{\textcolor[rgb]{0.38,0.63,0.69}{\textbf{\textit{#1}}}}
\ifxetex
  \usepackage[setpagesize=false, % page size defined by xetex
              unicode=false, % unicode breaks when used with xetex
              xetex]{hyperref}
\else
  \usepackage[unicode=true]{hyperref}
\fi
\hypersetup{breaklinks=true,
            bookmarks=true,
            pdfauthor={Justin Le},
            pdftitle={Introducing: the Auto library!},
            colorlinks=true,
            citecolor=blue,
            urlcolor=blue,
            linkcolor=magenta,
            pdfborder={0 0 0}}
\urlstyle{same}  % don't use monospace font for urls
% Make links footnotes instead of hotlinks:
\renewcommand{\href}[2]{#2\footnote{\url{#1}}}
\setlength{\parindent}{0pt}
\setlength{\parskip}{6pt plus 2pt minus 1pt}
\setlength{\emergencystretch}{3em}  % prevent overfull lines
\setcounter{secnumdepth}{0}

\title{Introducing: the Auto library!}
\author{Justin Le}
\date{March 24, 2015}

\begin{document}
\maketitle

\emph{Originally posted on
\textbf{\href{https://blog.jle.im/entry/introducing-the-auto-library.html}{in
Code}}.}

\textbf{Auto}:
\href{https://github.com/mstksg/auto/blob/master/README.md}{README} (with
examples) / \href{http://hackage.haskell.org/package/auto}{hackage} /
\href{https://github.com/mstksg/auto/blob/master/tutorial/tutorial.md}{tutorial}
/ \href{https://github.com/mstksg/auto-examples}{examples} /
\href{https://github.com/mstksg/auto}{github}

(Before anything, maybe take a quick look at the detailed description in the
\href{https://github.com/mstksg/auto/blob/master/README.md}{README} for a quick
motivating example and explanation of the library)

Today I'm announcing and beginning promotion of my \emph{auto} library, a
denotative and locally stateful programming DSL and platform, now
\href{http://hackage.haskell.org/package/auto}{on hackage}. \emph{auto} is
suitable when your program involves an input or output that is a discrete stream
of things --- events, views, etc., like turn based games, GUI's, numerical
computations\ldots{}; it allows you to state (possibly cyclic) complex
relationships between streams of values by composing simple, primitive ones. You
can read the \href{https://github.com/mstksg/auto/blob/master/README.md}{README}
too for a detailed buzz-word laden exposition with nice well-commented short
demos and examples, get started with
\href{https://github.com/mstksg/auto/blob/master/tutorial/tutorial.md}{the
tutorial}, check out the directory of
\href{https://github.com/mstksg/auto-examples}{sample projects}, and even see a
live running \href{http://mstksg.github.io/auto-examples/todo/}{todoMVC}
(\href{https://github.com/mstksg/auto-examples/blob/master/src/Todo.hs}{source})
example!

Over the next week or two I'm going to be breaking down real-world projects
written on \emph{auto}, and even be talking about the design processes of
programs written using \emph{auto}. You can follow along on
\href{http://blog.jle.im/entries/series/+all-about-auto}{the series page},
follow me on \href{https://twitter.com/mstk}{twitter}, or just subscribe to the
\href{http://blog.jle.im/rss}{rss feed} feed; expect a post on designing, from
start to finish,

\begin{enumerate}
\def\labelenumi{\arabic{enumi}.}
\tightlist
\item
  A fully running chat bot
\item
  A GUI-based todo app on ghcjs
\item
  A text-based adventure game a la the classic
  \href{http://en.wikipedia.org/wiki/Rogue_\%28video_game\%29}{rogue}
\item
  A numerical computation DSL
\end{enumerate}

But enough of that\ldots{}what is \emph{auto}, why does it exist, and what are
its design principles?

\hypertarget{auto}{%
\section{Auto}\label{auto}}

\hypertarget{on-state}{%
\subsection{on State}\label{on-state}}

I designed \emph{auto} because there really aren't any good solutions in Haskell
for declaratively describing locally stateful programs in a compositional way.
And a lack of denotational semantics to reason with them.

The go-to implementation for a turn-based game is to have a ``giant state
monad''. It is a clever ``hack'', but really, all we've done is began
programming a game with \emph{global mutable state}. People have tried getting
over this by using lenses and zoomers, but this processes doesn't quite scale.

Even now many games, GUI's, numerical computations, etc. are written as folds or
state compositions over a giant state. Surely there is a better way?

\emph{auto} provides denotational semantics for the composition and
transformation of \emph{stream transformers}. At a high level, it is an example
of ``locally stateful programming''. Each component and stream transformer
really operates as its own entity, separate from the world, in a composable way.
Typical programs involve building a (possibly cyclic) graph of
\emph{relationships} between quantities over a stream.

Composing two transformers side-by-side or end-to-end creates a new
transformer\ldots{}and the state of each trasnformer is ``closed off'' from the
other and the rest of the world.

\begin{Shaded}
\begin{Highlighting}[]
\NormalTok{sumAndProduct }\FunctionTok{=}\NormalTok{ proc x }\OtherTok{->} \KeywordTok{do}
\NormalTok{    sums  }\OtherTok{<-}\NormalTok{ sumFrom }\DecValTok{0}     \FunctionTok{-<}\NormalTok{ x}
\NormalTok{    prods }\OtherTok{<-}\NormalTok{ productFrom }\DecValTok{1} \FunctionTok{-<}\NormalTok{ x}
\NormalTok{    id }\FunctionTok{-<}\NormalTok{ sums }\FunctionTok{+}\NormalTok{ prods}
\end{Highlighting}
\end{Shaded}

\texttt{sumFrom\ 0} denotes a relationship between \texttt{x} and \texttt{sums}
such that \texttt{sums} is the cumulative sum of all \texttt{x}'s seen.
\texttt{productFrom\ 1} denotes a relationship between \texttt{x} and
\texttt{prods} that \texttt{prods} is the cumulative product of all \texttt{x}s
seen. With \texttt{sumAndProduct}, we \emph{built} a new relationship --- the
output is the sum of the cumulative sum and the cumulative product of the inputs
--- by composing two primitives.

\begin{Shaded}
\begin{Highlighting}[]
\CommentTok{-- running our Autos over the stream [1..10] to get a new stream}
\NormalTok{ghci}\FunctionTok{>}\NormalTok{ streamAuto' (sumFrom }\DecValTok{0}\NormalTok{) [}\DecValTok{1}\FunctionTok{..}\DecValTok{10}\NormalTok{]}
\NormalTok{[}\DecValTok{1}\NormalTok{,}\DecValTok{3}\NormalTok{, }\DecValTok{6}\NormalTok{,}\DecValTok{10}\NormalTok{, }\DecValTok{15}\NormalTok{, }\DecValTok{21}\NormalTok{,  }\DecValTok{28}\NormalTok{,   }\DecValTok{36}\NormalTok{,    }\DecValTok{45}\NormalTok{,     }\DecValTok{55}\NormalTok{]}
\NormalTok{ghci}\FunctionTok{>}\NormalTok{ streamAuto' (productFrom }\DecValTok{1}\NormalTok{) [}\DecValTok{1}\FunctionTok{..}\DecValTok{10}\NormalTok{]}
\NormalTok{[}\DecValTok{1}\NormalTok{,}\DecValTok{2}\NormalTok{, }\DecValTok{6}\NormalTok{,}\DecValTok{24}\NormalTok{,}\DecValTok{120}\NormalTok{,}\DecValTok{720}\NormalTok{,}\DecValTok{5040}\NormalTok{,}\DecValTok{40320}\NormalTok{,}\DecValTok{362880}\NormalTok{,}\DecValTok{3628800}\NormalTok{]}
\NormalTok{ghci}\FunctionTok{>}\NormalTok{ streamAuto' sumAndProduct [}\DecValTok{1}\FunctionTok{..}\DecValTok{10}\NormalTok{]}
\NormalTok{[}\DecValTok{2}\NormalTok{,}\DecValTok{5}\NormalTok{,}\DecValTok{12}\NormalTok{,}\DecValTok{34}\NormalTok{,}\DecValTok{135}\NormalTok{,}\DecValTok{741}\NormalTok{,}\DecValTok{5068}\NormalTok{,}\DecValTok{40356}\NormalTok{,}\DecValTok{362925}\NormalTok{,}\DecValTok{3628855}\NormalTok{]}
\end{Highlighting}
\end{Shaded}

Each of them maintain their own ``state''\ldots{}and even \texttt{sumAndProduct}
will maintain its own internal state as you compose it with other things.

You build complex programs. For games, you might have an enemy monster, or a
player character\ldots{}why should a player character's update be able to access
the state of the enemy monster? Why should the enemy monster be able to access
the state of the player? Now, the enemy monster's state is only accessible to
the enemy monster itself\ldots{}nobody else can touch it.

In my opinion, this really is the only sane way to have stateful loops. Separate
out your update into the composition of primitives, and describe it using
denotative, pure \emph{relationships}.

\hypertarget{on-architecture}{%
\section{on Architecture}\label{on-architecture}}

Using \emph{auto}, your entire program is structured as a stream transformer,
built from the composition of smaller primitives, and using the various
combinators and semantic streams of the library.

In the end, ``running'' it is simply gathering the input one at a time, pushing
it through, and acting on the result.

For example, in the \href{http://mstksg.github.io/auto-examples/todo/}{todo
example}, the entire application is just one \texttt{Auto} that takes a stream
of input commands and outputs a stream of output GUI views. It's launch so that
the \texttt{Auto} waits on a \texttt{Chan} queue to consume inputs as they come
in. All the javascript front-end has to do is render the output gui view, and
hook up the DOM elements to trigger events that add new inputs to the queue.
That's it!

How do we build that \texttt{Auto}? By composing smaller, simpler ones. Each
denoting their own relationship, each keeping track of their state locally. The
``full GUI state'' never \emph{exists} anywhere\ldots{}it is aggregated during
composition.

By the way, \emph{auto} does allow you to take ``snap shots'' of the actual
states of \texttt{Auto}s as they are run, as a binary\ldots{}so you can
serialize, freeze, and resume \texttt{Auto}s from any previous state at-will.
Free undos, \emph{free save states}. And this serialization \textbf{composes},
so the combination of two serialized \texttt{Auto}s with internal state will
also be serialized appropriately.

\hypertarget{on-comparisons}{%
\subsection{on Comparisons}\label{on-comparisons}}

Throughout its development \emph{auto} has been compared to FRP libraries like
netwire. A full address of this comparison is offered on
\href{https://github.com/mstksg/auto\#relation-to-frp}{the readme}. The main
difference is that FRP offers an expressive language for working with,
manipulating, and transforming continuous-time behaviors. \emph{auto} borrows
some aspects of FRP as well as some practical API aspects in order to build
something separate, manipulating and transforming causal (discrete) streams.
There are many situations where FRP is not quite suited --- it'd be like using a
vector art program to describe a bitmap. There are domains where \emph{auto} is
not suited --- that is, the semantic model doesn't allow you to say anything
meaningful about continuous time behaviors.

\emph{auto} has also been compared to pipes and conduit, but there are some
major differences in design and philosophy. pipes is a more general-purpose
co-routine library with an emphasis on being able to apply equational reasoning
to many different domains. conduit focuses around the problem of effective
streaming with resource management. Both work ``sources'' that come from
underlying monads like IO; \emph{auto} discourages this except for disciplined
exceptions, and it's definitely not the norm. \emph{auto} works as \emph{value
stream transformers}, working with ``pure'' one-by-one transformations on
streams of \emph{values}; pipes and conduit provide \emph{effect stream
manipulators}, managing streams of \emph{effects} in constant space, with
resource management, etc\ldots{}and often involving output effects as a result
(``consumers'').\footnote{One could still use a subset of pipes that does not
  stream effects, but merely values, and \emph{that} does somewhat fill a
  similar role; this is used in the
  \href{https://hackage.haskell.org/package/mvc}{mvc} library to build similar
  applications that \emph{auto} tries to build. However, due to mvc's ``global
  state'' nature, you lose many of the local statefulness idioms in \emph{auto},
  and a lot of \emph{auto}'s benefits and design philosophies go away, for other
  reasons as well.}

\hypertarget{on-the-future}{%
\section{on the Future}\label{on-the-future}}

So, I'll be using this blog to post complete walk-throughs on designing specific
apps like the ones in the
\href{https://github.com/mstksg/auto-examples}{examples} repo. Along the way
hopefully some general design principles will become apparent, too. I've been
working on this library and have been writing ``real world'' code for it for
almost a year now, so I have some experience with design practices and patterns
that work and don't work.

For now, you can try reading over the
\href{https://github.com/mstksg/auto/blob/master/tutorial/tutorial.md}{tutorial}
if you want, or peruse the
\href{https://github.com/mstksg/auto-examples}{examples} repo!

Also, I definitely welcome any criticism on the design of the library or the
semantic model, or of its use cases. I've been more or less working on this
alone for almost a year, so now is the time for any delusions of mine to be
vetted out in public!

\hypertarget{on-support}{%
\subsection{on Support}\label{on-support}}

For now, the official support channel is \emph{\#haskell-auto} on freenode (I'm
\emph{jle`}), but you can always use the
\href{https://github.com/mstksg/issues}{issue tracker} too,
\href{mailto:justin@jle.im}{email me}, or find me on twitter as
\href{https://twitter.com/mstk}{mstk}.

All this being said, \emph{auto} is still kind of technically in a sorta
pre-release state, because not all of the tests are written yet. But the API
should be stable and updates before \texttt{0.3.x} are going to all be backwards
compatible (API-wise) bug fixes or filling in holes.

\hypertarget{signoff}{%
\section{Signoff}\label{signoff}}

Hi, thanks for reading! You can reach me via email at
\href{mailto:justin@jle.im}{\nolinkurl{justin@jle.im}}, or at twitter at
\href{https://twitter.com/mstk}{@mstk}! This post and all others are published
under the \href{https://creativecommons.org/licenses/by-nc-nd/3.0/}{CC-BY-NC-ND
3.0} license. Corrections and edits via pull request are welcome and encouraged
at \href{https://github.com/mstksg/inCode}{the source repository}.

If you feel inclined, or this post was particularly helpful for you, why not
consider \href{https://www.patreon.com/justinle/overview}{supporting me on
Patreon}, or a \href{bitcoin:3D7rmAYgbDnp4gp4rf22THsGt74fNucPDU}{BTC donation}?
:)

\end{document}
