\documentclass[]{article}
\usepackage{lmodern}
\usepackage{amssymb,amsmath}
\usepackage{ifxetex,ifluatex}
\usepackage{fixltx2e} % provides \textsubscript
\ifnum 0\ifxetex 1\fi\ifluatex 1\fi=0 % if pdftex
  \usepackage[T1]{fontenc}
  \usepackage[utf8]{inputenc}
\else % if luatex or xelatex
  \ifxetex
    \usepackage{mathspec}
    \usepackage{xltxtra,xunicode}
  \else
    \usepackage{fontspec}
  \fi
  \defaultfontfeatures{Mapping=tex-text,Scale=MatchLowercase}
  \newcommand{\euro}{€}
\fi
% use upquote if available, for straight quotes in verbatim environments
\IfFileExists{upquote.sty}{\usepackage{upquote}}{}
% use microtype if available
\IfFileExists{microtype.sty}{\usepackage{microtype}}{}
\usepackage[margin=1in]{geometry}
\ifxetex
  \usepackage[setpagesize=false, % page size defined by xetex
              unicode=false, % unicode breaks when used with xetex
              xetex]{hyperref}
\else
  \usepackage[unicode=true]{hyperref}
\fi
\hypersetup{breaklinks=true,
            bookmarks=true,
            pdfauthor={Justin Le},
            pdftitle={Interpreters a la Carte},
            colorlinks=true,
            citecolor=blue,
            urlcolor=blue,
            linkcolor=magenta,
            pdfborder={0 0 0}}
\urlstyle{same}  % don't use monospace font for urls
% Make links footnotes instead of hotlinks:
\renewcommand{\href}[2]{#2\footnote{\url{#1}}}
\setlength{\parindent}{0pt}
\setlength{\parskip}{6pt plus 2pt minus 1pt}
\setlength{\emergencystretch}{3em}  % prevent overfull lines
\setcounter{secnumdepth}{0}

\title{Interpreters a la Carte}
\author{Justin Le}

\begin{document}
\maketitle

\emph{Originally posted on
\textbf{\href{https://blog.jle.im/entry/interpreters-a-la-carte.html}{in
Code}}.}

This post is just a fun one exploring a wide range of techniques that I applied
to solve the Day 18 puzzles of this year's great
\href{http://adventofcode.com/2017}{Advent of Code}. The puzzles involved
interpreting an assembly language on an abstract machine. The neat twist is that
Part 1 gave you a description of one abstract machine, and Part 2 gave you a
\emph{different} abstract machine to interpret the same language in. This twist
(one language, but different interpreters/abstract machines) is basically one of
the textbook applications of the \emph{interpreter pattern} in Haskell and
functional programming, so it was fun to implement my solution in that pattern
-- the assembly language source was ``compiled'' to an abstract data type once,
and the difference between Part 1 and Part 2 was just a different choice of
interpreter.

Even more interesting is that the two machines are only ``half different'' --
there's one aspect of the virtual machines that are the same between the two
parts, and aspect that is different. This means that we can apply the ``data
types a la carte'' technique in order to mix and match isolated components of
virtual machine interpreters, and re-use code whenever possible in assembling
our interpreters for our different machines!

\end{document}
