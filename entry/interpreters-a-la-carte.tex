\documentclass[]{article}
\usepackage{lmodern}
\usepackage{amssymb,amsmath}
\usepackage{ifxetex,ifluatex}
\usepackage{fixltx2e} % provides \textsubscript
\ifnum 0\ifxetex 1\fi\ifluatex 1\fi=0 % if pdftex
  \usepackage[T1]{fontenc}
  \usepackage[utf8]{inputenc}
\else % if luatex or xelatex
  \ifxetex
    \usepackage{mathspec}
    \usepackage{xltxtra,xunicode}
  \else
    \usepackage{fontspec}
  \fi
  \defaultfontfeatures{Mapping=tex-text,Scale=MatchLowercase}
  \newcommand{\euro}{€}
\fi
% use upquote if available, for straight quotes in verbatim environments
\IfFileExists{upquote.sty}{\usepackage{upquote}}{}
% use microtype if available
\IfFileExists{microtype.sty}{\usepackage{microtype}}{}
\usepackage[margin=1in]{geometry}
\ifxetex
  \usepackage[setpagesize=false, % page size defined by xetex
              unicode=false, % unicode breaks when used with xetex
              xetex]{hyperref}
\else
  \usepackage[unicode=true]{hyperref}
\fi
\hypersetup{breaklinks=true,
            bookmarks=true,
            pdfauthor={},
            pdftitle={},
            colorlinks=true,
            citecolor=blue,
            urlcolor=blue,
            linkcolor=magenta,
            pdfborder={0 0 0}}
\urlstyle{same}  % don't use monospace font for urls
% Make links footnotes instead of hotlinks:
\renewcommand{\href}[2]{#2\footnote{\url{#1}}}
\setlength{\parindent}{0pt}
\setlength{\parskip}{6pt plus 2pt minus 1pt}
\setlength{\emergencystretch}{3em}  % prevent overfull lines
\setcounter{secnumdepth}{0}


\begin{document}

\% Interpreters a la Carte (Advent of Code 2017 Duet) \% Justin Le

\emph{Originally posted on
\textbf{\href{https://blog.jle.im/entry/interpreters-a-la-carte.html}{in
Code}}.}

This post is just a fun one exploring a wide range of techniques that I applied
to solve the Day 18 puzzles of this year's great
\href{http://adventofcode.com/2017}{Advent of Code}. The puzzles involved
interpreting an assembly language on an abstract machine. The neat twist is that
Part A gave you a description of one abstract machine, and Part B gave you a
\emph{different} abstract machine to interpret the same language in. This twist
(one language, but different interpreters/abstract machines) is basically one of
the textbook applications of the \emph{interpreter pattern} in Haskell and
functional programming, so it was fun to implement my solution in that pattern
-\/- the assembly language source was "compiled" to an abstract data type once,
and the difference between Part A and Part B was just a different choice of
interpreter.

Even more interesting is that the two machines are only "half different" -\/-
there's one aspect of the virtual machines that are the same between the two
parts, and aspect that is different. This means that we can apply the "data
types a la carte" technique in order to mix and match isolated components of
virtual machine interpreters, and re-use code whenever possible in assembling
our interpreters for our different machines!

This blog post will not necessarily be a focused tutorial on this trick, but
rather an explanation on my solution centered around this pattern, hopefully
providing insight on how I approach and solve non-trivial Haskell problems.
Along the way we'll also use mtl typeclasses and classy lenses.

The source code is
\href{https://github.com/mstksg/inCode/tree/master/code-samples/interpreters/Duet.hs}{available
online} and is executable as a stack script.

\section{The Puzzle}

The puzzle is \href{http://adventofcode.com/2017/day/18}{Advent of Code 2017 Day
18}, and Part A is:

\begin{quote}
You discover a tablet containing some strange assembly code labeled simply
"Duet". Rather than bother the sound card with it, you decide to run the code
yourself. Unfortunately, you don't see any documentation, so you're left to
figure out what the instructions mean on your own.

It seems like the assembly is meant to operate on a set of \emph{registers} that
are each named with a single letter and that can each hold a single integer. You
suppose each register should start with a value of \texttt{0}.

There aren't that many instructions, so it shouldn't be hard to figure out what
they do. Here's what you determine:

\begin{itemize}
\tightlist
\item
  \texttt{snd\ X} \emph{plays a sound} with a frequency equal to the value of
  \texttt{X}.
\item
  \texttt{set\ X\ Y} \emph{sets} register \texttt{X} to the value of \texttt{Y}.
\item
  \texttt{add\ X\ Y} \emph{increases} register \texttt{X} by the value of
  \texttt{Y}.
\item
  \texttt{mul\ X\ Y} sets register \texttt{X} to the result of
  \emph{multiplying} the value contained in register \texttt{X} by the value of
  \texttt{Y}.
\item
  \texttt{mod\ X\ Y} sets register \texttt{X} to the \emph{remainder} of
  dividing the value contained in register \texttt{X} by the value of \texttt{Y}
  (that is, it sets \texttt{X} to the result of \texttt{X} modulo \texttt{Y}).
\item
  \texttt{rcv\ X} \emph{recovers} the frequency of the last sound played, but
  only when the value of \texttt{X} is not zero. (If it is zero, the command
  does nothing.)
\item
  \texttt{jgz\ X\ Y} \emph{jumps} with an offset of the value of \texttt{Y}, but
  only if the value of \texttt{X} is \emph{greater than zero}. (An offset of
  \texttt{2} skips the next instruction, an offset of \texttt{-1} jumps to the
  previous instruction, and so on.)
\end{itemize}

Many of the instructions can take either a register (a single letter) or a
number. The value of a register is the integer it contains; the value of a
number is that number.

After each \emph{jump} instruction, the program continues with the instruction
to which the \emph{jump} jumped. After any other instruction, the program
continues with the next instruction. Continuing (or jumping) off either end of
the program terminates it.

\emph{What is the value of the recovered frequency} (the value of the most
recently played sound) the \emph{first} time a \texttt{rcv} instruction is
executed with a non-zero value?
\end{quote}

Part B, however, says:

\begin{quote}
As you congratulate yourself for a job well done, you notice that the
documentation has been on the back of the tablet this entire time. While you
actually got most of the instructions correct, there are a few key differences.
This assembly code isn't about sound at all - it's meant to be run \emph{twice
at the same time}.

Each running copy of the program has its own set of registers and follows the
code independently - in fact, the programs don\textbackslash{}'t even
necessarily run at the same speed. To coordinate, they use the \emph{send}
(\texttt{snd}) and \emph{receive} (\texttt{rcv}) instructions:

\begin{itemize}
\tightlist
\item
  \texttt{snd\ X} \emph{sends} the value of \texttt{X} to the other program.
  These values wait in a queue until that program is ready to receive them. Each
  program has its own message queue, so a program can never receive a message it
  sent.
\item
  \texttt{rcv\ X} \emph{receives} the next value and stores it in register
  \texttt{X}. If no values are in the queue, the program \emph{waits for a value
  to be sent to it}. Programs do not continue to the next instruction until they
  have received a value. Values are received in the order they are sent.
\end{itemize}

Each program also has its own \emph{program ID} (one \texttt{0} and the other
\texttt{1}); the register \texttt{p} should begin with this value.

Once both of your programs have terminated (regardless of what caused them to do
so), \emph{how many times did program \texttt{1} send a value}?
\end{quote}

Note that in each of these, "the program" is a program (written in the Duet
assembly language), which is different for each user and given to us by the
site.

What's going on here is that both parts execute the same program in two
different virtual machines -\/- one has "sound" and "recover", and the other has
"send" and "receive". We are supposed to run the same program in \emph{both} of
these machines.

However, note that these two machines aren't \emph{completely} different -\/-
they both have the ability to manipulate memory and read/shift program data. So
really , we want to be able to create a "modular" spec and implementation of
these machines, so that we may re-use this memory manipulation aspect when
constructing our machine, without duplicating any code.

\section{Parsing Duet}

First, let's get the parsing of the actual input program out of the way. We'll
be parsing a program into a list of "ops" that we will read as our program.

Our program will be interpreted as a list of \texttt{Op} values, a data type
representing opcodes. There are four categories: "snd", "rcv", "jgz", and the
binary mathematical operations:

```haskell -\/- source:
https://github.com/mstksg/inCode/tree/master/code-samples/interpreters/Duet.hs\#L31-L36
type Addr = Either Char Int

data Op = OSnd Addr \textbar{} ORcv Char \textbar{} OJgz Addr Addr \textbar{}
OBin (Int -\textgreater{} Int -\textgreater{} Int) Char Addr ```

It's important to remember that "snd", "jgz", and the binary operations can all
take either numbers or other registers.

Now, parsing a single \texttt{Op} is just a matter of pattern matching on
\texttt{words}:

\texttt{haskell\ -\/-\ source:\ https://github.com/mstksg/inCode/tree/master/code-samples/interpreters/Duet.hs\#L38-L51\ parseOp\ ::\ String\ -\textgreater{}\ Op\ parseOp\ inp\ =\ case\ words\ inp\ of\ \ \ \ \ "snd":c\ \ \ \ :\_\ \ \ -\textgreater{}\ OSnd\ (addr\ c)\ \ \ \ \ "set":(x:\_):y:\_\ -\textgreater{}\ OBin\ (const\ id)\ x\ (addr\ y)\ \ \ \ \ "add":(x:\_):y:\_\ -\textgreater{}\ OBin\ (+)\ \ \ \ \ \ \ \ x\ (addr\ y)\ \ \ \ \ "mul":(x:\_):y:\_\ -\textgreater{}\ OBin\ (*)\ \ \ \ \ \ \ \ x\ (addr\ y)\ \ \ \ \ "mod":(x:\_):y:\_\ -\textgreater{}\ OBin\ mod\ \ \ \ \ \ \ \ x\ (addr\ y)\ \ \ \ \ "rcv":(x:\_):\_\ \ \ -\textgreater{}\ ORcv\ x\ \ \ \ \ "jgz":x\ \ \ \ :y:\_\ -\textgreater{}\ OJgz\ (addr\ x)\ (addr\ y)\ \ \ \ \ \_\ \ \ \ \ \ \ \ \ \ \ \ \ \ \ -\textgreater{}\ error\ "Bad\ parse"\ \ \ where\ \ \ \ \ addr\ ::\ String\ -\textgreater{}\ Addr\ \ \ \ \ addr\ {[}c{]}\ \textbar{}\ isAlpha\ c\ =\ Left\ c\ \ \ \ \ addr\ str\ =\ Right\ (read\ str)}

We're going to store our program in a \texttt{PointedList} from the
\emph{\href{http://hackage.haskell.org/package/pointedlist}{pointedlist}}
package, which is a non-empty list with a "focus" at a given index, which we use
to represent the program counter/program head/current instruction. Parsing our
program is then just parsing each line in the program string, and collecting
them into a \texttt{PointedList}. We're ready to go!

\texttt{haskell\ -\/-\ source:\ https://github.com/mstksg/inCode/tree/master/code-samples/interpreters/Duet.hs\#L53-L54\ parseProgram\ ::\ String\ -\textgreater{}\ P.PointedList\ Op\ parseProgram\ =\ fromJust\ .\ P.fromList\ .\ map\ parseOp\ .\ lines}

\section{Our Virtual Machine}

\subsection{MonadPrompt}

We're going to be using the great
\emph{\href{http://hackage.haskell.org/package/MonadPrompt}{MonadPrompt}}
library{[}\^{}mprompt{]} to build our representation of our interpreted
language. Another common choice is to use
\emph{\href{http://hackage.haskell.org/package/free}{free}}, and a lot of other
tutorials go down this route. However, \emph{free} is a bit more power than you
really need for the interpreter pattern, and I always felt like the
implementation of interpreter pattern programs in \emph{free} was a bit awkward.

\emph{MonadPrompt} lets us construct a language (and a monad) using GADTs to
represent command primitives. For example, to implement something like
\texttt{State\ Int}, you might use this GADT:

\texttt{haskell\ data\ StateCommand\ ::\ Type\ -\textgreater{}\ Type\ where\ \ \ \ \ Put\ ::\ Int\ -\textgreater{}\ StateCommand\ ()\ \ \ \ \ Get\ ::\ StateCommand\ Int}

Which says that the two "primitive" commands of \texttt{State\ Int} are
"putting" (which requires an \texttt{Int} and produces a \texttt{()} result) and
"getting" (which requires no inputs, and produces an \texttt{Int} result).

You can then write \texttt{State\ Int} as:

\texttt{haskell\ type\ IntState\ =\ Prompt\ StateCommand}

And our primitives can be constructed using:

```haskell prompt :: StateCommand a -\textgreater{} IntState a

prompt (Put 10) :: IntState () prompt Get :: IntState Int ```

Now, we \emph{interpret} an \texttt{IntState} in a monadic context using
\texttt{runPromptM}:

\texttt{haskell\ runPromptM\ \ \ \ \ ::\ Monad\ m\ \ \ \ \ \ \ \ \ \ \ \ \ \ \ \ \ \ \ \ \ \ \ \ \ \ \ \ \ \ -\/-\ m\ is\ the\ monad\ to\ interpret\ in\ \ \ \ \ =\textgreater{}\ (forall\ x.\ StateCommand\ x\ -\textgreater{}\ m\ x)\ \ \ \ -\/-\ a\ way\ to\ interpret\ each\ primitive\ in\ \textquotesingle{}m\textquotesingle{}\ \ \ \ \ -\textgreater{}\ IntState\ a\ \ \ \ \ \ \ \ \ \ \ \ \ \ \ \ \ \ \ \ \ \ \ \ \ \ \ -\/-\ IntState\ to\ interpret\ \ \ \ \ -\textgreater{}\ m\ a\ \ \ \ \ \ \ \ \ \ \ \ \ \ \ \ \ \ \ \ \ \ \ \ \ \ \ \ \ \ \ \ \ \ -\/-\ resulting\ action\ in\ \textquotesingle{}m\textquotesingle{}}

If you're unfamiliar with \emph{-XRankNTypes},
\texttt{forall\ x.\ StateCommand\ x\ -\textgreater{}\ m\ x} is the type of a
handler that can handle a \texttt{StateCommand} of \emph{any} type, and return a
value of \texttt{m\ x} (an action returning the \emph{same type} as the
\texttt{StateCommand}). So, you can't give it something like
\texttt{StateCommand\ Int\ -\textgreater{}\ m\ Bool}, or
\texttt{StateCommand\ x\ -\textgreater{}\ m\ ()}...it has to be able to handle a
\texttt{StateCommand\ a} of \emph{any} type \texttt{a} and return an action in
the interpreting context producing a result of the same type. If given a
\texttt{StateCommand\ Int}, it has to return an \texttt{m\ Int}, and if given a
\texttt{StateCommand\ ()}, it has to return an \texttt{m\ ()}, etc. etc.

Now, if we wanted to use \texttt{IO} and \texttt{IORefs} as the mechanism for
interpreting our \texttt{IntState}:

```haskell interpretIO :: IORef Int -\textgreater{} StateCommand a
-\textgreater{} IO a interpretIO r = \textbackslash{}case -\/- using
-XLambdaCase Put x -\textgreater{} writeIORef r x Get -\textgreater{} readIORef
r

runAsIO :: IntState a -\textgreater{} Int -\textgreater{} IO a runAsIO m s0 = do
r \textless{}- newIORef s0 runPromptM (interpretIO r) m ```

\texttt{interpretIO} is our interpreter, in \texttt{IO}. \texttt{runPromptM}
will interpret each primitive (\texttt{Put} and \texttt{Get}) using
\texttt{interpretIO}, and generate the result for us.

Note that the GADT property of \texttt{StateCommand} ensures us that the
\emph{result} of our \texttt{IO} action matches with the result that the GADT
constructor implies, due to the magic of dependent pattern matching. For the
\texttt{Put\ x\ ::\ StateCommand\ ()} branch, the result has to be
\texttt{IO\ ()}; for the \texttt{Get\ ::\ StateCommand\ Int} branch, the result
has to be \texttt{IO\ Int}.

We can also be boring and interpret it using \texttt{State\ Int}:

```haskell interpretState :: StateCommand a -\textgreater{} State Int a
interpretState = \textbackslash{}case Put x -\textgreater{} put x Get
-\textgreater{} get

runAsState :: IntState a -\textgreater{} State Int a runAsState = runPormptM
interpretState ```

Basically, an \texttt{IntState\ a} is an abstract representation of a program
(as a Monad), and \texttt{interpretIO} and \texttt{interpretState} are different
ways of \emph{interpreting} that program, in different monadic contexts. To
"run" or interpret our program in a context, we provide a function
\texttt{forall\ x.\ StateCommand\ x\ -\textgreater{}\ m\ x}, which interprets
each individual primitive command.

\subsection{Duet Commands}

Now let's specify the "primitives" of our program. It'll be useful to separate
out the "memory-based" primitive commands from the "communication-based"
primitive commands. This is so that we can write interpreters that operate on
each one individually.

For memory, we can access and modify register values, as well as jump around in
the program tape and read the \texttt{Op} at the current program head:

\texttt{haskell\ -\/-\ source:\ https://github.com/mstksg/inCode/tree/master/code-samples/interpreters/Duet.hs\#L56-L60\ data\ Mem\ ::\ Type\ -\textgreater{}\ Type\ where\ \ \ \ \ MGet\ ::\ Char\ -\textgreater{}\ Mem\ Int\ \ \ \ \ MSet\ ::\ Char\ -\textgreater{}\ Int\ -\textgreater{}\ Mem\ ()\ \ \ \ \ MJmp\ ::\ Int\ -\textgreater{}\ Mem\ ()\ \ \ \ \ MPk\ \ ::\ Mem\ Op}

For communication, we must be able to "snd" and "rcv".

\texttt{haskell\ -\/-\ source:\ https://github.com/mstksg/inCode/tree/master/code-samples/interpreters/Duet.hs\#L62-L64\ data\ Com\ ::\ Type\ -\textgreater{}\ Type\ where\ \ \ \ \ CSnd\ ::\ Int\ -\textgreater{}\ Com\ ()\ \ \ \ \ CRcv\ ::\ Int\ -\textgreater{}\ Com\ Int}

Part A requires \texttt{CRcv} to take, as an argument, a number, since whether
or not \texttt{CRcv} is a no-op depends on the value of a certain register for
Part A's virtual machine.

Now, we can leverage the \texttt{:\textbar{}:} type from
\emph{\href{http://hackage.haskell.org/package/type-combinators}{type-combinators}}:

\texttt{haskell\ data\ (f\ :\textbar{}:\ g)\ a\ =\ L\ (f\ a)\ \ \ \ \ \ \ \ \ \ \ \ \ \ \ \ \ \ \textbar{}\ R\ (g\ a)}

\texttt{:\textbar{}:} is a "functor disjunction" -\/- a value of type
\texttt{(f\ :\textbar{}:\ g)\ a} is either \texttt{f\ a} or \texttt{g\ a}.
\texttt{:\textbar{}:} is in \emph{base} twice, as \texttt{:+:} in
\emph{GHC.Generics} and as \texttt{Sum} in \emph{Data.Functor.Sum}. However, the
version in \emph{type-combinators} has some nice utility combinators we will be
using and is more fully-featured.

We can use \texttt{:\textbar{}:} to create the type
\texttt{Mem\ :\textbar{}:\ Com}. If \texttt{Mem} and \texttt{Com} represent
"primitives" in our Duet language, then \texttt{Mem\ :\textbar{}:\ Com}
represents \emph{primitives from \texttt{Mem} and \texttt{Com} together}. It's a
type that contains all of the primitives of \texttt{Mem} and the primitives of
\texttt{Com} -\/- that is, it contains:

\texttt{haskell\ L\ (MGet\ \textquotesingle{}c\textquotesingle{})\ ::\ (Mem\ :\textbar{}:\ Com)\ Int\ L\ MPk\ \ \ \ \ \ \ \ ::\ (Mem\ :\textbar{}:\ Com)\ Op\ R\ (CSnd\ 5)\ \ \ ::\ (Mem\ :\textbar{}:\ Com)\ ()}

etc.

Our final data type then -\/- a monad that encompasses \emph{all} possible Duet
primitive commands, is:

\texttt{haskell\ -\/-\ source:\ https://github.com/mstksg/inCode/tree/master/code-samples/interpreters/Duet.hs\#L66-L66\ type\ Duet\ =\ Prompt\ (Mem\ :\textbar{}:\ Com)}

We can write some convenient utility primitives to make things easier for us in
the long run:

```haskell -\/- source:
https://github.com/mstksg/inCode/tree/master/code-samples/interpreters/Duet.hs\#L68-L84
dGet :: Char -\textgreater{} Duet Int dGet = prompt . L . MGet

dSet :: Char -\textgreater{} Int -\textgreater{} Duet () dSet r = prompt . L .
MSet r

dJmp :: Int -\textgreater{} Duet () dJmp = prompt . L . MJmp

dPk :: Duet Op dPk = prompt (L MPk)

dSnd :: Int -\textgreater{} Duet () dSnd = prompt . R . CSnd

dRcv :: Int -\textgreater{} Duet Int dRcv = prompt . R . CRcv ```

\subsection{Constructing Duet Programs}

Armed with our \texttt{Duet} monad, we can now write a real-life \texttt{Duet}
action to represent \emph{one step} of our duet programs:

\texttt{haskell\ -\/-\ source:\ https://github.com/mstksg/inCode/tree/master/code-samples/interpreters/Duet.hs\#L86-L107\ stepProg\ ::\ Duet\ ()\ stepProg\ =\ dPk\ \textgreater{}\textgreater{}=\ \textbackslash{}case\ \ \ \ \ OSnd\ x\ -\textgreater{}\ do\ \ \ \ \ \ \ dSnd\ =\textless{}\textless{}\ addrVal\ x\ \ \ \ \ \ \ dJmp\ 1\ \ \ \ \ OBin\ f\ x\ y\ -\textgreater{}\ do\ \ \ \ \ \ \ yVal\ \textless{}-\ addrVal\ y\ \ \ \ \ \ \ xVal\ \textless{}-\ dGet\ \ \ \ x\ \ \ \ \ \ \ dSet\ x\ \$\ f\ xVal\ yVal\ \ \ \ \ \ \ dJmp\ 1\ \ \ \ \ ORcv\ x\ -\textgreater{}\ do\ \ \ \ \ \ \ y\ \textless{}-\ dRcv\ =\textless{}\textless{}\ dGet\ x\ \ \ \ \ \ \ dSet\ x\ y\ \ \ \ \ \ \ dJmp\ 1\ \ \ \ \ OJgz\ x\ y\ -\textgreater{}\ do\ \ \ \ \ \ \ xVal\ \textless{}-\ addrVal\ x\ \ \ \ \ \ \ dJmp\ =\textless{}\textless{}\ if\ xVal\ \textgreater{}\ 0\ \ \ \ \ \ \ \ \ \ \ \ \ \ \ \ \ \ then\ addrVal\ y\ \ \ \ \ \ \ \ \ \ \ \ \ \ \ \ \ \ else\ return\ 1\ \ \ where\ \ \ \ \ addrVal\ (Left\ r\ )\ =\ dGet\ r\ \ \ \ \ addrVal\ (Right\ x)\ =\ return\ x}

This is basically a straightforward interpretation of the "rules" of our
language, and what to do when encountering each op code.

The only non-trivial thing is the \texttt{ORcv} branch, where we include the
contents of the register in question, so that our interpreter will know whether
or not to treat it as a no-op.

\section{The Interpreters}

Now for the fun part!

\subsection{Interpreting Memory Primitives}

To interpret our \texttt{Mem} primitives, we need to be in some sort of stateful
monad that contains the program state. First, let's make a type describing our
relevant program state, along with classy lenses for operating on it
polymorphically:

\texttt{haskell\ -\/-\ source:\ https://github.com/mstksg/inCode/tree/master/code-samples/interpreters/Duet.hs\#L109-L112\ data\ ProgState\ =\ PS\ \{\ \_psTape\ ::\ P.PointedList\ Op\ \ \ \ \ \ \ \ \ \ \ \ \ \ \ \ \ \ \ \ \ ,\ \_psRegs\ ::\ M.Map\ Char\ Int\ \ \ \ \ \ \ \ \ \ \ \ \ \ \ \ \ \ \ \ \ \}\ makeClassy\ \textquotesingle{}\textquotesingle{}ProgState}

\subsubsection{Brief Aside on Lenses with State}

Using \emph{\href{http://hackage.haskell.org/package/lens}{lens}} with lenses
(especially classy ones) is one of the only things that makes programming
against \texttt{State} with non-trivial state bearable for me! We store the
current program and program head with the \texttt{PointedList}, and also
represent the register contents with a \texttt{Map\ Char\ Int}.

\texttt{makeClassy} gives us a typeclass \texttt{HasProgState}, which is for
things that "have" a \texttt{ProgState}, as well as lenses into the
\texttt{psTape} and \texttt{psRegs} field for that type. We can use these lenses
with \emph{lens} library machinery:

```haskell -\/- \textbar{} "get" based on a lens use :: MonadState s m
=\textgreater{} Lens' s a -\textgreater{} m a

-\/- \textbar{} "set" through on a lens (.=) :: MonadState s m =\textgreater{}
Lens' s a -\textgreater{} a -\textgreater{} m ()

-\/- \textbar{} "lift" a State action through a lens zoom :: Lens' s t
-\textgreater{} State t a -\textgreater{} State s a ```

So, for example, we have:

```haskell -\/- \textbar{} "get" the registers use psRegs :: (HasProgState s,
MonadState s m) =\textgreater{} m (M.Map Char Int)

-\/- \textbar{} "set" the PointedList (psTape .=) :: (HasProgState s, MonadState
s m) =\textgreater{} P.PointedList Op -\textgreater{} m () ```

The nice thing about lenses is that they compose, so, for example, we have:

```haskell at :: k -\textgreater{} Lens' (Map k v ) (Maybe v)

at 'h' :: Lens' (Map Char Int) (Maybe Int) ```

We can use \texttt{at\ \textquotesingle{}c\textquotesingle{}} to give us a lens
from our registers (\texttt{Map\ Char\ Int}) into the specific register
\texttt{\textquotesingle{}c\textquotesingle{}} as a \texttt{Maybe\ Int} -\/-
it's \texttt{Nothing} if the item is not in the \texttt{Map}, and \texttt{Just}
if it is (with the value).

However, we want to treat all registers as \texttt{0} by default, not as
\texttt{Nothing}, so we can use \texttt{non\ 0}:

\texttt{haskell\ non\ 0\ ::\ Lens\textquotesingle{}\ (Maybe\ Int)\ Int}

\texttt{non\ 0} is a \texttt{Lens} (actually an \texttt{Iso}, but who's
counting?) into a \texttt{Maybe\ Int} to treat \texttt{Nothing} as if it was
\texttt{0}, and to treat \texttt{Just\ x} as if it was \texttt{x}.

We can chain \texttt{at\ r} with \texttt{non\ 0} to get a lens into a
\texttt{Map\ Char\ Int}, which we can use to edit a specific item, treating
non-present-items as 0.

```haskell at 'h' . non 0 :: Lens' (Map Char Int) Int

psRegs . at 'h' . non 0 :: HasProgState s =\textgreater{} Lens' s Int ```

\subsubsection{Interpreting Mem}

With these tools to make life simpler, we can write an interpreter for our
\texttt{Mem} commands:

\texttt{haskell\ -\/-\ source:\ https://github.com/mstksg/inCode/tree/master/code-samples/interpreters/Duet.hs\#L114-L124\ interpMem\ \ \ \ \ ::\ (MonadState\ s\ m,\ MonadFail\ m,\ HasProgState\ s)\ \ \ \ \ =\textgreater{}\ Mem\ a\ \ \ \ \ -\textgreater{}\ m\ a\ interpMem\ =\ \textbackslash{}case\ \ \ \ \ MGet\ c\ \ \ -\textgreater{}\ use\ (psRegs\ .\ at\ c\ .\ non\ 0)\ \ \ \ \ MSet\ c\ x\ -\textgreater{}\ psRegs\ .\ at\ c\ .\ non\ 0\ .=\ x\ \ \ \ \ MJmp\ n\ \ \ -\textgreater{}\ do\ \ \ \ \ \ \ Just\ t\textquotesingle{}\ \textless{}-\ P.moveN\ n\ \textless{}\$\textgreater{}\ use\ psTape\ \ \ \ \ \ \ psTape\ .=\ t\textquotesingle{}\ \ \ \ \ MPk\ \ \ \ \ \ -\textgreater{}\ use\ (psTape\ .\ P.focus)}

We use \texttt{MonadFail} to explicitly state that we rely on a failed pattern
match for control flow.
\texttt{P.moveN\ ::\ Int\ -\textgreater{}\ P.PointedList\ a\ -\textgreater{}\ Maybe\ (P.PointedList\ a)}
will "shift" a \texttt{PointedList} by a given amount, but will return
\texttt{Nothing} if it goes out of bounds. Our program is meant to terminate if
we ever go out of bounds, so we can implement this by using a do block pattern
match with \texttt{MonadFail}. For instances like
\texttt{MaybeT}/\texttt{Maybe}, this means
\texttt{empty}/\texttt{Nothing}/short-circuit. So when we \texttt{P.move}, we
do-block pattern match on \texttt{Just\ t\textquotesingle{}}.

We also use \texttt{P.focus\ ::\ Lens\textquotesingle{}\ (P.PointedList\ a)\ a},
a lens that the \emph{pointedlist} library provides to the current "focus" of
the \texttt{PointedList}.

Note that most of this usage of lens with state is not exactly necessary (we can
manually use \texttt{modify}, \texttt{gets}, etc. instead of lenses and
operators), but it does make things a bit more convenient to write.

\subsubsection{GADT Property}

Again, the GADT-ness of \texttt{Mem} (and \texttt{Com}) works to enforce that
the "results" that each primitive expects is the result that we give.

For example, \texttt{MGet\ \textquotesingle{}c\textquotesingle{}\ ::\ Mem\ Int}
requires us to return \texttt{m\ Int}. This is what \texttt{use} gives us.
\texttt{MSet\ \textquotesingle{}c\textquotesingle{}\ 3\ ::\ Mem\ ()} requires us
to return \texttt{m\ ()}, which is what \texttt{(.=)} returns.

We have \texttt{MPk\ ::\ Mem\ Op}, which requires us to return \texttt{m\ Op}.
That's exactly what
\texttt{use\ (psTape\ .\ P.focus)\ ::\ (MonadState\ s\ m,\ HasProgState\ s)\ =\textgreater{}\ m\ Op}
gives.

The fact that we can use GADTs to specify the "result type" of each of our
primitives is a key part about how \texttt{Prompt} from \emph{MonadPrompt}
works, and how it implements the interpreter pattern.

This is enforced in Haskell's type system (through the "dependent pattern
match"), so GHC will complain to us if we ever return something of the wrong
type while handling a given constructor/primitive.

\subsection{Interpreting Com for Part A}

Now, Part A requires an environment where:

\begin{enumerate}
\tightlist
\item
  \texttt{CSnd} "emits" items into the void, keeping track only of the
  \emph{last} emitted item
\item
  \texttt{CRcv} "catches" the last thing seen by \texttt{CSnd}, keeping track of
  only the \emph{first} caught item
\end{enumerate}

We can keep track of this using \texttt{MonadWriter\ (First\ Int)} to interpret
\texttt{CRcv} (if there are two \emph{rcv}'s, we only care about the first
\emph{rcv}'d thing), and \texttt{MonadAccum\ (Last\ Int)} to interpret
\texttt{CSnd}. A \texttt{MonadAccum} is just like \texttt{MonadWriter} (where
you can "tell" things and accumulate things), but you also have the ability to
read the accumulated log at any time. We use \texttt{Last\ Int} because, if
there are two \emph{snd}'s, we only care about the last \emph{snd}'d thing.

\texttt{haskell\ -\/-\ source:\ https://github.com/mstksg/inCode/tree/master/code-samples/interpreters/Duet.hs\#L130-L140\ interpComA\ \ \ \ \ ::\ (MonadAccum\ (Last\ Int)\ m,\ MonadWriter\ (First\ Int)\ m)\ \ \ \ \ =\textgreater{}\ Com\ a\ \ \ \ \ -\textgreater{}\ m\ a\ interpComA\ =\ \textbackslash{}case\ \ \ \ \ CSnd\ x\ -\textgreater{}\ \ \ \ \ \ \ add\ (Last\ (Just\ x))\ \ \ \ \ CRcv\ x\ -\textgreater{}\ do\ \ \ \ \ \ \ when\ (x\ /=\ 0)\ \$\ \ \ \ \ \ \ \ \ -\/-\ don\textquotesingle{}t\ rcv\ if\ the\ register\ parameter\ is\ 0\ \ \ \ \ \ \ \ \ tell\ .\ First\ .\ getLast\ =\textless{}\textless{}\ look\ \ \ \ \ \ \ return\ x}

Note
\texttt{add\ ::\ MonadAccum\ w\ m\ =\textgreater{}\ w\ -\textgreater{}\ m\ ()}
and \texttt{look\ ::\ MonadAccum\ w\ w}, the functions to "tell" to a
\texttt{MonadAccum} and the function to "get"/"ask" from a \texttt{MonadAccum}.

\subsubsection{MonadAccum}

Small relevant note -\/- \texttt{MonadAccum} does not yet exist in \emph{mtl},
though it probably will in the next version. It's the classy version of
\texttt{AccumT}, which is already in
\emph{\href{https://hackage.haskell.org/package/transformers-0.5.5.0}{transformers-0.5.5.0}}.

For now, I've added \texttt{MonadAccum} and appropriate instances in the
\href{https://github.com/mstksg/inCode/tree/master/code-samples/interpreters/Duet.hs\#L126-L245}{sample
source code}, but when the new version of \emph{mtl} comes out, I'll be sure to
update this post to take this into account!

\subsection{Interpreting Com for Part B}

Part B requires an environment where:

\begin{enumerate}
\tightlist
\item
  \texttt{CSnd} "emits" items into into some accumulating log of items, and we
  need to keep track of all of them.
\item
  \texttt{CRcv} "consumes" items from some external environment, and fails when
  there are no more items to consume.
\end{enumerate}

We can interpret \texttt{CSnd}'s effects using \texttt{MonadWriter\ {[}Int{]}},
to collect all emitted \texttt{Int}s. We can interpret \texttt{CRcv}'s effects
using \texttt{MonadState\ s}, where \texttt{s} contains an \texttt{{[}Int{]}}
acting as a source of \texttt{Int}s to consume.

We're going to use a \texttt{Thread} type to keep track of all thread state. We
do this so we can merge the contexts of \texttt{interpMem} and
\texttt{interpComB}, and really treat them (using type inference) as both
working in the same interpretation context.

```haskell -\/- source:
https://github.com/mstksg/inCode/tree/master/code-samples/interpreters/Duet.hs\#L153-L159
data Thread = T \{ \emph{tState :: ProgState , }tBuffer :: {[}Int{]} \}
makeClassy ''Thread

instance HasProgState Thread where progState = tState ```

(We write an instance for \texttt{HasProgState\ Thread}, so we can use
\texttt{interpMem} in a \texttt{MonadState\ Thread\ m}, since
\texttt{psRegs\ ::\ Lens\textquotesingle{}\ Thread\ (M.Map\ Char\ Int)}, for
example, will refer to the \texttt{psRegs} inside the \texttt{ProgState} in the
\texttt{Thread})

And now, to interpret:

\texttt{haskell\ -\/-\ source:\ https://github.com/mstksg/inCode/tree/master/code-samples/interpreters/Duet.hs\#L161-L170\ interpComB\ \ \ \ \ ::\ (MonadWriter\ {[}Int{]}\ m,\ MonadFail\ m,\ MonadState\ Thread\ m)\ \ \ \ \ =\textgreater{}\ Com\ a\ \ \ \ \ -\textgreater{}\ m\ a\ interpComB\ =\ \textbackslash{}case\ \ \ \ \ CSnd\ x\ -\textgreater{}\ tell\ {[}x{]}\ \ \ \ \ CRcv\ \_\ -\textgreater{}\ do\ \ \ \ \ \ \ x:xs\ \textless{}-\ use\ tBuffer\ \ \ \ \ \ \ tBuffer\ .=\ xs\ \ \ \ \ \ \ return\ x}

Note again the usage of do block pattern matches and \texttt{MonadFail}.

\subsection{Combining Interpreters}

To combine interpreters, we're going to be using, from \emph{type-combinators}:

\texttt{haskell\ (\textgreater{}\textbar{}\textless{})\ ::\ (f\ a\ -\textgreater{}\ r)\ \ \ \ \ \ \ -\textgreater{}\ (g\ a\ -\textgreater{}\ r)\ \ \ \ \ \ \ -\textgreater{}\ ((f\ :\textbar{}:\ g)\ a\ -\textgreater{}\ r)}

Basically, \texttt{\textgreater{}\textbar{}\textless{}} lets us write a
"handler" for a \texttt{:\textbar{}:} by providing a handler for each side. For
example, with more concrete types:

\texttt{haskell\ (\textgreater{}\textbar{}\textless{})\ ::\ (Mem\ a\ -\textgreater{}\ r)\ \ \ \ \ \ \ -\textgreater{}\ (Com\ a\ -\textgreater{}\ r)\ \ \ \ \ \ \ -\textgreater{}\ ((Mem\ :\textbar{}:\ Com)\ a\ -\textgreater{}\ r)}

We can use this to build an interpreter for \texttt{Duet}:

\texttt{haskell\ runPromptM\ \ \ \ \ ::\ Monad\ m\ \ \ \ \ =\textgreater{}\ (forall\ x.\ (Mem\ x\ :\textbar{}:\ Com\ x)\ -\textgreater{}\ m\ x)\ \ \ \ \ -\textgreater{}\ Duet\ a\ \ \ \ \ -\textgreater{}\ m\ a}

By using \texttt{\textgreater{}\textbar{}\textless{}} to generate our compound
interpreters. This is how we can create interpreters on \texttt{Duet} by
"combining", in a modular way, interpreters for \texttt{Mem} and \texttt{Com}.
This is the essence of the "data types a la carte" technique.

\section{Getting the Results}

We now just have to pick concrete monads now for us to interpret into.

\subsection{Part A}

Our interpreter for Part A is
\texttt{interpMem\ \textgreater{}\textbar{}\textless{}\ interpComA} -\/- we
interpret the \texttt{Mem} primitives the usual way, and interpret the
\texttt{Com} primitives the Part A way.

Let's check what capabilities our interpreter must have:

\texttt{haskell\ ghci\textgreater{}\ :t\ interpMem\ \textgreater{}\textbar{}\textless{}\ interpComA\ interpMem\ \textgreater{}\textbar{}\textless{}\ interpComA\ \ \ \ \ ::\ (\ MonadWriter\ (First\ Int)\ m\ \ \ \ \ \ \ \ ,\ MonadAccum\ (Last\ Int)\ m\ \ \ \ \ \ \ \ ,\ MonadFail\ m\ \ \ \ \ \ \ \ ,\ MonadState\ s\ m\ \ \ \ \ \ \ \ ,\ HasProgState\ s\ \ \ \ \ \ \ \ )\ \ \ \ \ =\textgreater{}\ (Mem\ :\textbar{}:\ Com)\ a\ \ \ \ \ -\textgreater{}\ m\ a}

So it looks like we need to be \texttt{MonadWriter\ (First\ Int)},
\texttt{MonadAccum\ (Last\ Int)}, \texttt{MonadFail\ m}, and
\texttt{MonadState\ s\ m}, where \texttt{HasProgState\ s}.

Now, we can write such a Monad from scratch, or we can use the
\emph{transformers} library to generate a transformer with all of those
instances for us. For the sake of brevity and reducing duplicated code, let's go
that route. We can use:

\texttt{haskell\ MaybeT\ (StateT\ ProgState\ (WriterT\ (First\ Int)\ (A.Accum\ (Last\ Int))))}

And so we can write our final "step" function in that context:

\texttt{haskell\ -\/-\ source:\ https://github.com/mstksg/inCode/tree/master/code-samples/interpreters/Duet.hs\#L142-L143\ stepA\ ::\ MaybeT\ (StateT\ ProgState\ (WriterT\ (First\ Int)\ (A.Accum\ (Last\ Int))))\ ()\ stepA\ =\ runPromptM\ (interpMem\ \textgreater{}\textbar{}\textless{}\ interpComA)\ stepProg}

\texttt{stepA} will make a single step of the tape, according to the
interpreters \texttt{interpMem} and \texttt{interpComA}.

Our final answer is then just the result of \emph{repeating} this over and over
again until there's a failure. We take advantage of the fact that
\texttt{MaybeT}'s \texttt{Alternative} instance uses \texttt{empty} for
\texttt{fail}, so we can use
\texttt{many\ ::\ MaybeT\ m\ a\ -\textgreater{}\ MaybeT\ m\ {[}a{]}}, which
repeats a \texttt{MaybeT} action several times until a failure is encountered.
In our case, "failure" is when the tape goes out of bounds.

But, because of laziness, our computation terminates as soon as a valid
\texttt{CRcv} is found and a \texttt{First\ Int} is logged to the
\texttt{Writer}, so we don't actually need to run the computation until it goes
out of bounds.

Here is the entirety of running Part A -\/- as you can see, it consists mostly
of unwrapping \emph{transformers} newtype wrappers.

\texttt{haskell\ -\/-\ source:\ https://github.com/mstksg/inCode/tree/master/code-samples/interpreters/Duet.hs\#L145-L151\ partA\ ::\ P.PointedList\ Op\ -\textgreater{}\ Maybe\ Int\ partA\ ops\ =\ getFirst\ \ \ \ \ \ \ \ \ \ \ .\ flip\ A.evalAccum\ mempty\ \ \ \ \ \ \ \ \ \ \ .\ execWriterT\ \ \ \ \ \ \ \ \ \ \ .\ flip\ runStateT\ (PS\ ops\ M.empty)\ \ \ \ \ \ \ \ \ \ \ .\ runMaybeT\ \ \ \ \ \ \ \ \ \ \ \$\ many\ stepA}

A \texttt{Nothing} result means that the \texttt{Writer} log never received any
outputs before \texttt{many} ends looping, which means that the tape goes out of
bounds before a successful \emph{rcv}.

\subsection{Part B}

Our interpreter for Part B is a little simpler:

\texttt{haskell\ ghci\textgreater{}\ :t\ interpMem\ \textgreater{}\textbar{}\textless{}\ interpComB\ interpMem\ \textgreater{}\textbar{}\textless{}\ interpComB\ \ \ \ \ ::\ (\ MonadWriter\ {[}Int{]}\ m\ \ \ \ \ \ \ \ ,\ MonadFail\ m\ \ \ \ \ \ \ \ ,\ MonadState\ Thread\ m\ \ \ \ \ \ \ \ )\ \ \ \ \ =\textgreater{}\ (Mem\ :\textbar{}:\ Com)\ a\ \ \ \ \ -\textgreater{}\ m\ a}

We can really just use:

\texttt{haskell\ WriterT\ {[}Int{]}\ (MaybeT\ (State\ Thread))}

Writing our concrete \texttt{stepB} is a little more involved, since we have to
juggle the state of each thread separately. We can do this using:

\texttt{haskell\ zoom\ \_1\ ::\ MaybeT\ (State\ s)\ a\ -\textgreater{}\ MaybeT\ (State\ (s,\ t))\ a\ zoom\ \_2\ ::\ MaybeT\ (State\ t)\ a\ -\textgreater{}\ MaybeT\ (State\ (s,\ t))\ a}

To "lift" our actions on one thread to be actions on a "tuple" of threads. We
have, in the end:

\texttt{haskell\ -\/-\ source:\ https://github.com/mstksg/inCode/tree/master/code-samples/interpreters/Duet.hs\#L172-L181\ stepB\ ::\ MaybeT\ (State\ (Thread,\ Thread))\ Int\ stepB\ =\ do\ \ \ \ \ outA\ \textless{}-\ execWriterT\ \$\ \ \ \ \ \ \ zoom\ \_1\ .\ many\ \$\ runPromptM\ (interpMem\ \textgreater{}\textbar{}\textless{}\ interpComB)\ stepProg\ \ \ \ \ outB\ \textless{}-\ execWriterT\ \$\ \ \ \ \ \ \ zoom\ \_2\ .\ many\ \$\ runPromptM\ (interpMem\ \textgreater{}\textbar{}\textless{}\ interpComB)\ stepProg\ \ \ \ \ \_1\ .\ tBuffer\ .=\ outB\ \ \ \ \ \_2\ .\ tBuffer\ .=\ outA\ \ \ \ \ guard\ .\ not\ \$\ null\ outA\ \&\&\ null\ outB\ \ \ \ \ return\ \$\ length\ outB}

Our final \texttt{stepB} really doesn't need a \texttt{WriterT\ {[}Int{]}} -\/-
we just use that internally to collect \emph{snd} outputs. So we use
\texttt{execWriter} after "interpreting" our actions (along with \texttt{many},
to repeat our thread steps until they block) to just get the resulting logs.

We then reset the input buffers appropriately (by putting in the collected
outputs of the previous threads).

If both threads are blocking (they both have to external outputs to pass on),
then we're done (using \texttt{guard}).

We return the number of items that "Program 1" (the second thread) outputs,
because that's what we need for our answer.

This is one "single pass" of both of our threads. As you can anticipate, we'll
use \texttt{many} again to run these multiple times until both threads block.

\texttt{haskell\ -\/-\ source:\ https://github.com/mstksg/inCode/tree/master/code-samples/interpreters/Duet.hs\#L183-L191\ partB\ ::\ P.PointedList\ Op\ -\textgreater{}\ Int\ partB\ ops\ =\ sum\ .\ concat\ \ \ \ \ \ \ \ \ \ \ .\ flip\ evalState\ s0\ \ \ \ \ \ \ \ \ \ \ .\ runMaybeT\ \ \ \ \ \ \ \ \ \ \ \$\ many\ stepB\ \ \ where\ \ \ \ \ s0\ =\ (\ T\ (PS\ ops\ (M.singleton\ \textquotesingle{}p\textquotesingle{}\ 0))\ {[}{]}\ \ \ \ \ \ \ \ \ \ ,\ T\ (PS\ ops\ (M.singleton\ \textquotesingle{}p\textquotesingle{}\ 1))\ {[}{]}\ \ \ \ \ \ \ \ \ \ )}

\subsection{Examples}

In the
\href{https://github.com/mstksg/inCode/tree/master/code-samples/interpreters/Duet.hs}{sample
source code}, I've included
\href{https://github.com/mstksg/inCode/tree/master/code-samples/interpreters/Duet.hs\#L198-L241}{my
own puzzle input} provided to me from the advent of code website. We can now get
actual answers given some sample puzzle input:

\texttt{haskell\ -\/-\ source:\ https://github.com/mstksg/inCode/tree/master/code-samples/interpreters/Duet.hs\#L193-L196\ main\ ::\ IO\ ()\ main\ =\ do\ \ \ \ \ print\ \$\ partA\ (parseProgram\ testProg)\ \ \ \ \ print\ \$\ partB\ (parseProgram\ testProg)}

And, as a stack script, we can run this and see my own puzzle's answers:

\texttt{bash\ \$\ ./Duet.hs\ Just\ 7071\ 8001}

\end{document}
