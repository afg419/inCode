\documentclass[]{article}
\usepackage{lmodern}
\usepackage{amssymb,amsmath}
\usepackage{ifxetex,ifluatex}
\usepackage{fixltx2e} % provides \textsubscript
\ifnum 0\ifxetex 1\fi\ifluatex 1\fi=0 % if pdftex
  \usepackage[T1]{fontenc}
  \usepackage[utf8]{inputenc}
\else % if luatex or xelatex
  \ifxetex
    \usepackage{mathspec}
    \usepackage{xltxtra,xunicode}
  \else
    \usepackage{fontspec}
  \fi
  \defaultfontfeatures{Mapping=tex-text,Scale=MatchLowercase}
  \newcommand{\euro}{€}
\fi
% use upquote if available, for straight quotes in verbatim environments
\IfFileExists{upquote.sty}{\usepackage{upquote}}{}
% use microtype if available
\IfFileExists{microtype.sty}{\usepackage{microtype}}{}
\usepackage[margin=1in]{geometry}
\usepackage{color}
\usepackage{fancyvrb}
\newcommand{\VerbBar}{|}
\newcommand{\VERB}{\Verb[commandchars=\\\{\}]}
\DefineVerbatimEnvironment{Highlighting}{Verbatim}{commandchars=\\\{\}}
% Add ',fontsize=\small' for more characters per line
\newenvironment{Shaded}{}{}
\newcommand{\AlertTok}[1]{\textcolor[rgb]{1.00,0.00,0.00}{\textbf{#1}}}
\newcommand{\AnnotationTok}[1]{\textcolor[rgb]{0.38,0.63,0.69}{\textbf{\textit{#1}}}}
\newcommand{\AttributeTok}[1]{\textcolor[rgb]{0.49,0.56,0.16}{#1}}
\newcommand{\BaseNTok}[1]{\textcolor[rgb]{0.25,0.63,0.44}{#1}}
\newcommand{\BuiltInTok}[1]{#1}
\newcommand{\CharTok}[1]{\textcolor[rgb]{0.25,0.44,0.63}{#1}}
\newcommand{\CommentTok}[1]{\textcolor[rgb]{0.38,0.63,0.69}{\textit{#1}}}
\newcommand{\CommentVarTok}[1]{\textcolor[rgb]{0.38,0.63,0.69}{\textbf{\textit{#1}}}}
\newcommand{\ConstantTok}[1]{\textcolor[rgb]{0.53,0.00,0.00}{#1}}
\newcommand{\ControlFlowTok}[1]{\textcolor[rgb]{0.00,0.44,0.13}{\textbf{#1}}}
\newcommand{\DataTypeTok}[1]{\textcolor[rgb]{0.56,0.13,0.00}{#1}}
\newcommand{\DecValTok}[1]{\textcolor[rgb]{0.25,0.63,0.44}{#1}}
\newcommand{\DocumentationTok}[1]{\textcolor[rgb]{0.73,0.13,0.13}{\textit{#1}}}
\newcommand{\ErrorTok}[1]{\textcolor[rgb]{1.00,0.00,0.00}{\textbf{#1}}}
\newcommand{\ExtensionTok}[1]{#1}
\newcommand{\FloatTok}[1]{\textcolor[rgb]{0.25,0.63,0.44}{#1}}
\newcommand{\FunctionTok}[1]{\textcolor[rgb]{0.02,0.16,0.49}{#1}}
\newcommand{\ImportTok}[1]{#1}
\newcommand{\InformationTok}[1]{\textcolor[rgb]{0.38,0.63,0.69}{\textbf{\textit{#1}}}}
\newcommand{\KeywordTok}[1]{\textcolor[rgb]{0.00,0.44,0.13}{\textbf{#1}}}
\newcommand{\NormalTok}[1]{#1}
\newcommand{\OperatorTok}[1]{\textcolor[rgb]{0.40,0.40,0.40}{#1}}
\newcommand{\OtherTok}[1]{\textcolor[rgb]{0.00,0.44,0.13}{#1}}
\newcommand{\PreprocessorTok}[1]{\textcolor[rgb]{0.74,0.48,0.00}{#1}}
\newcommand{\RegionMarkerTok}[1]{#1}
\newcommand{\SpecialCharTok}[1]{\textcolor[rgb]{0.25,0.44,0.63}{#1}}
\newcommand{\SpecialStringTok}[1]{\textcolor[rgb]{0.73,0.40,0.53}{#1}}
\newcommand{\StringTok}[1]{\textcolor[rgb]{0.25,0.44,0.63}{#1}}
\newcommand{\VariableTok}[1]{\textcolor[rgb]{0.10,0.09,0.49}{#1}}
\newcommand{\VerbatimStringTok}[1]{\textcolor[rgb]{0.25,0.44,0.63}{#1}}
\newcommand{\WarningTok}[1]{\textcolor[rgb]{0.38,0.63,0.69}{\textbf{\textit{#1}}}}
\ifxetex
  \usepackage[setpagesize=false, % page size defined by xetex
              unicode=false, % unicode breaks when used with xetex
              xetex]{hyperref}
\else
  \usepackage[unicode=true]{hyperref}
\fi
\hypersetup{breaklinks=true,
            bookmarks=true,
            pdfauthor={Justin Le},
            pdftitle={Deploying Medium to Large Haskell Apps to Heroku by Precompiling},
            colorlinks=true,
            citecolor=blue,
            urlcolor=blue,
            linkcolor=magenta,
            pdfborder={0 0 0}}
\urlstyle{same}  % don't use monospace font for urls
% Make links footnotes instead of hotlinks:
\renewcommand{\href}[2]{#2\footnote{\url{#1}}}
\setlength{\parindent}{0pt}
\setlength{\parskip}{6pt plus 2pt minus 1pt}
\setlength{\emergencystretch}{3em}  % prevent overfull lines
\setcounter{secnumdepth}{0}

\title{Deploying Medium to Large Haskell Apps to Heroku by Precompiling}
\author{Justin Le}
\date{October 7, 2013}

\begin{document}
\maketitle

\emph{Originally posted on
\textbf{\href{https://blog.jle.im/entry/deploying-medium-to-large-haskell-apps-to-heroku.html}{in
Code}}.}

\textbf{UPDATE}: This post was written in 2013, where the options available to
someone looking to host a Haskell site on Heroku were fairly limited. It's (as
of the time of writing this) 2015 now and things have changed. Check out
\href{http://blog.jle.im/entry/deploying-medium-to-large-haskell-apps-to-heroku\#disqus_thread}{the
comments} for two good alternatives to this that are working today!

Consider the rest of this article obsolete, or look here if none of the
solutions given in the comments work :)

\hypertarget{old-article-written-2013}{%
\section{Old Article (Written 2013)}\label{old-article-written-2013}}

If you do a search on how to deploy Haskell apps to Heroku these days, chances
are you are going to find the very elegant method (here's
\href{http://adit.io/posts/2013-04-15-making-a-website-with-haskell.html\#deploying-to-heroku}{one
solution}, and
\href{http://blog.begriffs.com/2013/08/deploying-yesod-to-heroku-with-postgres.html}{another})
involving leveraging Heroku's powerful
\href{https://devcenter.heroku.com/articles/cedar}{Cedar stack} and having
Heroku use \texttt{cabal\ install} to download and compile your app and all of
its dependencies into a native binary on the server itself. It's a rather
beautiful solution to the problem of a truly polyglot automated production
server.

The \href{https://github.com/mstksg/blog}{blog engine} that runs this blog is
written in Haskell. When I tried to deploy it using those steps, I encountered a
rather frustrating roadblock:

Heroku enforces a hard time-out limit of \textbf{fifteen minutes} for all of its
apps to compile and deploy. And because \emph{cabal} needs time to download and
compile every dependency, a typical non-trivial app (like a blog) would reach
this limit very quickly with only a
\href{https://github.com/mstksg/blog/blob/master/blog.cabal\#L20-52}{modest
amount of dependencies}.

I did some searching on this and asked around on the irc channel, but I was not
able to find any real-world examples of \emph{non-trivial} apps being deployed
to Haskell using this method. It seemed like most articles simply deployed a toy
project, and left it at that. (On that note, if anyone has actually had success
with this, or knows someone who has, please let me know)

Until Heroku's time-out limit can be adjusted or bypassed, the only real
solution (besides incrementally pushing dependencies with a buildpack that
caches --- a solution even uglier and less practical) is to pre-compile your
binary to an architecture that Heroku supports.

There are some tutorials on this already, but few are up to date and
flexible/comprehensive, so here is my shot.

As an \textbf{important note}: this method is a lot less elegant and maintanable
(in principle) than the recommended buildpack method; only use it if you are
absolutely certain that buildpacks won't work for you.

Also note that this assumes that your app is already configured to work on
Heroku --- that it doesn't modify the filesystem after the deploy (so no
\emph{sqlite}), it uses the \texttt{\$DATABASE\_URL} environment variable to
establish database connections, it uses the \texttt{\$PATH} environment variable
to choose the port to listen to, and other small things you just have to worry
about that is out of the scope of this post.

Much of this post is owed to
\href{https://github.com/yesodweb/yesod/wiki/Deploying-Yesod-Apps-to-Heroku}{this
article on the Yesod wiki}.

\hypertarget{compiling-your-binary}{%
\section{Compiling Your Binary}\label{compiling-your-binary}}

\hypertarget{the-virtual-machine}{%
\subsection{The Virtual Machine}\label{the-virtual-machine}}

First of all, we need to find ourselves a machine with the same architecture as
the Heroku virtual machines. For most people, it is not practical to go out and
buy a physical machine that you can use for yourself, so we're going to be
setting up a virtual one here.

A lot of this is going to be verbatim from
\href{https://github.com/yesodweb/yesod/wiki/Setting-up-a-virtual-machine\%2C-using-VirtualBox-and-Vagrant}{this
reference}, with a few updates.

\begin{enumerate}
\def\labelenumi{\arabic{enumi}.}
\item
  You're going to need
  \href{https://www.virtualbox.org/wiki/Downloads}{VirtualBox}, from Oracle.
  Most modern work on virtual machines leverage this great tool.
\item
  Install \href{http://downloads.vagrantup.com/}{vagrant}, a really convenient
  wrapper around the raw VirtualBox functionality that integrates things like
  build scripts and configurations into an easy-to-use package.
\item
  Clone/download the
  \href{https://bitbucket.org/puffnfresh/vagrant-haskell-heroku}{vagrant-haskell-heroku}
  project from BitBucket. This is the vagrant project that will set up
  everything you need to build and deploy to Heroku --- an installation of GHC,
  the Haskell Platform, and the Heroku Toolbelt, all on a 64 bit Ubuntu 10.04
  image.
\item
  Edit the \texttt{Vagrantfile}; on the line starting with
  \texttt{chef.json.merge!}, change the values to the versions of
  \href{http://www.haskell.org/ghc/}{GHC} and
  \href{http://www.haskell.org/platform/}{Haskell Platform} you will be using.

\begin{Shaded}
\begin{Highlighting}[]
\NormalTok{chef.json.merge!(\{ }\StringTok{:ghc_version}\NormalTok{ => }\StringTok{'7.4.4'}\NormalTok{,}
                   \StringTok{:haskell_platform_version}\NormalTok{ => }\StringTok{'2012.4.0.0'}\NormalTok{\})}
\end{Highlighting}
\end{Shaded}

  I'm using \texttt{ghc-7.6.3} to develop, and as of October 2013, the lastest
  stable Haskell Platform is \texttt{2013.2.0.0}.
\item
  Launch your virtual machine with

\begin{Shaded}
\begin{Highlighting}[]
\NormalTok{$ }\ExtensionTok{vagrant}\NormalTok{ up}
\end{Highlighting}
\end{Shaded}

  This will launch the VM and install the given versions of GHC and the Haskell
  Platform. You are mostly good to go now -- log onto your machine using

\begin{Shaded}
\begin{Highlighting}[]
\NormalTok{$ }\ExtensionTok{vagrant}\NormalTok{ ssh}
\end{Highlighting}
\end{Shaded}

  If this doesn't work, try installing \texttt{libgc-dev}. \texttt{vagrant\ ssh}
  should send you into an ``ssh'' session on your VM. Once there, let's run some
  basic bookkeeping/updating that isn't handled by the vagrant project:

\begin{Shaded}
\begin{Highlighting}[]
\NormalTok{$ }\FunctionTok{sudo}\NormalTok{ apt-get update}
\NormalTok{$ }\FunctionTok{sudo}\NormalTok{ apt-get upgrade}
\NormalTok{$ }\FunctionTok{sudo}\NormalTok{ apt-get install git-core}
\NormalTok{$ }\ExtensionTok{cabal}\NormalTok{ update}
\end{Highlighting}
\end{Shaded}
\end{enumerate}

And you should have a fresh virtual machine compatible with Heroku ready to
build your project on.

\hypertarget{building-your-project}{%
\subsection{Building Your Project}\label{building-your-project}}

At this point, there are many ways to proceed. Building is now more or less the
same as if you were building on your own production server. Here is one way to
go about it.

\begin{enumerate}
\def\labelenumi{\arabic{enumi}.}
\item
  Get your project files onto your guest (virtual) machine.

  \begin{itemize}
  \item
    If your projects are on a version control repository like
    \href{https://www.github.com}{Github} or any accessible server, then getting
    your files on your guest machine and keeping them up-to-date is as easy as a
    \texttt{git\ pull} (substitute git for whatever version control you use).
  \item
    If not, the directory that your vagrant files are stored in is actually
    mounted onto the file system of the guest machine. You can access it at
    \texttt{/vagrant} on the guest machine.

    This is really handy for transferring things like ssh keys, but you can also
    use this to get your project files onto your guest machine. Simply copy them
    over somehow onto a folder in the vagrant directory, and you can then access
    them on your virtual machine and do what you want with them.

    However, if you want to keep your files up to date, you'll have to do this
    manually.
  \item
    If you are using version control like \emph{git}, but you aren't hosting it
    on a server (and why not? you can even
    \href{https://github.com/sitaramc/gitolite}{host a repo server locally on
    your own machine}.), see if you can use a local folder as a repository
    source.

    For \emph{git}, this is pretty simple. You only need to run
    \texttt{git\ init\ -\/-bare} on the folder you want to use as the
    repository, and add the local filesystem path as the remote url. There are
    many resources explaining this process in detail, like
    \href{http://www.jedi.be/blog/2009/05/06/8-ways-to-share-your-git-repository/}{this
    post} and
    \href{http://treeleaf.be/blog/2011/03/creating-a-new-git-repository-on-a-local-file-system/}{this
    one}
  \end{itemize}
\item
  Build the executable. This is the same as on any machine. However, I strongly
  recommend using some kind of sandboxing system like
  \href{http://hackage.haskell.org/package/cabal-dev}{cabal-dev}, or cabal
  1.18's built-in sandboxing, just to make sure you don't run into any problems
  in the future.

\begin{Shaded}
\begin{Highlighting}[]
\CommentTok{#   using cabal-dev}
\NormalTok{$ }\ExtensionTok{cabal}\NormalTok{ install cabal-dev}
\CommentTok{#   you can add ~/.cabal/bin to your $PATH if you want}
\NormalTok{$ }\ExtensionTok{~/.cabal/bin/cabal-dev}\NormalTok{ install}
\end{Highlighting}
\end{Shaded}

  If any of your cabal packages require developer libraries to be installed on
  your machine (anything involving Postgres comes to mind), you'll need to be
  sure that they are installed. A simple \texttt{apt-get} should take care of
  this for all relevant packages.

  After everything downloads, builds, installs, etc., your executable will be
  created as \texttt{dist/build/app-name/app-name}.

  This will also be a good test as to whether or not you specified your
  dependencies in your \texttt{.cabal} file properly.
\end{enumerate}

\hypertarget{deploying-it-all}{%
\section{Deploying it all}\label{deploying-it-all}}

Almost there! Your binary is now compiled; how are you going to deploy it to
Heroku?

\begin{enumerate}
\def\labelenumi{\arabic{enumi}.}
\item
  First, you have to commit your binary to version control. Some people
  recommend using a separate branch for this, but because your guest machine's
  project directory is kind of a transient thing, this really isn't that
  necessary.

  You can simply forcefully add the file to git as it is, because chances are
  you have it already in your \texttt{.gitignore}:

\begin{Shaded}
\begin{Highlighting}[]
\NormalTok{$ }\FunctionTok{git}\NormalTok{ add -f dist/build/app-name/app-name}
\end{Highlighting}
\end{Shaded}

  Alternatively, you can create a \texttt{bin/} folder and copy the executable
  there. It really doesn't make a difference, except that you don't have to
  modify your \texttt{.gitignore}.
\item
  Now, you need to create your \texttt{Procfile} --- this specifies the
  processes that Heroku will be executing.

\begin{Shaded}
\begin{Highlighting}[]
\CommentTok{# Procfile}
\FunctionTok{web:}\AttributeTok{ }\CommentTok{# system command to launch your server}
\end{Highlighting}
\end{Shaded}

  For some web servers, it is simply the path to the executable; for some
  frameworks like \emph{Yesod}, you need to specify the flag
  \texttt{-p\ \$PORT}, because Heroku specifies the port you are to listen to
  via the environment variable.
\item
  Heroku requires every project to have \emph{some} buildpack. Because the
  actual web processes are specified in your \texttt{Procfile}, buildpacks won't
  interfere with any actual execution of your server.

  There are three easy ways to do this --- you can either use a blank
  \texttt{requirements.txt} (the easiest way) to act like a \emph{Python} app, a
  valid but empty \texttt{package.json} to act like a \emph{Node.js} app, or a
  valid but empty \texttt{Gemfile} and \texttt{Gemfile.lock} combination to act
  like a \emph{ruby} app.

  But hey, if you use any node packages or gems or python packages in your
  project, then you can actually use this to your advantage! I personally use
  \emph{\href{http://compass-style.org/}{compass}} a lot for their extensions to
  \emph{sass}, so adding it is as simple as using a \emph{Gemfile} --- just like
  in any normal ruby app.

  If you want to mix and match libraries from different languages/ecosystems,
  you can use the
  \href{https://github.com/ddollar/heroku-buildpack-multi}{multi} buildpack and
  have Heroku check for packages in all of the normal package managers of the
  languages you specify.
\item
  Configure your \href{https://toolbelt.heroku.com/}{Heroku Toolbelt}, and
  deploy.

\begin{Shaded}
\begin{Highlighting}[]
\CommentTok{#   create your app}
\NormalTok{$ }\ExtensionTok{heroku}\NormalTok{ create appname}
\CommentTok{#   and, after making sure everything is set up, committed, and in order...}
\NormalTok{$ }\FunctionTok{git}\NormalTok{ push heroku master}
\end{Highlighting}
\end{Shaded}

  If you have done everything right, this should be succesful. Hooray!
\item
  Make sure your \texttt{web} process is running properly. You can do this by
  going to \url{https://dashboard.heroku.com/apps}, clicking on your app, and
  making sure under \textbf{Dynos} that the check box next to ``web'' is checked
  off.
\end{enumerate}

And that should be it!

\hypertarget{after-the-deploy}{%
\section{After the Deploy}\label{after-the-deploy}}

Your app should be running successfully now! Probably. Maybe. If you run into
any problems, let me know in the comments. But to preempt any issues that might
arise, here are some things that it might be important to pay attention to.

\begin{itemize}
\item
  If you are using Heroku's Postgres instances (and you should, they are amazing
  and probably more reliable than anything you could host yourself on cheap, for
  free), you will have to make sure to
  \href{https://devcenter.heroku.com/articles/heroku-postgresql}{configure them
  properly}.

  If you have problems making a connection, you can try the
  \href{http://hackage.haskell.org/package/heroku}{heroku} package on Hackage
  and integrate it with your database connection backends.
\item
  You are probably going to want to automate your entire re-deploy process ---
  the pull, the build/install, the copying of the executable, the committing of
  the binary to version control, and the deploy to Heroku.

  You can use your favorite task management system, like \emph{Make},
  \emph{Rake}, or even \emph{Shake} (see my
  \href{/entry/shake-task-automation-and-shell-scripting-in-haskell}{brief
  tutorial on Shake})

  Your basic workflow should consist of pushing your project files to your
  repository on your host machine, and \texttt{vagrant\ ssh}-ing onto your guest
  machine and executing one or two commands to automate the entire re-deploy
  process.
\item
  Be aware of good virtual machine management practices. Suspend your machine
  whenever you are not using it:

\begin{Shaded}
\begin{Highlighting}[]
\CommentTok{#   suspend}
\NormalTok{$ }\ExtensionTok{vagrant}\NormalTok{ suspend}
\CommentTok{#   resume}
\NormalTok{$ }\ExtensionTok{vagrant}\NormalTok{ resume}
\end{Highlighting}
\end{Shaded}

  and you will also prevent things from getting hairy in case of a system crash
  on the host side.
\end{itemize}

Good luck developing for the web on the Haskell platform, and welcome to the
club!

Hi, thanks for reading! You can reach me via email at
\href{mailto:justin@jle.im}{\nolinkurl{justin@jle.im}}, or at twitter at
\href{https://twitter.com/mstk}{@mstk}! This post and all others are published
under the \href{https://creativecommons.org/licenses/by-nc-nd/3.0/}{CC-BY-NC-ND
3.0} license. Corrections and edits via pull request are welcome and encouraged
at \href{https://github.com/mstksg/inCode}{the source repository}.

If you feel inclined, or this post was particularly helpful for you, why not
consider \href{https://www.patreon.com/justinle/overview}{supporting me on
Patreon}, or a \href{bitcoin:3D7rmAYgbDnp4gp4rf22THsGt74fNucPDU}{BTC donation}?
:)

\end{document}
