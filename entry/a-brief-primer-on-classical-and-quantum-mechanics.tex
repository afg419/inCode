\documentclass[]{article}
\usepackage{lmodern}
\usepackage{amssymb,amsmath}
\usepackage{ifxetex,ifluatex}
\usepackage{fixltx2e} % provides \textsubscript
\ifnum 0\ifxetex 1\fi\ifluatex 1\fi=0 % if pdftex
  \usepackage[T1]{fontenc}
  \usepackage[utf8]{inputenc}
\else % if luatex or xelatex
  \ifxetex
    \usepackage{mathspec}
    \usepackage{xltxtra,xunicode}
  \else
    \usepackage{fontspec}
  \fi
  \defaultfontfeatures{Mapping=tex-text,Scale=MatchLowercase}
  \newcommand{\euro}{€}
\fi
% use upquote if available, for straight quotes in verbatim environments
\IfFileExists{upquote.sty}{\usepackage{upquote}}{}
% use microtype if available
\IfFileExists{microtype.sty}{\usepackage{microtype}}{}
\usepackage[margin=1in]{geometry}
\usepackage{graphicx}
\makeatletter
\def\maxwidth{\ifdim\Gin@nat@width>\linewidth\linewidth\else\Gin@nat@width\fi}
\def\maxheight{\ifdim\Gin@nat@height>\textheight\textheight\else\Gin@nat@height\fi}
\makeatother
% Scale images if necessary, so that they will not overflow the page
% margins by default, and it is still possible to overwrite the defaults
% using explicit options in \includegraphics[width, height, ...]{}
\setkeys{Gin}{width=\maxwidth,height=\maxheight,keepaspectratio}
\ifxetex
  \usepackage[setpagesize=false, % page size defined by xetex
              unicode=false, % unicode breaks when used with xetex
              xetex]{hyperref}
\else
  \usepackage[unicode=true]{hyperref}
\fi
\hypersetup{breaklinks=true,
            bookmarks=true,
            pdfauthor={},
            pdftitle={},
            colorlinks=true,
            citecolor=blue,
            urlcolor=blue,
            linkcolor=magenta,
            pdfborder={0 0 0}}
\urlstyle{same}  % don't use monospace font for urls
% Make links footnotes instead of hotlinks:
\renewcommand{\href}[2]{#2\footnote{\url{#1}}}
\setlength{\parindent}{0pt}
\setlength{\parskip}{6pt plus 2pt minus 1pt}
\setlength{\emergencystretch}{3em}  % prevent overfull lines
\setcounter{secnumdepth}{0}


\begin{document}

\% A Brief Primer on Classical and Quantum Mechanics for Numerical Techniques \%
Justin Le \% November 29, 2013

\emph{Originally posted on
\textbf{\href{https://blog.jle.im/entry/a-brief-primer-on-classical-and-quantum-mechanics.html}{in
Code}}.}

Okay! In this series we will be going over many subjects in both physics and
computational techniques, including the Lagrangian formulation of classical
mechanics, basic principles of quantum mechanics, the Path Integral formatulion
of quantum mechanics, the Metropolis-Hastings Monte Carlo method, dealing with
entropy and randomness in a pure language, and general principles in numerical
computation! Fun stuff, right?

The end product will be a tool for deriving the ground state probability
distribution of arbitrary quantum systems, which is somewhat of a big deal in
any field that runs into quantum effects (which is basically every modern
field). But the real goal will be to hopefully impart some insight that can be
applied to broarder and more abstract applications. I am confident that these
techniques can be applied to many problems in computation to great results.

I'm going to assume little to no knowledge in Physics and a somewhat
intermediate working knowledge of programming. We're going to be working in both
my favorite imperative language and my favorite functional language.

In this first post I'm just going to go over the basics of the physics before we
dive into the simulation. Here we go!

\section{Classical Mechanics}

\subsection{Newtonian Mechanics}

Mechanics has always been a field in physics that has held a special place in my
heart. It is most likely the field most people are first exposed to in a physics
course. To me, there really is no more fundamental and pure form of physics. I
mean...it's the study of how things move under forces. How can you get any
deeper to the heart of physics than that?

When most people think of mechanics, they think of \$F = m a\$, inertia, and
that every reaction has an equal and opposite re-action. These are Newton's
"Laws of Motion" and they provide what can be referred to as a "state-updating
function": Given a state of the world at time \$t\emph{0\$, Newton's laws can be
used to "generate" the state of the world at time \$t}0 + \textbackslash{}Delta
t\$.

This sounds pretty useful, but it wasn't long before physicists began wishing
they had other tools with which to study the mechanics of certain systems.
Newton's equations worked very well for the cases that made it famous, but were
surprisingly unuseful, impractical, or clumsy in many others. And when we talk
about relativity, where things like \$\textbackslash{}Delta t\$ can't even be
trivially defined, it is almost completely useless without complicated
modifications.

So it was almost exactly one hundred years after Newton's laws that two people
named \href{http://en.wikipedia.org/wiki/Joseph-Louis_Lagrange}{Lagrange} and
\href{http://en.wikipedia.org/wiki/Leonhard_Euler}{Euler} (who is the "e" in
\$e\$) followed a wild hunch that ended up paying off.

\subsection{Lagrangian Mechanics}

To understand Lagrangian Mechanics, we must abandon our idea of "force" as the
fundamental phenomenon. Instead of forces, we deal with "potential fields".

You can imagine potential fields as a roller coaster track, or as a landscape of
rolling grassy hills. The height of a track or a landscape at that point
corresponds to the value of the potential at that point. Potential fields work
like this: Every object "wants" to go \emph{downwards} in a potential field
-\/-\/- it will want to go in the direction (backwards/forwards for the roller
coaster, north/south/east/west for the hilly landscape) that will take it
downwards. We don't care why, or how -\/-\/- it just "wants" to. And the steeper
the downwardness, the greater the compulsion.

We call this potential field \$U(\textbackslash{}vec\{r\})\$, which means "\$U\$
at the point \$\textbackslash{}vec\{r\}\$". (\$\textbackslash{}vec\{r\}\$
denotes a point in space)

Relating this to \$F = m a\$, the force on the object is now equal to the
steepness of the potential field at the point where the object is, and in the
direction that would allow the object to go downwards in potential. Objects
always wish to minimize their potential, and do so as fast as they can. In
mathematical terminology, we say that \$F(\textbackslash{}vec\{r\}) = -
\textbackslash{}vec\{\textbackslash{}nabla\} U(\textbackslash{}vec\{r\})\$.

\includegraphics{/img/entries/path-integral-intro/potential3d.png}

\includegraphics{/img/entries/path-integral-intro/gradient.png}

Now, for Lagrangian Mechanics:

Let's say I tell you an object's location at time \$t\emph{0\$, and its location
later at time \$t}1\$, and the potential energy field. What path did that object
take to get from point A to point B?

A pretty open question, right? You don't really have that much information to go
off of. You just know point A and point B. It could have taken any path, for all
we know! If we only knew \$F = m a\$, not only would we be at a complete loss at
how to even start, but we wouldn't even know if there was only one or even a
hundred valid paths a particle could have taken.

The solution to this problem is actually rather unexpected. Consider every
single path/curve from point A to point B. Every single one. Now, assign each
path a number known as the \textbf{Action}:

\begin{enumerate}
\tightlist
\item
  For every point, add up the "Kinetic Energy" at that point, which, for
  classical mechanics, is the square of the object's speed multiplied by
  \$\textbackslash{}frac\{1\}\{2\} m\$.
\item
  For every point, add up \$U(\textbackslash{}vec\{r\})\$ at that point.
\item
  Subtract (2) from (1).
\end{enumerate}

Think about every possible path. Calculate the action for each one. The path
that the object takes is \emph{the path with the lowest action}

It's almost as if the object "does the math" in its head: "I'm going to go from
here to there...let me calculate which path I can take has the lowest action.
Okay, got it!"

Lagrangian Mechanics provides for us a way to find out just what path an object
must have taken to get from point A to point B.

As it turns out, looking at things this way opens up entire worlds of
understanding. For example, just from this, we find that \emph{total energy is
conserved} over time for a closed system (trust me on this; the calculus is
slightly tricky). We also have a formulation that works fine under Special
Relativity in all frames of reference with almost no tweaks. And yes, if you
actually do find the path of lowest action, the path will somehow magically
always follow the state-updating equations \$F = m a\$. It's just now we have a
much more insighftul and meaningful way to look at the universe:

Paths \textbf{always attempt to minimize their action}.

Okay. We don't have that much time to spend on this, or its philosophical
implications, so we're going to move on now to Quantum Mechanics.

\section{Quantum Mechanics}

\subsection{Schrödinger Formulation}

If there was one thing that "everyone" knew about quantum mechanics, it would
either be
\href{http://en.wikipedia.org/wiki/Schr\%C3\%B6dinger's_cat}{Scrödinger's Cat}
or the fact that objects are no longer "for sure" anywhere. They are only
\emph{probably} somewhere.

How can we then analyze the behavior of \emph{anything}? If everything is just a
probability, and nothing is certain, we can't really use the same
"state-updating functions" that we used to rely on, because the positions and
velocities of the objects in question don't even have well-defined values.

Physicists' first solutions involved creating a new "state" that did not involve
particles at all. This "state" described the state of the universe, but not in
terms of particles and positions and velocities. It is a new \emph{abstract}
state. Then, they invented the equivalent of an \$F = m a\$ for this abstract
state -\/-\/- an equation that, for every abstract state at time \$t\emph{0\$,
gives you the abstract state at time \$t}0 + \textbackslash{}Delta t\$.

This approach is useful...just like \$F = m a\$ was useful. But it inherits all
of the problems of \$F = m a\$. How can we apply what we learned about actions
and Lagrangian mechanics to Quantum Mechanics? How do we make Lagrangian
mechanics "quantum"?

\subsection{Path Integral Formulation}

The answer is a bit simple, actually.

Instead of saying "the object will chose the path with the least action", we say
\textbf{the object chooses a random path, choosing lower-action paths more
often}. That is, if an electron is shot from point A to point B, the electron
picks a random path from point A to point B. It is a \emph{weighted random
choice} based on the action of each path -\/-\/- if Path
\$\textbackslash{}alpha\$ has lower action than Path \$\textbackslash{}beta\$,
the electron will pick path \$\textbackslash{}alpha\$ more often than path
\$\textbackslash{}beta\$.

There are some small technical differences (the process of calculating the
action is slightly different, and you end up summing over complex numbers for
certain reasons), but the fundamental principle remains the same.

So say we have an electron floating around a hydrogen atom (a hydrogen atom
creates a very pretty and easy to work with potential field). We know it is at
point A at time \$t\emph{0\$, and point B at time \$t}1\$. What path did the
electron take to get there?

Simple: We don't know. But we can say that it \emph{probably} took the path with
the least action. It \emph{could have also} taken the path with the \emph{second
to least} action...but that's just slightly less likely. It \emph{probably did
not} take the path with the greatest action...but who knows -\/-\/- it might
have! It's like it rolls a dice to determine which path it goes on, but the dice
is weighted so that lower-action paths are rolled more often than higher-action
paths.

The electron \emph{wants} to take the lowest-action path...but sometimes decides
not to.

So now we see what Lagrangian Mechanics in classical mechanics really \emph{is}:
It's quantum mechanics, except that the lowest-action path is \emph{so much
likelier} than any other path that we almost never see the second-to-least
action path taken.

As it turns out, like Lagrangian mechanics opened eyes to new worlds in
classical mechanics, the Path Integral formulation{[}\^{}naming{]} of quantum
mechanics opened up totally new worlds that the previous "state updating
formula" approach could have never dreamed of.

\section{Implications}

Okay, so what does this all have to do with us?

How many processes do we know that can be modeled by something trying to
minimize itself, but not always succeeding? What data patterns can be unveiled
by modeling things this way?

I'll leave this question slightly open-ended, but I'm also going to hint at the
next installment's contents.

\subsection{Particle in a potential}

Let's go back again to our electron next to an atom. Let's say that this
electron will move around and return back to its current position at time \$t\_0
+ \textbackslash{}Delta t\$, for very large \$\textbackslash{}Delta t\$. From
what we learned, this electron can really take any path it wants, going anywhere
in the universe and back again. Any closed loop that that zig zags or curls
anywhere is a valid path.

We can "pick" a random path, weighted by the action, and see where the electron
goes in that path. See where we find the electron along points in the path.
After many picks, we start seeing where the electron is "most likely to be". We
find the probability distribution of an electron in that potential.

We now have a way, given any quantum potential, to find the probability
distribution of a particle in that potential.

From here we can also find the particle's average energy, and many other
properties of a particle given an arbitrary quantum potential.

Now let's implement it.

\end{document}
