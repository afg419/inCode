\documentclass[]{article}
\usepackage{lmodern}
\usepackage{amssymb,amsmath}
\usepackage{ifxetex,ifluatex}
\usepackage{fixltx2e} % provides \textsubscript
\ifnum 0\ifxetex 1\fi\ifluatex 1\fi=0 % if pdftex
  \usepackage[T1]{fontenc}
  \usepackage[utf8]{inputenc}
\else % if luatex or xelatex
  \ifxetex
    \usepackage{mathspec}
    \usepackage{xltxtra,xunicode}
  \else
    \usepackage{fontspec}
  \fi
  \defaultfontfeatures{Mapping=tex-text,Scale=MatchLowercase}
  \newcommand{\euro}{€}
\fi
% use upquote if available, for straight quotes in verbatim environments
\IfFileExists{upquote.sty}{\usepackage{upquote}}{}
% use microtype if available
\IfFileExists{microtype.sty}{\usepackage{microtype}}{}
\usepackage[margin=1in]{geometry}
\usepackage{color}
\usepackage{fancyvrb}
\newcommand{\VerbBar}{|}
\newcommand{\VERB}{\Verb[commandchars=\\\{\}]}
\DefineVerbatimEnvironment{Highlighting}{Verbatim}{commandchars=\\\{\}}
% Add ',fontsize=\small' for more characters per line
\newenvironment{Shaded}{}{}
\newcommand{\AlertTok}[1]{\textcolor[rgb]{1.00,0.00,0.00}{\textbf{#1}}}
\newcommand{\AnnotationTok}[1]{\textcolor[rgb]{0.38,0.63,0.69}{\textbf{\textit{#1}}}}
\newcommand{\AttributeTok}[1]{\textcolor[rgb]{0.49,0.56,0.16}{#1}}
\newcommand{\BaseNTok}[1]{\textcolor[rgb]{0.25,0.63,0.44}{#1}}
\newcommand{\BuiltInTok}[1]{#1}
\newcommand{\CharTok}[1]{\textcolor[rgb]{0.25,0.44,0.63}{#1}}
\newcommand{\CommentTok}[1]{\textcolor[rgb]{0.38,0.63,0.69}{\textit{#1}}}
\newcommand{\CommentVarTok}[1]{\textcolor[rgb]{0.38,0.63,0.69}{\textbf{\textit{#1}}}}
\newcommand{\ConstantTok}[1]{\textcolor[rgb]{0.53,0.00,0.00}{#1}}
\newcommand{\ControlFlowTok}[1]{\textcolor[rgb]{0.00,0.44,0.13}{\textbf{#1}}}
\newcommand{\DataTypeTok}[1]{\textcolor[rgb]{0.56,0.13,0.00}{#1}}
\newcommand{\DecValTok}[1]{\textcolor[rgb]{0.25,0.63,0.44}{#1}}
\newcommand{\DocumentationTok}[1]{\textcolor[rgb]{0.73,0.13,0.13}{\textit{#1}}}
\newcommand{\ErrorTok}[1]{\textcolor[rgb]{1.00,0.00,0.00}{\textbf{#1}}}
\newcommand{\ExtensionTok}[1]{#1}
\newcommand{\FloatTok}[1]{\textcolor[rgb]{0.25,0.63,0.44}{#1}}
\newcommand{\FunctionTok}[1]{\textcolor[rgb]{0.02,0.16,0.49}{#1}}
\newcommand{\ImportTok}[1]{#1}
\newcommand{\InformationTok}[1]{\textcolor[rgb]{0.38,0.63,0.69}{\textbf{\textit{#1}}}}
\newcommand{\KeywordTok}[1]{\textcolor[rgb]{0.00,0.44,0.13}{\textbf{#1}}}
\newcommand{\NormalTok}[1]{#1}
\newcommand{\OperatorTok}[1]{\textcolor[rgb]{0.40,0.40,0.40}{#1}}
\newcommand{\OtherTok}[1]{\textcolor[rgb]{0.00,0.44,0.13}{#1}}
\newcommand{\PreprocessorTok}[1]{\textcolor[rgb]{0.74,0.48,0.00}{#1}}
\newcommand{\RegionMarkerTok}[1]{#1}
\newcommand{\SpecialCharTok}[1]{\textcolor[rgb]{0.25,0.44,0.63}{#1}}
\newcommand{\SpecialStringTok}[1]{\textcolor[rgb]{0.73,0.40,0.53}{#1}}
\newcommand{\StringTok}[1]{\textcolor[rgb]{0.25,0.44,0.63}{#1}}
\newcommand{\VariableTok}[1]{\textcolor[rgb]{0.10,0.09,0.49}{#1}}
\newcommand{\VerbatimStringTok}[1]{\textcolor[rgb]{0.25,0.44,0.63}{#1}}
\newcommand{\WarningTok}[1]{\textcolor[rgb]{0.38,0.63,0.69}{\textbf{\textit{#1}}}}
\usepackage{fancyvrb}
\ifxetex
  \usepackage[setpagesize=false, % page size defined by xetex
              unicode=false, % unicode breaks when used with xetex
              xetex]{hyperref}
\else
  \usepackage[unicode=true]{hyperref}
\fi
\hypersetup{breaklinks=true,
            bookmarks=true,
            pdfauthor={Justin Le},
            pdftitle={Introduction to Singletons (Part 2)},
            colorlinks=true,
            citecolor=blue,
            urlcolor=blue,
            linkcolor=magenta,
            pdfborder={0 0 0}}
\urlstyle{same}  % don't use monospace font for urls
% Make links footnotes instead of hotlinks:
\renewcommand{\href}[2]{#2\footnote{\url{#1}}}
\setlength{\parindent}{0pt}
\setlength{\parskip}{6pt plus 2pt minus 1pt}
\setlength{\emergencystretch}{3em}  % prevent overfull lines
\setcounter{secnumdepth}{0}
\VerbatimFootnotes % allows verbatim text in footnotes

\title{Introduction to Singletons (Part 2)}
\author{Justin Le}
\date{January 9, 2018}

\begin{document}
\maketitle

\emph{Originally posted on
\textbf{\href{https://blog.jle.im/entry/introduction-to-singletons-2.html}{in
Code}}.}

Welcome back to our journey through the singleton design pattern and the great
\emph{\href{http://hackage.haskell.org/package/singletons}{singletons}} library!

This post is a direct continuation of
\href{https://blog.jle.im/entry/introduction-to-singletons-1.html}{Part 1}, so
be sure to check that out first if you haven't already! If you hare just jumping
in now, I suggest taking some time to to through the exercises if you haven't
already!

Again, code is built on \emph{GHC 8.2.2} with the
\emph{\href{https://www.stackage.org/lts-10.0}{lts-10.0}} snapshot (so,
singletons-2.3.1).

\hypertarget{review}{%
\section{Review}\label{review}}

Let's return to our \texttt{Door} type:

\begin{Shaded}
\begin{Highlighting}[]
\CommentTok{-- source: https://github.com/mstksg/inCode/tree/master/code-samples/singletons/Door2.hs#L19-L25}

\FunctionTok{$}\NormalTok{(singletons [d|}
\NormalTok{  data DoorState = Opened | Closed | Locked}
\NormalTok{    deriving (Show, Eq)}
\NormalTok{  |])}

\KeywordTok{data} \DataTypeTok{Door}\OtherTok{ ::} \DataTypeTok{DoorState} \OtherTok{->} \DataTypeTok{Type} \KeywordTok{where}
    \DataTypeTok{UnsafeMkDoor}\OtherTok{ ::}\NormalTok{ \{}\OtherTok{ doorMaterial ::} \DataTypeTok{String}\NormalTok{ \} }\OtherTok{->} \DataTypeTok{Door}\NormalTok{ s}
\end{Highlighting}
\end{Shaded}

First, this derives the \emph{type} \texttt{DoorState} with the values
\texttt{Opened}, \texttt{Closed}, and \texttt{Locked}, and also the \emph{kind}
\texttt{DoorState} with the \emph{types} \texttt{\textquotesingle{}Opened},
\texttt{\textquotesingle{}Closed}, and \texttt{\textquotesingle{}Locked}. We
then also derive the singletons (and implicit-style typeclass instances,
reflectors, etc.) with the template haskell.

Then, there's \texttt{Door}. \texttt{Door} is great! It is an \emph{indexed data
type} (indexed by a type of kind \texttt{DoorState}) in that picking a different
type variable gives a different ``type'' of Door:

\begin{itemize}
\tightlist
\item
  \texttt{Door\ \textquotesingle{}Opened} is a type that represents the type of
  an opened door
\item
  \texttt{Door\ \textquotesingle{}Closed} is a \emph{different} type that
  represents the type of a \emph{closed} door
\item
  \texttt{Door\ \textquotesingle{}Locked} is yet another (third) type that
  represents the type of a \emph{locked} door.
\end{itemize}

So, really, when we define \texttt{Door\ s}, we really are defining \emph{three
distinct} types\footnote{And also a not-so-obvious fourth type,
  \texttt{forall\ s.\ Door\ s}, which is a subtype of all of those three!}.

This is great and all, but isn't Haskell a language with static, compile-time
types? Doesn't that mean that we have to know if our doors are opened, closed,
or locked at compile-time?

This is something we can foresee being a big issue. It's easy enough to create a
\texttt{Door\ s} if you know \texttt{s} at compile-time by just typing in a type
annotation (\texttt{UnsafeMkDoor\ "Oak"\ ::\ Door\ \textquotesingle{}Opened}) or
by using a monomorphic constructor (\texttt{mkDoor\ @SOpened\ "Oak"}). But what
if we \emph{don't} know \texttt{s} at compile-time?

To learn how to do this, we first need to learn how to \emph{not care}.

\hypertarget{ditching-the-phantom}{%
\section{Ditching the Phantom}\label{ditching-the-phantom}}

Sometimes we don't \emph{actually} care about the state of the door in the
\emph{type} of the door. We don't want \texttt{Door\ \textquotesingle{}Opened}
and \texttt{Door\ \textquotesingle{}Closed}\ldots{}we want a type to just
represent a door, without the status in its type.

This might come about a bunch of different ways. Maybe you're reading a
\texttt{Door} data from a serialization format, and you want to be able to parse
\emph{any} door (whatever door is serialized).

To learn how to not care, we can describe a type for a door that does \emph{not}
have its status in its type.

We have a couple of options here. First, we can create a new type
\texttt{SomeDoor} that is the same as \texttt{Door}, except instead of keeping
its status in its type, it keeps it as a runtime value:

\begin{Shaded}
\begin{Highlighting}[]
\KeywordTok{data} \DataTypeTok{SomeDoor} \FunctionTok{=} \DataTypeTok{MkSomeDoor}
\NormalTok{    \{}\OtherTok{ someDoorState    ::} \DataTypeTok{DoorState}
\NormalTok{    ,}\OtherTok{ someDoorMaterial ::} \DataTypeTok{String}
\NormalTok{    \}}

\CommentTok{-- or, in GADT syntax}
\KeywordTok{data} \DataTypeTok{SomeDoor}\OtherTok{ ::} \DataTypeTok{Type} \KeywordTok{where}
    \DataTypeTok{MkSomeDoor} \OtherTok{::}
\NormalTok{      \{}\OtherTok{ someDoorState    ::} \DataTypeTok{DoorState}
\NormalTok{      ,}\OtherTok{ someDoorMaterial ::} \DataTypeTok{String}
\NormalTok{      \} }\OtherTok{->} \DataTypeTok{SomeDoor}
\end{Highlighting}
\end{Shaded}

Note the similarity of \texttt{SomeDoor}'s declaration to \texttt{Door}'s
declaration above. It's mostly the same, except, instead of \texttt{DoorState}
being a type parameter, it is instead a runtime value inside \texttt{SomeDoor}.

Now, this is actually a type that we \emph{could} have been using this entire
time, if we didn't care about type safety. In the real world and in real
applications, we actually might have written \texttt{SomeDoor} \emph{before} we
ever thought about \texttt{Door} with a phantom type. It's definitely the more
typical ``standard'' Haskell thing.

\texttt{SomeDoor} is great. But because it's a completely different type, we
can't re-use any of our \texttt{Door} functions on this \texttt{SomeDoor}. We
potentially have to write the same function twice for both \texttt{Door} and
\texttt{SomeDoor}, because they have different implementations.

\hypertarget{the-existential-datatype}{%
\subsection{The Existential Datatype}\label{the-existential-datatype}}

However, there's another path we can take. With the power of singletons, we can
actually implement \texttt{SomeDoor} \emph{in terms of} \texttt{Door}, using an
\textbf{existential data type}:

\begin{Shaded}
\begin{Highlighting}[]
\CommentTok{-- using existential constructor syntax}
\KeywordTok{data} \DataTypeTok{SomeDoor} \FunctionTok{=}\NormalTok{ forall s}\FunctionTok{.} \DataTypeTok{MkSomeDoor}\NormalTok{ (}\DataTypeTok{Sing}\NormalTok{ s) (}\DataTypeTok{Door}\NormalTok{ s)}

\CommentTok{-- or, using GADT syntax (preferred)}
\CommentTok{-- source: https://github.com/mstksg/inCode/tree/master/code-samples/singletons/Door2.hs#L56-L57}

\KeywordTok{data} \DataTypeTok{SomeDoor}\OtherTok{ ::} \DataTypeTok{Type} \KeywordTok{where}
    \DataTypeTok{MkSomeDoor}\OtherTok{ ::} \DataTypeTok{Sing}\NormalTok{ s }\OtherTok{->} \DataTypeTok{Door}\NormalTok{ s }\OtherTok{->} \DataTypeTok{SomeDoor}
\end{Highlighting}
\end{Shaded}

\texttt{MkSomeDoor} is a constructor for an existential data type, meaning that
the data type ``hides'' a type variable \texttt{s}. Note the type
(\texttt{Sing\ s\ -\textgreater{}\ Door\ s\ -\textgreater{}\ SomeDoor}) and how
the result type (\texttt{SomeDoor}) \emph{forgets} the \texttt{s} and hides all
traces of it. Think of it like a type variable sponge -- type variable goes in,
but it's absorbed opaquely into the result type.

Note the similarities between our original \texttt{SomeDoor} and this one.

\begin{Shaded}
\begin{Highlighting}[]
\CommentTok{-- | Re-implementing door}
\KeywordTok{data} \DataTypeTok{SomeDoor} \KeywordTok{where}
    \DataTypeTok{MkSomeDoor}\OtherTok{ ::} \DataTypeTok{DoorState} \OtherTok{->} \DataTypeTok{String} \OtherTok{->} \DataTypeTok{SomeDoor}

\CommentTok{-- | Re-using Door, as an existential type}
\KeywordTok{data} \DataTypeTok{SomeDoor} \KeywordTok{where}
    \DataTypeTok{MkSomeDoor}\OtherTok{  ::} \DataTypeTok{Sing}\NormalTok{ s  }\OtherTok{->} \DataTypeTok{Door}\NormalTok{ s }\OtherTok{->} \DataTypeTok{SomeDoor}
                            \CommentTok{-- ^ data Door s = UnsafeMkDoor String}
\end{Highlighting}
\end{Shaded}

Basically, our type before re-implements \texttt{Door}. But the new one actually
directly uses the original \texttt{Door\ s}. This means we can \emph{directly}
re-use our \texttt{Door} functions on \texttt{SomeDoor}s, without needing to
write completely new implementations.

In Haskell, existential data types are pretty nice, syntactically, to work with.
Let's write some basic functions to see. First, a function to ``make'' a
\texttt{SomeDoor} from a \texttt{Door}:

\begin{Shaded}
\begin{Highlighting}[]
\CommentTok{-- source: https://github.com/mstksg/inCode/tree/master/code-samples/singletons/Door2.hs#L59-L63}

\OtherTok{fromDoor ::} \DataTypeTok{Sing}\NormalTok{ s }\OtherTok{->} \DataTypeTok{Door}\NormalTok{ s }\OtherTok{->} \DataTypeTok{SomeDoor}
\NormalTok{fromDoor }\FunctionTok{=} \DataTypeTok{MkSomeDoor}

\OtherTok{fromDoor_ ::} \DataTypeTok{SingI}\NormalTok{ s }\OtherTok{=>} \DataTypeTok{Door}\NormalTok{ s }\OtherTok{->} \DataTypeTok{SomeDoor}
\NormalTok{fromDoor_ }\FunctionTok{=}\NormalTok{ fromDoor sing}
\end{Highlighting}
\end{Shaded}

So that's how we \emph{make} one\ldots{}how do we \emph{use} it? Let's port our
\texttt{Door} functions to \texttt{SomeDoor}, by re-using our pre-existing
functions whenever we can, and \emph{pattern matching} on \texttt{MkSomeDoor}:

\begin{Shaded}
\begin{Highlighting}[]
\CommentTok{-- source: https://github.com/mstksg/inCode/tree/master/code-samples/singletons/Door2.hs#L65-L72}

\OtherTok{closeSomeOpenedDoor ::} \DataTypeTok{SomeDoor} \OtherTok{->} \DataTypeTok{Maybe} \DataTypeTok{SomeDoor}
\NormalTok{closeSomeOpenedDoor (}\DataTypeTok{MkSomeDoor}\NormalTok{ s d) }\FunctionTok{=} \KeywordTok{case}\NormalTok{ s }\KeywordTok{of}
    \DataTypeTok{SOpened} \OtherTok{->} \DataTypeTok{Just} \FunctionTok{.}\NormalTok{ fromDoor_ }\FunctionTok{$}\NormalTok{ closeDoor d}
    \DataTypeTok{SClosed} \OtherTok{->} \DataTypeTok{Nothing}
    \DataTypeTok{SLocked} \OtherTok{->} \DataTypeTok{Nothing}

\OtherTok{lockAnySomeDoor ::} \DataTypeTok{SomeDoor} \OtherTok{->} \DataTypeTok{SomeDoor}
\NormalTok{lockAnySomeDoor (}\DataTypeTok{MkSomeDoor}\NormalTok{ s d) }\FunctionTok{=}\NormalTok{ fromDoor_ }\FunctionTok{$}\NormalTok{ lockAnyDoor s d}
\end{Highlighting}
\end{Shaded}

Using an existential wrapper with a singleton makes this pretty simple -- just a
simple unwrapping and re-wrapping! Imagine having to re-implement all of these
functions for a completely different type, and having to re-implement all of our
previous \texttt{Door} functions.

It's important to remember that the secret ingredient here is the
\texttt{Sing\ s} we store inside \texttt{MkSomeDoor} -- it gives our pattern
matchers the ability to deduce the \texttt{s} type. Without it, the \texttt{s}
would be lost forever.

If \texttt{MkSomeDoor} did not have the \texttt{Sing}:

\begin{Shaded}
\begin{Highlighting}[]
\KeywordTok{data} \DataTypeTok{SomeDoor} \KeywordTok{where}
    \DataTypeTok{MkSomeDoor}\OtherTok{  ::} \DataTypeTok{Door}\NormalTok{ s }\OtherTok{->} \DataTypeTok{SomeDoor}       \CommentTok{-- no Sing s ???}
\end{Highlighting}
\end{Shaded}

It would then be impossible to write \texttt{closeSomeOpenedDoor} in a way that
only works on opened doors:

\begin{Shaded}
\begin{Highlighting}[]
\OtherTok{closeSomeOpenedDoor ::} \DataTypeTok{SomeDoor} \OtherTok{->} \DataTypeTok{Maybe} \DataTypeTok{SomeDoor}
\NormalTok{closeSomeOpenedDoor (}\DataTypeTok{MkSomeDoor}\NormalTok{ d) }\FunctionTok{=}
            \CommentTok{-- is the door opened, closed, or locked?}
            \CommentTok{-- there's no way to know!}
            \CommentTok{-- curses, type erasure!}
\end{Highlighting}
\end{Shaded}

\hypertarget{the-link}{%
\subsection{The Link}\label{the-link}}

It's important to remember that our original separate-implementation
\texttt{SomeDoor} is, functionally, identical to the new code-reusing
\texttt{Door}. All of the contents are isomorphic with each other, and you could
write a function converting one to the other. This is because \emph{having an
existentially quantified singleton is the same as having a value of the
corresponding type.} Having an existentially quantified \texttt{SingDS\ s} is
\emph{the same as} having a value of type \texttt{DoorState}.

In fact, the \emph{singletons} library gives us a direct existential wrapper:

\begin{Shaded}
\begin{Highlighting}[]
\CommentTok{-- from singletons (not the actual definition)}
\KeywordTok{data} \DataTypeTok{SomeSing} \DataTypeTok{DoorState}\OtherTok{ ::} \DataTypeTok{Type} \KeywordTok{where}
    \DataTypeTok{SomeSing}\OtherTok{ ::} \DataTypeTok{Sing}\NormalTok{ s }\OtherTok{->} \DataTypeTok{SomeSing} \DataTypeTok{DoorState}
\end{Highlighting}
\end{Shaded}

There are three values of type \texttt{SomeSing\ DoorState}:

\begin{Shaded}
\begin{Highlighting}[]
\DataTypeTok{SomeSing} \DataTypeTok{SOpened}\OtherTok{ ::} \DataTypeTok{SomeSing} \DataTypeTok{DoorState}
\DataTypeTok{SomeSing} \DataTypeTok{SClosed}\OtherTok{ ::} \DataTypeTok{SomeSing} \DataTypeTok{DoorState}
\DataTypeTok{SomeSing} \DataTypeTok{SLocked}\OtherTok{ ::} \DataTypeTok{SomeSing} \DataTypeTok{DoorState}
\end{Highlighting}
\end{Shaded}

A value of type \texttt{SomeSing\ DoorState} (which contains an existentially
quantified \texttt{Sing\ s} -- a \texttt{SingDS}) is \emph{the same} as a value
of type \texttt{DoorState}. The two types are identical! (Or, well, isomorphic.
As a fun exercise, write out the explicit isomorphism -- the
\texttt{SomeSing\ DoorState\ -\textgreater{}\ DoorState} and the
\texttt{DoorState\ -\textgreater{}\ SomeSing\ DoorState}).

Our new \texttt{SomeDoor} containing an existentially quantified
\texttt{Sing\ s} is the same as our first \texttt{SomeDoor} containing just a
\texttt{DoorState}.

\hypertarget{why-do-we-sing}{%
\subsubsection{Why do we sing?}\label{why-do-we-sing}}

If they're identical, why use a \texttt{Sing} or the new \texttt{SomeDoor} at
all? Why not just use a \texttt{DoorState} value?

One main reason (besides allowing code-reuse like we did earlier) is that
\emph{using the singleton lets us directly recover the type}. Essentially, a
\texttt{Sing\ s} not only contains whether it is Opened/Closed/Locked (like a
\texttt{DoorState} would), but also it contains it in a way that GHC can use to
\emph{bring it all back} to the type level.

A \texttt{forall\ s.\ SomeDoor\ (Sing\ s)\ (Door\ s)} essentially contains
\texttt{s} \emph{with} \texttt{Door\ s}. When you see this, you \emph{should
read this as} \texttt{forall\ s.\ SomeDoor\ s\ (Door\ s)} (and, indeed, this is
similar to how it is written in dependently typed languages.)

It's kind of like how, when you're used to reading Applicative style, you start
seeing \texttt{f\ \textless{}\$\textgreater{}\ x\ \textless{}*\textgreater{}\ y}
and reading it like \texttt{f\ x\ y}. When you see
\texttt{forall\ s.\ SomeDoor\ (Sing\ s)\ (Door\ s)}, you should read (the
pseudo-haskell) \texttt{forall\ s.\ SomeDoor\ s\ (Door\ s)}. The role of
\texttt{Sing\ s} there is, like in Part 1, simply to be a run-time stand-in for
the type \texttt{s} itself.

So, for our original \texttt{Door\ s} functions, we need to know \texttt{s} at
runtime -- storing the \texttt{Sing\ s} gives GHC exactly that. Once you get the
\texttt{Sing\ s} back, you can now use it in all of our type-safe functions from
Part 1, and you're back in type-safe land.\footnote{You might have noticed I was
  a bit sneaky by jumping straight \texttt{SomeDoor} when we already had a
  perfectly good ``I don't care'' option. We used it last post!

\begin{Shaded}
\begin{Highlighting}[]
\OtherTok{lockAnyDoor ::} \DataTypeTok{Sing}\NormalTok{ s }\OtherTok{->} \DataTypeTok{Door}\NormalTok{ s }\OtherTok{->} \DataTypeTok{Door}\NormalTok{ '}\DataTypeTok{Locked}
\end{Highlighting}
\end{Shaded}

  This does work! \texttt{lockAnyDoor} takes a \texttt{Door\ s} and doesn't
  ``care'' about what \texttt{s} it gets (it's parametrically polymorphic).

  So, this normal ``parametrically polymorphic'' way is how we have, in the
  past, treated functions that \emph{can take} a \texttt{Door} with an
  \texttt{s} we don't want the type system to care about. However, the reason we
  need \texttt{SomeDoor} and existentially quantified types is for the situation
  where we want to \emph{return} something that we want to the type system to
  not care about.}

\hypertarget{some-lingo}{%
\subsection{Some Lingo}\label{some-lingo}}

In the language of dependently typed programming, we call \texttt{SomeDoor} a
\textbf{dependent sum}, because you can imagine it basically as a sum type:

\begin{Shaded}
\begin{Highlighting}[]
\KeywordTok{data} \DataTypeTok{SomeDoor} \FunctionTok{=} \DataTypeTok{SDOpened}\NormalTok{ (}\DataTypeTok{Door}\NormalTok{ '}\DataTypeTok{Opened}\NormalTok{)}
              \FunctionTok{|} \DataTypeTok{SDClosed}\NormalTok{ (}\DataTypeTok{Door}\NormalTok{ '}\DataTypeTok{Closed}\NormalTok{)}
              \FunctionTok{|} \DataTypeTok{SDLocked}\NormalTok{ (}\DataTypeTok{Door}\NormalTok{ '}\DataTypeTok{Locked}\NormalTok{)}
\end{Highlighting}
\end{Shaded}

A three-way sum between a \texttt{Door\ \textquotesingle{}Opened}, a
\texttt{Door\ \textquotesingle{}Closed}, and a
\texttt{Door\ \textquotesingle{}Locked}, essentially. If you have a
\texttt{SomeDoor}, it's \emph{either} an opened door, a closed door, or a locked
door. Try looking at this new \texttt{SomeDoor} until you realize that this type
is the same type as the previous \texttt{SomeDoor}!

You might also see \texttt{SomeDoor} called a \textbf{dependent pair} -- it's a
``tuple'' where the \emph{type} of the second item (our \texttt{Door\ s}) is
determined by the \emph{value} of the first item (our \texttt{Sing\ s}).

In Idris, we could write \texttt{SomeDoor} as a type alias, using its native
\href{http://docs.idris-lang.org/en/latest/tutorial/typesfuns.html\#dependent-pairs}{dependent
pair syntactic sugar}, as \texttt{(s\ **\ Door\ s)}. The \emph{value} of the
first item reveals to us (through a pattern match, in Haskell) the \emph{type}
of the second.

\hypertarget{types-at-runtime}{%
\subsection{Types at Runtime}\label{types-at-runtime}}

With this new tool, we finally have enough to build a function to ``make'' a
door with the status unknown until runtime:

\begin{Shaded}
\begin{Highlighting}[]
\OtherTok{mkSomeDoor ::} \DataTypeTok{DoorState} \OtherTok{->} \DataTypeTok{String} \OtherTok{->} \DataTypeTok{SomeDoor}
\NormalTok{mkSomeDoor }\FunctionTok{=}\NormalTok{ \textbackslash{}}\KeywordTok{case}
    \DataTypeTok{Opened} \OtherTok{->}\NormalTok{ fromDoor_ }\FunctionTok{.}\NormalTok{ mkDoor }\DataTypeTok{SOpened}
    \DataTypeTok{Closed} \OtherTok{->}\NormalTok{ fromDoor_ }\FunctionTok{.}\NormalTok{ mkDoor }\DataTypeTok{SClosed}
    \DataTypeTok{Locked} \OtherTok{->}\NormalTok{ fromDoor_ }\FunctionTok{.}\NormalTok{ mkDoor }\DataTypeTok{SLocked}
\end{Highlighting}
\end{Shaded}

\begin{Shaded}
\begin{Highlighting}[]
\NormalTok{ghci}\FunctionTok{>} \KeywordTok{let}\NormalTok{ mySomeDoor }\FunctionTok{=}\NormalTok{ mkSomeDoor }\DataTypeTok{Opened} \StringTok{"Birch"}
\NormalTok{ghci}\FunctionTok{>} \FunctionTok{:}\NormalTok{t mySomeDoor}
\DataTypeTok{SomeDoor}
\NormalTok{ghci}\FunctionTok{>}\NormalTok{ putStrLn }\FunctionTok{$} \KeywordTok{case}\NormalTok{ mySomeDoor }\KeywordTok{of}
        \DataTypeTok{MkSomeDoor} \DataTypeTok{SOpened}\NormalTok{ _ }\OtherTok{->} \StringTok{"mySomeDoor was opened!"}
        \DataTypeTok{MkSomeDoor} \DataTypeTok{SClosed}\NormalTok{ _ }\OtherTok{->} \StringTok{"mySomeDoor was closed!"}
        \DataTypeTok{MkSomeDoor} \DataTypeTok{SLocked}\NormalTok{ _ }\OtherTok{->} \StringTok{"mySomeDoor was locked!"}
\NormalTok{mySomeDoor was opened}\FunctionTok{!}
\end{Highlighting}
\end{Shaded}

Using \texttt{mkSomeDoor}, we can truly pass in a \texttt{DoorState} that we
generate at runtime (from IO, or a user prompt, or a configuration file, maybe),
and create a \texttt{Door} based on it.

Take \emph{that}, type erasure! :D

\hypertarget{the-existential-type}{%
\subsection{The Existential Type}\label{the-existential-type}}

An \emph{existentially quantified} type is one that is hidden to the
user/consumer, but directly chosen by the producer. The producer chooses the
type, and the user has to handle any possible type that the producer gave.

This is in direct contrast to the \emph{universally quantified} type (which most
Haskellers are used to seeing), where the type is directly chosen by the
\emph{user}. The user chooses the type, and the producer has to handle any
possible type that the user asks for.

For example, a function like:

\begin{Shaded}
\begin{Highlighting}[]
\NormalTok{read}\OtherTok{ ::} \DataTypeTok{Read}\NormalTok{ a }\OtherTok{=>} \DataTypeTok{String} \OtherTok{->}\NormalTok{ a}
\end{Highlighting}
\end{Shaded}

Is universally quantified over \texttt{a}: The \emph{caller} of \texttt{read}
gets to pick which type is given. The burden is on the implementor of
\texttt{read} to be able to handle whatever \texttt{a} the user picks.

But, for a value like:

\begin{Shaded}
\begin{Highlighting}[]
\OtherTok{myDoor ::} \DataTypeTok{SomeDoor}
\end{Highlighting}
\end{Shaded}

The type variable \texttt{s} is existentially quantified. The person who
\emph{made} \texttt{myDoor} picked what \texttt{s} was. And, if you \emph{use}
\texttt{myDoor}, you have to be ready to handle \emph{any} \texttt{s} they could
have chosen.

In Haskell, there's another way to express an existentially quantified type: the
CPS-style encoding. This way is useful because it doesn't require creating an
intermediate helper data type. To help us understand it, let's compare a basic
function in both styles. We saw earlier \texttt{mkSomeDoor}, which takes a
\texttt{DoorState} and a \texttt{String} and returns an existentially quantified
\texttt{Door} in the form of \texttt{SomeDoor}:

\begin{Shaded}
\begin{Highlighting}[]
\NormalTok{mkSomeDoor}
\OtherTok{    ::} \DataTypeTok{DoorState}
    \OtherTok{->} \DataTypeTok{String}
    \OtherTok{->} \DataTypeTok{SomeDoor}
\NormalTok{mkSomeDoor s m }\FunctionTok{=} \KeywordTok{case}\NormalTok{ s }\KeywordTok{of}
    \DataTypeTok{Opened} \OtherTok{->}\NormalTok{ fromDoor_ (mkDoor }\DataTypeTok{SOpened}\NormalTok{ m)}
    \DataTypeTok{Closed} \OtherTok{->}\NormalTok{ fromDoor_ (mkDoor }\DataTypeTok{SClosed}\NormalTok{ m)}
    \DataTypeTok{Locked} \OtherTok{->}\NormalTok{ fromDoor_ (mkDoor }\DataTypeTok{SLocked}\NormalTok{ m)}
\end{Highlighting}
\end{Shaded}

The caller of the function can then break open the \texttt{SomeDoor} and must
handle whatever \texttt{s} they find inside.

We can write the same function using a \emph{CPS-style} existential instead:

\begin{Shaded}
\begin{Highlighting}[]
\NormalTok{withDoor}
\OtherTok{    ::} \DataTypeTok{DoorState}
    \OtherTok{->} \DataTypeTok{String}
    \OtherTok{->}\NormalTok{ (forall s}\FunctionTok{.} \DataTypeTok{Sing}\NormalTok{ s }\OtherTok{->} \DataTypeTok{Door}\NormalTok{ s }\OtherTok{->}\NormalTok{ r) }\OtherTok{->}\NormalTok{ r}
\NormalTok{withDoor s m f }\FunctionTok{=} \KeywordTok{case}\NormalTok{ s }\KeywordTok{of}
    \DataTypeTok{Opened} \OtherTok{->}\NormalTok{ f }\DataTypeTok{SOpened}\NormalTok{ (mkDoor }\DataTypeTok{SOpened}\NormalTok{ m)}
    \DataTypeTok{Closed} \OtherTok{->}\NormalTok{ f }\DataTypeTok{SClosed}\NormalTok{ (mkDoor }\DataTypeTok{SClosed}\NormalTok{ m)}
    \DataTypeTok{Locked} \OtherTok{->}\NormalTok{ f }\DataTypeTok{SLocked}\NormalTok{ (mkDoor }\DataTypeTok{SLocked}\NormalTok{ m)}
\end{Highlighting}
\end{Shaded}

With a Rank-N Type, \texttt{withDoor} takes a \texttt{DoorState} and a
\texttt{String} and a \emph{function to handle a \texttt{Door\ s}
polymorphically}. The caller of \texttt{withDoor} must provide a handler that
can handle \emph{any} \texttt{s}, in a uniform and parametrically polymorphic
way. The function then gives the result of the handler function called on the
resulting \texttt{Sing\ s} and \texttt{Door\ s}.

\begin{Shaded}
\begin{Highlighting}[]
\NormalTok{ghci}\FunctionTok{>}\NormalTok{ withDoor }\DataTypeTok{Opened} \StringTok{"Birch"} \FunctionTok{$}\NormalTok{ \textbackslash{}s d }\OtherTok{->} \KeywordTok{case}\NormalTok{ s }\KeywordTok{of}
         \DataTypeTok{SOpened} \OtherTok{->} \StringTok{"Opened door!"}
         \DataTypeTok{SClosed} \OtherTok{->} \StringTok{"Closed door!"}
         \DataTypeTok{SLocked} \OtherTok{->} \StringTok{"Locked door!"}
\DataTypeTok{Opened}\NormalTok{ door}\FunctionTok{!}
\end{Highlighting}
\end{Shaded}

The key to making this work is that your handler function \emph{has to be
polymorphic} over all possible \texttt{s}s. This way, it can handle any
potential \texttt{s} that the producer gives. Essentially, the producer is
``returning'' an \texttt{s} -- existentially quantified.

\hypertarget{reification}{%
\subsection{Reification}\label{reification}}

The general pattern we are exploring here is called \textbf{reification} --
we're taking a dynamic run-time value, and lifting it to the type level as a
type (here, the type variable \texttt{s}). You can think of reification as the
opposite of reflection, and imagine the two as being the ``gateway'' between the
type-safe and unsafe world. In the dynamic world of a \texttt{DoorState}
term-level value, you have no type safety. You live in the world of
\texttt{SomeDoor}, \texttt{closeSomeOpenedDoor}, \texttt{lockAnySomeDoor}, etc.
But, you can \emph{reify} your \texttt{DoorState} value to a \emph{type}, and
enter the type-safe world of \texttt{Door\ s}, \texttt{closeDoor},
\texttt{lockDoor}, and \texttt{lockAnyDoor}.

The \emph{singletons} library automatically generates functions to directly
reify \texttt{DoorState} values:

\begin{Shaded}
\begin{Highlighting}[]
\OtherTok{toSing       ::} \DataTypeTok{DoorState} \OtherTok{->} \DataTypeTok{SomeSing} \DataTypeTok{DoorState}
\OtherTok{withSomeSing ::} \DataTypeTok{DoorState} \OtherTok{->}\NormalTok{ (forall s}\FunctionTok{.} \DataTypeTok{Sing}\NormalTok{ s        }\OtherTok{->}\NormalTok{ r) }\OtherTok{->}\NormalTok{ r}
\OtherTok{withSomeSing ::} \DataTypeTok{DoorState} \OtherTok{->}\NormalTok{ (forall s}\FunctionTok{.} \DataTypeTok{SDoorState}\NormalTok{ s  }\OtherTok{->}\NormalTok{ r) }\OtherTok{->}\NormalTok{ r}
                                     \CommentTok{-- ^ using the convenience type synonym}
\end{Highlighting}
\end{Shaded}

The first one reifies a \texttt{DoorState} as an existentially quantified data
type, and the second one reifies one in CPS-style, without the intermediate data
type.

We can actually use these to write \texttt{mkSomeDoor} and \texttt{withDoor} in
a nicer way, without directly pattern matching on our constructors:

\begin{Shaded}
\begin{Highlighting}[]
\CommentTok{-- source: https://github.com/mstksg/inCode/tree/master/code-samples/singletons/Door2.hs#L74-L79}

\OtherTok{mkSomeDoor ::} \DataTypeTok{DoorState} \OtherTok{->} \DataTypeTok{String} \OtherTok{->} \DataTypeTok{SomeDoor}
\NormalTok{mkSomeDoor ds }\FunctionTok{=} \KeywordTok{case}\NormalTok{ toSing ds }\KeywordTok{of}
    \DataTypeTok{SomeSing}\NormalTok{ s }\OtherTok{->}\NormalTok{ fromDoor s }\FunctionTok{.}\NormalTok{ mkDoor s}

\OtherTok{withDoor ::} \DataTypeTok{DoorState} \OtherTok{->} \DataTypeTok{String} \OtherTok{->}\NormalTok{ (forall s}\FunctionTok{.} \DataTypeTok{Sing}\NormalTok{ s }\OtherTok{->} \DataTypeTok{Door}\NormalTok{ s }\OtherTok{->}\NormalTok{ r) }\OtherTok{->}\NormalTok{ r}
\NormalTok{withDoor ds m f }\FunctionTok{=}\NormalTok{ withSomeSing ds }\FunctionTok{$}\NormalTok{ \textbackslash{}s }\OtherTok{->}\NormalTok{ f s (mkDoor s m)}
\end{Highlighting}
\end{Shaded}

\hypertarget{zooming-out}{%
\section{Zooming Out}\label{zooming-out}}

Alright! We've spent two blog posts going over a lot of different things in the
context of our humble \texttt{Door\ s} type. Let's zoom out and take a
large-scale look at how \emph{singletons} (the design pattern, and the library)
helps us in general.

\hypertarget{sing}{%
\subsection{Sing}\label{sing}}

The crux of everything is the \texttt{Sing\ ::\ Type\ -\textgreater{}\ Type}
indexed type. If you see a value of type \texttt{Sing\ s}, you should really
just think ``a runtime witness for \texttt{s}''. If you see:

\begin{Shaded}
\begin{Highlighting}[]
\OtherTok{lockAnyDoor ::} \DataTypeTok{Sing}\NormalTok{ s }\OtherTok{->} \DataTypeTok{Door}\NormalTok{ s }\OtherTok{->} \DataTypeTok{Door}\NormalTok{ '}\DataTypeTok{Locked}
\DataTypeTok{MkSomeDoor}\OtherTok{  ::} \DataTypeTok{Sing}\NormalTok{ s }\OtherTok{->} \DataTypeTok{Door}\NormalTok{ s }\OtherTok{->} \DataTypeTok{SomeDoor}
\end{Highlighting}
\end{Shaded}

You should read it as (in pseudo-Haskell)

\begin{Shaded}
\begin{Highlighting}[]
\OtherTok{lockAnyDoor ::}\NormalTok{ \{ s \} }\OtherTok{->} \DataTypeTok{Door}\NormalTok{ s }\OtherTok{->} \DataTypeTok{Door}\NormalTok{ '}\DataTypeTok{Locked}
\DataTypeTok{MkSomeDoor}\OtherTok{  ::}\NormalTok{ \{ s \} }\OtherTok{->} \DataTypeTok{Door}\NormalTok{ s }\OtherTok{->} \DataTypeTok{SomeDoor}
\end{Highlighting}
\end{Shaded}

This is seen clearly if we look at the partially applied type signatures:

\begin{Shaded}
\begin{Highlighting}[]
\NormalTok{lockAnyDoor }\DataTypeTok{SOpened}\OtherTok{ ::} \DataTypeTok{Door}\NormalTok{ '}\DataTypeTok{Opened} \OtherTok{->} \DataTypeTok{Door}\NormalTok{ '}\DataTypeTok{Locked}
\DataTypeTok{MkSomeDoor}  \DataTypeTok{SLocked}\OtherTok{ ::} \DataTypeTok{Door}\NormalTok{ '}\DataTypeTok{Locked} \OtherTok{->} \DataTypeTok{SomeDoor}
\end{Highlighting}
\end{Shaded}

If you squint, this kinda looks like:

\begin{Shaded}
\begin{Highlighting}[]
\NormalTok{lockAnyDoor '}\DataTypeTok{Opened}\OtherTok{ ::} \DataTypeTok{Door}\NormalTok{ '}\DataTypeTok{Opened} \OtherTok{->} \DataTypeTok{Door}\NormalTok{ '}\DataTypeTok{Locked}
\DataTypeTok{MkSomeDoor}\NormalTok{  '}\DataTypeTok{Locked}\OtherTok{ ::} \DataTypeTok{Door}\NormalTok{ '}\DataTypeTok{Locked} \OtherTok{->} \DataTypeTok{SomeDoor}
\end{Highlighting}
\end{Shaded}

And indeed, when we get real dependent types in Haskell, we will really be
directly passing types (that act as their own runtime values) instead of
singletons.

It is important to remember that \texttt{Sing} is poly-kinded, so we can have
\texttt{Sing\ \textquotesingle{}Opened}, but also
\texttt{Sing\ \textquotesingle{}True}, \texttt{Sing\ 5}, and
\texttt{Sing\ \textquotesingle{}{[}\textquotesingle{}Just\ 3,\ \textquotesingle{}Nothing,\ \textquotesingle{}Just\ 0{]}}
as well. This is the real benefit of using the \emph{singletons} library instead
of writing our own singletons -- we get to work uniformly with singletons of all
kinds.

\hypertarget{singi}{%
\subsubsection{SingI}\label{singi}}

\texttt{SingI} is a bit of typeclass trickery that lets us implicitly pass
\texttt{Sing}s to functions:

\begin{Shaded}
\begin{Highlighting}[]
\KeywordTok{class} \DataTypeTok{SingI}\NormalTok{ s }\KeywordTok{where}
\OtherTok{    sing ::} \DataTypeTok{Sing}\NormalTok{ s}
\end{Highlighting}
\end{Shaded}

If you see:

\begin{Shaded}
\begin{Highlighting}[]
\OtherTok{lockAnyDoor ::} \DataTypeTok{Sing}\NormalTok{  s }\OtherTok{->} \DataTypeTok{Door}\NormalTok{ s }\OtherTok{->} \DataTypeTok{Door}\NormalTok{ '}\DataTypeTok{Locked}
\OtherTok{fromDoor    ::} \DataTypeTok{Sing}\NormalTok{  s }\OtherTok{->} \DataTypeTok{Door}\NormalTok{ s }\OtherTok{->} \DataTypeTok{SomeDoor}
\end{Highlighting}
\end{Shaded}

These are \emph{identical in power} to

\begin{Shaded}
\begin{Highlighting}[]
\OtherTok{lockAnyDoor ::} \DataTypeTok{SingI}\NormalTok{ s }\OtherTok{=>} \DataTypeTok{Door}\NormalTok{ s }\OtherTok{->} \DataTypeTok{Door}\NormalTok{ '}\DataTypeTok{Locked}
\OtherTok{fromDoor    ::} \DataTypeTok{SingI}\NormalTok{ s }\OtherTok{=>} \DataTypeTok{Door}\NormalTok{ s }\OtherTok{->} \DataTypeTok{SomeDoor}
\end{Highlighting}
\end{Shaded}

Either way, you're passing in the ability to get a runtime witness on \texttt{s}
-- just in one way, it is asked for as an explicit argument, and the second way,
it is passed in using a typeclass.

We can \emph{convert} from \texttt{SingI\ s\ -\textgreater{}} style to
\texttt{SingI\ s\ =\textgreater{}} style using \texttt{sing}:

\begin{Shaded}
\begin{Highlighting}[]
\CommentTok{-- source: https://github.com/mstksg/inCode/tree/master/code-samples/singletons/Door2.hs#L50-L63}

\OtherTok{lockAnyDoor_ ::} \DataTypeTok{SingI}\NormalTok{ s }\OtherTok{=>} \DataTypeTok{Door}\NormalTok{ s }\OtherTok{->} \DataTypeTok{Door}\NormalTok{ '}\DataTypeTok{Locked}
\NormalTok{lockAnyDoor_ }\FunctionTok{=}\NormalTok{ lockAnyDoor sing}

\OtherTok{fromDoor_ ::} \DataTypeTok{SingI}\NormalTok{ s }\OtherTok{=>} \DataTypeTok{Door}\NormalTok{ s }\OtherTok{->} \DataTypeTok{SomeDoor}
\NormalTok{fromDoor_ }\FunctionTok{=}\NormalTok{ fromDoor sing}
\end{Highlighting}
\end{Shaded}

And we can convert from \texttt{SingI\ s\ =\textgreater{}} style to
\texttt{SingI\ s\ -\textgreater{}} style using \texttt{withSingI}:

\begin{Shaded}
\begin{Highlighting}[]
\OtherTok{lockAnyDoor ::} \DataTypeTok{Sing}\NormalTok{ s }\OtherTok{->} \DataTypeTok{Door}\NormalTok{ s }\OtherTok{->} \DataTypeTok{Door}\NormalTok{ '}\DataTypeTok{Locked}
\NormalTok{lockAnyDoor s d }\FunctionTok{=}\NormalTok{ withSingI s (lockAnyDoor_ d)}

\OtherTok{fromDoor ::} \DataTypeTok{Sing}\NormalTok{ s }\OtherTok{->} \DataTypeTok{Door}\NormalTok{ s }\OtherTok{->} \DataTypeTok{SomeDoor}
\NormalTok{fromDoor s d }\FunctionTok{=}\NormalTok{ withSingI s (fromDoor_ d)}
\end{Highlighting}
\end{Shaded}

Again, the same function -- just two different styles of calling them.

\hypertarget{reflection-and-reification}{%
\subsection{Reflection and Reification}\label{reflection-and-reification}}

Reflection is the process of bringing a type-level thing to a value at the term
level (``losing'' the type information in the process) and reification is the
process of bringing a value at the \emph{term level} to the \emph{type level}.

Reflection and reification can be thought of as the gateways between the
untyped/unsafe world and the typed/safe world. Reflection takes you from the
typed world to the untyped world (from \texttt{Sing\ s} to \texttt{DoorState})
and reification takes you from the untyped world to the typed world (from
\texttt{DoorState} to \texttt{Sing\ s}).

One limitation in Haskell is that there is no actual link between the type
\texttt{DoorState} and its \emph{values} with the \emph{kind} \texttt{DoorState}
with its \emph{types}. Sure, the constructors have the same names, but the
language doesn't actually link them together for us.

\hypertarget{singkind}{%
\subsubsection{SingKind}\label{singkind}}

The \emph{singletons} library handles this by using a typeclass with associated
types to implement a generalized reflection and reification process. It gives us
the \texttt{SingKind} ``kindclass'':

\begin{Shaded}
\begin{Highlighting}[]
\KeywordTok{class} \DataTypeTok{SingKind}\NormalTok{ k }\KeywordTok{where}      \CommentTok{-- `k` is a kind!}
    \CommentTok{-- | Associate a kind k with its reflected type}
    \KeywordTok{type} \DataTypeTok{Demote}\NormalTok{ k }\FunctionTok{=}\NormalTok{ (}\OtherTok{r ::} \DataTypeTok{Type}\NormalTok{)}

    \CommentTok{-- | Reflect a singleton to its term-level value}
\OtherTok{    fromSing ::} \DataTypeTok{Sing}\NormalTok{ (}\OtherTok{a ::}\NormalTok{ k) }\OtherTok{->} \DataTypeTok{Demote}\NormalTok{ k}

    \CommentTok{-- | Reify a term-level value to the type level, as an existentially}
    \CommentTok{-- quantified singleton}
\OtherTok{    toSing ::} \DataTypeTok{Demote}\NormalTok{ k }\OtherTok{->} \DataTypeTok{SomeSing}\NormalTok{ k}
\end{Highlighting}
\end{Shaded}

Instances of \texttt{SingKind} are (promoted) \emph{kinds} like
\texttt{Bool}-the-kind, \texttt{DoorState}-the-kind, etc., and \texttt{Demote}
is an associated type/type family that associates each instance with the
\emph{type} it is promoted from. (Note -- writing these type signatures requires
the \texttt{-XTypeInType} extension, which lets us treat kinds as types)

For example, remember how
\texttt{data\ DoorState\ =\ Opened\ \textbar{}\ Closed\ \textbar{}\ Locked}
created the \emph{type} \texttt{DoorState} (with value constructors
\texttt{Opened}, \texttt{Closed}, and \texttt{Locked}), and also the \emph{kind}
\texttt{DoorState} (with \emph{type} constructors
\texttt{\textquotesingle{}Opened}, \texttt{\textquotesingle{}Closed}, and
\texttt{\textquotesingle{}Locked}). Our \emph{kind} \texttt{DoorState} would be
the instance of \texttt{SingKind}, and \texttt{Demote\ DoorState} would be the
\emph{type} \texttt{DoorState}.

The reason we need an explicit \texttt{Demote} associated type is, again, that
GHC doesn't actually link the type and its promoted kind. \texttt{Demote} lets
us explicitly specify what type a \texttt{Kind} should expect its term-level
reflected values to be. (And, like most things in this post, \texttt{Demote}
will hopefully one day become obsolete, along with the rest of
\texttt{SingKind})

\hypertarget{examples}{%
\subsubsection{Examples}\label{examples}}

To illustrate explicitly, here is the automatically generated instance of
\texttt{SingKind} for the \texttt{DoorState} \emph{kind}:

\begin{Shaded}
\begin{Highlighting}[]
\KeywordTok{instance} \DataTypeTok{SingKind} \DataTypeTok{DoorState} \KeywordTok{where}       \CommentTok{-- the *kind* DoorState}
    \KeywordTok{type} \DataTypeTok{Demote} \DataTypeTok{DoorState} \FunctionTok{=} \DataTypeTok{DoorState}   \CommentTok{-- the *type* DoorState}

\NormalTok{    fromSing}
\OtherTok{        ::} \DataTypeTok{Sing}\NormalTok{ (}\OtherTok{s ::} \DataTypeTok{DoorState}\NormalTok{)        }\CommentTok{-- the *kind* DoorState}
        \OtherTok{->} \DataTypeTok{DoorState}                    \CommentTok{-- the *type* DoorState}
\NormalTok{    fromSing }\FunctionTok{=}\NormalTok{ \textbackslash{}}\KeywordTok{case}
        \DataTypeTok{SOpened} \OtherTok{->} \DataTypeTok{Opened}
        \DataTypeTok{SClosed} \OtherTok{->} \DataTypeTok{Closed}
        \DataTypeTok{SLocked} \OtherTok{->} \DataTypeTok{Locked}

\NormalTok{    toSing}
\OtherTok{        ::} \DataTypeTok{DoorState}                    \CommentTok{-- the *type* DoorState}
        \OtherTok{->} \DataTypeTok{SomeSing} \DataTypeTok{DoorState}           \CommentTok{-- the *kind* DoorState}
\NormalTok{    toSing }\FunctionTok{=}\NormalTok{ \textbackslash{}}\KeywordTok{case}
        \DataTypeTok{Opened} \OtherTok{->} \DataTypeTok{SomeSing} \DataTypeTok{SOpened}
        \DataTypeTok{Closed} \OtherTok{->} \DataTypeTok{SomeSing} \DataTypeTok{SClosed}
        \DataTypeTok{Locked} \OtherTok{->} \DataTypeTok{SomeSing} \DataTypeTok{SLocked}
\end{Highlighting}
\end{Shaded}

If you are unfamiliar with how associated types work,
\texttt{type\ Demote\ DoorState\ =\ DoorState} means that wherever we see
\texttt{Demote\ DoorState} (with \texttt{DoorState} the \emph{kind}), we replace
it with \texttt{DoorState} (the \emph{type}). That's why the type of our
reflection function
\texttt{fromSing\ ::\ Sing\ s\ -\textgreater{}\ Demote\ DoorState} can be
simplified to \texttt{fromSing\ ::\ Sing\ s\ -\textgreater{}\ DoorState}.

Let's take a look at the instance for \texttt{Bool}, to compare:

\begin{Shaded}
\begin{Highlighting}[]
\CommentTok{-- Bool singletons have two constructors:}
\DataTypeTok{SFalse}\OtherTok{ ::} \DataTypeTok{Sing}\NormalTok{ '}\DataTypeTok{False}
\DataTypeTok{STrue}\OtherTok{  ::} \DataTypeTok{Sing}\NormalTok{ '}\DataTypeTok{True}

\KeywordTok{instance} \DataTypeTok{SingKind} \DataTypeTok{Bool} \KeywordTok{where}    \CommentTok{-- the *kind* Bool}
    \KeywordTok{type} \DataTypeTok{Demote} \DataTypeTok{Bool} \FunctionTok{=} \DataTypeTok{Bool}     \CommentTok{-- the *type* Bool}

\NormalTok{    fromSing}
\OtherTok{        ::} \DataTypeTok{Sing}\NormalTok{ (}\OtherTok{b ::} \DataTypeTok{Bool}\NormalTok{)        }\CommentTok{-- the *kind* Bool}
        \OtherTok{->} \DataTypeTok{Bool}                    \CommentTok{-- the *type* Bool}
\NormalTok{    fromSing }\FunctionTok{=}\NormalTok{ \textbackslash{}}\KeywordTok{case}
        \DataTypeTok{SFalse} \OtherTok{->} \DataTypeTok{False}
        \DataTypeTok{STrue}  \OtherTok{->} \DataTypeTok{True}

\NormalTok{    toSing}
\OtherTok{        ::} \DataTypeTok{Bool}                    \CommentTok{-- the *type* Bool}
        \OtherTok{->} \DataTypeTok{SomeSing} \DataTypeTok{Bool}           \CommentTok{-- the *kind* Bool}
\NormalTok{    toSing }\FunctionTok{=}\NormalTok{ \textbackslash{}}\KeywordTok{case}
        \DataTypeTok{False} \OtherTok{->} \DataTypeTok{SomeSing} \DataTypeTok{SFalse}
        \DataTypeTok{True}  \OtherTok{->} \DataTypeTok{SomeSing} \DataTypeTok{STrue}
\end{Highlighting}
\end{Shaded}

And a more sophisticated example, let's look at the instance for \texttt{Maybe}:

\begin{Shaded}
\begin{Highlighting}[]
\CommentTok{-- Maybe singletons have two constructors:}
\DataTypeTok{SNothing}\OtherTok{ ::} \DataTypeTok{Sing}\NormalTok{ '}\DataTypeTok{Nothing}
\DataTypeTok{SJust}\OtherTok{    ::} \DataTypeTok{Sing}\NormalTok{ x }\OtherTok{->} \DataTypeTok{Sing}\NormalTok{ ('}\DataTypeTok{Just}\NormalTok{ x)}

\KeywordTok{instance} \DataTypeTok{SingKind}\NormalTok{ k }\OtherTok{=>} \DataTypeTok{SingKind}\NormalTok{ (}\DataTypeTok{Maybe}\NormalTok{ k) }\KeywordTok{where}     \CommentTok{-- the *kind* Maybe}
    \KeywordTok{type} \DataTypeTok{Demote}\NormalTok{ (}\DataTypeTok{Maybe}\NormalTok{ k) }\FunctionTok{=} \DataTypeTok{Maybe}\NormalTok{ (}\DataTypeTok{Demote}\NormalTok{ k)        }\CommentTok{-- the *type* Maybe}

\NormalTok{    fromSing}
\OtherTok{        ::} \DataTypeTok{Sing}\NormalTok{ (}\OtherTok{m ::} \DataTypeTok{Maybe}\NormalTok{ k)        }\CommentTok{-- the *kind* Maybe}
        \OtherTok{->} \DataTypeTok{Maybe}\NormalTok{ (}\DataTypeTok{Demote}\NormalTok{ k)           }\CommentTok{-- the *type* Maybe}
\NormalTok{    fromSing }\FunctionTok{=}\NormalTok{ \textbackslash{}}\KeywordTok{case}
        \DataTypeTok{SNothing} \OtherTok{->} \DataTypeTok{Nothing}
        \DataTypeTok{SJust}\NormalTok{ sx }\OtherTok{->} \DataTypeTok{Just}\NormalTok{ (fromSing sx)}

\NormalTok{    toSing}
\OtherTok{        ::} \DataTypeTok{Maybe}\NormalTok{ (}\DataTypeTok{Demote}\NormalTok{ k)             }\CommentTok{-- the *type* Maybe}
        \OtherTok{->} \DataTypeTok{SomeSing}\NormalTok{ (}\DataTypeTok{Maybe}\NormalTok{ k)           }\CommentTok{-- the *kind* Maybe}
\NormalTok{    toSing }\FunctionTok{=}\NormalTok{ \textbackslash{}}\KeywordTok{case}
        \DataTypeTok{Nothing} \OtherTok{->} \DataTypeTok{SomeSing} \DataTypeTok{SNothing}
        \DataTypeTok{Just}\NormalTok{ x  }\OtherTok{->} \KeywordTok{case}\NormalTok{ toSing x }\KeywordTok{of}
          \DataTypeTok{SomeSing}\NormalTok{ sx }\OtherTok{->} \DataTypeTok{SomeSing}\NormalTok{ (}\DataTypeTok{SJust}\NormalTok{ sx)}
\end{Highlighting}
\end{Shaded}

This definition, I think, is a real testament to the usefulness of having all of
our singletons be unified under the same system. Because of how
\texttt{SingKind} works, \texttt{Demote\ (Maybe\ DoorState)} is evaluated to
\texttt{Maybe\ (Demote\ DoorState)}, which is simplified to
\texttt{Maybe\ DoorState}. This means that if we have a way to reify
\texttt{DoorState} values, we also have a way to reify \texttt{Maybe\ DoorState}
values! And, if we have a way to reflect \texttt{DoorState} singletons, we also
have a way to reflect \texttt{Maybe\ DoorState} singletons!

\hypertarget{somesing}{%
\subsubsection{SomeSing}\label{somesing}}

Throughout all of this, we utilize \texttt{SomeSing} as a generic poly-kinded
existential wrapper:

\begin{Shaded}
\begin{Highlighting}[]
\KeywordTok{data} \DataTypeTok{SomeSing}\OtherTok{ ::} \DataTypeTok{Type} \OtherTok{->} \DataTypeTok{Type} \KeywordTok{where}
    \DataTypeTok{SomeSing}\OtherTok{ ::} \DataTypeTok{Sing}\NormalTok{ (}\OtherTok{x ::}\NormalTok{ k) }\OtherTok{->} \DataTypeTok{SomeSing}\NormalTok{ k}
\end{Highlighting}
\end{Shaded}

Basically, this says that \texttt{SomeSing\ k} contains a \texttt{Sing\ x},
where \texttt{x} is of kind \texttt{k}. This is why we had, earlier:

\begin{Shaded}
\begin{Highlighting}[]
\DataTypeTok{SomeSing}\OtherTok{ ::} \DataTypeTok{Sing}\NormalTok{ (}\OtherTok{s ::} \DataTypeTok{DoorState}\NormalTok{) }\OtherTok{->} \DataTypeTok{SomeSing} \DataTypeTok{DoorState}
\DataTypeTok{SomeSing}\OtherTok{ ::} \DataTypeTok{Sing}\NormalTok{ (}\OtherTok{s ::} \DataTypeTok{Bool}\NormalTok{)      }\OtherTok{->} \DataTypeTok{SomeSing} \DataTypeTok{Bool}
\DataTypeTok{SomeSing}\OtherTok{ ::} \DataTypeTok{Sing}\NormalTok{ (}\OtherTok{s ::} \DataTypeTok{Maybe}\NormalTok{ k)   }\OtherTok{->} \DataTypeTok{SomeSing}\NormalTok{ (}\DataTypeTok{Maybe}\NormalTok{ k)}
\end{Highlighting}
\end{Shaded}

If we use \texttt{SomeSing} with, say, \texttt{SClosed}, we get
\texttt{SomeSing\ ::\ Sing\ \textquotesingle{}Closed\ -\textgreater{}\ SomeSing\ DoorState}.
\texttt{SomeSing} is an indexed type that tells us the \emph{kind} of the type
variable we existentially quantifying over. The value \texttt{SomeSing\ STrue}
would have the type \texttt{SomeSing\ Bool}. The value
\texttt{SomeSing\ (SJust\ SClosed)} would have the type
\texttt{SomeSing\ (Maybe\ DoorState)}.

\hypertarget{looking-forward}{%
\section{Looking Forward}\label{looking-forward}}

Between these first two parts, we explored a specific use case that would
benefit from dependent types (simple phantom types for state transitions) and
explored how the \emph{singletons} and design pattern help us implement the
functionality necessary to make things useful, and snuck in some concepts from
dependently typed programming as well. We then took a step back to explore the
\emph{singletons} library in a more ``universal'' way, and saw how it is
generalized to many different types.

The code is available
\href{https://github.com/mstksg/inCode/tree/master/code-samples/singletons/Door2.hs}{here}
for you to play around with yourself!

Now that the basics are out of the way, in Part 3 we'll jump deep into
type-level programming and being able to lift our term-level functions on values
up to become type-level functions, and how to use this to express complex
relationships and enhance our code!

Let me know in the comments if you have any questions! I'm also usually idling
on the freenode \texttt{\#haskell} channel, as well, as \emph{jle`}.

And, again, I definitely recommend checking out the
\href{https://cs.brynmawr.edu/~rae/papers/2012/singletons/paper.pdf}{original
singletons paper} for a really nice technical overview of all of these
techniques from the source itself.

\hypertarget{exercises}{%
\subsection{Exercises}\label{exercises}}

Check out the
\href{https://github.com/mstksg/inCode/tree/master/code-samples/singletons/Door2.hs}{sample
code} for solutions!

\begin{enumerate}
\def\labelenumi{\arabic{enumi}.}
\item
  Let's revisit our original redundant \texttt{SomeDoor}, compared to our final
  \texttt{SomeDoor}:

\begin{Shaded}
\begin{Highlighting}[]
\CommentTok{-- source: https://github.com/mstksg/inCode/tree/master/code-samples/singletons/Door2.hs#L56-L87}

\KeywordTok{data} \DataTypeTok{OldSomeDoor}\OtherTok{ ::} \DataTypeTok{Type} \KeywordTok{where}
    \DataTypeTok{OldMkSomeDoor}\OtherTok{ ::} \DataTypeTok{DoorState} \OtherTok{->} \DataTypeTok{String} \OtherTok{->} \DataTypeTok{OldSomeDoor}

\KeywordTok{data} \DataTypeTok{SomeDoor}\OtherTok{ ::} \DataTypeTok{Type} \KeywordTok{where}
    \DataTypeTok{MkSomeDoor}\OtherTok{ ::} \DataTypeTok{Sing}\NormalTok{ s }\OtherTok{->} \DataTypeTok{Door}\NormalTok{ s }\OtherTok{->} \DataTypeTok{SomeDoor}
\end{Highlighting}
\end{Shaded}

  To help convince yourself that the two are equal, write functions converting
  between the two:

\begin{Shaded}
\begin{Highlighting}[]
\CommentTok{-- source: https://github.com/mstksg/inCode/tree/master/code-samples/singletons/Door2.hs#L89-L92}

\OtherTok{toOld ::} \DataTypeTok{SomeDoor} \OtherTok{->} \DataTypeTok{OldSomeDoor}

\OtherTok{fromOld ::} \DataTypeTok{OldSomeDoor} \OtherTok{->} \DataTypeTok{SomeDoor}
\end{Highlighting}
\end{Shaded}

  \textbf{Avoid directly pattern matching on the singletons or constructors}.
  Instead, use \emph{singletons} library tools like \texttt{toSing},
  \texttt{withSomeSing}, \texttt{fromSing}, etc.
\item
  Previously, we had an \texttt{unlockDoor} function that took an \texttt{Int}
  (the ``password'') with a \texttt{Door\ \textquotesingle{}Locked} and returned
  a \texttt{Maybe\ (Door\ \textquotesingle{}Closed)}. It returns a
  \texttt{Door\ \textquotesingle{}Closed} (unlocked door) in \texttt{Just} if an
  odd number was given, and \texttt{Nothing} otherwise (a failed unlock)

  Use this to implement a that would return a \texttt{SomeDoor}. Re-use the
  ``password'' logic from the original \texttt{unlockDoor}. If the door is
  successfully unlocked (with a \texttt{Just}), return the unlocked door in a
  \texttt{SomeDoor}. Otherwise, \emph{return the original locked door} (in a
  \texttt{SomeDoor}).

\begin{Shaded}
\begin{Highlighting}[]
\CommentTok{-- source: https://github.com/mstksg/inCode/tree/master/code-samples/singletons/Door2.hs#L95-L100}

\OtherTok{unlockDoor ::} \DataTypeTok{Int} \OtherTok{->} \DataTypeTok{Door}\NormalTok{ '}\DataTypeTok{Locked} \OtherTok{->} \DataTypeTok{Maybe}\NormalTok{ (}\DataTypeTok{Door}\NormalTok{ '}\DataTypeTok{Closed}\NormalTok{)}
\NormalTok{unlockDoor n (}\DataTypeTok{UnsafeMkDoor}\NormalTok{ m)}
    \FunctionTok{|}\NormalTok{ n }\OtherTok{`mod`} \DecValTok{2} \FunctionTok{==} \DecValTok{1} \FunctionTok{=} \DataTypeTok{Just}\NormalTok{ (}\DataTypeTok{UnsafeMkDoor}\NormalTok{ m)}
    \FunctionTok{|}\NormalTok{ otherwise      }\FunctionTok{=} \DataTypeTok{Nothing}

\OtherTok{unlockSomeDoor ::} \DataTypeTok{Int} \OtherTok{->} \DataTypeTok{Door}\NormalTok{ '}\DataTypeTok{Locked} \OtherTok{->} \DataTypeTok{SomeDoor}
\NormalTok{unlockSomeDoor }\FunctionTok{=} \FunctionTok{???}
\end{Highlighting}
\end{Shaded}
\item
  Implement \texttt{openAnyDoor\textquotesingle{}} in the same style, with
  respect to \texttt{openAnyDoor}:

\begin{Shaded}
\begin{Highlighting}[]
\CommentTok{-- source: https://github.com/mstksg/inCode/tree/master/code-samples/singletons/Door2.hs#L105-L114}

\OtherTok{openAnyDoor ::} \DataTypeTok{SingI}\NormalTok{ s }\OtherTok{=>} \DataTypeTok{Int} \OtherTok{->} \DataTypeTok{Door}\NormalTok{ s }\OtherTok{->} \DataTypeTok{Maybe}\NormalTok{ (}\DataTypeTok{Door}\NormalTok{ '}\DataTypeTok{Opened}\NormalTok{)}
\NormalTok{openAnyDoor n }\FunctionTok{=}\NormalTok{ openAnyDoor_ sing}
  \KeywordTok{where}
\OtherTok{    openAnyDoor_ ::} \DataTypeTok{Sing}\NormalTok{ s }\OtherTok{->} \DataTypeTok{Door}\NormalTok{ s }\OtherTok{->} \DataTypeTok{Maybe}\NormalTok{ (}\DataTypeTok{Door}\NormalTok{ '}\DataTypeTok{Opened}\NormalTok{)}
\NormalTok{    openAnyDoor_ }\FunctionTok{=}\NormalTok{ \textbackslash{}}\KeywordTok{case}
      \DataTypeTok{SOpened} \OtherTok{->} \DataTypeTok{Just}
      \DataTypeTok{SClosed} \OtherTok{->} \DataTypeTok{Just} \FunctionTok{.}\NormalTok{ openDoor}
      \DataTypeTok{SLocked} \OtherTok{->}\NormalTok{ fmap openDoor }\FunctionTok{.}\NormalTok{ unlockDoor n}

\OtherTok{openAnySomeDoor ::} \DataTypeTok{Int} \OtherTok{->} \DataTypeTok{SomeDoor} \OtherTok{->} \DataTypeTok{SomeDoor}
\NormalTok{openAnySomeDoor }\FunctionTok{=} \FunctionTok{???}
\end{Highlighting}
\end{Shaded}

  Remember to re-use \texttt{openAnyDoor}.
\item
  Write the \texttt{SingKind} instance for the promoted kind of a custom list
  type:

\begin{Shaded}
\begin{Highlighting}[]
\CommentTok{-- source: https://github.com/mstksg/inCode/tree/master/code-samples/singletons/Door2.hs#L120-L126}

\KeywordTok{data} \DataTypeTok{List}\NormalTok{ a }\FunctionTok{=} \DataTypeTok{Nil} \FunctionTok{|} \DataTypeTok{Cons}\NormalTok{ a (}\DataTypeTok{List}\NormalTok{ a)}

\KeywordTok{data} \KeywordTok{instance} \DataTypeTok{Sing}\NormalTok{ (}\OtherTok{x ::} \DataTypeTok{List}\NormalTok{ k) }\KeywordTok{where}
    \DataTypeTok{SNil}\OtherTok{  ::} \DataTypeTok{Sing}\NormalTok{ '}\DataTypeTok{Nil}
    \DataTypeTok{SCons}\OtherTok{ ::} \DataTypeTok{Sing}\NormalTok{ x }\OtherTok{->} \DataTypeTok{Sing}\NormalTok{ xs }\OtherTok{->} \DataTypeTok{Sing}\NormalTok{ ('}\DataTypeTok{Cons}\NormalTok{ x xs)}

\KeywordTok{instance} \DataTypeTok{SingKind}\NormalTok{ k }\OtherTok{=>} \DataTypeTok{SingKind}\NormalTok{ (}\DataTypeTok{List}\NormalTok{ k) }\KeywordTok{where}
    \KeywordTok{type} \DataTypeTok{Demote}\NormalTok{ (}\DataTypeTok{List}\NormalTok{ k) }\FunctionTok{=} \FunctionTok{???}

\OtherTok{    fromSing ::} \DataTypeTok{Sing}\NormalTok{ (}\OtherTok{xs ::} \DataTypeTok{List}\NormalTok{ k) }\OtherTok{->} \DataTypeTok{List}\NormalTok{ (}\DataTypeTok{Demote}\NormalTok{ k)}
\NormalTok{    fromSing }\FunctionTok{=} \FunctionTok{???}

\OtherTok{    toSing ::} \DataTypeTok{List}\NormalTok{ (}\DataTypeTok{Demote}\NormalTok{ k) }\OtherTok{->} \DataTypeTok{SomeSing}\NormalTok{ (}\DataTypeTok{List}\NormalTok{ k)}
\NormalTok{    toSing }\FunctionTok{=} \FunctionTok{???}
\end{Highlighting}
\end{Shaded}

  The singletons for \texttt{List} are:

\begin{Shaded}
\begin{Highlighting}[]
\DataTypeTok{SNil}\OtherTok{  ::} \DataTypeTok{Sing}\NormalTok{ '}\DataTypeTok{Nil}
\DataTypeTok{SCons}\OtherTok{ ::} \DataTypeTok{Sing}\NormalTok{ x }\OtherTok{->} \DataTypeTok{Sing}\NormalTok{ xs }\OtherTok{->} \DataTypeTok{Sing}\NormalTok{ ('}\DataTypeTok{Cons}\NormalTok{ x xs)}
\end{Highlighting}
\end{Shaded}

  Note that the built-in singletons for the list type also uses these same
  constructor names, for \texttt{{[}{]}} and \texttt{:}.
\end{enumerate}

Hi, thanks for reading! You can reach me via email at
\href{mailto:justin@jle.im}{\nolinkurl{justin@jle.im}}, or at twitter at
\href{https://twitter.com/mstk}{@mstk}! This post and all others are published
under the \href{https://creativecommons.org/licenses/by-nc-nd/3.0/}{CC-BY-NC-ND
3.0} license. Corrections and edits via pull request are welcome and encouraged
at \href{https://github.com/mstksg/inCode}{the source repository}.

If you feel inclined, or this post was particularly helpful for you, why not
consider \href{https://www.patreon.com/justinle/overview}{supporting me on
Patreon}, or a \href{bitcoin:3D7rmAYgbDnp4gp4rf22THsGt74fNucPDU}{BTC donation}?
:)

\end{document}
