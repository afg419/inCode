\documentclass[]{article}
\usepackage{lmodern}
\usepackage{amssymb,amsmath}
\usepackage{ifxetex,ifluatex}
\usepackage{fixltx2e} % provides \textsubscript
\ifnum 0\ifxetex 1\fi\ifluatex 1\fi=0 % if pdftex
  \usepackage[T1]{fontenc}
  \usepackage[utf8]{inputenc}
\else % if luatex or xelatex
  \ifxetex
    \usepackage{mathspec}
    \usepackage{xltxtra,xunicode}
  \else
    \usepackage{fontspec}
  \fi
  \defaultfontfeatures{Mapping=tex-text,Scale=MatchLowercase}
  \newcommand{\euro}{€}
\fi
% use upquote if available, for straight quotes in verbatim environments
\IfFileExists{upquote.sty}{\usepackage{upquote}}{}
% use microtype if available
\IfFileExists{microtype.sty}{\usepackage{microtype}}{}
\usepackage[margin=1in]{geometry}
\usepackage{color}
\usepackage{fancyvrb}
\newcommand{\VerbBar}{|}
\newcommand{\VERB}{\Verb[commandchars=\\\{\}]}
\DefineVerbatimEnvironment{Highlighting}{Verbatim}{commandchars=\\\{\}}
% Add ',fontsize=\small' for more characters per line
\newenvironment{Shaded}{}{}
\newcommand{\AlertTok}[1]{\textcolor[rgb]{1.00,0.00,0.00}{\textbf{#1}}}
\newcommand{\AnnotationTok}[1]{\textcolor[rgb]{0.38,0.63,0.69}{\textbf{\textit{#1}}}}
\newcommand{\AttributeTok}[1]{\textcolor[rgb]{0.49,0.56,0.16}{#1}}
\newcommand{\BaseNTok}[1]{\textcolor[rgb]{0.25,0.63,0.44}{#1}}
\newcommand{\BuiltInTok}[1]{#1}
\newcommand{\CharTok}[1]{\textcolor[rgb]{0.25,0.44,0.63}{#1}}
\newcommand{\CommentTok}[1]{\textcolor[rgb]{0.38,0.63,0.69}{\textit{#1}}}
\newcommand{\CommentVarTok}[1]{\textcolor[rgb]{0.38,0.63,0.69}{\textbf{\textit{#1}}}}
\newcommand{\ConstantTok}[1]{\textcolor[rgb]{0.53,0.00,0.00}{#1}}
\newcommand{\ControlFlowTok}[1]{\textcolor[rgb]{0.00,0.44,0.13}{\textbf{#1}}}
\newcommand{\DataTypeTok}[1]{\textcolor[rgb]{0.56,0.13,0.00}{#1}}
\newcommand{\DecValTok}[1]{\textcolor[rgb]{0.25,0.63,0.44}{#1}}
\newcommand{\DocumentationTok}[1]{\textcolor[rgb]{0.73,0.13,0.13}{\textit{#1}}}
\newcommand{\ErrorTok}[1]{\textcolor[rgb]{1.00,0.00,0.00}{\textbf{#1}}}
\newcommand{\ExtensionTok}[1]{#1}
\newcommand{\FloatTok}[1]{\textcolor[rgb]{0.25,0.63,0.44}{#1}}
\newcommand{\FunctionTok}[1]{\textcolor[rgb]{0.02,0.16,0.49}{#1}}
\newcommand{\ImportTok}[1]{#1}
\newcommand{\InformationTok}[1]{\textcolor[rgb]{0.38,0.63,0.69}{\textbf{\textit{#1}}}}
\newcommand{\KeywordTok}[1]{\textcolor[rgb]{0.00,0.44,0.13}{\textbf{#1}}}
\newcommand{\NormalTok}[1]{#1}
\newcommand{\OperatorTok}[1]{\textcolor[rgb]{0.40,0.40,0.40}{#1}}
\newcommand{\OtherTok}[1]{\textcolor[rgb]{0.00,0.44,0.13}{#1}}
\newcommand{\PreprocessorTok}[1]{\textcolor[rgb]{0.74,0.48,0.00}{#1}}
\newcommand{\RegionMarkerTok}[1]{#1}
\newcommand{\SpecialCharTok}[1]{\textcolor[rgb]{0.25,0.44,0.63}{#1}}
\newcommand{\SpecialStringTok}[1]{\textcolor[rgb]{0.73,0.40,0.53}{#1}}
\newcommand{\StringTok}[1]{\textcolor[rgb]{0.25,0.44,0.63}{#1}}
\newcommand{\VariableTok}[1]{\textcolor[rgb]{0.10,0.09,0.49}{#1}}
\newcommand{\VerbatimStringTok}[1]{\textcolor[rgb]{0.25,0.44,0.63}{#1}}
\newcommand{\WarningTok}[1]{\textcolor[rgb]{0.38,0.63,0.69}{\textbf{\textit{#1}}}}
\usepackage{graphicx}
\makeatletter
\def\maxwidth{\ifdim\Gin@nat@width>\linewidth\linewidth\else\Gin@nat@width\fi}
\def\maxheight{\ifdim\Gin@nat@height>\textheight\textheight\else\Gin@nat@height\fi}
\makeatother
% Scale images if necessary, so that they will not overflow the page
% margins by default, and it is still possible to overwrite the defaults
% using explicit options in \includegraphics[width, height, ...]{}
\setkeys{Gin}{width=\maxwidth,height=\maxheight,keepaspectratio}
\ifxetex
  \usepackage[setpagesize=false, % page size defined by xetex
              unicode=false, % unicode breaks when used with xetex
              xetex]{hyperref}
\else
  \usepackage[unicode=true]{hyperref}
\fi
\hypersetup{breaklinks=true,
            bookmarks=true,
            pdfauthor={Justin Le},
            pdftitle={Introducing the backprop library},
            colorlinks=true,
            citecolor=blue,
            urlcolor=blue,
            linkcolor=magenta,
            pdfborder={0 0 0}}
\urlstyle{same}  % don't use monospace font for urls
% Make links footnotes instead of hotlinks:
\renewcommand{\href}[2]{#2\footnote{\url{#1}}}
\setlength{\parindent}{0pt}
\setlength{\parskip}{6pt plus 2pt minus 1pt}
\setlength{\emergencystretch}{3em}  % prevent overfull lines
\setcounter{secnumdepth}{0}

\title{Introducing the backprop library}
\author{Justin Le}
\date{February 12, 2018}

\begin{document}
\maketitle

\emph{Originally posted on
\textbf{\href{https://blog.jle.im/entry/introducing-the-backprop-library.html}{in
Code}}.}

\textbf{backprop}: \href{http://hackage.haskell.org/package/backprop}{hackage} /
\href{https://github.com/mstksg/backprop}{github}

I'm excited to announce the first official release of the
\emph{\href{http://hackage.haskell.org/package/backprop}{backprop}} library
(currently at version 0.1.2.0 on hackage)! This has been something I've been
working on for a while (trying to find a good API for \emph{heterogeneous}
automatic differentiation), and I'm happy to finally find something that I feel
good about, with the help of a
\emph{\href{http://hackage.haskell.org/package/lens}{lens}}-based API.

As a quick demonstration, this post will walk through the creation of a simple
neural network implementation, inspired by the
\href{https://www.tensorflow.org/versions/r1.2/get_started/mnist/beginners}{Tensorflow
Tutorial} for beginners, to learn handwritten digit recognition for the MNIST
data set. To help tell the story, we're going to be implementing it
``normally'', using the
\emph{\href{http://hackage.haskell.org/package/hmatrix}{hmatrix}} library API,
and then re-write the same thing using
\emph{\href{http://hackage.haskell.org/package/backprop}{backprop}} and
\emph{\href{http://hackage.haskell.org/package/hmatrix-backprop}{hmatrix-backprop}}
(a drop-in replacement for \emph{hmatrix}).

\hypertarget{the-basics}{%
\section{The Basics}\label{the-basics}}

For this network, we're not going to be doing anything super fancy. Our ``neural
network'' will just be simple series of matrix multiplications, vector
additions, and activation functions. We're going to make a neural network with a
single hidden layer using normal Haskell data types, parameterized by two weight
matrices and two bias vectors.

The purpose of the MNIST challenge is to take a vector of pixel data (28x28, so
784 elements total) and classify it as one of ten digits (0 through 9). To do
this, we're going to be building and training a model that takes in a 784-vector
of pixel data and produces a 10-item
\href{https://en.wikipedia.org/wiki/One-hot}{one-hot} vector of categorical
predictions (which is supposed to be 0 everywhere, except for 1 in the category
we predict the input picture to be in).

\hypertarget{types}{%
\subsection{Types}\label{types}}

For our types, our imports are pretty simple:

\begin{Shaded}
\begin{Highlighting}[]
\CommentTok{-- source: https://github.com/mstksg/inCode/tree/master/code-samples/backprop/intro-normal.hs#L10-L11}

\KeywordTok{import}           \DataTypeTok{Control.Lens} \KeywordTok{hiding}\NormalTok{          ((<.>))}
\KeywordTok{import}           \DataTypeTok{Numeric.LinearAlgebra.Static}
\end{Highlighting}
\end{Shaded}

Our \texttt{Net} type will just be a simple collection of all of the matrices
and vectors we want to optimize:

\begin{Shaded}
\begin{Highlighting}[]
\CommentTok{-- source: https://github.com/mstksg/inCode/tree/master/code-samples/backprop/intro-normal.hs#L18-L24}

\KeywordTok{data} \DataTypeTok{Net} \FunctionTok{=} \DataTypeTok{N}\NormalTok{ \{}\OtherTok{ _weights1 ::} \DataTypeTok{L} \DecValTok{250} \DecValTok{784}
\NormalTok{             ,}\OtherTok{ _bias1    ::} \DataTypeTok{R} \DecValTok{250}
\NormalTok{             ,}\OtherTok{ _weights2 ::} \DataTypeTok{L} \DecValTok{10} \DecValTok{250}
\NormalTok{             ,}\OtherTok{ _bias2    ::} \DataTypeTok{R} \DecValTok{10}
\NormalTok{             \}}
  \KeywordTok{deriving}\NormalTok{ (}\DataTypeTok{Generic}\NormalTok{)}
\NormalTok{makeLenses }\CharTok{''}\DataTypeTok{Net}
\end{Highlighting}
\end{Shaded}

We're using the matrix types from
\href{https://hackage.haskell.org/package/hmatrix/docs/Numeric-LinearAlgebra-Static.html}{\texttt{Numeric.LinearAlgebra.Static}}.
An \texttt{L\ 250\ 784} is a
\includegraphics{https://latex.codecogs.com/png.latex?250\%20\%5Ctimes\%20784}
matrix -- or, as we are using it, a linear transformation
\includegraphics{https://latex.codecogs.com/png.latex?\%5Cmathbb\%7BR\%7D\%5E\%7B784\%7D\%20\%5Crightarrow\%20\%5Cmathbb\%7BR\%7D\%5E\%7B250\%7D}.
An \texttt{R\ 250} is a 250-vector, etc.

Via the \emph{lens} library, four lenses are generated:

\begin{Shaded}
\begin{Highlighting}[]
\OtherTok{weights1 ::} \DataTypeTok{Lens'} \DataTypeTok{Net}\NormalTok{ (}\DataTypeTok{L} \DecValTok{250} \DecValTok{784}\NormalTok{)}
\OtherTok{bias1    ::} \DataTypeTok{Lens'} \DataTypeTok{Net}\NormalTok{ (}\DataTypeTok{R} \DecValTok{250}\NormalTok{)}
\OtherTok{weights2 ::} \DataTypeTok{Lens'} \DataTypeTok{Net}\NormalTok{ (}\DataTypeTok{L} \DecValTok{10}  \DecValTok{250}\NormalTok{)}
\OtherTok{bias2    ::} \DataTypeTok{Lens'} \DataTypeTok{Net}\NormalTok{ (}\DataTypeTok{R} \DecValTok{10}\NormalTok{)}
\end{Highlighting}
\end{Shaded}

These lenses give us ways to access components of our data type:

\begin{Shaded}
\begin{Highlighting}[]
\OtherTok{myNet             ::} \DataTypeTok{Net}
\NormalTok{myNet }\FunctionTok{^.}\OtherTok{ weights1 ::} \DataTypeTok{L} \DecValTok{250} \DecValTok{784}  \CommentTok{-- access the weights1 field in myNet}
\NormalTok{myNet }\FunctionTok{^.}\OtherTok{ bias2    ::} \DataTypeTok{R}  \DecValTok{10}      \CommentTok{-- access the bias2 field in myNet}
\end{Highlighting}
\end{Shaded}

I'm also going to define \texttt{Num} and \texttt{Fractional} instances for our
network, which makes it really easy to write code to ``gradient descend'' our
network (we can just add and scale our networks with each other). To do this,
I'm going to be using
\emph{\href{http://hackage.haskell.org/package/one-liner-instances}{one-liner-instances}}
to make a \texttt{Num} instance automatically using GHC Generics:

\begin{Shaded}
\begin{Highlighting}[]
\CommentTok{-- source: https://github.com/mstksg/inCode/tree/master/code-samples/backprop/intro-normal.hs#L61-L73}

\KeywordTok{instance} \DataTypeTok{Num} \DataTypeTok{Net} \KeywordTok{where}
\NormalTok{    (}\FunctionTok{+}\NormalTok{)         }\FunctionTok{=}\NormalTok{ gPlus}
\NormalTok{    (}\FunctionTok{-}\NormalTok{)         }\FunctionTok{=}\NormalTok{ gMinus}
\NormalTok{    (}\FunctionTok{*}\NormalTok{)         }\FunctionTok{=}\NormalTok{ gTimes}
\NormalTok{    negate      }\FunctionTok{=}\NormalTok{ gNegate}
\NormalTok{    abs         }\FunctionTok{=}\NormalTok{ gAbs}
\NormalTok{    signum      }\FunctionTok{=}\NormalTok{ gSignum}
\NormalTok{    fromInteger }\FunctionTok{=}\NormalTok{ gFromInteger}

\KeywordTok{instance} \DataTypeTok{Fractional} \DataTypeTok{Net} \KeywordTok{where}
\NormalTok{    (}\FunctionTok{/}\NormalTok{)          }\FunctionTok{=}\NormalTok{ gDivide}
\NormalTok{    recip        }\FunctionTok{=}\NormalTok{ gRecip}
\NormalTok{    fromRational }\FunctionTok{=}\NormalTok{ gFromRational}
\end{Highlighting}
\end{Shaded}

\hypertarget{without-backprop}{%
\section{Without Backprop}\label{without-backprop}}

\hypertarget{running}{%
\subsection{Running}\label{running}}

First, let's look at the picture if we just try to compute the error function
for our network directly.

Running our network is pretty textbook:

\begin{Shaded}
\begin{Highlighting}[]
\CommentTok{-- source: https://github.com/mstksg/inCode/tree/master/code-samples/backprop/intro-normal.hs#L42-L49}

\NormalTok{runNet}
\OtherTok{    ::} \DataTypeTok{Net}
    \OtherTok{->} \DataTypeTok{R} \DecValTok{784}
    \OtherTok{->} \DataTypeTok{R} \DecValTok{10}
\NormalTok{runNet n x }\FunctionTok{=}\NormalTok{ z}
  \KeywordTok{where}
\NormalTok{    y }\FunctionTok{=}\NormalTok{ logistic }\FunctionTok{$}\NormalTok{ (n }\FunctionTok{^.}\NormalTok{ weights1) }\FunctionTok{#>}\NormalTok{ x }\FunctionTok{+}\NormalTok{ (n }\FunctionTok{^.}\NormalTok{ bias1)}
\NormalTok{    z }\FunctionTok{=}\NormalTok{ softMax  }\FunctionTok{$}\NormalTok{ (n }\FunctionTok{^.}\NormalTok{ weights2) }\FunctionTok{#>}\NormalTok{ y }\FunctionTok{+}\NormalTok{ (n }\FunctionTok{^.}\NormalTok{ bias2)}
\end{Highlighting}
\end{Shaded}

\texttt{runNet} takes a network and produces the
\texttt{R\ 784\ -\textgreater{}\ R\ 10} function it encodes.

\texttt{\#\textgreater{}\ ::\ L\ m\ n\ -\textgreater{}\ R\ n\ -\textgreater{}\ R\ m}
is the matrix-vector multiplication operator from \emph{hmatrix} (its
\href{https://hackage.haskell.org/package/hmatrix/docs/Numeric-LinearAlgebra-Static.html}{static}
module); we can also just use \texttt{+} (from \texttt{Num}) to add vectors
together.

We use the \href{https://en.wikipedia.org/wiki/Logistic_function}{logistic
function} as our internal activation function and
\href{https://en.wikipedia.org/wiki/Softmax_function}{softmax} to normalize our
outputs:

\begin{Shaded}
\begin{Highlighting}[]
\CommentTok{-- source: https://github.com/mstksg/inCode/tree/master/code-samples/backprop/intro-normal.hs#L26-L34}

\OtherTok{logistic ::} \DataTypeTok{Floating}\NormalTok{ a }\OtherTok{=>}\NormalTok{ a }\OtherTok{->}\NormalTok{ a}
\NormalTok{logistic x }\FunctionTok{=} \DecValTok{1} \FunctionTok{/}\NormalTok{ (}\DecValTok{1} \FunctionTok{+}\NormalTok{ exp (}\FunctionTok{-}\NormalTok{x))}

\NormalTok{softMax}
\OtherTok{    ::} \DataTypeTok{R} \DecValTok{10}
    \OtherTok{->} \DataTypeTok{R} \DecValTok{10}
\NormalTok{softMax x }\FunctionTok{=}\NormalTok{ expx }\FunctionTok{/}\NormalTok{ konst (norm_1 expx)}
  \KeywordTok{where}
\NormalTok{    expx }\FunctionTok{=}\NormalTok{ exp x}
\end{Highlighting}
\end{Shaded}

We can define the logistic function using only \texttt{Num} operations, which
operate component-wise for \emph{hmatrix} types. \texttt{softMax} requires us to
\texttt{norm\_1} (to get the absolute sum of all items in a vector) from
\emph{hmatrix}, and also \texttt{konst} (to generate a vector of a single item
repeated). Still, though, pretty much a straightforward implementation of the
mathematical definitions.

\hypertarget{error-function}{%
\subsection{Error Function}\label{error-function}}

This neural network now makes predictions. However, in order to \emph{train} a
network, we actually need a scalar \emph{error function} that we want to
minimize. This is a function on the network that, given an input and its
expected output, computes how ``bad'' the currently network is. It computes the
error between the output of the network and the expected output, as a single
number.

To do this, we will be using the
\href{https://en.wikipedia.org/wiki/Cross_entropy}{cross entropy} between the
target output and the network output. This is a standard error function for
classification problems; smaller cross-entropies indicate ``better''
predictions.

\begin{Shaded}
\begin{Highlighting}[]
\CommentTok{-- source: https://github.com/mstksg/inCode/tree/master/code-samples/backprop/intro-normal.hs#L36-L56}

\NormalTok{crossEntropy}
\OtherTok{    ::} \DataTypeTok{R} \DecValTok{10}
    \OtherTok{->} \DataTypeTok{R} \DecValTok{10}
    \OtherTok{->} \DataTypeTok{Double}
\NormalTok{crossEntropy targ res }\FunctionTok{=} \FunctionTok{-}\NormalTok{(log res }\FunctionTok{<.>}\NormalTok{ targ)}

\NormalTok{netErr}
\OtherTok{    ::} \DataTypeTok{R} \DecValTok{784}
    \OtherTok{->} \DataTypeTok{R} \DecValTok{10}
    \OtherTok{->} \DataTypeTok{Net}
    \OtherTok{->} \DataTypeTok{Double}
\NormalTok{netErr x targ n }\FunctionTok{=}\NormalTok{ crossEntropy targ (runNet n x)}
\end{Highlighting}
\end{Shaded}

Computing the cross entropy requires us to use
\texttt{\textless{}.\textgreater{}} (the dot product) from \emph{hmatrix}, but
other than that we can just use \texttt{log} (from \texttt{Floating}) and
negation (from \texttt{Num}).

\hypertarget{training}{%
\subsection{Training}\label{training}}

At this point, we are supposed to find a way to compute the \emph{gradient} of
our error function. It's a function that computes the \emph{direction of
greatest change} of all of the components in our network, with respect to our
error function.

The gradient will take our \texttt{Net\ -\textgreater{}\ Double} error function
and produce a \texttt{Net} whose components contain the derivative of each
component with respect to the error. It tells us how to ``nudge'' each component
to increase the error function. \emph{Training} a neural network involves moving
in the opposite direction of the gradient, which causes the error to go
\emph{down}.

However, given \texttt{netErr}'s definition, it is not obvious how to compute
our gradient function. Doing so involves some careful multi-variable calculus
based on our knowledge of the operations we used. For simple situations we often
do it by hand, but for more complicated situations, this becomes impractical.
That's where \emph{automatic differentiation} comes into play.

We've gone as far as we can go now, so let's drop into the world of
\emph{backprop} and see what it can offer us!

\hypertarget{with-backprop}{%
\section{With Backprop}\label{with-backprop}}

Let's see what happens if we compute our error function using \emph{backprop},
instead!

We'll switch out our imports very slightly:

\begin{Shaded}
\begin{Highlighting}[]
\CommentTok{-- source: https://github.com/mstksg/inCode/tree/master/code-samples/backprop/intro-backprop.hs#L14-L16}

\KeywordTok{import}           \DataTypeTok{Control.Lens} \KeywordTok{hiding}\NormalTok{                   ((<.>))}
\KeywordTok{import}           \DataTypeTok{Numeric.Backprop}
\KeywordTok{import}           \DataTypeTok{Numeric.LinearAlgebra.Static.Backprop}
\end{Highlighting}
\end{Shaded}

First, we add \texttt{Numeric.Backprop}, the module where the magic happens.

Second, we switch from \texttt{Numeric.LinearAlgebra.Static} to
\texttt{Numeric.LinearAlgebra.Static.Backprop}, which exports the exact API as
\texttt{Numeric.LinearAlgebra.Static}, except with numeric operations that are
``lifted'' to work with \emph{backprop}. It's meant to act as a drop-in
replacement, and, because of this, most of our actual code will be more
identical.

\hypertarget{running-1}{%
\subsection{Running}\label{running-1}}

Writing functions that can be used with \emph{backprop} involves changing the
types slightly -- instead of working directly with values of type \texttt{a}, we
work with \texttt{BVar}s (backpropagatable variables) \emph{containing}
\texttt{a}s: a \texttt{BVar\ s\ a}.

For example, let's look a version \texttt{softMax} that works with
\emph{backprop}:

\begin{Shaded}
\begin{Highlighting}[]
\CommentTok{-- source: https://github.com/mstksg/inCode/tree/master/code-samples/backprop/intro-backprop.hs#L56-L62}

\NormalTok{softMax}
\OtherTok{    ::} \DataTypeTok{Reifies}\NormalTok{ s }\DataTypeTok{W}
    \OtherTok{=>} \DataTypeTok{BVar}\NormalTok{ s (}\DataTypeTok{R} \DecValTok{10}\NormalTok{)}
    \OtherTok{->} \DataTypeTok{BVar}\NormalTok{ s (}\DataTypeTok{R} \DecValTok{10}\NormalTok{)}
\NormalTok{softMax x }\FunctionTok{=}\NormalTok{ expx }\FunctionTok{/}\NormalTok{ konst (norm_1V expx)}
  \KeywordTok{where}
\NormalTok{    expx }\FunctionTok{=}\NormalTok{ exp x}
\end{Highlighting}
\end{Shaded}

Instead of \texttt{R\ 10\ -\textgreater{}\ R\ 10}, its type signature is now
\texttt{BVar\ s\ (R\ 10)\ -\textgreater{}\ BVar\ s\ (R\ 10)}. Instead of working
directly with \texttt{R\ 10}s (10-vectors), we work with
\texttt{BVar\ s\ (R\ 10)}s (\texttt{BVar}s containing 10-vectors).

\texttt{Numeric.LinearAlgebra.Static.Backprop} re-exports \texttt{konst} and
\texttt{norm\_1} (as \texttt{norm\_1V} --- \texttt{norm\_1} for vectors only)
lifted to work with \texttt{BVar}s:

\begin{Shaded}
\begin{Highlighting}[]
\CommentTok{-- normal}
\OtherTok{konst ::}        \DataTypeTok{Double} \OtherTok{->}         \DataTypeTok{R} \DecValTok{10}
\CommentTok{-- backprop}
\OtherTok{konst ::} \DataTypeTok{BVar}\NormalTok{ s }\DataTypeTok{Double} \OtherTok{->} \DataTypeTok{BVar}\NormalTok{ s (}\DataTypeTok{R} \DecValTok{10}\NormalTok{)}

\CommentTok{-- normal}
\OtherTok{norm_1  ::}         \DataTypeTok{R} \DecValTok{10}  \OtherTok{->}        \DataTypeTok{Double}
\CommentTok{-- backprop}
\OtherTok{norm_1V ::} \DataTypeTok{BVar}\NormalTok{ s (}\DataTypeTok{R} \DecValTok{10}\NormalTok{) }\OtherTok{->} \DataTypeTok{BVar}\NormalTok{ s }\DataTypeTok{Double}
\end{Highlighting}
\end{Shaded}

\texttt{BVar}s also have \texttt{Num}, \texttt{Fractional}, and
\texttt{Floating} instances, so \texttt{exp} and \texttt{/} already work
out-of-the-box.

With only a minimal and mechanical change in our code, \texttt{softMax} is now
automatically differentiable!

One neat trick --- because of \texttt{BVar}'s numeric instances, we can actually
re-use our original implementation of \texttt{logistic}:

\begin{Shaded}
\begin{Highlighting}[]
\CommentTok{-- source: https://github.com/mstksg/inCode/tree/master/code-samples/backprop/intro-backprop.hs#L53-L54}

\OtherTok{logistic ::} \DataTypeTok{Floating}\NormalTok{ a }\OtherTok{=>}\NormalTok{ a }\OtherTok{->}\NormalTok{ a}
\NormalTok{logistic x }\FunctionTok{=} \DecValTok{1} \FunctionTok{/}\NormalTok{ (}\DecValTok{1} \FunctionTok{+}\NormalTok{ exp (}\FunctionTok{-}\NormalTok{x))}
\end{Highlighting}
\end{Shaded}

To \emph{run} our network, things look pretty similar:

\begin{Shaded}
\begin{Highlighting}[]
\CommentTok{-- source: https://github.com/mstksg/inCode/tree/master/code-samples/backprop/intro-backprop.hs#L71-L79}

\NormalTok{runNet}
\OtherTok{    ::} \DataTypeTok{Reifies}\NormalTok{ s }\DataTypeTok{W}
    \OtherTok{=>} \DataTypeTok{BVar}\NormalTok{ s }\DataTypeTok{Net}
    \OtherTok{->} \DataTypeTok{BVar}\NormalTok{ s (}\DataTypeTok{R} \DecValTok{784}\NormalTok{)}
    \OtherTok{->} \DataTypeTok{BVar}\NormalTok{ s (}\DataTypeTok{R} \DecValTok{10}\NormalTok{)}
\NormalTok{runNet n x }\FunctionTok{=}\NormalTok{ z}
  \KeywordTok{where}
\NormalTok{    y }\FunctionTok{=}\NormalTok{ logistic }\FunctionTok{$}\NormalTok{ (n }\FunctionTok{^^.}\NormalTok{ weights1) }\FunctionTok{#>}\NormalTok{ x }\FunctionTok{+}\NormalTok{ (n }\FunctionTok{^^.}\NormalTok{ bias1)}
\NormalTok{    z }\FunctionTok{=}\NormalTok{ softMax  }\FunctionTok{$}\NormalTok{ (n }\FunctionTok{^^.}\NormalTok{ weights2) }\FunctionTok{#>}\NormalTok{ y }\FunctionTok{+}\NormalTok{ (n }\FunctionTok{^^.}\NormalTok{ bias2)}
\end{Highlighting}
\end{Shaded}

Again, pretty much the same, except with the lifted type signature. One notable
difference, however, is how we \emph{access} the weights and biases. Instead of
using \texttt{\^{}.} for lens access, we can use \texttt{\^{}\^{}.}, for lens
access into a \texttt{BVar}:

\begin{Shaded}
\begin{Highlighting}[]
\OtherTok{myNetVar                ::} \DataTypeTok{BVar}\NormalTok{ s }\DataTypeTok{Net}          \CommentTok{-- a Net inside a BVar}
\NormalTok{myNetVar }\FunctionTok{^^.}\OtherTok{ weights1   ::} \DataTypeTok{BVar}\NormalTok{ s (}\DataTypeTok{L} \DecValTok{250} \DecValTok{784}\NormalTok{)  }\CommentTok{-- access the weights1 field in myNetVar}
\NormalTok{myNetVar }\FunctionTok{^^.}\OtherTok{ bias2      ::} \DataTypeTok{BVar}\NormalTok{ s (}\DataTypeTok{R}  \DecValTok{10}\NormalTok{    )  }\CommentTok{-- access the bias2 field in myNetVar}
\end{Highlighting}
\end{Shaded}

Some insight may be gleamed from a comparison of their type signatures:

\begin{Shaded}
\begin{Highlighting}[]
\OtherTok{(^.)  ::}\NormalTok{        a }\OtherTok{->} \DataTypeTok{Lens'}\NormalTok{ a b }\OtherTok{->}\NormalTok{        b}
\OtherTok{(^^.) ::} \DataTypeTok{BVar}\NormalTok{ s a }\OtherTok{->} \DataTypeTok{Lens'}\NormalTok{ a b }\OtherTok{->} \DataTypeTok{BVar}\NormalTok{ s b}
\end{Highlighting}
\end{Shaded}

\texttt{\^{}.} is access to a value using a lens, and \texttt{\^{}\^{}.} is
access to a value inside a \texttt{BVar} using a lens.

Using lenses like this gives us essentially frictionless usage of
\texttt{BVar}s, allowing us to access items inside data types in a natural way.
We can also \emph{set} items using \texttt{.\textasciitilde{}\textasciitilde{}},
instead of \texttt{.\textasciitilde{}}, and access constructors in sum types
using \texttt{\^{}\^{}?} (which can implement pattern matching) and get matches
for \emph{multiple} targets using \texttt{\^{}\^{}..}:

\begin{Shaded}
\begin{Highlighting}[]
\OtherTok{(^..)  ::}\NormalTok{        a }\OtherTok{->} \DataTypeTok{Traversal'}\NormalTok{ a b }\OtherTok{->}\NormalTok{ [       b]}
\OtherTok{(^^..) ::} \DataTypeTok{BVar}\NormalTok{ s a }\OtherTok{->} \DataTypeTok{Traversal'}\NormalTok{ a b }\OtherTok{->}\NormalTok{ [}\DataTypeTok{BVar}\NormalTok{ s b]}
\end{Highlighting}
\end{Shaded}

Because of these, our translation from our normal \texttt{runNet} to our
\emph{backprop} \texttt{runNet} is more or less completely mechanical.

\hypertarget{error-function-1}{%
\subsection{Error Function}\label{error-function-1}}

At this point, the implementation of our updated error function should not be
too surprising:

\begin{Shaded}
\begin{Highlighting}[]
\CommentTok{-- source: https://github.com/mstksg/inCode/tree/master/code-samples/backprop/intro-backprop.hs#L64-L87}

\NormalTok{crossEntropy}
\OtherTok{    ::} \DataTypeTok{Reifies}\NormalTok{ s }\DataTypeTok{W}
    \OtherTok{=>} \DataTypeTok{BVar}\NormalTok{ s (}\DataTypeTok{R} \DecValTok{10}\NormalTok{)}
    \OtherTok{->} \DataTypeTok{BVar}\NormalTok{ s (}\DataTypeTok{R} \DecValTok{10}\NormalTok{)}
    \OtherTok{->} \DataTypeTok{BVar}\NormalTok{ s (}\DataTypeTok{Double}\NormalTok{)}
\NormalTok{crossEntropy targ res }\FunctionTok{=} \FunctionTok{-}\NormalTok{(log res }\FunctionTok{<.>}\NormalTok{ targ)}

\NormalTok{netErr}
\OtherTok{    ::} \DataTypeTok{Reifies}\NormalTok{ s }\DataTypeTok{W}
    \OtherTok{=>} \DataTypeTok{BVar}\NormalTok{ s (}\DataTypeTok{R} \DecValTok{784}\NormalTok{)}
    \OtherTok{->} \DataTypeTok{BVar}\NormalTok{ s (}\DataTypeTok{R} \DecValTok{10}\NormalTok{)}
    \OtherTok{->} \DataTypeTok{BVar}\NormalTok{ s }\DataTypeTok{Net}
    \OtherTok{->} \DataTypeTok{BVar}\NormalTok{ s }\DataTypeTok{Double}
\NormalTok{netErr x targ n }\FunctionTok{=}\NormalTok{ crossEntropy targ (runNet n x)}
\end{Highlighting}
\end{Shaded}

Both of these implementations are are 100\% lexicographically \emph{identical}
in implementation to our original ones -- the only difference is that
\texttt{\textless{}.\textgreater{}} comes from
\texttt{Numeric.LinearAlgebra.Static.Backprop}. Other than that, we re-use
\texttt{log} and negation.

\hypertarget{training-1}{%
\subsection{Training}\label{training-1}}

Time to gradient descend!

\begin{Shaded}
\begin{Highlighting}[]
\CommentTok{-- source: https://github.com/mstksg/inCode/tree/master/code-samples/backprop/intro-backprop.hs#L89-L93}

\OtherTok{stepNet ::} \DataTypeTok{R} \DecValTok{784} \OtherTok{->} \DataTypeTok{R} \DecValTok{10} \OtherTok{->} \DataTypeTok{Net} \OtherTok{->} \DataTypeTok{Net}
\NormalTok{stepNet x targ net0 }\FunctionTok{=}\NormalTok{ net0 }\FunctionTok{-} \FloatTok{0.02} \FunctionTok{*}\NormalTok{ gr}
  \KeywordTok{where}
\OtherTok{    gr ::} \DataTypeTok{Net}
\NormalTok{    gr }\FunctionTok{=}\NormalTok{ gradBP (netErr (constVar x) (constVar targ)) net0}
\end{Highlighting}
\end{Shaded}

And\ldots{}that's it!

To break this down:

\begin{enumerate}
\def\labelenumi{\arabic{enumi}.}
\item
  To train our network, we move in the opposite direction of our gradient. That
  means \texttt{net0\ -\ 0.02\ *\ gr} -- we subtract the gradient (scaled by
  0.02, a learning rate, to ensure we don't overshoot our goal) from our
  network.
\item
  To compute our gradient, we use \texttt{gradBP}:

\begin{Shaded}
\begin{Highlighting}[]
\OtherTok{gradBP ::}\NormalTok{ (forall s}\FunctionTok{.} \DataTypeTok{Reifies}\NormalTok{ s }\DataTypeTok{W} \OtherTok{=>} \DataTypeTok{BVar}\NormalTok{ s a }\OtherTok{->} \DataTypeTok{BVar}\NormalTok{ s b) }\OtherTok{->}\NormalTok{ a }\OtherTok{->}\NormalTok{ a}
\end{Highlighting}
\end{Shaded}

  If we ignore the RankN type/\texttt{Reifies} syntax noise, this can be read
  as:

\begin{Shaded}
\begin{Highlighting}[]
\OtherTok{gradBP ::}\NormalTok{ (}\DataTypeTok{BVar}\NormalTok{ s a }\OtherTok{->} \DataTypeTok{BVar}\NormalTok{ s b) }\OtherTok{->}\NormalTok{ a }\OtherTok{->}\NormalTok{ a}
\end{Highlighting}
\end{Shaded}

  Which says ``give a function from a \texttt{BVar} of \texttt{a} to a
  \texttt{BVar} of \texttt{b}, get the gradient function, from \texttt{a} to its
  gradient''

  This can be contrasted with \texttt{evalBP}:

\begin{Shaded}
\begin{Highlighting}[]
\OtherTok{evalBP ::}\NormalTok{ (}\DataTypeTok{BVar}\NormalTok{ s a }\OtherTok{->} \DataTypeTok{BVar}\NormalTok{ s b) }\OtherTok{->}\NormalTok{ a }\OtherTok{->}\NormalTok{ b}
\end{Highlighting}
\end{Shaded}

  Which ``runs'' the actual \texttt{a\ -\textgreater{}\ b} function that the
  \texttt{BVar\ s\ a\ -\textgreater{}\ BVar\ s\ b} encodes.
\item
  We want to use \texttt{gradBP} with our \texttt{Net\ -\textgreater{}\ Double}
  error function (or more accurately our
  \texttt{BVar\ s\ Net\ -\textgreater{}\ BVar\ s\ Double} function). That's
  exactly what \texttt{netErr} gives us.

  We use \texttt{constVar} to lift \texttt{x} and \texttt{targ}:

\begin{Shaded}
\begin{Highlighting}[]
\OtherTok{constVar ::}\NormalTok{ a }\OtherTok{->} \DataTypeTok{BVar}\NormalTok{ s a}
\end{Highlighting}
\end{Shaded}

  \texttt{constVar} simply lifts a value into a \texttt{BVar}, knowing that we
  don't care about its gradient.

  This means that we have:

\begin{Shaded}
\begin{Highlighting}[]
\NormalTok{netErr (constVar x) (constVar targ)}\OtherTok{ ::} \DataTypeTok{BVar}\NormalTok{ s }\DataTypeTok{Net} \OtherTok{->} \DataTypeTok{BVar}\NormalTok{ s }\DataTypeTok{Double}
\end{Highlighting}
\end{Shaded}

  We can pass this function to \texttt{gradBP} to get the gradient of the
  network \texttt{Net} with respect to the \texttt{Double} error.
\end{enumerate}

And that's really the whole gradient computation and descent code!

Kind of anti-climactic, isn't it?

\hypertarget{taking-it-for-a-spin}{%
\section{Taking it for a spin}\label{taking-it-for-a-spin}}

In the
\href{https://github.com/mstksg/inCode/tree/master/code-samples/backprop/intro-backprop.hs}{source
code} I've included some basic code for loading the mnist data set and training
the network, with some basic evaluations.

If you download it
\href{https://github.com/mstksg/inCode/tree/master/code-samples/backprop/intro-backprop.hs}{here},
you can compile it using a stack's self-compiling script feature (if
\emph{\href{https://docs.haskellstack.org/en/stable/README/}{stack}} is
installed on your computer):

\begin{Shaded}
\begin{Highlighting}[]
\NormalTok{$ }\ExtensionTok{./intro-backprop.hs}\NormalTok{    # compiles itself, managing dependencies automatically}
\end{Highlighting}
\end{Shaded}

The above command will cause the program to compile itself, installing the
necessary GHC (if needed) and also the automatically download the dependencies
from hackage. \emph{backprop} manages the automatic differentiation, and
\emph{stack} manages the automatic dependency management :)

If you are following along at home, you can download the
\href{http://yann.lecun.com/exdb/mnist/}{mnist data set files} and uncompress
them into a folder, and run it all with:

\begin{Shaded}
\begin{Highlighting}[]
\NormalTok{$ }\ExtensionTok{./intro-backprop}\NormalTok{ PATH_TO_DATA}
\ExtensionTok{Loaded}\NormalTok{ data.}
\NormalTok{[}\ExtensionTok{Epoch}\NormalTok{ 1]}
\KeywordTok{(}\ExtensionTok{Batch}\NormalTok{ 1}\KeywordTok{)}
\ExtensionTok{Trained}\NormalTok{ on 5000 points.}
\ExtensionTok{Training}\NormalTok{ error:   13.26%}
\ExtensionTok{Validation}\NormalTok{ error: 13.44%}
\KeywordTok{(}\ExtensionTok{Batch}\NormalTok{ 2}\KeywordTok{)}
\ExtensionTok{Trained}\NormalTok{ on 5000 points.}
\ExtensionTok{Training}\NormalTok{ error:   9.74%}
\ExtensionTok{Validation}\NormalTok{ error: 11.08%}
\KeywordTok{(}\ExtensionTok{Batch}\NormalTok{ 3}\KeywordTok{)}
\ExtensionTok{Trained}\NormalTok{ on 5000 points.}
\ExtensionTok{Training}\NormalTok{ error:   6.84%}
\ExtensionTok{Validation}\NormalTok{ error: 8.71%}
\KeywordTok{(}\ExtensionTok{Batch}\NormalTok{ 4}\KeywordTok{)}
\ExtensionTok{Trained}\NormalTok{ on 5000 points.}
\ExtensionTok{Training}\NormalTok{ error:   6.84%}
\ExtensionTok{Validation}\NormalTok{ error: 8.53%}
\KeywordTok{(}\ExtensionTok{Batch}\NormalTok{ 5}\KeywordTok{)}
\ExtensionTok{Trained}\NormalTok{ on 5000 points.}
\ExtensionTok{Training}\NormalTok{ error:   5.80%}
\ExtensionTok{Validation}\NormalTok{ error: 7.55%}
\KeywordTok{(}\ExtensionTok{Batch}\NormalTok{ 6}\KeywordTok{)}
\ExtensionTok{Trained}\NormalTok{ on 5000 points.}
\ExtensionTok{Training}\NormalTok{ error:   5.20%}
\ExtensionTok{Validation}\NormalTok{ error: 6.77%}
\KeywordTok{(}\ExtensionTok{Batch}\NormalTok{ 7}\KeywordTok{)}
\ExtensionTok{Trained}\NormalTok{ on 5000 points.}
\ExtensionTok{Training}\NormalTok{ error:   4.44%}
\ExtensionTok{Validation}\NormalTok{ error: 5.85%}
\end{Highlighting}
\end{Shaded}

After about 35000 training points, we get down to 94\% accuracy on our test set.
Neat!

\hypertarget{a-more-nuanced-look}{%
\section{A More Nuanced Look}\label{a-more-nuanced-look}}

That's the high level overview -- now let's look a bit at the details that might
be helpful before you go strike it out on your own.

The main API revolves around writing a
\texttt{BVar\ s\ a\ -\textgreater{}\ BVar\ s\ b} function (representing an
\texttt{a\ -\textgreater{}\ b} one), and then using one of the three runners:

\begin{Shaded}
\begin{Highlighting}[]
\CommentTok{-- Return the result and gradient}
\OtherTok{backprop ::}\NormalTok{ (}\DataTypeTok{Num}\NormalTok{ a, }\DataTypeTok{Num}\NormalTok{ b)}
         \OtherTok{=>}\NormalTok{ (forall s}\FunctionTok{.} \DataTypeTok{Reifies}\NormalTok{ s }\DataTypeTok{W} \OtherTok{=>} \DataTypeTok{BVar}\NormalTok{ s a }\OtherTok{->} \DataTypeTok{BVar}\NormalTok{ s b) }\OtherTok{->}\NormalTok{ a }\OtherTok{->}\NormalTok{ (a, b)}

\CommentTok{-- Return the result}
\OtherTok{evalBP   ::}\NormalTok{ (forall s}\FunctionTok{.} \DataTypeTok{Reifies}\NormalTok{ s }\DataTypeTok{W} \OtherTok{=>} \DataTypeTok{BVar}\NormalTok{ s a }\OtherTok{->} \DataTypeTok{BVar}\NormalTok{ s b) }\OtherTok{->}\NormalTok{ a }\OtherTok{->}\NormalTok{ b}

\CommentTok{-- Return the gradient}
\OtherTok{gradBP   ::}\NormalTok{ (}\DataTypeTok{Num}\NormalTok{ a, }\DataTypeTok{Num}\NormalTok{ b)}
         \OtherTok{=>}\NormalTok{ (forall s}\FunctionTok{.} \DataTypeTok{Reifies}\NormalTok{ s }\DataTypeTok{W} \OtherTok{=>} \DataTypeTok{BVar}\NormalTok{ s a }\OtherTok{->} \DataTypeTok{BVar}\NormalTok{ s b) }\OtherTok{->}\NormalTok{ a }\OtherTok{->}\NormalTok{ a}
\end{Highlighting}
\end{Shaded}

\texttt{evalBP} comes with virtually zero performance overhead (about 4\%) over
writing your functions directly, so there's pretty much no harm in writing your
entire application or library in \texttt{BVar}-based code.

\texttt{gradBP}, however, carries a some measurable performance overhead over
writing your gradient code ``manually'', but this heavily depends on exactly how
complex the code you are backpropagating is. The overhead comes from two
potential sources: the building of the function call graph, and also potentially
from the mechanical automatic differentiation process generating different
operations than what you might write by hand. See the
\href{https://github.com/mstksg/backprop\#readme}{README} for a deeper analysis.

You might have also noticed the RankN type signature (the
\texttt{forall\ s.\ ...}) that I glossed over earlier. This is here because
\emph{backprop} uses the RankN type trick from \texttt{Control.Monad.ST} and the
\emph{\href{http://hackage.haskell.org/package/ad}{ad}} library, for two
purposes:

\begin{enumerate}
\def\labelenumi{\arabic{enumi}.}
\tightlist
\item
  The prevent leakage of variables from the function. You can't use
  \texttt{evalBP} to get a \texttt{BVar} out in the end, just like you can't use
  \texttt{runST} to get an \texttt{STRef} out in the end. The type system
  prevents these variables from leaking out of the backprop/ST world.
\item
  The \texttt{Reifies\ s\ W} constraint allows \emph{backprop} to build a
  \href{https://dl.acm.org/citation.cfm?doid=355586.364791}{Wengert Tape} of
  your computation, which it uses internally to perform the reverse-mode
  automatic differentiation (The \texttt{W} stands for Wengert).
\end{enumerate}

\hypertarget{discussion-on-num}{%
\subsection{Discussion on Num}\label{discussion-on-num}}

Note that at the moment, \texttt{backprop} and \texttt{gradBP} (and
\texttt{(\^{}\^{}.)} and most \texttt{BVar}-based operations) all require a
\texttt{Num} instance on the things being backpropagated. This is an API
decision that is a compromise between different options, and the
\href{https://github.com/mstksg/backprop\#readme}{README} has a deeper
discussion on this.

Writing a \texttt{Num} instance for your types is some manageable boilerplate if
your type derives Generic (and we can use
\emph{\href{http://hackage.haskell.org/package/one-liner-instances}{one-liner-instances}}),
like we saw above with the \texttt{Num} instance for \texttt{Net}.

However, requiring a \texttt{Num} instance means you can't directly
backpropagate tuples. This can be an issue because of how pervasive tuples are
used for currying/uncurrying, and also because automatically generated prisms
use tuples for constructors with multiple fields.

To mitigate this issue, the library exports some convenient
tuples-with-Num-instances in \texttt{Numeric.Backprop.Tuple}. If you are writing
an application, you can use the orphan instances in
\emph{\href{https://hackage.haskell.org/package/NumInstances}{NumInstances}}.

\hypertarget{lifting-your-own-functions}{%
\subsection{Lifting your own functions}\label{lifting-your-own-functions}}

Of course, all of this would be useless unless you had a way to manipulate
\texttt{BVar}s. The library does provide lens-based accessors/setters. It also
provides \texttt{Num}, \texttt{Fractional}, and \texttt{Floating} instances for
\texttt{BVar}s so you can manipulate a \texttt{BVar\ s\ a} just like an
\texttt{a} using its numeric instances. We leveraged this heavily by using
\texttt{+}, \texttt{negate}, \texttt{log}, \texttt{/}, etc., and even going as
far as re-using our entire \texttt{logistic} implementation because it only
relied on numeric operations.

However, for our domain-specific operations (like matrix multiplication, norms,
and dot products), we needed to somehow lift those operations into
\emph{backprop}-land, to work with \texttt{BVar}s.

This isn't something that normal users should be expected to do -- ideally, this
would be done by library maintainers and authors, so that users can use their
types and operations with \emph{backprop}. However, writing them is not magical
-- it just requires providing the result and the gradient with respect to a
final total derivative. For example, let's look at the implementation of the
lifted \texttt{\textless{}.\textgreater{}}:

\begin{Shaded}
\begin{Highlighting}[]
\KeywordTok{import} \KeywordTok{qualified} \DataTypeTok{Numeric.LinearAlgebra.Static} \KeywordTok{as} \DataTypeTok{H}

\NormalTok{(}\FunctionTok{<.>}\NormalTok{)}
\OtherTok{    ::} \DataTypeTok{Reifies}\NormalTok{ s }\DataTypeTok{W}
    \OtherTok{=>} \DataTypeTok{BVar}\NormalTok{ s (}\DataTypeTok{R} \DecValTok{10}\NormalTok{)}
    \OtherTok{->} \DataTypeTok{BVar}\NormalTok{ s (}\DataTypeTok{R} \DecValTok{10}\NormalTok{)}
    \OtherTok{->} \DataTypeTok{BVar}\NormalTok{ s }\DataTypeTok{Double}
\NormalTok{(}\FunctionTok{<.>}\NormalTok{) }\FunctionTok{=}\NormalTok{ liftOp2 }\FunctionTok{.}\NormalTok{ op2 }\FunctionTok{$}\NormalTok{ \textbackslash{}x y }\OtherTok{->}
\NormalTok{    ( x }\FunctionTok{H.<.>}\NormalTok{ y}
\NormalTok{    , \textbackslash{}d }\OtherTok{->}\NormalTok{ (H.konst d }\FunctionTok{*}\NormalTok{ y, x }\FunctionTok{*}\NormalTok{ H.konst d)}
\NormalTok{    )}
\end{Highlighting}
\end{Shaded}

To lift \texttt{(\textless{}.\textgreater{})}, we provide a function that, given
its inputs \texttt{x} and \texttt{y}, gives the result
(\texttt{x\ H.\textless{}.\textgreater{}\ y}), and also its gradient with
respect to the total derivative of the result. For more details on the math, see
the
\href{https://hackage.haskell.org/package/backprop-0.1.2.0/docs/Numeric-Backprop-Op.html}{documentation
for \texttt{Numeric.Backprop.Op}}!

If you're interested in writing your own lifted operations, take a look at the
\href{https://github.com/mstksg/hmatrix-backprop/blob/master/src/Numeric/LinearAlgebra/Static/Backprop.hs}{source
of the lifted hmatrix module}, which lifts (most) of the functionality of
\emph{hmatrix} for backprop. (And if you're good at computing gradients, check
out the module notes for some of the current unimplemented operators -- any PR's
would definitely be appreciated!)

\hypertarget{conclusion}{%
\section{Conclusion}\label{conclusion}}

The world is now your oyster! Go out and feel emboldened to numerically optimize
everything you can get your hands on!

If you want to see an application to a more complex neural network type (and if
you're curious at how to implement the more ``extensible'' neural network types
like in my
\href{https://blog.jle.im/entries/series/+practical-dependent-types-in-haskell.html}{blog
series on extensible neural networks}), I wrote
\href{https://github.com/mstksg/backprop/blob/master/renders/extensible-neural.pdf}{a
quick write-up} on how to apply those type-level dependent programming
techniques to \emph{backprop} (also available in
\href{https://github.com/mstksg/backprop/blob/master/samples/extensible-neural.lhs}{literate
haskell}).

Really, though, the goal of backprop is to allow you to automatically
differentiate things you have \emph{already} written. Over the next few weeks
I'll be lifting operations from other libraries in the ecosystem. Let me know if
there are any that you might want me to look at first! Be also on the lookout
for some other posts I'll be writing on applying \emph{backprop} to optimize
things other than neural networks.

If you have any questions, feel free to leave a comment. You can also give me a
shout on \href{https://twitter.com/mstk}{twitter} (I'm \emph{@mstk}), on
freenode's \emph{\#haskell} (where I am usually idling as \emph{jle`}), or on
the \href{https://gitter.im/dataHaskell/Lobby}{DataHaskell gitter} (where I hang
out as \emph{@mstksg}).

Please let me know if you end up doing anything interesting with the library ---
I'd love to hear about it! And, until next time, happy Haskelling!

\end{document}
