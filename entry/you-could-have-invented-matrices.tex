\documentclass[]{article}
\usepackage{lmodern}
\usepackage{amssymb,amsmath}
\usepackage{ifxetex,ifluatex}
\usepackage{fixltx2e} % provides \textsubscript
\ifnum 0\ifxetex 1\fi\ifluatex 1\fi=0 % if pdftex
  \usepackage[T1]{fontenc}
  \usepackage[utf8]{inputenc}
\else % if luatex or xelatex
  \ifxetex
    \usepackage{mathspec}
    \usepackage{xltxtra,xunicode}
  \else
    \usepackage{fontspec}
  \fi
  \defaultfontfeatures{Mapping=tex-text,Scale=MatchLowercase}
  \newcommand{\euro}{€}
\fi
% use upquote if available, for straight quotes in verbatim environments
\IfFileExists{upquote.sty}{\usepackage{upquote}}{}
% use microtype if available
\IfFileExists{microtype.sty}{\usepackage{microtype}}{}
\usepackage[margin=1in]{geometry}
\usepackage{graphicx}
\makeatletter
\def\maxwidth{\ifdim\Gin@nat@width>\linewidth\linewidth\else\Gin@nat@width\fi}
\def\maxheight{\ifdim\Gin@nat@height>\textheight\textheight\else\Gin@nat@height\fi}
\makeatother
% Scale images if necessary, so that they will not overflow the page
% margins by default, and it is still possible to overwrite the defaults
% using explicit options in \includegraphics[width, height, ...]{}
\setkeys{Gin}{width=\maxwidth,height=\maxheight,keepaspectratio}
\ifxetex
  \usepackage[setpagesize=false, % page size defined by xetex
              unicode=false, % unicode breaks when used with xetex
              xetex]{hyperref}
\else
  \usepackage[unicode=true]{hyperref}
\fi
\hypersetup{breaklinks=true,
            bookmarks=true,
            pdfauthor={Justin Le},
            pdftitle={You Could Have Invented Matrices!},
            colorlinks=true,
            citecolor=blue,
            urlcolor=blue,
            linkcolor=magenta,
            pdfborder={0 0 0}}
\urlstyle{same}  % don't use monospace font for urls
% Make links footnotes instead of hotlinks:
\renewcommand{\href}[2]{#2\footnote{\url{#1}}}
\setlength{\parindent}{0pt}
\setlength{\parskip}{6pt plus 2pt minus 1pt}
\setlength{\emergencystretch}{3em}  % prevent overfull lines
\setcounter{secnumdepth}{0}

\title{You Could Have Invented Matrices!}
\author{Justin Le}

\begin{document}
\maketitle

\emph{Originally posted on
\textbf{\href{https://blog.jle.im/entry/you-could-have-invented-matrices.html}{in
Code}}.}

You could have invented matrices!

Let's talk about vectors. A \textbf{vector} (denoted as
\includegraphics{https://latex.codecogs.com/png.latex?\%5Cmathbf\%7Bv\%7D}, a
lower-case bold italicized letter) is an element in a \textbf{vector space},
which means that it can be ``scaled'', like
\includegraphics{https://latex.codecogs.com/png.latex?c\%20\%5Cmathbf\%7Bv\%7D}
(the \includegraphics{https://latex.codecogs.com/png.latex?c} is called a
``scalar'' --- creative name, right?) and added, like
\includegraphics{https://latex.codecogs.com/png.latex?\%5Cmathbf\%7Bv\%7D\%20\%2B\%20\%5Cmathbf\%7Bu\%7D}.

In order for vector spaces and their operations to be valid, they just have to
obey some
\href{https://en.wikipedia.org/wiki/Vector_space\#Definition}{common-sense
rules} (like associativity, commutativity, distributivity, etc.) that allow us
to make meaningful conclusions.

\hypertarget{dimensionality}{%
\section{Dimensionality}\label{dimensionality}}

One neat thing about vector spaces is that, in \emph{some} of them, you have the
ability to ``decompose'' any vector in it as a \emph{weighted sum} of some set
of ``basis'' vectors. If this is the case for your vector space, then the size
of smallest possible set of basis vectors is known as the \textbf{dimension} of
that vector space.

For example, for a 3-dimensional vector space
\includegraphics{https://latex.codecogs.com/png.latex?V}, any vector
\includegraphics{https://latex.codecogs.com/png.latex?\%5Cmathbf\%7Bv\%7D} can
be described as a weighted sum of three basis vectors
\includegraphics{https://latex.codecogs.com/png.latex?\%5Cmathbf\%7Be\%7D_1},
\includegraphics{https://latex.codecogs.com/png.latex?\%5Cmathbf\%7Be\%7D_2},
\includegraphics{https://latex.codecogs.com/png.latex?\%5Cmathbf\%7Be\%7D_3}:

{[} \textbackslash{}mathbf\{v\} = a \textbackslash{}mathbf\{e\}\_1 + b
\textbackslash{}mathbf\{e\}\_2 + c
\textbackslash{}mathbf\{e\}\_3{]}(https://latex.codecogs.com/png.latex?\%0A\%5Cmathbf\%7Bv\%7D\%20\%3D\%20a\%20\%5Cmathbf\%7Be\%7D\_1\%20\%2B\%20b\%20\%5Cmathbf\%7Be\%7D\_2\%20\%2B\%20c\%20\%5Cmathbf\%7Be\%7D\_3\%0A
" \mathbf{v} = a \mathbf{e}\_1 + b \mathbf{e}\_2 + c \mathbf{e}\_3 ``)

Where \includegraphics{https://latex.codecogs.com/png.latex?a},
\includegraphics{https://latex.codecogs.com/png.latex?b}, and
\includegraphics{https://latex.codecogs.com/png.latex?c} are scalars.

Dimensionality is really a statement about being able to decompose any vector in
that vector space into a useful set of bases. For a 3-dimensional vector space,
you can make a bases that can reproduce \emph{any} vector in your
space\ldots{}but that's only possible with at least three vectors.

In physics, we often treat reality as taking place in a three-dimensional vector
space. The basis vectors are often called
\includegraphics{https://latex.codecogs.com/png.latex?\%5Chat\%7B\%5Cmathbf\%7Bi\%7D\%7D},
\includegraphics{https://latex.codecogs.com/png.latex?\%5Chat\%7B\%5Cmathbf\%7Bj\%7D\%7D},
and
\includegraphics{https://latex.codecogs.com/png.latex?\%5Chat\%7B\%5Cmathbf\%7Bk\%7D\%7D},
and so we say that we can describe our 3D physics vectors as
\includegraphics{https://latex.codecogs.com/png.latex?\%5Cmathbf\%7Bv\%7D\%20\%3D\%20v_x\%20\%5Chat\%7B\%5Cmathbf\%7Bi\%7D\%7D\%20\%2B\%20v_y\%20\%5Chat\%7B\%5Cmathbf\%7Bj\%7D\%7D\%20\%2B\%20v_x\%20\%5Chat\%7B\%5Cmathbf\%7Bk\%7D\%7D}

\hypertarget{encoding}{%
\subsection{Encoding}\label{encoding}}

One neat thing that physicists take advantage of all the time is that if we
\emph{agree} on a set of basis vectors and a specific ordering, we can actually
\emph{encode} any vector
\includegraphics{https://latex.codecogs.com/png.latex?\%5Cmathbf\%7Bv\%7D} in
terms of those basis vectors.

So in physics, we can say ``Let's encode vectors in terms of
\includegraphics{https://latex.codecogs.com/png.latex?\%5Chat\%7B\%5Cmathbf\%7Bi\%7D\%7D},
\includegraphics{https://latex.codecogs.com/png.latex?\%5Chat\%7B\%5Cmathbf\%7Bj\%7D\%7D},
and
\includegraphics{https://latex.codecogs.com/png.latex?\%5Chat\%7B\%5Cmathbf\%7Bk\%7D\%7D},
in that order.'' Then, we can \emph{write}
\includegraphics{https://latex.codecogs.com/png.latex?\%5Cmathbf\%7Bv\%7D} as
\includegraphics{https://latex.codecogs.com/png.latex?\%5Clangle\%20v_x\%2C\%20v_y\%2C\%20v_z\%20\%5Crangle},
and understand that we really
mean\includegraphics{https://latex.codecogs.com/png.latex?\%5Cmathbf\%7Bv\%7D\%20\%3D\%20v_x\%20\%5Chat\%7B\%5Cmathbf\%7Bi\%7D\%7D\%20\%2B\%20v_y\%20\%5Chat\%7B\%5Cmathbf\%7Bj\%7D\%7D\%20\%2B\%20v_x\%20\%5Chat\%7B\%5Cmathbf\%7Bk\%7D\%7D}.

Note that
\includegraphics{https://latex.codecogs.com/png.latex?\%5Clangle\%20v_x\%2C\%20v_y\%2C\%20v_z\%20\%5Crangle}
is \textbf{not} the same thing as the \textbf{vector}
\includegraphics{https://latex.codecogs.com/png.latex?\%5Cmathbf\%7Bv\%7D}. It
is \emph{an encoding} of that vector, that only makes sense once we choose to
\emph{agree} on a specific set of basis.

For an N-dimensional vector space, it means that, with a minimum of N items, we
can represent any vector in that space. And, if we agree on those N items, we
can devise an encoding, such that:

{[} \textbackslash{}langle v\_1, v\_2 \textbackslash{}dots v\_N
\textbackslash{}rangle{]}(https://latex.codecogs.com/png.latex?\%0A\%5Clangle\%20v\_1\%2C\%20v\_2\%20\%5Cdots\%20v\_N\%20\%5Crangle\%0A
" \langle v\_1, v\_2 \dots v\_N \rangle ``)

will \emph{represent} the vector:

{[} v\_1 \textbackslash{}mathbf\{e\}\_1 + v\_2 \textbackslash{}mathbf\{e\}\_2 +
\textbackslash{}ldots + v\_N
\textbackslash{}mathbf\{e\}\_N{]}(https://latex.codecogs.com/png.latex?\%0Av\_1\%20\%5Cmathbf\%7Be\%7D\_1\%20\%2B\%20v\_2\%20\%5Cmathbf\%7Be\%7D\_2\%20\%2B\%20\%5Cldots\%20\%2B\%20v\_N\%20\%5Cmathbf\%7Be\%7D\_N\%0A
" v\_1 \mathbf{e}\_1 + v\_2 \mathbf{e}\_2 + \ldots + v\_N \mathbf{e}\_N ``)

Note that what this encoding represents is \emph{completely dependent} on what
\includegraphics{https://latex.codecogs.com/png.latex?\%5Cmathbf\%7Be\%7D_1\%2C\%20\%5Cmathbf\%7Be\%7D_2\%20\%5Cldots\%20\%5Cmathbf\%7Be\%7D_N}
we pick, and in what order. The basis vectors we pick are arbitrary, and
determine what our encoding looks like.

To highlight this, note that the same vector
\includegraphics{https://latex.codecogs.com/png.latex?\%5Cmathbf\%7Bv\%7D} has
many many different potential encodings --- all you have to do is pick a
different set of basis vectors, or even just re-arrange or re-scale the ones you
already have. However, all of those encodings correspond go the same vector
\includegraphics{https://latex.codecogs.com/png.latex?\%5Cmathbf\%7Bv\%7D}.

One interesting consequence of this is that any N-dimensional vector space whose
scalars are in
\includegraphics{https://latex.codecogs.com/png.latex?\%5Cmathbb\%7BR\%7D} is
actually isomorphic to
\includegraphics{https://latex.codecogs.com/png.latex?\%5Cmathbb\%7BR\%7D\%5EN}
--- the vector space of N-tuples of real numbers. This means that we can
basically treat any N-dimensional vector space with
\includegraphics{https://latex.codecogs.com/png.latex?\%5Cmathbb\%7BR\%7D}
scalars as if it was
\includegraphics{https://latex.codecogs.com/png.latex?\%5Cmathbb\%7BR\%7D\%5EN},
\emph{once we decide} on the basis vectors. Because of this, we often call
\emph{all} N-dimensional vector spaces (whose scalars are in
\includegraphics{https://latex.codecogs.com/png.latex?\%5Cmathbb\%7BR\%7D}) as
\includegraphics{https://latex.codecogs.com/png.latex?\%5Cmathbb\%7BR\%7D\%5EN}.
You will often hear physicists saying that the three-dimensional vector spaces
they use are
\includegraphics{https://latex.codecogs.com/png.latex?\%5Cmathbb\%7BR\%7D\%5E3}.
However, what they really mean is that their vector spaces is \emph{isomorphic}
to
\includegraphics{https://latex.codecogs.com/png.latex?\%5Cmathbb\%7BR\%7D\%5E3}.

\hypertarget{linear-transformations}{%
\section{Linear Transformations}\label{linear-transformations}}

Now, one of the most interesting things in mathematics is the idea of the
\textbf{linear transformation}. Linear transformations are useful to study
because:

\begin{enumerate}
\def\labelenumi{\arabic{enumi}.}
\tightlist
\item
  They are ubiquitous. They come up everywhere in engineering, physics,
  mathematics, data science, economics, and pretty much any mathematical theory.
  And there are even more situations which can be \emph{approximated} by linear
  transformations.
\item
  They are mathematically very nice to work with and study, in practice.
\end{enumerate}

A linear transformation,
\includegraphics{https://latex.codecogs.com/png.latex?f\%28\%5Cmathbf\%7Bx\%7D\%29},
is a function that ``respects'' addition and scaling:

{[} \textbackslash{}begin\{aligned\} f(c\textbackslash{}mathbf\{x\}) \& = c
f(\textbackslash{}mathbf\{x\}) \textbackslash{}\textbackslash{}
f(\textbackslash{}mathbf\{x\} + \textbackslash{}mathbf\{y\}) \& =
f(\textbackslash{}mathbf\{x\}) + f(\textbackslash{}mathbf\{y\})
\textbackslash{}end\{aligned\}{]}(https://latex.codecogs.com/png.latex?\%0A\%5Cbegin\%7Baligned\%7D\%0Af\%28c\%5Cmathbf\%7Bx\%7D\%29\%20\%26\%20\%3D\%20c\%20f\%28\%5Cmathbf\%7Bx\%7D\%29\%20\%5C\%5C\%0Af\%28\%5Cmathbf\%7Bx\%7D\%20\%2B\%20\%5Cmathbf\%7By\%7D\%29\%20\%26\%20\%3D\%20f\%28\%5Cmathbf\%7Bx\%7D\%29\%20\%2B\%20f\%28\%5Cmathbf\%7By\%7D\%29\%0A\%5Cend\%7Baligned\%7D\%0A
"

\begin{aligned}
f(c\mathbf{x}) & = c f(\mathbf{x}) \\
f(\mathbf{x} + \mathbf{y}) & = f(\mathbf{x}) + f(\mathbf{y})
\end{aligned}

``)

This means that if you scale the input, the output is scaled by the same amount.
And also, if you transform the sum of two things, it's the same as the sum of
the transformed things (it ``distributes'').

Note that I snuck in vector notation, because the concept of vectors are
\emph{perfectly suited} for studying linear transformations. That's because
talking about linear transformations requires talking about scaling and adding,
and\ldots{}hey, that's just exactly what vectors have!

From now on, we'll talk about linear transformations specifically on
\emph{N-dimensional vector spaces} (vector spaces that have dimensions and bases
we can use).

\hypertarget{studying-linear-transformations}{%
\subsection{Studying linear
transformations}\label{studying-linear-transformations}}

From first glance, a linear transformation's description doesn't look too useful
or analyzable. All you have is
\includegraphics{https://latex.codecogs.com/png.latex?f\%28\%5Cmathbf\%7Bv\%7D\%29}.
It could be anything! Right? Just a black box function?

But, actually, we can exploit its linearity and the fact that we're in a vector
space with a basis to analyze the heck out of any linear transformation, and see
that all of them actually have to follow some specific pattern.

Let's say that
\includegraphics{https://latex.codecogs.com/png.latex?A\%28\%5Cmathbf\%7Bx\%7D\%29}
is a linear transformation from N-dimensional vector space
\includegraphics{https://latex.codecogs.com/png.latex?V} to M-dimensional vector
space \includegraphics{https://latex.codecogs.com/png.latex?U}. That is,
\includegraphics{https://latex.codecogs.com/png.latex?A\%20\%3A\%20V\%20\%5Crightarrow\%20U}.

Because we know that, once we pick a set of basis vectors
\includegraphics{https://latex.codecogs.com/png.latex?\%5Cmathbf\%7Be\%7D_i},
any vector
\includegraphics{https://latex.codecogs.com/png.latex?\%5Cmathbf\%7Bv\%7D} in
\includegraphics{https://latex.codecogs.com/png.latex?V} can be decomposed as
\includegraphics{https://latex.codecogs.com/png.latex?v_1\%20\%5Cmathbf\%7Be\%7D_1\%20\%2B\%20v_2\%20\%5Cmathbf\%7Be\%7D_2\%20\%2B\%20\%5Cldots\%20v_n\%20\%5Cmathbf\%7Be\%7D_N},
we really can just look at how a transformation
\includegraphics{https://latex.codecogs.com/png.latex?A} acts on this
decomposition. For example, if
\includegraphics{https://latex.codecogs.com/png.latex?V} is three-dimensional:

{[} A(\textbackslash{}mathbf\{v\}) = A(v\_1 \textbackslash{}mathbf\{e\}\_1 +
v\_2 \textbackslash{}mathbf\{e\}\_2 + v\_3
\textbackslash{}mathbf\{e\}\_3){]}(https://latex.codecogs.com/png.latex?\%0AA\%28\%5Cmathbf\%7Bv\%7D\%29\%20\%3D\%20A\%28v\_1\%20\%5Cmathbf\%7Be\%7D\_1\%20\%2B\%20v\_2\%20\%5Cmathbf\%7Be\%7D\_2\%20\%2B\%20v\_3\%20\%5Cmathbf\%7Be\%7D\_3\%29\%0A
" A(\mathbf{v}) = A(v\_1 \mathbf{e}\_1 + v\_2 \mathbf{e}\_2 + v\_3
\mathbf{e}\_3) ``)

Hm. Doesn't seem very insightful, does it?

\hypertarget{a-simple-definition}{%
\subsection{A simple definition}\label{a-simple-definition}}

But! We can exploit the linearity of
\includegraphics{https://latex.codecogs.com/png.latex?A} (that it distributes
and scales) to rewrite that as:

{[} A(\textbackslash{}mathbf\{v\}) = v\_1 A(\textbackslash{}mathbf\{e\}\_1) +
v\_2 A(\textbackslash{}mathbf\{e\}\_2) + v\_3
A(\textbackslash{}mathbf\{e\}\_3){]}(https://latex.codecogs.com/png.latex?\%0AA\%28\%5Cmathbf\%7Bv\%7D\%29\%20\%3D\%20v\_1\%20A\%28\%5Cmathbf\%7Be\%7D\_1\%29\%20\%2B\%20v\_2\%20A\%28\%5Cmathbf\%7Be\%7D\_2\%29\%20\%2B\%20v\_3\%20A\%28\%5Cmathbf\%7Be\%7D\_3\%29\%0A
" A(\mathbf{v}) = v\_1 A(\mathbf{e}\_1) + v\_2 A(\mathbf{e}\_2) + v\_3
A(\mathbf{e}\_3) ``)

Okay, take a moment to pause and take that all in. This is actually a pretty big
deal! This just means that, to study
\includegraphics{https://latex.codecogs.com/png.latex?A}, \textbf{all you need
to study} is how \includegraphics{https://latex.codecogs.com/png.latex?A} acts
on our \emph{basis vectors}. If you know how
\includegraphics{https://latex.codecogs.com/png.latex?A} acts on our basis
vectors of our vector space, that's really ``all there is'' about
\includegraphics{https://latex.codecogs.com/png.latex?A}! Not such a black box
anymore!

That is, if I were to ask you, ``Hey, what is
\includegraphics{https://latex.codecogs.com/png.latex?A} like?'', \emph{all
you'd have to tell me} is the result of
\includegraphics{https://latex.codecogs.com/png.latex?A\%28\%5Cmathbf\%7Be\%7D_1\%29},
\includegraphics{https://latex.codecogs.com/png.latex?A\%28\%5Cmathbf\%7Be\%7D_2},
and
\includegraphics{https://latex.codecogs.com/png.latex?A\%28\%5Cmathbf\%7Be\%7D_3\%29}.
Just give me those three \emph{vectors}, and we \emph{uniquely determine
\includegraphics{https://latex.codecogs.com/png.latex?A}}.

To put in another way, \emph{any linear transformation} from a three-dimensional
vector space is uniquely characterized by \emph{three vectors}:
\includegraphics{https://latex.codecogs.com/png.latex?A\%28\%5Cmathbf\%7Be\%7D_1\%29},
\includegraphics{https://latex.codecogs.com/png.latex?A\%28\%5Cmathbf\%7Be\%7D_2\%29},
and
\includegraphics{https://latex.codecogs.com/png.latex?A\%28\%5Cmathbf\%7Be\%7D_3\%29}.

Those three vectors \emph{completely define}
\includegraphics{https://latex.codecogs.com/png.latex?A}.

In general, we see that \emph{any linear transformation} from an N-dimensional
vector space can be \emph{completely defined} by N vectors: the N results of
that transformation on each of N basis vectors we choose.

\hypertarget{enter-the-matrix}{%
\subsection{Enter the Matrix}\label{enter-the-matrix}}

Okay, so how do we ``give''/define/state those N vectors?

Well, recall that the result of
\includegraphics{https://latex.codecogs.com/png.latex?A\%28\%5Cmathbf\%7Bv\%7D\%29}
and
\includegraphics{https://latex.codecogs.com/png.latex?A\%28\%5Cmathbf\%7Be\%7D_1\%29},
etc. are \emph{themselves} vectors, in M-dimensional vector space
\includegraphics{https://latex.codecogs.com/png.latex?U}. Let's say that
\includegraphics{https://latex.codecogs.com/png.latex?U} is 2-dimensional, for
now.

This means that any vector
\includegraphics{https://latex.codecogs.com/png.latex?\%5Cmathbf\%7Bu\%7D} in
\includegraphics{https://latex.codecogs.com/png.latex?U} can be represented as
\includegraphics{https://latex.codecogs.com/png.latex?u_1\%20\%5Cmathbf\%7Bq\%7D_1\%20\%2B\%20u_2\%20\%5Cmathbf\%7Bq\%7D_2},
where
\includegraphics{https://latex.codecogs.com/png.latex?\%5Cmathbf\%7Bq\%7D_1} and
\includegraphics{https://latex.codecogs.com/png.latex?\%5Cmathbf\%7Bq\%7D_2} is
an arbitrary choice of basis vectors.

This means that
\includegraphics{https://latex.codecogs.com/png.latex?A\%28\%5Cmathbf\%7Be\%7D_1\%29}
etc. can also all be represented in terms of these basis vectors. So, laying it
all out:

{[} \textbackslash{}begin\{aligned\} A(\textbackslash{}mathbf\{e\}\_1) \& =
a\_\{11\} \textbackslash{}mathbf\{q\}\_1 + a\_\{21\}
\textbackslash{}mathbf\{q\}\_2 \textbackslash{}\textbackslash{}
A(\textbackslash{}mathbf\{e\}\_2) \& = a\_\{12\} \textbackslash{}mathbf\{q\}\_1
+ a\_\{22\} \textbackslash{}mathbf\{q\}\_2 \textbackslash{}\textbackslash{}
A(\textbackslash{}mathbf\{e\}\_3) \& = a\_\{13\} \textbackslash{}mathbf\{q\}\_1
+ a\_\{23\} \textbackslash{}mathbf\{q\}\_2
\textbackslash{}end\{aligned\}{]}(https://latex.codecogs.com/png.latex?\%0A\%5Cbegin\%7Baligned\%7D\%0AA\%28\%5Cmathbf\%7Be\%7D\_1\%29\%20\%26\%20\%3D\%20a\_\%7B11\%7D\%20\%5Cmathbf\%7Bq\%7D\_1\%20\%2B\%20a\_\%7B21\%7D\%20\%5Cmathbf\%7Bq\%7D\_2\%20\%5C\%5C\%0AA\%28\%5Cmathbf\%7Be\%7D\_2\%29\%20\%26\%20\%3D\%20a\_\%7B12\%7D\%20\%5Cmathbf\%7Bq\%7D\_1\%20\%2B\%20a\_\%7B22\%7D\%20\%5Cmathbf\%7Bq\%7D\_2\%20\%5C\%5C\%0AA\%28\%5Cmathbf\%7Be\%7D\_3\%29\%20\%26\%20\%3D\%20a\_\%7B13\%7D\%20\%5Cmathbf\%7Bq\%7D\_1\%20\%2B\%20a\_\%7B23\%7D\%20\%5Cmathbf\%7Bq\%7D\_2\%0A\%5Cend\%7Baligned\%7D\%0A
" \textbackslash{}begin\{aligned\} A(\mathbf{e}\emph{1) \& = a}\{11\}
\mathbf{q}\emph{1 + a}\{21\} \mathbf{q}\_2 \textbackslash{} A(\mathbf{e}\emph{2)
\& = a}\{12\} \mathbf{q}\emph{1 + a}\{22\} \mathbf{q}\_2 \textbackslash{}
A(\mathbf{e}\emph{3) \& = a}\{13\} \mathbf{q}\emph{1 + a}\{23\} \mathbf{q}\_2
\textbackslash{}end\{aligned\} ``)

Or, to use our bracket notation from before:

{[} \textbackslash{}begin\{aligned\} A(\textbackslash{}mathbf\{e\}\_1) \& =
\textbackslash{}langle a\_\{11\}, a\_\{21\} \textbackslash{}rangle
\textbackslash{}\textbackslash{} A(\textbackslash{}mathbf\{e\}\_2) \& =
\textbackslash{}langle a\_\{12\}, a\_\{22\} \textbackslash{}rangle
\textbackslash{}\textbackslash{} A(\textbackslash{}mathbf\{e\}\_3) \& =
\textbackslash{}langle a\_\{13\}, a\_\{23\} \textbackslash{}rangle
\textbackslash{}end\{aligned\}{]}(https://latex.codecogs.com/png.latex?\%0A\%5Cbegin\%7Baligned\%7D\%0AA\%28\%5Cmathbf\%7Be\%7D\_1\%29\%20\%26\%20\%3D\%20\%5Clangle\%20a\_\%7B11\%7D\%2C\%20a\_\%7B21\%7D\%20\%5Crangle\%20\%5C\%5C\%0AA\%28\%5Cmathbf\%7Be\%7D\_2\%29\%20\%26\%20\%3D\%20\%5Clangle\%20a\_\%7B12\%7D\%2C\%20a\_\%7B22\%7D\%20\%5Crangle\%20\%5C\%5C\%0AA\%28\%5Cmathbf\%7Be\%7D\_3\%29\%20\%26\%20\%3D\%20\%5Clangle\%20a\_\%7B13\%7D\%2C\%20a\_\%7B23\%7D\%20\%5Crangle\%0A\%5Cend\%7Baligned\%7D\%0A
" \textbackslash{}begin\{aligned\} A(\mathbf{e}\emph{1) \& = \langle a}\{11\},
a\_\{21\} \rangle \textbackslash{} A(\mathbf{e}\emph{2) \& = \langle a}\{12\},
a\_\{22\} \rangle \textbackslash{} A(\mathbf{e}\emph{3) \& = \langle a}\{13\},
a\_\{23\} \rangle \textbackslash{}end\{aligned\} ``)

So, we now see two facts:

\begin{enumerate}
\def\labelenumi{\arabic{enumi}.}
\tightlist
\item
  A linear transformation from an N dimensional vector space to an M dimensional
  vector space can be \emph{defined} using N vectors.
\item
  Each of those N vectors can, themselves, be defined using M scalars each.
\end{enumerate}

Our final conclusion: \emph{any} linear transformation from an N dimensional
vector space to an M dimensional vector space can be defined using
\includegraphics{https://latex.codecogs.com/png.latex?N\%20M} scalars.

That's right -- \emph{all} possible linear transformations from a 3-dimensional
vector space to a 2-dimensional are parameterized by only \emph{six} scalars!
These six scalars uniquely determine and define our linear transformation, given
a set of basis vectors that we agree on. All linear transformations
\includegraphics{https://latex.codecogs.com/png.latex?\%5Cmathbb\%7BR\%7D\%5E3\%20\%5Crightarrow\%20\%5Cmathbb\%7BR\%7D\%5E2}
can be defined/encoded/expressed with just six real numbers.

These six numbers are pretty important. Just like how we often talk about
3-dimensional vectors in terms of the encoding of their three coefficients, we
often talk about linear transformations from 3-d space to 2-d space in terms of
their six defining coefficients.

We group these things up in something called a \emph{matrix}.

If our linear transformation
\includegraphics{https://latex.codecogs.com/png.latex?A} from a 3-dimensional
vector space to a 2-dimensional vector space is defined by:

{[} \textbackslash{}begin\{aligned\} A(\textbackslash{}mathbf\{e\}\_1) \& =
a\_\{11\} \textbackslash{}mathbf\{q\}\_1 + a\_\{21\}
\textbackslash{}mathbf\{q\}\_2 \textbackslash{}\textbackslash{}
A(\textbackslash{}mathbf\{e\}\_2) \& = a\_\{12\} \textbackslash{}mathbf\{q\}\_1
+ a\_\{22\} \textbackslash{}mathbf\{q\}\_2 \textbackslash{}\textbackslash{}
A(\textbackslash{}mathbf\{e\}\_3) \& = a\_\{13\} \textbackslash{}mathbf\{q\}\_1
+ a\_\{23\} \textbackslash{}mathbf\{q\}\_2
\textbackslash{}end\{aligned\}{]}(https://latex.codecogs.com/png.latex?\%0A\%5Cbegin\%7Baligned\%7D\%0AA\%28\%5Cmathbf\%7Be\%7D\_1\%29\%20\%26\%20\%3D\%20a\_\%7B11\%7D\%20\%5Cmathbf\%7Bq\%7D\_1\%20\%2B\%20a\_\%7B21\%7D\%20\%5Cmathbf\%7Bq\%7D\_2\%20\%5C\%5C\%0AA\%28\%5Cmathbf\%7Be\%7D\_2\%29\%20\%26\%20\%3D\%20a\_\%7B12\%7D\%20\%5Cmathbf\%7Bq\%7D\_1\%20\%2B\%20a\_\%7B22\%7D\%20\%5Cmathbf\%7Bq\%7D\_2\%20\%5C\%5C\%0AA\%28\%5Cmathbf\%7Be\%7D\_3\%29\%20\%26\%20\%3D\%20a\_\%7B13\%7D\%20\%5Cmathbf\%7Bq\%7D\_1\%20\%2B\%20a\_\%7B23\%7D\%20\%5Cmathbf\%7Bq\%7D\_2\%0A\%5Cend\%7Baligned\%7D\%0A
" \textbackslash{}begin\{aligned\} A(\mathbf{e}\emph{1) \& = a}\{11\}
\mathbf{q}\emph{1 + a}\{21\} \mathbf{q}\_2 \textbackslash{} A(\mathbf{e}\emph{2)
\& = a}\{12\} \mathbf{q}\emph{1 + a}\{22\} \mathbf{q}\_2 \textbackslash{}
A(\mathbf{e}\emph{3) \& = a}\{13\} \mathbf{q}\emph{1 + a}\{23\} \mathbf{q}\_2
\textbackslash{}end\{aligned\} ``)

(for arbitrary choice of bases
\includegraphics{https://latex.codecogs.com/png.latex?\%5Cmathbf\%7Be\%7D_i} and
\includegraphics{https://latex.codecogs.com/png.latex?\%5Cmathbf\%7Bq\%7D_i})

We ``encode'' it as the matrix:

{[} \textbackslash{}begin\{bmatrix\} a\_\{11\} \& a\_\{12\} \& a\_\{13\}
\textbackslash{}\textbackslash{} a\_\{21\} \& a\_\{22\} \& a\_\{23\}
\textbackslash{}end\{bmatrix\}{]}(https://latex.codecogs.com/png.latex?\%0A\%5Cbegin\%7Bbmatrix\%7D\%0Aa\_\%7B11\%7D\%20\%26\%20a\_\%7B12\%7D\%20\%26\%20a\_\%7B13\%7D\%20\%5C\%5C\%0Aa\_\%7B21\%7D\%20\%26\%20a\_\%7B22\%7D\%20\%26\%20a\_\%7B23\%7D\%0A\%5Cend\%7Bbmatrix\%7D\%0A
" \textbackslash{}begin\{bmatrix\} a\_\{11\} \& a\_\{12\} \& a\_\{13\}
\textbackslash{} a\_\{21\} \& a\_\{22\} \& a\_\{23\}
\textbackslash{}end\{bmatrix\} ``)

And that's why we use matrices in linear algebra -- like how
\includegraphics{https://latex.codecogs.com/png.latex?\%5Clangle\%20x\%2C\%20y\%2C\%20z\%20\%5Crangle}
is a convenient way to represent and define a \emph{vector} (once we agree on a
bases), a
\includegraphics{https://latex.codecogs.com/png.latex?M\%20\%5Ctimes\%20N}
matrix is a convenient way to represent and define a \emph{linear
transformation} from an N-dimensional vector space to a M-dimensional vector
space (once we agree on the bases in both spaces).

\hypertarget{matrix-operations}{%
\section{Matrix Operations}\label{matrix-operations}}

In this light, we can understand the definition of the common matrix operations.

\hypertarget{matrix-vector-multiplication}{%
\subsection{Matrix-Vector Multiplication}\label{matrix-vector-multiplication}}

Matrix-vector multiplication is essentially the \emph{decoding} of the linear
transformation that the matrix represents.

Let's look at the
\includegraphics{https://latex.codecogs.com/png.latex?2\%20\%5Ctimes\%203}
example. Recall that we had:

{[} f(\textbackslash{}mathbf\{v\}) = v\_1 f(\textbackslash{}mathbf\{e\}\_1) +
v\_2 f(\textbackslash{}mathbf\{e\}\_2) + v\_3
f(\textbackslash{}mathbf\{e\}\_3){]}(https://latex.codecogs.com/png.latex?\%0Af\%28\%5Cmathbf\%7Bv\%7D\%29\%20\%3D\%20v\_1\%20f\%28\%5Cmathbf\%7Be\%7D\_1\%29\%20\%2B\%20v\_2\%20f\%28\%5Cmathbf\%7Be\%7D\_2\%29\%20\%2B\%20v\_3\%20f\%28\%5Cmathbf\%7Be\%7D\_3\%29\%0A
" f(\mathbf{v}) = v\_1 f(\mathbf{e}\_1) + v\_2 f(\mathbf{e}\_2) + v\_3
f(\mathbf{e}\_3) ``)

And we say that \includegraphics{https://latex.codecogs.com/png.latex?A} is
completely defined by:

{[} \textbackslash{}begin\{aligned\} f(\textbackslash{}mathbf\{e\}\_1) \& =
a\_\{11\} \textbackslash{}mathbf\{q\}\_1 + a\_\{21\}
\textbackslash{}mathbf\{q\}\_2 \textbackslash{}\textbackslash{}
f(\textbackslash{}mathbf\{e\}\_2) \& = a\_\{12\} \textbackslash{}mathbf\{q\}\_1
+ a\_\{22\} \textbackslash{}mathbf\{q\}\_2 \textbackslash{}\textbackslash{}
f(\textbackslash{}mathbf\{e\}\_3) \& = a\_\{13\} \textbackslash{}mathbf\{q\}\_1
+ a\_\{23\} \textbackslash{}mathbf\{q\}\_2
\textbackslash{}end\{aligned\}{]}(https://latex.codecogs.com/png.latex?\%0A\%5Cbegin\%7Baligned\%7D\%0Af\%28\%5Cmathbf\%7Be\%7D\_1\%29\%20\%26\%20\%3D\%20a\_\%7B11\%7D\%20\%5Cmathbf\%7Bq\%7D\_1\%20\%2B\%20a\_\%7B21\%7D\%20\%5Cmathbf\%7Bq\%7D\_2\%20\%5C\%5C\%0Af\%28\%5Cmathbf\%7Be\%7D\_2\%29\%20\%26\%20\%3D\%20a\_\%7B12\%7D\%20\%5Cmathbf\%7Bq\%7D\_1\%20\%2B\%20a\_\%7B22\%7D\%20\%5Cmathbf\%7Bq\%7D\_2\%20\%5C\%5C\%0Af\%28\%5Cmathbf\%7Be\%7D\_3\%29\%20\%26\%20\%3D\%20a\_\%7B13\%7D\%20\%5Cmathbf\%7Bq\%7D\_1\%20\%2B\%20a\_\%7B23\%7D\%20\%5Cmathbf\%7Bq\%7D\_2\%0A\%5Cend\%7Baligned\%7D\%0A
" \textbackslash{}begin\{aligned\} f(\mathbf{e}\emph{1) \& = a}\{11\}
\mathbf{q}\emph{1 + a}\{21\} \mathbf{q}\_2 \textbackslash{} f(\mathbf{e}\emph{2)
\& = a}\{12\} \mathbf{q}\emph{1 + a}\{22\} \mathbf{q}\_2 \textbackslash{}
f(\mathbf{e}\emph{3) \& = a}\{13\} \mathbf{q}\emph{1 + a}\{23\} \mathbf{q}\_2
\textbackslash{}end\{aligned\} ``)

This means that:

{[} \textbackslash{}begin\{aligned\} f(\textbackslash{}mathbf\{v\}) \& = v\_1
(a\_\{11\} \textbackslash{}mathbf\{q\}\_1 + a\_\{21\}
\textbackslash{}mathbf\{q\}\_2) \textbackslash{}\textbackslash{} \& + v\_2
(a\_\{12\} \textbackslash{}mathbf\{q\}\_1 + a\_\{22\}
\textbackslash{}mathbf\{q\}\_2) \textbackslash{}\textbackslash{} \& + v\_3
(a\_\{13\} \textbackslash{}mathbf\{q\}\_1 + a\_\{23\}
\textbackslash{}mathbf\{q\}\_2)
\textbackslash{}end\{aligned\}{]}(https://latex.codecogs.com/png.latex?\%0A\%5Cbegin\%7Baligned\%7D\%0Af\%28\%5Cmathbf\%7Bv\%7D\%29\%20\%26\%20\%3D\%20v\_1\%20\%28a\_\%7B11\%7D\%20\%5Cmathbf\%7Bq\%7D\_1\%20\%2B\%20a\_\%7B21\%7D\%20\%5Cmathbf\%7Bq\%7D\_2\%29\%20\%5C\%5C\%0A\%20\%20\%20\%20\%20\%20\%20\%20\%20\%20\%20\%20\%20\%20\%26\%20\%2B\%20v\_2\%20\%28a\_\%7B12\%7D\%20\%5Cmathbf\%7Bq\%7D\_1\%20\%2B\%20a\_\%7B22\%7D\%20\%5Cmathbf\%7Bq\%7D\_2\%29\%20\%5C\%5C\%0A\%20\%20\%20\%20\%20\%20\%20\%20\%20\%20\%20\%20\%20\%20\%26\%20\%2B\%20v\_3\%20\%28a\_\%7B13\%7D\%20\%5Cmathbf\%7Bq\%7D\_1\%20\%2B\%20a\_\%7B23\%7D\%20\%5Cmathbf\%7Bq\%7D\_2\%29\%0A\%5Cend\%7Baligned\%7D\%0A
" \textbackslash{}begin\{aligned\} f(\mathbf{v}) \& = v\_1 (a\_\{11\}
\mathbf{q}\emph{1 + a}\{21\} \mathbf{q}\emph{2) \textbackslash{} \& + v\_2
(a}\{12\} \mathbf{q}\emph{1 + a}\{22\} \mathbf{q}\emph{2) \textbackslash{} \& +
v\_3 (a}\{13\} \mathbf{q}\emph{1 + a}\{23\} \mathbf{q}\_2)
\textbackslash{}end\{aligned\} ``)

Which is itself a vector in
\includegraphics{https://latex.codecogs.com/png.latex?U}, so let's write this as
a combination of its components
\includegraphics{https://latex.codecogs.com/png.latex?\%5Cmathbf\%7Bq\%7D_1} and
\includegraphics{https://latex.codecogs.com/png.latex?\%5Cmathbf\%7Bq\%7D_2}, by
distributing and rearranging terms:

{[} \textbackslash{}begin\{aligned\} f(\textbackslash{}mathbf\{v\}) \& = (v\_1
a\_\{11\} + v\_2 a\_\{12\} + v\_3 a\_\{13\}) \textbackslash{}mathbf\{q\}\_1
\textbackslash{}\textbackslash{} \& + (v\_1 a\_\{21\} + v\_2 a\_\{22\} + v\_3
a\_\{23\}) \textbackslash{}mathbf\{q\}\_2
\textbackslash{}end\{aligned\}{]}(https://latex.codecogs.com/png.latex?\%0A\%5Cbegin\%7Baligned\%7D\%0Af\%28\%5Cmathbf\%7Bv\%7D\%29\%20\%26\%20\%3D\%20\%28v\_1\%20a\_\%7B11\%7D\%20\%2B\%20v\_2\%20a\_\%7B12\%7D\%20\%2B\%20v\_3\%20a\_\%7B13\%7D\%29\%20\%5Cmathbf\%7Bq\%7D\_1\%20\%5C\%5C\%0A\%20\%20\%20\%20\%20\%20\%20\%20\%20\%20\%20\%20\%20\%20\%26\%20\%2B\%20\%28v\_1\%20a\_\%7B21\%7D\%20\%2B\%20v\_2\%20a\_\%7B22\%7D\%20\%2B\%20v\_3\%20a\_\%7B23\%7D\%29\%20\%5Cmathbf\%7Bq\%7D\_2\%0A\%5Cend\%7Baligned\%7D\%0A
" \textbackslash{}begin\{aligned\} f(\mathbf{v}) \& = (v\_1 a\_\{11\} + v\_2
a\_\{12\} + v\_3 a\_\{13\}) \mathbf{q}\emph{1 \textbackslash{} \& + (v\_1
a}\{21\} + v\_2 a\_\{22\} + v\_3 a\_\{23\}) \mathbf{q}\_2
\textbackslash{}end\{aligned\} ``)

And this is exactly the formula for matrix-vector multiplication!

{[} \textbackslash{}begin\{bmatrix\} a\_\{11\} \& a\_\{12\} \& a\_\{13\}
\textbackslash{}\textbackslash{} a\_\{21\} \& a\_\{22\} \& a\_\{23\}
\textbackslash{}end\{bmatrix\} \textbackslash{}begin\{bmatrix\} v\_1
\textbackslash{}\textbackslash{} v\_2 \textbackslash{}\textbackslash{} v\_3
\textbackslash{}end\{bmatrix\} = \textbackslash{}begin\{bmatrix\} v\_1 a\_\{11\}
+ v\_2 a\_\{12\} + v\_3 a\_\{13\} \textbackslash{}\textbackslash{} v\_2
a\_\{21\} + v\_2 a\_\{22\} + v\_3 a\_\{23\}
\textbackslash{}end\{bmatrix\}{]}(https://latex.codecogs.com/png.latex?\%0A\%5Cbegin\%7Bbmatrix\%7D\%0Aa\_\%7B11\%7D\%20\%26\%20a\_\%7B12\%7D\%20\%26\%20a\_\%7B13\%7D\%20\%5C\%5C\%0Aa\_\%7B21\%7D\%20\%26\%20a\_\%7B22\%7D\%20\%26\%20a\_\%7B23\%7D\%0A\%5Cend\%7Bbmatrix\%7D\%0A\%5Cbegin\%7Bbmatrix\%7D\%0Av\_1\%20\%5C\%5C\%0Av\_2\%20\%5C\%5C\%0Av\_3\%0A\%5Cend\%7Bbmatrix\%7D\%0A\%3D\%0A\%5Cbegin\%7Bbmatrix\%7D\%0Av\_1\%20a\_\%7B11\%7D\%20\%2B\%20v\_2\%20a\_\%7B12\%7D\%20\%2B\%20v\_3\%20a\_\%7B13\%7D\%20\%5C\%5C\%0Av\_2\%20a\_\%7B21\%7D\%20\%2B\%20v\_2\%20a\_\%7B22\%7D\%20\%2B\%20v\_3\%20a\_\%7B23\%7D\%0A\%5Cend\%7Bbmatrix\%7D\%0A
" \textbackslash{}begin\{bmatrix\} a\_\{11\} \& a\_\{12\} \& a\_\{13\}
\textbackslash{} a\_\{21\} \& a\_\{22\} \& a\_\{23\}
\textbackslash{}end\{bmatrix\} \textbackslash{}begin\{bmatrix\} v\_1
\textbackslash{} v\_2 \textbackslash{} v\_3 \textbackslash{}end\{bmatrix\} =
\textbackslash{}begin\{bmatrix\} v\_1 a\_\{11\} + v\_2 a\_\{12\} + v\_3
a\_\{13\} \textbackslash{} v\_2 a\_\{21\} + v\_2 a\_\{22\} + v\_3 a\_\{23\}
\textbackslash{}end\{bmatrix\} ``)

Again, remember that what we are doing is manipulating \emph{specific encodings}
of our vectors and our linear transformations. Namely, we encode linear
transformations as matrices, and vectors in their component encoding. The reason
we can do these is that we agree upon a set of bases for our source and target
vector spaces, and express these encodings in terms of those.

\hypertarget{addition-of-linear-transformations}{%
\subsection{Addition of linear
transformations}\label{addition-of-linear-transformations}}

One neat thing about linear transformation is that they ``add'' well -- you can
add them together by simply applying them both and adding the results. The
result is another linear transformation.

{[} (f + g)(\textbackslash{}mathbf\{x\}) \textbackslash{}equiv
f(\textbackslash{}mathbf\{x\}) +
g(\textbackslash{}mathbf\{x\}){]}(https://latex.codecogs.com/png.latex?\%0A\%28f\%20\%2B\%20g\%29\%28\%5Cmathbf\%7Bx\%7D\%29\%20\%5Cequiv\%20f\%28\%5Cmathbf\%7Bx\%7D\%29\%20\%2B\%20g\%28\%5Cmathbf\%7Bx\%7D\%29\%0A
" (f + g)(\mathbf{x}) \equiv f(\mathbf{x}) + g(\mathbf{x}) ``)

If
\includegraphics{https://latex.codecogs.com/png.latex?f\%20\%3A\%20V\%20\%5Crightarrow\%20U}
and
\includegraphics{https://latex.codecogs.com/png.latex?g\%20\%3A\%20V\%20\%5Crightarrow\%20U}
are linear transformations between the \emph{same} vector spaces, then
\includegraphics{https://latex.codecogs.com/png.latex?f\%20\%2B\%20g\%20\%3A\%20V\%20\%5Crightarrow\%20U},
as we defined it, is also one:

{[} \textbackslash{}begin\{aligned\} (f + g)(c \textbackslash{}mathbf\{x\}) \& =
f(c \textbackslash{}mathbf\{x\}) + g(c \textbackslash{}mathbf\{x\})
\textbackslash{}\textbackslash{} \& = c f(\textbackslash{}mathbf\{x\}) + c
g(\textbackslash{}mathbf\{x\}) \textbackslash{}\textbackslash{} \& = c (
f(\textbackslash{}mathbf\{x\}) + g(\textbackslash{}mathbf\{x\}) )
\textbackslash{}\textbackslash{} (f + g)(c \textbackslash{}mathbf\{x\}) \& = c
(f + g)(\textbackslash{}mathbf\{x\})
\textbackslash{}end\{aligned\}{]}(https://latex.codecogs.com/png.latex?\%0A\%5Cbegin\%7Baligned\%7D\%0A\%28f\%20\%2B\%20g\%29\%28c\%20\%5Cmathbf\%7Bx\%7D\%29\%20\%26\%20\%3D\%20f\%28c\%20\%5Cmathbf\%7Bx\%7D\%29\%20\%2B\%20g\%28c\%20\%5Cmathbf\%7Bx\%7D\%29\%20\%5C\%5C\%0A\%20\%20\%20\%20\%20\%20\%20\%20\%20\%20\%20\%20\%20\%20\%20\%20\%20\%20\%20\%20\%20\%20\%26\%20\%3D\%20c\%20f\%28\%5Cmathbf\%7Bx\%7D\%29\%20\%2B\%20c\%20g\%28\%5Cmathbf\%7Bx\%7D\%29\%20\%5C\%5C\%0A\%20\%20\%20\%20\%20\%20\%20\%20\%20\%20\%20\%20\%20\%20\%20\%20\%20\%20\%20\%20\%20\%20\%26\%20\%3D\%20c\%20\%28\%20f\%28\%5Cmathbf\%7Bx\%7D\%29\%20\%2B\%20g\%28\%5Cmathbf\%7Bx\%7D\%29\%20\%29\%20\%5C\%5C\%0A\%28f\%20\%2B\%20g\%29\%28c\%20\%5Cmathbf\%7Bx\%7D\%29\%20\%26\%20\%3D\%20c\%20\%28f\%20\%2B\%20g\%29\%28\%5Cmathbf\%7Bx\%7D\%29\%0A\%5Cend\%7Baligned\%7D\%0A
"

\begin{aligned}
(f + g)(c \mathbf{x}) & = f(c \mathbf{x}) + g(c \mathbf{x}) \\
                      & = c f(\mathbf{x}) + c g(\mathbf{x}) \\
                      & = c ( f(\mathbf{x}) + g(\mathbf{x}) ) \\
(f + g)(c \mathbf{x}) & = c (f + g)(\mathbf{x})
\end{aligned}

``)

(Showing that it respects addition is something you can look at if you want to
have some fun!)

So, if \includegraphics{https://latex.codecogs.com/png.latex?f} is encoded as
matrix \includegraphics{https://latex.codecogs.com/png.latex?\%5Chat\%7BA\%7D}
for given bases, and \includegraphics{https://latex.codecogs.com/png.latex?g} is
encoded as matrix
\includegraphics{https://latex.codecogs.com/png.latex?\%5Chat\%7BB\%7D}, what is
the encoding of
\includegraphics{https://latex.codecogs.com/png.latex?f\%20\%2B\%20g} ?

Let's say that, if \includegraphics{https://latex.codecogs.com/png.latex?V} and
\includegraphics{https://latex.codecogs.com/png.latex?U} are 3-dimensional and
2-dimensional, respectively:

{[} \textbackslash{}begin\{aligned\} f(\textbackslash{}mathbf\{v\}) \& = (v\_1
a\_\{11\} + v\_2 a\_\{12\} + v\_3 a\_\{13\}) \textbackslash{}mathbf\{q\}\_1
\textbackslash{}\textbackslash{} \& + (v\_1 a\_\{21\} + v\_2 a\_\{22\} + v\_3
a\_\{23\}) \textbackslash{}mathbf\{q\}\_2 \textbackslash{}\textbackslash{}
g(\textbackslash{}mathbf\{v\}) \& = (v\_1 b\_\{11\} + v\_2 b\_\{12\} + v\_3
b\_\{13\}) \textbackslash{}mathbf\{q\}\_1 \textbackslash{}\textbackslash{} \& +
(v\_1 b\_\{21\} + v\_2 b\_\{22\} + v\_3 b\_\{23\})
\textbackslash{}mathbf\{q\}\_2
\textbackslash{}end\{aligned\}{]}(https://latex.codecogs.com/png.latex?\%0A\%5Cbegin\%7Baligned\%7D\%0Af\%28\%5Cmathbf\%7Bv\%7D\%29\%20\%26\%20\%3D\%20\%28v\_1\%20a\_\%7B11\%7D\%20\%2B\%20v\_2\%20a\_\%7B12\%7D\%20\%2B\%20v\_3\%20a\_\%7B13\%7D\%29\%20\%5Cmathbf\%7Bq\%7D\_1\%20\%5C\%5C\%0A\%20\%20\%20\%20\%20\%20\%20\%20\%20\%20\%20\%20\%20\%20\%26\%20\%2B\%20\%28v\_1\%20a\_\%7B21\%7D\%20\%2B\%20v\_2\%20a\_\%7B22\%7D\%20\%2B\%20v\_3\%20a\_\%7B23\%7D\%29\%20\%5Cmathbf\%7Bq\%7D\_2\%20\%5C\%5C\%0Ag\%28\%5Cmathbf\%7Bv\%7D\%29\%20\%26\%20\%3D\%20\%28v\_1\%20b\_\%7B11\%7D\%20\%2B\%20v\_2\%20b\_\%7B12\%7D\%20\%2B\%20v\_3\%20b\_\%7B13\%7D\%29\%20\%5Cmathbf\%7Bq\%7D\_1\%20\%5C\%5C\%0A\%20\%20\%20\%20\%20\%20\%20\%20\%20\%20\%20\%20\%20\%20\%26\%20\%2B\%20\%28v\_1\%20b\_\%7B21\%7D\%20\%2B\%20v\_2\%20b\_\%7B22\%7D\%20\%2B\%20v\_3\%20b\_\%7B23\%7D\%29\%20\%5Cmathbf\%7Bq\%7D\_2\%0A\%5Cend\%7Baligned\%7D\%0A
" \textbackslash{}begin\{aligned\} f(\mathbf{v}) \& = (v\_1 a\_\{11\} + v\_2
a\_\{12\} + v\_3 a\_\{13\}) \mathbf{q}\emph{1 \textbackslash{} \& + (v\_1
a}\{21\} + v\_2 a\_\{22\} + v\_3 a\_\{23\}) \mathbf{q}\emph{2 \textbackslash{}
g(\mathbf{v}) \& = (v\_1 b}\{11\} + v\_2 b\_\{12\} + v\_3 b\_\{13\})
\mathbf{q}\emph{1 \textbackslash{} \& + (v\_1 b}\{21\} + v\_2 b\_\{22\} + v\_3
b\_\{23\}) \mathbf{q}\_2 \textbackslash{}end\{aligned\} ``)

Then the breakdown of
\includegraphics{https://latex.codecogs.com/png.latex?f\%20\%2B\%20g} is:

{[} \textbackslash{}begin\{aligned\} (f + g)(\textbackslash{}mathbf\{v\}) \& =
(v\_1 a\_\{11\} + v\_2 a\_\{12\} + v\_3 a\_\{13\})
\textbackslash{}mathbf\{q\}\_1 \textbackslash{}\textbackslash{} \& + (v\_1
a\_\{21\} + v\_2 a\_\{22\} + v\_3 a\_\{23\}) \textbackslash{}mathbf\{q\}\_2
\textbackslash{}\textbackslash{} \& + (v\_1 b\_\{11\} + v\_2 b\_\{12\} + v\_3
b\_\{13\}) \textbackslash{}mathbf\{q\}\_1 \textbackslash{}\textbackslash{} \& +
(v\_1 b\_\{21\} + v\_2 b\_\{22\} + v\_3 b\_\{23\})
\textbackslash{}mathbf\{q\}\_2 \textbackslash{}\textbackslash{} \& = (v\_1
(a\_\{11\} + b\_\{11\}) + v\_2 (a\_\{12\} + b\_\{12\}) + v\_3 (a\_\{13\} +
b\_\{13\})) \textbackslash{}mathbf\{q\} \textbackslash{}\textbackslash{} \& +
(v\_1 (a\_\{21\} + b\_\{21\}) + v\_2 (a\_\{22\} + b\_\{22\}) + v\_3 (a\_\{23\} +
b\_\{23\})) \textbackslash{}mathbf\{q\}
\textbackslash{}end\{aligned\}{]}(https://latex.codecogs.com/png.latex?\%0A\%5Cbegin\%7Baligned\%7D\%0A\%28f\%20\%2B\%20g\%29\%28\%5Cmathbf\%7Bv\%7D\%29\%20\%26\%20\%3D\%20\%28v\_1\%20a\_\%7B11\%7D\%20\%2B\%20v\_2\%20a\_\%7B12\%7D\%20\%2B\%20v\_3\%20a\_\%7B13\%7D\%29\%20\%5Cmathbf\%7Bq\%7D\_1\%20\%5C\%5C\%0A\%20\%20\%20\%20\%20\%20\%20\%20\%20\%20\%20\%20\%20\%20\%20\%20\%20\%20\%20\%20\%26\%20\%2B\%20\%28v\_1\%20a\_\%7B21\%7D\%20\%2B\%20v\_2\%20a\_\%7B22\%7D\%20\%2B\%20v\_3\%20a\_\%7B23\%7D\%29\%20\%5Cmathbf\%7Bq\%7D\_2\%20\%5C\%5C\%0A\%20\%20\%20\%20\%20\%20\%20\%20\%20\%20\%20\%20\%20\%20\%20\%20\%20\%20\%20\%20\%26\%20\%2B\%20\%28v\_1\%20b\_\%7B11\%7D\%20\%2B\%20v\_2\%20b\_\%7B12\%7D\%20\%2B\%20v\_3\%20b\_\%7B13\%7D\%29\%20\%5Cmathbf\%7Bq\%7D\_1\%20\%5C\%5C\%0A\%20\%20\%20\%20\%20\%20\%20\%20\%20\%20\%20\%20\%20\%20\%20\%20\%20\%20\%20\%20\%26\%20\%2B\%20\%28v\_1\%20b\_\%7B21\%7D\%20\%2B\%20v\_2\%20b\_\%7B22\%7D\%20\%2B\%20v\_3\%20b\_\%7B23\%7D\%29\%20\%5Cmathbf\%7Bq\%7D\_2\%20\%5C\%5C\%0A\%20\%20\%20\%20\%20\%20\%20\%20\%20\%20\%20\%20\%20\%20\%20\%20\%20\%20\%20\%20\%26\%20\%3D\%20\%28v\_1\%20\%28a\_\%7B11\%7D\%20\%2B\%20b\_\%7B11\%7D\%29\%20\%2B\%20v\_2\%20\%28a\_\%7B12\%7D\%20\%2B\%20b\_\%7B12\%7D\%29\%20\%2B\%20v\_3\%20\%28a\_\%7B13\%7D\%20\%2B\%20b\_\%7B13\%7D\%29\%29\%20\%5Cmathbf\%7Bq\%7D\%20\%5C\%5C\%0A\%20\%20\%20\%20\%20\%20\%20\%20\%20\%20\%20\%20\%20\%20\%20\%20\%20\%20\%20\%20\%26\%20\%2B\%20\%28v\_1\%20\%28a\_\%7B21\%7D\%20\%2B\%20b\_\%7B21\%7D\%29\%20\%2B\%20v\_2\%20\%28a\_\%7B22\%7D\%20\%2B\%20b\_\%7B22\%7D\%29\%20\%2B\%20v\_3\%20\%28a\_\%7B23\%7D\%20\%2B\%20b\_\%7B23\%7D\%29\%29\%20\%5Cmathbf\%7Bq\%7D\%0A\%5Cend\%7Baligned\%7D\%0A
" \textbackslash{}begin\{aligned\} (f + g)(\mathbf{v}) \& = (v\_1 a\_\{11\} +
v\_2 a\_\{12\} + v\_3 a\_\{13\}) \mathbf{q}\emph{1 \textbackslash{} \& + (v\_1
a}\{21\} + v\_2 a\_\{22\} + v\_3 a\_\{23\}) \mathbf{q}\emph{2 \textbackslash{}
\& + (v\_1 b}\{11\} + v\_2 b\_\{12\} + v\_3 b\_\{13\}) \mathbf{q}\emph{1
\textbackslash{} \& + (v\_1 b}\{21\} + v\_2 b\_\{22\} + v\_3 b\_\{23\})
\mathbf{q}\emph{2 \textbackslash{} \& = (v\_1 (a}\{11\} + b\_\{11\}) + v\_2
(a\_\{12\} + b\_\{12\}) + v\_3 (a\_\{13\} + b\_\{13\})) \mathbf{q}
\textbackslash{} \& + (v\_1 (a\_\{21\} + b\_\{21\}) + v\_2 (a\_\{22\} +
b\_\{22\}) + v\_3 (a\_\{23\} + b\_\{23\})) \mathbf{q}
\textbackslash{}end\{aligned\} ``)

Note that if we say that
\includegraphics{https://latex.codecogs.com/png.latex?f\%20\%2B\%20g} is encoded
as matrix
\includegraphics{https://latex.codecogs.com/png.latex?\%5Chat\%7BC\%7D}, and
call the components
\includegraphics{https://latex.codecogs.com/png.latex?c_\%7B11\%7D},
\includegraphics{https://latex.codecogs.com/png.latex?c_\%7B12\%7D}, etc., then
we can rewrite that as:

{[} \textbackslash{}begin\{aligned\} (f + g)(\textbackslash{}mathbf\{v\}) \& =
(v\_1 c\_\{11\} + v\_2 c\_\{12\} + v\_3 c\_\{13\})
\textbackslash{}mathbf\{q\}\_1 \textbackslash{}\textbackslash{} \& + (v\_1
c\_\{21\} + v\_2 c\_\{22\} + v\_3 c\_\{23\}) \textbackslash{}mathbf\{q\}\_2
\textbackslash{}end\{aligned\}{]}(https://latex.codecogs.com/png.latex?\%0A\%5Cbegin\%7Baligned\%7D\%0A\%28f\%20\%2B\%20g\%29\%28\%5Cmathbf\%7Bv\%7D\%29\%20\%26\%20\%3D\%20\%28v\_1\%20c\_\%7B11\%7D\%20\%2B\%20v\_2\%20c\_\%7B12\%7D\%20\%2B\%20v\_3\%20c\_\%7B13\%7D\%29\%20\%5Cmathbf\%7Bq\%7D\_1\%20\%5C\%5C\%0A\%20\%20\%20\%20\%20\%20\%20\%20\%20\%20\%20\%20\%20\%20\%20\%20\%20\%20\%20\%20\%26\%20\%2B\%20\%28v\_1\%20c\_\%7B21\%7D\%20\%2B\%20v\_2\%20c\_\%7B22\%7D\%20\%2B\%20v\_3\%20c\_\%7B23\%7D\%29\%20\%5Cmathbf\%7Bq\%7D\_2\%0A\%5Cend\%7Baligned\%7D\%0A
" \textbackslash{}begin\{aligned\} (f + g)(\mathbf{v}) \& = (v\_1 c\_\{11\} +
v\_2 c\_\{12\} + v\_3 c\_\{13\}) \mathbf{q}\emph{1 \textbackslash{} \& + (v\_1
c}\{21\} + v\_2 c\_\{22\} + v\_3 c\_\{23\}) \mathbf{q}\_2
\textbackslash{}end\{aligned\} ``)

Where
\includegraphics{https://latex.codecogs.com/png.latex?c_\%7B11\%7D\%20\%3D\%20a_\%7B11\%7D\%20\%2B\%20b_\%7B11\%7D},
\includegraphics{https://latex.codecogs.com/png.latex?c_\%7B12\%7D\%20\%3D\%20a_\%7B12\%7D\%20\%2B\%20b_\%7B13\%7D},
etc.

So, if \includegraphics{https://latex.codecogs.com/png.latex?\%5Chat\%7BA\%7D}
and \includegraphics{https://latex.codecogs.com/png.latex?\%5Chat\%7BB\%7D}
encode linear transformations
\includegraphics{https://latex.codecogs.com/png.latex?f} and
\includegraphics{https://latex.codecogs.com/png.latex?g}, then we can encode
\includegraphics{https://latex.codecogs.com/png.latex?f\%20\%2B\%20g} as matrix
\includegraphics{https://latex.codecogs.com/png.latex?\%5Chat\%7BC\%7D}, where
the components of
\includegraphics{https://latex.codecogs.com/png.latex?\%5Chat\%7BC\%7D} are just
the sum of their corresponding components in
\includegraphics{https://latex.codecogs.com/png.latex?\%5Chat\%7BA\%7D} and
\includegraphics{https://latex.codecogs.com/png.latex?\%5Chat\%7BB\%7D}.

And that's why we define
\includegraphics{https://latex.codecogs.com/png.latex?\%5Chat\%7BA\%7D\%20\%2B\%20\%5Chat\%7BB\%7D},
matrix-matrix addition, as component-wise addition: component-wise addition
perfectly ``simulates'' the addition of the linear transformation!

What's happening here is we can represent manipulations of the functions
themselves by manipulating \emph{their encodings}.

\hypertarget{multiplication-of-linear-transformations}{%
\subsection{Multiplication of linear
transformations}\label{multiplication-of-linear-transformations}}

We might be tempted to define \emph{multiplication} of linear transformations
the same way. However, this doesn't quite make sense.

Remember that we talked about adding linear transformations as the addition of
their results. However, we can't talk about multiplying linear transformations
as the multiplication of their results because the idea of a vector space
doesn't come with any notion of multiplication.

However, even if we talk specifically about linear transformations to
\emph{scalars}, this still doesn't quite work:

{[} \textbackslash{}begin\{aligned\} (f * g)(c \textbackslash{}mathbf\{x\}) \& =
f(c \textbackslash{}mathbf\{x\}) * g(c \textbackslash{}mathbf\{x\})
\textbackslash{}\textbackslash{} \& = c f(\textbackslash{}mathbf\{x\}) * c
g(\textbackslash{}mathbf\{x\}) \textbackslash{}\textbackslash{} \& = c\^{}2 (
f(\textbackslash{}mathbf\{x\}) * g(\textbackslash{}mathbf\{x\}) )
\textbackslash{}\textbackslash{} (f * g)(c \textbackslash{}mathbf\{x\}) \& =
c\^{}2 (f * g)(\textbackslash{}mathbf\{x\})
\textbackslash{}end\{aligned\}{]}(https://latex.codecogs.com/png.latex?\%0A\%5Cbegin\%7Baligned\%7D\%0A\%28f\%20\%2A\%20g\%29\%28c\%20\%5Cmathbf\%7Bx\%7D\%29\%20\%26\%20\%3D\%20f\%28c\%20\%5Cmathbf\%7Bx\%7D\%29\%20\%2A\%20g\%28c\%20\%5Cmathbf\%7Bx\%7D\%29\%20\%5C\%5C\%0A\%20\%20\%20\%20\%20\%20\%20\%20\%20\%20\%20\%20\%20\%20\%20\%20\%20\%20\%20\%20\%20\%20\%26\%20\%3D\%20c\%20f\%28\%5Cmathbf\%7Bx\%7D\%29\%20\%2A\%20c\%20g\%28\%5Cmathbf\%7Bx\%7D\%29\%20\%5C\%5C\%0A\%20\%20\%20\%20\%20\%20\%20\%20\%20\%20\%20\%20\%20\%20\%20\%20\%20\%20\%20\%20\%20\%20\%26\%20\%3D\%20c\%5E2\%20\%28\%20f\%28\%5Cmathbf\%7Bx\%7D\%29\%20\%2A\%20g\%28\%5Cmathbf\%7Bx\%7D\%29\%20\%29\%20\%5C\%5C\%0A\%28f\%20\%2A\%20g\%29\%28c\%20\%5Cmathbf\%7Bx\%7D\%29\%20\%26\%20\%3D\%20c\%5E2\%20\%28f\%20\%2A\%20g\%29\%28\%5Cmathbf\%7Bx\%7D\%29\%0A\%5Cend\%7Baligned\%7D\%0A
"

\begin{aligned}
(f * g)(c \mathbf{x}) & = f(c \mathbf{x}) * g(c \mathbf{x}) \\
                      & = c f(\mathbf{x}) * c g(\mathbf{x}) \\
                      & = c^2 ( f(\mathbf{x}) * g(\mathbf{x}) ) \\
(f * g)(c \mathbf{x}) & = c^2 (f * g)(\mathbf{x})
\end{aligned}

``)

That's right,
\includegraphics{https://latex.codecogs.com/png.latex?f\%20\%2A\%20g}, defined
point-wise, does not create a linear transformation.

So, \emph{there is no matrix} that could would even represent or encode
\includegraphics{https://latex.codecogs.com/png.latex?f\%20\%2A\%20g}, as we
defined it. So, since
\includegraphics{https://latex.codecogs.com/png.latex?f\%20\%2A\%20g} isn't even
representable as a matrix in our encoding scheme, it doesn't make sense to treat
it as a matrix operation.

\hypertarget{composition-of-linear-transformations}{%
\subsection{Composition of linear
transformations}\label{composition-of-linear-transformations}}

Since linear transformations are functions, we can compose them:

{[} (f \textbackslash{}circ g)(\textbackslash{}mathbf\{x\})
\textbackslash{}equiv
f(g(\textbackslash{}mathbf\{x\})){]}(https://latex.codecogs.com/png.latex?\%0A\%28f\%20\%5Ccirc\%20g\%29\%28\%5Cmathbf\%7Bx\%7D\%29\%20\%5Cequiv\%20f\%28g\%28\%5Cmathbf\%7Bx\%7D\%29\%29\%0A
" (f \circ g)(\mathbf{x}) \equiv f(g(\mathbf{x})) ``)

Is the composition of linear transformations also a linear transformation?

{[} \textbackslash{}begin\{aligned\} (f \textbackslash{}circ g)(c
\textbackslash{}mathbf\{x\}) \& = f(g(c \textbackslash{}mathbf\{x\}))
\textbackslash{}\textbackslash{} \& = f(c g(\textbackslash{}mathbf\{x\}))
\textbackslash{}\textbackslash{} \& = c f(g(\textbackslash{}mathbf\{x\}))
\textbackslash{}\textbackslash{} (f \textbackslash{}circ g)(c
\textbackslash{}mathbf\{x\}) \& = c (f \textbackslash{}circ
g)(\textbackslash{}mathbf\{x\})
\textbackslash{}end\{aligned\}{]}(https://latex.codecogs.com/png.latex?\%0A\%5Cbegin\%7Baligned\%7D\%0A\%28f\%20\%5Ccirc\%20g\%29\%28c\%20\%5Cmathbf\%7Bx\%7D\%29\%20\%26\%20\%3D\%20f\%28g\%28c\%20\%5Cmathbf\%7Bx\%7D\%29\%29\%20\%5C\%5C\%0A\%20\%20\%20\%20\%20\%20\%20\%20\%20\%20\%20\%20\%20\%20\%20\%20\%20\%20\%20\%20\%20\%20\%26\%20\%3D\%20f\%28c\%20g\%28\%5Cmathbf\%7Bx\%7D\%29\%29\%20\%5C\%5C\%0A\%20\%20\%20\%20\%20\%20\%20\%20\%20\%20\%20\%20\%20\%20\%20\%20\%20\%20\%20\%20\%20\%20\%26\%20\%3D\%20c\%20f\%28g\%28\%5Cmathbf\%7Bx\%7D\%29\%29\%20\%5C\%5C\%0A\%28f\%20\%5Ccirc\%20g\%29\%28c\%20\%5Cmathbf\%7Bx\%7D\%29\%20\%26\%20\%3D\%20c\%20\%28f\%20\%5Ccirc\%20g\%29\%28\%5Cmathbf\%7Bx\%7D\%29\%0A\%5Cend\%7Baligned\%7D\%0A
"

\begin{aligned}
(f \circ g)(c \mathbf{x}) & = f(g(c \mathbf{x})) \\
                      & = f(c g(\mathbf{x})) \\
                      & = c f(g(\mathbf{x})) \\
(f \circ g)(c \mathbf{x}) & = c (f \circ g)(\mathbf{x})
\end{aligned}

``)

Yes! (Well, once you prove that it respects addition. I'll leave the fun to
you!)

Okay, so we know that
\includegraphics{https://latex.codecogs.com/png.latex?f\%20\%5Ccirc\%20g} is
indeed a linear transformation. That means that it can \emph{also} be encoded as
a matrix.

So, let's say that
\includegraphics{https://latex.codecogs.com/png.latex?f\%20\%3A\%20U\%20\%5Crightarrow\%20W},
then
\includegraphics{https://latex.codecogs.com/png.latex?g\%20\%3A\%20V\%20\%5Crightarrow\%20U}.
\includegraphics{https://latex.codecogs.com/png.latex?f} is a linear
transformation from \includegraphics{https://latex.codecogs.com/png.latex?U} to
\includegraphics{https://latex.codecogs.com/png.latex?W}, and
\includegraphics{https://latex.codecogs.com/png.latex?g} is a linear
transformation from \includegraphics{https://latex.codecogs.com/png.latex?V} to
\includegraphics{https://latex.codecogs.com/png.latex?U}. That means that
\includegraphics{https://latex.codecogs.com/png.latex?f\%20\%5Ccirc\%20g\%20\%3A\%20V\%20\%5Crightarrow\%20W}
is a linear transformation from
\includegraphics{https://latex.codecogs.com/png.latex?V} to
\includegraphics{https://latex.codecogs.com/png.latex?W}.

Let's say that \includegraphics{https://latex.codecogs.com/png.latex?V} is
3-dimensional, \includegraphics{https://latex.codecogs.com/png.latex?U} is
2-dimensional, and \includegraphics{https://latex.codecogs.com/png.latex?W} is
4-dimensional.

If \includegraphics{https://latex.codecogs.com/png.latex?f} is encoded by the
\includegraphics{https://latex.codecogs.com/png.latex?4\%20\%5Ctimes\%202}
matrix \includegraphics{https://latex.codecogs.com/png.latex?\%5Chat\%7BA\%7D},
and \includegraphics{https://latex.codecogs.com/png.latex?g} is encoded by
\includegraphics{https://latex.codecogs.com/png.latex?2\%20\%5Ctimes\%203}
matrix \includegraphics{https://latex.codecogs.com/png.latex?\%5Chat\%7BB\%7D},
then we can represent
\includegraphics{https://latex.codecogs.com/png.latex?f\%20\%5Ccirc\%20g} as the
\includegraphics{https://latex.codecogs.com/png.latex?4\%20\%5Ctimes\%203}
matrix \includegraphics{https://latex.codecogs.com/png.latex?\%5Chat\%7BC\%7D}.

If you've taken a linear algebra class, you might recognize this pattern.
Combining a
\includegraphics{https://latex.codecogs.com/png.latex?4\%20\%5Ctimes\%202} and a
\includegraphics{https://latex.codecogs.com/png.latex?2\%20\%5Ctimes\%203} to
make a
\includegraphics{https://latex.codecogs.com/png.latex?4\%20\%5Ctimes\%203} ?

We \emph{can} compute
\includegraphics{https://latex.codecogs.com/png.latex?\%5Chat\%7BC\%7D} using
only the encodings
\includegraphics{https://latex.codecogs.com/png.latex?\%5Chat\%7BA\%7D} and
\includegraphics{https://latex.codecogs.com/png.latex?\%5Chat\%7BB\%7D}! We call
this \textbf{matrix multiplication}. It's typically denoted as
\includegraphics{https://latex.codecogs.com/png.latex?\%5Chat\%7BC\%7D\%20\%3D\%20\%5Chat\%7BA\%7D\%20\%5Chat\%7BB\%7D}.

That's exactly what \emph{matrix multiplication} is defined as. If:

\begin{itemize}
\tightlist
\item
  \includegraphics{https://latex.codecogs.com/png.latex?\%5Chat\%7BA\%7D} is a
  \includegraphics{https://latex.codecogs.com/png.latex?O\%20\%5Ctimes\%20M}
  matrix representing a linear transformation from a M-dimensional space to an
  O-dimensional space
\item
  \hat{B} is an
  \includegraphics{https://latex.codecogs.com/png.latex?M\%20\%5Ctimes\%20N}
  matrix representing a linear transformation from an N-dimensional space to an
  M-dimensional space
\end{itemize}

Then
\includegraphics{https://latex.codecogs.com/png.latex?\%5Chat\%7BC\%7D\%20\%3D\%20\%5Chat\%7BA\%7D\%5Chat\%7BB\%7D}
is a \includegraphics{https://latex.codecogs.com/png.latex?O\%20\%5Ctimes\%20N}
matrix representing a linear transformation from an N-dimensional space to an
O-dimensional space.

\end{document}
