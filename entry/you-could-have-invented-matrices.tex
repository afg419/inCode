\documentclass[]{article}
\usepackage{lmodern}
\usepackage{amssymb,amsmath}
\usepackage{ifxetex,ifluatex}
\usepackage{fixltx2e} % provides \textsubscript
\ifnum 0\ifxetex 1\fi\ifluatex 1\fi=0 % if pdftex
  \usepackage[T1]{fontenc}
  \usepackage[utf8]{inputenc}
\else % if luatex or xelatex
  \ifxetex
    \usepackage{mathspec}
    \usepackage{xltxtra,xunicode}
  \else
    \usepackage{fontspec}
  \fi
  \defaultfontfeatures{Mapping=tex-text,Scale=MatchLowercase}
  \newcommand{\euro}{€}
\fi
% use upquote if available, for straight quotes in verbatim environments
\IfFileExists{upquote.sty}{\usepackage{upquote}}{}
% use microtype if available
\IfFileExists{microtype.sty}{\usepackage{microtype}}{}
\usepackage[margin=1in]{geometry}
\usepackage{graphicx}
\makeatletter
\def\maxwidth{\ifdim\Gin@nat@width>\linewidth\linewidth\else\Gin@nat@width\fi}
\def\maxheight{\ifdim\Gin@nat@height>\textheight\textheight\else\Gin@nat@height\fi}
\makeatother
% Scale images if necessary, so that they will not overflow the page
% margins by default, and it is still possible to overwrite the defaults
% using explicit options in \includegraphics[width, height, ...]{}
\setkeys{Gin}{width=\maxwidth,height=\maxheight,keepaspectratio}
\ifxetex
  \usepackage[setpagesize=false, % page size defined by xetex
              unicode=false, % unicode breaks when used with xetex
              xetex]{hyperref}
\else
  \usepackage[unicode=true]{hyperref}
\fi
\hypersetup{breaklinks=true,
            bookmarks=true,
            pdfauthor={Justin Le},
            pdftitle={You Could Have Invented Matrices!},
            colorlinks=true,
            citecolor=blue,
            urlcolor=blue,
            linkcolor=magenta,
            pdfborder={0 0 0}}
\urlstyle{same}  % don't use monospace font for urls
% Make links footnotes instead of hotlinks:
\renewcommand{\href}[2]{#2\footnote{\url{#1}}}
\setlength{\parindent}{0pt}
\setlength{\parskip}{6pt plus 2pt minus 1pt}
\setlength{\emergencystretch}{3em}  % prevent overfull lines
\setcounter{secnumdepth}{0}

\title{You Could Have Invented Matrices!}
\author{Justin Le}

\begin{document}
\maketitle

\emph{Originally posted on
\textbf{\href{https://blog.jle.im/entry/you-could-have-invented-matrices.html}{in
Code}}.}

You could have invented matrices!

Let's talk about vectors. A \textbf{vector} (denoted as
\includegraphics{https://latex.codecogs.com/png.latex?\%5Cmathbf\%7Bv\%7D}, a
lower-case bold italicized letter) is an element in a \textbf{vector space},
which means that it can be ``scaled'', like
\includegraphics{https://latex.codecogs.com/png.latex?c\%20\%5Cmathbf\%7Bv\%7D}
(the \includegraphics{https://latex.codecogs.com/png.latex?c} is called a
``scalar'' --- creative name, right?) and added, like
\includegraphics{https://latex.codecogs.com/png.latex?\%5Cmathbf\%7Bv\%7D\%20\%2B\%20\%5Cmathbf\%7Bu\%7D}.

In order for vector spaces and their operations to be valid, they just have to
obey some common-sense rules (like associativity, commutativity, distributivity,
etc.) that allow us to make meaningful conclusions.\footnote{In short, vector
  spaces form an Abelian group (which is another way of just saying that
  addition is commutative, associative, has an identity, and an inverse), and
  scalars have to play nice with addition
  (\includegraphics{https://latex.codecogs.com/png.latex?c\%28\%5Cmathbf\%7Bv\%7D\%20\%2B\%20\%5Cmathbf\%7Bu\%7D\%29\%20\%3D\%20c\%20\%5Cmathbf\%7Bv\%7D\%20\%2B\%20c\%20\%5Cmathbf\%7Bu\%7D},
  and
  \includegraphics{https://latex.codecogs.com/png.latex?\%28c\%20\%2B\%20d\%29\%5Cmathbf\%7Bv\%7D\%20\%3D\%20c\%20\%5Cmathbf\%7Bv\%7D\%20\%2B\%20d\%20\%5Cmathbf\%7Bv\%7D}).
  Also, scalars themselves form a field.}

\hypertarget{dimensionality}{%
\section{Dimensionality}\label{dimensionality}}

One neat thing about vector spaces is that \emph{some} of them (if you're lucky)
have a notion of \textbf{dimensionality}. We say that a vector space is
N-dimensional if there exists N ``basis'' vectors
\includegraphics{https://latex.codecogs.com/png.latex?\%5Cmathbf\%7Be\%7D_1\%2C\%20\%5Cmathbf\%7Be\%7D_2\%20\%5Cldots\%20\%5Cmathbf\%7Be\%7D_N}
where \emph{any} vector can be described as scaled sums of all of them, and that
N is the lowest number of basis vectors needed. For example, if a vector space
is 3-dimensional, then it means that \emph{any} vector
\includegraphics{https://latex.codecogs.com/png.latex?\%5Cmathbf\%7Bv\%7D} in
that space can be broken down as:

{[} \textbackslash{}mathbf\{v\} = a \textbackslash{}mathbf\{e\}\_1 + b
\textbackslash{}mathbf\{e\}\_2 + c
\textbackslash{}mathbf\{e\}\_3{]}(https://latex.codecogs.com/png.latex?\%0A\%5Cmathbf\%7Bv\%7D\%20\%3D\%20a\%20\%5Cmathbf\%7Be\%7D\_1\%20\%2B\%20b\%20\%5Cmathbf\%7Be\%7D\_2\%20\%2B\%20c\%20\%5Cmathbf\%7Be\%7D\_3\%0A
" \mathbf{v} = a \mathbf{e}\_1 + b \mathbf{e}\_2 + c \mathbf{e}\_3 ``)

Where \includegraphics{https://latex.codecogs.com/png.latex?a},
\includegraphics{https://latex.codecogs.com/png.latex?b}, and
\includegraphics{https://latex.codecogs.com/png.latex?c} are scalars.

Dimensionality is really a statement about being able to decompose any vector in
that vector space into a useful set of bases. For a 3-dimensional vector space,
you need at least 3 vectors to make a bases that can reproduce \emph{any} vector
in your space.

In physics, we often treat reality as taking place in a three-dimensional vector
space. The basis vectors are often called
\includegraphics{https://latex.codecogs.com/png.latex?\%5Chat\%7B\%5Cmathbf\%7Bi\%7D\%7D},
\includegraphics{https://latex.codecogs.com/png.latex?\%5Chat\%7B\%5Cmathbf\%7Bj\%7D\%7D},
and
\includegraphics{https://latex.codecogs.com/png.latex?\%5Chat\%7B\%5Cmathbf\%7Bk\%7D\%7D},
and so we say that we can describe our 3D physics vectors as
\includegraphics{https://latex.codecogs.com/png.latex?\%5Cmathbf\%7Bv\%7D\%20\%3D\%20v_x\%20\%5Chat\%7B\%5Cmathbf\%7Bi\%7D\%7D\%20\%2B\%20v_y\%20\%5Chat\%7B\%5Cmathbf\%7Bj\%7D\%7D\%20\%2B\%20v_x\%20\%5Chat\%7B\%5Cmathbf\%7Bk\%7D\%7D}

\hypertarget{encoding}{%
\subsection{Encoding}\label{encoding}}

One neat thing that physicists take advantage of all the time is that if we
\emph{agree} on a set of basis vectors and a specific ordering, we can actually
\emph{encode} any vector
\includegraphics{https://latex.codecogs.com/png.latex?\%5Cmathbf\%7Bv\%7D} in
terms of those basis vectors.

So in physics, we can say ``Let's encode vectors in terms of
\includegraphics{https://latex.codecogs.com/png.latex?\%5Chat\%7B\%5Cmathbf\%7Bi\%7D\%7D},
\includegraphics{https://latex.codecogs.com/png.latex?\%5Chat\%7B\%5Cmathbf\%7Bj\%7D\%7D},
and
\includegraphics{https://latex.codecogs.com/png.latex?\%5Chat\%7B\%5Cmathbf\%7Bk\%7D\%7D},
in that order.'' Then, we can \emph{write}
\includegraphics{https://latex.codecogs.com/png.latex?\%5Cmathbf\%7Bv\%7D} as
\includegraphics{https://latex.codecogs.com/png.latex?\%5Clangle\%20v_x\%2C\%20v_y\%2C\%20v_z\%20\%5Crangle},
and understand that we really
mean\includegraphics{https://latex.codecogs.com/png.latex?\%5Cmathbf\%7Bv\%7D\%20\%3D\%20v_x\%20\%5Chat\%7B\%5Cmathbf\%7Bi\%7D\%7D\%20\%2B\%20v_y\%20\%5Chat\%7B\%5Cmathbf\%7Bj\%7D\%7D\%20\%2B\%20v_x\%20\%5Chat\%7B\%5Cmathbf\%7Bk\%7D\%7D}.

Note that
\includegraphics{https://latex.codecogs.com/png.latex?\%5Clangle\%20v_x\%2C\%20v_y\%2C\%20v_z\%20\%5Crangle}
is \textbf{not} the same thing as the \textbf{vector}
\includegraphics{https://latex.codecogs.com/png.latex?\%5Cmathbf\%7Bv\%7D}. It
is \emph{an encoding} of that vector, that only makes sense once we choose to
\emph{agree} on a specific set of basis.

For an N-dimensional vector space, it means that, with a minimum of N items, we
can represent any vector in that space. And, if we agree on those N items, we
can devise an encoding, such that:

{[} \textbackslash{}langle v\_1, v\_2 \textbackslash{}dots v\_N
\textbackslash{}rangle{]}(https://latex.codecogs.com/png.latex?\%0A\%5Clangle\%20v\_1\%2C\%20v\_2\%20\%5Cdots\%20v\_N\%20\%5Crangle\%0A
" \langle v\_1, v\_2 \dots v\_N \rangle ``)

will \emph{represent} the vector:

{[} v\_1 \textbackslash{}mathbf\{e\}\_1 + v\_2 \textbackslash{}mathbf\{e\}\_2 +
\textbackslash{}ldots + v\_N
\textbackslash{}mathbf\{e\}\_N{]}(https://latex.codecogs.com/png.latex?\%0Av\_1\%20\%5Cmathbf\%7Be\%7D\_1\%20\%2B\%20v\_2\%20\%5Cmathbf\%7Be\%7D\_2\%20\%2B\%20\%5Cldots\%20\%2B\%20v\_N\%20\%5Cmathbf\%7Be\%7D\_N\%0A
" v\_1 \mathbf{e}\_1 + v\_2 \mathbf{e}\_2 + \ldots + v\_N \mathbf{e}\_N ``)

Note that what this encoding represents is \emph{completely dependent} on what
\includegraphics{https://latex.codecogs.com/png.latex?\%5Cmathbf\%7Be\%7D_1\%2C\%20\%5Cmathbf\%7Be\%7D_2\%20\%5Cldots\%20\%5Cmathbf\%7Be\%7D_N}
we pick, and in what order. The basis vectors we pick are arbitrary, and
determine what our encoding looks like.

To highlight this, note that the same vector
\includegraphics{https://latex.codecogs.com/png.latex?\%5Cmathbf\%7Bv\%7D} has
many different potential encodings --- all you have to do is pick a different
set of basis vectors, or even just re-arrange the ones you already have.
However, all of those encodings correspond go the same vector
\includegraphics{https://latex.codecogs.com/png.latex?\%5Cmathbf\%7Bv\%7D}.

One interesting consequence of this is that any N-dimensional vector space whose
scalars are in
\includegraphics{https://latex.codecogs.com/png.latex?\%5Cmathbb\%7BR\%7D} is
actually isomorphic to
\includegraphics{https://latex.codecogs.com/png.latex?\%5Cmathbf\%7BR\%7D\%5EN}
--- the vector space of N-tuples of real numbers. Because of this, we often call
\emph{all} N-dimensional vector spaces (whose scalars are in
\includegraphics{https://latex.codecogs.com/png.latex?\%5Cmathbb\%7BR\%7D}) as
\includegraphics{https://latex.codecogs.com/png.latex?\%5Cmathbb\%7BR\%7D\%5EN}.
You will often hear physicists saying that the three-dimensional vector spaces
they use are
\includegraphics{https://latex.codecogs.com/png.latex?\%5Cmathbb\%7BR\%7D\%5E3}.
However, what they really mean is that their vectors are isomorphic to
\includegraphics{https://latex.codecogs.com/png.latex?\%5Cmathbb\%7BR\%7D\%5E3}.

\hypertarget{linear-transformations}{%
\section{Linear Transformations}\label{linear-transformations}}

Now, one of the most interesting things in mathematics is the idea of the
\textbf{linear transformation}. Linear transformations are useful to study
because:

\begin{enumerate}
\def\labelenumi{\arabic{enumi}.}
\tightlist
\item
  They are ubiquitious. They come up everywhere in engineering, physics,
  mathematics, data science, economics, and pretty much any mathematical theory.
  And there are even more situations which can be \emph{approximated} by linear
  transformations.
\item
  They are mathematically very nice to work with and study, in practice.
\end{enumerate}

A linear transformation,
\includegraphics{https://latex.codecogs.com/png.latex?f\%28\%5Cmathbf\%7Bx\%7D\%29},
is a function that ``respects'' addition and scaling:

{[} \textbackslash{}begin\{aligned\} f(c\textbackslash{}mathbf\{x\}) \& = c
f(\textbackslash{}mathbf\{x\}) \textbackslash{}\textbackslash{}
f(\textbackslash{}mathbf\{x\} + \textbackslash{}mathbf\{y\}) \& =
f(\textbackslash{}mathbf\{x\}) + f(\textbackslash{}mathbf\{y\})
\textbackslash{}end\{aligned\}{]}(https://latex.codecogs.com/png.latex?\%0A\%5Cbegin\%7Baligned\%7D\%0Af\%28c\%5Cmathbf\%7Bx\%7D\%29\%20\%26\%20\%3D\%20c\%20f\%28\%5Cmathbf\%7Bx\%7D\%29\%20\%5C\%5C\%0Af\%28\%5Cmathbf\%7Bx\%7D\%20\%2B\%20\%5Cmathbf\%7By\%7D\%29\%20\%26\%20\%3D\%20f\%28\%5Cmathbf\%7Bx\%7D\%29\%20\%2B\%20f\%28\%5Cmathbf\%7By\%7D\%29\%0A\%5Cend\%7Baligned\%7D\%0A
"

\begin{aligned}
f(c\mathbf{x}) & = c f(\mathbf{x}) \\
f(\mathbf{x} + \mathbf{y}) & = f(\mathbf{x}) + f(\mathbf{y})
\end{aligned}

``)

This means that if you scale the input, the output is scaled by the same amount.
And also, if you transform the sum of two things, it's the same as the sum of
the transformed things (it ``distributes'').

Note that I snuck in vector notation, because the concept of vectors are
\emph{perfectly suited} for studying linear transformations. That's because
talking about linear transformations requires talking about scaling and adding,
and\ldots{}hey, that's just exactly what vectors have!

From now on, we'll talk about linear transformations specifically on
\emph{N-dimensional vector spaces} (vector spaces that have dimensions and bases
we can use).

\hypertarget{studying-linear-transformations}{%
\subsection{Studying linear
transformations}\label{studying-linear-transformations}}

From first glance, a linear transformation's description doesn't look too useful
or analyzable. All you have is
\includegraphics{https://latex.codecogs.com/png.latex?f\%28\%5Cmathbf\%7Bv\%7D\%29}.
It could be anything! Right? Just a black box function?

But, actually, we can exploit its linearity and the fact that we're in a vector
space with a basis to analyze the heck out of any linear transformation, and see
that all of them actually have to follow some specific pattern.

Let's say that
\includegraphics{https://latex.codecogs.com/png.latex?A\%28\%5Cmathbf\%7Bx\%7D\%29}
is a linear transformation from N-dimensional vector space
\includegraphics{https://latex.codecogs.com/png.latex?V} to M-dimensional vector
space \includegraphics{https://latex.codecogs.com/png.latex?U}. That is,
\includegraphics{https://latex.codecogs.com/png.latex?A\%20\%3A\%20V\%20\%5Crightarrow\%20U}.

Because we know that any vector
\includegraphics{https://latex.codecogs.com/png.latex?\%5Cmathbf\%7Bv\%7D} in
\includegraphics{https://latex.codecogs.com/png.latex?V} can be decomposed as
\includegraphics{https://latex.codecogs.com/png.latex?v_1\%20\%5Cmathbf\%7Be\%7D_1\%20\%2B\%20v_2\%20\%5Cmathbf\%7Be\%7D_2\%20\%2B\%20\%5Cldots\%20v_n\%20\%5Cmathbf\%7Be\%7D_N},
we really can just look at how a transformation
\includegraphics{https://latex.codecogs.com/png.latex?A} acts on this
decomposition. For example, if
\includegraphics{https://latex.codecogs.com/png.latex?V} is three-dimensional:

{[} A(\textbackslash{}mathbf\{v\}) = A(v\_1 \textbackslash{}mathbf\{e\}\_1 +
v\_2 \textbackslash{}mathbf\{e\}\_2 + v\_3
\textbackslash{}mathbf\{e\}\_3){]}(https://latex.codecogs.com/png.latex?\%0AA\%28\%5Cmathbf\%7Bv\%7D\%29\%20\%3D\%20A\%28v\_1\%20\%5Cmathbf\%7Be\%7D\_1\%20\%2B\%20v\_2\%20\%5Cmathbf\%7Be\%7D\_2\%20\%2B\%20v\_3\%20\%5Cmathbf\%7Be\%7D\_3\%29\%0A
" A(\mathbf{v}) = A(v\_1 \mathbf{e}\_1 + v\_2 \mathbf{e}\_2 + v\_3
\mathbf{e}\_3) ``)

Hm. Doesn't seem very insightful, does it?

\hypertarget{a-simple-definition}{%
\subsection{A simple definition}\label{a-simple-definition}}

But! We can exploit the linearity of
\includegraphics{https://latex.codecogs.com/png.latex?A} (that it distributes
and scales) to rewrite that as:

{[} A(\textbackslash{}mathbf\{v\}) = v\_1 A(\textbackslash{}mathbf\{e\}\_1) +
v\_2 A(\textbackslash{}mathbf\{e\}\_2) + v\_3
A(\textbackslash{}mathbf\{e\}\_3){]}(https://latex.codecogs.com/png.latex?\%0AA\%28\%5Cmathbf\%7Bv\%7D\%29\%20\%3D\%20v\_1\%20A\%28\%5Cmathbf\%7Be\%7D\_1\%29\%20\%2B\%20v\_2\%20A\%28\%5Cmathbf\%7Be\%7D\_2\%29\%20\%2B\%20v\_3\%20A\%28\%5Cmathbf\%7Be\%7D\_3\%29\%0A
" A(\mathbf{v}) = v\_1 A(\mathbf{e}\_1) + v\_2 A(\mathbf{e}\_2) + v\_3
A(\mathbf{e}\_3) ``)

Okay, take a moment to pause and take that all in. This is actually a pretty big
deal! This just means that, to study
\includegraphics{https://latex.codecogs.com/png.latex?A}, \textbf{all you need
to study} is how \includegraphics{https://latex.codecogs.com/png.latex?A} acts
on our \emph{basis vectors}. If you know how
\includegraphics{https://latex.codecogs.com/png.latex?A} acts on our basis
vectors of our vector space, that's really ``all there is'' about
\includegraphics{https://latex.codecogs.com/png.latex?A}! Not such a black box
anymore!

That is, if I were to ask you, ``Hey, what is
\includegraphics{https://latex.codecogs.com/png.latex?A} like?'', \emph{all
you'd have to tell me} is the result of
\includegraphics{https://latex.codecogs.com/png.latex?A\%28\%5Cmathbf\%7Be\%7D_1\%29},
\includegraphics{https://latex.codecogs.com/png.latex?A\%28\%5Cmathbf\%7Be\%7D_2},
and
\includegraphics{https://latex.codecogs.com/png.latex?A\%28\%5Cmathbf\%7Be\%7D_3\%29}.
Just give me those three \emph{vectors}, and we \emph{uniquely determine
\includegraphics{https://latex.codecogs.com/png.latex?A}}.

To put in another way, \emph{any linear transformation} from a three-dimensional
vector space is uniquely characterized by \emph{three vectors}:
\includegraphics{https://latex.codecogs.com/png.latex?A\%28\%5Cmathbf\%7Be\%7D_1\%29},
\includegraphics{https://latex.codecogs.com/png.latex?A\%28\%5Cmathbf\%7Be\%7D_2\%29},
and
\includegraphics{https://latex.codecogs.com/png.latex?A\%28\%5Cmathbf\%7Be\%7D_3\%29}.

Those three vectors \emph{completely define}
\includegraphics{https://latex.codecogs.com/png.latex?A}.

In general, we see that \emph{any linear transformation} from an N-dimensional
vector space can be \emph{completely defined} by N vectors: the N results of
that transformation on each of N basis vectors we choose.

\hypertarget{enter-the-matrix}{%
\subsection{Enter the Matrix}\label{enter-the-matrix}}

Okay, so how do we ``give''/define/state those N vectors?

Well, recall that the result of
\includegraphics{https://latex.codecogs.com/png.latex?A\%28\%5Cmathbf\%7Bv\%7D\%29}
and
\includegraphics{https://latex.codecogs.com/png.latex?A\%28\%5Cmathbf\%7Be\%7D_1\%29},
etc. are \emph{themselves} vectors, in M-dimensional vector space
\includegraphics{https://latex.codecogs.com/png.latex?U}. Let's say that
\includegraphics{https://latex.codecogs.com/png.latex?U} is 2-dimensional, for
now.

This means that any vector
\includegraphics{https://latex.codecogs.com/png.latex?\%5Cmathbf\%7Bu\%7D} in
\includegraphics{https://latex.codecogs.com/png.latex?U} can be represented as
\includegraphics{https://latex.codecogs.com/png.latex?u_1\%20\%5Cmathbf\%7Bq\%7D_1\%20\%2B\%20u_2\%20\%5Cmathbf\%7Bq\%7D_2},
where
\includegraphics{https://latex.codecogs.com/png.latex?\%5Cmathbf\%7Bq\%7D_1} and
\includegraphics{https://latex.codecogs.com/png.latex?\%5Cmathbf\%7Bq\%7D_2} is
an arbitrary choice of basis vectors.

This means that
\includegraphics{https://latex.codecogs.com/png.latex?A\%28\%5Cmathbf\%7Be\%7D_1\%29}
etc. can also all be represented in terms of these basis vectors. So, laying it
all out:

{[} \textbackslash{}begin\{aligned\} A(\textbackslash{}mathbf\{e\}\_1) \& =
a\_\{11\} \textbackslash{}mathbf\{q\}\_1 + a\_\{21\}
\textbackslash{}mathbf\{q\}\_2 \textbackslash{}\textbackslash{}
A(\textbackslash{}mathbf\{e\}\_2) \& = a\_\{12\} \textbackslash{}mathbf\{q\}\_1
+ a\_\{22\} \textbackslash{}mathbf\{q\}\_2 \textbackslash{}\textbackslash{}
A(\textbackslash{}mathbf\{e\}\_3) \& = a\_\{13\} \textbackslash{}mathbf\{q\}\_1
+ a\_\{23\} \textbackslash{}mathbf\{q\}\_2
\textbackslash{}end\{aligned\}{]}(https://latex.codecogs.com/png.latex?\%0A\%5Cbegin\%7Baligned\%7D\%0AA\%28\%5Cmathbf\%7Be\%7D\_1\%29\%20\%26\%20\%3D\%20a\_\%7B11\%7D\%20\%5Cmathbf\%7Bq\%7D\_1\%20\%2B\%20a\_\%7B21\%7D\%20\%5Cmathbf\%7Bq\%7D\_2\%20\%5C\%5C\%0AA\%28\%5Cmathbf\%7Be\%7D\_2\%29\%20\%26\%20\%3D\%20a\_\%7B12\%7D\%20\%5Cmathbf\%7Bq\%7D\_1\%20\%2B\%20a\_\%7B22\%7D\%20\%5Cmathbf\%7Bq\%7D\_2\%20\%5C\%5C\%0AA\%28\%5Cmathbf\%7Be\%7D\_3\%29\%20\%26\%20\%3D\%20a\_\%7B13\%7D\%20\%5Cmathbf\%7Bq\%7D\_1\%20\%2B\%20a\_\%7B23\%7D\%20\%5Cmathbf\%7Bq\%7D\_2\%0A\%5Cend\%7Baligned\%7D\%0A
" \textbackslash{}begin\{aligned\} A(\mathbf{e}\emph{1) \& = a}\{11\}
\mathbf{q}\emph{1 + a}\{21\} \mathbf{q}\_2 \textbackslash{} A(\mathbf{e}\emph{2)
\& = a}\{12\} \mathbf{q}\emph{1 + a}\{22\} \mathbf{q}\_2 \textbackslash{}
A(\mathbf{e}\emph{3) \& = a}\{13\} \mathbf{q}\emph{1 + a}\{23\} \mathbf{q}\_2
\textbackslash{}end\{aligned\} ``)

Or, to use our bracket notation from before:

{[} \textbackslash{}begin\{aligned\} A(\textbackslash{}mathbf\{e\}\_1) \& =
\textbackslash{}langle a\_\{11\}, a\_\{21\} \textbackslash{}rangle
\textbackslash{}\textbackslash{} A(\textbackslash{}mathbf\{e\}\_2) \& =
\textbackslash{}langle a\_\{12\}, a\_\{22\} \textbackslash{}rangle
\textbackslash{}\textbackslash{} A(\textbackslash{}mathbf\{e\}\_3) \& =
\textbackslash{}langle a\_\{13\}, a\_\{23\} \textbackslash{}rangle
\textbackslash{}end\{aligned\}{]}(https://latex.codecogs.com/png.latex?\%0A\%5Cbegin\%7Baligned\%7D\%0AA\%28\%5Cmathbf\%7Be\%7D\_1\%29\%20\%26\%20\%3D\%20\%5Clangle\%20a\_\%7B11\%7D\%2C\%20a\_\%7B21\%7D\%20\%5Crangle\%20\%5C\%5C\%0AA\%28\%5Cmathbf\%7Be\%7D\_2\%29\%20\%26\%20\%3D\%20\%5Clangle\%20a\_\%7B12\%7D\%2C\%20a\_\%7B22\%7D\%20\%5Crangle\%20\%5C\%5C\%0AA\%28\%5Cmathbf\%7Be\%7D\_3\%29\%20\%26\%20\%3D\%20\%5Clangle\%20a\_\%7B13\%7D\%2C\%20a\_\%7B23\%7D\%20\%5Crangle\%0A\%5Cend\%7Baligned\%7D\%0A
" \textbackslash{}begin\{aligned\} A(\mathbf{e}\emph{1) \& = \langle a}\{11\},
a\_\{21\} \rangle \textbackslash{} A(\mathbf{e}\emph{2) \& = \langle a}\{12\},
a\_\{22\} \rangle \textbackslash{} A(\mathbf{e}\emph{3) \& = \langle a}\{13\},
a\_\{23\} \rangle \textbackslash{}end\{aligned\} ``)

So, we now see two facts:

\begin{enumerate}
\def\labelenumi{\arabic{enumi}.}
\tightlist
\item
  A linear transformation from an N dimensional vector space to an M dimensional
  vector space can be \emph{defined} using N vectors.
\item
  Each of those N vectors can, themselves, be defined using M scalars each.
\end{enumerate}

Our final conclusion: \emph{any} linear transformation from an N dimensional
vector space to an M dimensional vector space can be defined using
\includegraphics{https://latex.codecogs.com/png.latex?N\%20\%5Ctimes\%20M}
scalars.

That's right -- \emph{all} possible linear transformations from a 3-dimensional
vector space to a 2-dimensional are parameterized by only \emph{six} scalars.
These six scalars uniquely determine and define our linear transformation, given
a set of basis vectors that we agree on.

These six numbers are pretty important. Just like how we often talk about
3-dimensional vectors in terms of the encoding of their three coefficients, we
often talk about linear transformations from 3-d space to 2-d space in terms of
their six defining coefficients.

We group these things up in something called a \emph{matrix}.

If our linear transformation
\includegraphics{https://latex.codecogs.com/png.latex?A} from a 3-dimensional
vector space to a 2-dimensional vector space is defined by:

{[} \textbackslash{}begin\{aligned\} A(\textbackslash{}mathbf\{e\}\_1) \& =
a\_\{11\} \textbackslash{}mathbf\{q\}\_1 + a\_\{21\}
\textbackslash{}mathbf\{q\}\_2 \textbackslash{}\textbackslash{}
A(\textbackslash{}mathbf\{e\}\_2) \& = a\_\{12\} \textbackslash{}mathbf\{q\}\_1
+ a\_\{22\} \textbackslash{}mathbf\{q\}\_2 \textbackslash{}\textbackslash{}
A(\textbackslash{}mathbf\{e\}\_3) \& = a\_\{13\} \textbackslash{}mathbf\{q\}\_1
+ a\_\{23\} \textbackslash{}mathbf\{q\}\_2
\textbackslash{}end\{aligned\}{]}(https://latex.codecogs.com/png.latex?\%0A\%5Cbegin\%7Baligned\%7D\%0AA\%28\%5Cmathbf\%7Be\%7D\_1\%29\%20\%26\%20\%3D\%20a\_\%7B11\%7D\%20\%5Cmathbf\%7Bq\%7D\_1\%20\%2B\%20a\_\%7B21\%7D\%20\%5Cmathbf\%7Bq\%7D\_2\%20\%5C\%5C\%0AA\%28\%5Cmathbf\%7Be\%7D\_2\%29\%20\%26\%20\%3D\%20a\_\%7B12\%7D\%20\%5Cmathbf\%7Bq\%7D\_1\%20\%2B\%20a\_\%7B22\%7D\%20\%5Cmathbf\%7Bq\%7D\_2\%20\%5C\%5C\%0AA\%28\%5Cmathbf\%7Be\%7D\_3\%29\%20\%26\%20\%3D\%20a\_\%7B13\%7D\%20\%5Cmathbf\%7Bq\%7D\_1\%20\%2B\%20a\_\%7B23\%7D\%20\%5Cmathbf\%7Bq\%7D\_2\%0A\%5Cend\%7Baligned\%7D\%0A
" \textbackslash{}begin\{aligned\} A(\mathbf{e}\emph{1) \& = a}\{11\}
\mathbf{q}\emph{1 + a}\{21\} \mathbf{q}\_2 \textbackslash{} A(\mathbf{e}\emph{2)
\& = a}\{12\} \mathbf{q}\emph{1 + a}\{22\} \mathbf{q}\_2 \textbackslash{}
A(\mathbf{e}\emph{3) \& = a}\{13\} \mathbf{q}\emph{1 + a}\{23\} \mathbf{q}\_2
\textbackslash{}end\{aligned\} ``)

(for arbitrary choice of bases
\includegraphics{https://latex.codecogs.com/png.latex?\%5Cmathbf\%7Be\%7D_i} and
\includegraphics{https://latex.codecogs.com/png.latex?\%5Cmathbf\%7Bq\%7D_i})

We ``encode'' it as the matrix:

{[} \textbackslash{}begin\{bmatrix\} a\_\{11\} \& a\_\{12\} \& a\_\{13\}
\textbackslash{}\textbackslash{} a\_\{21\} \& a\_\{22\} \& a\_\{23\}
\textbackslash{}end\{bmatrix\}{]}(https://latex.codecogs.com/png.latex?\%0A\%5Cbegin\%7Bbmatrix\%7D\%0Aa\_\%7B11\%7D\%20\%26\%20a\_\%7B12\%7D\%20\%26\%20a\_\%7B13\%7D\%20\%5C\%5C\%0Aa\_\%7B21\%7D\%20\%26\%20a\_\%7B22\%7D\%20\%26\%20a\_\%7B23\%7D\%0A\%5Cend\%7Bbmatrix\%7D\%0A
" \textbackslash{}begin\{bmatrix\} a\_\{11\} \& a\_\{12\} \& a\_\{13\}
\textbackslash{} a\_\{21\} \& a\_\{22\} \& a\_\{23\}
\textbackslash{}end\{bmatrix\} ``)

And that's why we use matrices in linear algebra -- like how
\includegraphics{https://latex.codecogs.com/png.latex?\%5Clangle\%20x\%2C\%20y\%2C\%20z\%20\%5Crangle}
is a convenient way to represent and define a \emph{vector} (once we agree on a
bases), a
\includegraphics{https://latex.codecogs.com/png.latex?M\%20\%5Ctimes\%20N}
matrix is a convenient way to represent and define a \emph{linear
transformation} from an N-dimensional vector space to a M-dimensional vector
space (once we agree on the bases in both spaces).

And, sometimes we just think of the
\includegraphics{https://latex.codecogs.com/png.latex?\%5Clangle\%20x\%20y\%2C\%20z\%20\%5Crangle}
encoding as the vector itself, we often also talk about the
\includegraphics{https://latex.codecogs.com/png.latex?M\%20\%5Ctimes\%20N}
matrix as if it were the linear transformation itself. The matrix ``is'' the
linear transformation, informally.

\hypertarget{matrix-operations}{%
\section{Matrix Operations}\label{matrix-operations}}

In this light, we can understand the definition of the common matrix operations.

\end{document}
