\documentclass[]{article}
\usepackage{lmodern}
\usepackage{amssymb,amsmath}
\usepackage{ifxetex,ifluatex}
\usepackage{fixltx2e} % provides \textsubscript
\ifnum 0\ifxetex 1\fi\ifluatex 1\fi=0 % if pdftex
  \usepackage[T1]{fontenc}
  \usepackage[utf8]{inputenc}
\else % if luatex or xelatex
  \ifxetex
    \usepackage{mathspec}
    \usepackage{xltxtra,xunicode}
  \else
    \usepackage{fontspec}
  \fi
  \defaultfontfeatures{Mapping=tex-text,Scale=MatchLowercase}
  \newcommand{\euro}{€}
\fi
% use upquote if available, for straight quotes in verbatim environments
\IfFileExists{upquote.sty}{\usepackage{upquote}}{}
% use microtype if available
\IfFileExists{microtype.sty}{\usepackage{microtype}}{}
\usepackage[margin=1in]{geometry}
\ifxetex
  \usepackage[setpagesize=false, % page size defined by xetex
              unicode=false, % unicode breaks when used with xetex
              xetex]{hyperref}
\else
  \usepackage[unicode=true]{hyperref}
\fi
\hypersetup{breaklinks=true,
            bookmarks=true,
            pdfauthor={},
            pdftitle={},
            colorlinks=true,
            citecolor=blue,
            urlcolor=blue,
            linkcolor=magenta,
            pdfborder={0 0 0}}
\urlstyle{same}  % don't use monospace font for urls
% Make links footnotes instead of hotlinks:
\renewcommand{\href}[2]{#2\footnote{\url{#1}}}
\setlength{\parindent}{0pt}
\setlength{\parskip}{6pt plus 2pt minus 1pt}
\setlength{\emergencystretch}{3em}  % prevent overfull lines
\setcounter{secnumdepth}{0}


\begin{document}

\% Introducing the "Prompt" library \% Justin Le \% June 30, 2015

\emph{Originally posted on
\textbf{\href{https://blog.jle.im/entry/introducing-the-prompt-library.html}{in
Code}}.}

\textbf{Prompt}:
\href{https://github.com/mstksg/prompt/blob/master/README.md}{README} /
\href{http://hackage.haskell.org/package/prompt}{hackage} /
\href{https://github.com/mstksg/prompt}{github}

Have you ever wanted to specify a computation involving some limited form of IO
-\/-\/- like querying a database, or asking stdio -\/-\/- but didn't want a
computation in the \texttt{IO} monad, opening the entire can of worms that is
arbitrary \texttt{IO}? Have you ever looked at complicated \texttt{IO\ a} you
wrote last week at 4am and prayed that it didn't launch missiles if you decided
to execute it? Do you want to be able to run an effectful computation and
explicitly \emph{say} what IO it can or cannot do?

Introducing the \emph{\href{http://hackage.haskell.org/package/prompt}{prompt}}
library! It's a small little lightweight library that allows you to specify and
describe computations involving forms of effects where you "ask" with a value
and receive a value in return (such as a database query, etc.), but not ever
care about how the effects are fulfilled -\/-\/- freeing you from working
directly with IO.

\textasciitilde{}\textasciitilde{}\textasciitilde{}haskell data Foo = Foo \{
fooBar :: String , fooBaz :: Int \} deriving Show

-\/- ask with a String, receive a String as an answer promptFoo :: Prompt String
String Foo promptFoo = Foo \textless{}\$\textgreater{} prompt "bar"
\textless{}*\textgreater{} fmap length (prompt "baz")
\textasciitilde{}\textasciitilde{}\textasciitilde{}

\section{Running}

You can now "run it" in IO, by talking to stdio -\/-\/-

\textasciitilde{}\textasciitilde{}\textasciitilde{}haskell ghci\textgreater{}
runPromptM promptFoo \$ \textbackslash{}str -\textgreater{} putStrLn str
\textgreater{}\textgreater{} getLine bar -\/- stdout prompt

\begin{quote}
hello! -\/- stdin response typed in baz -\/- stdout prompt i am baz -\/- stdin
response typed in Foo "hello!" 8 -\/- result
\textasciitilde{}\textasciitilde{}\textasciitilde{}
\end{quote}

(this is also just \texttt{interactP\ promptFoo})

Or you can maybe request it from the environment variables:

\textasciitilde{}\textasciitilde{}\textasciitilde{}haskell ghci\textgreater{}
import System.Environment ghci\textgreater{} setEnv "bar" "hello!"
ghci\textgreater{} setEnv "baz" "i am baz" ghci\textgreater{} runPromptM
promptFoo getEnv Foo "hello!" 8
\textasciitilde{}\textasciitilde{}\textasciitilde{}

Or maybe you want to fulfill the prompts purely:

\textasciitilde{}\textasciitilde{}\textasciitilde{}haskell ghci\textgreater{}
import qualified Data.Map as M ghci\textgreater{} let testMap = M.fromList
{[}("bar", "hello!"), ("baz", "i am baz"){]} ghci\textgreater{} runPrompt
promptFoo (testMap M.!) Foo "hello!" 8
\textasciitilde{}\textasciitilde{}\textasciitilde{}

With \texttt{Prompt}, specify the computation and your logic \emph{without
involving any IO}, so you can write safe code without arbitrary side effects. If
you ever receive a \texttt{Prompt}, you know it can't wipe out your hard drive
or do any IO other than exactly what you allow it to do! I'd feel more safe
running a \texttt{Prompt\ a\ b\ r} than an \texttt{IO\ r}.

You can also do some cute tricks; \texttt{Prompt\ a\ ()\ r} with a "prompt
response function" like \texttt{putStrLn} lets you do streaming logging, and
defer \emph{how} the logging is done -\/-\/- to IO, to a list?

\textasciitilde{}\textasciitilde{}\textasciitilde{}haskell ghci\textgreater{}
let logHelloWord = mapM\_ prompt {[}"hello", "world"{]} ghci\textgreater{}
runPromptM logHelloWorld putStrLn hello world ghci\textgreater{} execWriter \$
runPromptM logHelloWorld tell "helloworld"
\textasciitilde{}\textasciitilde{}\textasciitilde{}

\texttt{Prompt\ ()\ b\ r} is like a fancy \texttt{ReaderT\ b\ m\ r}, where you
"defer" the choice of the Monad.

\section{Combining with other effects}

\texttt{Prompt} can be used as an underlying "effects" source for libraries like
\emph{pipes}, \emph{conduit}, and \emph{auto}. If your effects are only ever
asking and prompting and receiving, there's really no need to put the entire
power of \texttt{IO} underneath your DSL as an effects source. That's just
crazy!

\texttt{Prompt} can be used with monad transformers to give you safe underlying
effect sources, like \texttt{StateT\ s\ (Prompt\ a\ b)\ r}, which is a stateful
computation which can sometimes sequence "prompty" effects. \texttt{Prompt} is
also itself a "Traversable transformer", with \texttt{PrompT\ a\ b\ t\ r}. It
can perform computations in the context of a Traversable \texttt{t}, to be able
to incorporate built-in short-circuiting and logging, etc.

This is all abstracted over with \texttt{MonadPrompt}, \texttt{MonadError},
\texttt{MonadPlus}, etc., typeclasses -\/-\/-

\textasciitilde{}\textasciitilde{}\textasciitilde{}haskell promptFoo2 ::
(MonadPlus m, MonadPrompt String String m) =\textgreater{} m Foo promptFoo2 = do
bar \textless{}- prompt "bar" str \textless{}- prompt "baz" case readMaybe str
of Just baz -\textgreater{} return \$ Foo bar baz Nothing -\textgreater{} mzero

-\/- more polymorphic promptFoo :: MonadPrompt String String m =\textgreater{} m
Foo promptFoo = Foo \textless{}\$\textgreater{} prompt "bar"
\textless{}*\textgreater{} fmap length (prompt "baz")
\textasciitilde{}\textasciitilde{}\textasciitilde{}

You can run \texttt{promptFoo} as a
\texttt{MaybeT\ (Prompt\ String\ String)\ Foo}, and manually unwrap:

\textasciitilde{}\textasciitilde{}\textasciitilde{}haskell ghci\textgreater{}
interactP . runMaybeT \$ promptFoo2 bar

\begin{quote}
hello! baz i am baz Nothing ghci\textgreater{} interactP . runMaybeT \$
promptFoo2 bar hello! baz 19 Just (Foo "hello!" 19)
\textasciitilde{}\textasciitilde{}\textasciitilde{}
\end{quote}

Or you can run it as a \texttt{PromptT\ String\ String\ MaybeT\ Foo}, to have
\texttt{PromptT} handle the wrapping/unwrapping itself:

\textasciitilde{}\textasciitilde{}\textasciitilde{}haskell ghci\textgreater{}
interactPT promptFoo2 bar

\begin{quote}
hello! baz i am baz Nothing ghci\textgreater{} interactPT \$ promptFoo2
\textless{}\textbar{}\textgreater{} promptFoo bar hello! baz i am baz bar -\/-
failed to parse -\/-\/- retrying with promptFoo! hello! baz i am baz Just (Foo
"hello" 8) \textasciitilde{}\textasciitilde{}\textasciitilde{}
\end{quote}

The previous example of \texttt{logHelloWorld}?

\textasciitilde{}\textasciitilde{}\textasciitilde{}haskell ghci\textgreater{}
runPromptT (logHelloWorld :: PromptT String () (Writer String) ()) tell
"helloworld" \textasciitilde{}\textasciitilde{}\textasciitilde{}

\section{Runners}

The "runners" are:

\textasciitilde{}\textasciitilde{}\textasciitilde{}haskell interactP :: Prompt
String String r -\textgreater{} IO r interactPT :: Applicative t =\textgreater{}
PromptT String String t r -\textgreater{} IO (t r)

runPrompt :: Prompt a b r -\textgreater{} (a -\textgreater{} b) -\textgreater{}
r runPromptM :: Monad m =\textgreater{} Prompt a b r -\textgreater{} (a
-\textgreater{} m b) -\textgreater{} m r

runPromptT :: PromptT a b t r -\textgreater{} (a -\textgreater{} t b)
-\textgreater{} t r runPromptTM :: Monad m =\textgreater{} PromptT a b t r
-\textgreater{} (a -\textgreater{} m (t b)) -\textgreater{} m (t r)
\textasciitilde{}\textasciitilde{}\textasciitilde{}

Note that \texttt{runPromptM} and \texttt{runPromptTM} can run in monads (like
\texttt{IO}) that are \emph{completely unrelated} to the \texttt{Prompt} type
itself. It sequences them all "after the fact". It's also interesting to note
that \texttt{runPrompt} is just a glorified
\texttt{Reader\ (a\ -\textgreater{}\ b)\ r}.

With \texttt{runPromptTM}, you can incorporate \texttt{t} in your "prompt
response" function, too. Which brings us to our grand finale -\/- environment
variable parsing!

\textasciitilde{}\textasciitilde{}\textasciitilde{}haskell import
Control.Monad.Error.Class import Control.Monad.Prompt import Text.Read import
qualified Data.Map as M

type Key = String type Val = String

data MyError = MENoParse Key Val \textbar{} MENotFound Key deriving Show

promptRead :: (MonadError MyError m, MonadPrompt Key Val m, Read b)
=\textgreater{} Key -\textgreater{} m b -\/- promptRead :: Read b
=\textgreater{} Key -\textgreater{} PromptT Key Val (Either MyError) b
promptRead k = do resp \textless{}- prompt k case readMaybe resp of Nothing
-\textgreater{} throwError \$ MEParse k resp Just v -\textgreater{} return v

promptFoo3 :: MonadPrompt Key Val m =\textgreater{} m Foo -\/- promptFoo3 ::
Applicative t =\textgreater{} PromptT Key Val t Foo promptFoo3 = Foo
\textless{}\$\textgreater{} prompt "bar" \textless{}*\textgreater{} promptRead
"baz"

-\/- -\/- running!

-\/- Lookup environment variables, and "throw" an error if not found throughEnv
:: IO (Either MyError Foo) throughEnv = runPromptTM parseFoo3 \$
\textbackslash{}k -\textgreater{} do env \textless{}- lookupEnv k return \$ case
env of Nothing -\textgreater{} Left (MENotFound k) Just v -\textgreater{} Right
v

-\/- Fulfill the prompt through user input throughStdIO :: IO (Either MyError
Foo) throughStdIO = interactPT parseFoo3

-\/- Fulfill the prompt through user input; count blank responses as "not found"
throughStdIOBlankIsError :: IO (Either MyError Foo) throughStdIOBlankIsError =
runPromptTM parseFoo3 \$ \textbackslash{}k -\textgreater{} do putStrLn k resp
\textless{}- getLine return \$ if null resp then Left (MENotFound k) else Right
resp

-\/- Fulfill the prompt purely through a Map lookup throughMap :: M.Map Key Val
-\textgreater{} Either MyError Foo throughMap m = runPromptT parseFoo3 \$
\textbackslash{}k -\textgreater{} case M.lookup k m of Nothing -\textgreater{}
Left (MENotFound k) Just v -\textgreater{} Right v
\textasciitilde{}\textasciitilde{}\textasciitilde{}

Hope you enjoy! Please feel free to leave a comment, find me on
\href{https://twitter.com/mstk}{twitter}, leave an issue on the
\href{https://github.com/mstksg/prompt}{github}, etc. -\/-\/- and I'm usually on
freenode's \emph{\#haskell} as \emph{jle`} if you have any questions!

\section{Comparisons}

To lay it all on the floor,

\textasciitilde{}\textasciitilde{}\textasciitilde{}haskell newtype PromptT a b t
r = PromptT \{ runPromptTM :: forall m. Monad m =\textgreater{} (a
-\textgreater{} m (t b)) -\textgreater{} m (t r) \}
\textasciitilde{}\textasciitilde{}\textasciitilde{}

There is admittedly a popular misconception that I've seen going around that
equates this sort of type to \texttt{Free} from the \emph{free} package.
However, \texttt{Free} doesn't really have anything significant to do with this.
Sure, you might be able to generate this type by using \texttt{FreeT} over a
specifically chosen Functor, but...this is the case for literally any Monad
ever, so that doesn't really mean much :)

It's also unrelated in this same manner to \texttt{Prompt} from the
\emph{MonadPrompt} package, and \texttt{Program} from \emph{operational} too.

One close relative to this type is
\texttt{forall\ m.\ ReaderT\ (a\ -\textgreater{}\ m\ b)\ m\ r}, where
\texttt{prompt\ k\ =\ ReaderT\ (\$\ k)}. This is more or less equivalent to
\texttt{Prompt}, but still can't do the things that \texttt{PromptT} can do
without a special instance of Monad.

This type is also similar in structure to \texttt{Bazaar}, from the \emph{lens}
package. The biggest difference that makes \texttt{Bazaar} unusable is because
the RankN constraint is only \texttt{Applicative}, not \texttt{Monad}, so a
\texttt{Monad} instance is impossible. Ignoring that (or if it's okay for you to
only use the \texttt{Applicative} instance), \texttt{Bazaar} forces the
"prompting effect" to take place in the same context as the \texttt{Traversable}
\texttt{t}...which really defeats the purpose of this whole thing in the first
place (the idea is to be able to separate your prompting effect from your
application logic). If the \texttt{Traversable} you want to transform has a
"monad transformer" version, then you can somewhat simulate \texttt{PromptT} for
that specifc \texttt{t} with the transformer version.

It's also somewhat similar to the \texttt{Client} type from \emph{pipes}, but
it's also a bit tricky to use that with a different effect type than the logic
\texttt{Traversable}, as well...so it has a lot of the same difference as
\texttt{Bazaar} here.

But this type is common/simple enough that I'm sure someone has it somewhere in
a library that I haven't been able to find. If you find it, let me know!

\end{document}
