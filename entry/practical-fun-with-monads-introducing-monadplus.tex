\documentclass[]{article}
\usepackage{lmodern}
\usepackage{amssymb,amsmath}
\usepackage{ifxetex,ifluatex}
\usepackage{fixltx2e} % provides \textsubscript
\ifnum 0\ifxetex 1\fi\ifluatex 1\fi=0 % if pdftex
  \usepackage[T1]{fontenc}
  \usepackage[utf8]{inputenc}
\else % if luatex or xelatex
  \ifxetex
    \usepackage{mathspec}
    \usepackage{xltxtra,xunicode}
  \else
    \usepackage{fontspec}
  \fi
  \defaultfontfeatures{Mapping=tex-text,Scale=MatchLowercase}
  \newcommand{\euro}{€}
\fi
% use upquote if available, for straight quotes in verbatim environments
\IfFileExists{upquote.sty}{\usepackage{upquote}}{}
% use microtype if available
\IfFileExists{microtype.sty}{\usepackage{microtype}}{}
\usepackage[margin=1in]{geometry}
\usepackage{color}
\usepackage{fancyvrb}
\newcommand{\VerbBar}{|}
\newcommand{\VERB}{\Verb[commandchars=\\\{\}]}
\DefineVerbatimEnvironment{Highlighting}{Verbatim}{commandchars=\\\{\}}
% Add ',fontsize=\small' for more characters per line
\newenvironment{Shaded}{}{}
\newcommand{\AlertTok}[1]{\textcolor[rgb]{1.00,0.00,0.00}{\textbf{#1}}}
\newcommand{\AnnotationTok}[1]{\textcolor[rgb]{0.38,0.63,0.69}{\textbf{\textit{#1}}}}
\newcommand{\AttributeTok}[1]{\textcolor[rgb]{0.49,0.56,0.16}{#1}}
\newcommand{\BaseNTok}[1]{\textcolor[rgb]{0.25,0.63,0.44}{#1}}
\newcommand{\BuiltInTok}[1]{#1}
\newcommand{\CharTok}[1]{\textcolor[rgb]{0.25,0.44,0.63}{#1}}
\newcommand{\CommentTok}[1]{\textcolor[rgb]{0.38,0.63,0.69}{\textit{#1}}}
\newcommand{\CommentVarTok}[1]{\textcolor[rgb]{0.38,0.63,0.69}{\textbf{\textit{#1}}}}
\newcommand{\ConstantTok}[1]{\textcolor[rgb]{0.53,0.00,0.00}{#1}}
\newcommand{\ControlFlowTok}[1]{\textcolor[rgb]{0.00,0.44,0.13}{\textbf{#1}}}
\newcommand{\DataTypeTok}[1]{\textcolor[rgb]{0.56,0.13,0.00}{#1}}
\newcommand{\DecValTok}[1]{\textcolor[rgb]{0.25,0.63,0.44}{#1}}
\newcommand{\DocumentationTok}[1]{\textcolor[rgb]{0.73,0.13,0.13}{\textit{#1}}}
\newcommand{\ErrorTok}[1]{\textcolor[rgb]{1.00,0.00,0.00}{\textbf{#1}}}
\newcommand{\ExtensionTok}[1]{#1}
\newcommand{\FloatTok}[1]{\textcolor[rgb]{0.25,0.63,0.44}{#1}}
\newcommand{\FunctionTok}[1]{\textcolor[rgb]{0.02,0.16,0.49}{#1}}
\newcommand{\ImportTok}[1]{#1}
\newcommand{\InformationTok}[1]{\textcolor[rgb]{0.38,0.63,0.69}{\textbf{\textit{#1}}}}
\newcommand{\KeywordTok}[1]{\textcolor[rgb]{0.00,0.44,0.13}{\textbf{#1}}}
\newcommand{\NormalTok}[1]{#1}
\newcommand{\OperatorTok}[1]{\textcolor[rgb]{0.40,0.40,0.40}{#1}}
\newcommand{\OtherTok}[1]{\textcolor[rgb]{0.00,0.44,0.13}{#1}}
\newcommand{\PreprocessorTok}[1]{\textcolor[rgb]{0.74,0.48,0.00}{#1}}
\newcommand{\RegionMarkerTok}[1]{#1}
\newcommand{\SpecialCharTok}[1]{\textcolor[rgb]{0.25,0.44,0.63}{#1}}
\newcommand{\SpecialStringTok}[1]{\textcolor[rgb]{0.73,0.40,0.53}{#1}}
\newcommand{\StringTok}[1]{\textcolor[rgb]{0.25,0.44,0.63}{#1}}
\newcommand{\VariableTok}[1]{\textcolor[rgb]{0.10,0.09,0.49}{#1}}
\newcommand{\VerbatimStringTok}[1]{\textcolor[rgb]{0.25,0.44,0.63}{#1}}
\newcommand{\WarningTok}[1]{\textcolor[rgb]{0.38,0.63,0.69}{\textbf{\textit{#1}}}}
\ifxetex
  \usepackage[setpagesize=false, % page size defined by xetex
              unicode=false, % unicode breaks when used with xetex
              xetex]{hyperref}
\else
  \usepackage[unicode=true]{hyperref}
\fi
\hypersetup{breaklinks=true,
            bookmarks=true,
            pdfauthor={Justin Le},
            pdftitle={Practical Fun with Monads --- Introducing: MonadPlus!},
            colorlinks=true,
            citecolor=blue,
            urlcolor=blue,
            linkcolor=magenta,
            pdfborder={0 0 0}}
\urlstyle{same}  % don't use monospace font for urls
% Make links footnotes instead of hotlinks:
\renewcommand{\href}[2]{#2\footnote{\url{#1}}}
\setlength{\parindent}{0pt}
\setlength{\parskip}{6pt plus 2pt minus 1pt}
\setlength{\emergencystretch}{3em}  % prevent overfull lines
\setcounter{secnumdepth}{0}

\title{Practical Fun with Monads --- Introducing: MonadPlus!}
\author{Justin Le}
\date{December 9, 2013}

\begin{document}
\maketitle

\emph{Originally posted on
\textbf{\href{https://blog.jle.im/entry/practical-fun-with-monads-introducing-monadplus.html}{in
Code}}.}

Monads. Haskell's famous for them, but they are one of the most ill-understood
concepts to the public. They are mostly shrouded in mystery because of their
association with how Haskell models I/O. This reputation is undeserved. Monads
don't have anything to do with I/O.

This series is a part of a global effort to pull away the shroud of mystery
behind monads and show that they are fun! And exciting! And really just a way of
chaining together functions that allow for new ways of approaching puzzles.

The first sub-series (chapter?) will be on a specific class/family of monads
known as \emph{MonadPlus}. At the end of it all, we are going to be solving the
classic logic puzzle, as old as time itself, using \textbf{only} the List monad
instance, and no loops, queues, or fancy stuff like that:

\begin{quote}
A farmer has a wolf, a goat, and a cabbage that he wishes to transport across a
river. Unfortunately, his boat can carry only one thing at a time with him. He
can't leave the wolf alone with the goat, or the wolf will eat the goat. He
can't leave the goat alone with the cabbage, or the goat will eat the cabbage.
How can he properly transport his belongings to the other side one at a time,
without any disasters?
\end{quote}

Let us enter a brave new world!

\hypertarget{a-quick-review-of-monads}{%
\subsection{A quick review of monads}\label{a-quick-review-of-monads}}

\begin{center}\rule{0.5\linewidth}{\linethickness}\end{center}

\textbf{Note}

This article is written for both beginners --- people who have a fuzzy idea of
monads and a minimal understanding of functional programming principles, but who
have some experience in Object-Oriented Programming in a language like Java or
C++ --- and intermediate Haskell users --- people who have a somewhat firm grasp
on monads, but want to know about monads on a broader context (in particular,
the MonadPlus typeclass).

Intermediate Haskell users will most likely find this post to be review, and
I'll put a link in this paragraph when the next part is up so we can get to
``real'' Haskelling. However, this post might be beneficial if you read it while
asking, at every point, ``How can this be abstracted and generalized?''. It's a
fun exercise!

This article attempts to explain all Haskell syntax that might be foreign to
beginners. That being said, if you ever run into anything you can't understand,
feel free to either read the articles above, give
\href{http://learnyouahaskell.com/}{Learn You A Haskell} a quick read (you won't
regret it!), or leave a comment --- I'd love to answer your questions or hear
your responses!

This first post will cover the basics of MonadPlus with the simplest MonadPlus
of all; the second part will explore the List MonadPlus, and the third will
finally tackle the Wolf/Goat/Cabbage puzzle with our combined knowledge.

\begin{center}\rule{0.5\linewidth}{\linethickness}\end{center}

Okay, so as a Haskell blogger, I'm actually not allowed to write any monad
tutorials. Luckily for you, however, I don't need too --- there are a wealth of
great ones.
\href{http://adit.io/posts/2013-04-17-functors,_applicatives,_and_monads_in_pictures.html}{Adit
provides a great concise one}, and, if you want,
\href{http://www.haskell.org/haskellwiki/All_About_Monads}{a more in depth one}
is on the haskell.org wiki about all sorts of monads and using them in real
life.

Remember --- different monads do not actually have any non-superficial
relationship to one another. When we say monads, we just mean objects for which
we have defined a way to chain together functions ``inside'' wrappers,
containers, or contexts.

\hypertarget{maybe-maybe-not}{%
\section{Maybe, maybe not}\label{maybe-maybe-not}}

Monads are very useful when you are dealing with objects that are containers.
Let's look at the most obvious container -- a \texttt{Maybe\ a}. A
\texttt{Maybe\ a} is a container that can either be \texttt{Just\ x}
(representing a successful result \texttt{x} of type \texttt{a}) or a
\texttt{Nothing} (representing a failed result).

\begin{center}\rule{0.5\linewidth}{\linethickness}\end{center}

\textbf{Welcome to Haskell!}

Hi! These ``Welcome to Haskell'' asides are going to be for you readers that are
unfamiliar with Haskell syntax. Feel free to ignore them if you already feel
comfortable.

Anyways, if you've ever done any object-oriented programming, you might be able
to think of \texttt{Maybe\ a} as an abstract/virtual superclass with
templates/generics --- \texttt{Maybe\textless{}a\textgreater{}}, kinda. And that
superclass has two subclasses: \texttt{Just\textless{}a\textgreater{}}, which
has one public instance variable \texttt{x} of type \texttt{a}, and
\texttt{Nothing}, which contains no instance variables.

\begin{center}\rule{0.5\linewidth}{\linethickness}\end{center}

Often times you'll have functions that fail, and you want to chain them. The
easiest way is that any function that is chained onto a failed value will be
skipped; a failure is the final result.

Consider the \texttt{halve} function, which returns
\texttt{Just\ (x\ \textasciigrave{}div\textasciigrave{}\ 2)} on a successful
halving, or \texttt{Nothing} on an unsuccessful halving:

\begin{Shaded}
\begin{Highlighting}[]
\OtherTok{halve ::} \DataTypeTok{Int} \OtherTok{->} \DataTypeTok{Maybe} \DataTypeTok{Int}                       \CommentTok{-- 1}
\NormalTok{halve x }\FunctionTok{|}\NormalTok{ even x    }\FunctionTok{=} \DataTypeTok{Just}\NormalTok{ (x }\OtherTok{`div`} \DecValTok{2}\NormalTok{)          }\CommentTok{-- 2}
        \FunctionTok{|}\NormalTok{ otherwise }\FunctionTok{=} \DataTypeTok{Nothing}                   \CommentTok{-- 3}
\end{Highlighting}
\end{Shaded}

\begin{center}\rule{0.5\linewidth}{\linethickness}\end{center}

\textbf{Welcome to Haskell!}

Hi again! There are some quick syntax features here.

\begin{enumerate}
\def\labelenumi{\arabic{enumi}.}
\tightlist
\item
  This first line declares that the type of the function \texttt{halve} is
  \texttt{Int\ -\textgreater{}\ Maybe\ Int}, which means that it takes in an
  \texttt{Int} and returns a \texttt{Maybe\ Int} --- an integer wrapped in a
  ``Maybe'' container.
\item
  This says that if x is even, then return a successful
  \texttt{Just\ (x\ \textasciigrave{}div\textasciigrave{}\ 2)}.
  \texttt{x\ \textasciigrave{}div\textasciigrave{}\ 2} is x divided by two, in
  case you couldn't guess already.
\item
  Otherwise, return \texttt{Nothing} --- a failure.
\end{enumerate}

\begin{center}\rule{0.5\linewidth}{\linethickness}\end{center}

Because Maybe comes built-in as a monad, we can now chain \texttt{halve}s on
results of other \texttt{halves}, and have any failures automatically propagate
to the end and short circuit your entire computation:

\begin{Shaded}
\begin{Highlighting}[]
\NormalTok{ghci}\FunctionTok{>}\NormalTok{ halve }\DecValTok{8}
\DataTypeTok{Just} \DecValTok{4}
\NormalTok{ghci}\FunctionTok{>}\NormalTok{ halve }\DecValTok{7}
\DataTypeTok{Nothing}
\NormalTok{ghci}\FunctionTok{>}\NormalTok{ halve }\DecValTok{8} \FunctionTok{>>=}\NormalTok{ halve}
\DataTypeTok{Just} \DecValTok{2}
\NormalTok{ghci}\FunctionTok{>}\NormalTok{ halve }\DecValTok{7} \FunctionTok{>>=}\NormalTok{ halve}
\DataTypeTok{Nothing}                         \CommentTok{-- 1}
\NormalTok{ghci}\FunctionTok{>}\NormalTok{ halve }\DecValTok{6} \FunctionTok{>>=}\NormalTok{ halve}
\DataTypeTok{Nothing}
\NormalTok{ghci}\FunctionTok{>}\NormalTok{ halve }\DecValTok{6} \FunctionTok{>>=}\NormalTok{ halve }\FunctionTok{>>=}\NormalTok{ halve}
\DataTypeTok{Nothing}
\NormalTok{ghci}\FunctionTok{>}\NormalTok{ halve }\DecValTok{32} \FunctionTok{>>} \DataTypeTok{Nothing}
\DataTypeTok{Nothing}                         \CommentTok{-- 2}
\NormalTok{ghci}\FunctionTok{>}\NormalTok{ halve }\DecValTok{32} \FunctionTok{>>=}\NormalTok{ halve }\FunctionTok{>>=}\NormalTok{ halve }\FunctionTok{>>=}\NormalTok{ halve}
\DataTypeTok{Just} \DecValTok{2}
\NormalTok{ghci}\FunctionTok{>}\NormalTok{ halve }\DecValTok{32} \FunctionTok{>>} \DataTypeTok{Nothing} \FunctionTok{>>=}\NormalTok{ halve }\FunctionTok{>>=}\NormalTok{ halve }\FunctionTok{>>=}\NormalTok{ halve}
\DataTypeTok{Nothing}                         \CommentTok{-- 3}
\end{Highlighting}
\end{Shaded}

You can play with this yourself by
\href{https://github.com/mstksg/inCode/blob/master/code-samples/monad-plus/Halve.hs}{loading
up the function yourself}.

\begin{center}\rule{0.5\linewidth}{\linethickness}\end{center}

\textbf{Welcome to Haskell!}

In this article, code that begins with \texttt{ghci\textgreater{}} represents
commands to be entered at the interactive prompt, ghci. Code that doesn't is
actual source code.

\begin{center}\rule{0.5\linewidth}{\linethickness}\end{center}

Remember, \texttt{\textgreater{}\textgreater{}=} means ``use the results of the
last thing to calculate this next thing'' --- it ``chains'' the functions.

How does this work, exactly? That's not really in the scope of this article (any
monad tutorial will explain this in more detail). But here are some interesting
points:

\begin{enumerate}
\def\labelenumi{\arabic{enumi}.}
\tightlist
\item
  Note that this command doesn't even bother with the second \texttt{halve}. It
  knows that the end result will be \texttt{Nothing} no matter what (because
  \texttt{halve\ 7} is \texttt{Nothing}), so it just skips right past the second
  \texttt{halve}.
\item
  \texttt{\textgreater{}\textgreater{}} is a special variation of
  \texttt{\textgreater{}\textgreater{}=}. \texttt{\textgreater{}\textgreater{}=}
  says ``take the result of the last thing and use it on this'', while
  \texttt{\textgreater{}\textgreater{}} says ``ignore the result of the last
  thing and always return this''. So
  \texttt{\textgreater{}\textgreater{}\ Nothing} means ``I don't care what the
  last thing succeeded with, I'm going to fail right here.''
\item
  Disastrous! Even though halving 32 four times usually is fine (giving
  \texttt{Just\ 2}), having just one failure along the way means that the entire
  thing is a failure. \texttt{halve\ 32\ \textgreater{}\textgreater{}\ Nothing}
  is \texttt{Nothing}, so the whole thing is just
  \texttt{(Nothing)\ \textgreater{}\textgreater{}=\ halve\ \textgreater{}\textgreater{}=\ halve\ \textgreater{}\textgreater{}=\ halve}.
\end{enumerate}

You can think of this failing phenomenon like this: At every step, Haskell
attempts to apply \texttt{halve} to the result of the previous step. However,
you can't \texttt{halve} a \texttt{Nothing} because a \texttt{Nothing} has no
value to halve!

\hypertarget{do-notation}{%
\subsection{Do notation}\label{do-notation}}

Haskell provides a convenient way of writing chained
\texttt{\textgreater{}\textgreater{}=}'s called do notation; here are a few
samples matched with their equivalent \texttt{\textgreater{}\textgreater{}=}
form:

\begin{Shaded}
\begin{Highlighting}[]
\NormalTok{ghci}\FunctionTok{>}\NormalTok{ half }\DecValTok{8}
\DataTypeTok{Just} \DecValTok{4}
\NormalTok{ghci}\FunctionTok{>} \KeywordTok{do}\NormalTok{  half }\DecValTok{8}
\DataTypeTok{Just} \DecValTok{4}

\NormalTok{ghci}\FunctionTok{>}\NormalTok{ halve }\DecValTok{8} \FunctionTok{>>=}\NormalTok{ halve}
\DataTypeTok{Just} \DecValTok{2}
\NormalTok{ghci}\FunctionTok{>} \KeywordTok{do}\NormalTok{  x }\OtherTok{<-}\NormalTok{ halve }\DecValTok{8}
    \FunctionTok{|}\NormalTok{     halve x}
\DataTypeTok{Just} \DecValTok{2}

\NormalTok{ghci}\FunctionTok{>}\NormalTok{ halve }\DecValTok{32} \FunctionTok{>>=}\NormalTok{ halve }\FunctionTok{>>=}\NormalTok{ halve }\FunctionTok{>>=}\NormalTok{ halve}
\DataTypeTok{Just} \DecValTok{2}
\NormalTok{ghci}\FunctionTok{>} \KeywordTok{do}\NormalTok{  x }\OtherTok{<-}\NormalTok{ halve }\DecValTok{32}
    \FunctionTok{|}\NormalTok{     y }\OtherTok{<-}\NormalTok{ halve x}
    \FunctionTok{|}\NormalTok{     z }\OtherTok{<-}\NormalTok{ halve y}
    \FunctionTok{|}\NormalTok{     halve z}
\DataTypeTok{Just} \DecValTok{2}

\NormalTok{ghci}\FunctionTok{>}\NormalTok{ halve }\DecValTok{32} \FunctionTok{>>} \DataTypeTok{Nothing} \FunctionTok{>>=}\NormalTok{ halve }\FunctionTok{>>=}\NormalTok{ halve}
\DataTypeTok{Nothing}
\NormalTok{ghci}\FunctionTok{>} \KeywordTok{do}\NormalTok{  x }\OtherTok{<-}\NormalTok{ halve }\DecValTok{32}
    \FunctionTok{|}     \DataTypeTok{Nothing}
    \FunctionTok{|}\NormalTok{     y }\OtherTok{<-}\NormalTok{ halve x}
    \FunctionTok{|}\NormalTok{     z }\OtherTok{<-}\NormalTok{ halve y}
    \FunctionTok{|}\NormalTok{     halve z}
\DataTypeTok{Nothing}
\end{Highlighting}
\end{Shaded}

In this notation, \texttt{x}, \texttt{y}, and \texttt{z}'s do not contain the
\texttt{Just}/\texttt{Nothing}'s --- they represent the actual \textbf{Ints
inside them}, so we can so something like \texttt{halve\ x} (where
\texttt{halve} only takes Ints, not \texttt{Maybe\ Int}'s)

It kind of feels very imperative-y --- ``do \texttt{halve\ 32} and assign the
result (16) to \texttt{x}\ldots{}do \texttt{halve\ x} and assign the result (8)
to \texttt{y}\ldots{}'' --- but remember, it's still just a bunch of chained
\texttt{\textgreater{}\textgreater{}=}s in the end.

\hypertarget{failure-is-an-option}{%
\section{Failure is an option}\label{failure-is-an-option}}

The important thing to note here is that ``do'' notation basically builds up one
``giant'' object. Remember the last two examples --- the second to last one, all
of those lines were in an effort to build one giant \texttt{Just\ 2} value. In
the last example, all of those lines were in an effort to build one giant
\texttt{Nothing} value. That's why one \texttt{Nothing} ``ruined'' the whole
thing. The entire computation is one big \texttt{Maybe\ a}. If at any point in
your attempt to build that \texttt{Maybe\ a}, you fail, then you have
\texttt{Nothing}.

Now, remember, saying ``x is a monad'' just means ``we have defined a way of
chaining functions/operations on x''. Just like how we can now chain multiple
functions that return Maybe's (that don't take Maybe's as input). However, given
any object, there is probably more than one way to meaningfully define this
``chaining''.

Sometimes, it's useful to base your definition of chaining on the idea of a
failure/success process. Sometimes it's useful to define chaining as ``We are
building up either a success or a failure\ldots{}and if at any point I fail, the
whole thing is a failure''.

There is a special name for this design pattern. In Haskell, we call something
like this a
``\href{http://hackage.haskell.org/package/base-4.6.0.1/docs/Control-Monad.html\#t:MonadPlus}{MonadPlus}''\footnote{I
  have to give a fair disclaimer here. MonadPlus, as it is currently
  implemented, actually serves two functionalities/purposes. However, its
  functionality not related to success/failure is actually (except for a few
  cases) mostly redundant, due to another typeclass called
  \href{http://hackage.haskell.org/package/base-4.6.0.1/docs/Control-Applicative.html\#t:Alternative}{Alternative}
  that now handles it in all nearly all modern usage. The redundancy actually
  stems from one of the more famous embarassing misakes in the design of the
  Haskell standard library --- the infamous
  \href{http://www.haskell.org/haskellwiki/Functor-Applicative-Monad_Proposal}{monad-applicative-functor
  hierarchy issue}. In practice, however, simply using the appropriate typeclass
  for the appropriate property is the norm. For this article and this series, I
  will be addressing specifically this non-redundant functionality, the
  success/failureness; just be aware that in some other places, you will find
  other explanations of MonadPlus as a ``whole'' that includes the redundant
  parts.} \footnote{Actually, there is one noteworthy success/failure monad that
  isn't implemented as a MonadPlus in Haskell --- the Either. Arguably, Either
  embodies the ``spirit'' of MonadPlus; the problem is that Haskell requires
  that ``fail''/``mzero'' must not take any parameters, and Either must always
  have a ``reason'' when it fails. However, one could easily instance their own
  Either instance with a ``default reason'' if the \texttt{Left} type is known
  or constrained. The easiest way is to constrain the \texttt{Left} type to be a
  monoid and make \texttt{mzero\ =\ Left\ mempty}. Alternatively, if your Left
  is a String, you can just put in whatever default error message you want.}.

I know, it's an embarrassingly bad name, and it's like this is for historical
reasons (related to the footnote above). The name doesn't even hint at a
fail/succeedness. But we're stuck with it for pretty much the entire foreseeable
future, so when you chose to adopt a success/failure model for your chaining
process, you have a \emph{MonadPlus}.

There is a vocabulary we can use so we can talk about all MonadPlus's in a
general way:

\begin{itemize}
\tightlist
\item
  We call a success a ``return''. Yeah\ldots{}the name is super confusing
  because of how the word ``return'' is used in almost every other context in
  computer science. But hey. Oh well.
\item
  We call a failure an ``mzero''. Yes, this name is pretty lame too.
\end{itemize}

For Maybe, a ``return'' with the value \texttt{x} is \texttt{Just\ x}, and an
``mzero'' is a \texttt{Nothing}.

Something cool about Haskell is that if we type \texttt{return\ x}, it'll
interpret it as an auto-success of value \texttt{x}. If we type \texttt{mzero},
it'll be an ``alias'' of whatever your failure is.

That means that for Maybe, \texttt{return\ x} is the same as \texttt{Just\ x},
and \texttt{mzero} is an alias for \texttt{Nothing}.

\begin{center}\rule{0.5\linewidth}{\linethickness}\end{center}

\textbf{Welcome to Haskell!}

If you are familiar with object oriented languages like Java, MonadPlus is
really like an \textbf{interface}. That is, if something is a MonadPlus, there
is a ``guarantee'' that that something will implement/define \texttt{return} and
\texttt{mzero} for that particular object. In this way, \texttt{return} and
\texttt{mzero} are \emph{polymorphic functions} that change their behavior based
on what type you are talking about, and you can write code that works with all
MonadPlus's generically without worrying about their actual type by using only
\texttt{return} and \texttt{mzero} (instead of say, \texttt{Just} and
\texttt{Nothing}).

In Haskell, the term we use (instead of ``interface'') is
``\textbf{typeclass}''. There are some subtle differences --- typeclasses are in
general more powerful of a tool than interfaces --- but the two concepts provide
similar roles in their respective languages.

\begin{center}\rule{0.5\linewidth}{\linethickness}\end{center}

As a small note, the term/command ``return''/\texttt{return} is shared by all
monads. However, monads don't ascribe any (general) conceptual ``meaning'' or
``purpose'' to return. For any old monad, it can mean whatever you want it to
mean for that specific monad. However, in the context of MonadPlus, ``return''
has a very specific meaning: \emph{succeed}. Because of this, ``return'' and
``succeed'' will be treated as synonyms in this article.

\hypertarget{monadplus-examples}{%
\subsection{MonadPlus examples}\label{monadplus-examples}}

To see this in action, let's revisit the last do block and make it more generic,
and just rephrase it in a form that we are going to be encountering more when we
solve our problem with the List monad (which is (spoilers) also a MonadPlus):

\begin{Shaded}
\begin{Highlighting}[]
\OtherTok{halveThriceOops ::} \DataTypeTok{Int} \OtherTok{->} \DataTypeTok{Maybe} \DataTypeTok{Int}
\NormalTok{halveThriceOops n }\FunctionTok{=} \KeywordTok{do}          \CommentTok{-- call with n = 32}
\NormalTok{    x }\OtherTok{<-}\NormalTok{ halve n                }\CommentTok{-- Just 16              -- 1}
\NormalTok{    mzero                       }\CommentTok{-- Nothing              -- 2}
\NormalTok{    y }\OtherTok{<-}\NormalTok{ halve x                }\CommentTok{-- (skip)               -- 3}
\NormalTok{    z }\OtherTok{<-}\NormalTok{ halve y                }\CommentTok{-- (skip)}
\NormalTok{    return z                    }\CommentTok{-- (skip)               -- 4}
\end{Highlighting}
\end{Shaded}

Note that I've also included a line-by-line `trace' of the do block with what
the monad ``is'' at that point. It is what is calculated on that line, and it
would be the value returned if you just exited at that step.

\begin{enumerate}
\def\labelenumi{\arabic{enumi}.}
\tightlist
\item
  Business as usual. Halve \texttt{n} if possible and place the reuslt in
  \texttt{x}. If \texttt{n} is 32, then \texttt{x} will be 16.
\item
  The failure. Remember, \texttt{mzero} means ``fail here automatically'',
  which, in a Maybe object, means \texttt{Nothing}.
\item
  Now from here on, nothing else even matters\ldots{}the entire block is a
  failure!
\item
  If possible, succeed with the value in \texttt{z}. This is supposed to be a
  \texttt{Just} with the value of \texttt{z}. Unfortunately, the entire block
  failed a long time ago. So sad!
\end{enumerate}

\begin{center}\rule{0.5\linewidth}{\linethickness}\end{center}

\textbf{Diversion}

A small diversion.

This is a little nicety, but there is the common library monad function
\texttt{sequence\ ::\ Monad\ m\ =\textgreater{}\ {[}m\ a{]}\ -\textgreater{}\ m\ {[}a{]}},
which turns a \texttt{{[}Maybe\ a{]}} into a \texttt{Maybe\ {[}a{]}}.
Conceptually, \texttt{sequence} turns a list of monads into a monad containing a
list.

In the context of MonadPlus, it would be turning a list of Success/Failures into
a succesful or failed list. It builds a succesful/failed list.

From what we have learned, if any part of that building process is a failure,
the entire thing is necessarily a failure. This is reflected in
\texttt{sequence}:

\begin{Shaded}
\begin{Highlighting}[]
\NormalTok{ghci}\FunctionTok{>}\NormalTok{ sequence [}\DataTypeTok{Just} \DecValTok{1}\NormalTok{, }\DataTypeTok{Just} \DecValTok{4}\NormalTok{, }\DataTypeTok{Just} \DecValTok{6}\NormalTok{]}
\DataTypeTok{Just}\NormalTok{ [}\DecValTok{1}\NormalTok{,}\DecValTok{4}\NormalTok{,}\DecValTok{6}\NormalTok{]}
\NormalTok{ghci}\FunctionTok{>}\NormalTok{ sequence [}\DataTypeTok{Just} \DecValTok{1}\NormalTok{, }\DataTypeTok{Nothing}\NormalTok{, }\DataTypeTok{Just} \DecValTok{6}\NormalTok{]}
\DataTypeTok{Nothing}
\end{Highlighting}
\end{Shaded}

If you already know a few other common library monad functions (like
\texttt{replicateM}, \texttt{forM}, etc.), try reasoning about how they would
work on Maybe's and MonadPlus's in general --- they aren't just for IO!

\begin{center}\rule{0.5\linewidth}{\linethickness}\end{center}

\hypertarget{guards}{%
\subsection{Guards}\label{guards}}

It feels like just slapping in \texttt{mzero} willy-nilly is not that useful,
because then things just fail always no matter what. Wouldn't it be handy to
have a function that says ``fail right\ldots{}here, if this condition is not
met''? Like \texttt{mzero}, but instead of always failing, fails on certain
conditions.

Luckily, Haskell gives us one in the standard library:

\begin{Shaded}
\begin{Highlighting}[]
\OtherTok{guard ::} \DataTypeTok{MonadPlus}\NormalTok{ m }\OtherTok{=>} \DataTypeTok{Bool} \OtherTok{->}\NormalTok{ m ()        }\CommentTok{-- 1}
\NormalTok{guard }\DataTypeTok{True}  \FunctionTok{=}\NormalTok{ return ()}
\NormalTok{guard }\DataTypeTok{False} \FunctionTok{=}\NormalTok{ mzero}
\end{Highlighting}
\end{Shaded}

\begin{center}\rule{0.5\linewidth}{\linethickness}\end{center}

\textbf{Welcome to Haskell!}

\begin{enumerate}
\def\labelenumi{\arabic{enumi}.}
\item
  This is a type signature, like before. We say that \texttt{guard} is a
  function that takes a \texttt{Bool} and returns a \texttt{m\ ()} --- a monad
  containing \texttt{()}. But we say that \texttt{m}, the monad, must be a
  MonadPlus.

  For example, if we applied this to Maybe, the concrete signature would be
  \texttt{guard\ ::\ Bool\ -\textgreater{}\ Maybe\ ()}
\end{enumerate}

\begin{center}\rule{0.5\linewidth}{\linethickness}\end{center}

So \texttt{guard} will make sure a condition is met, or else it fails the entire
thing. If the condition is met, then it succeeds and places a \texttt{()} in the
value.

We can use this to re-implement \texttt{halve}, using do notation, aware of
Maybe's MonadPlus-ness:

\begin{Shaded}
\begin{Highlighting}[]
\OtherTok{halve ::} \DataTypeTok{Int} \OtherTok{->} \DataTypeTok{Maybe} \DataTypeTok{Int}
\NormalTok{halve n }\FunctionTok{=} \KeywordTok{do}                \CommentTok{-- <halve 8>   <halve 7>}
\NormalTok{    guard }\FunctionTok{$}\NormalTok{ even n          }\CommentTok{-- Just ()      Nothing}
\NormalTok{    return }\FunctionTok{$}\NormalTok{ n }\OtherTok{`div`} \DecValTok{2}      \CommentTok{-- Just 4       (skip)}
\end{Highlighting}
\end{Shaded}

\begin{center}\rule{0.5\linewidth}{\linethickness}\end{center}

\textbf{Welcome to Haskell!}

\texttt{guard\ \$\ even\ n} seems confusing, but it is just shorthand for
\texttt{guard\ (even\ n)}. We just don't like writing all those parentheses out.

\begin{center}\rule{0.5\linewidth}{\linethickness}\end{center}

So, first, \texttt{halve} is \texttt{Just\ ()} (succeeds with a blank value
\texttt{()}) if \texttt{n} is even, or else \texttt{Nothing} (fails
automatically) otherwise. Finally, if it has not yet failed, it attempts to
succeed with \texttt{n\ \textasciigrave{}div\textasciigrave{}\ 2}.

You can trust me when I say this works the exact same way! You can
\href{https://github.com/mstksg/inCode/blob/master/code-samples/monad-plus/HalveGuard.hs}{try
it out yourself}!

As a friendly reminder, this entire block is ``compiled''/desugared to:

\begin{Shaded}
\begin{Highlighting}[]
\NormalTok{halve}\OtherTok{ n ::} \DataTypeTok{Int} \OtherTok{->} \DataTypeTok{Maybe} \DataTypeTok{Int}
\NormalTok{halve n }\FunctionTok{=}\NormalTok{ guard (even n) }\FunctionTok{>>}\NormalTok{ return (n }\OtherTok{`div`} \DecValTok{2}\NormalTok{)}
\end{Highlighting}
\end{Shaded}

\hypertarget{a-practical-use}{%
\section{A practical use}\label{a-practical-use}}

We aren't where we need to be to begin tackling that Wolf/Goat/Cabbage puzzle
yet\ldots{}so to let this article not be a complete anticlimax (as a result of
my bad planning --- I had originally intended to do the entire three-part series
as one article), let's look at a practical problem that you can solve using the
Maybe monad.

The obvious examples are situations where it is useful to be able to chain
failable operations such as retrieving things from a database or a network
connection or applying partial functions (functions that only work on some
values, like our \texttt{halve}).

However, here is a neat one.

Let's say we are making a game where you can lose health by being hit or gain
health by picking up powerups. We want to calculate the final health at the end
of the game. It seems a bit easy: just add up all the losses and gains!
Unfortunately, it's not so simple --- it needs to be implemented such that if
your health ever dips below 0, you are dead. Forever. No powerups will ever help
you.

Think about how you would implement this normally. You might have a state object
that stores the current health as well as a flag with the current dead/alive
state, and at every step, check if the health is 0 or lower; if it is, swap the
flag to be dead and ignore all other updates.

But let's try doing this instead with the Maybe monad:

\begin{Shaded}
\begin{Highlighting}[]
\CommentTok{-- source: https://github.com/mstksg/inCode/tree/master/code-samples/monad-plus/MaybeGame.hs#L26-L51}

\CommentTok{-- die or fail immediately}
\OtherTok{die ::} \DataTypeTok{Maybe} \DataTypeTok{Int}
\NormalTok{die }\FunctionTok{=}\NormalTok{ mzero                         }\CommentTok{-- or die = mzero}

\CommentTok{-- if not dead, sets the health to the given level}
\OtherTok{setHealth ::} \DataTypeTok{Int} \OtherTok{->} \DataTypeTok{Maybe} \DataTypeTok{Int}
\NormalTok{setHealth n }\FunctionTok{=}\NormalTok{ return n              }\CommentTok{-- or setHealth n = return n}

\CommentTok{-- damage the player (from its previous health) and check for death}
\OtherTok{hit ::} \DataTypeTok{Int} \OtherTok{->} \DataTypeTok{Maybe} \DataTypeTok{Int}
\NormalTok{hit currHealth }\FunctionTok{=} \KeywordTok{do}
    \KeywordTok{let}\NormalTok{ newHealth }\FunctionTok{=}\NormalTok{ currHealth }\FunctionTok{-} \DecValTok{1}
\NormalTok{    guard }\FunctionTok{$}\NormalTok{ newHealth }\FunctionTok{>} \DecValTok{0}           \CommentTok{-- fail/die immediately unless newHealth}
                                    \CommentTok{--     is positive}
\NormalTok{    return newHealth                }\CommentTok{-- succeed with newHealth if not already}
                                    \CommentTok{--     dead}

\CommentTok{-- an alternative but identical definition of `hit`, using >>= and >>}
\OtherTok{hit' ::} \DataTypeTok{Int} \OtherTok{->} \DataTypeTok{Maybe} \DataTypeTok{Int}
\NormalTok{hit' currHealth }\FunctionTok{=}\NormalTok{ guard (newHealth }\FunctionTok{>} \DecValTok{0}\NormalTok{) }\FunctionTok{>>}\NormalTok{ return newHealth}
    \KeywordTok{where}
\NormalTok{        newHealth }\FunctionTok{=}\NormalTok{ currHealth }\FunctionTok{-} \DecValTok{1}

\CommentTok{-- increase the player's health from its previous health}
\OtherTok{powerup ::} \DataTypeTok{Int} \OtherTok{->} \DataTypeTok{Maybe} \DataTypeTok{Int}
\NormalTok{powerup currHealth }\FunctionTok{=}\NormalTok{ return }\FunctionTok{$}\NormalTok{ currHealth }\FunctionTok{+} \DecValTok{1}
\end{Highlighting}
\end{Shaded}

\begin{Shaded}
\begin{Highlighting}[]
\NormalTok{ghci}\FunctionTok{>}\NormalTok{ setHealth }\DecValTok{2} \FunctionTok{>>=}\NormalTok{ hit }\FunctionTok{>>=}\NormalTok{ powerup }\FunctionTok{>>=}\NormalTok{ hit }\FunctionTok{>>=}\NormalTok{ powerup }\FunctionTok{>>=}\NormalTok{ powerup}
\DataTypeTok{Just} \DecValTok{3}
\NormalTok{ghci}\FunctionTok{>}\NormalTok{ setHealth }\DecValTok{2} \FunctionTok{>>=}\NormalTok{ hit }\FunctionTok{>>=}\NormalTok{ powerup }\FunctionTok{>>=}\NormalTok{ hit }\FunctionTok{>>=}\NormalTok{ hit }\FunctionTok{>>=}\NormalTok{ powerup}
\DataTypeTok{Nothing}
\NormalTok{ghci}\FunctionTok{>}\NormalTok{ setHealth }\DecValTok{10} \FunctionTok{>>=}\NormalTok{ powerup }\FunctionTok{>>}\NormalTok{ die }\FunctionTok{>>=}\NormalTok{ powerup }\FunctionTok{>>=}\NormalTok{ powerup}
\DataTypeTok{Nothing}
\NormalTok{ghci}\FunctionTok{>} \KeywordTok{do}\NormalTok{  h0 }\OtherTok{<-}\NormalTok{ setHealth }\DecValTok{2}        \CommentTok{-- Just 2}
    \FunctionTok{|}\NormalTok{     h1 }\OtherTok{<-}\NormalTok{ hit h0             }\CommentTok{-- Just 1}
    \FunctionTok{|}\NormalTok{     h2 }\OtherTok{<-}\NormalTok{ powerup h1         }\CommentTok{-- Just 2}
    \FunctionTok{|}\NormalTok{     h3 }\OtherTok{<-}\NormalTok{ hit h2             }\CommentTok{-- Just 1}
    \FunctionTok{|}\NormalTok{     h4 }\OtherTok{<-}\NormalTok{ hit h3             }\CommentTok{-- Nothing}
    \FunctionTok{|}\NormalTok{     h5 }\OtherTok{<-}\NormalTok{ powerup h4         }\CommentTok{-- (skip)}
    \FunctionTok{|}\NormalTok{     h6 }\OtherTok{<-}\NormalTok{ powerup h5         }\CommentTok{-- (skip)}
    \FunctionTok{|}\NormalTok{     return h6                }\CommentTok{-- (skip)}
\DataTypeTok{Nothing}
\end{Highlighting}
\end{Shaded}

And voilà!
\href{https://github.com/mstksg/inCode/blob/master/code-samples/monad-plus/MaybeGame.hs}{Fire
it up yourself} if you want to test it out in person!

You can think of the last do block conceptually this way: remember, \texttt{h3}
does not represent the \texttt{Just\ 1} value --- \texttt{h3} represents the
number \emph{inside} the \texttt{Just\ 1} --- the 1. So \texttt{h4} is supposed
to represent the number inside its value, too. But because \texttt{hit\ h3}
results in \texttt{Nothing}; \texttt{Nothing} has no value ``inside'', so
\texttt{h4} doesn't even have a value! So obviously it doesn't even make sense
to call \texttt{powerup\ h4}\ldots{}therefore \texttt{h5} has no value either!
It's therefore meaningless to call \texttt{powerup\ h5}, so meaningless to
\texttt{return\ h6}\ldots{}the entire thing is a beautiful disaster. A fiasco.
Mission accomplished!

The whole thing works as expected; you can even die suddenly with \texttt{die},
which ignores your current health.

Interestingly enough, we could actually eliminate all references to Maybe
altogether by always using \texttt{return} and \texttt{mzero} instead of
\texttt{Just} and \texttt{Nothing}. And if we make our type signatures generic
enough, we could use this with \emph{any} MonadPlus! But that is for another
day.

\hypertarget{looking-forward}{%
\section{Looking forward}\label{looking-forward}}

Wow, who knew you could spend so much time talking about failure. Anyways, this
is a good place to stop before we move onto how List is also a MonadPlus. Okay,
so what have we learned?

\begin{itemize}
\tightlist
\item
  Monads are just a way of chaining functions on objects, and of course, every
  object's chaining process is different. In fact there might be even more than
  one way to meaningfully chain functions on an object!
\item
  One useful ``chaining approach'' is to model things as success-failure chains,
  where you are building something from successes, but if you fail once in the
  process, the entire process fails. An object that uses this approach/design
  pattern is called a MonadPlus.
\item
  The Maybe object is one such example. We can define `chaining' failable
  functions as functions that continue if the previous function succeeded, or
  propagate a failure if the previous function fails. A failable function, for a
  Maybe object, is a function \texttt{::\ a\ -\textgreater{}\ Maybe\ b} or even
  \texttt{::\ Maybe\ b}.
\item
  There is a common vocabulary for talking about MonadPlus concepts ---
  ``return'' means ``succeed with this value'', and ``mzero'' means ``fail
  now''.
\item
  Due to Haskell's polymorphism, we can ``forget'' we are using Maybe and in
  fact talk about/write for ``general'' MonadPlus's, with \texttt{return\ x} and
  \texttt{mzero} resulting in the appropriate success/fail objects.
\end{itemize}

For the mean time, think about how it might make sense to chain operations on
lists (ie, repeatedly applying functions
\texttt{::\ a\ -\textgreater{}\ {[}b{]}} to lists).

By this, I mean, given a function that turns a value into a list of values
\texttt{f\ ::\ a\ -\textgreater{}\ {[}b{]}}, find a way to meaningfully
``chain'' that function to a previous list and get a new list:

\begin{Shaded}
\begin{Highlighting}[]
\NormalTok{ghci}\FunctionTok{>}\NormalTok{ oldList }\FunctionTok{>>=}\NormalTok{ f}
\NormalTok{newList             }\CommentTok{-- a new list based on old list; f "chained" to `oldList`.}
\end{Highlighting}
\end{Shaded}

Is there more than one way to think about chaining them, even? And in what ways
we can define this ``chaining'' to represent success/failure? Until next time!

\end{document}
