\documentclass[]{article}
\usepackage{lmodern}
\usepackage{amssymb,amsmath}
\usepackage{ifxetex,ifluatex}
\usepackage{fixltx2e} % provides \textsubscript
\ifnum 0\ifxetex 1\fi\ifluatex 1\fi=0 % if pdftex
  \usepackage[T1]{fontenc}
  \usepackage[utf8]{inputenc}
\else % if luatex or xelatex
  \ifxetex
    \usepackage{mathspec}
    \usepackage{xltxtra,xunicode}
  \else
    \usepackage{fontspec}
  \fi
  \defaultfontfeatures{Mapping=tex-text,Scale=MatchLowercase}
  \newcommand{\euro}{€}
\fi
% use upquote if available, for straight quotes in verbatim environments
\IfFileExists{upquote.sty}{\usepackage{upquote}}{}
% use microtype if available
\IfFileExists{microtype.sty}{\usepackage{microtype}}{}
\usepackage[margin=1in]{geometry}
\ifxetex
  \usepackage[setpagesize=false, % page size defined by xetex
              unicode=false, % unicode breaks when used with xetex
              xetex]{hyperref}
\else
  \usepackage[unicode=true]{hyperref}
\fi
\hypersetup{breaklinks=true,
            bookmarks=true,
            pdfauthor={Justin Le},
            pdftitle={In Memory of Ertugrul Söylemez (1985 -- 2018)},
            colorlinks=true,
            citecolor=blue,
            urlcolor=blue,
            linkcolor=magenta,
            pdfborder={0 0 0}}
\urlstyle{same}  % don't use monospace font for urls
% Make links footnotes instead of hotlinks:
\renewcommand{\href}[2]{#2\footnote{\url{#1}}}
\setlength{\parindent}{0pt}
\setlength{\parskip}{6pt plus 2pt minus 1pt}
\setlength{\emergencystretch}{3em}  % prevent overfull lines
\setcounter{secnumdepth}{0}

\title{In Memory of Ertugrul Söylemez (1985 -- 2018)}
\author{Justin Le}

\begin{document}
\maketitle

\emph{Originally posted on
\textbf{\href{https://blog.jle.im/entry/in-memory-of-ertugrul-soylemez.html}{in
Code}}.}

I found out
\href{https://byorgey.wordpress.com/2018/05/21/ertugrul-soylemez-1985-2018/}{come
to my attention} recently that \href{http://ertes.eu/about.html}{Ertugrul
Solemez} has passed away suddenly. Many have come forward to express their
sadness about this passing and how much this has impacted the Haskell community,
and how much of a loss it is for functional programming at large.

They aren't wrong; Ertugrul was one of the faces of the friendly, warm,
encouraging, patient Haskell teaching that Haskell has grown to be known for.
Ertugrul was also one of the original pioneers in the implementation and theory
of Functional Reactive Programming and continued to innovate even through this
year. His name is now and forever will be synonymous with the ``pull-based''
variant of functional reactive programming. And the freenode \#haskell channel
and the Haskell community at large will have one less friendly face who always
enjoys helping new people learn.

However, I wanted to just put some words down about his personal influence in my
Haskell, Academic, and FOSS career.

When I was a new Haskeller, a lot of things confused me. But the passion of
people like Ertugrul to help me understand concepts that I found interesting
late into the night was one of the things that really made it worth it.

One of these lead to the creation of my first ever Haskell library --
\emph{\href{https://hackage.haskell.org/package/auto}{auto}}. \emph{auto} is
basically literally a direct translation of one of our conversations (and
somewhat of a derivative of his own library
\emph{\href{https://hackage.haskell.org/package/netwire}{netwire}}), and
throughout the entire implementation process he was open to the many questions I
had. And some of the features of the library (including implicit serialization)
were directly his innovations put into practice.

In a slightly different context --- as a new PhD student, I was told to follow
wherever my curiosity lead me. One of those lines lead me to ``comonadic'' image
processing, which was directly inspired by
\href{https://hub.darcs.net/ertes/articles/browse/media-processing.lhs}{this
unfinished article} of his.

Thanks to this (and along with more help from him) I gave my first graduate
research presentation. And, following from this, I submitted my work to a CFP
for a conference and was \href{http://talks.jle.im/lambdaconf-2016/}{accepted to
give my first ever conference talk}.

I've come along way since those times. I now have fifteen packages published on
Hackage, and a few more in the pipeline. I've given
\href{http://talks.jle.im/}{many more research talks} all over the world, as far
as Ukraine. I'm no longer a fresh PhD student new to research; I'm a PhD
candidate getting ready to defend my dissertation four years in the making in
the upcoming months. I'm still learning Haskell, but inspired by an example of a
helpful Haskell attitude, I maintain a Haskell blog year after year that I hope
people have, too, found helpful.

But I know I'll always remember my roots --- and the person who inspired and
helped me with my first Haskell questions, my first library, and my first
graduate research presentations. I can say with confidence that there is no
single person who has personally influenced my Haskell career more. Ertugrul was
the mentor I was blessed to have, and I know anyone who has worked with him or
talked to him feels the same way.

Haskell and Functional programming lost a great innovator and educator, but I
know a lot of people will say that they, like me, have lost a great mentor and
friend.

Rest in peace, friend.

\end{document}
