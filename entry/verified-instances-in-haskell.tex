\documentclass[]{article}
\usepackage{lmodern}
\usepackage{amssymb,amsmath}
\usepackage{ifxetex,ifluatex}
\usepackage{fixltx2e} % provides \textsubscript
\ifnum 0\ifxetex 1\fi\ifluatex 1\fi=0 % if pdftex
  \usepackage[T1]{fontenc}
  \usepackage[utf8]{inputenc}
\else % if luatex or xelatex
  \ifxetex
    \usepackage{mathspec}
    \usepackage{xltxtra,xunicode}
  \else
    \usepackage{fontspec}
  \fi
  \defaultfontfeatures{Mapping=tex-text,Scale=MatchLowercase}
  \newcommand{\euro}{€}
\fi
% use upquote if available, for straight quotes in verbatim environments
\IfFileExists{upquote.sty}{\usepackage{upquote}}{}
% use microtype if available
\IfFileExists{microtype.sty}{\usepackage{microtype}}{}
\usepackage[margin=1in]{geometry}
\usepackage{color}
\usepackage{fancyvrb}
\newcommand{\VerbBar}{|}
\newcommand{\VERB}{\Verb[commandchars=\\\{\}]}
\DefineVerbatimEnvironment{Highlighting}{Verbatim}{commandchars=\\\{\}}
% Add ',fontsize=\small' for more characters per line
\newenvironment{Shaded}{}{}
\newcommand{\AlertTok}[1]{\textcolor[rgb]{1.00,0.00,0.00}{\textbf{#1}}}
\newcommand{\AnnotationTok}[1]{\textcolor[rgb]{0.38,0.63,0.69}{\textbf{\textit{#1}}}}
\newcommand{\AttributeTok}[1]{\textcolor[rgb]{0.49,0.56,0.16}{#1}}
\newcommand{\BaseNTok}[1]{\textcolor[rgb]{0.25,0.63,0.44}{#1}}
\newcommand{\BuiltInTok}[1]{#1}
\newcommand{\CharTok}[1]{\textcolor[rgb]{0.25,0.44,0.63}{#1}}
\newcommand{\CommentTok}[1]{\textcolor[rgb]{0.38,0.63,0.69}{\textit{#1}}}
\newcommand{\CommentVarTok}[1]{\textcolor[rgb]{0.38,0.63,0.69}{\textbf{\textit{#1}}}}
\newcommand{\ConstantTok}[1]{\textcolor[rgb]{0.53,0.00,0.00}{#1}}
\newcommand{\ControlFlowTok}[1]{\textcolor[rgb]{0.00,0.44,0.13}{\textbf{#1}}}
\newcommand{\DataTypeTok}[1]{\textcolor[rgb]{0.56,0.13,0.00}{#1}}
\newcommand{\DecValTok}[1]{\textcolor[rgb]{0.25,0.63,0.44}{#1}}
\newcommand{\DocumentationTok}[1]{\textcolor[rgb]{0.73,0.13,0.13}{\textit{#1}}}
\newcommand{\ErrorTok}[1]{\textcolor[rgb]{1.00,0.00,0.00}{\textbf{#1}}}
\newcommand{\ExtensionTok}[1]{#1}
\newcommand{\FloatTok}[1]{\textcolor[rgb]{0.25,0.63,0.44}{#1}}
\newcommand{\FunctionTok}[1]{\textcolor[rgb]{0.02,0.16,0.49}{#1}}
\newcommand{\ImportTok}[1]{#1}
\newcommand{\InformationTok}[1]{\textcolor[rgb]{0.38,0.63,0.69}{\textbf{\textit{#1}}}}
\newcommand{\KeywordTok}[1]{\textcolor[rgb]{0.00,0.44,0.13}{\textbf{#1}}}
\newcommand{\NormalTok}[1]{#1}
\newcommand{\OperatorTok}[1]{\textcolor[rgb]{0.40,0.40,0.40}{#1}}
\newcommand{\OtherTok}[1]{\textcolor[rgb]{0.00,0.44,0.13}{#1}}
\newcommand{\PreprocessorTok}[1]{\textcolor[rgb]{0.74,0.48,0.00}{#1}}
\newcommand{\RegionMarkerTok}[1]{#1}
\newcommand{\SpecialCharTok}[1]{\textcolor[rgb]{0.25,0.44,0.63}{#1}}
\newcommand{\SpecialStringTok}[1]{\textcolor[rgb]{0.73,0.40,0.53}{#1}}
\newcommand{\StringTok}[1]{\textcolor[rgb]{0.25,0.44,0.63}{#1}}
\newcommand{\VariableTok}[1]{\textcolor[rgb]{0.10,0.09,0.49}{#1}}
\newcommand{\VerbatimStringTok}[1]{\textcolor[rgb]{0.25,0.44,0.63}{#1}}
\newcommand{\WarningTok}[1]{\textcolor[rgb]{0.38,0.63,0.69}{\textbf{\textit{#1}}}}
\ifxetex
  \usepackage[setpagesize=false, % page size defined by xetex
              unicode=false, % unicode breaks when used with xetex
              xetex]{hyperref}
\else
  \usepackage[unicode=true]{hyperref}
\fi
\hypersetup{breaklinks=true,
            bookmarks=true,
            pdfauthor={Justin Le},
            pdftitle={Verify your Typeclass Instances in Haskell Today!},
            colorlinks=true,
            citecolor=blue,
            urlcolor=blue,
            linkcolor=magenta,
            pdfborder={0 0 0}}
\urlstyle{same}  % don't use monospace font for urls
% Make links footnotes instead of hotlinks:
\renewcommand{\href}[2]{#2\footnote{\url{#1}}}
\setlength{\parindent}{0pt}
\setlength{\parskip}{6pt plus 2pt minus 1pt}
\setlength{\emergencystretch}{3em}  % prevent overfull lines
\setcounter{secnumdepth}{0}

\title{Verify your Typeclass Instances in Haskell Today!}
\author{Justin Le}
\date{April 1, 2017}

\begin{document}
\maketitle

\emph{Originally posted on
\textbf{\href{https://blog.jle.im/entry/verified-instances-in-haskell.html}{in
Code}}.}

One of the most common gripes people have when learning Haskell is the fact that
typeclass ``laws'' are only laws by convention, and aren't enforced by the
language and compiler. When asked why, the typical response is ``Haskell can't
do that'', followed by a well-intentioned redirection to quickcheck or some
other fuzzing library.

But, to any experienced Haskeller, ``Haskell's type system can't express X'' is
always interpreted as a (personal) challenge.

GHC Haskell's type system has been advanced enough to provide verified
typeclasses for a long time, since the introduction of data kinds and associated
types. And with the
\emph{\href{http://hackage.haskell.org/package/singletons}{singletons}} library,
it's now as easy as ever.

(The code for this post is available
\href{https://github.com/mstksg/inCode/tree/master/code-samples/verified-instances/VerifiedInstances.hs}{here}
if you want to follow along!)

\hypertarget{semigroups}{%
\section{Semigroups}\label{semigroups}}

Let's start simple -- everyone's favorite structural addition to magmas,
\href{http://hackage.haskell.org/package/base-4.9.1.0/docs/Data-Semigroup.html}{semigroups}.
A semigroup is a type with an associative binary operation,
\texttt{(\textless{}\textgreater{})}:

\begin{Shaded}
\begin{Highlighting}[]
\KeywordTok{class} \DataTypeTok{Semigroup}\NormalTok{ a }\KeywordTok{where}
\OtherTok{    (<>) ::}\NormalTok{ a }\OtherTok{->}\NormalTok{ a }\OtherTok{->}\NormalTok{ a}
\end{Highlighting}
\end{Shaded}

Its one law is associativity:

\begin{Shaded}
\begin{Highlighting}[]
\NormalTok{(x }\FunctionTok{<>}\NormalTok{ y) }\FunctionTok{<>}\NormalTok{ z }\FunctionTok{=}\NormalTok{ x }\FunctionTok{<>}\NormalTok{ (y }\FunctionTok{<>}\NormalTok{ z)}
\end{Highlighting}
\end{Shaded}

But, this class stinks, because it's super easy to write bad instances:

\begin{Shaded}
\begin{Highlighting}[]
\KeywordTok{data} \DataTypeTok{List}\NormalTok{ a }\FunctionTok{=} \DataTypeTok{Nil} \FunctionTok{|} \DataTypeTok{Cons}\NormalTok{ a (}\DataTypeTok{List}\NormalTok{ a)}
    \KeywordTok{deriving} \DataTypeTok{Show}

\KeywordTok{infixr} \DecValTok{5} \OtherTok{`Cons`}

\KeywordTok{instance} \DataTypeTok{Semigroup}\NormalTok{ (}\DataTypeTok{List}\NormalTok{ a) }\KeywordTok{where}
    \DataTypeTok{Nil}       \FunctionTok{<>}\NormalTok{ ys }\FunctionTok{=}\NormalTok{ ys}
    \DataTypeTok{Cons}\NormalTok{ x xs }\FunctionTok{<>}\NormalTok{ ys }\FunctionTok{=} \DataTypeTok{Cons}\NormalTok{ x (ys }\FunctionTok{<>}\NormalTok{ xs)}
\end{Highlighting}
\end{Shaded}

This instance isn't associative:

\begin{Shaded}
\begin{Highlighting}[]
\NormalTok{ghci}\FunctionTok{>}\NormalTok{ ((}\DecValTok{1} \OtherTok{`Cons`} \DecValTok{2} \OtherTok{`Cons`} \DataTypeTok{Nil}\NormalTok{) }\FunctionTok{<>}\NormalTok{ (}\DecValTok{3} \OtherTok{`Cons`} \DecValTok{4} \OtherTok{`Cons`} \DataTypeTok{Nil}\NormalTok{)) }\FunctionTok{<>}\NormalTok{ (}\DecValTok{5} \OtherTok{`Cons`} \DecValTok{6} \OtherTok{`Cons`} \DataTypeTok{Nil}\NormalTok{)}
\DecValTok{1} \OtherTok{`Cons`} \DecValTok{5} \OtherTok{`Cons`} \DecValTok{3} \OtherTok{`Cons`} \DecValTok{6} \OtherTok{`Cons`} \DecValTok{2} \OtherTok{`Cons`} \DecValTok{4} \OtherTok{`Cons`} \DataTypeTok{Nil}
\NormalTok{ghci}\FunctionTok{>}\NormalTok{ (}\DecValTok{1} \OtherTok{`Cons`} \DecValTok{2} \OtherTok{`Cons`} \DataTypeTok{Nil}\NormalTok{) }\FunctionTok{<>}\NormalTok{ ((}\DecValTok{3} \OtherTok{`Cons`} \DecValTok{4} \OtherTok{`Cons`} \DataTypeTok{Nil}\NormalTok{) }\FunctionTok{<>}\NormalTok{ (}\DecValTok{5} \OtherTok{`Cons`} \DecValTok{6} \OtherTok{`Cons`} \DataTypeTok{Nil}\NormalTok{))}
\DecValTok{1} \OtherTok{`Cons`} \DecValTok{3} \OtherTok{`Cons`} \DecValTok{2} \OtherTok{`Cons`} \DecValTok{5} \OtherTok{`Cons`} \DecValTok{4} \OtherTok{`Cons`} \DecValTok{6} \OtherTok{`Cons`} \DataTypeTok{Nil}
\end{Highlighting}
\end{Shaded}

But if you try to compile it, GHC doesn't complain at all. Is this an error on
the part of Haskell? Not quite; it's an error on the part of the
\texttt{Semigroup} typeclass not requiring proofs that the instance is indeed
associative.

Let's try again.

\hypertarget{verify-me-captain}{%
\subsection{Verify me, Captain}\label{verify-me-captain}}

We will now define \texttt{Semigroup} on the \emph{kind} \texttt{List}, using
\texttt{-XDataKinds}, instead of the type.

\begin{Shaded}
\begin{Highlighting}[]
\KeywordTok{class} \DataTypeTok{Semigroup}\NormalTok{ a }\KeywordTok{where}
    \KeywordTok{type}\NormalTok{ (}\OtherTok{x ::}\NormalTok{ a) }\FunctionTok{<>}\NormalTok{ (}\OtherTok{y ::}\NormalTok{ a)}\OtherTok{ ::}\NormalTok{ a}

\OtherTok{    (%<>) ::} \DataTypeTok{Sing}\NormalTok{ (}\OtherTok{x ::}\NormalTok{ a) }\OtherTok{->} \DataTypeTok{Sing}\NormalTok{ (}\OtherTok{y ::}\NormalTok{ a) }\OtherTok{->} \DataTypeTok{Sing}\NormalTok{ (x }\FunctionTok{<>}\NormalTok{ y)}

\NormalTok{    appendAssoc}
\OtherTok{        ::} \DataTypeTok{Sing}\NormalTok{ (}\OtherTok{x ::}\NormalTok{ a)}
        \OtherTok{->} \DataTypeTok{Sing}\NormalTok{ (}\OtherTok{y ::}\NormalTok{ a)}
        \OtherTok{->} \DataTypeTok{Sing}\NormalTok{ (}\OtherTok{z ::}\NormalTok{ a)}
        \OtherTok{->}\NormalTok{ ((x }\FunctionTok{<>}\NormalTok{ y) }\FunctionTok{<>}\NormalTok{ z) }\FunctionTok{:~:}\NormalTok{ (x }\FunctionTok{<>}\NormalTok{ (y }\FunctionTok{<>}\NormalTok{ z))}
\end{Highlighting}
\end{Shaded}

Now, \texttt{\textless{}\textgreater{}} exists not as a function on
\emph{values}, but as a function on \emph{types}.
\texttt{\%\textless{}\textgreater{}} is a function that performs
\texttt{\textless{}\textgreater{}} at the value level, written to work with
singletons representing the input types, so that GHC can verify that it is
identical to the type family \texttt{\textless{}\textgreater{}}. (it's 100\%
boilerplate and should pretty much exactly match the
\texttt{\textless{}\textgreater{}} type family).\footnote{In full
  \emph{singletons} style, this should actually be expressed in terms of the the
  \emph{partially applied} (defunctionalized)
  \texttt{\textless{}\textgreater{}}. However, I'm giving the
  non-defunctionalized versions here for clarity.} Finally, \texttt{appendAssoc}
is a proof that the type family \texttt{\textless{}\textgreater{}} is
associative, using \texttt{:\textasciitilde{}:} (type equality witness) from
\texttt{Data.Type.Equality}.

This means that, if a type is an instance of \texttt{Semigroup}, it not only has
to provide
\texttt{\textless{}\textgreater{}}/\texttt{\%\textless{}\textgreater{}}, but
also a \emph{proof that they are associative}. You can't write the full instance
without it!

\texttt{Semigroup} is a ``kind-class'', because it is a bunch of methods and
types associated with a certain kind. Which \texttt{\textless{}\textgreater{}}
is dispatched when you do something like
\texttt{x\ \textless{}\textgreater{}\ y} depends on the \emph{kind} of
\texttt{x} and \texttt{y}. GHC does ``kind inference'' and uses the
\texttt{\textless{}\textgreater{}} corresponding to the kinds of \texttt{x} and
\texttt{y}.

Using the \texttt{SingKind} typeclass from the singletons library, we can move
back and forth from \texttt{Sing\ x} and \texttt{x}, and get our original
(value-level) \texttt{\textless{}\textgreater{}} back:

\begin{Shaded}
\begin{Highlighting}[]
\NormalTok{(}\FunctionTok{<>}\NormalTok{)}
\OtherTok{    ::}\NormalTok{ (}\DataTypeTok{SingKind}\NormalTok{ m, }\DataTypeTok{Semigroup}\NormalTok{ m)}
    \OtherTok{=>} \DataTypeTok{Demote}\NormalTok{ m}
    \OtherTok{->} \DataTypeTok{Demote}\NormalTok{ m}
    \OtherTok{->} \DataTypeTok{Demote}\NormalTok{ m}
\NormalTok{x }\FunctionTok{<>}\NormalTok{ y }\FunctionTok{=}\NormalTok{ withSomeSing x }\FunctionTok{$}\NormalTok{ \textbackslash{}sX }\OtherTok{->}
\NormalTok{           withSomeSing y }\FunctionTok{$}\NormalTok{ \textbackslash{}sY }\OtherTok{->}
\NormalTok{             fromSing (sX }\FunctionTok{%<>}\NormalTok{ sY)}
\end{Highlighting}
\end{Shaded}

Now, let's write the instance for \texttt{List}. First, we need to define the
singletons:

\begin{Shaded}
\begin{Highlighting}[]
\KeywordTok{data} \KeywordTok{instance} \DataTypeTok{Sing}\NormalTok{ (}\OtherTok{xs ::} \DataTypeTok{List}\NormalTok{ a) }\KeywordTok{where}
    \DataTypeTok{SNil}\OtherTok{  ::} \DataTypeTok{Sing} \DataTypeTok{Nil}
    \DataTypeTok{SCons}\OtherTok{ ::} \DataTypeTok{Sing}\NormalTok{ x }\OtherTok{->} \DataTypeTok{Sing}\NormalTok{ xs }\OtherTok{->} \DataTypeTok{Sing}\NormalTok{ (}\DataTypeTok{Cons}\NormalTok{ x xs)}
\end{Highlighting}
\end{Shaded}

Then, we can define the instance, using the traditional \texttt{(++)} appending
that lists famously have:

\begin{Shaded}
\begin{Highlighting}[]
\KeywordTok{instance} \DataTypeTok{Semigroup}\NormalTok{ (}\DataTypeTok{List}\NormalTok{ a) }\KeywordTok{where}
    \KeywordTok{type} \DataTypeTok{Nil}       \FunctionTok{<>}\NormalTok{ ys }\FunctionTok{=}\NormalTok{ ys}
    \KeywordTok{type} \DataTypeTok{Cons}\NormalTok{ x xs }\FunctionTok{<>}\NormalTok{ ys }\FunctionTok{=} \DataTypeTok{Cons}\NormalTok{ x (xs }\FunctionTok{<>}\NormalTok{ ys)}

    \DataTypeTok{SNil}       \FunctionTok{%<>}\NormalTok{ ys }\FunctionTok{=}\NormalTok{ ys}
    \DataTypeTok{SCons}\NormalTok{ x xs }\FunctionTok{%<>}\NormalTok{ ys }\FunctionTok{=} \DataTypeTok{SCons}\NormalTok{ x (xs }\FunctionTok{%<>}\NormalTok{ ys)}

\NormalTok{    appendAssoc }\FunctionTok{=}\NormalTok{ \textbackslash{}}\KeywordTok{case}
      \DataTypeTok{SNil}       \OtherTok{->}\NormalTok{ \textbackslash{}_ _ }\OtherTok{->} \DataTypeTok{Refl}
      \DataTypeTok{SCons}\NormalTok{ x xs }\OtherTok{->}\NormalTok{ \textbackslash{}ys zs }\OtherTok{->}
        \KeywordTok{case}\NormalTok{ appendAssoc xs ys zs }\KeywordTok{of}
          \DataTypeTok{Refl} \OtherTok{->} \DataTypeTok{Refl}
\end{Highlighting}
\end{Shaded}

Like I promised, \texttt{\%\textless{}\textgreater{}} is a boilerplate
re-implementation of \texttt{\textless{}\textgreater{}}, to manipulate
value-level witnesses. \texttt{appendAssoc} is the interesting bit: It's our
proof. It reads like this:

\begin{enumerate}
\def\labelenumi{\arabic{enumi}.}
\item
  If the first list is \texttt{Nil}:

\begin{Shaded}
\begin{Highlighting}[]
\CommentTok{-- left hand side}
\NormalTok{(}\DataTypeTok{Nil} \FunctionTok{<>}\NormalTok{ ys) }\FunctionTok{<>}\NormalTok{ zs}
  \FunctionTok{=}\NormalTok{ ys }\FunctionTok{<>}\NormalTok{ zs        }\CommentTok{-- definition of `(Nil <>)`}
\CommentTok{-- right hand side}
\DataTypeTok{Nil} \FunctionTok{<>}\NormalTok{ (ys }\FunctionTok{<>}\NormalTok{ zs)}
  \FunctionTok{=}\NormalTok{ ys }\FunctionTok{<>}\NormalTok{ zs        }\CommentTok{-- definition of `(Nil <>)`}
\end{Highlighting}
\end{Shaded}

  So, no work needed. QED! (Or, as we say in Haskell, \texttt{Refl}!)
\item
  If the first list is \texttt{Cons\ x\ xs}:

\begin{Shaded}
\begin{Highlighting}[]
\CommentTok{-- left hand side}
\NormalTok{(}\DataTypeTok{Cons}\NormalTok{ x xs }\FunctionTok{<>}\NormalTok{ ys) }\FunctionTok{<>}\NormalTok{ zs}
  \FunctionTok{=}\NormalTok{ (}\DataTypeTok{Cons}\NormalTok{ x (xs }\FunctionTok{<>}\NormalTok{ ys)) }\FunctionTok{<>}\NormalTok{ zs   }\CommentTok{-- definition of `(Cons x xs <>)`}
  \FunctionTok{=} \DataTypeTok{Cons}\NormalTok{ x ((xs }\FunctionTok{<>}\NormalTok{ ys) }\FunctionTok{<>}\NormalTok{ zs)   }\CommentTok{-- definition of `(Cons x xs <>)`}
\CommentTok{-- right hand side}
\DataTypeTok{Cons}\NormalTok{ x xs }\FunctionTok{<>}\NormalTok{ (ys }\FunctionTok{<>}\NormalTok{ zs)}
  \FunctionTok{=} \DataTypeTok{Cons}\NormalTok{ x (xs }\FunctionTok{<>}\NormalTok{ (ys }\FunctionTok{<>}\NormalTok{ zs))   }\CommentTok{-- definition of `(Cons x xs <>)`}
\end{Highlighting}
\end{Shaded}

  So, the problem reduces to proving that
  \texttt{(xs\ \textless{}\textgreater{}\ ys)\ \textless{}\textgreater{}\ zs} is
  equal to
  \texttt{xs\ \textless{}\textgreater{}\ (ys\ \textless{}\textgreater{}\ zs)}.
  If we can do that, then we can prove that the whole things are equal. We
  generate that proof using \texttt{appendAssoc\ xs\ ys\ zs}, and, wit that
  proof in scope\ldots{}QED!
\end{enumerate}

And, we're done!

Note that if you had tried any \emph{non-associative} implementation of
\texttt{\textless{}\textgreater{}} (and \texttt{\%\textless{}\textgreater{}}),
GHC would reject it because you wouldn't have been able to write the proof!

\hypertarget{automatic-singletons}{%
\subsubsection{Automatic Singletons}\label{automatic-singletons}}

Deriving \texttt{Sing} and \texttt{SingKind} and both versions of
\texttt{\textless{}\textgreater{}} is kind of tedious, so it's useful to use
template haskell to do it all for us:

\begin{Shaded}
\begin{Highlighting}[]
\FunctionTok{$}\NormalTok{(singletons [d|}
\NormalTok{  data List a = Nil | Cons a (List a)}
\NormalTok{      deriving (Show)}

\NormalTok{  infixr 5 `Cons`}

\NormalTok{  appendList :: List a -> List a -> List a}
\NormalTok{  appendList Nil         ys = ys}
\NormalTok{  appendList (Cons x xs) ys = Cons x (appendList xs ys)}
\NormalTok{  |])}

\KeywordTok{instance} \DataTypeTok{Semigroup}\NormalTok{ (}\DataTypeTok{List}\NormalTok{ a) }\KeywordTok{where}
    \KeywordTok{type}\NormalTok{ xs }\FunctionTok{<>}\NormalTok{ ys }\FunctionTok{=} \DataTypeTok{AppendList}\NormalTok{ xs ys}
\NormalTok{    (}\FunctionTok{%<>}\NormalTok{) }\FunctionTok{=}\NormalTok{ sAppendList}

\NormalTok{    appendAssoc }\FunctionTok{=}\NormalTok{ \textbackslash{}}\KeywordTok{case}
      \DataTypeTok{SNil}       \OtherTok{->}\NormalTok{ \textbackslash{}_ _ }\OtherTok{->} \DataTypeTok{Refl}
      \DataTypeTok{SCons}\NormalTok{ _ xs }\OtherTok{->}\NormalTok{ \textbackslash{}ys zs }\OtherTok{->}
        \KeywordTok{case}\NormalTok{ appendAssoc xs ys zs }\KeywordTok{of}
          \DataTypeTok{Refl} \OtherTok{->} \DataTypeTok{Refl}
\end{Highlighting}
\end{Shaded}

The boilerplate of re-defining \texttt{\textless{}\textgreater{}} as
\texttt{\%\textless{}\textgreater{}} goes away!

And now, we we can do:

\begin{Shaded}
\begin{Highlighting}[]
\NormalTok{ghci}\FunctionTok{>}\NormalTok{ print }\FunctionTok{$}\NormalTok{ ((}\DecValTok{1}\OtherTok{::}\DataTypeTok{Integer}\NormalTok{) }\OtherTok{`Cons`} \DecValTok{2} \OtherTok{`Cons`} \DataTypeTok{Nil}\NormalTok{) }\FunctionTok{<>}\NormalTok{ (}\DecValTok{3} \OtherTok{`Cons`} \DecValTok{4} \OtherTok{`Cons`} \DataTypeTok{Nil}\NormalTok{)}
\DecValTok{1} \OtherTok{`Cons`} \DecValTok{2} \OtherTok{`Cons`} \DecValTok{3} \OtherTok{`Cons`} \DecValTok{4} \OtherTok{`Cons`} \DataTypeTok{Nil}
\end{Highlighting}
\end{Shaded}

Ta dah!

\hypertarget{naturally-maybe}{%
\subsection{Naturally, Maybe}\label{naturally-maybe}}

Now that we have our basic infrastructure, let's implement some other famous
semigroups:

First, the inductive nats, \texttt{data\ N\ =\ Z\ \textbar{}\ S\ N:}

\begin{Shaded}
\begin{Highlighting}[]
\FunctionTok{$}\NormalTok{(singletons [d|}
\NormalTok{  data N = Z | S N}
\NormalTok{    deriving (Show)}

\NormalTok{  plus :: N -> N -> N}
\NormalTok{  plus Z     y = y}
\NormalTok{  plus (S x) y = S (plus x y)}
\NormalTok{  |])}

\KeywordTok{instance} \DataTypeTok{Semigroup} \DataTypeTok{N} \KeywordTok{where}
    \KeywordTok{type}\NormalTok{ xs }\FunctionTok{<>}\NormalTok{ ys }\FunctionTok{=} \DataTypeTok{Plus}\NormalTok{ xs ys}
\NormalTok{    (}\FunctionTok{%<>}\NormalTok{) }\FunctionTok{=}\NormalTok{ sPlus}

\NormalTok{    appendAssoc }\FunctionTok{=}\NormalTok{ \textbackslash{}}\KeywordTok{case}
      \DataTypeTok{SZ} \OtherTok{->}\NormalTok{ \textbackslash{}_ _ }\OtherTok{->} \DataTypeTok{Refl}
      \DataTypeTok{SS}\NormalTok{ x }\OtherTok{->}\NormalTok{ \textbackslash{}y z }\OtherTok{->}
        \KeywordTok{case}\NormalTok{ appendAssoc x y z }\KeywordTok{of}
          \DataTypeTok{Refl} \OtherTok{->} \DataTypeTok{Refl}
\end{Highlighting}
\end{Shaded}

And the standard instance for \texttt{Maybe}, which lifts the underlying
semigroup:

\begin{Shaded}
\begin{Highlighting}[]
\FunctionTok{$}\NormalTok{(singletons [d|}
\NormalTok{  data Option a = None | Some a}
\NormalTok{      deriving (Show)}
\NormalTok{  |])}

\KeywordTok{instance} \DataTypeTok{Semigroup}\NormalTok{ a }\OtherTok{=>} \DataTypeTok{Semigroup}\NormalTok{ (}\DataTypeTok{Option}\NormalTok{ a) }\KeywordTok{where}
    \KeywordTok{type} \DataTypeTok{None}   \FunctionTok{<>}\NormalTok{ y      }\FunctionTok{=}\NormalTok{ y}
    \KeywordTok{type}\NormalTok{ x      }\FunctionTok{<>} \DataTypeTok{None}   \FunctionTok{=}\NormalTok{ x}
    \KeywordTok{type} \DataTypeTok{Some}\NormalTok{ x }\FunctionTok{<>} \DataTypeTok{Some}\NormalTok{ y }\FunctionTok{=} \DataTypeTok{Some}\NormalTok{ (x }\FunctionTok{<>}\NormalTok{ y)}

    \DataTypeTok{SNone}   \FunctionTok{%<>}\NormalTok{ y       }\FunctionTok{=}\NormalTok{ y}
\NormalTok{    x       }\FunctionTok{%<>} \DataTypeTok{SNone}   \FunctionTok{=}\NormalTok{ x}
    \DataTypeTok{SSome}\NormalTok{ x }\FunctionTok{%<>} \DataTypeTok{SSome}\NormalTok{ y }\FunctionTok{=} \DataTypeTok{SSome}\NormalTok{ (x }\FunctionTok{%<>}\NormalTok{ y)}

\NormalTok{    appendAssoc }\FunctionTok{=}\NormalTok{ \textbackslash{}}\KeywordTok{case}
        \DataTypeTok{SNone}   \OtherTok{->}\NormalTok{ \textbackslash{}_ _ }\OtherTok{->} \DataTypeTok{Refl}
        \DataTypeTok{SSome}\NormalTok{ x }\OtherTok{->}\NormalTok{ \textbackslash{}}\KeywordTok{case}
          \DataTypeTok{SNone} \OtherTok{->}\NormalTok{ \textbackslash{}_ }\OtherTok{->} \DataTypeTok{Refl}
          \DataTypeTok{SSome}\NormalTok{ y }\OtherTok{->}\NormalTok{ \textbackslash{}}\KeywordTok{case}
            \DataTypeTok{SNone} \OtherTok{->} \DataTypeTok{Refl}
            \DataTypeTok{SSome}\NormalTok{ z }\OtherTok{->}
              \KeywordTok{case}\NormalTok{ appendAssoc x y z }\KeywordTok{of}
                \DataTypeTok{Refl} \OtherTok{->} \DataTypeTok{Refl}
\end{Highlighting}
\end{Shaded}

\begin{Shaded}
\begin{Highlighting}[]
\NormalTok{ghci}\FunctionTok{>}\NormalTok{ print }\FunctionTok{$} \DataTypeTok{S}\NormalTok{ (}\DataTypeTok{S} \DataTypeTok{Z}\NormalTok{) }\FunctionTok{<>} \DataTypeTok{S} \DataTypeTok{Z}
\DataTypeTok{S}\NormalTok{ (}\DataTypeTok{S}\NormalTok{ (}\DataTypeTok{S} \DataTypeTok{Z}\NormalTok{))}
\NormalTok{ghci}\FunctionTok{>}\NormalTok{ print }\FunctionTok{$} \DataTypeTok{Some}\NormalTok{ (}\DataTypeTok{S} \DataTypeTok{Z}\NormalTok{) }\FunctionTok{<>} \DataTypeTok{Some}\NormalTok{ (}\DataTypeTok{S}\NormalTok{ (}\DataTypeTok{S}\NormalTok{ (}\DataTypeTok{S} \DataTypeTok{Z}\NormalTok{)))}
\DataTypeTok{Some}\NormalTok{ (}\DataTypeTok{S}\NormalTok{ (}\DataTypeTok{S}\NormalTok{ (}\DataTypeTok{S}\NormalTok{ (}\DataTypeTok{S} \DataTypeTok{Z}\NormalTok{))))}
\NormalTok{ghci}\FunctionTok{>}\NormalTok{ print }\FunctionTok{$} \DataTypeTok{None}       \FunctionTok{<>} \DataTypeTok{Some}\NormalTok{ (}\DataTypeTok{S}\NormalTok{ (}\DataTypeTok{S}\NormalTok{ (}\DataTypeTok{S} \DataTypeTok{Z}\NormalTok{)))}
\DataTypeTok{Some}\NormalTok{ (}\DataTypeTok{S}\NormalTok{ (}\DataTypeTok{S}\NormalTok{ (}\DataTypeTok{S} \DataTypeTok{Z}\NormalTok{)))}
\end{Highlighting}
\end{Shaded}

\hypertarget{going-monoidal}{%
\section{Going Monoidal}\label{going-monoidal}}

Of course, we can now introduce the \texttt{Monoid} typeclass, which introduces
a new element \texttt{empty}, along with the laws that appending with empty
leaves things unchanged:

\begin{Shaded}
\begin{Highlighting}[]
\KeywordTok{class} \DataTypeTok{Semigroup}\NormalTok{ a }\OtherTok{=>} \DataTypeTok{Monoid}\NormalTok{ a }\KeywordTok{where}
    \KeywordTok{type} \DataTypeTok{Empty}\OtherTok{ a ::}\NormalTok{ a}

\OtherTok{    sEmpty ::} \DataTypeTok{Sing}\NormalTok{ (}\DataTypeTok{Empty}\NormalTok{ a)}

\NormalTok{    emptyIdentLeft}
\OtherTok{        ::} \DataTypeTok{Sing}\NormalTok{ x}
        \OtherTok{->}\NormalTok{ (}\DataTypeTok{Empty}\NormalTok{ a }\FunctionTok{<>}\NormalTok{ x) }\FunctionTok{:~:}\NormalTok{ x}

\NormalTok{    emptyIdentRight}
\OtherTok{        ::} \DataTypeTok{Sing}\NormalTok{ x}
        \OtherTok{->}\NormalTok{ (x }\FunctionTok{<>} \DataTypeTok{Empty}\NormalTok{ a) }\FunctionTok{:~:}\NormalTok{ x}

\NormalTok{empty}
\OtherTok{    ::}\NormalTok{ (}\DataTypeTok{SingKind}\NormalTok{ m, }\DataTypeTok{Monoid}\NormalTok{ m)}
    \OtherTok{=>} \DataTypeTok{Demote}\NormalTok{ m}
\NormalTok{empty }\FunctionTok{=}\NormalTok{ fromSing sEmpty}
\end{Highlighting}
\end{Shaded}

Because working implicitly return-type polymorphism at the type level can be
annoying sometimes, we have \texttt{Empty} take the \emph{kind} \texttt{a} as a
parameter, instead of having it be inferred through kind inference like we did
for \texttt{\textless{}\textgreater{}}. That is, \texttt{Empty\ (List\ a)} is
\texttt{Empty} for the \emph{kind} \texttt{List\ a}.

As usual in Haskell, the instances write themselves!

\begin{Shaded}
\begin{Highlighting}[]
\KeywordTok{instance} \DataTypeTok{Monoid}\NormalTok{ (}\DataTypeTok{List}\NormalTok{ a) }\KeywordTok{where}
    \KeywordTok{type} \DataTypeTok{Empty}\NormalTok{ (}\DataTypeTok{List}\NormalTok{ a) }\FunctionTok{=} \DataTypeTok{Nil}

\NormalTok{    sEmpty }\FunctionTok{=} \DataTypeTok{SNil}
\NormalTok{    emptyIdentLeft _ }\FunctionTok{=} \DataTypeTok{Refl}
\NormalTok{    emptyIdentRight  }\FunctionTok{=}\NormalTok{ \textbackslash{}}\KeywordTok{case}
      \DataTypeTok{SNil} \OtherTok{->} \DataTypeTok{Refl}
      \DataTypeTok{SCons}\NormalTok{ _ xs }\OtherTok{->}
        \KeywordTok{case}\NormalTok{ emptyIdentRight xs }\KeywordTok{of}
          \DataTypeTok{Refl} \OtherTok{->} \DataTypeTok{Refl}

\KeywordTok{instance} \DataTypeTok{Monoid} \DataTypeTok{N} \KeywordTok{where}
    \KeywordTok{type} \DataTypeTok{Empty} \DataTypeTok{N} \FunctionTok{=} \DataTypeTok{Z}

\NormalTok{    sEmpty }\FunctionTok{=} \DataTypeTok{SZ}
\NormalTok{    emptyIdentLeft _ }\FunctionTok{=} \DataTypeTok{Refl}
\NormalTok{    emptyIdentRight  }\FunctionTok{=}\NormalTok{ \textbackslash{}}\KeywordTok{case}
      \DataTypeTok{SZ} \OtherTok{->} \DataTypeTok{Refl}
      \DataTypeTok{SS}\NormalTok{ x }\OtherTok{->} \KeywordTok{case}\NormalTok{ emptyIdentRight x }\KeywordTok{of}
        \DataTypeTok{Refl} \OtherTok{->} \DataTypeTok{Refl}

\KeywordTok{instance} \DataTypeTok{Semigroup}\NormalTok{ a }\OtherTok{=>} \DataTypeTok{Monoid}\NormalTok{ (}\DataTypeTok{Option}\NormalTok{ a) }\KeywordTok{where}
    \KeywordTok{type} \DataTypeTok{Empty}\NormalTok{ (}\DataTypeTok{Option}\NormalTok{ a) }\FunctionTok{=} \DataTypeTok{None}

\NormalTok{    sEmpty }\FunctionTok{=} \DataTypeTok{SNone}
\NormalTok{    emptyIdentLeft  _ }\FunctionTok{=} \DataTypeTok{Refl}
\NormalTok{    emptyIdentRight _ }\FunctionTok{=} \DataTypeTok{Refl}
\end{Highlighting}
\end{Shaded}

\hypertarget{play-that-funcy-music}{%
\section{Play that Funcy Music}\label{play-that-funcy-music}}

How about some higher-kinded typeclasses?

\begin{Shaded}
\begin{Highlighting}[]
\KeywordTok{class} \DataTypeTok{Functor}\NormalTok{ f }\KeywordTok{where}
    \KeywordTok{type} \DataTypeTok{Fmap}\NormalTok{ a b (}\OtherTok{g ::}\NormalTok{ a }\FunctionTok{~>}\NormalTok{ b) (}\OtherTok{x ::}\NormalTok{ f a)}\OtherTok{ ::}\NormalTok{ f b}

\NormalTok{    sFmap}
\OtherTok{        ::} \DataTypeTok{Sing}\NormalTok{ (}\OtherTok{g            ::}\NormalTok{ a }\FunctionTok{~>}\NormalTok{ b)}
        \OtherTok{->} \DataTypeTok{Sing}\NormalTok{ (}\OtherTok{x            ::}\NormalTok{ f a   )}
        \OtherTok{->} \DataTypeTok{Sing}\NormalTok{ (}\DataTypeTok{Fmap}\NormalTok{ a b g}\OtherTok{ x ::}\NormalTok{ f b   )}

    \CommentTok{-- | fmap id x == x}
\NormalTok{    fmapId}
\OtherTok{        ::} \DataTypeTok{Sing}\NormalTok{ (}\OtherTok{x ::}\NormalTok{ f a)}
        \OtherTok{->} \DataTypeTok{Fmap}\NormalTok{ a a }\DataTypeTok{IdSym0}\NormalTok{ x }\FunctionTok{:~:}\NormalTok{ x}

    \CommentTok{-- | fmap f (fmap g x) = fmap (f . g) x}
\NormalTok{    fmapCompose}
\OtherTok{        ::} \DataTypeTok{Sing}\NormalTok{ (}\OtherTok{g ::}\NormalTok{ b }\FunctionTok{~>}\NormalTok{ c)}
        \OtherTok{->} \DataTypeTok{Sing}\NormalTok{ (}\OtherTok{h ::}\NormalTok{ a }\FunctionTok{~>}\NormalTok{ b)}
        \OtherTok{->} \DataTypeTok{Sing}\NormalTok{ (}\OtherTok{x ::}\NormalTok{ f a   )}
        \OtherTok{->} \DataTypeTok{Fmap}\NormalTok{ b c g (}\DataTypeTok{Fmap}\NormalTok{ a b h x) }\FunctionTok{:~:} \DataTypeTok{Fmap}\NormalTok{ a c (((}\FunctionTok{:.$}\NormalTok{) }\FunctionTok{@@}\NormalTok{ g) }\FunctionTok{@@}\NormalTok{ h) x}
\end{Highlighting}
\end{Shaded}

\texttt{Fmap\ a\ b\ g\ x} maps the \emph{type-level function}
\texttt{g\ ::\ a\ \textasciitilde{}\textgreater{}\ b} over \texttt{x\ ::\ f\ a},
and returns a type of kind \texttt{f\ b}. Like with \texttt{Empty}, to help with
kind inference, we have \texttt{Fmap} explicitly requre the \emph{kinds} of the
input and results of \texttt{g} (\texttt{a} and \texttt{b}) so GHC doesn't have
to struggle to infer it implicitly.

And, of course, along with \texttt{sFmap} (the singleton mirror of
\texttt{Fmap}), we have our laws: \texttt{fmap\ id\ x\ =\ x}, and
\texttt{fmap\ g\ (fmap\ h)\ x\ =\ fmap\ (g\ .\ h)\ x}.

But, what are \texttt{a\ \textasciitilde{}\textgreater{}\ b}, \texttt{IdSym0},
\texttt{:.\$}, and \texttt{@@}? They're a part of the \emph{defunctionalization}
system that the singletons library uses. A
\texttt{g\ ::\ a\ \textasciitilde{}\textgreater{}\ b} means that \texttt{g}
represents a type-level function taking a type of kind \texttt{a} to a type of
kind \texttt{b}, but, importantly, encodes it in a way that makes Haskell happy.
This hack is required because you can't partially apply type families in
Haskell. If \texttt{g} was a regular old \texttt{a\ -\textgreater{}\ b} type
family, you wouldn't be able to pass just \texttt{g} into \texttt{Fmap\ a\ b\ g}
(because it'd be partially applied, and type families always have to appear
fully saturated).

You can convert a \texttt{g\ ::\ a\ \textasciitilde{}\textgreater{}\ b} back
into a regular old \texttt{g\ ::\ a\ -\textgreater{}\ b} using \texttt{Apply},
or its convenient infix synonym \texttt{@@}, like
\texttt{g\ @@\ (x\ ::\ a)\ ::\ b}

The singletons library provides
\texttt{type\ family\ Id\ a\ where\ Id\ a\ =\ a}, but we can't pass in
\texttt{Id} directly into \texttt{Fmap}. We have to pass in its
``defunctionalized'' encoding,
\texttt{IdSym0\ ::\ a\ \textasciitilde{}\textgreater{}\ a}.

For the composition law, we use \texttt{(:.\$)} (which is a defunctionalized
type-level \texttt{.}) and apply it to \texttt{g} and \texttt{h} to get,
essentially, \texttt{g\ :.\ h}, where \texttt{:.} is type-level function
composition.

Now we Haskell.

\begin{Shaded}
\begin{Highlighting}[]
\FunctionTok{$}\NormalTok{(singletons [d|}
\NormalTok{  mapOption :: (a -> b) -> Option a -> Option b}
\NormalTok{  mapOption _ None     = None}
\NormalTok{  mapOption f (Some x) = Some (f x)}

\NormalTok{  mapList :: (a -> b) -> List a -> List b}
\NormalTok{  mapList _ Nil         = Nil}
\NormalTok{  mapList f (Cons x xs) = Cons (f x) (mapList f xs)}
\NormalTok{  |])}

\KeywordTok{instance} \DataTypeTok{Functor} \DataTypeTok{Option} \KeywordTok{where}
    \KeywordTok{type} \DataTypeTok{Fmap}\NormalTok{ a b g x }\FunctionTok{=} \DataTypeTok{MapOption}\NormalTok{ g x}

\NormalTok{    sFmap }\FunctionTok{=}\NormalTok{ sMapOption}
\NormalTok{    fmapId }\FunctionTok{=}\NormalTok{ \textbackslash{}}\KeywordTok{case}
      \DataTypeTok{SNone}   \OtherTok{->} \DataTypeTok{Refl}
      \DataTypeTok{SSome}\NormalTok{ _ }\OtherTok{->} \DataTypeTok{Refl}

\NormalTok{    fmapCompose _ _ }\FunctionTok{=}\NormalTok{ \textbackslash{}}\KeywordTok{case}
      \DataTypeTok{SNone}   \OtherTok{->} \DataTypeTok{Refl}
      \DataTypeTok{SSome}\NormalTok{ _ }\OtherTok{->} \DataTypeTok{Refl}

\KeywordTok{instance} \DataTypeTok{Functor} \DataTypeTok{List} \KeywordTok{where}
    \KeywordTok{type} \DataTypeTok{Fmap}\NormalTok{ a b g x }\FunctionTok{=} \DataTypeTok{MapList}\NormalTok{ g x}

\NormalTok{    sFmap }\FunctionTok{=}\NormalTok{ sMapList}
\NormalTok{    fmapId }\FunctionTok{=}\NormalTok{ \textbackslash{}}\KeywordTok{case}
      \DataTypeTok{SNil}       \OtherTok{->} \DataTypeTok{Refl}
      \DataTypeTok{SCons}\NormalTok{ _ xs }\OtherTok{->}
        \KeywordTok{case}\NormalTok{ fmapId xs }\KeywordTok{of}
          \DataTypeTok{Refl} \OtherTok{->} \DataTypeTok{Refl}

\NormalTok{    fmapCompose g h }\FunctionTok{=}\NormalTok{ \textbackslash{}}\KeywordTok{case}
      \DataTypeTok{SNil} \OtherTok{->} \DataTypeTok{Refl}
      \DataTypeTok{SCons}\NormalTok{ _ xs }\OtherTok{->}
        \KeywordTok{case}\NormalTok{ fmapCompose g h xs }\KeywordTok{of}
          \DataTypeTok{Refl} \OtherTok{->} \DataTypeTok{Refl}
\end{Highlighting}
\end{Shaded}

And there you have it. A verified \texttt{Functor} typeclass, ensuring that all
instances are lawful. Never tell me that Haskell's type system can't do anything
ever again!

Note that any mistakes in implementation (like, for example, having
\texttt{mapOption\ \_\ \_\ =\ None}) will cause a compile-time error now,
because the proofs are impossible to provide.

As a side note, I'm not quite sure how to implement the value-level
\texttt{fmap} from this, since I can't figure out how to promote functions
nicely. Using \texttt{sFmap} is the only way to work with this at the value
level that I can see, but it's probably because of my own lack of understanding.
If anyone knows how to do this, please let me know!

Anyway, what an exciting journey and a wonderful conclusion. I hope you enjoyed
this and will begin using this in your normal day-to-day Haskell. Goodbye, until
next time!

\hypertarget{just-one-more}{%
\section{Just one more}\label{just-one-more}}

Hah! Of course we aren't done. I wouldn't let you down like that. I know that
you probably saw that the entire last section's only purpose was to build up to
the pièce de résistance: the crown jewel of every Haskell article, the Monad.

\begin{Shaded}
\begin{Highlighting}[]
\KeywordTok{class} \DataTypeTok{Functor}\NormalTok{ f }\OtherTok{=>} \DataTypeTok{Monad}\NormalTok{ f }\KeywordTok{where}
    \KeywordTok{type} \DataTypeTok{Return}\NormalTok{ a   (}\OtherTok{x ::}\NormalTok{ a)}\OtherTok{                   ::}\NormalTok{ f a}
    \KeywordTok{type} \DataTypeTok{Bind}\NormalTok{   a b (}\OtherTok{m ::}\NormalTok{ f a) (}\OtherTok{g ::}\NormalTok{ a }\FunctionTok{~>}\NormalTok{ f b)}\OtherTok{ ::}\NormalTok{ f b}

\NormalTok{    sReturn}
\OtherTok{        ::} \DataTypeTok{Sing}\NormalTok{ (}\OtherTok{x ::}\NormalTok{ a)}
        \OtherTok{->} \DataTypeTok{Sing}\NormalTok{ (}\DataTypeTok{Return}\NormalTok{ a}\OtherTok{ x ::}\NormalTok{ f a)}

\NormalTok{    sBind}
\OtherTok{        ::} \DataTypeTok{Sing}\NormalTok{ (}\OtherTok{m ::}\NormalTok{ f a)}
        \OtherTok{->} \DataTypeTok{Sing}\NormalTok{ (}\OtherTok{g ::}\NormalTok{ a }\FunctionTok{~>}\NormalTok{ f b)}
        \OtherTok{->} \DataTypeTok{Sing}\NormalTok{ (}\DataTypeTok{Bind}\NormalTok{ a b m g)}

    \CommentTok{-- | (return x >>= f) == f x}
\NormalTok{    returnIdentLeft}
\OtherTok{        ::} \DataTypeTok{Sing}\NormalTok{ (}\OtherTok{x ::}\NormalTok{ a)}
        \OtherTok{->} \DataTypeTok{Sing}\NormalTok{ (}\OtherTok{g ::}\NormalTok{ a }\FunctionTok{~>}\NormalTok{ f b)}
        \OtherTok{->} \DataTypeTok{Bind}\NormalTok{ a b (}\DataTypeTok{Return}\NormalTok{ a x) g }\FunctionTok{:~:}\NormalTok{ (g }\FunctionTok{@@}\NormalTok{ x)}

    \CommentTok{-- | (m >>= return) == m}
\NormalTok{    returnIdentRight}
\OtherTok{        ::} \DataTypeTok{Sing}\NormalTok{ (}\OtherTok{m ::}\NormalTok{ f a)}
        \OtherTok{->} \DataTypeTok{Bind}\NormalTok{ a a m }\DataTypeTok{ReturnSym0} \FunctionTok{:~:}\NormalTok{ m}

    \CommentTok{-- | m >>= (\textbackslash{}x -> f x >>= h) == (m >>= f) >>= h}
\NormalTok{    bindCompose}
\OtherTok{        ::} \DataTypeTok{Sing}\NormalTok{ (}\OtherTok{m ::}\NormalTok{ f a)}
        \OtherTok{->} \DataTypeTok{Sing}\NormalTok{ (}\OtherTok{g ::}\NormalTok{ a }\FunctionTok{~>}\NormalTok{ f b)}
        \OtherTok{->} \DataTypeTok{Sing}\NormalTok{ (}\OtherTok{h ::}\NormalTok{ b }\FunctionTok{~>}\NormalTok{ f c)}
        \OtherTok{->} \DataTypeTok{Bind}\NormalTok{ a c m (}\DataTypeTok{KCompSym2}\NormalTok{ a b c g h) }\FunctionTok{:~:} \DataTypeTok{Bind}\NormalTok{ b c (}\DataTypeTok{Bind}\NormalTok{ a b m g) h}

\KeywordTok{data} \DataTypeTok{ReturnSym0}\OtherTok{ ::}\NormalTok{ a }\FunctionTok{~>}\NormalTok{ f a}
\KeywordTok{type} \KeywordTok{instance} \DataTypeTok{Apply}\NormalTok{ (}\DataTypeTok{ReturnSym0}\OtherTok{ ::}\NormalTok{ a }\FunctionTok{~>}\NormalTok{ f a) (}\OtherTok{x ::}\NormalTok{ a) }\FunctionTok{=} \DataTypeTok{Return}\NormalTok{ a x}

\KeywordTok{type} \DataTypeTok{KComp}\NormalTok{ a b c (}\OtherTok{g ::}\NormalTok{ a }\FunctionTok{~>}\NormalTok{ f b) (}\OtherTok{h ::}\NormalTok{ b }\FunctionTok{~>}\NormalTok{ f c) (}\OtherTok{x ::}\NormalTok{ a) }\FunctionTok{=} \DataTypeTok{Bind}\NormalTok{ b c (g }\FunctionTok{@@}\NormalTok{ x) h}
\KeywordTok{data} \DataTypeTok{KCompSym2}\NormalTok{ a b c g}\OtherTok{ h ::}\NormalTok{ (a }\FunctionTok{~>}\NormalTok{ f c)}
\KeywordTok{type} \KeywordTok{instance} \DataTypeTok{Apply}\NormalTok{ (}\DataTypeTok{KCompSym2}\NormalTok{ a b c g}\OtherTok{ h ::}\NormalTok{ a }\FunctionTok{~>}\NormalTok{ f c) (}\OtherTok{x ::}\NormalTok{ a) }\FunctionTok{=} \DataTypeTok{KComp}\NormalTok{ a b c g h x}

\NormalTok{return}
\OtherTok{    ::}\NormalTok{ (}\DataTypeTok{SingKind}\NormalTok{ a, }\DataTypeTok{SingKind}\NormalTok{ (f a), }\DataTypeTok{Monad}\NormalTok{ f)}
    \OtherTok{=>} \DataTypeTok{Demote}\NormalTok{ a}
    \OtherTok{->} \DataTypeTok{Demote}\NormalTok{ (f a)}
\NormalTok{return x }\FunctionTok{=}\NormalTok{ withSomeSing x }\FunctionTok{$}\NormalTok{ \textbackslash{}sX }\OtherTok{->}
\NormalTok{             fromSing (sReturn sX)}
\end{Highlighting}
\end{Shaded}

To help with kind inference, again, we provide explicit kind arguments for
\texttt{Return} (the kind of the thing that is being lifted) and \texttt{Bind}
(the original \texttt{a} and the resulting \texttt{b}).

Some boilerplate exists there at the bottom --- it's the plumbing for the
defunctionalization system. \texttt{returnIdentRight} requires a
defunctionalized version of \texttt{Return}, so we can provide that by defining
\texttt{ReturnSym0}, and writing an \texttt{Apply} instance for it (which
``applies'' it the parameter \texttt{x}).

We introduce \texttt{KComp} (kleisli composition) and its defunctionalized
version in order to express the third law, because we don't yet have type-level
lambdas in Haskell. The actual function it is expressing is
\texttt{\textbackslash{}x\ -\textgreater{}\ f\ x\ \textgreater{}\textgreater{}=\ g},
and that definition is given on the
\texttt{type\ KComp\ a\ b\ c\ ...\ =\ Bind\ ...} line. \texttt{KCompSym2} is the
defunctioanlized version, which is not a \texttt{a\ -\textgreater{}\ f\ c} but
rather an \texttt{a\ \textasciitilde{}\textgreater{}\ f\ c}, which allows it to
be partially applied (like we do for \texttt{composeBind}). And, finally, to
hook all of this up into the defunctionalization system, we write an
\texttt{Apply} instance yet again.

And, again, if anyone knows how I can write a value-level \texttt{Bind}, I'd
definitely appreciate hearing!

Let's see some sample implementations.

\begin{Shaded}
\begin{Highlighting}[]
\FunctionTok{$}\NormalTok{(singletons [d|}
\NormalTok{  bindOption :: Option a -> (a -> Option b) -> Option b}
\NormalTok{  bindOption None     _ = None}
\NormalTok{  bindOption (Some x) f = f x}

\NormalTok{  concatMapList :: (a -> List b) -> List a -> List b}
\NormalTok{  concatMapList _ Nil         = Nil}
\NormalTok{  concatMapList f (Cons x xs) = f x `appendList` concatMapList f xs}
\NormalTok{  |])}

\KeywordTok{instance} \DataTypeTok{Monad} \DataTypeTok{Option} \KeywordTok{where}
    \KeywordTok{type} \DataTypeTok{Return}\NormalTok{ a   x   }\FunctionTok{=} \DataTypeTok{Some}\NormalTok{ x}
    \KeywordTok{type} \DataTypeTok{Bind}\NormalTok{   a b m g }\FunctionTok{=} \DataTypeTok{BindOption}\NormalTok{ m g}

\NormalTok{    sReturn }\FunctionTok{=} \DataTypeTok{SSome}
\NormalTok{    sBind   }\FunctionTok{=}\NormalTok{ sBindOption}

\NormalTok{    returnIdentLeft _ _ }\FunctionTok{=} \DataTypeTok{Refl}
\NormalTok{    returnIdentRight }\FunctionTok{=}\NormalTok{ \textbackslash{}}\KeywordTok{case}
      \DataTypeTok{SNone}   \OtherTok{->} \DataTypeTok{Refl}
      \DataTypeTok{SSome}\NormalTok{ x }\OtherTok{->} \KeywordTok{case}\NormalTok{ sReturn x }\KeywordTok{of}
        \DataTypeTok{SSome}\NormalTok{ _ }\OtherTok{->} \DataTypeTok{Refl}
\NormalTok{    bindCompose }\FunctionTok{=}\NormalTok{ \textbackslash{}}\KeywordTok{case}
      \DataTypeTok{SNone}   \OtherTok{->}\NormalTok{ \textbackslash{}_ _ }\OtherTok{->} \DataTypeTok{Refl}
      \DataTypeTok{SSome}\NormalTok{ _ }\OtherTok{->}\NormalTok{ \textbackslash{}_ _ }\OtherTok{->} \DataTypeTok{Refl}

\KeywordTok{instance} \DataTypeTok{Monad} \DataTypeTok{List} \KeywordTok{where}
    \KeywordTok{type} \DataTypeTok{Return}\NormalTok{ a   x   }\FunctionTok{=} \DataTypeTok{PureList}\NormalTok{ x}
    \KeywordTok{type} \DataTypeTok{Bind}\NormalTok{   a b m g }\FunctionTok{=} \DataTypeTok{ConcatMapList}\NormalTok{ g m}

\NormalTok{    sReturn   }\FunctionTok{=}\NormalTok{ sPureList}
\NormalTok{    sBind x f }\FunctionTok{=}\NormalTok{ sConcatMapList f x}

\NormalTok{    returnIdentLeft x g }\FunctionTok{=} \KeywordTok{case}\NormalTok{ sReturn x }\KeywordTok{of}
      \DataTypeTok{SCons}\NormalTok{ y }\DataTypeTok{SNil} \OtherTok{->} \KeywordTok{case}\NormalTok{ emptyIdentRight (unSingFun1 g y) }\KeywordTok{of}
        \DataTypeTok{Refl} \OtherTok{->} \DataTypeTok{Refl}

\NormalTok{    returnIdentRight }\FunctionTok{=}\NormalTok{ \textbackslash{}}\KeywordTok{case}
      \DataTypeTok{SNil}       \OtherTok{->} \DataTypeTok{Refl}
      \DataTypeTok{SCons}\NormalTok{ _ xs }\OtherTok{->} \KeywordTok{case}\NormalTok{ returnIdentRight xs }\KeywordTok{of}
        \DataTypeTok{Refl} \OtherTok{->} \DataTypeTok{Refl}

\NormalTok{    bindCompose }\FunctionTok{=}\NormalTok{ \textbackslash{}}\KeywordTok{case}
      \DataTypeTok{SNil}       \OtherTok{->}\NormalTok{ \textbackslash{}_ _ }\OtherTok{->} \DataTypeTok{Refl}
      \DataTypeTok{SCons}\NormalTok{ x xs }\OtherTok{->}\NormalTok{ \textbackslash{}g h }\OtherTok{->} \KeywordTok{case}\NormalTok{ bindCompose xs g h }\KeywordTok{of}
        \DataTypeTok{Refl} \OtherTok{->} \KeywordTok{case}\NormalTok{ unSingFun1 g x }\KeywordTok{of}
          \DataTypeTok{SNil}       \OtherTok{->} \DataTypeTok{Refl}
          \DataTypeTok{SCons}\NormalTok{ y ys }\OtherTok{->}
            \KeywordTok{let}\NormalTok{ gxs  }\FunctionTok{=}\NormalTok{ sConcatMapList g xs}
\NormalTok{                hgxs }\FunctionTok{=}\NormalTok{ sConcatMapList h gxs}
\NormalTok{                hy   }\FunctionTok{=}\NormalTok{ unSingFun1 h y}
\NormalTok{                hys  }\FunctionTok{=}\NormalTok{ sConcatMapList h ys}
            \KeywordTok{in}  \KeywordTok{case}\NormalTok{ distribConcatMap h ys gxs }\KeywordTok{of}
                  \DataTypeTok{Refl} \OtherTok{->} \KeywordTok{case}\NormalTok{ appendAssoc hy hys hgxs }\KeywordTok{of}
                    \DataTypeTok{Refl} \OtherTok{->} \DataTypeTok{Refl}

\CommentTok{-- | Proving that concatMap distributes over <>}
\NormalTok{distribConcatMap}
\OtherTok{    ::} \DataTypeTok{Sing}\NormalTok{ (}\OtherTok{g ::}\NormalTok{ a }\FunctionTok{~>} \DataTypeTok{List}\NormalTok{ b)}
    \OtherTok{->} \DataTypeTok{Sing}\NormalTok{ (}\OtherTok{xs ::} \DataTypeTok{List}\NormalTok{ a)}
    \OtherTok{->} \DataTypeTok{Sing}\NormalTok{ (}\OtherTok{ys ::} \DataTypeTok{List}\NormalTok{ a)}
    \OtherTok{->} \DataTypeTok{ConcatMapList}\NormalTok{ g (xs }\FunctionTok{<>}\NormalTok{ ys) }\FunctionTok{:~:}\NormalTok{ (}\DataTypeTok{ConcatMapList}\NormalTok{ g xs }\FunctionTok{<>} \DataTypeTok{ConcatMapList}\NormalTok{ g ys)}
\NormalTok{distribConcatMap g }\FunctionTok{=}\NormalTok{ \textbackslash{}}\KeywordTok{case}
    \DataTypeTok{SNil} \OtherTok{->}\NormalTok{ \textbackslash{}_ }\OtherTok{->} \DataTypeTok{Refl}
    \DataTypeTok{SCons}\NormalTok{ x xs }\OtherTok{->}\NormalTok{ \textbackslash{}ys }\OtherTok{->}
      \KeywordTok{case}\NormalTok{ distribConcatMap g xs ys }\KeywordTok{of}
        \DataTypeTok{Refl} \OtherTok{->}
          \KeywordTok{let}\NormalTok{ gx    }\FunctionTok{=}\NormalTok{ unSingFun1 g x}
\NormalTok{              cmgxs }\FunctionTok{=}\NormalTok{ sConcatMapList g xs}
\NormalTok{              cmgys }\FunctionTok{=}\NormalTok{ sConcatMapList g ys}
          \KeywordTok{in}  \KeywordTok{case}\NormalTok{ appendAssoc gx cmgxs cmgys }\KeywordTok{of}
                \DataTypeTok{Refl} \OtherTok{->} \DataTypeTok{Refl}
\end{Highlighting}
\end{Shaded}

Here we use \texttt{unSingFun1}, which converts a singleton of a type-level
function into a value-level function on singletons:

\begin{Shaded}
\begin{Highlighting}[]
\NormalTok{unSingFun1}
\OtherTok{    ::} \DataTypeTok{Sing}\NormalTok{  (}\OtherTok{f      ::}\NormalTok{ a }\FunctionTok{~>}\NormalTok{ b)}
    \OtherTok{->} \DataTypeTok{Sing}\NormalTok{  (}\OtherTok{x      ::}\NormalTok{ a)}
    \OtherTok{->} \DataTypeTok{Sing}\NormalTok{  (f }\FunctionTok{@@}\OtherTok{ x ::}\NormalTok{ b)}
\end{Highlighting}
\end{Shaded}

The crux is that, given a
\texttt{Sing\ (f\ ::\ a\ \textasciitilde{}\textgreater{}\ b)} and a
\texttt{Sing\ (x\ ::\ a)}, we can ``apply'' them to get
\texttt{Sing\ (f\ @@\ x\ ::\ b)}

The proofs for the list instance is admittedly ugly to write, due to the fact
that \texttt{List} is a recursive type. It's also tricky because Haskell has
poor to little support for theorem proving and no real tools to help you write
them efficiently. But, the proofs for \texttt{Option} are really something,
aren't they? It's kind of amazing how much GHC can do on its own without
requiring any manual proving on the part of the user.

\hypertarget{disclaimer}{%
\section{Disclaimer}\label{disclaimer}}

Don't do this in actual code, please
(\href{https://twitter.com/mstk/status/848677244478279680}{why?}). This post
started off as an April Fools joke that accidentally compiled correctly for
reasons which I cannot explain.

While I don't recommend that you do this in actual code, but definitely do
recommend that you do it for fun! The code in this post is available
\href{https://github.com/mstksg/inCode/tree/master/code-samples/verified-instances/VerifiedInstances.hs}{here}
if you want to play around!

\hypertarget{signoff}{%
\section{Signoff}\label{signoff}}

Hi, thanks for reading! You can reach me via email at
\href{mailto:justin@jle.im}{\nolinkurl{justin@jle.im}}, or at twitter at
\href{https://twitter.com/mstk}{@mstk}! This post and all others are published
under the \href{https://creativecommons.org/licenses/by-nc-nd/3.0/}{CC-BY-NC-ND
3.0} license. Corrections and edits via pull request are welcome and encouraged
at \href{https://github.com/mstksg/inCode}{the source repository}.

If you feel inclined, or this post was particularly helpful for you, why not
consider \href{https://www.patreon.com/justinle/overview}{supporting me on
Patreon}, or a \href{bitcoin:3D7rmAYgbDnp4gp4rf22THsGt74fNucPDU}{BTC donation}?
:)

\end{document}
