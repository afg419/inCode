\documentclass[]{article}
\usepackage{lmodern}
\usepackage{amssymb,amsmath}
\usepackage{ifxetex,ifluatex}
\usepackage{fixltx2e} % provides \textsubscript
\ifnum 0\ifxetex 1\fi\ifluatex 1\fi=0 % if pdftex
  \usepackage[T1]{fontenc}
  \usepackage[utf8]{inputenc}
\else % if luatex or xelatex
  \ifxetex
    \usepackage{mathspec}
    \usepackage{xltxtra,xunicode}
  \else
    \usepackage{fontspec}
  \fi
  \defaultfontfeatures{Mapping=tex-text,Scale=MatchLowercase}
  \newcommand{\euro}{€}
\fi
% use upquote if available, for straight quotes in verbatim environments
\IfFileExists{upquote.sty}{\usepackage{upquote}}{}
% use microtype if available
\IfFileExists{microtype.sty}{\usepackage{microtype}}{}
\usepackage[margin=1in]{geometry}
\usepackage{color}
\usepackage{fancyvrb}
\newcommand{\VerbBar}{|}
\newcommand{\VERB}{\Verb[commandchars=\\\{\}]}
\DefineVerbatimEnvironment{Highlighting}{Verbatim}{commandchars=\\\{\}}
% Add ',fontsize=\small' for more characters per line
\usepackage{framed}
\definecolor{shadecolor}{RGB}{248,248,248}
\newenvironment{Shaded}{\begin{snugshade}}{\end{snugshade}}
\newcommand{\AlertTok}[1]{\textcolor[rgb]{0.94,0.16,0.16}{#1}}
\newcommand{\AnnotationTok}[1]{\textcolor[rgb]{0.56,0.35,0.01}{\textbf{\textit{#1}}}}
\newcommand{\AttributeTok}[1]{\textcolor[rgb]{0.77,0.63,0.00}{#1}}
\newcommand{\BaseNTok}[1]{\textcolor[rgb]{0.00,0.00,0.81}{#1}}
\newcommand{\BuiltInTok}[1]{#1}
\newcommand{\CharTok}[1]{\textcolor[rgb]{0.31,0.60,0.02}{#1}}
\newcommand{\CommentTok}[1]{\textcolor[rgb]{0.56,0.35,0.01}{\textit{#1}}}
\newcommand{\CommentVarTok}[1]{\textcolor[rgb]{0.56,0.35,0.01}{\textbf{\textit{#1}}}}
\newcommand{\ConstantTok}[1]{\textcolor[rgb]{0.00,0.00,0.00}{#1}}
\newcommand{\ControlFlowTok}[1]{\textcolor[rgb]{0.13,0.29,0.53}{\textbf{#1}}}
\newcommand{\DataTypeTok}[1]{\textcolor[rgb]{0.13,0.29,0.53}{#1}}
\newcommand{\DecValTok}[1]{\textcolor[rgb]{0.00,0.00,0.81}{#1}}
\newcommand{\DocumentationTok}[1]{\textcolor[rgb]{0.56,0.35,0.01}{\textbf{\textit{#1}}}}
\newcommand{\ErrorTok}[1]{\textcolor[rgb]{0.64,0.00,0.00}{\textbf{#1}}}
\newcommand{\ExtensionTok}[1]{#1}
\newcommand{\FloatTok}[1]{\textcolor[rgb]{0.00,0.00,0.81}{#1}}
\newcommand{\FunctionTok}[1]{\textcolor[rgb]{0.00,0.00,0.00}{#1}}
\newcommand{\ImportTok}[1]{#1}
\newcommand{\InformationTok}[1]{\textcolor[rgb]{0.56,0.35,0.01}{\textbf{\textit{#1}}}}
\newcommand{\KeywordTok}[1]{\textcolor[rgb]{0.13,0.29,0.53}{\textbf{#1}}}
\newcommand{\NormalTok}[1]{#1}
\newcommand{\OperatorTok}[1]{\textcolor[rgb]{0.81,0.36,0.00}{\textbf{#1}}}
\newcommand{\OtherTok}[1]{\textcolor[rgb]{0.56,0.35,0.01}{#1}}
\newcommand{\PreprocessorTok}[1]{\textcolor[rgb]{0.56,0.35,0.01}{\textit{#1}}}
\newcommand{\RegionMarkerTok}[1]{#1}
\newcommand{\SpecialCharTok}[1]{\textcolor[rgb]{0.00,0.00,0.00}{#1}}
\newcommand{\SpecialStringTok}[1]{\textcolor[rgb]{0.31,0.60,0.02}{#1}}
\newcommand{\StringTok}[1]{\textcolor[rgb]{0.31,0.60,0.02}{#1}}
\newcommand{\VariableTok}[1]{\textcolor[rgb]{0.00,0.00,0.00}{#1}}
\newcommand{\VerbatimStringTok}[1]{\textcolor[rgb]{0.31,0.60,0.02}{#1}}
\newcommand{\WarningTok}[1]{\textcolor[rgb]{0.56,0.35,0.01}{\textbf{\textit{#1}}}}
\ifxetex
  \usepackage[setpagesize=false, % page size defined by xetex
              unicode=false, % unicode breaks when used with xetex
              xetex]{hyperref}
\else
  \usepackage[unicode=true]{hyperref}
\fi
\hypersetup{breaklinks=true,
            bookmarks=true,
            pdfauthor={Justin Le},
            pdftitle={Fixed-Length Vector Types in Haskell (an Update for 2017)},
            colorlinks=true,
            citecolor=blue,
            urlcolor=blue,
            linkcolor=magenta,
            pdfborder={0 0 0}}
\urlstyle{same}  % don't use monospace font for urls
% Make links footnotes instead of hotlinks:
\renewcommand{\href}[2]{#2\footnote{\url{#1}}}
\setlength{\parindent}{0pt}
\setlength{\parskip}{6pt plus 2pt minus 1pt}
\setlength{\emergencystretch}{3em}  % prevent overfull lines
\setcounter{secnumdepth}{0}

\title{Fixed-Length Vector Types in Haskell (an Update for 2017)}
\author{Justin Le}
\date{August 25, 2017}

\begin{document}
\maketitle

\emph{Originally posted on
\textbf{\href{https://blog.jle.im/entry/fixed-length-vector-types-in-haskell.html}{in
Code}}.}

This post is a follow-up to my
\href{https://blog.jle.im/entry/fixed-length-vector-types-in-haskell-2015.html}{fixed-length
vectors in haskell in 2015} post! When I was writing the post originally, I was
new to the whole type-level game in Haskell; I didn't know what I was talking
about, and that post was a way for me to push myself to learn more. Immediately
after it was posted, of course, people taught me where I went wrong in the
idioms I explained, and better and more idiomatic ways to do things. And that's
great! Learning is awesome!

Unfortunately, however, to my horror, I began noticing people referring to the
post in a canonical/authoritative way\ldots{}so the post became an immediate
source of guilt to me. I tried correcting things with my
\href{https://blog.jle.im/entries/series/+practical-dependent-types-in-haskell.html}{practical
dependent types in haskell} series the next year. But I still saw my 2015 post
being used as a reference even after that post, so I figured that writing a
direct replacement/follow-up as the only way I would ever fix this!

So here we are in 2017. GHC 8.2 is here, and \emph{base} is in version
\emph{4.10}. What's the ``right'' way to do fixed-length vectors in Haskell?

This post doesn't attempt to present anything groundbreaking or new, but is
meant to be a sort of \emph{reference/introduction} to fixed-length vectors in
Haskell, as of GHC 8.2 and the 2017 Haskell ecosystem.

We'll be looking at two methods here: The first one we will be looking at is a
\emph{performant} fixed-length vector that you will probably be using for any
code that requires a fixed-length container --- especially for tight numeric
code and situations where performance matters. We'll see how to implement them
using the universal native \texttt{KnownNat} mechanisms, and also how we can
implement them using
\emph{\href{http://hackage.haskell.org/package/singletons}{singletons}} to help
us make things a bit smoother and more well-integrated. For most people, this is
all they actually need. (I claim the canonical haskell ecosystem source to be
the \emph{\href{http://hackage.haskell.org/package/vector-sized}{vector-sized}}
library)

The second method is a \emph{structural} fixed-length inductive vector.
It's\ldots{}actually more like a fixed-length (lazily linked) \emph{list} than a
vector. The length of the list is enforced by the very structure of the data
type. This type is more useful as a streaming data type, and also in situations
where you want take advantage of the structural characteristics of lengths in
the context of a dependently typed program. (I claim the canonical haskell
ecosystem source to be the
\emph{\href{http://hackage.haskell.org/package/type-combinators}{type-combinators}}
library)

\hypertarget{the-non-structural-way}{%
\section{The Non-Structural Way}\label{the-non-structural-way}}

(Code for this section (as well as exercise solutions) are
\href{https://github.com/mstksg/inCode/tree/master/code-samples/fixvec-2/VecWrapped.hs}{available
here})

In most situations, if you use vectors, you want some sort of constant-time
indexed data structure. The best way to do this in Haskell is to wrap the
heavily optimized
\emph{\href{http://hackage.haskell.org/package/vector}{vector}} library with a
newtype wrapper that contains its length as a phantom type parameter.

\begin{Shaded}
\begin{Highlighting}[]
\KeywordTok{import} \KeywordTok{qualified} \DataTypeTok{Data.Vector} \KeywordTok{as} \DataTypeTok{V}
\KeywordTok{import}           \DataTypeTok{GHC.TypeNats}

\CommentTok{-- source: https://github.com/mstksg/inCode/tree/master/code-samples/fixvec-2/VecWrapped.hs#L19-L20}
\KeywordTok{data} \DataTypeTok{Vec}\NormalTok{ (}\OtherTok{n ::} \DataTypeTok{Nat}\NormalTok{) a }\FunctionTok{=} \DataTypeTok{UnsafeMkVec}\NormalTok{ \{}\OtherTok{ getVector ::} \DataTypeTok{V.Vector}\NormalTok{ a \}}
    \KeywordTok{deriving} \DataTypeTok{Show}
\end{Highlighting}
\end{Shaded}

A \texttt{Vec\ n\ a} will represent an \texttt{n}-element vector of \texttt{a}s.
So, a \texttt{Vec\ 5\ Int} will be a vector of five \texttt{Int}s, a
\texttt{Vec\ 10\ String} is a vector of 10 \texttt{String}s, etc.

For our numeric types, we're using the fancy ``type literals'' that GHC offers
us with the \texttt{DataKinds} extension. Basically, alongside the normal kinds
\texttt{*}, \texttt{*\ -\textgreater{}\ *}, etc., we also have the \texttt{Nat}
kind; type literals \texttt{1}, \texttt{5}, \texttt{100}, etc. are all
\emph{types} with the \emph{kind} \texttt{Nat}, from the
\emph{\href{http://hackage.haskell.org/package/base/docs/GHC-TypeNats.html}{GHC.TypeNats}}
module\footnote{Users who are used to GHC 8.0 and below might remember
  \texttt{Nat} coming from
  \emph{\href{http://hackage.haskell.org/package/base/docs/GHC-TypeLits.html}{GHC.TypeLits}}.
  Well, GHC 8.2 is here, \texttt{TypeLits} is out, \texttt{TypeNats} is in. The
  difference is that, in \texttt{TypeLits}, the \texttt{Nat} kind
  reifies/reflects with \texttt{Integer}. In \texttt{TypeNats}, the \texttt{Nat}
  kind reifies/reflects with \texttt{Natural} from
  \emph{\href{http://hackage.haskell.org/package/base/docs/Numeric-Natural.html}{Numeric.Natural}}.}.

\begin{Shaded}
\begin{Highlighting}[]
\NormalTok{ghci}\FunctionTok{>} \FunctionTok{:}\NormalTok{k }\DecValTok{5}
\DataTypeTok{Nat}
\NormalTok{ghci}\FunctionTok{>} \FunctionTok{:}\NormalTok{k }\DataTypeTok{Vec}
\DataTypeTok{Vec}\OtherTok{ ::} \DataTypeTok{Nat} \OtherTok{->} \FunctionTok{*} \OtherTok{->} \FunctionTok{*}
\end{Highlighting}
\end{Shaded}

You can ``reflect'' the type-level numeral as a value using the
\texttt{KnownNat} typeclass, provided by GHC, which lets you gain back the
number as a run-time value using \texttt{natVal}: (This process is called
``reflection'')

\begin{Shaded}
\begin{Highlighting}[]
\OtherTok{natVal ::} \DataTypeTok{KnownNat}\NormalTok{ n }\OtherTok{=>}\NormalTok{ p n }\OtherTok{->} \DataTypeTok{Natural}
\end{Highlighting}
\end{Shaded}

(Where \texttt{Natural}, from
\emph{\href{http://hackage.haskell.org/package/base/docs/Numeric-Natural.html}{Numeric.Natural}},
is a non-negative \texttt{Integer} type.)

\begin{Shaded}
\begin{Highlighting}[]
\NormalTok{ghci}\FunctionTok{>}\NormalTok{ natVal (}\DataTypeTok{Proxy} \FunctionTok{@}\DecValTok{10}\NormalTok{)   }\CommentTok{-- or, natVal (Proxy :: Proxy 10)}
\DecValTok{10}
\NormalTok{ghci}\FunctionTok{>}\NormalTok{ natVal (}\DataTypeTok{Proxy} \FunctionTok{@}\DecValTok{7}\NormalTok{)}
\DecValTok{7}
\end{Highlighting}
\end{Shaded}

Super low-level utility functions for the \texttt{Nat} kind (like
\texttt{natVal}) are found in the
\emph{\href{http://hackage.haskell.org/package/base/docs/GHC-TypeNats.html}{GHC.TypeNats}}
module (and also in
\emph{\href{http://hackage.haskell.org/package/base/docs/GHC-TypeLits.html}{GHC.TypeLits}}
for a slightly different API)

\hypertarget{the-smart-constructor}{%
\subsection{The Smart Constructor}\label{the-smart-constructor}}

We can use \texttt{natVal} with the \texttt{KnownNat} typeclass to write a
``smart constructor'' for our type -- make a \texttt{Vec} from a
\texttt{Vector}, but only if the length is the correct type:

\begin{Shaded}
\begin{Highlighting}[]
\CommentTok{-- source: https://github.com/mstksg/inCode/tree/master/code-samples/fixvec-2/VecWrapped.hs#L22-L26}
\OtherTok{mkVec ::}\NormalTok{ forall n a}\FunctionTok{.} \DataTypeTok{KnownNat}\NormalTok{ n }\OtherTok{=>} \DataTypeTok{V.Vector}\NormalTok{ a }\OtherTok{->} \DataTypeTok{Maybe}\NormalTok{ (}\DataTypeTok{Vec}\NormalTok{ n a)}
\NormalTok{mkVec v }\FunctionTok{|}\NormalTok{ V.length v }\FunctionTok{==}\NormalTok{ l }\FunctionTok{=} \DataTypeTok{Just}\NormalTok{ (}\DataTypeTok{UnsafeMkVec}\NormalTok{ v)}
        \FunctionTok{|}\NormalTok{ otherwise       }\FunctionTok{=} \DataTypeTok{Nothing}
  \KeywordTok{where}
\NormalTok{    l }\FunctionTok{=}\NormalTok{ fromIntegral (natVal (}\DataTypeTok{Proxy} \FunctionTok{@}\NormalTok{n))}
\end{Highlighting}
\end{Shaded}

Here, we use \texttt{ScopedTypeVariables} so we can refer to the \texttt{n} in
the type signature in the function body (for \texttt{natVal\ (Proxy\ @n)}), and
we need to use an explicit forall, then, to bring the \texttt{n} into scope.

We also use \texttt{TypeApplications} syntax (\texttt{Proxy\ @n}) so we can say
``We want a \texttt{Proxy\ ::\ Proxy\ n}'' and tell \texttt{natVal} that we want
the \texttt{Natural} for \texttt{n}.

\hypertarget{utilizing-type-level-guarantees}{%
\subsection{Utilizing type-level
guarantees}\label{utilizing-type-level-guarantees}}

Another operation we might want to do with vectors is do things with them that
might change their length in a predetermined way. We might want the type of our
vectors to describe the nature of the operations they are undergoing. For
example, if you saw a function:

\begin{Shaded}
\begin{Highlighting}[]
\OtherTok{someFunc ::}\NormalTok{ (a }\OtherTok{->}\NormalTok{ b) }\OtherTok{->} \DataTypeTok{Vec}\NormalTok{ n a }\OtherTok{->} \DataTypeTok{Vec}\NormalTok{ n b}
\end{Highlighting}
\end{Shaded}

You can see that it takes a function and a vector of length \texttt{n}, and
returns another vector of length \texttt{n}. Clearly, this function might be a
``map'' function, which applies the function to all of the values in the
\texttt{Vec}! We know that it must have the same length, so it can't drop or add
items. (However, it could still be shuffling or duplicating or permuting the
items, as long as the resulting length is the same)

In this situation, we can write such a mapping function in an ``unsafe'' way,
and then give it our type signature as a form of documentation:

\begin{Shaded}
\begin{Highlighting}[]
\CommentTok{-- source: https://github.com/mstksg/inCode/tree/master/code-samples/fixvec-2/VecWrapped.hs#L28-L29}
\OtherTok{mapVec ::}\NormalTok{ (a }\OtherTok{->}\NormalTok{ b) }\OtherTok{->} \DataTypeTok{Vec}\NormalTok{ n a }\OtherTok{->} \DataTypeTok{Vec}\NormalTok{ n b}
\NormalTok{mapVec f v }\FunctionTok{=} \DataTypeTok{UnsafeMkVec} \FunctionTok{$}\NormalTok{ V.map f (getVector v)}

\CommentTok{-- just for fun}
\CommentTok{-- source: https://github.com/mstksg/inCode/tree/master/code-samples/fixvec-2/VecWrapped.hs#L31-L32}
\KeywordTok{instance} \DataTypeTok{Functor}\NormalTok{ (}\DataTypeTok{Vec}\NormalTok{ n) }\KeywordTok{where}
\NormalTok{    fmap }\FunctionTok{=}\NormalTok{ mapVec}
\end{Highlighting}
\end{Shaded}

The compiler didn't help us write this function, and we have to be pretty
careful that the guarantees we specify in our types are reflected in the actual
unsafe operations. This is because our types don't \emph{structurally} enforce
their type-level lengths.

So, why bother? For us, here, our fixed-length vector types basically act as
``active documentation'', in a way. Compare:

\begin{Shaded}
\begin{Highlighting}[]
\CommentTok{-- | Maps the function over the items in the vector, returning a vector of the}
\CommentTok{-- same length}
\NormalTok{V.map}\OtherTok{ ::}\NormalTok{ (a }\OtherTok{->}\NormalTok{ b) }\OtherTok{->} \DataTypeTok{V.Vector}\NormalTok{ a }\OtherTok{->} \DataTypeTok{V.Vector}\NormalTok{ b}
\end{Highlighting}
\end{Shaded}

Which is the type signature we'd have to write for a map of an ``unsized''
vector (like from the \emph{Data.Vector} module)

We have to rely on the documentation to \emph{tell} us what the length of the
final resulting vector is, even though it can be known statically if you know
the length of the input vectors. The vectors have a \emph{static relationship}
in their length, but this isn't specified in a way that the compiler can take
advantage of.

By having our
\texttt{mapVec\ ::\ (a\ -\textgreater{}\ b)\ -\textgreater{}\ Vec\ n\ a\ -\textgreater{}\ Vec\ n\ b},
the relationship between the input lengths and output length is right there in
the types, when you \emph{use} \texttt{mapVec}, GHC is aware of the
relationships and can give you help in the form of typed hole suggestions and
informative type errors. You can even catch errors in logic at compile-time
instead of runtime!

\begin{Shaded}
\begin{Highlighting}[]
\CommentTok{-- the resulting vector's length is the sum of the input vectors' lengths}
\CommentTok{-- source: https://github.com/mstksg/inCode/tree/master/code-samples/fixvec-2/VecWrapped.hs#L34-L35}
\OtherTok{(++) ::} \DataTypeTok{Vec}\NormalTok{ n a }\OtherTok{->} \DataTypeTok{Vec}\NormalTok{ m a }\OtherTok{->} \DataTypeTok{Vec}\NormalTok{ (n }\FunctionTok{+}\NormalTok{ m) a}
\DataTypeTok{UnsafeMkVec}\NormalTok{ xs }\FunctionTok{++} \DataTypeTok{UnsafeMkVec}\NormalTok{ ys }\FunctionTok{=} \DataTypeTok{UnsafeMkVec}\NormalTok{ (xs }\FunctionTok{V.++}\NormalTok{ ys)}

\CommentTok{-- you must zip two vectors of the same length}
\CommentTok{-- source: https://github.com/mstksg/inCode/tree/master/code-samples/fixvec-2/VecWrapped.hs#L37-L38}
\OtherTok{zipVec ::} \DataTypeTok{Vec}\NormalTok{ n a }\OtherTok{->} \DataTypeTok{Vec}\NormalTok{ n b }\OtherTok{->} \DataTypeTok{Vec}\NormalTok{ n (a, b)}
\NormalTok{zipVec (}\DataTypeTok{UnsafeMkVec}\NormalTok{ xs) (}\DataTypeTok{UnsafeMkVec}\NormalTok{ ys) }\FunctionTok{=} \DataTypeTok{UnsafeMkVec}\NormalTok{ (V.zip xs ys)}

\CommentTok{-- type-level arithmetic to let us 'take'}
\CommentTok{-- source: https://github.com/mstksg/inCode/tree/master/code-samples/fixvec-2/VecWrapped.hs#L40-L43}
\OtherTok{takeVec ::}\NormalTok{ forall n m a}\FunctionTok{.} \DataTypeTok{KnownNat}\NormalTok{ n }\OtherTok{=>} \DataTypeTok{Vec}\NormalTok{ (n }\FunctionTok{+}\NormalTok{ m) a }\OtherTok{->} \DataTypeTok{Vec}\NormalTok{ n a}
\NormalTok{takeVec (}\DataTypeTok{UnsafeMkVec}\NormalTok{ xs) }\FunctionTok{=} \DataTypeTok{UnsafeMkVec}\NormalTok{ (V.take l xs)}
  \KeywordTok{where}
\NormalTok{    l }\FunctionTok{=}\NormalTok{ fromIntegral (natVal (}\DataTypeTok{Proxy} \FunctionTok{@}\NormalTok{n))}

\CommentTok{-- splitAt, as well}
\CommentTok{-- source: https://github.com/mstksg/inCode/tree/master/code-samples/fixvec-2/VecWrapped.hs#L45-L49}
\OtherTok{splitVec ::}\NormalTok{ forall n m a}\FunctionTok{.} \DataTypeTok{KnownNat}\NormalTok{ n }\OtherTok{=>} \DataTypeTok{Vec}\NormalTok{ (n }\FunctionTok{+}\NormalTok{ m) a }\OtherTok{->}\NormalTok{ (}\DataTypeTok{Vec}\NormalTok{ n a, }\DataTypeTok{Vec}\NormalTok{ m a)}
\NormalTok{splitVec (}\DataTypeTok{UnsafeMkVec}\NormalTok{ xs) }\FunctionTok{=}\NormalTok{ (}\DataTypeTok{UnsafeMkVec}\NormalTok{ ys, }\DataTypeTok{UnsafeMkVec}\NormalTok{ zs)}
  \KeywordTok{where}
\NormalTok{    l }\FunctionTok{=}\NormalTok{ fromIntegral (natVal (}\DataTypeTok{Proxy} \FunctionTok{@}\NormalTok{n))}
\NormalTok{    (ys, zs) }\FunctionTok{=}\NormalTok{ V.splitAt l xs}
\end{Highlighting}
\end{Shaded}

Here, \texttt{(+)} comes from
\emph{\href{http://hackage.haskell.org/package/base/docs/GHC-TypeNats.html}{GHC.TypeNats}},
which provides it as a type family (type-level function) we can use, with proper
meaning and semantics.

Note the usage of \texttt{(+)} for \texttt{takeVec} and \texttt{splitVec} to let
the function ensure that the input vector has ``enough'' (at least \texttt{n})
elements to do the taking.

\hypertarget{notes-on-the-typechecker}{%
\subsubsection{Notes on the typechecker}\label{notes-on-the-typechecker}}

GHC's typechecker works very well with concrete, monomorphic \texttt{Nat}s. For
example, \texttt{5\ +\ 3} will always typecheck as \texttt{8}, so you don't have
to worry at all about \texttt{takeVec}, \texttt{(++)}, and \texttt{splitVec}'s
usage of \texttt{(+)} if you work with monomorphic, specific \texttt{Nat}s.

However, GHC treats \texttt{(+)} ``opaquely'' when using using it with
polymorphic type variables. That means that \texttt{n\ +\ (m\ +\ o)} is seen as
a completely different type to GHC than \texttt{(n\ +\ m)\ +\ o} -- GHC doesn't
reduce \texttt{+}, and to it, they both just look like different trees. Remember
that one is \texttt{(+)\ n\ ((+)\ m\ o)}, and the other is
\texttt{(+)\ ((+)\ n\ m)\ o}. Completely different structure!

This comes up as an issue when you start doing non-trivial things, so it
sometimes helps to augment GHC's typechecker.

The
\emph{\href{https://hackage.haskell.org/package/ghc-typelits-natnormalise}{ghc-typelits-natnormalise}}
package provides such a plugin. If we pass it as a flag to GHC (as
\texttt{-fplugin\ GHC.TypeLits.NatNormalise}) or as a pragma:

\begin{Shaded}
\begin{Highlighting}[]
\OtherTok{\{-# OPTIONS_GHC -fplugin GHC.TypeLits.Normalise #-\}}
\end{Highlighting}
\end{Shaded}

Then GHC will be able to recognize the fact that \texttt{n\ +\ (m\ +\ o)} and
\texttt{(n\ +\ m)\ +\ o} are the same, and will unify them during typechecking.
It also provides normalization/unification for many other situations that we
``expect'' to work when using \texttt{*} and \texttt{+} on type variables.

\hypertarget{indexing}{%
\subsection{Indexing}\label{indexing}}

We need an appropriate type for indexing our vectors, but we'd like a type where
indexing is ``safe'' -- that is, that you can't compile a program that will
result in an index error.

For this, we can use the
\emph{\href{http://hackage.haskell.org/package/finite-typelits}{finite-typelits}}
package, which provides the \texttt{Finite\ n} type.

A \texttt{Finite\ n} type is a type with exactly \texttt{n} distinct
inhabitants/values. For example, \texttt{Finite\ 4} contains four ``anonymous''
inhabitants. For convenience, sometimes we like to name them 0, 1, 2, and 3. In
general, we sometimes refer to the values of type \texttt{Finite\ n} as 0
\ldots{} (n - 1).

So, we can imagine that \texttt{Finite\ 6} has inhabitants corresponding to 0,
1, 2, 3, 4, and 5. We can convert back and forth between a \texttt{Finite\ n}
and its \texttt{Integer} representation using \texttt{packFinite} and
\texttt{getFinite}:

\begin{Shaded}
\begin{Highlighting}[]
\OtherTok{packFinite ::} \DataTypeTok{KnownNat}\NormalTok{ n }\OtherTok{=>} \DataTypeTok{Integer}  \OtherTok{->} \DataTypeTok{Maybe}\NormalTok{ (}\DataTypeTok{Finite}\NormalTok{ n)}
\OtherTok{getFinite  ::}               \DataTypeTok{Finite}\NormalTok{ n }\OtherTok{->} \DataTypeTok{Integer}
\end{Highlighting}
\end{Shaded}

\begin{Shaded}
\begin{Highlighting}[]
\NormalTok{ghci}\FunctionTok{>}\NormalTok{ map packFinite [}\DecValTok{0}\FunctionTok{..}\DecValTok{3}\NormalTok{]}\OtherTok{ ::}\NormalTok{ [}\DataTypeTok{Maybe}\NormalTok{ (}\DataTypeTok{Finite} \DecValTok{3}\NormalTok{)]}
\NormalTok{[}\DataTypeTok{Just}\NormalTok{ (finite }\DecValTok{0}\NormalTok{), }\DataTypeTok{Just}\NormalTok{ (finite }\DecValTok{1}\NormalTok{), }\DataTypeTok{Just}\NormalTok{ (finite }\DecValTok{2}\NormalTok{), }\DataTypeTok{Nothing}\NormalTok{]}
\NormalTok{ghci}\FunctionTok{>}\NormalTok{ getFinite (finite }\DecValTok{2}\OtherTok{ ::} \DataTypeTok{Finite} \DecValTok{5}\NormalTok{)}
\DecValTok{2}
\end{Highlighting}
\end{Shaded}

We can use a \texttt{Finite\ n} to ``index'' a \texttt{Vector\ n\ a}. A
\texttt{Vector\ n\ a} has exactly \texttt{n} slots, and a \texttt{Finite\ n} has
\texttt{n} possible values. Clearly, \texttt{Finite\ n} only contains valid
indices into our vector!

\begin{Shaded}
\begin{Highlighting}[]
\CommentTok{-- source: https://github.com/mstksg/inCode/tree/master/code-samples/fixvec-2/VecWrapped.hs#L51-L52}
\NormalTok{index}\OtherTok{ ::} \DataTypeTok{Vec}\NormalTok{ n a }\OtherTok{->} \DataTypeTok{Finite}\NormalTok{ n }\OtherTok{->}\NormalTok{ a}
\NormalTok{index v i }\FunctionTok{=}\NormalTok{ getVector v }\FunctionTok{V.!}\NormalTok{ fromIntegral (getFinite i)}
\end{Highlighting}
\end{Shaded}

\texttt{index} will never fail at runtime due to a bad index --- do you see why?
Valid indices of a \texttt{Vector\ 5\ a} are the integers 0 to 4, and that is
precisely the exact things that \texttt{Finite\ 5} can store!

\hypertarget{generating}{%
\subsection{Generating}\label{generating}}

We can directly generate these vectors in interesting ways. Using return-type
polymorphism, we can have the user \emph{directly} request a vector length,
\emph{just} by using type inference or a type annotation. (kind of like with
\texttt{read})

For example, we can write a version of \texttt{replicate}:

\begin{Shaded}
\begin{Highlighting}[]
\CommentTok{-- source: https://github.com/mstksg/inCode/tree/master/code-samples/fixvec-2/VecWrapped.hs#L54-L57}
\NormalTok{replicate}\OtherTok{ ::}\NormalTok{ forall n a}\FunctionTok{.} \DataTypeTok{KnownNat}\NormalTok{ n }\OtherTok{=>}\NormalTok{ a }\OtherTok{->} \DataTypeTok{Vec}\NormalTok{ n a}
\NormalTok{replicate x }\FunctionTok{=} \DataTypeTok{UnsafeMkVec} \FunctionTok{$}\NormalTok{ V.replicate l x}
  \KeywordTok{where}
\NormalTok{    l }\FunctionTok{=}\NormalTok{ fromIntegral (natVal (}\DataTypeTok{Proxy} \FunctionTok{@}\NormalTok{n))}
\end{Highlighting}
\end{Shaded}

\begin{Shaded}
\begin{Highlighting}[]
\NormalTok{ghci}\FunctionTok{>}\NormalTok{ replicate }\CharTok{'a'}\OtherTok{ ::} \DataTypeTok{Vec} \DecValTok{5} \DataTypeTok{Char}
\DataTypeTok{UnsafeMkVec}\NormalTok{ (V.fromList [}\CharTok{'a'}\NormalTok{,}\CharTok{'a'}\NormalTok{,}\CharTok{'a'}\NormalTok{,}\CharTok{'a'}\NormalTok{,}\CharTok{'a'}\NormalTok{])}
\end{Highlighting}
\end{Shaded}

Note that normally, \texttt{replicate} takes an \texttt{Int} argument so that
the user can give how long the resulting vector needs to be. However, with our
new \texttt{replicate}, we don't need that \texttt{Int} argument --- the size of
the vector we want can more often than not be inferred auto-magically using type
inference!

With this new cleaner type signature, we can actually see that
\texttt{replicate}'s type is something very familiar. Look at it carefully:

\begin{Shaded}
\begin{Highlighting}[]
\NormalTok{replicate}\OtherTok{ ::} \DataTypeTok{KnownNat}\NormalTok{ n }\OtherTok{=>}\NormalTok{ a }\OtherTok{->} \DataTypeTok{Vec}\NormalTok{ n a}
\end{Highlighting}
\end{Shaded}

You might recognize it as very similar to haskellism \texttt{pure}:

\begin{Shaded}
\begin{Highlighting}[]
\NormalTok{pure}\OtherTok{ ::} \DataTypeTok{Applicative}\NormalTok{ f }\OtherTok{=>}\NormalTok{ a }\OtherTok{->}\NormalTok{ f a}
\end{Highlighting}
\end{Shaded}

\texttt{replicate} is actually \texttt{pure} for the Applicative instance of
\texttt{Vec\ n}! As an extra challenge, what would
\texttt{\textless{}*\textgreater{}} be? See
\href{https://github.com/mstksg/inCode/tree/master/code-samples/fixvec-2/VecWrapped.hs\#L59-L61}{the
solution} if you want to check your answer!

\hypertarget{generating-with-indices}{%
\subsubsection{Generating with indices}\label{generating-with-indices}}

We can be a little more fancy with \texttt{replicate}, to get what we normally
call \texttt{generate}:

\begin{Shaded}
\begin{Highlighting}[]
\CommentTok{-- source: https://github.com/mstksg/inCode/tree/master/code-samples/fixvec-2/VecWrapped.hs#L63-L66}
\OtherTok{generate ::}\NormalTok{ forall n a}\FunctionTok{.} \DataTypeTok{KnownNat}\NormalTok{ n }\OtherTok{=>}\NormalTok{ (}\DataTypeTok{Finite}\NormalTok{ n }\OtherTok{->}\NormalTok{ a) }\OtherTok{->} \DataTypeTok{Vec}\NormalTok{ n a}
\NormalTok{generate f }\FunctionTok{=} \DataTypeTok{UnsafeMkVec} \FunctionTok{$}\NormalTok{ V.generate l (f }\FunctionTok{.}\NormalTok{ fromIntegral)}
  \KeywordTok{where}
\NormalTok{    l }\FunctionTok{=}\NormalTok{ fromIntegral (natVal (}\DataTypeTok{Proxy} \FunctionTok{@}\NormalTok{n))}
\end{Highlighting}
\end{Shaded}

\hypertarget{type-safety-and-positives-and-negatives}{%
\subsubsection{Type-Safety and positives and
negatives}\label{type-safety-and-positives-and-negatives}}

I think it's an interesting point that we're using \texttt{Finite\ n} in a
different sense here than in \texttt{index}, and for different reasons. In
\texttt{index}, \texttt{Finite} is in the ``negative'' position --- it's
something that the function ``takes''. In \texttt{generate}, \texttt{Finite} is
in the ``positive'' position --- it's something that the function ``gives'' (to
the \texttt{f} in \texttt{generate\ f}).

In the negative position, \texttt{Finite\ n} and type-safety is useful because:

\begin{enumerate}
\def\labelenumi{\arabic{enumi}.}
\tightlist
\item
  It tells the user what sort of values that the function expects. The user
  \emph{knows}, just from the type, that indexing a \texttt{Vec\ 5\ a} requires
  a \texttt{Finite\ 5}, or a number between 0 and 4.
\item
  It guarantees that whatever \texttt{Finite\ n} index you give to
  \texttt{index} is a \emph{valid one}. It's impossible to give \texttt{index}
  an ``invalid index'', so \texttt{index} is allowed to use ``unsafe indexing''
  in its implementation, knowing that nothing bad can be given.
\item
  It lets you develop code in ``typed-hole'' style: if a function requires a
  \texttt{Finite\ 4}, put an underscore there, and GHC will tell you about all
  the \texttt{Finite\ 4}s you have in scope. It can help you write your code for
  you!
\end{enumerate}

In the positive position, \texttt{Finite\ n} and the type-safety have different
uses and advantages: it tells the user what sort of values the function can
return, and also also the type of values that the user has to be expected to
handle. For example, in \texttt{generate}, the fact that the user has to provide
a \texttt{Finite\ n\ -\textgreater{}\ a} tells the user that they have to handle
every number between 0 and n-1, and nothing else.

\hypertarget{moving-between-sized-and-unsized}{%
\subsection{Moving between Sized and
Unsized}\label{moving-between-sized-and-unsized}}

One key part of our API is missing: how to convert between ``sized'' and
``unsized'' vectors.

Converting from sized to unsized is easy, and we already have it:

\begin{Shaded}
\begin{Highlighting}[]
\OtherTok{getVector ::} \DataTypeTok{Vec}\NormalTok{ n a }\OtherTok{->} \DataTypeTok{V.Vector}\NormalTok{ a}
\end{Highlighting}
\end{Shaded}

Converting from unsized to sized is harder. We already saw a ``shoe-horning''
method, if we know the size we want at compile-time:

\begin{Shaded}
\begin{Highlighting}[]
\OtherTok{mkVec ::}\NormalTok{ forall n}\FunctionTok{.} \DataTypeTok{KnownNat}\NormalTok{ n }\OtherTok{=>} \DataTypeTok{V.Vector}\NormalTok{ a }\OtherTok{->} \DataTypeTok{Maybe}\NormalTok{ (}\DataTypeTok{Vec}\NormalTok{ n a)}
\end{Highlighting}
\end{Shaded}

But what if we don't know what size \texttt{n} we want? What if we want
\texttt{n} to be whatever the actual size of the input vector is?

In general we can't predict the size of our input vector at compile-time, so we
can't just directly put in a size we want. What we want is a method to return a
\texttt{Vec\ n}, where \texttt{n} is the length of the input vector, determined
at runtime.

I'm going to try to convince you that a plausible API is:

\begin{Shaded}
\begin{Highlighting}[]
\NormalTok{withVec}
\OtherTok{    ::} \DataTypeTok{V.Vector}\NormalTok{ a}
    \OtherTok{->}\NormalTok{ (forall n}\FunctionTok{.} \DataTypeTok{KnownNat}\NormalTok{ n }\OtherTok{=>} \DataTypeTok{Vec}\NormalTok{ n a }\OtherTok{->}\NormalTok{ r)}
    \OtherTok{->}\NormalTok{ r}
\end{Highlighting}
\end{Shaded}

(Note: this does require \texttt{RankNTypes})

People familiar with dependent types might recognize that \texttt{withVec} is a
function that takes an unsized vector and returns an \emph{existentially
quantified} sized vector, in CPS-style. Basically, give the function a vector,
and a way to ``handle'' a \texttt{Vec\ n} of \emph{any possible size}. The
function will then give your handler a \texttt{Vec\ n} of the proper type/size.
The \emph{function} gets to chose the \texttt{n} that you must handle.

Within your continuation/handler, you can take advantage of the size type, and
do take advantage of all of the type-level guarantees and benefits of a
length-indexed vector. In a way, it is its own ``world'' where your vector has a
fixed size. However, the caveat is that you have to treat the size
\emph{universally} --- you have to be able to handle any possible size given to
you, in a parametrically polymorphic way.

For example:

\begin{Shaded}
\begin{Highlighting}[]
\NormalTok{ghci}\FunctionTok{>}\NormalTok{ myVector }\FunctionTok{=}\NormalTok{ V.fromList [}\DecValTok{10}\NormalTok{,}\DecValTok{5}\NormalTok{,}\DecValTok{8}\NormalTok{]}\OtherTok{ ::} \DataTypeTok{V.Vector} \DataTypeTok{Int}
\NormalTok{ghci}\FunctionTok{>}\NormalTok{ withVec myVector }\FunctionTok{$}\NormalTok{ \textbackslash{}(}\OtherTok{v ::} \DataTypeTok{Vec}\NormalTok{ n }\DataTypeTok{Int}\NormalTok{) }\OtherTok{->}
          \CommentTok{-- in this function body, `v :: Vec 3 Int`, and `n ~ 3`}
          \CommentTok{-- whatever I return here will be the return value of the entire line}
          \KeywordTok{case}\NormalTok{ packFinite }\DecValTok{1}\OtherTok{ ::} \DataTypeTok{Maybe}\NormalTok{ (}\DataTypeTok{Finite}\NormalTok{ n) }\KeywordTok{of}      \CommentTok{-- Finite 3}
            \DataTypeTok{Nothing} \OtherTok{->} \DecValTok{0}
            \DataTypeTok{Just}\NormalTok{ i  }\OtherTok{->}\NormalTok{ v }\OtherTok{`index`}\NormalTok{ i}
\DecValTok{5}
\end{Highlighting}
\end{Shaded}

We could write, say, a function to always safely get the third item:

\begin{Shaded}
\begin{Highlighting}[]
\CommentTok{-- source: https://github.com/mstksg/inCode/tree/master/code-samples/fixvec-2/VecWrapped.hs#L68-L69}
\OtherTok{getThird ::} \DataTypeTok{V.Vector}\NormalTok{ a }\OtherTok{->} \DataTypeTok{Maybe}\NormalTok{ a}
\NormalTok{getThird v }\FunctionTok{=}\NormalTok{ withVec v }\FunctionTok{$}\NormalTok{ \textbackslash{}v' }\OtherTok{->}\NormalTok{ fmap (v' }\OtherTok{`index`}\NormalTok{) (packFinite }\DecValTok{2}\NormalTok{)}
\end{Highlighting}
\end{Shaded}

And we can run it:

\begin{Shaded}
\begin{Highlighting}[]
\NormalTok{ghci}\FunctionTok{>}\NormalTok{ getThird }\FunctionTok{$}\NormalTok{ V.fromList [}\DecValTok{1}\NormalTok{,}\DecValTok{2}\NormalTok{,}\DecValTok{3}\NormalTok{]}
\DataTypeTok{Just} \DecValTok{3}
\NormalTok{ghci}\FunctionTok{>}\NormalTok{ getThird }\FunctionTok{$}\NormalTok{ V.fromList [}\DecValTok{1}\NormalTok{,}\DecValTok{2}\NormalTok{]}
\DataTypeTok{Nothing}
\end{Highlighting}
\end{Shaded}

We can even do something silly like convert an unsized vector to a sized vector
and then back again:

\begin{Shaded}
\begin{Highlighting}[]
\CommentTok{-- source: https://github.com/mstksg/inCode/tree/master/code-samples/fixvec-2/VecWrapped.hs#L71-L72}
\OtherTok{vectorToVector ::} \DataTypeTok{V.Vector}\NormalTok{ a }\OtherTok{->} \DataTypeTok{V.Vector}\NormalTok{ a}
\NormalTok{vectorToVector v }\FunctionTok{=}\NormalTok{ withVec v getVector}
\end{Highlighting}
\end{Shaded}

Now that I've (hopefully) convinced you that this function really does convert
an unsized vector into a sized vector that you can use, let's see how we can
implement it!

To do this, we can take advantage of the \texttt{someNatVal} function (from
\emph{\href{http://hackage.haskell.org/package/base/docs/GHC-TypeNats.html}{GHC.TypeNats}}):

\begin{Shaded}
\begin{Highlighting}[]
\KeywordTok{data} \DataTypeTok{SomeNat} \FunctionTok{=}\NormalTok{ forall n}\FunctionTok{.} \DataTypeTok{KnownNat}\NormalTok{ n }\OtherTok{=>} \DataTypeTok{SomeNat}\NormalTok{ (}\DataTypeTok{Proxy}\NormalTok{ n)}

\OtherTok{someNatVal ::} \DataTypeTok{Natural} \OtherTok{->} \DataTypeTok{SomeNat}
\end{Highlighting}
\end{Shaded}

\texttt{SomeNat} contains what we call an existentially quantified type,
\texttt{n}. Basically, a value of \texttt{SomeNat} contains a \texttt{Proxy\ n}
with \emph{some specific \texttt{n}}, that is hidden ``inside'' the constructor.
The only way to figure it out is to pattern match on the constructor and use it
in a generic and parametrically polymorphic way. Sound familiar?

\texttt{someNatVal} converts \texttt{Natural} (a non-negative Integer type) into
a \texttt{SomeNat} --- it ``picks'' the right \texttt{n} (the one that
corresponds to that \texttt{Natural}) and stuffs/hides it into \texttt{SomeNat}.
We can leverage this to write our \texttt{withVec}:

\begin{Shaded}
\begin{Highlighting}[]
\CommentTok{-- source: https://github.com/mstksg/inCode/tree/master/code-samples/fixvec-2/VecWrapped.hs#L74-L76}
\OtherTok{withVec ::} \DataTypeTok{V.Vector}\NormalTok{ a }\OtherTok{->}\NormalTok{ (forall n}\FunctionTok{.} \DataTypeTok{KnownNat}\NormalTok{ n }\OtherTok{=>} \DataTypeTok{Vec}\NormalTok{ n a }\OtherTok{->}\NormalTok{ r) }\OtherTok{->}\NormalTok{ r}
\NormalTok{withVec v f }\FunctionTok{=} \KeywordTok{case}\NormalTok{ someNatVal (fromIntegral (V.length v)) }\KeywordTok{of}
    \DataTypeTok{SomeNat}\NormalTok{ (}\DataTypeTok{Proxy}\OtherTok{ ::} \DataTypeTok{Proxy}\NormalTok{ m) }\OtherTok{->}\NormalTok{ f (}\DataTypeTok{UnsafeMkVec} \FunctionTok{@}\NormalTok{m v)}
\end{Highlighting}
\end{Shaded}

(The \texttt{TypeApplications} syntax \texttt{@m} is used with
\texttt{UnsafeMkVec} to specify that we want a \texttt{Vec\ m\ a}.)

This process is actually called ``reification'' -- we take a value-level runtime
property (the length) and ``reify'' it, bringing it up to the type-level.

And now, we have both of our conversion functions! We can convert from sized to
unsized using \texttt{getVector}, and from unsized to sized using
\texttt{withVec}.

\hypertarget{verifying-properties}{%
\subsection{Verifying Properties}\label{verifying-properties}}

The final useful API aspect we will be looking at is how to verify properties of
our vector lengths at the type level, and let us use those properties.

One common thing we might want to do is ensure that two vectors have the same
length. This might happen when we use \texttt{withVec} from two different
vectors, and we get a \texttt{Vec\ n\ a} and \texttt{Vec\ m\ a} of two
(potentially) different lengths.

We can do this using \texttt{sameNat} from
\emph{\href{http://hackage.haskell.org/package/base/docs/GHC-TypeNats.html}{GHC.TypeNats}}:

\begin{Shaded}
\begin{Highlighting}[]
\CommentTok{-- `Type` is just a synonym for * from Data.Kind}
\CommentTok{-- from the module Data.Type.Equality}
\KeywordTok{data}\OtherTok{ (:~:) ::}\NormalTok{ k }\OtherTok{->}\NormalTok{ k }\OtherTok{->} \DataTypeTok{Type} \KeywordTok{where}
    \DataTypeTok{Refl}\OtherTok{ ::}\NormalTok{ x }\FunctionTok{:~:}\NormalTok{ x}

\NormalTok{sameNat}
\OtherTok{    ::}\NormalTok{ (}\DataTypeTok{KnownNat}\NormalTok{ n, }\DataTypeTok{KnownNat}\NormalTok{ m)}
    \OtherTok{=>} \DataTypeTok{Proxy}\NormalTok{ n}
    \OtherTok{->} \DataTypeTok{Proxy}\NormalTok{ m}
    \OtherTok{->} \DataTypeTok{Maybe}\NormalTok{ (n }\FunctionTok{:~:}\NormalTok{ m)}
\end{Highlighting}
\end{Shaded}

The only way we can have a non-bottom value of type
\texttt{n\ :\textasciitilde{}:\ m} is with the \texttt{Refl} constructor, which
can only be used in the case that \texttt{n} and \texttt{m} are equal.
\texttt{sameNat} gives us that \texttt{Refl}, if possible --- that is, if
\texttt{n} and \texttt{m} are equal. If not, it gives us \texttt{Nothing}.

Now, we can write:

\begin{Shaded}
\begin{Highlighting}[]
\CommentTok{-- source: https://github.com/mstksg/inCode/tree/master/code-samples/fixvec-2/VecWrapped.hs#L78-L81}
\OtherTok{exactLength ::}\NormalTok{ forall n m a}\FunctionTok{.}\NormalTok{ (}\DataTypeTok{KnownNat}\NormalTok{ n, }\DataTypeTok{KnownNat}\NormalTok{ m) }\OtherTok{=>} \DataTypeTok{Vec}\NormalTok{ n a }\OtherTok{->} \DataTypeTok{Maybe}\NormalTok{ (}\DataTypeTok{Vec}\NormalTok{ m a)}
\NormalTok{exactLength v }\FunctionTok{=} \KeywordTok{case}\NormalTok{ sameNat (}\DataTypeTok{Proxy} \FunctionTok{@}\NormalTok{n) (}\DataTypeTok{Proxy} \FunctionTok{@}\NormalTok{m) }\KeywordTok{of}
    \DataTypeTok{Just} \DataTypeTok{Refl} \OtherTok{->} \DataTypeTok{Just}\NormalTok{ v     }\CommentTok{-- here, n ~ m, so a `Vec n a` is a `Vec m a`, too}
    \DataTypeTok{Nothing}   \OtherTok{->} \DataTypeTok{Nothing}
\end{Highlighting}
\end{Shaded}

(We could also write this by using \texttt{getVector} and \texttt{mkVec}, which
wraps and unwraps, but let's pretend it is expensive to construct and
re-construct).

Now we can do:

\begin{Shaded}
\begin{Highlighting}[]
\CommentTok{-- source: https://github.com/mstksg/inCode/tree/master/code-samples/fixvec-2/VecWrapped.hs#L37-L37}
\OtherTok{zipVec ::} \DataTypeTok{Vec}\NormalTok{ n a }\OtherTok{->} \DataTypeTok{Vec}\NormalTok{ n b }\OtherTok{->} \DataTypeTok{Vec}\NormalTok{ n (a, b)}

\CommentTok{-- source: https://github.com/mstksg/inCode/tree/master/code-samples/fixvec-2/VecWrapped.hs#L83-L89}
\OtherTok{zipSame ::}\NormalTok{ forall a b}\FunctionTok{.} \DataTypeTok{V.Vector}\NormalTok{ a }\OtherTok{->} \DataTypeTok{V.Vector}\NormalTok{ b }\OtherTok{->} \DataTypeTok{Maybe}\NormalTok{ (}\DataTypeTok{V.Vector}\NormalTok{ (a, b))}
\NormalTok{zipSame v1 v2 }\FunctionTok{=}\NormalTok{ withVec v1 }\FunctionTok{$}\NormalTok{ \textbackslash{}(}\OtherTok{v1' ::} \DataTypeTok{Vec}\NormalTok{ n a) }\OtherTok{->}
\NormalTok{                withVec v2 }\FunctionTok{$}\NormalTok{ \textbackslash{}(}\OtherTok{v2' ::} \DataTypeTok{Vec}\NormalTok{ m b) }\OtherTok{->}
      \KeywordTok{case}\NormalTok{ exactLength v1' }\KeywordTok{of}
        \DataTypeTok{Just}\NormalTok{ v1Same }\OtherTok{->} \DataTypeTok{Just} \FunctionTok{$}\NormalTok{ getVector}
\NormalTok{                          (zipVec v1Same v2')     }\CommentTok{-- v1' has the same length as v2'}
        \DataTypeTok{Nothing}     \OtherTok{->} \DataTypeTok{Nothing}
\end{Highlighting}
\end{Shaded}

Which will zip two unsized vectors, but only if their lengths are the same.

Now, ``checking that the length is a certain length'' is literally the least
interesting property we can test about our vectors. There are definitely more
interesting properties we can test, like whether or not our lengths are even or
odd, if they are greater than a certain number, etc.; for these, the process is
essentially the same: find some way, \emph{at runtime}, to get some sort of
witness for the property you want. In our case, our witness was
\texttt{n\ :\textasciitilde{}:\ m}, which witnessed the fact that
\texttt{n\ \textasciitilde{}\ m}. Different libraries might provide different
witnesses that might be useful. But the general process is

\begin{enumerate}
\def\labelenumi{\arabic{enumi}.}
\tightlist
\item
  Find a way to get your witness, using some runtime function (that will
  probably return \texttt{Maybe})
\item
  Pattern match on your witness, and see that property realized and usable by
  GHC/the type checker!
\end{enumerate}

\hypertarget{help-from-singletons}{%
\subsection{Help from singletons}\label{help-from-singletons}}

You have probably heard that \texttt{TypeNats} provides a very bare-bones and
primitive interface. This is true. Its interface also sometimes doesn't play
well with other type-level mechanisms you might want to try. To prepare you for
the real world, let's re-implement these things using the
\emph{\href{http://hackage.haskell.org/package/singletons}{singletons}} library,
which provides a unified interface for type-level programming in general.

Instead of \texttt{KnownNat}, \texttt{Proxy}, \texttt{natVal}, \texttt{SomeNat},
and \texttt{someNatVal}, we can use the singletons equivalents, \texttt{Sing},
\texttt{fromSing}, \texttt{SomeSing}, and \texttt{toSing}:\footnote{For
  singletons \textgreater{} 2.3 \texttt{fromSing} and \texttt{toSing} give and
  take \texttt{Natural} when going to \texttt{Nat}. However, for 2.3.1 and
  below, they give/take \texttt{Integer} instead.}

\begin{Shaded}
\begin{Highlighting}[]
\CommentTok{-- TypeNats style}
\OtherTok{natVal ::} \DataTypeTok{KnownNat}\NormalTok{ n }\OtherTok{=>}\NormalTok{ p n }\OtherTok{->} \DataTypeTok{Natural}

\CommentTok{-- Singletons style}
\OtherTok{sing     ::} \DataTypeTok{KnownNat}\NormalTok{ n }\OtherTok{=>} \DataTypeTok{Sing}\NormalTok{ n}
\OtherTok{fromSing ::} \DataTypeTok{Sing}\NormalTok{ n }\OtherTok{->} \DataTypeTok{Natural}       \CommentTok{-- (for n :: Nat)}

\CommentTok{-- TypeNats style}
\KeywordTok{data} \DataTypeTok{SomeNat} \FunctionTok{=}\NormalTok{ forall n}\FunctionTok{.} \DataTypeTok{KnownNat}\NormalTok{ n }\OtherTok{=>} \DataTypeTok{SomeNat}\NormalTok{ (}\DataTypeTok{Proxy}\NormalTok{ n)}
\OtherTok{someNatVal ::} \DataTypeTok{Natural} \OtherTok{->} \DataTypeTok{SomeNat}

\CommentTok{-- Singletons style}
\KeywordTok{data} \DataTypeTok{SomeSing} \DataTypeTok{Nat} \FunctionTok{=}\NormalTok{ forall n}\FunctionTok{.} \DataTypeTok{SomeSing}\NormalTok{ (}\DataTypeTok{Sing}\NormalTok{ n)}
\OtherTok{toSing ::} \DataTypeTok{Natural} \OtherTok{->} \DataTypeTok{SomeSing} \DataTypeTok{Nat}

\OtherTok{withSomeSing ::} \DataTypeTok{Natural} \OtherTok{->}\NormalTok{ (forall n}\FunctionTok{.} \DataTypeTok{Sing}\NormalTok{ n }\OtherTok{->}\NormalTok{ r) }\OtherTok{->}\NormalTok{ r}

\CommentTok{-- TypeNats style}
\OtherTok{sameNat ::}\NormalTok{ (}\DataTypeTok{KnownNat}\NormalTok{ n, }\DataTypeTok{KnownNat}\NormalTok{ m) }\OtherTok{=>} \DataTypeTok{Proxy}\NormalTok{ n }\OtherTok{->} \DataTypeTok{Proxy}\NormalTok{ m }\OtherTok{->} \DataTypeTok{Maybe}\NormalTok{ (n }\FunctionTok{:~:}\NormalTok{ m)}

\CommentTok{-- Singletons style}
\CommentTok{-- from Data.Singletons.Decide}
\CommentTok{-- for our purposes, Decision is basically a fancy Maybe}
\KeywordTok{data} \DataTypeTok{Decision}\NormalTok{ a }\FunctionTok{=} \DataTypeTok{Proved}\NormalTok{ a }\FunctionTok{|} \DataTypeTok{Disproved}\NormalTok{ (a }\OtherTok{->} \DataTypeTok{Void}\NormalTok{)}
\OtherTok{(%~) ::} \DataTypeTok{Sing}\NormalTok{ n }\OtherTok{->} \DataTypeTok{Sing}\NormalTok{ m }\OtherTok{->} \DataTypeTok{Decision}\NormalTok{ (n }\FunctionTok{:~:}\NormalTok{ m)}
\end{Highlighting}
\end{Shaded}

Hopefully the above should give you a nice ``key'' for translating between the
two styles. But here are some practical translations:

\begin{Shaded}
\begin{Highlighting}[]
\CommentTok{-- "explicit Sing" style}
\CommentTok{-- source: https://github.com/mstksg/inCode/tree/master/code-samples/fixvec-2/VecWrappedSingletons.hs#L20-L24}
\OtherTok{mkVec_ ::} \DataTypeTok{Sing}\NormalTok{ n }\OtherTok{->} \DataTypeTok{V.Vector}\NormalTok{ a }\OtherTok{->} \DataTypeTok{Maybe}\NormalTok{ (}\DataTypeTok{Vec}\NormalTok{ n a)}
\NormalTok{mkVec_ s v }\FunctionTok{|}\NormalTok{ V.length v }\FunctionTok{==}\NormalTok{ l }\FunctionTok{=} \DataTypeTok{Just}\NormalTok{ (}\DataTypeTok{UnsafeMkVec}\NormalTok{ v)}
           \FunctionTok{|}\NormalTok{ otherwise       }\FunctionTok{=} \DataTypeTok{Nothing}
  \KeywordTok{where}
\NormalTok{    l }\FunctionTok{=}\NormalTok{ fromIntegral (fromSing s)}

\CommentTok{-- "implicit" style}
\CommentTok{-- source: https://github.com/mstksg/inCode/tree/master/code-samples/fixvec-2/VecWrappedSingletons.hs#L26-L30}
\OtherTok{mkVec ::}\NormalTok{ forall n a}\FunctionTok{.} \DataTypeTok{KnownNat}\NormalTok{ n }\OtherTok{=>} \DataTypeTok{V.Vector}\NormalTok{ a }\OtherTok{->} \DataTypeTok{Maybe}\NormalTok{ (}\DataTypeTok{Vec}\NormalTok{ n a)}
\NormalTok{mkVec v }\FunctionTok{|}\NormalTok{ V.length v }\FunctionTok{==}\NormalTok{ l }\FunctionTok{=} \DataTypeTok{Just}\NormalTok{ (}\DataTypeTok{UnsafeMkVec}\NormalTok{ v)}
        \FunctionTok{|}\NormalTok{ otherwise       }\FunctionTok{=} \DataTypeTok{Nothing}
  \KeywordTok{where}
\NormalTok{    l }\FunctionTok{=}\NormalTok{ fromIntegral (fromSing (}\OtherTok{sing ::} \DataTypeTok{Sing}\NormalTok{ n))}

\CommentTok{-- alternatively, re-using `mkVec_`}
\OtherTok{mkVec ::} \DataTypeTok{KnownNat}\NormalTok{ n }\OtherTok{=>} \DataTypeTok{V.Vector}\NormalTok{ a }\OtherTok{->} \DataTypeTok{Maybe}\NormalTok{ (}\DataTypeTok{Vec}\NormalTok{ n a)}
\NormalTok{mkVec }\FunctionTok{=}\NormalTok{ mkVec_ sing}
\end{Highlighting}
\end{Shaded}

As you can see, in singletons, we have the luxury of defining our functions in
``explicit'' style (where the user passes in a \texttt{Sing} token which reveals
what length they want) or ``implicit'' style (where the length is inferred from
the return type, requiring a \texttt{KnownNat\ n\ =\textgreater{}} constraint),
like we have been writing up to this point. \texttt{Sing\ n\ -\textgreater{}}
and \texttt{KnownNat\ n\ =\textgreater{}} really have the same power. You can
think of \texttt{Sing\ n} as a token that carries around
\texttt{KnownNat\ n\ =\textgreater{}}, in a way.

\begin{Shaded}
\begin{Highlighting}[]
\CommentTok{-- source: https://github.com/mstksg/inCode/tree/master/code-samples/fixvec-2/VecWrappedSingletons.hs#L32-L42}
\OtherTok{replicate_ ::} \DataTypeTok{Sing}\NormalTok{ n }\OtherTok{->}\NormalTok{ a }\OtherTok{->} \DataTypeTok{Vec}\NormalTok{ n a}
\NormalTok{replicate_ s x }\FunctionTok{=} \DataTypeTok{UnsafeMkVec} \FunctionTok{$}\NormalTok{ V.replicate l x}
  \KeywordTok{where}
\NormalTok{    l }\FunctionTok{=}\NormalTok{ fromIntegral (fromSing s)}

\NormalTok{replicate}\OtherTok{ ::} \DataTypeTok{KnownNat}\NormalTok{ n }\OtherTok{=>}\NormalTok{ a }\OtherTok{->} \DataTypeTok{Vec}\NormalTok{ n a}
\NormalTok{replicate }\FunctionTok{=}\NormalTok{ replicate_ sing}

\OtherTok{withVec ::} \DataTypeTok{V.Vector}\NormalTok{ a }\OtherTok{->}\NormalTok{ (forall n}\FunctionTok{.} \DataTypeTok{Sing}\NormalTok{ n }\OtherTok{->} \DataTypeTok{Vec}\NormalTok{ n a }\OtherTok{->}\NormalTok{ r) }\OtherTok{->}\NormalTok{ r}
\NormalTok{withVec v f }\FunctionTok{=} \KeywordTok{case}\NormalTok{ toSing (fromIntegral (V.length v)) }\KeywordTok{of}
    \DataTypeTok{SomeSing}\NormalTok{ s }\OtherTok{->}\NormalTok{ f s (}\DataTypeTok{UnsafeMkVec}\NormalTok{ v)}

\CommentTok{-- alternatively, skipping `SomeSing` altogether:}
\CommentTok{-- source: https://github.com/mstksg/inCode/tree/master/code-samples/fixvec-2/VecWrappedSingletons.hs#L44-L54}
\OtherTok{withVec' ::} \DataTypeTok{V.Vector}\NormalTok{ a }\OtherTok{->}\NormalTok{ (forall n}\FunctionTok{.} \DataTypeTok{Sing}\NormalTok{ n }\OtherTok{->} \DataTypeTok{Vec}\NormalTok{ n a }\OtherTok{->}\NormalTok{ r) }\OtherTok{->}\NormalTok{ r}
\NormalTok{withVec' v0 f }\FunctionTok{=}\NormalTok{ withSomeSing (fromIntegral (V.length v0)) }\FunctionTok{$}\NormalTok{ \textbackslash{}s }\OtherTok{->}
\NormalTok{    f s (}\DataTypeTok{UnsafeMkVec}\NormalTok{ v0)}

\OtherTok{exactLength_ ::} \DataTypeTok{Sing}\NormalTok{ m }\OtherTok{->} \DataTypeTok{Sing}\NormalTok{ n }\OtherTok{->} \DataTypeTok{Vec}\NormalTok{ n a }\OtherTok{->} \DataTypeTok{Maybe}\NormalTok{ (}\DataTypeTok{Vec}\NormalTok{ m a)}
\NormalTok{exactLength_ sM sN v }\FunctionTok{=} \KeywordTok{case}\NormalTok{ sM }\FunctionTok{%~}\NormalTok{ sN }\KeywordTok{of}
    \DataTypeTok{Proved} \DataTypeTok{Refl} \OtherTok{->} \DataTypeTok{Just}\NormalTok{ v}
    \DataTypeTok{Disproved}\NormalTok{ _  }\OtherTok{->} \DataTypeTok{Nothing}

\OtherTok{exactLength ::}\NormalTok{ (}\DataTypeTok{KnownNat}\NormalTok{ m, }\DataTypeTok{KnownNat}\NormalTok{ n) }\OtherTok{=>} \DataTypeTok{Vec}\NormalTok{ n a }\OtherTok{->} \DataTypeTok{Maybe}\NormalTok{ (}\DataTypeTok{Vec}\NormalTok{ m a)}
\NormalTok{exactLength }\FunctionTok{=}\NormalTok{ exactLength_ sing sing}
\end{Highlighting}
\end{Shaded}

Note that you \emph{aren't} required to implement both a \texttt{replicate\_}
and \texttt{replicate} --- I'm just including them here to show that both API's
(implicit and explicit) are possible. (You can always just directly use
\texttt{sing} right away before getting started to get the \texttt{Sing\ n} that
those functions use, and so skip \texttt{replicate\_} and other explicit
variants)

One slight bit of friction comes when using libraries that work with
\texttt{KnownNat}, like \emph{finite-typelits} and the \texttt{Finite} type. But
we can convert between the two using \texttt{SNat} or \texttt{withKnownNat}

\begin{Shaded}
\begin{Highlighting}[]
\CommentTok{-- SNat can be used to construct a `Sing` if we have a `KnownNat` constraint}
\CommentTok{-- It can also be pattern matched on to reveal a `KnownNat constraint`}
\DataTypeTok{SNat}\OtherTok{ ::} \DataTypeTok{KnownNat}\NormalTok{ n }\OtherTok{=>} \DataTypeTok{Sing}\NormalTok{ n}

\CommentTok{-- we can give a `Sing n` and be able to execute something in the context where}
\CommentTok{-- that `n` has a `KnownNat` constraint}
\OtherTok{withKnownNat ::} \DataTypeTok{Sing}\NormalTok{ n }\OtherTok{->}\NormalTok{ (}\DataTypeTok{KnownNat}\NormalTok{ n }\OtherTok{=>}\NormalTok{ r) }\OtherTok{->}\NormalTok{ r}
\end{Highlighting}
\end{Shaded}

\begin{Shaded}
\begin{Highlighting}[]
\CommentTok{-- source: https://github.com/mstksg/inCode/tree/master/code-samples/fixvec-2/VecWrappedSingletons.hs#L56-L60}
\OtherTok{generate_ ::} \DataTypeTok{Sing}\NormalTok{ n }\OtherTok{->}\NormalTok{ (}\DataTypeTok{Finite}\NormalTok{ n }\OtherTok{->}\NormalTok{ a) }\OtherTok{->} \DataTypeTok{Vec}\NormalTok{ n a}
\NormalTok{generate_ s f }\FunctionTok{=}\NormalTok{ withKnownNat s }\FunctionTok{$}
    \DataTypeTok{UnsafeMkVec} \FunctionTok{$}\NormalTok{ V.generate l (f }\FunctionTok{.}\NormalTok{ fromIntegral)}
  \KeywordTok{where}
\NormalTok{    l }\FunctionTok{=}\NormalTok{ fromIntegral (fromSing s)}

\CommentTok{-- alternatively, via pattern matching:}
\CommentTok{-- source: https://github.com/mstksg/inCode/tree/master/code-samples/fixvec-2/VecWrappedSingletons.hs#L63-L66}
\OtherTok{generate'_ ::} \DataTypeTok{Sing}\NormalTok{ n }\OtherTok{->}\NormalTok{ (}\DataTypeTok{Finite}\NormalTok{ n }\OtherTok{->}\NormalTok{ a) }\OtherTok{->} \DataTypeTok{Vec}\NormalTok{ n a}
\NormalTok{generate'_ s}\FunctionTok{@}\DataTypeTok{SNat}\NormalTok{ f }\FunctionTok{=} \DataTypeTok{UnsafeMkVec} \FunctionTok{$}\NormalTok{ V.generate l (f }\FunctionTok{.}\NormalTok{ fromIntegral)}
  \KeywordTok{where}
\NormalTok{    l }\FunctionTok{=}\NormalTok{ fromIntegral (fromSing s)}

\CommentTok{-- source: https://github.com/mstksg/inCode/tree/master/code-samples/fixvec-2/VecWrappedSingletons.hs#L68-L69}
\OtherTok{generate ::} \DataTypeTok{KnownNat}\NormalTok{ n }\OtherTok{=>}\NormalTok{ (}\DataTypeTok{Finite}\NormalTok{ n }\OtherTok{->}\NormalTok{ a) }\OtherTok{->} \DataTypeTok{Vec}\NormalTok{ n a}
\NormalTok{generate }\FunctionTok{=}\NormalTok{ generate_ sing}
\end{Highlighting}
\end{Shaded}

You can see most of our original code (with pure \texttt{KnownNat}) rewritten to
work with singletons in
\href{https://github.com/mstksg/inCode/tree/master/code-samples/fixvec-2/VecWrappedSingletons.hs}{this
file}.

\hypertarget{why-singletons}{%
\subsubsection{Why Singletons?}\label{why-singletons}}

As you can see, singletons-style programming completely subsumes programming
with \texttt{TypeNats} and \texttt{KnownNat}. What we don't see here is that
singletons style integrates very well with the rest of the singletons
ecosystem\ldots{}so you might just have to take my word for it :)

What we have just witnessed is the bridge between the singletons ecosystem and
the rest of the Haskell ecosystem's usage of \texttt{GHC.TypeNats}.
\texttt{KnownNat}, because it is provided by GHC itself, is universal. However,
I recommend any new projects or libraries you write that do \emph{anything} more
than the most trivial of usages of \texttt{KnownNat} should take a look at doing
things singletons-style.

Working with just \texttt{GHC.TypeNats} and \texttt{KnownNat}, you run into
limitations very quickly unless you stick to very basic things. And, if you ever
work with any other type-level stuff, \emph{singletons} integrates very well and
very smoothly with everything else type-level you do. If you plan on doing other
type-level things besides just the most basic, you will not regret starting
singletons-style from the beginning.

\hypertarget{real-world-examples}{%
\subsection{Real-World Examples}\label{real-world-examples}}

This exact pattern is used in many real-world libraries. The canonical
fixed-length vector library implemented in this style is
\emph{\href{http://hackage.haskell.org/package/vector-sized}{vector-sized}},
which more or less re-exports the entire
\emph{\href{http://hackage.haskell.org/package/vector}{vector}} library, but
with a statically-sized interface. This is the library I use for all my my
modern sized-vector needs.

It's also used to great benefit by the
\emph{\href{http://hackage.haskell.org/package/hmatrix}{hmatrix}} library, which
I take advantage of in my
\href{https://blog.jle.im/entries/series/+practical-dependent-types-in-haskell.html}{dependently
typed neural networks} tutorial series.

It's also provided in the
\emph{\href{http://hackage.haskell.org/package/linear-1.20.7/docs/Linear-V.html}{linear}}
library, which was one of the first major libraries to adopt this style.
However, it offers an incomplete API, and requires lens --- its main purpose is
for integration with the rest of the
\emph{\href{http://hackage.haskell.org/package/linear-1.20.7/docs/Linear-V.html}{linear}}
library, which it does very well.

Anyway, if all you really wanted was performant fixed-size containers, feel free
to stop reading now (or jump to the conclusion). But if you want to explore a
bit deeper into the world of inductive dependent types \ldots{} continue on :)

\hypertarget{the-structural-way}{%
\section{The Structural Way}\label{the-structural-way}}

So, the (a?) problem with \texttt{TypeNats} from GHC is that it has no internal
structure. It's basically the same as the \texttt{Integer} or \texttt{Natural}
type --- every single value (constructor) is completely structurally unrelated
to the next.

Just like we can imagine

\begin{Shaded}
\begin{Highlighting}[]
\KeywordTok{data} \DataTypeTok{Int} \FunctionTok{=} \FunctionTok{..} \FunctionTok{-}\DecValTok{2} \FunctionTok{|} \FunctionTok{-}\DecValTok{1} \FunctionTok{|} \DecValTok{0} \FunctionTok{|} \DecValTok{1} \FunctionTok{|} \DecValTok{2} \FunctionTok{...}
\end{Highlighting}
\end{Shaded}

We can also think of \texttt{Nat} as just
\texttt{0\ \textbar{}\ 1\ \textbar{}\ 2\ \textbar{}\ 3\ \textbar{}\ 4\ ...}.
Each constructor is completely distinct.

This is useful for most practical applications. However, when we want to use our
fixed-length types in a more subtle and nuanced way, it might help to work with
a length type that is more\ldots{}structurally aware.

We've also noticed that the structure of our \texttt{Vec} and the structure of
our \texttt{Nat} have nothing in common, so we can't take advantage of any
shared structure to help us with type-safety in our
implementation\ldots{}everything we wrote was pretty much implemented using
``unsafe'' functions.

So, enough of this non-structural blasphemy. We are proper dependent type
programmers, dangit! We want structural verification! Compiler verification from
the very bottom!

For this, we'll dig into \emph{inductive} type-level nats.

\begin{Shaded}
\begin{Highlighting}[]
\KeywordTok{data} \DataTypeTok{Nat} \FunctionTok{=} \DataTypeTok{Z} \FunctionTok{|} \DataTypeTok{S} \DataTypeTok{Nat}
  \KeywordTok{deriving} \DataTypeTok{Eq}
\end{Highlighting}
\end{Shaded}

We're using the \texttt{DataKinds} extension, so not only does that define the
\emph{type} \texttt{Nat} with the \emph{values} \texttt{Z} and
\texttt{S\ ::\ Nat\ -\textgreater{}\ Nat}, it also defines the \emph{kind}
\texttt{Nat} with the \emph{types} \texttt{\textquotesingle{}Z} and
\texttt{\textquotesingle{}S\ ::\ Nat\ -\textgreater{}\ Nat}! (note the
backticks)

\begin{Shaded}
\begin{Highlighting}[]
\NormalTok{ghci}\FunctionTok{>} \FunctionTok{:}\NormalTok{t }\DataTypeTok{S} \DataTypeTok{Z}
\DataTypeTok{Nat}
\NormalTok{ghci}\FunctionTok{>} \FunctionTok{:}\NormalTok{k }\CharTok{'S '}\DataTypeTok{Z}
\DataTypeTok{Nat}
\end{Highlighting}
\end{Shaded}

So \texttt{\textquotesingle{}Z} represents 0, and \texttt{\textquotesingle{}S}
represents the ``successor'' function: one plus whatever number it contains.
\texttt{\textquotesingle{}S\ \textquotesingle{}Z} represents 1,
\texttt{\textquotesingle{}S\ (\textquotesingle{}S\ \textquotesingle{}Z)}
represents 2, etc.

And now we can define a fixed-length \emph{list}, which is basically a normal
haskell list ``zipped'' with \texttt{S}s.

\begin{Shaded}
\begin{Highlighting}[]
\CommentTok{-- source: https://github.com/mstksg/inCode/tree/master/code-samples/fixvec-2/VecInductive.hs#L26-L30}
\KeywordTok{data} \DataTypeTok{Vec}\OtherTok{ ::} \DataTypeTok{Nat} \OtherTok{->} \DataTypeTok{Type} \OtherTok{->} \DataTypeTok{Type} \KeywordTok{where}
    \DataTypeTok{VNil}\OtherTok{ ::} \DataTypeTok{Vec} \CharTok{'Z a}
\OtherTok{    (:+) ::}\NormalTok{ a }\OtherTok{->} \DataTypeTok{Vec}\NormalTok{ n a }\OtherTok{->} \DataTypeTok{Vec}\NormalTok{ (}\CharTok{'S n) a}

\KeywordTok{infixr} \DecValTok{5} \FunctionTok{:+}
\end{Highlighting}
\end{Shaded}

Here, we're using \texttt{GADT} syntax to define our type using its
constructors: the \texttt{VNil} constructor (which creates a
\texttt{Vec\ \textquotesingle{}Z\ a}, or the empty vector, like \texttt{{[}{]}})
and the \texttt{(:+)} constructor (like cons, or \texttt{(:)}), which conses an
item to a \texttt{Vec\ n\ a} to get a \texttt{Vec\ (\textquotesingle{}S\ n)\ a},
or a vector with one more element.

Basically, all usage of nil and cons (\texttt{VNil} and \texttt{:+}) keeps track
of the current ``length'' of the vectors in its type. Observe that the only way
to construct a
\texttt{Vec\ (\textquotesingle{}S\ (\textquotesingle{}S\ \textquotesingle{}Z))\ a}
is by using two \texttt{:+}s and a \texttt{VNil}!

\begin{Shaded}
\begin{Highlighting}[]
\NormalTok{ghci}\FunctionTok{>} \FunctionTok{:}\NormalTok{t }\DataTypeTok{VNil}
\DataTypeTok{Vec} \CharTok{'Z a}
\NormalTok{ghci}\FunctionTok{>} \FunctionTok{:}\NormalTok{t }\DataTypeTok{True} \FunctionTok{:+} \DataTypeTok{VNil}
\DataTypeTok{Vec}\NormalTok{ (}\CharTok{'S '}\DataTypeTok{Z}\NormalTok{) }\DataTypeTok{Bool}
\NormalTok{ghci}\FunctionTok{>} \FunctionTok{:}\NormalTok{t }\DataTypeTok{False} \FunctionTok{:+} \DataTypeTok{True} \FunctionTok{:+} \DataTypeTok{VNil}
\DataTypeTok{Vec}\NormalTok{ (}\CharTok{'S ('}\DataTypeTok{S} \CharTok{'Z)) Bool}
\end{Highlighting}
\end{Shaded}

\hypertarget{type-level-guarantees-are-structurally-free}{%
\subsection{Type-level Guarantees are Structurally
Free}\label{type-level-guarantees-are-structurally-free}}

One nice thing about this is that there is no ``unsafe'' way to construct a
\texttt{Vec}. Any \texttt{Vec} is \emph{inherently of the correct size}. The
very act of constructing it enforces its length.

Remember our ``unsafe'' \texttt{mapVec}? We had to implement it unsafely, and
trust that our implementation is correct. Even worse --- our \emph{users} have
to trust that our implementation is correct!

But writing such a \texttt{mapVec} function using \texttt{Vec} is guaranteed to
preserve the lengths:

\begin{Shaded}
\begin{Highlighting}[]
\CommentTok{-- source: https://github.com/mstksg/inCode/tree/master/code-samples/fixvec-2/VecInductive.hs#L32-L35}
\OtherTok{mapVec ::}\NormalTok{ (a }\OtherTok{->}\NormalTok{ b) }\OtherTok{->} \DataTypeTok{Vec}\NormalTok{ n a }\OtherTok{->} \DataTypeTok{Vec}\NormalTok{ n b}
\NormalTok{mapVec f }\FunctionTok{=}\NormalTok{ \textbackslash{}}\KeywordTok{case}
    \DataTypeTok{VNil}    \OtherTok{->} \DataTypeTok{VNil}
\NormalTok{    x }\FunctionTok{:+}\NormalTok{ xs }\OtherTok{->}\NormalTok{ f x }\FunctionTok{:+}\NormalTok{ mapVec f xs}

\CommentTok{-- compare to}
\NormalTok{map}\OtherTok{ ::}\NormalTok{ (a }\OtherTok{->}\NormalTok{ b) }\OtherTok{->}\NormalTok{ [a] }\OtherTok{->}\NormalTok{ [b]}
\NormalTok{map f }\FunctionTok{=}\NormalTok{ \textbackslash{}}\KeywordTok{case}
\NormalTok{    [] }\OtherTok{->}\NormalTok{ []}
\NormalTok{    x}\FunctionTok{:}\NormalTok{xs }\OtherTok{->}\NormalTok{ f x }\FunctionTok{:}\NormalTok{ map f xs}
\end{Highlighting}
\end{Shaded}

Our implementation is guaranteed to have the correct length. Neat! We get all of
the documentation benefits described in our previous discussion of
\texttt{mapVec}, plus more.

We can write \texttt{zip} too:

\begin{Shaded}
\begin{Highlighting}[]
\CommentTok{-- source: https://github.com/mstksg/inCode/tree/master/code-samples/fixvec-2/VecInductive.hs#L37-L42}
\OtherTok{zipVec ::} \DataTypeTok{Vec}\NormalTok{ n a }\OtherTok{->} \DataTypeTok{Vec}\NormalTok{ n b }\OtherTok{->} \DataTypeTok{Vec}\NormalTok{ n (a, b)}
\NormalTok{zipVec }\FunctionTok{=}\NormalTok{ \textbackslash{}}\KeywordTok{case}
    \DataTypeTok{VNil} \OtherTok{->}\NormalTok{ \textbackslash{}}\KeywordTok{case}
      \DataTypeTok{VNil} \OtherTok{->} \DataTypeTok{VNil}
\NormalTok{    x }\FunctionTok{:+}\NormalTok{ xs }\OtherTok{->}\NormalTok{ \textbackslash{}}\KeywordTok{case}
\NormalTok{      y }\FunctionTok{:+}\NormalTok{ ys }\OtherTok{->}\NormalTok{ (x,y) }\FunctionTok{:+}\NormalTok{ zipVec xs ys}
\end{Highlighting}
\end{Shaded}

Isn't it neat how the code reads exactly like the code for map/zip for
\emph{lists}? Because their structure is identical, their only real difference
is the type-level tag. All of the functions we write are the same.

\hypertarget{type-level-arithmentic}{%
\subsubsection{Type-Level Arithmentic}\label{type-level-arithmentic}}

GHC provided our \texttt{+} before, so we have to write it ourselves if we want
to be able to use it for our \texttt{Nat}s. We can write it as a type family:

\begin{Shaded}
\begin{Highlighting}[]
\CommentTok{-- source: https://github.com/mstksg/inCode/tree/master/code-samples/fixvec-2/VecInductive.hs#L54-L61}
\KeywordTok{type}\NormalTok{ family (}\OtherTok{n ::} \DataTypeTok{Nat}\NormalTok{) }\FunctionTok{+}\NormalTok{ (}\OtherTok{m ::} \DataTypeTok{Nat}\NormalTok{)}\OtherTok{ ::} \DataTypeTok{Nat} \KeywordTok{where}
    \CharTok{'Z   + m = m}
    \CharTok{'S n + m = '}\DataTypeTok{S}\NormalTok{ (n }\FunctionTok{+}\NormalTok{ m)}

\OtherTok{(++) ::} \DataTypeTok{Vec}\NormalTok{ n a }\OtherTok{->} \DataTypeTok{Vec}\NormalTok{ m a }\OtherTok{->} \DataTypeTok{Vec}\NormalTok{ (n }\FunctionTok{+}\NormalTok{ m) a}
\NormalTok{(}\FunctionTok{++}\NormalTok{) }\FunctionTok{=}\NormalTok{ \textbackslash{}}\KeywordTok{case}
    \DataTypeTok{VNil}    \OtherTok{->}\NormalTok{ \textbackslash{}ys }\OtherTok{->}\NormalTok{ ys}
\NormalTok{    x }\FunctionTok{:+}\NormalTok{ xs }\OtherTok{->}\NormalTok{ \textbackslash{}ys }\OtherTok{->}\NormalTok{ x }\FunctionTok{:+}\NormalTok{ (xs }\FunctionTok{++}\NormalTok{ ys)}
\end{Highlighting}
\end{Shaded}

This works! However, we have to be careful that GHC can verify that the final
vector \emph{really does} have the length \texttt{n\ +\ m}. GHC can do this
automatically only in very simple situations. In our situation, it is possible
because \texttt{+} and \texttt{++} have the \emph{exact same structure}.

Take a moment to stare at the definition of \texttt{+} and \texttt{++} very
closely, and then squint really hard. You can see that \texttt{+} and
\texttt{++} really describe the ``same function'', using the exact same
structure. First, if the first item is a Z-y thing, return the second item
as-is. If the first item is a consy thing, return the second item consed with
the rest of the first item. Roughly speaking, of course.

This is a part of what we mean when we say that we can take advantage of the
\emph{structure} of the length type. Here, the structure of \texttt{Nat} aligns
so well with the structure of \texttt{Vec} what we can prove structural
properties about \texttt{Nat} and the \texttt{Vec} together by exploiting their
shared inductive structure.

Unfortunately, for examples where the function we write doesn't exactly match
the structure as the type family we write, this won't work. And sometimes, the
structural properties might get in the way of what we are trying to
prove/produce. An example here would be a \texttt{snoc} function (cons to the
end of a list). If you try writing it, you'll see that the structure of
\texttt{Nat} and \texttt{Vec} fight back against you pretty hard. So, exploiting
structure isn't universally useful, but it definitely helps in many situations!
Handling tricky cases like this is a subject for a whole other blog post.

\hypertarget{indexing-1}{%
\subsection{Indexing}\label{indexing-1}}

To index our previous type, we used some abstract \texttt{Finite} type, where
\texttt{Finite\ n} conveniently represented the type of all possible indices to
a \texttt{Vec\ n\ a}. We can do something similar, inductively, as well:

\begin{Shaded}
\begin{Highlighting}[]
\CommentTok{-- source: https://github.com/mstksg/inCode/tree/master/code-samples/fixvec-2/VecInductive.hs#L63-L67}
\KeywordTok{data} \DataTypeTok{Fin}\OtherTok{ ::} \DataTypeTok{Nat} \OtherTok{->} \DataTypeTok{Type} \KeywordTok{where}
    \DataTypeTok{FZ}\OtherTok{ ::} \DataTypeTok{Fin}\NormalTok{ (}\CharTok{'S n)}
    \DataTypeTok{FS}\OtherTok{ ::} \DataTypeTok{Fin}\NormalTok{ n }\OtherTok{->} \DataTypeTok{Fin}\NormalTok{ (}\CharTok{'S n)}

\KeywordTok{deriving} \KeywordTok{instance} \DataTypeTok{Show}\NormalTok{ (}\DataTypeTok{Fin}\NormalTok{ n)}
\end{Highlighting}
\end{Shaded}

I always thought of this inductive definition of \texttt{Fin} as a cute trick,
because I don't think there was any way I could have thought of it on my own.
But if you play around it enough, you might be able to convince yourself that
there are exactly \texttt{n} inhabitants of \texttt{Fin\ n}.

For example, for \texttt{Fin\ (\textquotesingle{}S\ \textquotesingle{}Z)}
(indices for a one-item vector), there should be only one inhabitant. And there
is! It's \texttt{FZ}. \texttt{FS\ FZ} is not a valid inhabitant, because it has
type \texttt{Fin\ (\textquotesingle{}S\ (\textquotesingle{}S\ m))} for some
\texttt{m}, so cannot possibly have the type
\texttt{Fin\ (\textquotesingle{}S\ \textquotesingle{}Z)}.

Let's see the inhabitants of
\texttt{Fin\ (\textquotesingle{}S\ (\textquotesingle{}S\ (\textquotesingle{}S\ \textquotesingle{}Z)))}
(indices for three-item vectors):

\begin{Shaded}
\begin{Highlighting}[]
\NormalTok{ghci}\FunctionTok{>} \DataTypeTok{FZ}\OtherTok{              ::} \DataTypeTok{Fin}\NormalTok{ (}\CharTok{'S ('}\DataTypeTok{S}\NormalTok{ (}\CharTok{'S '}\DataTypeTok{Z}\NormalTok{)))}
\DataTypeTok{FZ}
\NormalTok{ghci}\FunctionTok{>} \DataTypeTok{FS} \DataTypeTok{FZ}\OtherTok{           ::} \DataTypeTok{Fin}\NormalTok{ (}\CharTok{'S ('}\DataTypeTok{S}\NormalTok{ (}\CharTok{'S '}\DataTypeTok{Z}\NormalTok{)))}
\DataTypeTok{FS} \DataTypeTok{FZ}
\NormalTok{ghci}\FunctionTok{>} \DataTypeTok{FS}\NormalTok{ (}\DataTypeTok{FS} \DataTypeTok{FZ}\NormalTok{)}\OtherTok{      ::} \DataTypeTok{Fin}\NormalTok{ (}\CharTok{'S ('}\DataTypeTok{S}\NormalTok{ (}\CharTok{'S '}\DataTypeTok{Z}\NormalTok{)))}
\DataTypeTok{FS}\NormalTok{ (}\DataTypeTok{FS} \DataTypeTok{FZ}\NormalTok{)}
\NormalTok{ghci}\FunctionTok{>} \DataTypeTok{FS}\NormalTok{ (}\DataTypeTok{FS}\NormalTok{ (}\DataTypeTok{FS} \DataTypeTok{FZ}\NormalTok{))}\OtherTok{ ::} \DataTypeTok{Fin}\NormalTok{ (}\CharTok{'S ('}\DataTypeTok{S}\NormalTok{ (}\CharTok{'S '}\DataTypeTok{Z}\NormalTok{)))}
\DataTypeTok{TYPE} \DataTypeTok{ERROR}\FunctionTok{!}  \DataTypeTok{TYPE} \DataTypeTok{ERROR}\FunctionTok{!}  \DataTypeTok{TYPE} \DataTypeTok{ERROR}\FunctionTok{!}
\end{Highlighting}
\end{Shaded}

As GHC informs us, \texttt{FS\ (FS\ (FS\ FZ))} is not an inhabitant of
\texttt{Fin\ (\textquotesingle{}S\ (\textquotesingle{}S\ (\textquotesingle{}S\ \textquotesingle{}Z)))},
which is exactly the behavior we wanted. This is because
\texttt{FS\ (FS\ (FS\ FZ))} has type
\texttt{Fin\ (\textquotesingle{}S\ (\textquotesingle{}S\ (\textquotesingle{}S\ (\textquotesingle{}S\ m))))}
for some \texttt{m}, and this can't fit
\texttt{Fin\ (\textquotesingle{}S\ (\textquotesingle{}S\ (\textquotesingle{}S\ \textquotesingle{}Z)))}.

Also, note that there are no inhabitants of \texttt{Fin\ \textquotesingle{}Z}.
There is no constructor or combination of constructors that can yield a value of
that type.

Armed with this handy \texttt{Fin} type, we can do structural type-safe
indexing:

\begin{Shaded}
\begin{Highlighting}[]
\CommentTok{-- source: https://github.com/mstksg/inCode/tree/master/code-samples/fixvec-2/VecInductive.hs#L69-L74}
\NormalTok{index}\OtherTok{ ::} \DataTypeTok{Fin}\NormalTok{ n }\OtherTok{->} \DataTypeTok{Vec}\NormalTok{ n a }\OtherTok{->}\NormalTok{ a}
\NormalTok{index }\FunctionTok{=}\NormalTok{ \textbackslash{}}\KeywordTok{case}
    \DataTypeTok{FZ} \OtherTok{->}\NormalTok{ \textbackslash{}}\KeywordTok{case}
\NormalTok{      x }\FunctionTok{:+}\NormalTok{ _ }\OtherTok{->}\NormalTok{ x}
    \DataTypeTok{FS}\NormalTok{ i }\OtherTok{->}\NormalTok{ \textbackslash{}}\KeywordTok{case}
\NormalTok{      _ }\FunctionTok{:+}\NormalTok{ xs }\OtherTok{->}\NormalTok{ index i xs}
\end{Highlighting}
\end{Shaded}

Note that our \texttt{Fin} type structurally precludes us from being able to
index into a \texttt{Vec\ \textquotesingle{}Z\ a} (an empty vector), because to
do that, we would have to pass in a \texttt{Fin\ \textquotesingle{}Z}\ldots{}but
there is no such value with that type!

\hypertarget{generating-1}{%
\subsection{Generating}\label{generating-1}}

Now, generating these requires some more thought. Naively writing a
\texttt{replicate\ ::\ a\ -\textgreater{}\ Vec\ n\ a} is not possible; ideally,
we'd want to ``pattern match'' on our length \texttt{n}, and use \texttt{VNil}
if it's \texttt{\textquotesingle{}Z}, etc.

However, we can't pattern match on types in Haskell, because types are
\emph{erased} at runtime. They're just used by the compiler to verify your code,
but they don't exist at runtime. So, you can't just say ``do this if \texttt{n}
is \texttt{\textquotesingle{}Z}, otherwise do this''.

Recall that, in our previous vector type, we needed to use a
\texttt{KnownNat\ n} constraint to be able to \emph{reflect} a \texttt{n} type
down to the value level. We can do something similar using the \emph{singletons}
machinery!

First, we need to get singletons for our \texttt{Nat}:

\begin{Shaded}
\begin{Highlighting}[]
\CommentTok{-- source: https://github.com/mstksg/inCode/tree/master/code-samples/fixvec-2/VecInductive.hs#L21-L24}
\FunctionTok{$}\NormalTok{(singletons [d|}
\NormalTok{  data Nat = Z | S Nat}
\NormalTok{    deriving Eq}
\NormalTok{  |])}

\CommentTok{-- this creates:}
\KeywordTok{data} \KeywordTok{instance} \DataTypeTok{Sing}\OtherTok{ ::} \DataTypeTok{Nat} \OtherTok{->} \DataTypeTok{Type} \KeywordTok{where}
    \DataTypeTok{SZ}\OtherTok{ ::} \DataTypeTok{Sing} \CharTok{'Z}
    \DataTypeTok{SS}\OtherTok{ ::} \DataTypeTok{Sing}\NormalTok{ n }\OtherTok{->} \DataTypeTok{Sing}\NormalTok{ (}\CharTok{'S n)}
\end{Highlighting}
\end{Shaded}

\texttt{Sing\ n} is a singleton for our \texttt{Nat}, in that there is only one
\texttt{Sing\ n} for every \texttt{n}. So, if we receive a value of type
\texttt{Sing\ n}, we can pattern match on it to figure out what \texttt{n} is.
Essentially, we can \emph{pattern match} on \texttt{n}.

\begin{Shaded}
\begin{Highlighting}[]
\CommentTok{-- source: https://github.com/mstksg/inCode/tree/master/code-samples/fixvec-2/VecInductive.hs#L76-L83}
\OtherTok{singSize ::} \DataTypeTok{Sing}\NormalTok{ (}\OtherTok{n ::} \DataTypeTok{Nat}\NormalTok{) }\OtherTok{->} \DataTypeTok{String}
\NormalTok{singSize }\FunctionTok{=}\NormalTok{ \textbackslash{}}\KeywordTok{case}
    \CommentTok{-- here, n is 'Z}
    \DataTypeTok{SZ}        \OtherTok{->} \StringTok{"Size of zero!"}
    \CommentTok{-- here, n is ('S 'Z)}
    \DataTypeTok{SS} \DataTypeTok{SZ}     \OtherTok{->} \StringTok{"Size of one!"}
    \CommentTok{-- here, n is ('S ('S n))}
    \DataTypeTok{SS}\NormalTok{ (}\DataTypeTok{SS}\NormalTok{ _) }\OtherTok{->} \StringTok{"Wow, so big!"}
\end{Highlighting}
\end{Shaded}

We can now branch depending on what \texttt{n} is!

Basically, \emph{we can use a singleton} if we ever want to ``pattern match'' or
branch our program's output based on the type. This is a general rule you will
observe as we continue on this article.

Note that because of the inductive nature of our original \texttt{Nat} type, the
singletons are also inductive, as well. This is handy, because then our whole
ecosystem remains inductive.

Now, to write \texttt{replicate}:

\begin{Shaded}
\begin{Highlighting}[]
\CommentTok{-- source: https://github.com/mstksg/inCode/tree/master/code-samples/fixvec-2/VecInductive.hs#L90-L93}
\OtherTok{replicate_ ::} \DataTypeTok{Sing}\NormalTok{ n }\OtherTok{->}\NormalTok{ a }\OtherTok{->} \DataTypeTok{Vec}\NormalTok{ n a}
\NormalTok{replicate_ }\FunctionTok{=}\NormalTok{ \textbackslash{}}\KeywordTok{case}
    \DataTypeTok{SZ}   \OtherTok{->}\NormalTok{ \textbackslash{}_ }\OtherTok{->} \DataTypeTok{VNil}
    \DataTypeTok{SS}\NormalTok{ l }\OtherTok{->}\NormalTok{ \textbackslash{}x }\OtherTok{->}\NormalTok{ x }\FunctionTok{:+}\NormalTok{ replicate_ l x}
\end{Highlighting}
\end{Shaded}

And we can recover our original ``implicit'' style, with type-inference-driven
lengths, using \texttt{SingI} and
\texttt{sing\ ::\ SingI\ n\ =\textgreater{}\ Sing\ n}:

\begin{Shaded}
\begin{Highlighting}[]
\CommentTok{-- source: https://github.com/mstksg/inCode/tree/master/code-samples/fixvec-2/VecInductive.hs#L95-L96}
\NormalTok{replicate}\OtherTok{ ::} \DataTypeTok{SingI}\NormalTok{ n }\OtherTok{=>}\NormalTok{ a }\OtherTok{->} \DataTypeTok{Vec}\NormalTok{ n a}
\NormalTok{replicate }\FunctionTok{=}\NormalTok{ replicate_ sing}
\end{Highlighting}
\end{Shaded}

You can think of \texttt{SingI} as the ``generic singletons'' equivalent of
\texttt{KnownNat}. \texttt{KnownNat} lets us reflect out a
\texttt{GHC.TypeNats.Nat} to a \texttt{Sing}\ldots{}\texttt{SingI} lets us
reflect any type that has singletons defined to its corresponding \texttt{Sing}.
Since our new \texttt{Nat} type has singletons, we basically get a free
``\texttt{KnownNat} equivalent''!

See how useful the whole singletons ecosystem is? :)

\hypertarget{generating-with-indices-1}{%
\subsubsection{Generating with indices}\label{generating-with-indices-1}}

Writing \texttt{generate} using the inductive \texttt{Fin} and \texttt{Nat} is
an interesting challenge. It's actually a fairly standard pattern that comes up
when working with inductive types like these. I'm going to leave it as an
exercise to the reader -- click the link at the top corner of the text box to
see the solution, and see how it compares to your own :)

\begin{Shaded}
\begin{Highlighting}[]
\CommentTok{-- source: https://github.com/mstksg/inCode/tree/master/code-samples/fixvec-2/VecInductive.hs#L98-L104}
\OtherTok{generate_ ::} \DataTypeTok{Sing}\NormalTok{ n }\OtherTok{->}\NormalTok{ (}\DataTypeTok{Fin}\NormalTok{ n }\OtherTok{->}\NormalTok{ a) }\OtherTok{->} \DataTypeTok{Vec}\NormalTok{ n a}

\OtherTok{generate ::} \DataTypeTok{SingI}\NormalTok{ n }\OtherTok{=>}\NormalTok{ (}\DataTypeTok{Fin}\NormalTok{ n }\OtherTok{->}\NormalTok{ a) }\OtherTok{->} \DataTypeTok{Vec}\NormalTok{ n a}
\NormalTok{generate }\FunctionTok{=}\NormalTok{ generate_ sing}
\end{Highlighting}
\end{Shaded}

The one thing I will point out is that it is very useful that GHC verifies our
code for us, and that we have typed holes to help us develop our code. If we
ever don't know something, we can just use a typed hole \texttt{\_}, and GHC
will tell us what type it expects, and what values in scope have that type. It
is infinitely useful for situations like this, especially when you are new to
this sort of dependently typed inductive programming!

If you ever get stuck, try throwing in a \texttt{\_} and seeing what types GHC
expects\ldots{}these clues will help you get your bearings!

\hypertarget{between-sized-and-unsized}{%
\subsection{Between Sized and Unsized}\label{between-sized-and-unsized}}

Converting from sized to unsized vectors (to lists) is something that is pretty
straightforward, and can be done by just pattern matching on the vector and
recursing on the tail. I've
\href{https://github.com/mstksg/inCode/tree/master/code-samples/fixvec-2/VecInductive.hs\#L85-L88}{left
it as an excercise} to write \texttt{Vec\ n\ a\ -\textgreater{}\ {[}a{]}}.

More interesting is the other way around; our the API of converting unsized to
sized vectors will be the same:

\begin{Shaded}
\begin{Highlighting}[]
\CommentTok{-- source: https://github.com/mstksg/inCode/tree/master/code-samples/fixvec-2/VecInductive.hs#L106-L106}
\OtherTok{withVec ::}\NormalTok{ [a] }\OtherTok{->}\NormalTok{ (forall n}\FunctionTok{.} \DataTypeTok{Sing}\NormalTok{ n }\OtherTok{->} \DataTypeTok{Vec}\NormalTok{ n a }\OtherTok{->}\NormalTok{ r) }\OtherTok{->}\NormalTok{ r}
\end{Highlighting}
\end{Shaded}

But implementing it inductively is also an interesting challenge. See my tip
above about typed holes (\texttt{\_}). I recommend taking a break here to try to
solve it yourself.

Ready?

Welcome back! Hope you had a fun time :) Here's the solution!

\begin{Shaded}
\begin{Highlighting}[]
\CommentTok{-- source: https://github.com/mstksg/inCode/tree/master/code-samples/fixvec-2/VecInductive.hs#L106-L110}
\OtherTok{withVec ::}\NormalTok{ [a] }\OtherTok{->}\NormalTok{ (forall n}\FunctionTok{.} \DataTypeTok{Sing}\NormalTok{ n }\OtherTok{->} \DataTypeTok{Vec}\NormalTok{ n a }\OtherTok{->}\NormalTok{ r) }\OtherTok{->}\NormalTok{ r}
\NormalTok{withVec }\FunctionTok{=}\NormalTok{ \textbackslash{}}\KeywordTok{case}
\NormalTok{    []   }\OtherTok{->}\NormalTok{ \textbackslash{}f }\OtherTok{->}\NormalTok{ f }\DataTypeTok{SZ} \DataTypeTok{VNil}
\NormalTok{    x}\FunctionTok{:}\NormalTok{xs }\OtherTok{->}\NormalTok{ \textbackslash{}f }\OtherTok{->}\NormalTok{ withVec xs }\FunctionTok{$}\NormalTok{ \textbackslash{}l ys }\OtherTok{->}
\NormalTok{        f (}\DataTypeTok{SS}\NormalTok{ l) (x }\FunctionTok{:+}\NormalTok{ ys)}
\end{Highlighting}
\end{Shaded}

To handle the empty list, we just return immediately, giving \texttt{f} the
proper singleton and vector (\texttt{SZ} and \texttt{VNil}). For the non-empty
list, first we convert the tail \texttt{xs} into a vector (\texttt{ys}) and its
corresponding length-singleton (\texttt{l}), and then we give \texttt{f} the
``correct'' length singleton of our complete vector (\texttt{SS\ l}) and the
correct complete vector (\texttt{x\ :+\ ys})

One nice property where (in contrast with our previous non-structural
\texttt{withVec}) is that GHC ensures that the length of the vector we give to
\texttt{f} is actually what we claim it is.

\hypertarget{verifying-properties-1}{%
\subsection{Verifying properties}\label{verifying-properties-1}}

We can create some corresponding example of \texttt{exactLength} using the exact
same process we did before

First, it'd be nice to get a witness for the length of a given vector just from
the vector itself:

\begin{Shaded}
\begin{Highlighting}[]
\CommentTok{-- source: https://github.com/mstksg/inCode/tree/master/code-samples/fixvec-2/VecInductive.hs#L112-L115}
\OtherTok{vecLength ::} \DataTypeTok{Vec}\NormalTok{ n a }\OtherTok{->} \DataTypeTok{Sing}\NormalTok{ n}
\NormalTok{vecLength }\FunctionTok{=}\NormalTok{ \textbackslash{}}\KeywordTok{case}
    \DataTypeTok{VNil}    \OtherTok{->} \DataTypeTok{SZ}
\NormalTok{    _ }\FunctionTok{:+}\NormalTok{ xs }\OtherTok{->} \DataTypeTok{SS}\NormalTok{ (vecLength xs)}
\end{Highlighting}
\end{Shaded}

The type of \texttt{vecLength\ ::\ Vec\ n\ a\ -\textgreater{}\ Sing\ n} says
that it is possible, from the structure of the vector given alone, to get a
witness to its length. And, because the structure of the vector and the
structure of the length type are so similar, this is possible! (Note that this
is not possible for our non-structural ``wrapped'' \texttt{Vec}, without some
unsafe operations)

Now, our code will be identical to the code for our wrapped/non-structural
vectors, using \texttt{\%\textasciitilde{}} and \texttt{Decision} and
\texttt{Refl}:

\begin{Shaded}
\begin{Highlighting}[]
\CommentTok{-- source: https://github.com/mstksg/inCode/tree/master/code-samples/fixvec-2/VecInductive.hs#L117-L123}
\OtherTok{exactLength_ ::} \DataTypeTok{Sing}\NormalTok{ m }\OtherTok{->} \DataTypeTok{Vec}\NormalTok{ n a }\OtherTok{->} \DataTypeTok{Maybe}\NormalTok{ (}\DataTypeTok{Vec}\NormalTok{ m a)}
\NormalTok{exactLength_ sM v }\FunctionTok{=} \KeywordTok{case}\NormalTok{ sM }\FunctionTok{%~}\NormalTok{ vecLength v }\KeywordTok{of}
    \DataTypeTok{Proved} \DataTypeTok{Refl} \OtherTok{->} \DataTypeTok{Just}\NormalTok{ v}
    \DataTypeTok{Disproved}\NormalTok{ _ }\OtherTok{->} \DataTypeTok{Nothing}

\OtherTok{exactLength ::} \DataTypeTok{SingI}\NormalTok{ m }\OtherTok{=>} \DataTypeTok{Vec}\NormalTok{ n a }\OtherTok{->} \DataTypeTok{Maybe}\NormalTok{ (}\DataTypeTok{Vec}\NormalTok{ m a)}
\NormalTok{exactLength }\FunctionTok{=}\NormalTok{ exactLength_ sing}
\end{Highlighting}
\end{Shaded}

It's nice that this is exactly the same as before, and that's a testament to how
useful the singletons library is at unifying all of these distinct type-level
stuffs.

We could also write \texttt{exactLength} in a cute way by inducting on the
length we want and the vector, so it might be fun to look at this version
instead --

\begin{Shaded}
\begin{Highlighting}[]
\CommentTok{-- source: https://github.com/mstksg/inCode/tree/master/code-samples/fixvec-2/VecInductive.hs#L125-L135}
\OtherTok{exactLengthInductive_ ::} \DataTypeTok{Sing}\NormalTok{ m }\OtherTok{->} \DataTypeTok{Vec}\NormalTok{ n a }\OtherTok{->} \DataTypeTok{Maybe}\NormalTok{ (}\DataTypeTok{Vec}\NormalTok{ m a)}
\NormalTok{exactLengthInductive_ }\FunctionTok{=}\NormalTok{ \textbackslash{}}\KeywordTok{case}
    \DataTypeTok{SZ} \OtherTok{->}\NormalTok{ \textbackslash{}}\KeywordTok{case}
      \DataTypeTok{VNil}   \OtherTok{->} \DataTypeTok{Just} \DataTypeTok{VNil}
\NormalTok{      _ }\FunctionTok{:+}\NormalTok{ _ }\OtherTok{->} \DataTypeTok{Nothing}
    \DataTypeTok{SS}\NormalTok{ l }\OtherTok{->}\NormalTok{ \textbackslash{}}\KeywordTok{case}
      \DataTypeTok{VNil}    \OtherTok{->} \DataTypeTok{Nothing}
\NormalTok{      x }\FunctionTok{:+}\NormalTok{ xs }\OtherTok{->}\NormalTok{ (x }\FunctionTok{:+}\NormalTok{) }\FunctionTok{<$>}\NormalTok{ exactLengthInductive_ l xs}

\OtherTok{exactLengthInductive ::} \DataTypeTok{SingI}\NormalTok{ m }\OtherTok{=>} \DataTypeTok{Vec}\NormalTok{ n a }\OtherTok{->} \DataTypeTok{Maybe}\NormalTok{ (}\DataTypeTok{Vec}\NormalTok{ m a)}
\NormalTok{exactLengthInductive }\FunctionTok{=}\NormalTok{ exactLengthInductive_ sing}
\end{Highlighting}
\end{Shaded}

This is another way you can take advantage of the \emph{structure} of the length
type. Here, we explicitly take advantage of the inductive structure of the
\texttt{Nat} type and how it matches with the structure of the \texttt{Vec}
type, and do bold things with it!\footnote{Note, however, that if you unroll the
  definition of \texttt{\%\textasciitilde{}} for \texttt{Nat}, you pretty much
  get the exact same thing.}

But I digress. Like in the last section, checking for a given length is
literally the least interesting property you can check for. But, again, the same
process is usable here: find a way to get your witness, and then pattern match
on that witness.

For example, we can make a witness that \texttt{n} is less than or equal to
\texttt{m}, as well as a way to construct such a witness:

\begin{Shaded}
\begin{Highlighting}[]
\CommentTok{-- source: https://github.com/mstksg/inCode/tree/master/code-samples/fixvec-2/VecInductive.hs#L137-L149}
\KeywordTok{data} \DataTypeTok{LTE}\OtherTok{ ::} \DataTypeTok{Nat} \OtherTok{->} \DataTypeTok{Nat} \OtherTok{->} \DataTypeTok{Type} \KeywordTok{where}
    \DataTypeTok{LEZ}\OtherTok{ ::} \DataTypeTok{LTE} \CharTok{'Z n}
    \DataTypeTok{LES}\OtherTok{ ::} \DataTypeTok{LTE}\NormalTok{ n m }\OtherTok{->} \DataTypeTok{LTE}\NormalTok{ (}\CharTok{'S n) ('}\DataTypeTok{S}\NormalTok{ m)}

\OtherTok{isLTE ::} \DataTypeTok{Sing}\NormalTok{ n }\OtherTok{->} \DataTypeTok{Sing}\NormalTok{ m }\OtherTok{->} \DataTypeTok{Decision}\NormalTok{ (}\DataTypeTok{LTE}\NormalTok{ n m)}
\NormalTok{isLTE }\FunctionTok{=}\NormalTok{ \textbackslash{}}\KeywordTok{case}
    \DataTypeTok{SZ}   \OtherTok{->}\NormalTok{ \textbackslash{}_ }\OtherTok{->} \DataTypeTok{Proved} \DataTypeTok{LEZ}
    \DataTypeTok{SS}\NormalTok{ n }\OtherTok{->}\NormalTok{ \textbackslash{}}\KeywordTok{case}
      \DataTypeTok{SZ} \OtherTok{->} \DataTypeTok{Disproved} \FunctionTok{$}\NormalTok{ \textbackslash{}}\KeywordTok{case}       \CommentTok{-- EmptyCase}
      \DataTypeTok{SS}\NormalTok{ m }\OtherTok{->} \KeywordTok{case}\NormalTok{ isLTE n m }\KeywordTok{of}
        \DataTypeTok{Proved}\NormalTok{ l    }\OtherTok{->} \DataTypeTok{Proved} \FunctionTok{$} \DataTypeTok{LES}\NormalTok{ l}
        \DataTypeTok{Disproved}\NormalTok{ p }\OtherTok{->} \DataTypeTok{Disproved} \FunctionTok{$}\NormalTok{ \textbackslash{}}\KeywordTok{case}
          \DataTypeTok{LES}\NormalTok{ l }\OtherTok{->}\NormalTok{ p l}
\end{Highlighting}
\end{Shaded}

So, it is impossible to construct an \texttt{LTE\ n\ m} if \texttt{n} is
\emph{not} less than or equal to \texttt{m}. I dare you to try!

We can write code to check for this property in our vectors:

\begin{Shaded}
\begin{Highlighting}[]
\CommentTok{-- source: https://github.com/mstksg/inCode/tree/master/code-samples/fixvec-2/VecInductive.hs#L151-L157}
\OtherTok{atLeast_ ::} \DataTypeTok{Sing}\NormalTok{ n }\OtherTok{->} \DataTypeTok{Vec}\NormalTok{ m a }\OtherTok{->} \DataTypeTok{Maybe}\NormalTok{ (}\DataTypeTok{LTE}\NormalTok{ n m, }\DataTypeTok{Vec}\NormalTok{ m a)}
\NormalTok{atLeast_ sN v }\FunctionTok{=} \KeywordTok{case}\NormalTok{ isLTE sN (vecLength v) }\KeywordTok{of}
    \DataTypeTok{Proved}\NormalTok{ l    }\OtherTok{->} \DataTypeTok{Just}\NormalTok{ (l, v)}
    \DataTypeTok{Disproved}\NormalTok{ _ }\OtherTok{->} \DataTypeTok{Nothing}

\OtherTok{atLeast ::} \DataTypeTok{SingI}\NormalTok{ n }\OtherTok{=>} \DataTypeTok{Vec}\NormalTok{ m a }\OtherTok{->} \DataTypeTok{Maybe}\NormalTok{ (}\DataTypeTok{LTE}\NormalTok{ n m, }\DataTypeTok{Vec}\NormalTok{ m a)}
\NormalTok{atLeast }\FunctionTok{=}\NormalTok{ atLeast_ sing}
\end{Highlighting}
\end{Shaded}

\texttt{atLeast\_\ sN} will only return our vector if its length is \emph{at
least} the length of the length indicated by \texttt{sN}. Basically, we check if
our vector is ``at least'' a certain length.

We can write a function that can ``take'' an arbitrary amount from a vector,
given (via proof) that the vector has at least that many elements:

\begin{Shaded}
\begin{Highlighting}[]
\CommentTok{-- source: https://github.com/mstksg/inCode/tree/master/code-samples/fixvec-2/VecInductive.hs#L159-L163}
\OtherTok{takeVec ::} \DataTypeTok{LTE}\NormalTok{ n m }\OtherTok{->} \DataTypeTok{Vec}\NormalTok{ m a }\OtherTok{->} \DataTypeTok{Vec}\NormalTok{ n a}
\NormalTok{takeVec }\FunctionTok{=}\NormalTok{ \textbackslash{}}\KeywordTok{case}
    \DataTypeTok{LEZ}   \OtherTok{->}\NormalTok{ \textbackslash{}_ }\OtherTok{->} \DataTypeTok{VNil}
    \DataTypeTok{LES}\NormalTok{ l }\OtherTok{->}\NormalTok{ \textbackslash{}}\KeywordTok{case}
\NormalTok{      x }\FunctionTok{:+}\NormalTok{ xs }\OtherTok{->}\NormalTok{ x }\FunctionTok{:+}\NormalTok{ takeVec l xs}
\end{Highlighting}
\end{Shaded}

And, we can combine that with our \texttt{atLeast} function, to be able to take
(maybe)\footnote{Remember the whole point of this exercise --- that the
  \texttt{Maybe} is required only in the completely polymorphic case, where we
  get our lengths at runtime and don't know them at compile-time. If we
  \emph{knew} \texttt{n} and \texttt{m} at compile-time, and knew that
  \texttt{n} was less than or equal to \texttt{m}, we could construct an
  \texttt{LTE\ n\ m} and call \texttt{takeVec} directly, and not return a
  \texttt{Maybe}.} from any vector:

\begin{Shaded}
\begin{Highlighting}[]
\CommentTok{-- source: https://github.com/mstksg/inCode/tree/master/code-samples/fixvec-2/VecInductive.hs#L165-L169}
\OtherTok{takeVecMaybe_ ::} \DataTypeTok{Sing}\NormalTok{ n }\OtherTok{->} \DataTypeTok{Vec}\NormalTok{ m a }\OtherTok{->} \DataTypeTok{Maybe}\NormalTok{ (}\DataTypeTok{Vec}\NormalTok{ n a)}
\NormalTok{takeVecMaybe_ sN v }\FunctionTok{=}\NormalTok{ uncurry takeVec }\FunctionTok{<$>}\NormalTok{ atLeast_ sN v}

\OtherTok{takeVecMaybe ::} \DataTypeTok{SingI}\NormalTok{ n }\OtherTok{=>} \DataTypeTok{Vec}\NormalTok{ m a }\OtherTok{->} \DataTypeTok{Maybe}\NormalTok{ (}\DataTypeTok{Vec}\NormalTok{ n a)}
\NormalTok{takeVecMaybe v }\FunctionTok{=}\NormalTok{ uncurry takeVec }\FunctionTok{<$>}\NormalTok{ atLeast v}
\end{Highlighting}
\end{Shaded}

\hypertarget{in-the-real-world}{%
\subsection{In the Real World}\label{in-the-real-world}}

This type is more like a list than a vector, so it's in a bit of an awkward
position, utility-wise. You usually chose a list over a vector in Haskell when
you want some sort of lazy streaming, but the cases where you want to lazily
stream something \emph{and} you know exactly how many items you want to stream
are admittedly a bit rare. GHC can't handle infinite \texttt{Vec}s, so there's
that, too. For ``containers'', \emph{vector} is great, so the non-structural
\texttt{Vec} is seen a lot more.

However, if you are working with a lot of other inductive types, \texttt{Vec}
works very naturally alongside them. It makes sense, then, that a ``canonical''
package offering \texttt{Vec} is
\emph{\href{http://hackage.haskell.org/package/type-combinators}{type-combinators}},
an actively maintained library with loads of useful inductive types for
type-level programming, exporting its own \texttt{Nat} and
\texttt{Sing}-equivalents. If I am doing the sort of type-level programming that
\texttt{Vec} is useful for, chances are I already have \emph{type-combinators}
imported. This is the library that I personally suggest if you want to use this
\texttt{Vec} in the real world.

\hypertarget{wrapping-up}{%
\section{Wrapping up}\label{wrapping-up}}

There's obviously more to look at, and much more we can do with fixed-length
vectors and inductive types. And, there will definitely be more issues that come
up when you start working with these in the real world, with real applications.

If you plan on moving into learning about dependent types, I hope that guide
would be a good launching point. But if all you wanted to do was learn how to
use fixed-length vectors effectively in Haskell\ldots{}hopefully after reading
this, you have confidence to work with these things directly, and to know what
to google if anything else comes up :)

Feel free as always to leave a comment or a
\href{https://twitter.com/mstk}{tweet}, or find me the freenode
\texttt{\#haskell} channel, as \emph{jle`}. I always welcome feedback,
suggestions, or questions!

\end{document}
