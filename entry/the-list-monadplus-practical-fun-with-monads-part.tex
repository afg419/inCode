\documentclass[]{article}
\usepackage{lmodern}
\usepackage{amssymb,amsmath}
\usepackage{ifxetex,ifluatex}
\usepackage{fixltx2e} % provides \textsubscript
\ifnum 0\ifxetex 1\fi\ifluatex 1\fi=0 % if pdftex
  \usepackage[T1]{fontenc}
  \usepackage[utf8]{inputenc}
\else % if luatex or xelatex
  \ifxetex
    \usepackage{mathspec}
    \usepackage{xltxtra,xunicode}
  \else
    \usepackage{fontspec}
  \fi
  \defaultfontfeatures{Mapping=tex-text,Scale=MatchLowercase}
  \newcommand{\euro}{€}
\fi
% use upquote if available, for straight quotes in verbatim environments
\IfFileExists{upquote.sty}{\usepackage{upquote}}{}
% use microtype if available
\IfFileExists{microtype.sty}{\usepackage{microtype}}{}
\usepackage[margin=1in]{geometry}
\usepackage{color}
\usepackage{fancyvrb}
\newcommand{\VerbBar}{|}
\newcommand{\VERB}{\Verb[commandchars=\\\{\}]}
\DefineVerbatimEnvironment{Highlighting}{Verbatim}{commandchars=\\\{\}}
% Add ',fontsize=\small' for more characters per line
\newenvironment{Shaded}{}{}
\newcommand{\AlertTok}[1]{\textcolor[rgb]{1.00,0.00,0.00}{\textbf{#1}}}
\newcommand{\AnnotationTok}[1]{\textcolor[rgb]{0.38,0.63,0.69}{\textbf{\textit{#1}}}}
\newcommand{\AttributeTok}[1]{\textcolor[rgb]{0.49,0.56,0.16}{#1}}
\newcommand{\BaseNTok}[1]{\textcolor[rgb]{0.25,0.63,0.44}{#1}}
\newcommand{\BuiltInTok}[1]{#1}
\newcommand{\CharTok}[1]{\textcolor[rgb]{0.25,0.44,0.63}{#1}}
\newcommand{\CommentTok}[1]{\textcolor[rgb]{0.38,0.63,0.69}{\textit{#1}}}
\newcommand{\CommentVarTok}[1]{\textcolor[rgb]{0.38,0.63,0.69}{\textbf{\textit{#1}}}}
\newcommand{\ConstantTok}[1]{\textcolor[rgb]{0.53,0.00,0.00}{#1}}
\newcommand{\ControlFlowTok}[1]{\textcolor[rgb]{0.00,0.44,0.13}{\textbf{#1}}}
\newcommand{\DataTypeTok}[1]{\textcolor[rgb]{0.56,0.13,0.00}{#1}}
\newcommand{\DecValTok}[1]{\textcolor[rgb]{0.25,0.63,0.44}{#1}}
\newcommand{\DocumentationTok}[1]{\textcolor[rgb]{0.73,0.13,0.13}{\textit{#1}}}
\newcommand{\ErrorTok}[1]{\textcolor[rgb]{1.00,0.00,0.00}{\textbf{#1}}}
\newcommand{\ExtensionTok}[1]{#1}
\newcommand{\FloatTok}[1]{\textcolor[rgb]{0.25,0.63,0.44}{#1}}
\newcommand{\FunctionTok}[1]{\textcolor[rgb]{0.02,0.16,0.49}{#1}}
\newcommand{\ImportTok}[1]{#1}
\newcommand{\InformationTok}[1]{\textcolor[rgb]{0.38,0.63,0.69}{\textbf{\textit{#1}}}}
\newcommand{\KeywordTok}[1]{\textcolor[rgb]{0.00,0.44,0.13}{\textbf{#1}}}
\newcommand{\NormalTok}[1]{#1}
\newcommand{\OperatorTok}[1]{\textcolor[rgb]{0.40,0.40,0.40}{#1}}
\newcommand{\OtherTok}[1]{\textcolor[rgb]{0.00,0.44,0.13}{#1}}
\newcommand{\PreprocessorTok}[1]{\textcolor[rgb]{0.74,0.48,0.00}{#1}}
\newcommand{\RegionMarkerTok}[1]{#1}
\newcommand{\SpecialCharTok}[1]{\textcolor[rgb]{0.25,0.44,0.63}{#1}}
\newcommand{\SpecialStringTok}[1]{\textcolor[rgb]{0.73,0.40,0.53}{#1}}
\newcommand{\StringTok}[1]{\textcolor[rgb]{0.25,0.44,0.63}{#1}}
\newcommand{\VariableTok}[1]{\textcolor[rgb]{0.10,0.09,0.49}{#1}}
\newcommand{\VerbatimStringTok}[1]{\textcolor[rgb]{0.25,0.44,0.63}{#1}}
\newcommand{\WarningTok}[1]{\textcolor[rgb]{0.38,0.63,0.69}{\textbf{\textit{#1}}}}
\usepackage{graphicx}
\makeatletter
\def\maxwidth{\ifdim\Gin@nat@width>\linewidth\linewidth\else\Gin@nat@width\fi}
\def\maxheight{\ifdim\Gin@nat@height>\textheight\textheight\else\Gin@nat@height\fi}
\makeatother
% Scale images if necessary, so that they will not overflow the page
% margins by default, and it is still possible to overwrite the defaults
% using explicit options in \includegraphics[width, height, ...]{}
\setkeys{Gin}{width=\maxwidth,height=\maxheight,keepaspectratio}
\ifxetex
  \usepackage[setpagesize=false, % page size defined by xetex
              unicode=false, % unicode breaks when used with xetex
              xetex]{hyperref}
\else
  \usepackage[unicode=true]{hyperref}
\fi
\hypersetup{breaklinks=true,
            bookmarks=true,
            pdfauthor={Justin Le},
            pdftitle={The List MonadPlus --- Practical Fun with Monads (Part 2 of 3)},
            colorlinks=true,
            citecolor=blue,
            urlcolor=blue,
            linkcolor=magenta,
            pdfborder={0 0 0}}
\urlstyle{same}  % don't use monospace font for urls
% Make links footnotes instead of hotlinks:
\renewcommand{\href}[2]{#2\footnote{\url{#1}}}
\setlength{\parindent}{0pt}
\setlength{\parskip}{6pt plus 2pt minus 1pt}
\setlength{\emergencystretch}{3em}  % prevent overfull lines
\setcounter{secnumdepth}{0}

\title{The List MonadPlus --- Practical Fun with Monads (Part 2 of 3)}
\author{Justin Le}
\date{December 18, 2013}

\begin{document}
\maketitle

\emph{Originally posted on
\textbf{\href{https://blog.jle.im/entry/the-list-monadplus-practical-fun-with-monads-part.html}{in
Code}}.}

Part two of an exploration of a very useful design pattern in Haskell known as
MonadPlus, a part of an effort to make ``practical'' monads less of a mystery
and fun to the good peoples of this earth.

When we last left off on the
\href{http://blog.jle.im/entry/practical-fun-with-monads-introducing-monadplus}{MonadPlus
introduction}, we understood that there are times when you want to chain
functions on objects in a way that ``resembles'' a failure/success process. We
did this by exploring the most simple of all MonadPlus's: a simple ``dumb''
container for a value is either in a success or a failure. We looked at how the
MonadPlus design pattern really ``behaved''.

This time we're going to look at another MonadPlus --- the List. By the end of
this series we're going to be using nothing but the list's MonadPlus properties
to solve this classic logic problem:

\begin{quote}
A farmer has a wolf, a goat, and a cabbage that he wishes to transport across a
river. Unfortunately, his boat can carry only one thing at a time with him. He
can't leave the wolf alone with the goat, or the wolf will eat the goat. He
can't leave the goat alone with the cabbage, or the goat will eat the cabbage.
How can he properly transport his belongings to the other side one at a time,
without any disasters?
\end{quote}

Let's get to it!

\hypertarget{monadwhat-a-review}{%
\subsection{MonadWhat? A review}\label{monadwhat-a-review}}

Let's take a quick review! Remember, a monad is just an object where you have
defined a way to chain functions inside it. You'll find that you can be creative
this ``chaining'' behavior, and for any given type of object you can definitely
define more than one way to ``chain'' functions on that type of object. One
``design pattern'' of chaining is MonadPlus, where we use this chaining to model
success/failure.

\begin{itemize}
\tightlist
\item
  \texttt{mzero} means ``failure'', and chaining anything onto a failure will
  still be a failure.
\item
  \texttt{return\ x} means ``succeed with \texttt{x}'', and will return a
  ``successful'' result with a value of \texttt{x}.
\end{itemize}

You can read through the
\href{http://blog.jle.im/entry/practical-fun-with-monads-introducing-monadplus}{previous
article} for examples of seeing these principles in action and in real code.

Without further ado, let us start on the list monad.

\hypertarget{starting-on-the-list-monad}{%
\section{Starting on the List Monad}\label{starting-on-the-list-monad}}

Now, when I say ``list monad'', I mean ``one way that you can implement chaining
operations on a list''. To be more precise, I should say ``haskell's default
choice of chaining method on lists''. Technically, \textbf{there is no ``the
list monad''}\ldots{}there is ``a way we can make the List data structure a
monad''.

And what's one way we can do this? You could probably take a wild guess. Yup, we
can model lists as a MonadPlus --- we can model chaining in a way that revolves
around successes and failures.

So, how can a list model success/failure? Does that even make sense?

Let's take a look at last article's \texttt{halve} function:

\begin{Shaded}
\begin{Highlighting}[]
\CommentTok{-- the built in function `guard`, to refresh your memory}
\OtherTok{guard ::} \DataTypeTok{MonadPlus}\NormalTok{ m }\OtherTok{=>} \DataTypeTok{Bool} \OtherTok{->}\NormalTok{ m ()}
\NormalTok{guard }\DataTypeTok{True}  \FunctionTok{=}\NormalTok{ return ()}
\NormalTok{guard }\DataTypeTok{False} \FunctionTok{=}\NormalTok{ mzero}

\CommentTok{-- source: https://github.com/mstksg/inCode/tree/master/code-samples/monad-plus/Halves.hs#L30-L33}

\OtherTok{halve ::} \DataTypeTok{Int} \OtherTok{->} \DataTypeTok{Maybe} \DataTypeTok{Int}
\NormalTok{halve n }\FunctionTok{=} \KeywordTok{do}
\NormalTok{    guard }\FunctionTok{$}\NormalTok{ even n}
\NormalTok{    return }\FunctionTok{$}\NormalTok{ n }\OtherTok{`div`} \DecValTok{2}
\end{Highlighting}
\end{Shaded}

\begin{Shaded}
\begin{Highlighting}[]
\NormalTok{ghci}\FunctionTok{>}\NormalTok{ halve }\DecValTok{6}
\DataTypeTok{Just} \DecValTok{3}
\NormalTok{ghci}\FunctionTok{>}\NormalTok{ halve }\DecValTok{7}
\DataTypeTok{Nothing}
\NormalTok{ghci}\FunctionTok{>}\NormalTok{ halve }\DecValTok{8} \FunctionTok{>>=}\NormalTok{ halve}
\DataTypeTok{Just} \DecValTok{2}
\NormalTok{ghci}\FunctionTok{>}\NormalTok{ halve }\DecValTok{7} \FunctionTok{>>=}\NormalTok{ halve}
\DataTypeTok{Nothing}
\end{Highlighting}
\end{Shaded}

Here, our success/fail mechanism was built into the Maybe container. Remember,
first, it fails automatically if \texttt{n} is not even; then, it auto-succeeds
with \texttt{n\ \textasciigrave{}div\textasciigrave{}\ 2} (which only works if
it has not already failed). But note that we didn't actually really ``need''
Maybe here\ldots{}we could have used anything that had an \texttt{mzero}
(insta-fail, which is used in \texttt{guard}) and a \texttt{return}
(auto-succeed).

Let's see what happens when we replace our Maybe container with a list:

\begin{Shaded}
\begin{Highlighting}[]
\CommentTok{-- source: https://github.com/mstksg/inCode/tree/master/code-samples/monad-plus/Halves.hs#L35-L38}

\OtherTok{halve' ::} \DataTypeTok{Int} \OtherTok{->}\NormalTok{ [}\DataTypeTok{Int}\NormalTok{]}
\NormalTok{halve' n }\FunctionTok{=} \KeywordTok{do}
\NormalTok{    guard }\FunctionTok{$}\NormalTok{ even n}
\NormalTok{    return }\FunctionTok{$}\NormalTok{ n }\OtherTok{`div`} \DecValTok{2}
\end{Highlighting}
\end{Shaded}

This is\ldots{}the exact same function body. We didn't do anything but change
the type signature. But because you believe me when I say that List is a
MonadPlus\ldots{}this should work, right? \texttt{guard} should work for
\emph{any} MonadPlus, because every MonadPlus has an \texttt{mzero} (fail).
\texttt{return} should work for any MonadPlus, too --- it wouldn't be a
MonadPlus without \texttt{return} implemented! (Remember, typeclasses are
similar to interfaces in OOP) We don't know exactly what failing and succeeding
actually \emph{looks} like in a list yet\ldots{}but if you know it's a MonadPlus
(which List is, in the standard library), you know that it \emph{has} these
concepts defined somewhere.

So, how is list a meaningful MonadPlus? Simple: a ``failure'' is an empty list.
A ``success'' is a non-empty list.

Watch:

\begin{Shaded}
\begin{Highlighting}[]
\NormalTok{ghci}\FunctionTok{>}\NormalTok{ halve' }\DecValTok{6}
\NormalTok{[}\DecValTok{3}\NormalTok{]}
\NormalTok{ghci}\FunctionTok{>}\NormalTok{ halve' }\DecValTok{7}
\NormalTok{[]}
\NormalTok{ghci}\FunctionTok{>}\NormalTok{ halve' }\DecValTok{8} \FunctionTok{>>=}\NormalTok{ halve'}
\NormalTok{[}\DecValTok{2}\NormalTok{]}
\NormalTok{ghci}\FunctionTok{>}\NormalTok{ halve' }\DecValTok{7} \FunctionTok{>>=}\NormalTok{ halve'}
\NormalTok{[]}
\NormalTok{ghci}\FunctionTok{>}\NormalTok{ halve' }\DecValTok{32} \FunctionTok{>>=}\NormalTok{ halve' }\FunctionTok{>>=}\NormalTok{ halve' }\FunctionTok{>>=}\NormalTok{ halve'}
\NormalTok{[}\DecValTok{2}\NormalTok{]}
\NormalTok{ghci}\FunctionTok{>}\NormalTok{ halve' }\DecValTok{32} \FunctionTok{>>}\NormalTok{ mzero }\FunctionTok{>>=}\NormalTok{ halve' }\FunctionTok{>>=}\NormalTok{ halve' }\FunctionTok{>>=}\NormalTok{ halve'}
\NormalTok{[]}
\end{Highlighting}
\end{Shaded}

So there we have it! \texttt{Nothing} is just like \texttt{{[}{]}},
\texttt{Just\ x} is just like \texttt{{[}x{]}}. This whole time! It's all so
clear now! Why does \texttt{Maybe} even exist, anyway, when we can just use
\texttt{{[}{]}} and \texttt{{[}x{]}} for \texttt{Nothing} and \texttt{Just\ x}
and be none the wiser? (Take some time to think about it if you want!)

In fact, if we generalize our type signature for \texttt{halve}, we can do some
crazy things\ldots{}

\begin{Shaded}
\begin{Highlighting}[]
\CommentTok{-- source: https://github.com/mstksg/inCode/tree/master/code-samples/monad-plus/Halves.hs#L40-L43}

\OtherTok{genericHalve ::} \DataTypeTok{MonadPlus}\NormalTok{ m }\OtherTok{=>} \DataTypeTok{Int} \OtherTok{->}\NormalTok{ m }\DataTypeTok{Int}
\NormalTok{genericHalve n }\FunctionTok{=} \KeywordTok{do}
\NormalTok{    guard }\FunctionTok{$}\NormalTok{ even n}
\NormalTok{    return }\FunctionTok{$}\NormalTok{ n }\OtherTok{`div`} \DecValTok{2}
\end{Highlighting}
\end{Shaded}

\begin{Shaded}
\begin{Highlighting}[]
\NormalTok{ghci}\FunctionTok{>}\NormalTok{ genericHalve }\DecValTok{8}\OtherTok{ ::} \DataTypeTok{Maybe} \DataTypeTok{Int}
\DataTypeTok{Just} \DecValTok{4}
\NormalTok{ghci}\FunctionTok{>}\NormalTok{ genericHalve }\DecValTok{8}\OtherTok{ ::}\NormalTok{ [}\DataTypeTok{Int}\NormalTok{]}
\NormalTok{[}\DecValTok{4}\NormalTok{]}
\NormalTok{ghci}\FunctionTok{>}\NormalTok{ genericHalve }\DecValTok{7}\OtherTok{ ::} \DataTypeTok{Maybe} \DataTypeTok{Int}
\DataTypeTok{Nothing}
\NormalTok{ghci}\FunctionTok{>}\NormalTok{ genericHalve }\DecValTok{7}\OtherTok{ ::}\NormalTok{ [}\DataTypeTok{Int}\NormalTok{]}
\NormalTok{[]}
\end{Highlighting}
\end{Shaded}

\begin{center}\rule{0.5\linewidth}{\linethickness}\end{center}

\textbf{Welcome to Haskell!}

Now, when we say something like \texttt{genericHalve\ 8\ ::\ Maybe\ Int}, it
means ``I want \texttt{genericHalve\ 8}\ldots{}and I want the type to be
\texttt{Maybe\ Int}.'' This is necessary here because in our
\texttt{genericHalve} can be \emph{any} MonadPlus, so we have to tell ghci which
MonadPlus we want.

\begin{center}\rule{0.5\linewidth}{\linethickness}\end{center}

(\href{https://github.com/mstksg/inCode/blob/master/code-samples/monad-plus/Halves.hs}{All
three versions of \texttt{halve} available for playing around with})

So there you have it. Maybe and lists are one and the same. Lists \emph{do} too
represent the concept of failure and success. So\ldots{}what's the difference?

\hypertarget{a-list-apart}{%
\section{A List Apart}\label{a-list-apart}}

Lists can model failure the same way that Maybe can. But it should be apparent
that lists can do a little ``more'' than Maybe\ldots{}

Consider \texttt{{[}3,\ 5{]}}. Clearly this is to represent some sort of
``success'' (because a failure would be an empty list). But what kind of
``success'' could it represent?

How about we look at it this way: \texttt{{[}3,\ 5{]}} represents two separate
\emph{paths} to success. When we look at a \texttt{Just\ 5}, we see a
computation that succeeded with a 5. When we see a \texttt{{[}3,\ 5{]}}, we may
interpret it as a computation that had two possible succesful paths: one
succeeding with a 3 and another with a 5.

You can also say that it represents a computation that \emph{could have chosen}
to succeed in a 3, or a 5. In this way, the list monad is often referred to as
``the choice monad''.

This view of a list as a collection of possible successes or choices of
successes is not the only way to think of a list as a monad\ldots{}but it is the
way that the Haskell community has adopted as arguably the most useful. (The
other main way is to approach it completely differently, making list not even a
MonadPlus and therefore not representing failure or success at all)

Think of it this way: A value goes through a long and arduous journey with many
choices and possible paths and forks. At the end of it, you have the result of
every path that could have lead to a success. Contrast this to the Maybe monad,
where a value goes through this arduous journey, but never has any choice. There
is only one path --- successful, or otherwise. A Maybe is deterministic\ldots{}a
list provides a choice in paths.

\hypertarget{halveordouble}{%
\section{halveOrDouble}\label{halveordouble}}

Let's take a simple example: \texttt{halveOrDouble}. It provides two successful
paths if you are even: halving and doubling. It only provides one choice or
possible path to success if you are odd: doubling. In this way it is slightly
racist.

\begin{Shaded}
\begin{Highlighting}[]
\CommentTok{-- source: https://github.com/mstksg/inCode/tree/master/code-samples/monad-plus/HalveOrDouble.hs#L19-L21}

\OtherTok{halveOrDouble ::} \DataTypeTok{Int} \OtherTok{->}\NormalTok{ [}\DataTypeTok{Int}\NormalTok{]}
\NormalTok{halveOrDouble n }\FunctionTok{|}\NormalTok{ even n    }\FunctionTok{=}\NormalTok{ [n }\OtherTok{`div`} \DecValTok{2}\NormalTok{, n }\FunctionTok{*} \DecValTok{2}\NormalTok{]}
                \FunctionTok{|}\NormalTok{ otherwise }\FunctionTok{=}\NormalTok{ [n }\FunctionTok{*} \DecValTok{2}\NormalTok{]}
\end{Highlighting}
\end{Shaded}

\begin{Shaded}
\begin{Highlighting}[]
\NormalTok{ghci}\FunctionTok{>}\NormalTok{ halveOrDouble }\DecValTok{6}
\NormalTok{[ }\DecValTok{3}\NormalTok{,}\DecValTok{12}\NormalTok{]}
\NormalTok{ghci}\FunctionTok{>}\NormalTok{ halveOrDouble }\DecValTok{7}
\NormalTok{[   }\DecValTok{14}\NormalTok{]}
\end{Highlighting}
\end{Shaded}

(\href{https://github.com/mstksg/inCode/blob/master/code-samples/monad-plus/HalveOrDouble.hs}{Play
with this and other functions this section on your own})

As you can see in the first case, with the 6, there are two paths to success:
the halve, and the double. In the second case, with the 7, there is only one ---
the double.

How about we subject a number to this halving-or-doubling journey twice? What do
we expect?

\begin{enumerate}
\def\labelenumi{\arabic{enumi}.}
\tightlist
\item
  The path of halve-halve only works if the number is divisible by two twice. So
  this is only a successful path if the number is divisible by four.
\item
  The path of halve-double only works if the number is even. So this is only a
  successful path in that case.
\item
  The path of double-halve will work in all cases! It is a success always.
\item
  The path of double-double will also work in all cases\ldots{}it'll never fail
  for our sojourning number!
\end{enumerate}

So\ldots{}halving-or-doubling twice has two possible successful paths for an odd
number, three successful paths for a number divisible by two but not four, and
four successful paths for a number divisible by four.

Let's try it out:

\begin{Shaded}
\begin{Highlighting}[]
\NormalTok{ghci}\FunctionTok{>}\NormalTok{ halveOrDouble }\DecValTok{5} \FunctionTok{>>=}\NormalTok{ halveOrDouble}
\NormalTok{[       }\DecValTok{5}\NormalTok{, }\DecValTok{20}\NormalTok{]}
\NormalTok{ghci}\FunctionTok{>}\NormalTok{ halveOrDouble }\DecValTok{6} \FunctionTok{>>=}\NormalTok{ halveOrDouble}
\NormalTok{[    }\DecValTok{6}\NormalTok{, }\DecValTok{6}\NormalTok{, }\DecValTok{24}\NormalTok{]}
\NormalTok{ghci}\FunctionTok{>}\NormalTok{ halveOrDouble }\DecValTok{8} \FunctionTok{>>=}\NormalTok{ halveOrDouble}
\NormalTok{[ }\DecValTok{2}\NormalTok{, }\DecValTok{8}\NormalTok{, }\DecValTok{8}\NormalTok{, }\DecValTok{32}\NormalTok{]}
\end{Highlighting}
\end{Shaded}

The first list represents the results of all of the possible successful paths 5
could have taken to ``traverse'' the dreaded \texttt{halveOrDouble} landscape
twice --- double-halve, or double-double. The second, 6 could have emerged
successful with halve-double, double-halve, or double-double. For 8, all paths
are successful, incidentally. He better check his privilege.

\hypertarget{do-notation}{%
\subsection{Do notation}\label{do-notation}}

Let's look at the same thing in do notation form to offer some possible insight:

\begin{Shaded}
\begin{Highlighting}[]
\CommentTok{-- source: https://github.com/mstksg/inCode/tree/master/code-samples/monad-plus/HalveOrDouble.hs#L24-L27}

\OtherTok{halveOrDoubleTwice ::} \DataTypeTok{Int} \OtherTok{->}\NormalTok{ [}\DataTypeTok{Int}\NormalTok{]}
\NormalTok{halveOrDoubleTwice n }\FunctionTok{=} \KeywordTok{do}
\NormalTok{    x }\OtherTok{<-}\NormalTok{ halveOrDouble n}
\NormalTok{    halveOrDouble x}
\end{Highlighting}
\end{Shaded}

Do notation describes \textbf{a single path of a value}. This is slightly
confusing at first. But look at it --- it has the \emph{exact same form} as a
Maybe monad do block.

This thing describes, in general terms, the path of a \textbf{single value}.
\texttt{x} is \textbf{not} a list --- it represents a single value, in the
middle of its treacherous journey.

Here is an illustration, tracing out ``individual paths'':

\begin{Shaded}
\begin{Highlighting}[]
\OtherTok{halveOrDoubleTwice ::} \DataTypeTok{Int} \OtherTok{->}\NormalTok{ [}\DataTypeTok{Int}\NormalTok{]}
\NormalTok{halveOrDoubleTwice n }\FunctionTok{=} \KeywordTok{do}       \CommentTok{-- halveOrDoubleTwice 6}
\NormalTok{    x }\OtherTok{<-}\NormalTok{ halveOrDouble n        }\CommentTok{-- x <-     Just 3          Just 12}
\NormalTok{    halveOrDouble x             }\CommentTok{--      Nothing  Just 6  Just 6  Just 24}
\end{Highlighting}
\end{Shaded}

where you take the left path if you want to halve, and the right path if you
want to double.

Remember, just like in the Maybe monad, the \texttt{x} represents the value
``inside'' the object --- \texttt{x} represents a 3 \textbf{or} a 12 (but not
``both''), depending on what path you are taking/are ``in''. That's why we can
call \texttt{halveOrDouble\ x}: \texttt{halveOrDouble} only takes \texttt{Int}s
and \texttt{x} is \emph{one} \texttt{Int} along the path.

\hypertarget{a-winding-journey}{%
\subsection{A winding journey}\label{a-winding-journey}}

Note that once you bind a value to a variable (like \texttt{x}), then that is
the value for \texttt{x} for the entire rest of the journey. In fact, let's see
it in action:

\begin{Shaded}
\begin{Highlighting}[]
\CommentTok{-- source: https://github.com/mstksg/inCode/tree/master/code-samples/monad-plus/HalveOrDouble.hs#L29-L29}

\OtherTok{hod2PlusOne ::} \DataTypeTok{Int} \OtherTok{->}\NormalTok{ [}\DataTypeTok{Int}\NormalTok{]}
\NormalTok{hod2PlusOne n }\FunctionTok{=} \KeywordTok{do}              \CommentTok{-- hod2PlusOne 6}
\NormalTok{    x }\OtherTok{<-}\NormalTok{ halveOrDouble n        }\CommentTok{-- x <-     Just 3          Just 12}
\NormalTok{    halveOrDouble x             }\CommentTok{--      Nothing  Just 6  Just 6  Just 24}
\NormalTok{    return }\FunctionTok{$}\NormalTok{ x }\FunctionTok{+} \DecValTok{1}              \CommentTok{--      (skip)   Just 4  Just 13 Just 13}
\end{Highlighting}
\end{Shaded}

\begin{Shaded}
\begin{Highlighting}[]
\NormalTok{ghci}\FunctionTok{>}\NormalTok{ hod2PlusOne }\DecValTok{6}
\NormalTok{[   }\DecValTok{4}\NormalTok{,}\DecValTok{13}\NormalTok{,}\DecValTok{13}\NormalTok{]}
\end{Highlighting}
\end{Shaded}

Okay! This is getting interesting now. What's going on? Well, there are four
possible ``paths''.

\begin{enumerate}
\def\labelenumi{\arabic{enumi}.}
\tightlist
\item
  In the half-half path, \texttt{x} (the result of the first halving) is 3.
  However, the half-half path is a failure --- 6 cannot be halved twice.
  Therefore, even though \texttt{x} is three, the path has already failed before
  we get to the \texttt{return\ (x\ +\ 1)}. Just like in the case with Maybe,
  once something fails during the process of the journey, the entire journey is
  a failure.
\item
  In the half-double path, \texttt{x} is also 3. However, this journey doesn't
  fail. It survives to the end. After the doubling, the value of the journey at
  that point is ``Just 6''. Afterwards, it ``auto-succeeds'' and replaces the
  current value with the value of \texttt{x} on that path (3) plus 1 --- 4. This
  is just like how in the Maybe monad, we return a new value after the guard.
\item
  In the double-halve path, \texttt{x} (the result of the first operation, a
  double) is 12. The second operation makes the value in the journey a 6; At the
  end of it all, we succeed with whatever the value of \texttt{x} is on that
  specific journey (12) is, plus one. 13.
\item
  Same story here, but for double-double; \texttt{x} is 12. At the end of it
  all, the journey never fails, so it succeeds with \texttt{x\ +\ 1}, or 13.
\end{enumerate}

\hypertarget{trying-out-every-path}{%
\subsubsection{Trying out every path}\label{trying-out-every-path}}

If this doesn't satisfy you, here is an example of four Maybe do blocks where we
``flesh out'' each possible path, with the value of the block at each line in
comments:

\begin{Shaded}
\begin{Highlighting}[]
\OtherTok{double ::} \DataTypeTok{Int} \OtherTok{->} \DataTypeTok{Maybe} \DataTypeTok{Int}
\NormalTok{double n }\FunctionTok{=} \DataTypeTok{Just}\NormalTok{ n}

\OtherTok{halveHalvePlusOne ::} \DataTypeTok{Int} \OtherTok{->} \DataTypeTok{Maybe} \DataTypeTok{Int}
\NormalTok{halveHalvePlusOne n }\FunctionTok{=} \KeywordTok{do}                \CommentTok{-- n = 6}
\NormalTok{    x }\OtherTok{<-}\NormalTok{ halve n                        }\CommentTok{-- Just 3 (x = 3)}
\NormalTok{    halve x                             }\CommentTok{-- Nothing}
\NormalTok{    return }\FunctionTok{$}\NormalTok{ x }\FunctionTok{+} \DecValTok{1}                      \CommentTok{-- (skip)}

\OtherTok{halveDoublePlusOne ::} \DataTypeTok{Int} \OtherTok{->} \DataTypeTok{Maybe} \DataTypeTok{Int}
\NormalTok{halveDoublePlusOne }\FunctionTok{=} \KeywordTok{do}                 \CommentTok{-- n = 6}
\NormalTok{    x }\OtherTok{<-}\NormalTok{ halve n                        }\CommentTok{-- Just 3 (x = 3)}
\NormalTok{    double x                            }\CommentTok{-- Just 6}
\NormalTok{    return }\FunctionTok{$}\NormalTok{ x }\FunctionTok{+} \DecValTok{1}                      \CommentTok{-- Just 4}

\OtherTok{doubleHalvePlusOne ::} \DataTypeTok{Int} \OtherTok{->} \DataTypeTok{Maybe} \DataTypeTok{Int}
\NormalTok{doubleHalvePlusOne }\FunctionTok{=} \KeywordTok{do}                 \CommentTok{-- n = 6}
\NormalTok{    x }\OtherTok{<-}\NormalTok{ double n                       }\CommentTok{-- Just 12 (x = 12)}
\NormalTok{    halve x                             }\CommentTok{-- Just 6}
\NormalTok{    return }\FunctionTok{$}\NormalTok{ x }\FunctionTok{+} \DecValTok{1}                      \CommentTok{-- Just 13}

\OtherTok{doubleDoublePlusOne ::} \DataTypeTok{Int} \OtherTok{->} \DataTypeTok{Maybe} \DataTypeTok{Int}
\NormalTok{doubleDoublePlusOne }\FunctionTok{=} \KeywordTok{do}                \CommentTok{-- n = 6}
\NormalTok{    x }\OtherTok{<-}\NormalTok{ double n                       }\CommentTok{-- Just 12 (x = 12)}
\NormalTok{    double x                            }\CommentTok{-- Just 6}
\NormalTok{    return }\FunctionTok{$}\NormalTok{ x }\FunctionTok{+} \DecValTok{1}                      \CommentTok{-- Just 13}
\end{Highlighting}
\end{Shaded}

\hypertarget{a-graphical-look}{%
\subsubsection{A graphical look}\label{a-graphical-look}}

This tree might also be a nice illustration, showing what happens at each stage
of the journey.

\begin{figure}
\centering
\includegraphics{/img/entries/monad-plus/halvedouble.png}
\caption{\emph{hod2PlusOne 6}, all journeys illustrated}
\end{figure}

Every complete ``journey'' is a complete path from top to bottom. You can see
that the left-left journey (the half-halve journey) fails. The left-right
journey (the halve-double journey) passes, and at the end is given the value of
\texttt{x\ +\ 1} for the \texttt{x} in that particular journey. The other
journeys work the same way!

\hypertarget{solving-real-ish-problems}{%
\section{Solving real-ish problems}\label{solving-real-ish-problems}}

That wasn't too bad, was it? We're actually just about ready to start
implementing our solution to the Wolf/Goat/Cabbage puzzle!

Before we end this post let's build some more familiarity with the List monad
and try out a very common practical example.

\hypertarget{finding-the-right-combinations}{%
\subsection{Finding the right
combinations}\label{finding-the-right-combinations}}

Here is probably the most common of all examples involving the list monad:
finding Pythagorean triples.

\begin{Shaded}
\begin{Highlighting}[]
\CommentTok{-- source: https://github.com/mstksg/inCode/tree/master/code-samples/monad-plus/TriplesUnder.hs#L12-L18}

\OtherTok{triplesUnder ::} \DataTypeTok{Int} \OtherTok{->}\NormalTok{ [}\DataTypeTok{Int}\NormalTok{]}
\NormalTok{triplesUnder n }\FunctionTok{=} \KeywordTok{do}
\NormalTok{    a }\OtherTok{<-}\NormalTok{ [}\DecValTok{1}\FunctionTok{..}\NormalTok{n]}
\NormalTok{    b }\OtherTok{<-}\NormalTok{ [a}\FunctionTok{..}\NormalTok{n]}
\NormalTok{    c }\OtherTok{<-}\NormalTok{ [b}\FunctionTok{..}\NormalTok{n]}
\NormalTok{    guard }\FunctionTok{$}\NormalTok{ a}\FunctionTok{^}\DecValTok{2} \FunctionTok{+}\NormalTok{ b}\FunctionTok{^}\DecValTok{2} \FunctionTok{==}\NormalTok{ c}\FunctionTok{^}\DecValTok{2}
\NormalTok{    return (a,b,c)}
\end{Highlighting}
\end{Shaded}

(\href{https://github.com/mstksg/inCode/blob/master/code-samples/monad-plus/TriplesUnder.hs}{Download
it and try it out yourself!})

\begin{enumerate}
\def\labelenumi{\arabic{enumi}.}
\tightlist
\item
  Our journey begins with picking a number between 1 and \texttt{n} and setting
  it to \texttt{a}.
\item
  Next, we pick a number between \texttt{a} and \texttt{n} and set it to
  \texttt{b}. We start from \texttt{a} because if we don't, we are probably
  going to be testing the same tuple twice.
\item
  Next, we pick a number between \texttt{b} and \texttt{n}. This is our
  hypotenuse, and of course all hypontenii are larger than either side.
\item
  Now, we mercilessly and ruthlessly end all journeys who were unfortunate
  enough to pick a non-Pythagorean combination --- combinations where
  \texttt{a\^{}2\ +\ b\^{}2} is not \texttt{c\^{}2}
\item
  For those successful journeys, we succeed with a tuple containing our
  victorious triple \texttt{(a,b,c)}.
\end{enumerate}

Let's try ``following'' this path with some arbitrary choices, looking at
arbitrary journeys for \texttt{n\ =\ 10}:

\begin{itemize}
\tightlist
\item
  We pick \texttt{a} as 2, \texttt{b} as 3, and \texttt{c} as 9. All is good
  until we get to the guard. \texttt{a\^{}2\ +\ b\^{}2} is 10, which is not
  \texttt{c\^{}2} (81), unfortunately. This \texttt{(2,3,10)} journey ends here.
\item
  We pick \texttt{a} as 3, \texttt{b} as 4, and \texttt{c} as 5. On the guard,
  we succeed: \texttt{a\^{}2\ +\ b\^{}2} is 25, which indeed is \texttt{c\^{}2}.
  Our journey passes the guard, and then succeeds with a value of
  \texttt{(3,4,5)}. This is indeed counted among the successful paths --- among
  the victorious!
\end{itemize}

Paths like \texttt{a\ =\ 5} and \texttt{b\ =\ 3} do not even happen. This is
because if we pick \texttt{a\ =\ 5}, then in that particular journey, \texttt{b}
can only be chosen between \texttt{5} and \texttt{n} inclusive.

Remember, the final result is the accumulation of \textbf{all such successful
journeys}. A little bit of combinatorics will show that there are
\includegraphics{https://latex.codecogs.com/png.latex?\%5Cfrac\%7B1\%7D\%7B6\%7D\%20\%5Ctimes\%20\%5Cfrac\%7B\%28n\%2B2\%29\%21\%7D\%7B\%28n-1\%29\%21\%7D}
possible journeys to attempt. Only the ones that do not fail (at the guard) will
make it to the end. Remember how MonadPlus works --- one failure along the
journey means that the \emph{entire journey} is a failure.

Let's see what we get when we try it at the prompt:

\begin{Shaded}
\begin{Highlighting}[]
\NormalTok{ghci}\FunctionTok{>}\NormalTok{ triplesUnder }\DecValTok{10}
\NormalTok{[ ( }\DecValTok{3}\NormalTok{, }\DecValTok{4}\NormalTok{, }\DecValTok{5}\NormalTok{),( }\DecValTok{6}\NormalTok{, }\DecValTok{8}\NormalTok{,}\DecValTok{10}\NormalTok{) ]}
\NormalTok{ghci}\FunctionTok{>}\NormalTok{ triplesUnder }\DecValTok{25}
\NormalTok{[ ( }\DecValTok{3}\NormalTok{, }\DecValTok{4}\NormalTok{, }\DecValTok{5}\NormalTok{),( }\DecValTok{5}\NormalTok{,}\DecValTok{12}\NormalTok{,}\DecValTok{13}\NormalTok{),( }\DecValTok{6}\NormalTok{, }\DecValTok{8}\NormalTok{,}\DecValTok{10}\NormalTok{),( }\DecValTok{7}\NormalTok{,}\DecValTok{24}\NormalTok{,}\DecValTok{25}\NormalTok{)}
\NormalTok{ ,( }\DecValTok{8}\NormalTok{,}\DecValTok{15}\NormalTok{,}\DecValTok{17}\NormalTok{),( }\DecValTok{9}\NormalTok{,}\DecValTok{12}\NormalTok{,}\DecValTok{15}\NormalTok{),(}\DecValTok{12}\NormalTok{,}\DecValTok{16}\NormalTok{,}\DecValTok{20}\NormalTok{),(}\DecValTok{15}\NormalTok{,}\DecValTok{20}\NormalTok{,}\DecValTok{25}\NormalTok{) ]}
\end{Highlighting}
\end{Shaded}

Perfect! You can probably quickly verify that all of these solutions are indeed
Pythagorean triples. Out of the 220 journeys undertaken by
\texttt{triplesUnder\ 10}, only two of them survived to the end to be
successful. Out of the 2925 journeys in \texttt{triplesUnder\ 25}, only eight of
them made it to the end. The rest ``died''/failed, and as a result we do not
even observe their remains. It is a cruel and unforgiving world.

While the full diagram of \texttt{triplesUnder\ 5} has 35 branches, here is a
diagram for those branches with
\includegraphics{https://latex.codecogs.com/png.latex?a\%20\%3E\%202}, which has
10:

\begin{figure}
\centering
\includegraphics{/img/entries/monad-plus/triplesunder.png}
\caption{\emph{triplesUnder 5}, all journeys (where a \textgreater{} 2)
illustrated}
\end{figure}

\hypertarget{almost-there}{%
\section{Almost There!}\label{almost-there}}

Let's do a quick review:

\begin{itemize}
\tightlist
\item
  You can really treat List exactly as if it were Maybe by using the general
  MonadPlus terms \texttt{mzero} and \texttt{return}. If you do this,
  \texttt{Nothing} is equivalent to \texttt{{[}{]}}, and \texttt{Just\ x} is
  equivalent to \texttt{{[}x{]}}. Trippy!
\item
  However, whereas Maybe is a ``deterministic'' success, for a list, a list of
  successes represents the end results of \emph{possible paths} to success.
  Chaining two ``path splits'' results in the item having to traverse both
  splits one after another.
\item
  If any of these paths meet a failure at some point in their journey, the
  entire path is a failure and doesn't show up in the list of successes.
  \emph{This} is the ``MonadPlus''ness of it all.
\item
  When you use a do block (or reason about paths), it helps to think of each do
  block as representing one specific path in a Maybe monad, with arbitrary
  choices. Your \texttt{\textless{}-} binds all represent \emph{one specific
  element}, \emph{just} like for Maybe.
\end{itemize}

The last point is particularly important and is pretty pivotal in understanding
what is coming up next. Remember that all Maybe blocks and List blocks really
essentially look \emph{exactly the same}. This keeping-track-of-separate-paths
thing is all handled behind-the scenes.

In fact you should be able to look at code like:

\begin{Shaded}
\begin{Highlighting}[]
\CommentTok{-- source: https://github.com/mstksg/inCode/tree/master/code-samples/monad-plus/TriplesUnder.hs#L12-L18}

\OtherTok{triplesUnder ::} \DataTypeTok{Int} \OtherTok{->}\NormalTok{ [}\DataTypeTok{Int}\NormalTok{]}
\NormalTok{triplesUnder n }\FunctionTok{=} \KeywordTok{do}
\NormalTok{    a }\OtherTok{<-}\NormalTok{ [}\DecValTok{1}\FunctionTok{..}\NormalTok{n]}
\NormalTok{    b }\OtherTok{<-}\NormalTok{ [a}\FunctionTok{..}\NormalTok{n]}
\NormalTok{    c }\OtherTok{<-}\NormalTok{ [b}\FunctionTok{..}\NormalTok{n]}
\NormalTok{    guard }\FunctionTok{$}\NormalTok{ a}\FunctionTok{^}\DecValTok{2} \FunctionTok{+}\NormalTok{ b}\FunctionTok{^}\DecValTok{2} \FunctionTok{==}\NormalTok{ c}\FunctionTok{^}\DecValTok{2}
\NormalTok{    return (a,b,c)}
\end{Highlighting}
\end{Shaded}

and see that it is structurally identical to

\begin{Shaded}
\begin{Highlighting}[]
\OtherTok{triplesUnder' ::} \DataTypeTok{Int} \OtherTok{->} \DataTypeTok{Maybe} \DataTypeTok{Int}
\NormalTok{triplesUnder' n }\FunctionTok{=} \KeywordTok{do}
\NormalTok{    a }\OtherTok{<-} \DataTypeTok{Just} \DecValTok{3}
\NormalTok{    b }\OtherTok{<-} \DataTypeTok{Just} \DecValTok{5}
\NormalTok{    c }\OtherTok{<-} \DataTypeTok{Just} \DecValTok{8}
\NormalTok{    guard }\FunctionTok{$}\NormalTok{ a}\FunctionTok{^}\DecValTok{2} \FunctionTok{+}\NormalTok{ b}\FunctionTok{^}\DecValTok{2} \FunctionTok{==}\NormalTok{ c}\FunctionTok{^}\DecValTok{2}
\NormalTok{    return (a,b,c)}
\end{Highlighting}
\end{Shaded}

for any arbitrary choice of \texttt{a}, \texttt{b}, and \texttt{c}, except
instead of \texttt{Just\ 3} (or \texttt{{[}3{]}}), you have
\texttt{{[}2,3,4{]}}, etc.

In fact recall that this block:

\begin{Shaded}
\begin{Highlighting}[]
\CommentTok{-- source: https://github.com/mstksg/inCode/tree/master/code-samples/monad-plus/Halves.hs#L40-L43}

\OtherTok{genericHalve ::} \DataTypeTok{MonadPlus}\NormalTok{ m }\OtherTok{=>} \DataTypeTok{Int} \OtherTok{->}\NormalTok{ m }\DataTypeTok{Int}
\NormalTok{genericHalve n }\FunctionTok{=} \KeywordTok{do}
\NormalTok{    guard }\FunctionTok{$}\NormalTok{ even n}
\NormalTok{    return }\FunctionTok{$}\NormalTok{ n }\OtherTok{`div`} \DecValTok{2}
\end{Highlighting}
\end{Shaded}

is general enough that it works for both.

Hopefully this all serves to show that \textbf{in do blocks, Lists and Maybes
are structurally identical}. You reason with them the exact same way you do with
Maybe's. In something like \texttt{x\ \textless{}-\ Just\ 5}, \texttt{x}
represents a \textbf{single value}, the 5. In something like
\texttt{x\ \textless{}-\ {[}1,2,3{]}}, \texttt{x} \emph{also} represents a
single value --- the 1, the 2, or the 3, depending on which path you are
currently on. Then later in the block, you can refer to \texttt{x}, and
\texttt{x} refers to \emph{that} one specific \texttt{x} for that path.

\hypertarget{until-next-time}{%
\subsection{Until next time}\label{until-next-time}}

So I feel like we are at all we need to know to really use the list monad to
solve a large class of logic problems (because who needs Prolog, anyway?).

Between now and next time, think about how you would approach a logic problem
like the Wolf/Goat/Cabbage problem with the concepts of MonadPlus? What would
\texttt{mzero}/fail be useful for? What would the idea of a success be useful
for, and what would the idea of ``multiple paths to success'' in a journey even
mean? What is the journey?

Until next!

Hi, thanks for reading! You can reach me via email at
\href{mailto:justin@jle.im}{\nolinkurl{justin@jle.im}}, or at twitter at
\href{https://twitter.com/mstk}{@mstk}! This post and all others are published
under the \href{https://creativecommons.org/licenses/by-nc-nd/3.0/}{CC-BY-NC-ND
3.0} license. Corrections and edits via pull request are welcome and encouraged
at \href{https://github.com/mstksg/inCode}{the source repository}.

If you would like to donate, I am currently accepting bitcoin donations at
\emph{\href{bitcoin:3D7rmAYgbDnp4gp4rf22THsGt74fNucPDU}{3D7rmAYgbDnp4gp4rf22THsGt74fNucPDU}}!

\end{document}
