\documentclass[]{article}
\usepackage{lmodern}
\usepackage{amssymb,amsmath}
\usepackage{ifxetex,ifluatex}
\usepackage{fixltx2e} % provides \textsubscript
\ifnum 0\ifxetex 1\fi\ifluatex 1\fi=0 % if pdftex
  \usepackage[T1]{fontenc}
  \usepackage[utf8]{inputenc}
\else % if luatex or xelatex
  \ifxetex
    \usepackage{mathspec}
    \usepackage{xltxtra,xunicode}
  \else
    \usepackage{fontspec}
  \fi
  \defaultfontfeatures{Mapping=tex-text,Scale=MatchLowercase}
  \newcommand{\euro}{€}
\fi
% use upquote if available, for straight quotes in verbatim environments
\IfFileExists{upquote.sty}{\usepackage{upquote}}{}
% use microtype if available
\IfFileExists{microtype.sty}{\usepackage{microtype}}{}
\usepackage[margin=1in]{geometry}
\usepackage{color}
\usepackage{fancyvrb}
\newcommand{\VerbBar}{|}
\newcommand{\VERB}{\Verb[commandchars=\\\{\}]}
\DefineVerbatimEnvironment{Highlighting}{Verbatim}{commandchars=\\\{\}}
% Add ',fontsize=\small' for more characters per line
\newenvironment{Shaded}{}{}
\newcommand{\AlertTok}[1]{\textcolor[rgb]{1.00,0.00,0.00}{\textbf{#1}}}
\newcommand{\AnnotationTok}[1]{\textcolor[rgb]{0.38,0.63,0.69}{\textbf{\textit{#1}}}}
\newcommand{\AttributeTok}[1]{\textcolor[rgb]{0.49,0.56,0.16}{#1}}
\newcommand{\BaseNTok}[1]{\textcolor[rgb]{0.25,0.63,0.44}{#1}}
\newcommand{\BuiltInTok}[1]{#1}
\newcommand{\CharTok}[1]{\textcolor[rgb]{0.25,0.44,0.63}{#1}}
\newcommand{\CommentTok}[1]{\textcolor[rgb]{0.38,0.63,0.69}{\textit{#1}}}
\newcommand{\CommentVarTok}[1]{\textcolor[rgb]{0.38,0.63,0.69}{\textbf{\textit{#1}}}}
\newcommand{\ConstantTok}[1]{\textcolor[rgb]{0.53,0.00,0.00}{#1}}
\newcommand{\ControlFlowTok}[1]{\textcolor[rgb]{0.00,0.44,0.13}{\textbf{#1}}}
\newcommand{\DataTypeTok}[1]{\textcolor[rgb]{0.56,0.13,0.00}{#1}}
\newcommand{\DecValTok}[1]{\textcolor[rgb]{0.25,0.63,0.44}{#1}}
\newcommand{\DocumentationTok}[1]{\textcolor[rgb]{0.73,0.13,0.13}{\textit{#1}}}
\newcommand{\ErrorTok}[1]{\textcolor[rgb]{1.00,0.00,0.00}{\textbf{#1}}}
\newcommand{\ExtensionTok}[1]{#1}
\newcommand{\FloatTok}[1]{\textcolor[rgb]{0.25,0.63,0.44}{#1}}
\newcommand{\FunctionTok}[1]{\textcolor[rgb]{0.02,0.16,0.49}{#1}}
\newcommand{\ImportTok}[1]{#1}
\newcommand{\InformationTok}[1]{\textcolor[rgb]{0.38,0.63,0.69}{\textbf{\textit{#1}}}}
\newcommand{\KeywordTok}[1]{\textcolor[rgb]{0.00,0.44,0.13}{\textbf{#1}}}
\newcommand{\NormalTok}[1]{#1}
\newcommand{\OperatorTok}[1]{\textcolor[rgb]{0.40,0.40,0.40}{#1}}
\newcommand{\OtherTok}[1]{\textcolor[rgb]{0.00,0.44,0.13}{#1}}
\newcommand{\PreprocessorTok}[1]{\textcolor[rgb]{0.74,0.48,0.00}{#1}}
\newcommand{\RegionMarkerTok}[1]{#1}
\newcommand{\SpecialCharTok}[1]{\textcolor[rgb]{0.25,0.44,0.63}{#1}}
\newcommand{\SpecialStringTok}[1]{\textcolor[rgb]{0.73,0.40,0.53}{#1}}
\newcommand{\StringTok}[1]{\textcolor[rgb]{0.25,0.44,0.63}{#1}}
\newcommand{\VariableTok}[1]{\textcolor[rgb]{0.10,0.09,0.49}{#1}}
\newcommand{\VerbatimStringTok}[1]{\textcolor[rgb]{0.25,0.44,0.63}{#1}}
\newcommand{\WarningTok}[1]{\textcolor[rgb]{0.38,0.63,0.69}{\textbf{\textit{#1}}}}
\usepackage{graphicx}
\makeatletter
\def\maxwidth{\ifdim\Gin@nat@width>\linewidth\linewidth\else\Gin@nat@width\fi}
\def\maxheight{\ifdim\Gin@nat@height>\textheight\textheight\else\Gin@nat@height\fi}
\makeatother
% Scale images if necessary, so that they will not overflow the page
% margins by default, and it is still possible to overwrite the defaults
% using explicit options in \includegraphics[width, height, ...]{}
\setkeys{Gin}{width=\maxwidth,height=\maxheight,keepaspectratio}
\ifxetex
  \usepackage[setpagesize=false, % page size defined by xetex
              unicode=false, % unicode breaks when used with xetex
              xetex]{hyperref}
\else
  \usepackage[unicode=true]{hyperref}
\fi
\hypersetup{breaklinks=true,
            bookmarks=true,
            pdfauthor={Justin Le},
            pdftitle={Lenses embody Products, Prisms embody Sums},
            colorlinks=true,
            citecolor=blue,
            urlcolor=blue,
            linkcolor=magenta,
            pdfborder={0 0 0}}
\urlstyle{same}  % don't use monospace font for urls
% Make links footnotes instead of hotlinks:
\renewcommand{\href}[2]{#2\footnote{\url{#1}}}
\setlength{\parindent}{0pt}
\setlength{\parskip}{6pt plus 2pt minus 1pt}
\setlength{\emergencystretch}{3em}  % prevent overfull lines
\setcounter{secnumdepth}{0}

\title{Lenses embody Products, Prisms embody Sums}
\author{Justin Le}

\begin{document}
\maketitle

\emph{Originally posted on
\textbf{\href{https://blog.jle.im/entry/lenses-products-prisms-sums.html}{in
Code}}.}

I've written about a variety of topics on this blog, but one thing I haven't
touched in too much detail is the topic of lenses and optics. A big part of this
is because there are already so many great resources on lenses.

This post won't be a ``lens tutorial'', but rather a dive into an insightful
perspective on lenses and prisms that I've heard repeated many times, but not
yet all compiled into a single place. In particular, I'm going to talk about the
perspective of lenses and prisms as embodying the essences of products and sums
(respectively), and how that observation can help you with a more ``practical''
understanding of lenses and prisms.

\hypertarget{an-algebraic-recap}{%
\section{An Algebraic Recap}\label{an-algebraic-recap}}

In Haskell, ``products and sums'' can roughly be said to correspond to ``tuples
and \texttt{Either}''. If I have two types \texttt{A} and \texttt{B},
\texttt{(A,\ B)} is their ``product'' type. It's often called an ``anonymous
product'', because we can make one without having to give it a fancy name. It's
called a product type because \texttt{A} has
\includegraphics{https://latex.codecogs.com/png.latex?n} possible values and
\texttt{B} has \includegraphics{https://latex.codecogs.com/png.latex?m} possible
values, then \texttt{(A,\ B)} has
\includegraphics{https://latex.codecogs.com/png.latex?n\%20\%5Ctimes\%20m}
possible values\footnote{All of this is disregarding the notorious ``bottom''
  value that inhabits every type.}. And, \texttt{Either\ A\ B} is their
(anonymous) ``sum'' type. It's called a sum type because \texttt{Either\ A\ B}
has \includegraphics{https://latex.codecogs.com/png.latex?n\%20\%2B\%20m}
possible values. I won't go much deeper into this, but there are
\href{https://codewords.recurse.com/issues/three/algebra-and-calculus-of-algebraic-data-types}{many
useful summaries already online} on this topic!

\hypertarget{lets-get-productive}{%
\section{Let's Get Productive!}\label{lets-get-productive}}

It's easy to recognize \texttt{(Int,\ Double)} as a product between \texttt{Int}
and \texttt{Bool}. However, did you know that some types are secretly product
types in disguise?

For example, here's a classic example of a data type often used with
\emph{lens}:

\begin{Shaded}
\begin{Highlighting}[]
\KeywordTok{data} \DataTypeTok{Person} \FunctionTok{=} \DataTypeTok{P}\NormalTok{ \{}\OtherTok{ _pName ::} \DataTypeTok{String}
\NormalTok{                ,}\OtherTok{ _pAge  ::} \DataTypeTok{Int}
\NormalTok{                \}}
\end{Highlighting}
\end{Shaded}

\texttt{Person} is an algebraic data type --- so-called because it is actually a
\emph{product} between a \texttt{String} and \texttt{Int}. \texttt{Person} is
\emph{isomorphic} to \texttt{(String,\ Int)}. I will be writing this as
\texttt{Person\ \textless{}\textasciitilde{}\textgreater{}\ (String,\ Int)}.

By \emph{isomorphic}, I mean that there are functions
\texttt{split\ ::\ Person\ -\textgreater{}\ (String,\ Int)} and
\texttt{unsplit\ ::\ (String,\ Int)\ -\textgreater{}\ Person} where
\texttt{unsplit\ .\ split\ =\ id} and \texttt{split\ .\ unsplit\ =\ id}. You can
think of this property as stating formally that you should be able to go from
one type to the other without ``losing any information''.

In our case, we have:

\begin{Shaded}
\begin{Highlighting}[]
\OtherTok{split ::} \DataTypeTok{Person} \OtherTok{->}\NormalTok{ (}\DataTypeTok{String}\NormalTok{, }\DataTypeTok{Int}\NormalTok{)}
\NormalTok{split (}\DataTypeTok{P}\NormalTok{ n a) }\FunctionTok{=}\NormalTok{ (n, a)}

\OtherTok{unsplit ::}\NormalTok{ (}\DataTypeTok{String}\NormalTok{, }\DataTypeTok{Int}\NormalTok{) }\OtherTok{->} \DataTypeTok{Person}
\NormalTok{unsplit (n, a) }\FunctionTok{=} \DataTypeTok{P}\NormalTok{ n a}
\end{Highlighting}
\end{Shaded}

And we can verify that \texttt{unsplit\ .\ split} is \texttt{id}:

\begin{Shaded}
\begin{Highlighting}[]
\NormalTok{unsplit }\FunctionTok{.}\OtherTok{ split ::} \DataTypeTok{Person} \OtherTok{->} \DataTypeTok{Person}
\NormalTok{unsplit }\FunctionTok{.}\NormalTok{ split}
    \FunctionTok{=}\NormalTok{ \textbackslash{}x          }\OtherTok{->}\NormalTok{ unsplit (split x)        }\CommentTok{-- substitute definition of (.)}
    \FunctionTok{=}\NormalTok{ \textbackslash{}}\KeywordTok{case} \DataTypeTok{P}\NormalTok{ n a }\OtherTok{->}\NormalTok{ unsplit (split (}\DataTypeTok{P}\NormalTok{ n a))  }\CommentTok{-- expand patterns}
    \FunctionTok{=}\NormalTok{ \textbackslash{}}\KeywordTok{case} \DataTypeTok{P}\NormalTok{ n a }\OtherTok{->}\NormalTok{ unsplit (n, a)           }\CommentTok{-- substitute definition of split}
    \FunctionTok{=}\NormalTok{ \textbackslash{}}\KeywordTok{case} \DataTypeTok{P}\NormalTok{ n a }\OtherTok{->} \DataTypeTok{P}\NormalTok{ n a                    }\CommentTok{-- substitute definition of unsplit}
    \FunctionTok{=}\NormalTok{ \textbackslash{}x      }\OtherTok{->}\NormalTok{ x                            }\CommentTok{-- condense patterns}
    \FunctionTok{=}\NormalTok{ id                                      }\CommentTok{-- definition of id}
\end{Highlighting}
\end{Shaded}

And verification of \texttt{split\ .\ unsplit\ =\ id} is left as an exercise.

There are some other interesting products in Haskell, too. One such example is
\texttt{NonEmpty\ a} being a product between \texttt{a} (the head/first item)
and \texttt{{[}a{]}} (the tail/rest of the items). This means that
\texttt{NonEmpty\ a} is isomorphic to \texttt{(a,\ {[}a{]})} --- we have
\texttt{NonEmpty\ a\ \textless{}\textasciitilde{}\textgreater{}\ (a,\ {[}a{]})}!

Another curious product is the fact that every type \texttt{a} is a product
between \emph{itself} and unit, \texttt{()}. That is, every type \texttt{a} is
isomorphic to \texttt{(a,\ ())} (which follows from the algebraic property
\includegraphics{https://latex.codecogs.com/png.latex?x\%20\%2A\%201\%20\%3D\%20x}).
Freaky, right?

\begin{Shaded}
\begin{Highlighting}[]
\CommentTok{-- a <~> (a, ())}

\OtherTok{split ::}\NormalTok{ a }\OtherTok{->}\NormalTok{ (a, ())}
\NormalTok{split x }\FunctionTok{=}\NormalTok{ (x, ())}

\OtherTok{unsplit ::}\NormalTok{ (a, ()) }\OtherTok{->}\NormalTok{ a}
\NormalTok{unsplit (x, _) }\FunctionTok{=}\NormalTok{ x}
\end{Highlighting}
\end{Shaded}

One final interesting ``product in disguise'' is \texttt{Either\ a\ a}. ``But
wait,'' you say. ``That's a sum\ldots{}right??''

Well, yeah. But in addition, any \texttt{Either\ a\ a} is the product between
\texttt{Bool} and \texttt{a}. That is, \texttt{Either\ a\ a} is isomorphic to
\texttt{(Bool,\ a)}. The \texttt{Bool} tells you ``left or right?'' and the
\texttt{a} is the contents!

\begin{Shaded}
\begin{Highlighting}[]
\CommentTok{-- Either a a <~> (Bool, a)}

\OtherTok{split ::} \DataTypeTok{Either}\NormalTok{ a a }\OtherTok{->}\NormalTok{ (}\DataTypeTok{Bool}\NormalTok{, a)}
\NormalTok{split (}\DataTypeTok{Left}\NormalTok{  x) }\FunctionTok{=}\NormalTok{ (}\DataTypeTok{False}\NormalTok{, x)}
\NormalTok{split (}\DataTypeTok{Right}\NormalTok{ x) }\FunctionTok{=}\NormalTok{ (}\DataTypeTok{True}\NormalTok{ , x)}

\OtherTok{unsplit ::}\NormalTok{ (}\DataTypeTok{Bool}\NormalTok{, a) }\OtherTok{->} \DataTypeTok{Either}\NormalTok{ a a}
\NormalTok{unsplit (}\DataTypeTok{False}\NormalTok{, x) }\FunctionTok{=} \DataTypeTok{Left}\NormalTok{  x}
\NormalTok{unsplit (}\DataTypeTok{True}\NormalTok{ , x) }\FunctionTok{=} \DataTypeTok{Right}\NormalTok{ x}
\end{Highlighting}
\end{Shaded}

Proving that \texttt{unsplit\ .\ split\ =\ id}:

\begin{Shaded}
\begin{Highlighting}[]
\NormalTok{unsplit }\FunctionTok{.}\OtherTok{ split ::} \DataTypeTok{Either}\NormalTok{ a a }\OtherTok{->} \DataTypeTok{Either}\NormalTok{ a a}
\NormalTok{unsplit }\FunctionTok{.}\NormalTok{ split }\FunctionTok{=}
    \FunctionTok{=}\NormalTok{ \textbackslash{}x            }\OtherTok{->}\NormalTok{ unsplit (split x)          }\CommentTok{-- substitute definition of (.)}
      \CommentTok{-- trying case 1}
    \FunctionTok{=}\NormalTok{ \textbackslash{}}\KeywordTok{case} \DataTypeTok{Left}\NormalTok{  y }\OtherTok{->}\NormalTok{ unsplit (split (}\DataTypeTok{Left}\NormalTok{  y))  }\CommentTok{-- expand pattern for case 1}
    \FunctionTok{=}\NormalTok{ \textbackslash{}}\KeywordTok{case} \DataTypeTok{Left}\NormalTok{  y }\OtherTok{->}\NormalTok{ unsplit (}\DataTypeTok{False}\NormalTok{, y)         }\CommentTok{-- substitute definition of split}
    \FunctionTok{=}\NormalTok{ \textbackslash{}}\KeywordTok{case} \DataTypeTok{Left}\NormalTok{  y }\OtherTok{->} \DataTypeTok{Left}\NormalTok{  y                    }\CommentTok{-- substitute definition of unsplit}
    \FunctionTok{=}\NormalTok{ \textbackslash{}x            }\OtherTok{->}\NormalTok{ x                          }\CommentTok{-- condense pattern for case 1}
    \FunctionTok{=}\NormalTok{ id                                          }\CommentTok{-- definition of id}
      \CommentTok{-- trying case 2}
    \FunctionTok{=}\NormalTok{ \textbackslash{}}\KeywordTok{case} \DataTypeTok{Right}\NormalTok{ y }\OtherTok{->}\NormalTok{ unsplit (split (}\DataTypeTok{Right}\NormalTok{ y))  }\CommentTok{-- expand pattern for case 2}
    \FunctionTok{=}\NormalTok{ \textbackslash{}}\KeywordTok{case} \DataTypeTok{Right}\NormalTok{ y }\OtherTok{->}\NormalTok{ unsplit (}\DataTypeTok{True}\NormalTok{ , y)         }\CommentTok{-- substitute definition of split}
    \FunctionTok{=}\NormalTok{ \textbackslash{}}\KeywordTok{case} \DataTypeTok{Right}\NormalTok{ y }\OtherTok{->} \DataTypeTok{Right}\NormalTok{ y                    }\CommentTok{-- substitute definition of unsplit}
    \FunctionTok{=}\NormalTok{ \textbackslash{}x            }\OtherTok{->}\NormalTok{ x                          }\CommentTok{-- condense pattern for case 2}
    \FunctionTok{=}\NormalTok{ id                                          }\CommentTok{-- definition of id}
\end{Highlighting}
\end{Shaded}

And \texttt{split\ .\ unsplit\ =\ id} is again left as an exercise.

(\texttt{\textbackslash{}case} here is from the \emph{-XLambdaCase} extension)

\hypertarget{lenses}{%
\subsection{Lenses}\label{lenses}}

So, how do lenses come into the picture?

Let's review a bit. A \texttt{Lens\textquotesingle{}\ s\ a} is a way to
``access'' an \texttt{a} ``inside'' an \texttt{s}, respecting some laws.

A \texttt{Lens\textquotesingle{}\ s\ a} is a data type with the following API:

\begin{Shaded}
\begin{Highlighting}[]
\OtherTok{view ::} \DataTypeTok{Lens'}\NormalTok{ s a }\OtherTok{->}\NormalTok{ (s }\OtherTok{->}\NormalTok{ a)                }\CommentTok{-- get the 'a' from an 's'}
\OtherTok{set  ::} \DataTypeTok{Lens'}\NormalTok{ s a }\OtherTok{->}\NormalTok{ (a }\OtherTok{->}\NormalTok{ s }\OtherTok{->}\NormalTok{ s)           }\CommentTok{-- set the 'a' inside an 's'}
\end{Highlighting}
\end{Shaded}

respecting
\href{https://www.schoolofhaskell.com/school/to-infinity-and-beyond/pick-of-the-week/a-little-lens-starter-tutorial\#the-lens-laws-}{some
laws} --- get-put, put-get, and put-put. Abstract mathematical laws are great
and all, but I'm going to tell you a secret that gives a nice restatement of
those laws.

At first, you might naively implement lenses like:

\begin{Shaded}
\begin{Highlighting}[]
\KeywordTok{data} \DataTypeTok{Lens'}\NormalTok{ s a }\FunctionTok{=} \DataTypeTok{Lens'}\NormalTok{ \{}\OtherTok{ view ::}\NormalTok{ s }\OtherTok{->}\NormalTok{ a}
\NormalTok{                       ,}\OtherTok{ set  ::}\NormalTok{ a }\OtherTok{->}\NormalTok{ s }\OtherTok{->}\NormalTok{ s}
\NormalTok{                       \}}
\end{Highlighting}
\end{Shaded}

But this is bad bad bad. That's because you can use this to represent lenses
that ``break the laws''. This representation is, to use the technical term,
``too big''. It allows more more values than are actual lenses.

So, here's the secret: A \texttt{Lens\textquotesingle{}\ s\ a} means that
\emph{\texttt{s} is a product between \texttt{a} and some type \texttt{q}}.

That means that if it is possible to represent \texttt{s} as some
\texttt{(a,\ q)} (that is,
\texttt{s\ \textless{}\textasciitilde{}\textgreater{}\ (a,\ q)}), \emph{then you
have two lenses}! Lenses are nothing more than \emph{descriptions of products}!
Another way to think of this is that if you are able to ``split'' a type into
two parts without losing any information, then each part represents a lens.

A \texttt{Lens\textquotesingle{}\ s\ a} is nothing more than a witness for an
\texttt{exists\ q.\ s\ \textless{}\textasciitilde{}\textgreater{}\ (a,\ q)}
isomorphism.\footnote{The
  \texttt{exists\ q.\ s\ \textless{}\textasciitilde{}\textgreater{}\ (a,\ q)} is
  a way of saying that \texttt{Lens\textquotesingle{}\ s\ a} witnesses an
  isomorphism between \texttt{s} and the product of \texttt{a} and some
  ``hidden'' type \texttt{q}. A \texttt{Lens\textquotesingle{}\ s\ a} is a
  statement that some \texttt{q} \emph{exists} in the first place. If no such
  type exists, no lens is possible. And, for many lenses, \texttt{q} might be an
  abstract type.}

With that in mind, let's re-visit a saner definition of lenses based on the idea
that lenses embody descriptions of products:

\begin{Shaded}
\begin{Highlighting}[]
\KeywordTok{data} \DataTypeTok{Lens'}\NormalTok{ s a }\FunctionTok{=}\NormalTok{ forall q}\FunctionTok{.}
                 \DataTypeTok{Lens'}\NormalTok{ \{}\OtherTok{ split   ::}\NormalTok{ s }\OtherTok{->}\NormalTok{ (a, q)}
\NormalTok{                       ,}\OtherTok{ unsplit ::}\NormalTok{ (a, q) }\OtherTok{->}\NormalTok{ s}
\NormalTok{                       \}    }\CommentTok{-- ^ s <~> (a, q)}
\end{Highlighting}
\end{Shaded}

(the \texttt{forall\ q.} is the \emph{-XExistentialQuantification} extension,
and allows us to hide type variables in consructors)

Now, if \texttt{split} and \texttt{join} form an isomorphism, \emph{this can
only represent valid lenses}!\footnote{This type is technically also ``too big''
  (you can write a value where \texttt{split} and \texttt{unsplit} do not form
  an isomorphism), but I think, to me, ``\texttt{split} and \texttt{join} must
  form an isomorphism'' is a much clearer and natural law than
  get-put/put-get/put-put.}

We can implement our necessary lens API as so:

\begin{Shaded}
\begin{Highlighting}[]
\OtherTok{view ::} \DataTypeTok{Lens'}\NormalTok{ s a }\OtherTok{->}\NormalTok{ (s }\OtherTok{->}\NormalTok{ a)}
\NormalTok{view }\DataTypeTok{Lens'}\NormalTok{\{}\FunctionTok{..}\NormalTok{\} }\FunctionTok{=}\NormalTok{ fst }\FunctionTok{.}\NormalTok{ split}

\OtherTok{set ::} \DataTypeTok{Lens'}\NormalTok{ s a }\OtherTok{->}\NormalTok{ (a }\OtherTok{->}\NormalTok{ s }\OtherTok{->}\NormalTok{ s)}
\NormalTok{set }\DataTypeTok{Lens'}\NormalTok{\{}\FunctionTok{..}\NormalTok{\} newVal x }\FunctionTok{=} \KeywordTok{case}\NormalTok{ split x }\KeywordTok{of}
\NormalTok{    (_, q) }\OtherTok{->}\NormalTok{ unsplit (newVal, q)      }\CommentTok{-- "replace" the `a`}
\end{Highlighting}
\end{Shaded}

(Using the \emph{-XRecordWildcards} extension, where
\texttt{Lens\textquotesingle{}\{..\}} binds \texttt{split} and \texttt{unsplit}
to the fields of the lens)

The implementation of the helper function \texttt{over} (which modifies the
\texttt{a} with a function) is also particularly elegant:

\begin{Shaded}
\begin{Highlighting}[]
\OtherTok{over ::} \DataTypeTok{Lens'}\NormalTok{ s a }\OtherTok{->}\NormalTok{ (a }\OtherTok{->}\NormalTok{ a) }\OtherTok{->}\NormalTok{ (s }\OtherTok{->}\NormalTok{ s)}
\NormalTok{over }\DataTypeTok{Lens'}\NormalTok{\{}\FunctionTok{..}\NormalTok{\} f }\FunctionTok{=}\NormalTok{ unsplit }\FunctionTok{.}\NormalTok{ first f }\FunctionTok{.}\NormalTok{ split}
\end{Highlighting}
\end{Shaded}

The surprising result of this perspective is that \textbf{every product yields
lenses} (one for every item in the product), and \textbf{every lens witnesses
one side of a product}.

\hypertarget{insights-gleamed}{%
\subsection{Insights Gleamed}\label{insights-gleamed}}

Let's take a look at our first product we talked about:

\begin{Shaded}
\begin{Highlighting}[]
\KeywordTok{data} \DataTypeTok{Person} \FunctionTok{=} \DataTypeTok{P}\NormalTok{ \{}\OtherTok{ _pName ::} \DataTypeTok{String}
\NormalTok{                ,}\OtherTok{ _pAge  ::} \DataTypeTok{Int}
\NormalTok{                \}}

\OtherTok{split ::} \DataTypeTok{Person} \OtherTok{->}\NormalTok{ (}\DataTypeTok{String}\NormalTok{, }\DataTypeTok{Int}\NormalTok{)}
\NormalTok{split (}\DataTypeTok{P}\NormalTok{ n a) }\FunctionTok{=}\NormalTok{ (n, a)}

\OtherTok{unsplit ::}\NormalTok{ (}\DataTypeTok{String}\NormalTok{, }\DataTypeTok{Int}\NormalTok{) }\OtherTok{->} \DataTypeTok{Person}
\NormalTok{unsplit (n, a) }\FunctionTok{=} \DataTypeTok{P}\NormalTok{ n a}
\end{Highlighting}
\end{Shaded}

Because \texttt{Person} is a product between \texttt{String} and \texttt{Int},
we get \emph{two lenses}: a \texttt{Lens\textquotesingle{}\ Person\ String} and
\texttt{Lens\textquotesingle{}\ Person\ Int}. \emph{Every product} gives us a
lens for every item in the product.

\begin{Shaded}
\begin{Highlighting}[]
\CommentTok{-- Person <~> (String, Int)}

\OtherTok{pName ::} \DataTypeTok{Lens'} \DataTypeTok{Person} \DataTypeTok{String}
\NormalTok{pName }\FunctionTok{=} \DataTypeTok{Lens'}\NormalTok{ \{ split   }\FunctionTok{=}\NormalTok{ \textbackslash{}(}\DataTypeTok{P}\NormalTok{ n a) }\OtherTok{->}\NormalTok{ (n, a)}
\NormalTok{              , unsplit }\FunctionTok{=}\NormalTok{ \textbackslash{}(n, a)  }\OtherTok{->} \DataTypeTok{P}\NormalTok{ n a}
\NormalTok{              \}}

\OtherTok{pAge ::} \DataTypeTok{Lens'} \DataTypeTok{Person} \DataTypeTok{Int}
\NormalTok{pAge }\FunctionTok{=} \DataTypeTok{Lens'}\NormalTok{ \{ split   }\FunctionTok{=}\NormalTok{ \textbackslash{}(}\DataTypeTok{P}\NormalTok{ n a) }\OtherTok{->}\NormalTok{ (a, n)}
\NormalTok{             , unsplit }\FunctionTok{=}\NormalTok{ \textbackslash{}(a, n)  }\OtherTok{->} \DataTypeTok{P}\NormalTok{ n a}
\NormalTok{             \}}
\end{Highlighting}
\end{Shaded}

These are actually the typical lenses associated with records! You get exactly
these lenses if you use \texttt{makeLenses} from the \emph{lens} package.

The inverse is true too. \textbf{Every lens witnesses a product}. The fact that
we have a lawful \texttt{pName\ ::\ Lens\textquotesingle{}\ Person\ String}
means that a \texttt{Person} \emph{must} be a product between \texttt{String}
and some other (hidden) type.

It can be insightful to look at products that we know and see what lenses those
correspond to.

For example, our
\texttt{NonEmpty\ a\ \textless{}\textasciitilde{}\textgreater{}\ (a,\ {[}a{]})}
product tells us that \texttt{NonEmpty\ a} has at least two lenses: a ``head''
lens \texttt{Lens\textquotesingle{}\ (NonEmpty\ a)\ a} and a ``tail'' lens
\texttt{Lens\textquotesingle{}\ (NonEmpty\ a)\ {[}a{]}}.

Our \texttt{a\ \textless{}\textasciitilde{}\textgreater{}\ (a,\ ())} product
gives some interesting insight. This tells us that we always have an
``identity'' lens \texttt{Lens\textquotesingle{}\ a\ a}, and a ``unit'' lens
\texttt{Lens\textquotesingle{}\ a\ ()}, for any \texttt{a}:

\begin{Shaded}
\begin{Highlighting}[]
\CommentTok{-- a <~> (a, ())}

\OtherTok{identity ::} \DataTypeTok{Lens'}\NormalTok{ a a}
\NormalTok{identity }\FunctionTok{=} \DataTypeTok{Lens'}\NormalTok{ \{ split   }\FunctionTok{=}\NormalTok{ \textbackslash{}x      }\OtherTok{->}\NormalTok{ (x, ())}
\NormalTok{                 , unsplit }\FunctionTok{=}\NormalTok{ \textbackslash{}(x, _) }\OtherTok{->}\NormalTok{ x}
\NormalTok{                 \}}

\OtherTok{united ::} \DataTypeTok{Lens'}\NormalTok{ a ()}
\NormalTok{united }\FunctionTok{=} \DataTypeTok{Lens'}\NormalTok{ \{ split   }\FunctionTok{=}\NormalTok{ \textbackslash{}x       }\OtherTok{->}\NormalTok{ ((), x)}
\NormalTok{               , unsplit }\FunctionTok{=}\NormalTok{ \textbackslash{}((), x) }\OtherTok{->}\NormalTok{ x}
\NormalTok{               \}}
\end{Highlighting}
\end{Shaded}

In the language of lens, \texttt{identity\ ::\ Lens\textquotesingle{}\ a\ a}
tells us that all \texttt{a}s have an \texttt{a} ``inside'' them. However, in
the language of products, this just tells us that \texttt{a} can be represented
as \texttt{(a,\ ())}. In the language of lens,
\texttt{united\ ::\ Lens\textquotesingle{}\ a\ ()} tells us that all \texttt{a}s
have a \texttt{()} ``inside'' them. In the language of products, this just tells
us that \texttt{a\ \textless{}\textasciitilde{}\textgreater{}\ (a,\ ())}.

What insight does our
\texttt{Either\ a\ a\ \textless{}\textasciitilde{}\textgreater{}\ (Bool,\ a)}
product perspective give us? Well, let's write out their types and see what it
might suggest:

\begin{Shaded}
\begin{Highlighting}[]
\OtherTok{mysteryLens1 ::} \DataTypeTok{Lens'}\NormalTok{ (}\DataTypeTok{Either}\NormalTok{ a a) }\DataTypeTok{Bool}
\OtherTok{mysteryLens2 ::} \DataTypeTok{Lens'}\NormalTok{ (}\DataTypeTok{Either}\NormalTok{ a a) a}
\end{Highlighting}
\end{Shaded}

Looking at
\texttt{mysteryLens1\ ::\ Lens\textquotesingle{}\ (Either\ a\ a)\ Bool}, we are
saying that every \texttt{Either\ a\ a} has some \texttt{Bool} ``inside'' it.
From our knowledge of our product, we know that this \texttt{Bool} is really a
\emph{flag} for left-ness or right-ness. Getting the \texttt{Bool} is finding
out if we're in \texttt{Left} or \texttt{Right}, and flipping the \texttt{Bool}
``inside'' is really just swapping from \texttt{Left} to \texttt{Right}.

\begin{Shaded}
\begin{Highlighting}[]
\OtherTok{flipEither ::} \DataTypeTok{Either}\NormalTok{ a a }\OtherTok{->} \DataTypeTok{Either}\NormalTok{ a a}
\NormalTok{flipEither }\FunctionTok{=}\NormalTok{ over mysteryLens1 not}

\OtherTok{isLeft ::} \DataTypeTok{Either}\NormalTok{ a a }\OtherTok{->} \DataTypeTok{Bool}
\NormalTok{isLeft }\FunctionTok{=}\NormalTok{ view mysteryLens1}
\end{Highlighting}
\end{Shaded}

\begin{Shaded}
\begin{Highlighting}[]
\NormalTok{ghci}\FunctionTok{>}\NormalTok{ flipEither (}\DataTypeTok{Left} \CharTok{'a'}\NormalTok{)}
\DataTypeTok{Right} \CharTok{'a'}
\NormalTok{ghci}\FunctionTok{>}\NormalTok{ flipEither (}\DataTypeTok{Right} \CharTok{'a'}\NormalTok{)}
\DataTypeTok{Left} \CharTok{'a'}
\NormalTok{ghci}\FunctionTok{>}\NormalTok{ isLeft (}\DataTypeTok{Left} \CharTok{'a'}\NormalTok{)}
\DataTypeTok{True}
\NormalTok{ghci}\FunctionTok{>}\NormalTok{ isLeft (}\DataTypeTok{Right} \CharTok{'a'}\NormalTok{)}
\DataTypeTok{False}
\end{Highlighting}
\end{Shaded}

Note that if we look at lenses as embodying ``record fields'' (things that give
you the ability to ``get'' a field, and ``modify'' a field --- corresponding
with \texttt{view} and \texttt{set}), we can think of \texttt{mysteryLens1} as
an \emph{abstract record field} into the Leftness/Rightness of a value. Thing of
lenses as defining abstract record fields is a
\href{http://blog.ezyang.com/2016/12/a-tale-of-backwards-compatibility-in-asts/}{common
tool for backwards compatiblity}.

Looking at \texttt{mysteryLens2\ ::\ Lens\textquotesingle{}\ (Either\ a\ a)\ a},
we are saying that every \texttt{Either\ a\ a} has some \texttt{a} ``inside''
it. From what we know about the underlying product, the \texttt{a} is just the
``contained value'', \emph{ignoring} leftness or rightness. Getting the
\texttt{a} is getting the contained value and losing leftness/rightness, and
re-setting the \texttt{a} inside is modifying the contained value but preserving
leftness/rightness.

\begin{Shaded}
\begin{Highlighting}[]
\OtherTok{fromEither ::} \DataTypeTok{Either}\NormalTok{ a a }\OtherTok{->}\NormalTok{ a}
\NormalTok{fromEither }\FunctionTok{=}\NormalTok{ view mysteryLens2}

\OtherTok{mapEither ::}\NormalTok{ (a }\OtherTok{->}\NormalTok{ a) }\OtherTok{->} \DataTypeTok{Either}\NormalTok{ a a }\OtherTok{->} \DataTypeTok{Either}\NormalTok{ a a}
\NormalTok{mapEither }\FunctionTok{=}\NormalTok{ over mysteryLens2}
\end{Highlighting}
\end{Shaded}

\begin{Shaded}
\begin{Highlighting}[]
\NormalTok{ghci}\FunctionTok{>}\NormalTok{ fromEither (}\DataTypeTok{Left} \CharTok{'a'}\NormalTok{)}
\CharTok{'a'}
\NormalTok{ghci}\FunctionTok{>}\NormalTok{ mapEither negate (}\DataTypeTok{Right} \DecValTok{4}\NormalTok{)}
\DataTypeTok{Right}\NormalTok{ (}\FunctionTok{-}\DecValTok{4}\NormalTok{)}
\end{Highlighting}
\end{Shaded}

So that's really the essence of what a \texttt{Lens\textquotesingle{}} is. A
\texttt{Lens\textquotesingle{}\ s\ a} is the embodiment of the fact that
\texttt{s} can be represented as a product between \texttt{a} and something else
--- that \texttt{s\ \textless{}\textasciitilde{}\textgreater{}\ (a,\ q)}. All of
the lens laws just boil down to this. \textbf{Lenses embody products}.

\hypertarget{what-isnt-a-lens}{%
\subsection{What Isn't a Lens?}\label{what-isnt-a-lens}}

This perspective also gives you some insight into when things \emph{aren't}
lenses. For example, is it possible to make a lens that gives you the first item
in a list?

No, there isn't, because there isn't any type \texttt{q} that you could factor
out \texttt{{[}a{]}} into as \texttt{(a,\ q)}. That would imply that
\texttt{{[}a{]}} is always ``an \texttt{a} with something else''\ldots{}but this
isn't true with \texttt{{[}{]}}.

\hypertarget{sum-thing-interesting}{%
\section{"Sum-thing" Interesting}\label{sum-thing-interesting}}

It's easy to recognize \texttt{Either\ Int\ Bool} as a sum between \texttt{Int}
and \texttt{Bool}. However, did you know that some types are secretly sums in
disguise?

For example, here's a data type you might encounter out there in the real world:

\begin{Shaded}
\begin{Highlighting}[]
\KeywordTok{data} \DataTypeTok{Shape} \FunctionTok{=} \DataTypeTok{Circle}  \DataTypeTok{Double}           \CommentTok{-- radius}
           \FunctionTok{|} \DataTypeTok{RegPoly} \DataTypeTok{Natural} \DataTypeTok{Double}   \CommentTok{-- number of sides, length of sides}
\end{Highlighting}
\end{Shaded}

\texttt{Circle\ 2.9} represents a circle with radius 2.9, and
\texttt{RegPoly\ 8\ 4.6} represents a octagon (8-sided figure) whose sides all
have length 4.6.

\texttt{Shape} is an algebraic data type --- so-called because it is actually a
\emph{sum} between \texttt{Double} and \texttt{(Natural,\ Double)} (a
\texttt{Natural} is the non-negative \texttt{Integer} type). \texttt{Shape} is
\emph{isomorphic} to \texttt{Either\ Double\ (Natural,\ Double)}. To prove it,
let's witness
\texttt{Shape\ \textless{}\textasciitilde{}\textgreater{}\ Either\ Double\ (Natural,\ Double)}
using the functions \texttt{match} and \texttt{inject}:

\begin{Shaded}
\begin{Highlighting}[]
\CommentTok{-- Shape <~> Either Double (Natural, Double)}

\OtherTok{match ::} \DataTypeTok{Shape} \OtherTok{->} \DataTypeTok{Either} \DataTypeTok{Double}\NormalTok{ (}\DataTypeTok{Natural}\NormalTok{, }\DataTypeTok{Double}\NormalTok{)}
\NormalTok{match (}\DataTypeTok{Circle}\NormalTok{  r  ) }\FunctionTok{=} \DataTypeTok{Left}\NormalTok{ r}
\NormalTok{match (}\DataTypeTok{RegPoly}\NormalTok{ n s) }\FunctionTok{=} \DataTypeTok{Right}\NormalTok{ (n, s)}

\OtherTok{inject ::} \DataTypeTok{Either} \DataTypeTok{Double}\NormalTok{ (}\DataTypeTok{Natural}\NormalTok{, }\DataTypeTok{Double}\NormalTok{) }\OtherTok{->} \DataTypeTok{Shape}
\NormalTok{inject (}\DataTypeTok{Left}\NormalTok{   r    ) }\FunctionTok{=} \DataTypeTok{Circle}\NormalTok{  r}
\NormalTok{inject (}\DataTypeTok{Right}\NormalTok{ (n, s)) }\FunctionTok{=} \DataTypeTok{RegPoly}\NormalTok{ n s}
\end{Highlighting}
\end{Shaded}

Since \texttt{inject\ .\ match\ =\ id} and \texttt{match\ .\ inject\ =\ id},
this proves that \texttt{Shape} is a sum in disguise.

Another interesting ``hidden sum'' is the fact that \texttt{{[}a{]}} in Haskell
is actually a sum between \texttt{()} and \texttt{(a,\ {[}a{]})}. That's right
--- it's a sum between \texttt{()} and\ldots{}itself? Indeed it is pretty
bizarre.

However, if we think of \texttt{()} as the possibility of an empty list, and
\texttt{(a,\ {[}a{]})} as the possibility of \texttt{NonEmpty\ a} (the ``head''
of a list consed with the rest of the list), then saying that \texttt{{[}a{]}}
is a sum between \texttt{()} and \texttt{NonEmpty\ a} is saying that
\texttt{{[}a{]}} is ``either an empty list or a non-empty list''. Whoa. Take
\emph{that},
\href{https://en.wikipedia.org/wiki/Constructivism_(mathematics)}{LEM
denialists}.

\begin{Shaded}
\begin{Highlighting}[]
\CommentTok{-- [a] <~> Either () (NonEmpty a)}

\OtherTok{match ::}\NormalTok{ [a] }\OtherTok{->} \DataTypeTok{Either}\NormalTok{ () (}\DataTypeTok{NonEmpty}\NormalTok{ a)}
\NormalTok{match []     }\FunctionTok{=} \DataTypeTok{Left}\NormalTok{  ()}
\NormalTok{match (x}\FunctionTok{:}\NormalTok{xs) }\FunctionTok{=} \DataTypeTok{Right}\NormalTok{ (x }\FunctionTok{:|}\NormalTok{ xs)}

\OtherTok{inject ::} \DataTypeTok{Either}\NormalTok{ () (}\DataTypeTok{NonEmpty}\NormalTok{ a) }\OtherTok{->}\NormalTok{ [a]}
\NormalTok{inject (}\DataTypeTok{Left}\NormalTok{   _       ) }\FunctionTok{=}\NormalTok{ []}
\NormalTok{inject (}\DataTypeTok{Right}\NormalTok{ (x }\FunctionTok{:|}\NormalTok{ xs)) }\FunctionTok{=}\NormalTok{ x}\FunctionTok{:}\NormalTok{xs}
\end{Highlighting}
\end{Shaded}

And, actually, there is another way to deconstruct \texttt{{[}a{]}} as a sum in
Haskell. You can treat it as a sum between \texttt{()} and
\texttt{({[}a{]},\ a)} --- where the \texttt{()} represents the empty list and
the \texttt{({[}a{]},\ a)} represents an ``all but the last item'' list and
``the last item'':

\begin{Shaded}
\begin{Highlighting}[]
\CommentTok{-- [a] <~> Either () ([a], a)}

\OtherTok{match  ::}\NormalTok{ [a] }\OtherTok{->} \DataTypeTok{Either}\NormalTok{ () ([a], a)}
\NormalTok{match xs}
  \FunctionTok{|}\NormalTok{ null xs   }\FunctionTok{=} \DataTypeTok{Left}\NormalTok{  ()}
  \FunctionTok{|}\NormalTok{ otherwise }\FunctionTok{=} \DataTypeTok{Right}\NormalTok{ (init xs, last xs)}

\CommentTok{-- init gives you all but the last item:}
\CommentTok{-- > init [1,2,3] = [1,2]}

\OtherTok{inject ::} \DataTypeTok{Either}\NormalTok{ () (a, [a]) }\OtherTok{->}\NormalTok{ [a]}
\NormalTok{inject (}\DataTypeTok{Left}\NormalTok{   _     ) }\FunctionTok{=}\NormalTok{ []}
\NormalTok{inject (}\DataTypeTok{Right}\NormalTok{ (xs, x)) }\FunctionTok{=}\NormalTok{ xs }\FunctionTok{++}\NormalTok{ [x]}
\end{Highlighting}
\end{Shaded}

I just think it's interesting that the same type can be ``decomposed'' into a
sum of two different types in multiple ways.

Another curious sum: if we consider the ``empty data type'' \texttt{Void}, the
type with no inhabitants:

\begin{Shaded}
\begin{Highlighting}[]
\KeywordTok{data} \DataTypeTok{Void}           \CommentTok{-- no constructors, no valid inhabitants}
\end{Highlighting}
\end{Shaded}

then we have a curious sum: every type \texttt{a} is a sum between \emph{itself}
and \texttt{Void}. In other words, \texttt{a} is isomorphic to
\texttt{Either\ a\ Void} (which follows from the algebraic property
\includegraphics{https://latex.codecogs.com/png.latex?x\%20\%2B\%200\%20\%3D\%20x}):

\begin{Shaded}
\begin{Highlighting}[]
\CommentTok{-- a <~> Either a Void}

\CommentTok{-- | A useful helper function when working with `Void`}
\OtherTok{absurd ::} \DataTypeTok{Void} \OtherTok{->}\NormalTok{ a}
\NormalTok{absurd }\FunctionTok{=}\NormalTok{ \textbackslash{}}\KeywordTok{case} \CommentTok{-- empty case statement because we have}
               \CommentTok{-- no constructors of 'Void' we need to}
               \CommentTok{-- match on}

\OtherTok{match ::}\NormalTok{ a }\OtherTok{->} \DataTypeTok{Either}\NormalTok{ a }\DataTypeTok{Void}
\NormalTok{match x }\FunctionTok{=} \DataTypeTok{Left}\NormalTok{ x}

\OtherTok{inject ::} \DataTypeTok{Either}\NormalTok{ a }\DataTypeTok{Void} \OtherTok{->}\NormalTok{ a}
\NormalTok{inject (}\DataTypeTok{Left}\NormalTok{  x) }\FunctionTok{=}\NormalTok{ x}
\NormalTok{inject (}\DataTypeTok{Right}\NormalTok{ v) }\FunctionTok{=}\NormalTok{ absurd v}
\end{Highlighting}
\end{Shaded}

Again, if you don't believe me, verify that \texttt{inject\ .\ match\ =\ id} and
\texttt{match\ .\ inject\ =\ id}!

One final example: earlier, we said that every type can be decomposed as a
\emph{product} involving \texttt{()}. Algebraically, finding that mystery type
is easy --- we solve
\includegraphics{https://latex.codecogs.com/png.latex?x\%20\%3D\%201\%20\%2A\%20y}
for \includegraphics{https://latex.codecogs.com/png.latex?y} (since \texttt{()}
is 1), and we see
\includegraphics{https://latex.codecogs.com/png.latex?y\%20\%3D\%20x}. This
tells us that every type is a product between \texttt{()} and itself
(\texttt{a\ \textless{}\textasciitilde{}\textgreater{}\ ((),\ a)}).

However, can every type be decomposed as a \emph{sum} involving \texttt{()}?

Algebraically, we need to find this mystery type by solving
\includegraphics{https://latex.codecogs.com/png.latex?x\%20\%3D\%201\%20\%2B\%20y}
for \includegraphics{https://latex.codecogs.com/png.latex?y}, and the result is
\includegraphics{https://latex.codecogs.com/png.latex?y\%20\%3D\%20x\%20-\%201}.
We can interpret
\includegraphics{https://latex.codecogs.com/png.latex?x\%20-\%201} as
``\texttt{a}, minus one potential element''.

This type isn't expressible in general in Haskell, so \emph{no}, not
\emph{every} type can be decomposed as a sum involving \texttt{()}. The
necessary and sufficient condition is that there must exist some type that is
the same as your original type but with one missing element.

Oh, hey! Remember our
\texttt{{[}a{]}\ \textless{}\textasciitilde{}\textgreater{}\ Either\ ()\ (NonEmpty\ a)}
decomposition? That's exactly this! We can \texttt{NonEmpty\ a} is our mystery
type: it's exactly a list \texttt{{[}a{]}} minus one potential element (the
empty list).

There's another way to go about this: we can talk about
\includegraphics{https://latex.codecogs.com/png.latex?x\%20-\%201} by specifying
one single ``forbidden element''. This isn't explicitly possible in Haskell, but
we can simulate this by using an abstract type. We have this ability using
``refinement types''. For example, using the
\href{http://hackage.haskell.org/package/refined}{refined} library, a
\texttt{Refined\ (NotEqualTo\ 4)\ Int} is a type that is the same as
\texttt{Int}, except the \texttt{4} value is forbidden.

We can use it to implement a
\texttt{Int\ \textless{}\textasciitilde{}\textgreater{}\ Either\ ()\ (Refined\ (NotEqualTo\ 4)\ Int)}
witness:

\begin{Shaded}
\begin{Highlighting}[]
\CommentTok{-- | Like `Int`, but cannot be constructed if it is 4}
\KeywordTok{type} \DataTypeTok{Not4} \FunctionTok{=} \DataTypeTok{Refined}\NormalTok{ (}\DataTypeTok{NotEqualTo} \DecValTok{4}\NormalTok{) }\DataTypeTok{Int}

\CommentTok{-- | Provided by the 'refined' library that lets us refine and unrefine a type}
\OtherTok{refineFail ::} \DataTypeTok{Int}  \OtherTok{->} \DataTypeTok{Maybe} \DataTypeTok{Not4}
\OtherTok{unrefine   ::} \DataTypeTok{Not4} \OtherTok{->} \DataTypeTok{Int}


\CommentTok{-- | The "safe constructor"}
\OtherTok{match ::} \DataTypeTok{Int} \OtherTok{->} \DataTypeTok{Either}\NormalTok{ () }\DataTypeTok{Not4}
\NormalTok{match n }\FunctionTok{=} \KeywordTok{case}\NormalTok{ refineFail n }\KeywordTok{of}
    \DataTypeTok{Nothing} \OtherTok{->} \DataTypeTok{Left}\NormalTok{ ()          }\CommentTok{-- the value was 4, so we return `Left`}
    \DataTypeTok{Just}\NormalTok{ x  }\OtherTok{->} \DataTypeTok{Right}\NormalTok{ x          }\CommentTok{-- value was succesfully refined}

\CommentTok{-- | The "safe extractor"}
\OtherTok{inject ::} \DataTypeTok{Either}\NormalTok{ () }\DataTypeTok{Not4} \OtherTok{->} \DataTypeTok{Int}
\NormalTok{inject (}\DataTypeTok{Left}\NormalTok{  _) }\FunctionTok{=} \DecValTok{4}
\NormalTok{inject (}\DataTypeTok{Right}\NormalTok{ x) }\FunctionTok{=}\NormalTok{ unrefine x}
\end{Highlighting}
\end{Shaded}

In fact, if we can parameterize an isomorphism on a specific value, \emph{all}
types with at least one value can be expressed as a sum involving \texttt{()}!
It's always \texttt{()} plus the type itself minus that given specific value.

\hypertarget{through-the-looking-prism}{%
\subsection{Through the Looking-Prism}\label{through-the-looking-prism}}

Now let's bring prisms into the picture. A
\texttt{Prism\textquotesingle{}\ s\ a} also refers to some \texttt{a} ``inside''
an \texttt{s}, with the following API: \texttt{preview} and
\texttt{review}\footnote{I didn't invent these names}

\begin{Shaded}
\begin{Highlighting}[]
\OtherTok{preview ::} \DataTypeTok{Prism'}\NormalTok{ s a }\OtherTok{->}\NormalTok{ (s }\OtherTok{->} \DataTypeTok{Maybe}\NormalTok{ a)   }\CommentTok{-- get the 'a' in the 's' if it exists}
\OtherTok{review  ::} \DataTypeTok{Prism'}\NormalTok{ s a }\OtherTok{->}\NormalTok{ (a }\OtherTok{->}\NormalTok{ s)         }\CommentTok{-- reconstruct the 's' from an 'a'}
\end{Highlighting}
\end{Shaded}

Naively you might implement a prism like this:

\begin{Shaded}
\begin{Highlighting}[]
\KeywordTok{data} \DataTypeTok{Prism'}\NormalTok{ s a }\FunctionTok{=} \DataTypeTok{Prism'}\NormalTok{ \{}\OtherTok{ preview ::}\NormalTok{ s }\OtherTok{->} \DataTypeTok{Maybe}\NormalTok{ a}
\NormalTok{                         ,}\OtherTok{ review  ::}\NormalTok{ a }\OtherTok{->}\NormalTok{ s}
\NormalTok{                         \}}
\end{Highlighting}
\end{Shaded}

But, again, this implementation space is too big. There are way too many values
of this type that aren't \emph{actual} ``lawful'' prisms. And the laws are kind
of muddled here.

You might be able to guess where I'm going at this point. Whereas a
\texttt{Lens\textquotesingle{}\ s\ a} is nothing more than a witness to the fact
that \texttt{s} is a \emph{product} \texttt{(a,\ q)} \ldots{} a
\texttt{Prism\textquotesingle{}\ s\ a} is nothing more than a witness to the
fact that \texttt{s} is a \emph{sum} \texttt{Either\ a\ q}. If it is possible to
represent \texttt{s} as some \texttt{Either\ a\ q}\ldots{}then you have two
prisms! Prisms are nothing more than \emph{descriptions of sums}!

A \texttt{Prism\textquotesingle{}\ s\ a} is nothing more than a witness for an
\texttt{exists\ q.\ s\ \textless{}\textasciitilde{}\textgreater{}\ Either\ a\ q}
isomorphism.

Under this interpretation, we can write a nice representation of
\texttt{Prism\textquotesingle{}}:

\begin{Shaded}
\begin{Highlighting}[]
\KeywordTok{data} \DataTypeTok{Prism'}\NormalTok{ s a }\FunctionTok{=}\NormalTok{ forall q}\FunctionTok{.}
                  \DataTypeTok{Prism'}\NormalTok{ \{}\OtherTok{ match  ::}\NormalTok{ s }\OtherTok{->} \DataTypeTok{Either}\NormalTok{ a q}
\NormalTok{                         ,}\OtherTok{ inject ::} \DataTypeTok{Either}\NormalTok{ a q }\OtherTok{->}\NormalTok{ s}
\NormalTok{                         \}    }\CommentTok{-- ^ s <~> Either a q}
\end{Highlighting}
\end{Shaded}

Now, if \texttt{match} and \texttt{inject} form an isomorphism, \emph{this can
only represent valid prisms}!

We can now implement the prism API:

\begin{Shaded}
\begin{Highlighting}[]
\OtherTok{preview ::} \DataTypeTok{Prism'}\NormalTok{ s a }\OtherTok{->}\NormalTok{ (s }\OtherTok{->} \DataTypeTok{Maybe}\NormalTok{ a)}
\NormalTok{preview }\DataTypeTok{Prism'}\NormalTok{\{}\FunctionTok{..}\NormalTok{\} x }\FunctionTok{=} \KeywordTok{case}\NormalTok{ match x }\KeywordTok{of}
    \DataTypeTok{Left}\NormalTok{ _  }\OtherTok{->} \DataTypeTok{Nothing}
    \DataTypeTok{Right}\NormalTok{ y }\OtherTok{->} \DataTypeTok{Just}\NormalTok{ y}

\OtherTok{review  ::} \DataTypeTok{Prism'}\NormalTok{ s a }\OtherTok{->}\NormalTok{ (a }\OtherTok{->}\NormalTok{ s)}
\NormalTok{review }\DataTypeTok{Prism'}\NormalTok{\{}\FunctionTok{..}\NormalTok{\} }\FunctionTok{=}\NormalTok{ inject }\FunctionTok{.} \DataTypeTok{Left}
\end{Highlighting}
\end{Shaded}

Like for lenses, prisms also admit a particularly elegant formulation for
\texttt{over}, which maps a function over the \texttt{a} in the \texttt{s} if it
exists:

\begin{Shaded}
\begin{Highlighting}[]
\OtherTok{over ::} \DataTypeTok{Lens'}\NormalTok{ s a  }\OtherTok{->}\NormalTok{ (a }\OtherTok{->}\NormalTok{ a) }\OtherTok{->}\NormalTok{ (s }\OtherTok{->}\NormalTok{ s)}
\NormalTok{over }\DataTypeTok{Lens'}\NormalTok{\{}\FunctionTok{..}\NormalTok{\}  f }\FunctionTok{=}\NormalTok{ inject }\FunctionTok{.}\NormalTok{ first f }\FunctionTok{.}\NormalTok{ match    }\CommentTok{-- instance Bifunctor (,)}

\OtherTok{over ::} \DataTypeTok{Prism'}\NormalTok{ s a }\OtherTok{->}\NormalTok{ (a }\OtherTok{->}\NormalTok{ a) }\OtherTok{->}\NormalTok{ (s }\OtherTok{->}\NormalTok{ s)}
\NormalTok{over }\DataTypeTok{Prism'}\NormalTok{\{}\FunctionTok{..}\NormalTok{\} f }\FunctionTok{=}\NormalTok{ inject }\FunctionTok{.}\NormalTok{ first f }\FunctionTok{.}\NormalTok{ match    }\CommentTok{-- instance Bifunctor Either}
\end{Highlighting}
\end{Shaded}

Neat, they're actually exactly identical! Who would have thought?

So we see now, similar to lenses, \textbf{every sum yields prisms}, and
\textbf{every prism witnesses one side of a sum}.

\hypertarget{prism-tour}{%
\subsection{Prism Tour}\label{prism-tour}}

Let's go back at our example prisms and see what sort of insight we can gain
from this perspective.

\begin{Shaded}
\begin{Highlighting}[]
\KeywordTok{data} \DataTypeTok{Shape} \FunctionTok{=} \DataTypeTok{Circle}  \DataTypeTok{Double}
           \FunctionTok{|} \DataTypeTok{RegPoly} \DataTypeTok{Natural} \DataTypeTok{Double}

\OtherTok{match ::} \DataTypeTok{Shape} \OtherTok{->} \DataTypeTok{Either} \DataTypeTok{Double}\NormalTok{ (}\DataTypeTok{Natural}\NormalTok{, }\DataTypeTok{Double}\NormalTok{)}
\NormalTok{match (}\DataTypeTok{Circle}\NormalTok{  r  ) }\FunctionTok{=} \DataTypeTok{Left}\NormalTok{ r}
\NormalTok{match (}\DataTypeTok{RegPoly}\NormalTok{ n s) }\FunctionTok{=} \DataTypeTok{Right}\NormalTok{ (n, s)}

\OtherTok{inject ::} \DataTypeTok{Either} \DataTypeTok{Double}\NormalTok{ (}\DataTypeTok{Natural}\NormalTok{, }\DataTypeTok{Double}\NormalTok{) }\OtherTok{->} \DataTypeTok{Shape}
\NormalTok{inject (}\DataTypeTok{Left}\NormalTok{   r    ) }\FunctionTok{=} \DataTypeTok{Circle}\NormalTok{  r}
\NormalTok{inject (}\DataTypeTok{Right}\NormalTok{ (n, s)) }\FunctionTok{=} \DataTypeTok{RegPoly}\NormalTok{ n s}
\end{Highlighting}
\end{Shaded}

Because \texttt{Shape} is a sum between \texttt{Double} and
\texttt{(Natural,\ Double)}, we get \emph{two prisms}:

\begin{Shaded}
\begin{Highlighting}[]
\CommentTok{-- Shape <~> Either Natural (Natural, Double)}

\OtherTok{_Circle ::} \DataTypeTok{Prism'} \DataTypeTok{Shape} \DataTypeTok{Natural}
\NormalTok{_Circle }\FunctionTok{=} \DataTypeTok{Prism'}\NormalTok{ \{ match  }\FunctionTok{=}\NormalTok{ \textbackslash{}}\KeywordTok{case} \DataTypeTok{Circle}\NormalTok{  r    }\OtherTok{->} \DataTypeTok{Left}\NormalTok{ r}
                                  \DataTypeTok{RegPoly}\NormalTok{ n s  }\OtherTok{->} \DataTypeTok{Right}\NormalTok{ (n, s)}
\NormalTok{                 , inject }\FunctionTok{=}\NormalTok{ \textbackslash{}}\KeywordTok{case} \DataTypeTok{Left}\NormalTok{   r     }\OtherTok{->} \DataTypeTok{Circle}\NormalTok{ r}
                                  \DataTypeTok{Right}\NormalTok{ (n, s) }\OtherTok{->} \DataTypeTok{RegPoly}\NormalTok{ n s}
\NormalTok{                 \}}

\OtherTok{_RegPoly ::} \DataTypeTok{Prism'} \DataTypeTok{Shape}\NormalTok{ (}\DataTypeTok{Natural}\NormalTok{, }\DataTypeTok{Double}\NormalTok{)}
\NormalTok{_RegPoly }\FunctionTok{=} \DataTypeTok{Prism'}\NormalTok{ \{ match  }\FunctionTok{=}\NormalTok{ \textbackslash{}}\KeywordTok{case} \DataTypeTok{Circle}\NormalTok{  r    }\OtherTok{->} \DataTypeTok{Right}\NormalTok{ r}
                                   \DataTypeTok{RegPoly}\NormalTok{ n s  }\OtherTok{->} \DataTypeTok{Left}\NormalTok{ (n, s)}
\NormalTok{                  , inject }\FunctionTok{=}\NormalTok{ \textbackslash{}}\KeywordTok{case} \DataTypeTok{Left}\NormalTok{  (n, s) }\OtherTok{->} \DataTypeTok{RegPoly}\NormalTok{ n s}
                                   \DataTypeTok{Right}\NormalTok{  r     }\OtherTok{->} \DataTypeTok{Circle}\NormalTok{ r}
\NormalTok{                  \}}
\end{Highlighting}
\end{Shaded}

And these are actually the typical prisms associated with an ADT. You actually
get exactly these if you use \texttt{makePrisms} from the \emph{lens} package.

What can we get out of our decomposition of \texttt{{[}a{]}} as a sum between
\texttt{()} and \texttt{NonEmpty\ a}? Let's look at them:

\begin{Shaded}
\begin{Highlighting}[]
\CommentTok{-- [a] <~> Either () (NonEmpty a)}

\OtherTok{_Nil ::} \DataTypeTok{Prism'}\NormalTok{ [a] ()}
\NormalTok{_Nil }\FunctionTok{=} \DataTypeTok{Prism'}
\NormalTok{    \{ match  }\FunctionTok{=}\NormalTok{ \textbackslash{}}\KeywordTok{case}\NormalTok{ []              }\OtherTok{->} \DataTypeTok{Left}\NormalTok{ ()}
\NormalTok{                     x}\FunctionTok{:}\NormalTok{xs            }\OtherTok{->} \DataTypeTok{Right}\NormalTok{ (x }\FunctionTok{:|}\NormalTok{ xs)}
\NormalTok{    , inject }\FunctionTok{=}\NormalTok{ \textbackslash{}}\KeywordTok{case} \DataTypeTok{Left}\NormalTok{ _          }\OtherTok{->}\NormalTok{ []}
                     \DataTypeTok{Right}\NormalTok{ (x }\FunctionTok{:|}\NormalTok{ xs) }\OtherTok{->}\NormalTok{ x}\FunctionTok{:}\NormalTok{xs}
\NormalTok{    \}}

\OtherTok{_Cons ::} \DataTypeTok{Prism'}\NormalTok{ [a] (a, [a])}
\NormalTok{_Cons }\FunctionTok{=} \DataTypeTok{Prism'}
\NormalTok{    \{ match  }\FunctionTok{=}\NormalTok{ \textbackslash{}}\KeywordTok{case}\NormalTok{ []              }\OtherTok{->} \DataTypeTok{Right}\NormalTok{ ()}
\NormalTok{                     x}\FunctionTok{:}\NormalTok{xs            }\OtherTok{->} \DataTypeTok{Left}\NormalTok{ (x }\FunctionTok{:|}\NormalTok{ xs)}
\NormalTok{    , inject }\FunctionTok{=}\NormalTok{ \textbackslash{}}\KeywordTok{case} \DataTypeTok{Left}\NormalTok{  (x }\FunctionTok{:|}\NormalTok{ xs) }\OtherTok{->}\NormalTok{ x}\FunctionTok{:}\NormalTok{xs}
                     \DataTypeTok{Right}\NormalTok{ _         }\OtherTok{->}\NormalTok{ []}
\NormalTok{    \}}
\end{Highlighting}
\end{Shaded}

We see a sort of pattern here. And, if we look deeper, we will see that
\emph{all prisms} correspond to some sort of ``constructor''.

After all, what do constructors give you? Two things:

\begin{enumerate}
\def\labelenumi{\arabic{enumi}.}
\tightlist
\item
  The ability to ``create'' a value. This corresponds to \texttt{review}, or
  \texttt{inject}
\item
  The ability to do ``case-analysis'' or check if a value was created using that
  constructor. This corresponds to \texttt{preview}, or \texttt{match}.
\end{enumerate}

The API of a ``constructor'' is pretty much exactly the Prism API. In fact, we
often use Prisms to simulate ``abstract'' constructors.

An \emph{abstract constructor} is exactly what our \emph{other} \texttt{{[}a{]}}
sum decomposition gives us! If we look at that isomorphism
\texttt{{[}a{]}\ \textless{}\textasciitilde{}\textgreater{}\ Either\ ()\ ({[}a{]},\ a)}
(the ``tail-and-last'' breakdown) and write out the prisms, we see that they
correspond to the abstract constructors \texttt{\_Nil} and \texttt{\_Snoc}:

\begin{Shaded}
\begin{Highlighting}[]
\CommentTok{-- [a] <~> Either () ([a], a)}

\OtherTok{_Nil ::} \DataTypeTok{Prism'}\NormalTok{ [a] ()}
\NormalTok{_Nil }\FunctionTok{=} \DataTypeTok{Prism'}\NormalTok{ \{ match  }\FunctionTok{=}\NormalTok{ \textbackslash{}xs }\OtherTok{->} \KeywordTok{if}\NormalTok{ null xs}
                                  \KeywordTok{then} \DataTypeTok{Left}\NormalTok{  ()}
                                  \KeywordTok{else} \DataTypeTok{Right}\NormalTok{ (init xs, last xs)}
\NormalTok{              , inject }\FunctionTok{=}\NormalTok{ \textbackslash{}}\KeywordTok{case} \DataTypeTok{Left}\NormalTok{ _        }\OtherTok{->}\NormalTok{ []}
                               \DataTypeTok{Right}\NormalTok{ (xs, x) }\OtherTok{->}\NormalTok{ xs }\FunctionTok{++}\NormalTok{ [x]}
\NormalTok{              \}}

\OtherTok{_Snoc ::} \DataTypeTok{Prism'}\NormalTok{ [a] ([a], a)}
\NormalTok{_Snoc }\FunctionTok{=} \DataTypeTok{Prism'}\NormalTok{ \{ match  }\FunctionTok{=}\NormalTok{ \textbackslash{}xs }\OtherTok{->} \KeywordTok{if}\NormalTok{ null xs}
                                   \KeywordTok{then} \DataTypeTok{Right}\NormalTok{ ()}
                                   \KeywordTok{else} \DataTypeTok{Left}\NormalTok{  (init xs, last xs)}
\NormalTok{               , inject }\FunctionTok{=}\NormalTok{ \textbackslash{}}\KeywordTok{case} \DataTypeTok{Left}\NormalTok{  (xs, x) }\OtherTok{->}\NormalTok{ xs }\FunctionTok{++}\NormalTok{ [x]}
                                \DataTypeTok{Right}\NormalTok{ _       }\OtherTok{->}\NormalTok{ []}
\NormalTok{               \}}
\end{Highlighting}
\end{Shaded}

\texttt{\_Snoc} (\texttt{mysteryPrism2}) is an abstract constructor for a list
that lets us:

\begin{enumerate}
\def\labelenumi{\arabic{enumi}.}
\tightlist
\item
  ``Construct'' an \texttt{{[}a{]}} given an original list \texttt{{[}a{]}} and
  an item to add to the end, \texttt{a}
\item
  ``Deconstruct'' an \texttt{{[}a{]}} into an initial run \texttt{{[}a{]}} and
  its last element \texttt{a} (as a pattern match that might ``fail'').
\end{enumerate}

And, our final sum,
\texttt{a\ \textless{}\textasciitilde{}\textgreater{}\ Either\ a\ Void}\ldots{}what
does that decomposition give us, conceptually?

\begin{Shaded}
\begin{Highlighting}[]
\CommentTok{-- a <~> Either a Void}

\OtherTok{identity ::} \DataTypeTok{Prism'}\NormalTok{ a a}
\NormalTok{identity }\FunctionTok{=} \DataTypeTok{Prism'}\NormalTok{ \{ match }\FunctionTok{=} \DataTypeTok{Left}
\NormalTok{                  , inject }\FunctionTok{=}\NormalTok{ \textbackslash{}}\KeywordTok{case}
                      \DataTypeTok{Left}\NormalTok{  x }\OtherTok{->}\NormalTok{ x}
                      \DataTypeTok{Right}\NormalTok{ v }\OtherTok{->}\NormalTok{ absurd v}
\NormalTok{                  \}}


\OtherTok{_Void ::} \DataTypeTok{Prism'}\NormalTok{ a }\DataTypeTok{Void}
\NormalTok{_Void }\FunctionTok{=} \DataTypeTok{Prism'}\NormalTok{ \{ match }\FunctionTok{=} \DataTypeTok{Right}
\NormalTok{               , inject }\FunctionTok{=}\NormalTok{ \textbackslash{}}\KeywordTok{case}
                   \DataTypeTok{Left}\NormalTok{  v }\OtherTok{->}\NormalTok{ absurd v}
                   \DataTypeTok{Right}\NormalTok{ x }\OtherTok{->}\NormalTok{ x}
\NormalTok{               \}}
\end{Highlighting}
\end{Shaded}

In lens-speak, \texttt{identity\ ::\ Prism\textquotesingle{}\ a\ a} tells us
that all \texttt{a}s have an \texttt{a} ``inside'' them (since \texttt{match}
always matches) and that you can construct an \texttt{a} with only an \texttt{a}
(whoa). In our ``sum'' perspective, however, it just witnesses that an
\texttt{a\ \textless{}\textasciitilde{}\textgreater{}\ Either\ a\ Void} sum.

In lens-speak, \texttt{\_Void\ ::\ Prism\textquotesingle{}\ a\ Void} tells us
that you can pattern match a \texttt{Void} out of any \texttt{a}\ldots{}but that
that pattern match will never fail. Furthermore, it tells us that if you have a
value of type \texttt{Void}, you can use the \texttt{\_Void} ``constructor'' to
make a value of any type \texttt{a}! We have
\texttt{review\ \_Void\ ::\ Void\ -\textgreater{}\ a}!

However, in our ``sum'' perspective, it is nothing more than the witness of the
fact that \texttt{a} is the sum of \texttt{a} and \texttt{Void}.

And finally, let's look at our deconstruction of \texttt{Int} and
\texttt{Reinfed\ (NotEqualTo\ 4)\ Int}. What prisms does this yield, and what
insight do we get?

\begin{Shaded}
\begin{Highlighting}[]
\OtherTok{only4 ::} \DataTypeTok{Prism'} \DataTypeTok{Int}\NormalTok{ ()}
\NormalTok{only4 }\FunctionTok{=} \DataTypeTok{Prism'}
\NormalTok{    \{ match  }\FunctionTok{=}\NormalTok{ \textbackslash{}n }\OtherTok{->} \KeywordTok{case}\NormalTok{ refineFail n }\KeywordTok{of}
                       \DataTypeTok{Nothing} \OtherTok{->} \DataTypeTok{Left}\NormalTok{ ()}
                       \DataTypeTok{Just}\NormalTok{ x  }\OtherTok{->} \DataTypeTok{Right}\NormalTok{ x}
\NormalTok{    , inject }\FunctionTok{=}\NormalTok{ \textbackslash{}}\KeywordTok{case} \DataTypeTok{Left}\NormalTok{  _ }\OtherTok{->} \DecValTok{4}
                     \DataTypeTok{Right}\NormalTok{ x }\OtherTok{->}\NormalTok{ unrefine x}
\NormalTok{    \}}

\OtherTok{refined4 ::} \DataTypeTok{Prism'} \DataTypeTok{Int} \DataTypeTok{Not4}
\NormalTok{refined4 }\FunctionTok{=} \DataTypeTok{Prism'}
\NormalTok{    \{ match  }\FunctionTok{=}\NormalTok{ \textbackslash{}n }\OtherTok{->} \KeywordTok{case}\NormalTok{ refineFail n }\KeywordTok{of}
                       \DataTypeTok{Nothing} \OtherTok{->} \DataTypeTok{Right}\NormalTok{ ()}
                       \DataTypeTok{Just}\NormalTok{ x  }\OtherTok{->} \DataTypeTok{Left}\NormalTok{ x}
\NormalTok{    , inject }\FunctionTok{=}\NormalTok{ \textbackslash{}}\KeywordTok{case} \DataTypeTok{Left}\NormalTok{  x }\OtherTok{->}\NormalTok{ unrefine x}
                     \DataTypeTok{Right}\NormalTok{ _ }\OtherTok{->} \DecValTok{4}
\NormalTok{    \}}
\end{Highlighting}
\end{Shaded}

The first prism, \texttt{only4}, is a prism that basically ``only matches'' on
the \texttt{Int} if it is \texttt{4}. We can use it to implement ``is equal to
four'', and ``get a 4''

\begin{Shaded}
\begin{Highlighting}[]
\CommentTok{-- | Checks if a value is 4}
\OtherTok{isEqualTo4 ::} \DataTypeTok{Int} \OtherTok{->} \DataTypeTok{Bool}
\NormalTok{isEqualTo4 }\FunctionTok{=}\NormalTok{ isJust }\FunctionTok{.}\NormalTok{ preview only4}

\CommentTok{-- | Just is '4'}
\OtherTok{four ::} \DataTypeTok{Int}
\NormalTok{four }\FunctionTok{=}\NormalTok{ review ()}
\end{Highlighting}
\end{Shaded}

The name \texttt{only4} is inspired by the \texttt{only} combinator from the
\emph{lens} library, which lets you provide a value you want to ``restrict''.

\begin{Shaded}
\begin{Highlighting}[]
\CommentTok{-- | From the lens library; lets you provide the value you want to "restrict"}
\OtherTok{only ::} \DataTypeTok{Eq}\NormalTok{ a }\OtherTok{=>}\NormalTok{ a }\OtherTok{->} \DataTypeTok{Prism'}\NormalTok{ a ()}

\OtherTok{only4 ::} \DataTypeTok{Prism'} \DataTypeTok{Int}\NormalTok{ ()}
\NormalTok{only4 }\FunctionTok{=}\NormalTok{ only }\DecValTok{4}
\end{Highlighting}
\end{Shaded}

The second prism, \texttt{refined4}, basically acts like a ``smart constructor''
for \texttt{Not4}, essentially \texttt{refineFail} and \texttt{unrefine}:

\begin{Shaded}
\begin{Highlighting}[]
\OtherTok{makeNot4 ::} \DataTypeTok{Int} \OtherTok{->} \DataTypeTok{Maybe} \DataTypeTok{Not4}
\NormalTok{makeNot4 }\FunctionTok{=}\NormalTok{ preview refined4}

\OtherTok{fromNot4 ::} \DataTypeTok{Not4} \OtherTok{->} \DataTypeTok{Int}
\NormalTok{fromNot4 }\FunctionTok{=}\NormalTok{ review refined4}
\end{Highlighting}
\end{Shaded}

\hypertarget{prism-or-not}{%
\subsection{Prism or Not}\label{prism-or-not}}

To me, however, one of the most useful things about this prism perspective is
that it helps me see what \emph{isn't} a prism.

For example, is it possible to have a prism into the \emph{head} of a list? That
is, is the following prism possible?

\begin{Shaded}
\begin{Highlighting}[]
\OtherTok{_head ::} \DataTypeTok{Prism'}\NormalTok{ [a] a           }\CommentTok{-- get the head of a list}
\end{Highlighting}
\end{Shaded}

If you think of a prism as just ``a lens that might fail'' (as it's often
taught), you might think yes. If you think of a prism as just ``a constructor
and deconstructor'', you might also think yes, since you can construct an
\texttt{{[}a{]}} with only a single \texttt{a}. You can definitely ``implement''
this prism using our ``too big'' representation, in terms of \texttt{preview}
and \texttt{review}. These viewpoints of prisms will fail you and lead you
astray.

However, if you think of it as witnessing a sum, you might see that this prism
isn't possible. There is no possible type \texttt{q} where \texttt{{[}a{]}} is a
sum of \texttt{a} and \texttt{q}. The isomorphism
\texttt{{[}a{]}\ \textless{}\textasciitilde{}\textgreater{}\ Either\ a\ q}
cannot be made for \emph{any} type \texttt{q}. There is no way to express
\texttt{{[}a{]}} as the sum of \texttt{a} and some other type. Try thinking of a
type \texttt{q} --- it's just not possible!

\hypertarget{exercises}{%
\section{Exercises}\label{exercises}}

\begin{enumerate}
\def\labelenumi{\arabic{enumi}.}
\tightlist
\item
  Is (a, Void) a decomp
\item
  what does the (Bool, a) \textless{}\textasciitilde{}\textgreater{} Either a a
  sum give us
\item
  Write a lens composition
\end{enumerate}

(Fun haskell challenge: the version of \texttt{match} I wrote there is
conceptually simple, but very inefficient. It traverses the input list three
times, uses two partial functions, and uses a \texttt{Bool}. Can you write a
\texttt{match} that does the same thing while traversing the input list only
once and using no partial functions or \texttt{Bool}s?)

Hi, thanks for reading! You can reach me via email at
\href{mailto:justin@jle.im}{\nolinkurl{justin@jle.im}}, or at twitter at
\href{https://twitter.com/mstk}{@mstk}! This post and all others are published
under the \href{https://creativecommons.org/licenses/by-nc-nd/3.0/}{CC-BY-NC-ND
3.0} license. Corrections and edits via pull request are welcome and encouraged
at \href{https://github.com/mstksg/inCode}{the source repository}.

If you feel inclined, or this post was particularly helpful for you, why not
consider \href{https://www.patreon.com/justinle/overview}{supporting me on
Patreon}, or a \href{bitcoin:3D7rmAYgbDnp4gp4rf22THsGt74fNucPDU}{BTC donation}?
:)

\end{document}
