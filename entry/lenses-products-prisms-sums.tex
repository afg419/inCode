\documentclass[]{article}
\usepackage{lmodern}
\usepackage{amssymb,amsmath}
\usepackage{ifxetex,ifluatex}
\usepackage{fixltx2e} % provides \textsubscript
\ifnum 0\ifxetex 1\fi\ifluatex 1\fi=0 % if pdftex
  \usepackage[T1]{fontenc}
  \usepackage[utf8]{inputenc}
\else % if luatex or xelatex
  \ifxetex
    \usepackage{mathspec}
    \usepackage{xltxtra,xunicode}
  \else
    \usepackage{fontspec}
  \fi
  \defaultfontfeatures{Mapping=tex-text,Scale=MatchLowercase}
  \newcommand{\euro}{€}
\fi
% use upquote if available, for straight quotes in verbatim environments
\IfFileExists{upquote.sty}{\usepackage{upquote}}{}
% use microtype if available
\IfFileExists{microtype.sty}{\usepackage{microtype}}{}
\usepackage[margin=1in]{geometry}
\usepackage{color}
\usepackage{fancyvrb}
\newcommand{\VerbBar}{|}
\newcommand{\VERB}{\Verb[commandchars=\\\{\}]}
\DefineVerbatimEnvironment{Highlighting}{Verbatim}{commandchars=\\\{\}}
% Add ',fontsize=\small' for more characters per line
\newenvironment{Shaded}{}{}
\newcommand{\AlertTok}[1]{\textcolor[rgb]{1.00,0.00,0.00}{\textbf{#1}}}
\newcommand{\AnnotationTok}[1]{\textcolor[rgb]{0.38,0.63,0.69}{\textbf{\textit{#1}}}}
\newcommand{\AttributeTok}[1]{\textcolor[rgb]{0.49,0.56,0.16}{#1}}
\newcommand{\BaseNTok}[1]{\textcolor[rgb]{0.25,0.63,0.44}{#1}}
\newcommand{\BuiltInTok}[1]{#1}
\newcommand{\CharTok}[1]{\textcolor[rgb]{0.25,0.44,0.63}{#1}}
\newcommand{\CommentTok}[1]{\textcolor[rgb]{0.38,0.63,0.69}{\textit{#1}}}
\newcommand{\CommentVarTok}[1]{\textcolor[rgb]{0.38,0.63,0.69}{\textbf{\textit{#1}}}}
\newcommand{\ConstantTok}[1]{\textcolor[rgb]{0.53,0.00,0.00}{#1}}
\newcommand{\ControlFlowTok}[1]{\textcolor[rgb]{0.00,0.44,0.13}{\textbf{#1}}}
\newcommand{\DataTypeTok}[1]{\textcolor[rgb]{0.56,0.13,0.00}{#1}}
\newcommand{\DecValTok}[1]{\textcolor[rgb]{0.25,0.63,0.44}{#1}}
\newcommand{\DocumentationTok}[1]{\textcolor[rgb]{0.73,0.13,0.13}{\textit{#1}}}
\newcommand{\ErrorTok}[1]{\textcolor[rgb]{1.00,0.00,0.00}{\textbf{#1}}}
\newcommand{\ExtensionTok}[1]{#1}
\newcommand{\FloatTok}[1]{\textcolor[rgb]{0.25,0.63,0.44}{#1}}
\newcommand{\FunctionTok}[1]{\textcolor[rgb]{0.02,0.16,0.49}{#1}}
\newcommand{\ImportTok}[1]{#1}
\newcommand{\InformationTok}[1]{\textcolor[rgb]{0.38,0.63,0.69}{\textbf{\textit{#1}}}}
\newcommand{\KeywordTok}[1]{\textcolor[rgb]{0.00,0.44,0.13}{\textbf{#1}}}
\newcommand{\NormalTok}[1]{#1}
\newcommand{\OperatorTok}[1]{\textcolor[rgb]{0.40,0.40,0.40}{#1}}
\newcommand{\OtherTok}[1]{\textcolor[rgb]{0.00,0.44,0.13}{#1}}
\newcommand{\PreprocessorTok}[1]{\textcolor[rgb]{0.74,0.48,0.00}{#1}}
\newcommand{\RegionMarkerTok}[1]{#1}
\newcommand{\SpecialCharTok}[1]{\textcolor[rgb]{0.25,0.44,0.63}{#1}}
\newcommand{\SpecialStringTok}[1]{\textcolor[rgb]{0.73,0.40,0.53}{#1}}
\newcommand{\StringTok}[1]{\textcolor[rgb]{0.25,0.44,0.63}{#1}}
\newcommand{\VariableTok}[1]{\textcolor[rgb]{0.10,0.09,0.49}{#1}}
\newcommand{\VerbatimStringTok}[1]{\textcolor[rgb]{0.25,0.44,0.63}{#1}}
\newcommand{\WarningTok}[1]{\textcolor[rgb]{0.38,0.63,0.69}{\textbf{\textit{#1}}}}
\usepackage{graphicx}
\makeatletter
\def\maxwidth{\ifdim\Gin@nat@width>\linewidth\linewidth\else\Gin@nat@width\fi}
\def\maxheight{\ifdim\Gin@nat@height>\textheight\textheight\else\Gin@nat@height\fi}
\makeatother
% Scale images if necessary, so that they will not overflow the page
% margins by default, and it is still possible to overwrite the defaults
% using explicit options in \includegraphics[width, height, ...]{}
\setkeys{Gin}{width=\maxwidth,height=\maxheight,keepaspectratio}
\ifxetex
  \usepackage[setpagesize=false, % page size defined by xetex
              unicode=false, % unicode breaks when used with xetex
              xetex]{hyperref}
\else
  \usepackage[unicode=true]{hyperref}
\fi
\hypersetup{breaklinks=true,
            bookmarks=true,
            pdfauthor={Justin Le},
            pdftitle={Lenses embody Products, Prisms embody Sums},
            colorlinks=true,
            citecolor=blue,
            urlcolor=blue,
            linkcolor=magenta,
            pdfborder={0 0 0}}
\urlstyle{same}  % don't use monospace font for urls
% Make links footnotes instead of hotlinks:
\renewcommand{\href}[2]{#2\footnote{\url{#1}}}
\setlength{\parindent}{0pt}
\setlength{\parskip}{6pt plus 2pt minus 1pt}
\setlength{\emergencystretch}{3em}  % prevent overfull lines
\setcounter{secnumdepth}{0}

\title{Lenses embody Products, Prisms embody Sums}
\author{Justin Le}

\begin{document}
\maketitle

\emph{Originally posted on
\textbf{\href{https://blog.jle.im/entry/lenses-products-prisms-sums.html}{in
Code}}.}

I've written about a variety of topics on this blog, but one thing I haven't
touched in too much detail is the topic of lenses and optics. A big part of this
is because there are already so many great resources on lenses, like the famous
(and my favorite) \href{https://artyom.me/lens-over-tea-1}{lenses over tea}
series.

This post won't be a ``lens tutorial'', but rather a dive into a (what I believe
is an) insightful perspective on lenses and prisms that I've heard repeated many
times, but not yet all gathered together into a single place. In particular, I'm
going to talk about the perspective of lenses and prisms as embodying the
essences of products and sums (respectively), and how that observation can help
you with a more ``practical'' understanding of lenses and prisms.

\hypertarget{products-and-sums}{%
\section{Products and Sums}\label{products-and-sums}}

In Haskell, ``products and sums'' can roughly be said to correspond to ``tuples
and \texttt{Either}''. If I have two types \texttt{A} and \texttt{B},
\texttt{(A,\ B)} is their ``product'' type. It's often called an ``anonymous
product'', because we can make one without having to give it a fancy name. It's
called a product type because \texttt{A} has
\includegraphics{https://latex.codecogs.com/png.latex?n} possible values and
\texttt{B} has \includegraphics{https://latex.codecogs.com/png.latex?m} possible
values, then \texttt{(A,\ B)} has
\includegraphics{https://latex.codecogs.com/png.latex?n\%20\%5Ctimes\%20m}
possible values\footnote{All of this is disregarding the notorious ``bottom''
  value that inhabits every type.}. And, \texttt{Either\ A\ B} is their
(anonymous) ``sum'' type. It's called a sum type because \texttt{Either\ A\ B}
has \includegraphics{https://latex.codecogs.com/png.latex?n\%20\%2B\%20m}
possible values. I won't go much deeper into this, but there are
\href{https://codewords.recurse.com/issues/three/algebra-and-calculus-of-algebraic-data-types}{many
useful tutorials already online} on this topic!

\hypertarget{be-productive}{%
\section{Be Productive!}\label{be-productive}}

It's easy to recognize \texttt{(Int,\ Double)} as a product between \texttt{Int}
and \texttt{Bool}. However, did you know that some types are secretly product
types in disguise?

For example, here's a classic example of a lensable data type

\begin{Shaded}
\begin{Highlighting}[]
\KeywordTok{data} \DataTypeTok{Person} \FunctionTok{=} \DataTypeTok{P}\NormalTok{ \{}\OtherTok{ _pName ::} \DataTypeTok{String}
\NormalTok{                ,}\OtherTok{ _pAge  ::} \DataTypeTok{Int}
\NormalTok{                \}}
\end{Highlighting}
\end{Shaded}

\texttt{Person} is an algebraic data type --- so-called because it is actually a
\emph{product} between a \texttt{String} and \texttt{Int}. \texttt{Person} is
\emph{isomorphic} to \texttt{(String,\ Int)}. I will be writing this as
\texttt{Person\ \textless{}\textasciitilde{}\textgreater{}\ (String,\ Int)}.

By \emph{isomorphic}, I mean that there are functions
\texttt{split\ ::\ Person\ -\textgreater{}\ (String,\ Int)} and
\texttt{unsplit\ ::\ (String,\ Int)\ -\textgreater{}\ Person} where
\texttt{unsplit\ .\ split\ =\ id} and \texttt{split\ .\ unsplit\ =\ id}.

In our case, we have:

\begin{Shaded}
\begin{Highlighting}[]
\OtherTok{split ::} \DataTypeTok{Person} \OtherTok{->}\NormalTok{ (}\DataTypeTok{String}\NormalTok{, }\DataTypeTok{Int}\NormalTok{)}
\NormalTok{split (}\DataTypeTok{P}\NormalTok{ n a) }\FunctionTok{=}\NormalTok{ (n, a)}

\OtherTok{unsplit ::}\NormalTok{ (}\DataTypeTok{String}\NormalTok{, }\DataTypeTok{Int}\NormalTok{) }\OtherTok{->} \DataTypeTok{Person}
\NormalTok{unsplit (n, a) }\FunctionTok{=} \DataTypeTok{P}\NormalTok{ n a}
\end{Highlighting}
\end{Shaded}

And we can verify that \texttt{unsplit\ .\ split} is \texttt{id}:

\begin{Shaded}
\begin{Highlighting}[]
\NormalTok{unsplit }\FunctionTok{.}\OtherTok{ split ::} \DataTypeTok{Person} \OtherTok{->} \DataTypeTok{Person}
\NormalTok{unsplit }\FunctionTok{.}\NormalTok{ split}
    \FunctionTok{=}\NormalTok{ \textbackslash{}x          }\OtherTok{->}\NormalTok{ unsplit (split x)        }\CommentTok{-- substitute definition of (.)}
    \FunctionTok{=}\NormalTok{ \textbackslash{}}\KeywordTok{case} \DataTypeTok{P}\NormalTok{ n a }\OtherTok{->}\NormalTok{ unsplit (split (}\DataTypeTok{P}\NormalTok{ n a))  }\CommentTok{-- expand patterns}
    \FunctionTok{=}\NormalTok{ \textbackslash{}}\KeywordTok{case} \DataTypeTok{P}\NormalTok{ n a }\OtherTok{->}\NormalTok{ unsplit (n, a)           }\CommentTok{-- substitute definition of split}
    \FunctionTok{=}\NormalTok{ \textbackslash{}}\KeywordTok{case} \DataTypeTok{P}\NormalTok{ n a }\OtherTok{->} \DataTypeTok{P}\NormalTok{ n a                    }\CommentTok{-- substitute definition of unsplit}
    \FunctionTok{=}\NormalTok{ \textbackslash{}x      }\OtherTok{->}\NormalTok{ x                            }\CommentTok{-- condense patterns}
    \FunctionTok{=}\NormalTok{ id                                      }\CommentTok{-- definition of id}
\end{Highlighting}
\end{Shaded}

And verification of \texttt{split\ .\ unsplit\ =\ id} is left as an exercise.

There are some other interesting products in Haskell, too. One such example is
\texttt{NonEmpty\ a} being a product between \texttt{a} (the head/first item)
and \texttt{{[}a{]}} (the tail/rest of the items). This means that
\texttt{NonEmpty\ a} is isomorphic to \texttt{(a,\ {[}a{]})} --- we have
\texttt{NonEmpty\ a\ \textless{}\textasciitilde{}\textgreater{}\ (a,\ {[}a{]})}!

Another curious product is the fact that every type \texttt{a} is a product
between \emph{itself} and unit, \texttt{()}. That is, every type \texttt{a} is
isomorphic to \texttt{(a,\ ())}. Freaky, right?

\begin{Shaded}
\begin{Highlighting}[]
\CommentTok{-- a <~> (a, ())}

\OtherTok{split ::}\NormalTok{ a }\OtherTok{->}\NormalTok{ (a, ())}
\NormalTok{split x }\FunctionTok{=}\NormalTok{ (x, ())}

\OtherTok{unsplit ::}\NormalTok{ (a, ()) }\OtherTok{->}\NormalTok{ a}
\NormalTok{unsplit (x, _) }\FunctionTok{=}\NormalTok{ x}
\end{Highlighting}
\end{Shaded}

One final interesting ``product in disguise'' is \texttt{Either\ a\ a}. ``But
wait,'' you say. ``That's a sum\ldots{}right??''

Well, yeah. But in addition, any \texttt{Either\ a\ a} is the product between
\texttt{Bool} and \texttt{a}. That is, \texttt{Either\ a\ a} is isomorphic to
\texttt{(Bool,\ a)}. The \texttt{Bool} tells you ``left or right?'' and the
\texttt{a} is the contents!

\begin{Shaded}
\begin{Highlighting}[]
\CommentTok{-- Either a a <~> (Bool, a)}

\OtherTok{split ::} \DataTypeTok{Either}\NormalTok{ a a }\OtherTok{->}\NormalTok{ (}\DataTypeTok{Bool}\NormalTok{, a)}
\NormalTok{split (}\DataTypeTok{Left}\NormalTok{  x) }\FunctionTok{=}\NormalTok{ (}\DataTypeTok{False}\NormalTok{, x)}
\NormalTok{split (}\DataTypeTok{Right}\NormalTok{ x) }\FunctionTok{=}\NormalTok{ (}\DataTypeTok{True}\NormalTok{ , x)}

\OtherTok{unsplit ::}\NormalTok{ (}\DataTypeTok{Bool}\NormalTok{, a) }\OtherTok{->} \DataTypeTok{Either}\NormalTok{ a a}
\NormalTok{unsplit (}\DataTypeTok{False}\NormalTok{, x) }\FunctionTok{=} \DataTypeTok{Left}\NormalTok{  x}
\NormalTok{unsplit (}\DataTypeTok{True}\NormalTok{ , x) }\FunctionTok{=} \DataTypeTok{Right}\NormalTok{ x}
\end{Highlighting}
\end{Shaded}

Proving that \texttt{unsplit\ .\ split\ =\ id}:

\begin{Shaded}
\begin{Highlighting}[]
\NormalTok{unsplit }\FunctionTok{.}\OtherTok{ split ::} \DataTypeTok{Either}\NormalTok{ a a }\OtherTok{->} \DataTypeTok{Either}\NormalTok{ a a}
\NormalTok{unsplit }\FunctionTok{.}\NormalTok{ split }\FunctionTok{=}
    \FunctionTok{=}\NormalTok{ \textbackslash{}x            }\OtherTok{->}\NormalTok{ unsplit (split x)          }\CommentTok{-- substitute definition of (.)}
      \CommentTok{-- trying case 1}
    \FunctionTok{=}\NormalTok{ \textbackslash{}}\KeywordTok{case} \DataTypeTok{Left}\NormalTok{  y }\OtherTok{->}\NormalTok{ unsplit (split (}\DataTypeTok{Left}\NormalTok{  y))  }\CommentTok{-- expand pattern for case 1}
    \FunctionTok{=}\NormalTok{ \textbackslash{}}\KeywordTok{case} \DataTypeTok{Left}\NormalTok{  y }\OtherTok{->}\NormalTok{ unsplit (}\DataTypeTok{False}\NormalTok{, y)         }\CommentTok{-- substitute definition of split}
    \FunctionTok{=}\NormalTok{ \textbackslash{}}\KeywordTok{case} \DataTypeTok{Left}\NormalTok{  y }\OtherTok{->} \DataTypeTok{Left}\NormalTok{  y                    }\CommentTok{-- substitute definition of unsplit}
    \FunctionTok{=}\NormalTok{ \textbackslash{}x            }\OtherTok{->}\NormalTok{ x                          }\CommentTok{-- condense pattern for case 1}
    \FunctionTok{=}\NormalTok{ id                                          }\CommentTok{-- definition of id}
      \CommentTok{-- trying case 2}
    \FunctionTok{=}\NormalTok{ \textbackslash{}}\KeywordTok{case} \DataTypeTok{Right}\NormalTok{ y }\OtherTok{->}\NormalTok{ unsplit (split (}\DataTypeTok{Right}\NormalTok{ y))  }\CommentTok{-- expand pattern for case 2}
    \FunctionTok{=}\NormalTok{ \textbackslash{}}\KeywordTok{case} \DataTypeTok{Right}\NormalTok{ y }\OtherTok{->}\NormalTok{ unsplit (}\DataTypeTok{True}\NormalTok{ , y)         }\CommentTok{-- substitute definition of split}
    \FunctionTok{=}\NormalTok{ \textbackslash{}}\KeywordTok{case} \DataTypeTok{Right}\NormalTok{ y }\OtherTok{->} \DataTypeTok{Right}\NormalTok{ y                    }\CommentTok{-- substitute definition of unsplit}
    \FunctionTok{=}\NormalTok{ \textbackslash{}x            }\OtherTok{->}\NormalTok{ x                          }\CommentTok{-- condense pattern for case 2}
    \FunctionTok{=}\NormalTok{ id                                          }\CommentTok{-- definition of id}
\end{Highlighting}
\end{Shaded}

And \texttt{split\ .\ unsplit\ =\ id} is again left as an exercise.

(\texttt{\textbackslash{}case} here is from the \emph{-XLambdaCase} extension)

\hypertarget{lenses}{%
\subsection{Lenses}\label{lenses}}

So, how do lenses come into the picture?

Let's review a bit. A \texttt{Lens\textquotesingle{}\ s\ a} is a way to
``access'' an \texttt{a} ``inside'' an \texttt{s}, respecting some laws.

A \texttt{Lens\textquotesingle{}\ s\ a} is a data type with the following API:

\begin{Shaded}
\begin{Highlighting}[]
\OtherTok{view ::} \DataTypeTok{Lens'}\NormalTok{ s a }\OtherTok{->}\NormalTok{ (s }\OtherTok{->}\NormalTok{ a)                }\CommentTok{-- get the 'a' from an 's'}
\OtherTok{set  ::} \DataTypeTok{Lens'}\NormalTok{ s a }\OtherTok{->}\NormalTok{ (a }\OtherTok{->}\NormalTok{ s }\OtherTok{->}\NormalTok{ s)           }\CommentTok{-- set the 'a' inside an 's'}
\end{Highlighting}
\end{Shaded}

respecting
\href{https://www.schoolofhaskell.com/school/to-infinity-and-beyond/pick-of-the-week/a-little-lens-starter-tutorial\#the-lens-laws-}{some
laws} --- get-put, put-get, and put-put.

Abstract mathematical laws are great and all, but I'm going to tell you a secret
that will render those laws obsolete.

At first, you might naively implement lenses like:

\begin{Shaded}
\begin{Highlighting}[]
\KeywordTok{data} \DataTypeTok{Lens'}\NormalTok{ s a }\FunctionTok{=} \DataTypeTok{Lens'}\NormalTok{ \{}\OtherTok{ view ::}\NormalTok{ s }\OtherTok{->}\NormalTok{ a}
\NormalTok{                       ,}\OtherTok{ set  ::}\NormalTok{ a }\OtherTok{->}\NormalTok{ s }\OtherTok{->}\NormalTok{ s}
\NormalTok{                       \}}
\end{Highlighting}
\end{Shaded}

But this is bad bad bad. That's because you can use this to represent lenses
that ``break the laws''. This representation is, to use the technical term,
``too big''. It allows more more values than are actual lenses.

So, here's the secret: A \texttt{Lens\textquotesingle{}\ s\ a} means that
\emph{\texttt{s} is a product between \texttt{a} and some type \texttt{x}}.

That means that if it is possible to represent \texttt{s} as some
\texttt{(a,\ q)} (that is,
\texttt{s\ \textless{}\textasciitilde{}\textgreater{}\ (a,\ q)}, \emph{then you
have a lens}! Lenses are nothing more than \textbf{descriptions of products}!

With that in mind, let's re-visit a saner definition of lenses based on the idea
that lenses embody descriptions of products:

\begin{Shaded}
\begin{Highlighting}[]
\KeywordTok{data} \DataTypeTok{Lens'}\NormalTok{ s a }\FunctionTok{=}\NormalTok{ forall q}\FunctionTok{.}
                 \DataTypeTok{Lens'}\NormalTok{ \{}\OtherTok{ split   ::}\NormalTok{ s }\OtherTok{->}\NormalTok{ (a, q)}
\NormalTok{                       ,}\OtherTok{ unsplit ::}\NormalTok{ (a, q) }\OtherTok{->}\NormalTok{ s}
\NormalTok{                       \}    }\CommentTok{-- ^ s <~> (a, q)}
\end{Highlighting}
\end{Shaded}

Now, if \texttt{split} and \texttt{join} form an isomorphism, \emph{this can
only represent valid lenses}!{[}\^{}big{]}

This type is technically also ``too big'' (you can write a value where
\texttt{split} and \texttt{unsplit} do not form an isomorphism), but I think, to
me, ``\texttt{split} and \texttt{join} must form an isomorphism'' is a much
clearer and natural law than get-put/put-get/put-put.

We can implement our necessary lens API as so:

\begin{Shaded}
\begin{Highlighting}[]
\OtherTok{view ::} \DataTypeTok{Lens'}\NormalTok{ s a }\OtherTok{->}\NormalTok{ (s }\OtherTok{->}\NormalTok{ a)}
\NormalTok{view (}\DataTypeTok{Lens'}\NormalTok{ spl _) }\FunctionTok{=}\NormalTok{ fst }\FunctionTok{.}\NormalTok{ spl}

\OtherTok{set ::} \DataTypeTok{Lens'}\NormalTok{ s a }\OtherTok{->}\NormalTok{ (a }\OtherTok{->}\NormalTok{ s }\OtherTok{->}\NormalTok{ s)}
\NormalTok{set (}\DataTypeTok{Lens'}\NormalTok{ spl unspl) x y }\FunctionTok{=} \KeywordTok{let}\NormalTok{ (_, q) }\FunctionTok{=}\NormalTok{ spl y}
                            \KeywordTok{in}\NormalTok{  unspl (x, q)        }\CommentTok{-- "replace" the `a`}
\end{Highlighting}
\end{Shaded}

TODO: record wildcards?

The surprising result is that \textbf{every product yields lenses} (one for
every item in the product), and \textbf{every lens witnesses a product}.

Let's take a look at our first product we talked about:

\begin{Shaded}
\begin{Highlighting}[]
\KeywordTok{data} \DataTypeTok{Person} \FunctionTok{=} \DataTypeTok{P}\NormalTok{ \{}\OtherTok{ _pName ::} \DataTypeTok{String}
\NormalTok{                ,}\OtherTok{ _pAge  ::} \DataTypeTok{Int}
\NormalTok{                \}}

\OtherTok{split ::} \DataTypeTok{Person} \OtherTok{->}\NormalTok{ (}\DataTypeTok{String}\NormalTok{, }\DataTypeTok{Int}\NormalTok{)}
\NormalTok{split (}\DataTypeTok{P}\NormalTok{ n a) }\FunctionTok{=}\NormalTok{ (n, a)}

\OtherTok{unsplit ::}\NormalTok{ (}\DataTypeTok{String}\NormalTok{, }\DataTypeTok{Int}\NormalTok{) }\OtherTok{->} \DataTypeTok{Person}
\NormalTok{unsplit (n, a) }\FunctionTok{=} \DataTypeTok{P}\NormalTok{ n a}
\end{Highlighting}
\end{Shaded}

Because \texttt{Person} is a product between \texttt{String} and \texttt{Int},
we get \emph{two lenses}: a \texttt{Lens\textquotesingle{}\ Person\ String} and
\texttt{Lens\textquotesingle{}\ Person\ Int}. \emph{Every product} gives us a
lens for every item in the product.

\begin{Shaded}
\begin{Highlighting}[]
\OtherTok{pName ::} \DataTypeTok{Lens'} \DataTypeTok{Person} \DataTypeTok{String}
\NormalTok{pName }\FunctionTok{=} \DataTypeTok{Lens'}\NormalTok{ \{ split   }\FunctionTok{=}\NormalTok{ \textbackslash{}(}\DataTypeTok{P}\NormalTok{ n a) }\OtherTok{->}\NormalTok{ (n, a)}
\NormalTok{              , unsplit }\FunctionTok{=}\NormalTok{ \textbackslash{}(n, a)  }\OtherTok{->} \DataTypeTok{P}\NormalTok{ n a}
\NormalTok{              \}}

\OtherTok{pAge ::} \DataTypeTok{Lens'} \DataTypeTok{Person} \DataTypeTok{String}
\NormalTok{pAge }\FunctionTok{=} \DataTypeTok{Lens'}\NormalTok{ \{ split   }\FunctionTok{=}\NormalTok{ \textbackslash{}(}\DataTypeTok{P}\NormalTok{ n a) }\OtherTok{->}\NormalTok{ (a, n)}
\NormalTok{             , unsplit }\FunctionTok{=}\NormalTok{ \textbackslash{}(a, n)  }\OtherTok{->} \DataTypeTok{P}\NormalTok{ n a}
\NormalTok{             \}}
\end{Highlighting}
\end{Shaded}

The inverse is true too. \textbf{Every lens witnesses a product}. The fact that
we have a lawful \texttt{pName\ ::\ Lens\textquotesingle{}\ Person\ String}
means that a \texttt{Person} \emph{must} be a product between \texttt{String}
and some other (hidden) type.

It can be insightful to look at products that we know and see what lenses those
correspond to.

For example, our
\texttt{NonEmpty\ a\ \textless{}\textasciitilde{}\textgreater{}\ (a,\ {[}a{]})}
product tells us that \texttt{NonEmpty\ a} has at least two lenses: a ``head''
lens \texttt{Lens\textquotesingle{}\ (NonEmpty\ a)\ a} and a ``tail'' lens
\texttt{Lens\textquotesingle{}\ (NonEmpty\ a)\ {[}a{]}}.

Our \texttt{a\ \textless{}\textasciitilde{}\textgreater{}\ (a,\ ())} product
gives some interesting insight. This tells us that we always have an
``identity'' lens \texttt{Lens\textquotesingle{}\ a\ a}, and a ``unit'' lens
\texttt{Lens\textquotesingle{}\ a\ ()}, for any \texttt{a}:

\begin{Shaded}
\begin{Highlighting}[]
\OtherTok{identity ::} \DataTypeTok{Lens'}\NormalTok{ a a}
\NormalTok{identity }\FunctionTok{=} \DataTypeTok{Lens'}\NormalTok{ \{ split   }\FunctionTok{=}\NormalTok{ \textbackslash{}x      }\OtherTok{->}\NormalTok{ (x, ())}
\NormalTok{                 , unsplit }\FunctionTok{=}\NormalTok{ \textbackslash{}(x, _) }\OtherTok{->}\NormalTok{ x}
\NormalTok{                 \}}

\OtherTok{unital ::} \DataTypeTok{Lens'}\NormalTok{ a ()}
\NormalTok{unital }\FunctionTok{=} \DataTypeTok{Lens'}\NormalTok{ \{ split   }\FunctionTok{=}\NormalTok{ \textbackslash{}x       }\OtherTok{->}\NormalTok{ ((), x)}
\NormalTok{               , unsplit }\FunctionTok{=}\NormalTok{ \textbackslash{}((), x) }\OtherTok{->}\NormalTok{ x}
\NormalTok{               \}}
\end{Highlighting}
\end{Shaded}

In the language of lens, \texttt{identity\ ::\ Lens\textquotesingle{}\ a\ a}
tells us that all \texttt{a}s have an \texttt{a} ``inside'' them. However, in
the language of products, this just tells us that \texttt{a} can be represented
as \texttt{(a,\ ())}. In the language of lens,
\texttt{unital\ ::\ Lens\textquotesingle{}\ a\ ()} tells us that all \texttt{a}s
have a \texttt{()} ``inside'' them. In the language of products, this just tells
us that \texttt{a\ \textless{}\textasciitilde{}\textgreater{}\ (a,\ ())}.

What insight does our
\texttt{Either\ a\ a\ \textless{}\textasciitilde{}\textgreater{}\ (Bool,\ a)}
product perspective give us? Well, let's write out their types and see what it
might suggest:

\begin{Shaded}
\begin{Highlighting}[]
\OtherTok{mysteryLens1 ::} \DataTypeTok{Lens'}\NormalTok{ (}\DataTypeTok{Either}\NormalTok{ a a) }\DataTypeTok{Bool}
\OtherTok{mysteryLens2 ::} \DataTypeTok{Lens'}\NormalTok{ (}\DataTypeTok{Either}\NormalTok{ a a) a}
\end{Highlighting}
\end{Shaded}

Looking at
\texttt{mysteryLens1\ ::\ Lens\textquotesingle{}\ (Either\ a\ a)\ Bool}, we are
saying that every \texttt{Either\ a\ a} has some \texttt{Bool} ``inside'' it.
From our knowledge of our product, we know that this \texttt{Bool} is really a
\emph{flag} for left-ness or right-ness. Getting the \texttt{Bool} is finding
out if we're in \texttt{Left} or \texttt{Right}, and flipping the \texttt{Bool}
``inside'' is really just swapping from \texttt{Left} to \texttt{Right}.

Looking at \texttt{mysteryLens2\ ::\ Lens\textquotesingle{}\ (Either\ a\ a)\ a},
we are saying that every \texttt{Either\ a\ a} has some \texttt{a} ``inside''
it. From what we know about the underlying product, the \texttt{a} is just the
``contained value'', \emph{ignoring} leftness or rightness. Getting the
\texttt{a} is getting the contained value and losing leftness/rightness, and
re-setting the \texttt{a} inside is modifying the contained value but preserving
leftness/rightness.

So that's really the essence of what a \texttt{Lens\textquotesingle{}} is. A
\texttt{Lens\textquotesingle{}\ s\ a} is the embodiment of the fact that
\texttt{s} can be represented as a product between \texttt{a} and something else
--- that \texttt{s\ \textless{}\textasciitilde{}\textgreater{}\ (a,\ q)}. All of
the lens laws just boil down to this. \textbf{Lenses embody products}.

\hypertarget{strength}{%
\subsection{Strength}\label{strength}}

You might have heard that lenses aren't \emph{actually} implemented as an ADT in
practice. In most modern lens libraries, they are represented as higher-order
functions. In my opinion, however, this is similarly boring :)

If we really do mean that a \texttt{Lens\textquotesingle{}\ s\ a} witnesses an
isomorphism between \texttt{s} and \texttt{(a,\ q)}\ldots{}then we can represent
that using a RankN function polymorphic over a \texttt{Profunctor}.

If you are not familiar with profunctors, there's a
\href{https://ocharles.org.uk/blog/guest-posts/2013-12-22-24-days-of-hackage-profunctors.html}{great
introduction} on Oliver Charles's Blog written by Tom Ellis. The typeclass
requires that we have ways to ``re-map'' both ends of the relationship:

\begin{Shaded}
\begin{Highlighting}[]
\OtherTok{lmap ::} \DataTypeTok{Profunctor}\NormalTok{ p }\OtherTok{=>}\NormalTok{ (a' }\OtherTok{->}\NormalTok{ a)  }\OtherTok{->}\NormalTok{ p a b }\OtherTok{->}\NormalTok{ p a' b}
\OtherTok{rmap ::} \DataTypeTok{Profunctor}\NormalTok{ p }\OtherTok{=>}\NormalTok{ (b  }\OtherTok{->}\NormalTok{ b') }\OtherTok{->}\NormalTok{ p a b }\OtherTok{->}\NormalTok{ p a  b'}
\end{Highlighting}
\end{Shaded}

\texttt{lmap} lets you ``pre-apply'' a function, essentially, to the ``input''
of a \texttt{p\ a\ b}, and \texttt{rmap} lets you ``post-apply'' a function to
the ``output'' of a \texttt{p\ a\ b}. These definitions are ``generic'' and
vague enough to allow for many different colorful \texttt{Profunctor} instances
that have all sorts of interesting implementations of ``pre-apply'' and
``post-apply''.

However, for this post, I want to propose a specific interpretation of the
\texttt{Profunctor} typeclass: a \texttt{p\ a\ b} represents a
\emph{relationship} from \texttt{a} to \texttt{b}. The \texttt{Profunctor}
typeclass is delightfully vague enough that all of the many different instances
and choices of \texttt{p} all give very unique and interesting perspectives on
what a ``relationship from \texttt{a} to \texttt{b}'' even means.

What does this mean for lenses?

Well, a \texttt{Lens\textquotesingle{}\ s\ a} can be thought of as a way to
``promote'' a \texttt{p\ a\ a} to a \texttt{p\ s\ s}. It lets you promote some
relationship on \texttt{a} (the internals) to become a full-fledged relationship
on \texttt{s}.

What does this mean from our ``lenses are products'' perspective? It means that
if we can turn \texttt{p\ a\ a} into \texttt{p\ (a,\ q)\ (a,\ q)} that
``ignores'' the \texttt{q}\ldots{}then we can just pre-map and post-map
\texttt{split} and \texttt{unsplit} to get a \texttt{p\ s\ s}:

\begin{Shaded}
\begin{Highlighting}[]
\NormalTok{rejoinProfunctor}
\OtherTok{    ::} \DataTypeTok{Profunctor}\NormalTok{ p}
    \OtherTok{=>}\NormalTok{ p (}\DataTypeTok{String}\NormalTok{, }\DataTypeTok{Int}\NormalTok{) (}\DataTypeTok{String}\NormalTok{, }\DataTypeTok{Int}\NormalTok{)}
    \OtherTok{->}\NormalTok{ p }\DataTypeTok{Person} \DataTypeTok{Person}
\NormalTok{rejoinProfunctor p }\FunctionTok{=}\NormalTok{ p''}
  \KeywordTok{where}
\OtherTok{    p'  ::}\NormalTok{ p }\DataTypeTok{Person}\NormalTok{ (}\DataTypeTok{String}\NormalTok{, }\DataTypeTok{Int}\NormalTok{)}
\NormalTok{    p' }\FunctionTok{=}\NormalTok{ lmap split p         }\CommentTok{-- pre-map split}
\OtherTok{    p'' ::}\NormalTok{ p }\DataTypeTok{Person} \DataTypeTok{Person}
\NormalTok{    p'' }\FunctionTok{=}\NormalTok{ rmap unsplit p          }\CommentTok{-- post-map unsplit}
\NormalTok{    split   (}\DataTypeTok{P}\NormalTok{ n a) }\FunctionTok{=}\NormalTok{ (n, a)}
\NormalTok{    unsplit (n, a)  }\FunctionTok{=} \DataTypeTok{P}\NormalTok{ n a}
\end{Highlighting}
\end{Shaded}

We can make a typeclass for profunctors that have such a ``promote to a
relationship over a tuple, ignoring the second field''. It's typically called
\texttt{Strong}:

\begin{Shaded}
\begin{Highlighting}[]
\KeywordTok{class} \DataTypeTok{Profunctor}\NormalTok{ p }\OtherTok{=>} \DataTypeTok{Strong}\NormalTok{ p }\KeywordTok{where}
\OtherTok{    first ::}\NormalTok{ p a b }\OtherTok{->}\NormalTok{ p (a, q) (b, q)}
\end{Highlighting}
\end{Shaded}

\texttt{first} takes a \texttt{p\ a\ b} (some relationship from \texttt{a} to
\texttt{b}) and promotes it to a \texttt{p\ (a,\ q)\ (b,\ q)} (some relationship
from \texttt{(a,\ q)} to \texttt{(b,\ q)} that \emph{ignores} the \texttt{q}).

With this in mind, we can now use any \texttt{Lens\textquotesingle{}\ s\ a} to
transform a \texttt{p\ a\ a} into a \texttt{p\ s\ s}:

\begin{Shaded}
\begin{Highlighting}[]
\NormalTok{fromLens'}
\OtherTok{    ::} \DataTypeTok{Strong}\NormalTok{ p}
    \OtherTok{=>} \DataTypeTok{Lens'}\NormalTok{ s a}
    \OtherTok{->}\NormalTok{ p a a}
    \OtherTok{->}\NormalTok{ p s s}
\NormalTok{fromLens' (}\DataTypeTok{Lens'}\NormalTok{ spl uns) p }\FunctionTok{=}\NormalTok{ rmap split (lmap unsplit (first p))}
\end{Highlighting}
\end{Shaded}

Hi, thanks for reading! You can reach me via email at
\href{mailto:justin@jle.im}{\nolinkurl{justin@jle.im}}, or at twitter at
\href{https://twitter.com/mstk}{@mstk}! This post and all others are published
under the \href{https://creativecommons.org/licenses/by-nc-nd/3.0/}{CC-BY-NC-ND
3.0} license. Corrections and edits via pull request are welcome and encouraged
at \href{https://github.com/mstksg/inCode}{the source repository}.

If you feel inclined, or this post was particularly helpful for you, why not
consider \href{https://www.patreon.com/justinle}{supporting me on Patreon}, or a
\href{bitcoin:3D7rmAYgbDnp4gp4rf22THsGt74fNucPDU}{BTC donation}? :)

\end{document}
