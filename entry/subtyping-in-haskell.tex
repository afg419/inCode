\documentclass[]{article}
\usepackage{lmodern}
\usepackage{amssymb,amsmath}
\usepackage{ifxetex,ifluatex}
\usepackage{fixltx2e} % provides \textsubscript
\ifnum 0\ifxetex 1\fi\ifluatex 1\fi=0 % if pdftex
  \usepackage[T1]{fontenc}
  \usepackage[utf8]{inputenc}
\else % if luatex or xelatex
  \ifxetex
    \usepackage{mathspec}
    \usepackage{xltxtra,xunicode}
  \else
    \usepackage{fontspec}
  \fi
  \defaultfontfeatures{Mapping=tex-text,Scale=MatchLowercase}
  \newcommand{\euro}{€}
\fi
% use upquote if available, for straight quotes in verbatim environments
\IfFileExists{upquote.sty}{\usepackage{upquote}}{}
% use microtype if available
\IfFileExists{microtype.sty}{\usepackage{microtype}}{}
\usepackage[margin=1in]{geometry}
\ifxetex
  \usepackage[setpagesize=false, % page size defined by xetex
              unicode=false, % unicode breaks when used with xetex
              xetex]{hyperref}
\else
  \usepackage[unicode=true]{hyperref}
\fi
\hypersetup{breaklinks=true,
            bookmarks=true,
            pdfauthor={},
            pdftitle={},
            colorlinks=true,
            citecolor=blue,
            urlcolor=blue,
            linkcolor=magenta,
            pdfborder={0 0 0}}
\urlstyle{same}  % don't use monospace font for urls
% Make links footnotes instead of hotlinks:
\renewcommand{\href}[2]{#2\footnote{\url{#1}}}
\setlength{\parindent}{0pt}
\setlength{\parskip}{6pt plus 2pt minus 1pt}
\setlength{\emergencystretch}{3em}  % prevent overfull lines
\setcounter{secnumdepth}{0}


\begin{document}

\% Subtyping in Haskell \% Justin Le

\emph{Originally posted on
\textbf{\href{https://blog.jle.im/entry/subtyping-in-haskell.html}{in Code}}.}

It is often said that Haskell does not have subtyping. While it is indeed true
that Haskell doesn't have \emph{ad-hoc} subtyping, you can achieve something
similar with Haskell, RankN types, and some key choice of data types or
typeclasses. And, in many situations, you can build programs around it!

As a simple example, let's redesign the API of the
\emph{Control.Monad.Trans.State} module from the \emph{transformers} package.
Here, they define

```haskell data StateT s m a = StateT (s -\textgreater{} m (a, s))

runStateT :: StateT s m a -\textgreater{} s -\textgreater{} m (a, s) evalStateT
:: StateT s m a -\textgreater{} s -\textgreater{} m a execStateT :: StateT s m a
-\textgreater{} s -\textgreater{} m s

get :: StateT s m s put :: s -\textgreater{} StateT s m () modify :: (s
-\textgreater{} s) -\textgreater{} StateT s m () state :: (s -\textgreater{} (a,
s)) -\textgreater{} StateT s m a

type State s = StateT s Identity

runState :: State s a -\textgreater{} s -\textgreater{} (a, s) evalState ::
State s a -\textgreater{} s -\textgreater{} a execState :: State s a
-\textgreater{} s -\textgreater{} s

```

\end{document}
