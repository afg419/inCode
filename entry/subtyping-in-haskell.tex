\documentclass[]{article}
\usepackage{lmodern}
\usepackage{amssymb,amsmath}
\usepackage{ifxetex,ifluatex}
\usepackage{fixltx2e} % provides \textsubscript
\ifnum 0\ifxetex 1\fi\ifluatex 1\fi=0 % if pdftex
  \usepackage[T1]{fontenc}
  \usepackage[utf8]{inputenc}
\else % if luatex or xelatex
  \ifxetex
    \usepackage{mathspec}
    \usepackage{xltxtra,xunicode}
  \else
    \usepackage{fontspec}
  \fi
  \defaultfontfeatures{Mapping=tex-text,Scale=MatchLowercase}
  \newcommand{\euro}{€}
\fi
% use upquote if available, for straight quotes in verbatim environments
\IfFileExists{upquote.sty}{\usepackage{upquote}}{}
% use microtype if available
\IfFileExists{microtype.sty}{\usepackage{microtype}}{}
\usepackage[margin=1in]{geometry}
\usepackage{color}
\usepackage{fancyvrb}
\newcommand{\VerbBar}{|}
\newcommand{\VERB}{\Verb[commandchars=\\\{\}]}
\DefineVerbatimEnvironment{Highlighting}{Verbatim}{commandchars=\\\{\}}
% Add ',fontsize=\small' for more characters per line
\newenvironment{Shaded}{}{}
\newcommand{\KeywordTok}[1]{\textcolor[rgb]{0.00,0.44,0.13}{\textbf{#1}}}
\newcommand{\DataTypeTok}[1]{\textcolor[rgb]{0.56,0.13,0.00}{#1}}
\newcommand{\DecValTok}[1]{\textcolor[rgb]{0.25,0.63,0.44}{#1}}
\newcommand{\BaseNTok}[1]{\textcolor[rgb]{0.25,0.63,0.44}{#1}}
\newcommand{\FloatTok}[1]{\textcolor[rgb]{0.25,0.63,0.44}{#1}}
\newcommand{\ConstantTok}[1]{\textcolor[rgb]{0.53,0.00,0.00}{#1}}
\newcommand{\CharTok}[1]{\textcolor[rgb]{0.25,0.44,0.63}{#1}}
\newcommand{\SpecialCharTok}[1]{\textcolor[rgb]{0.25,0.44,0.63}{#1}}
\newcommand{\StringTok}[1]{\textcolor[rgb]{0.25,0.44,0.63}{#1}}
\newcommand{\VerbatimStringTok}[1]{\textcolor[rgb]{0.25,0.44,0.63}{#1}}
\newcommand{\SpecialStringTok}[1]{\textcolor[rgb]{0.73,0.40,0.53}{#1}}
\newcommand{\ImportTok}[1]{#1}
\newcommand{\CommentTok}[1]{\textcolor[rgb]{0.38,0.63,0.69}{\textit{#1}}}
\newcommand{\DocumentationTok}[1]{\textcolor[rgb]{0.73,0.13,0.13}{\textit{#1}}}
\newcommand{\AnnotationTok}[1]{\textcolor[rgb]{0.38,0.63,0.69}{\textbf{\textit{#1}}}}
\newcommand{\CommentVarTok}[1]{\textcolor[rgb]{0.38,0.63,0.69}{\textbf{\textit{#1}}}}
\newcommand{\OtherTok}[1]{\textcolor[rgb]{0.00,0.44,0.13}{#1}}
\newcommand{\FunctionTok}[1]{\textcolor[rgb]{0.02,0.16,0.49}{#1}}
\newcommand{\VariableTok}[1]{\textcolor[rgb]{0.10,0.09,0.49}{#1}}
\newcommand{\ControlFlowTok}[1]{\textcolor[rgb]{0.00,0.44,0.13}{\textbf{#1}}}
\newcommand{\OperatorTok}[1]{\textcolor[rgb]{0.40,0.40,0.40}{#1}}
\newcommand{\BuiltInTok}[1]{#1}
\newcommand{\ExtensionTok}[1]{#1}
\newcommand{\PreprocessorTok}[1]{\textcolor[rgb]{0.74,0.48,0.00}{#1}}
\newcommand{\AttributeTok}[1]{\textcolor[rgb]{0.49,0.56,0.16}{#1}}
\newcommand{\RegionMarkerTok}[1]{#1}
\newcommand{\InformationTok}[1]{\textcolor[rgb]{0.38,0.63,0.69}{\textbf{\textit{#1}}}}
\newcommand{\WarningTok}[1]{\textcolor[rgb]{0.38,0.63,0.69}{\textbf{\textit{#1}}}}
\newcommand{\AlertTok}[1]{\textcolor[rgb]{1.00,0.00,0.00}{\textbf{#1}}}
\newcommand{\ErrorTok}[1]{\textcolor[rgb]{1.00,0.00,0.00}{\textbf{#1}}}
\newcommand{\NormalTok}[1]{#1}
\ifxetex
  \usepackage[setpagesize=false, % page size defined by xetex
              unicode=false, % unicode breaks when used with xetex
              xetex]{hyperref}
\else
  \usepackage[unicode=true]{hyperref}
\fi
\hypersetup{breaklinks=true,
            bookmarks=true,
            pdfauthor={Justin Le},
            pdftitle={Subtyping in Haskell},
            colorlinks=true,
            citecolor=blue,
            urlcolor=blue,
            linkcolor=magenta,
            pdfborder={0 0 0}}
\urlstyle{same}  % don't use monospace font for urls
% Make links footnotes instead of hotlinks:
\renewcommand{\href}[2]{#2\footnote{\url{#1}}}
\setlength{\parindent}{0pt}
\setlength{\parskip}{6pt plus 2pt minus 1pt}
\setlength{\emergencystretch}{3em}  % prevent overfull lines
\setcounter{secnumdepth}{0}

\title{Subtyping in Haskell}
\author{Justin Le}

\begin{document}
\maketitle

\emph{Originally posted on
\textbf{\href{https://blog.jle.im/entry/subtyping-in-haskell.html}{in Code}}.}

It is often said that Haskell does not have subtyping. While it is indeed true
that Haskell doesn't have \emph{ad-hoc} subtyping, you can achieve something
similar with Haskell, RankN types, and some key choice of data types or
typeclasses. And, in many situations, you can build programs around it!

As a simple example, let's redesign the API of the
\emph{Control.Monad.Trans.State} module from the \emph{transformers} package.
Here, they define

\begin{Shaded}
\begin{Highlighting}[]
\KeywordTok{data} \DataTypeTok{StateT}\NormalTok{ s m a }\FunctionTok{=} \DataTypeTok{StateT}\NormalTok{ (s }\OtherTok{->}\NormalTok{ m (a, s))}

\OtherTok{runStateT  ::} \DataTypeTok{StateT}\NormalTok{ s m a }\OtherTok{->}\NormalTok{ s }\OtherTok{->}\NormalTok{ m (a, s)}
\OtherTok{evalStateT ::} \DataTypeTok{StateT}\NormalTok{ s m a }\OtherTok{->}\NormalTok{ s }\OtherTok{->}\NormalTok{ m a}
\OtherTok{execStateT ::} \DataTypeTok{StateT}\NormalTok{ s m a }\OtherTok{->}\NormalTok{ s }\OtherTok{->}\NormalTok{ m s}

\OtherTok{get    ::}                  \DataTypeTok{StateT}\NormalTok{ s m s}
\OtherTok{put    ::}\NormalTok{ s             }\OtherTok{->} \DataTypeTok{StateT}\NormalTok{ s m ()}
\OtherTok{modify ::}\NormalTok{ (s }\OtherTok{->}\NormalTok{ s)      }\OtherTok{->} \DataTypeTok{StateT}\NormalTok{ s m ()}
\OtherTok{state  ::}\NormalTok{ (s }\OtherTok{->}\NormalTok{ (a, s)) }\OtherTok{->} \DataTypeTok{StateT}\NormalTok{ s m a}

\KeywordTok{type} \DataTypeTok{State}\NormalTok{ s }\FunctionTok{=} \DataTypeTok{StateT}\NormalTok{ s }\DataTypeTok{Identity}

\OtherTok{runState  ::} \DataTypeTok{State}\NormalTok{ s a  }\OtherTok{->}\NormalTok{ s }\OtherTok{->}\NormalTok{ (a, s)}
\OtherTok{evalState ::} \DataTypeTok{State}\NormalTok{ s a }\OtherTok{->}\NormalTok{ s }\OtherTok{->}\NormalTok{ a}
\OtherTok{execState ::} \DataTypeTok{State}\NormalTok{ s a }\OtherTok{->}\NormalTok{ s }\OtherTok{->}\NormalTok{ s}
\end{Highlighting}
\end{Shaded}

\end{document}
