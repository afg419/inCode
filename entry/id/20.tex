\documentclass[]{article}
\usepackage{lmodern}
\usepackage{amssymb,amsmath}
\usepackage{ifxetex,ifluatex}
\usepackage{fixltx2e} % provides \textsubscript
\ifnum 0\ifxetex 1\fi\ifluatex 1\fi=0 % if pdftex
  \usepackage[T1]{fontenc}
  \usepackage[utf8]{inputenc}
\else % if luatex or xelatex
  \ifxetex
    \usepackage{mathspec}
    \usepackage{xltxtra,xunicode}
  \else
    \usepackage{fontspec}
  \fi
  \defaultfontfeatures{Mapping=tex-text,Scale=MatchLowercase}
  \newcommand{\euro}{€}
\fi
% use upquote if available, for straight quotes in verbatim environments
\IfFileExists{upquote.sty}{\usepackage{upquote}}{}
% use microtype if available
\IfFileExists{microtype.sty}{\usepackage{microtype}}{}
\usepackage[margin=1in]{geometry}
\usepackage{color}
\usepackage{fancyvrb}
\newcommand{\VerbBar}{|}
\newcommand{\VERB}{\Verb[commandchars=\\\{\}]}
\DefineVerbatimEnvironment{Highlighting}{Verbatim}{commandchars=\\\{\}}
% Add ',fontsize=\small' for more characters per line
\newenvironment{Shaded}{}{}
\newcommand{\KeywordTok}[1]{\textcolor[rgb]{0.00,0.44,0.13}{\textbf{#1}}}
\newcommand{\DataTypeTok}[1]{\textcolor[rgb]{0.56,0.13,0.00}{#1}}
\newcommand{\DecValTok}[1]{\textcolor[rgb]{0.25,0.63,0.44}{#1}}
\newcommand{\BaseNTok}[1]{\textcolor[rgb]{0.25,0.63,0.44}{#1}}
\newcommand{\FloatTok}[1]{\textcolor[rgb]{0.25,0.63,0.44}{#1}}
\newcommand{\ConstantTok}[1]{\textcolor[rgb]{0.53,0.00,0.00}{#1}}
\newcommand{\CharTok}[1]{\textcolor[rgb]{0.25,0.44,0.63}{#1}}
\newcommand{\SpecialCharTok}[1]{\textcolor[rgb]{0.25,0.44,0.63}{#1}}
\newcommand{\StringTok}[1]{\textcolor[rgb]{0.25,0.44,0.63}{#1}}
\newcommand{\VerbatimStringTok}[1]{\textcolor[rgb]{0.25,0.44,0.63}{#1}}
\newcommand{\SpecialStringTok}[1]{\textcolor[rgb]{0.73,0.40,0.53}{#1}}
\newcommand{\ImportTok}[1]{#1}
\newcommand{\CommentTok}[1]{\textcolor[rgb]{0.38,0.63,0.69}{\textit{#1}}}
\newcommand{\DocumentationTok}[1]{\textcolor[rgb]{0.73,0.13,0.13}{\textit{#1}}}
\newcommand{\AnnotationTok}[1]{\textcolor[rgb]{0.38,0.63,0.69}{\textbf{\textit{#1}}}}
\newcommand{\CommentVarTok}[1]{\textcolor[rgb]{0.38,0.63,0.69}{\textbf{\textit{#1}}}}
\newcommand{\OtherTok}[1]{\textcolor[rgb]{0.00,0.44,0.13}{#1}}
\newcommand{\FunctionTok}[1]{\textcolor[rgb]{0.02,0.16,0.49}{#1}}
\newcommand{\VariableTok}[1]{\textcolor[rgb]{0.10,0.09,0.49}{#1}}
\newcommand{\ControlFlowTok}[1]{\textcolor[rgb]{0.00,0.44,0.13}{\textbf{#1}}}
\newcommand{\OperatorTok}[1]{\textcolor[rgb]{0.40,0.40,0.40}{#1}}
\newcommand{\BuiltInTok}[1]{#1}
\newcommand{\ExtensionTok}[1]{#1}
\newcommand{\PreprocessorTok}[1]{\textcolor[rgb]{0.74,0.48,0.00}{#1}}
\newcommand{\AttributeTok}[1]{\textcolor[rgb]{0.49,0.56,0.16}{#1}}
\newcommand{\RegionMarkerTok}[1]{#1}
\newcommand{\InformationTok}[1]{\textcolor[rgb]{0.38,0.63,0.69}{\textbf{\textit{#1}}}}
\newcommand{\WarningTok}[1]{\textcolor[rgb]{0.38,0.63,0.69}{\textbf{\textit{#1}}}}
\newcommand{\AlertTok}[1]{\textcolor[rgb]{1.00,0.00,0.00}{\textbf{#1}}}
\newcommand{\ErrorTok}[1]{\textcolor[rgb]{1.00,0.00,0.00}{\textbf{#1}}}
\newcommand{\NormalTok}[1]{#1}
\ifxetex
  \usepackage[setpagesize=false, % page size defined by xetex
              unicode=false, % unicode breaks when used with xetex
              xetex]{hyperref}
\else
  \usepackage[unicode=true]{hyperref}
\fi
\hypersetup{breaklinks=true,
            bookmarks=true,
            pdfauthor={Justin Le},
            pdftitle={Auto as Category, Applicative \& Arrow (Intro to Machines/Arrows Part 2)},
            colorlinks=true,
            citecolor=blue,
            urlcolor=blue,
            linkcolor=magenta,
            pdfborder={0 0 0}}
\urlstyle{same}  % don't use monospace font for urls
% Make links footnotes instead of hotlinks:
\renewcommand{\href}[2]{#2\footnote{\url{#1}}}
\setlength{\parindent}{0pt}
\setlength{\parskip}{6pt plus 2pt minus 1pt}
\setlength{\emergencystretch}{3em}  % prevent overfull lines
\setcounter{secnumdepth}{0}

\title{Auto as Category, Applicative \& Arrow (Intro to Machines/Arrows Part 2)}
\author{Justin Le}
\date{July 14, 2014}

\begin{document}
\maketitle

\emph{Originally posted on
\textbf{\href{https://blog.jle.im/entry/auto-as-category-applicative-arrow-intro-to-machines.html}{in
Code}}.}

Welcome back! It's been a while since the last post, admittedly; sorry! In this
post we'll be continuing on from
\href{http://blog.jle.im/entry/intro-to-machines-arrows-part-1-stream-and}{the
previous post}. In particular, we're going to be looking at the \texttt{Auto}
type as something that is a part of a pretty powerful pattern of abstraction,
and try to exploit it to write concise, expressive code using Auto composition
and proc notation. We'll also see first hands the principles of locally stateful
composition, and how much more expressive and safe it makes our code.

One motivating factor that I will eventually write about is that we can use this
to implement the semantics of Functional Reactive Programming, yay! But through
this, I hope you can actually see that it is useful for much, much more!

As always, feel free to leave a comment if you have any questions, or try to
find me on \href{https://twitter.com/mstk}{twitter}, or drop by the \#haskell
Freenode IRC channel! (I go by \emph{jle`})

Note that all of the code in this post can be downloaded (from
\href{https://github.com/mstksg/inCode/tree/master/code-samples/machines/Auto.hs}{Auto.hs}
for the last post, and
\href{https://github.com/mstksg/inCode/tree/master/code-samples/machines/Auto2.hs}{Auto2.hs}
for this post's new material) so you can play along on GHCi, or write your own
code using it the concepts and types here :) You can also run it
\href{https://www.fpcomplete.com/user/jle/machines}{online interactively}.

A fair warning: at times this post might feel a bit fragmented; but remember
that we really are just going to be exploring and getting very familiar with the
Auto type and building an intuition. Everything comes to a mini-climax at the
end, and a final satisfying one at the next post --- kind of like every Part 2
in every trilogy ever, you know? :)

\section{Recap}\label{recap}

We left off in our last post having looked at \texttt{Auto}:

\begin{Shaded}
\begin{Highlighting}[]
\CommentTok{-- source: https://github.com/mstksg/inCode/tree/master/code-samples/machines/Auto.hs#L12-L12}
\CommentTok{-- interactive: https://www.fpcomplete.com/user/jle/machines}
\KeywordTok{newtype} \DataTypeTok{Auto}\NormalTok{ a b }\FunctionTok{=} \DataTypeTok{ACons}\NormalTok{ \{}\OtherTok{ runAuto ::}\NormalTok{ a }\OtherTok{->}\NormalTok{ (b, }\DataTypeTok{Auto}\NormalTok{ a b) \}}
\end{Highlighting}
\end{Shaded}

which we saw as a stream that had an influencing input of type \texttt{a}, an
internal, opaque state (a function of the input and of the previous state), and
an output ``head'' of type \texttt{b} (also a function of the input and of the
previous state).

And we looked at
\href{https://github.com/mstksg/inCode/tree/master/code-samples/machines/Auto.hs\#L46-L54}{a
simple auto} which acted like a constantly incrementing stream, but where you
could reset the counter by passing in a \texttt{Just}.

Then we took another approach to looking at this --- we thought about Autos as
functions ``with state''. As in, \texttt{Auto\ a\ b} was like a function
\texttt{a\ -\textgreater{}\ b}, but which had an internal state that updated
every time it was called.

We saw this in an auto that
\href{https://github.com/mstksg/inCode/tree/master/code-samples/machines/Auto.hs\#L66-L73}{returns
the sum} of everything you have given it.

Autos are ``function-like things''\ldots{}they map or ``morph'' things of type
\texttt{a} to things of type \texttt{b} in some form, just like functions. It
looks like we stumbled onto some sort of design pattern. Wouldn't it be cool if
we could treat Autos the same way we treat functions? And reason about them the
same way, think about them using the same logic?

What is the \emph{essence} of function-like-ness?

\section{The Essence of
Function-like-ness}\label{the-essence-of-function-like-ness}

I'm going to introduce some formality and call things ``function-like-things''
\emph{morphisms} (with some laws). Sometimes you'll see them called ``arrows'',
but this is a slightly loaded word as there is an Arrow typeclass in Haskell
that doesn't exactly refer to the same thing.

Here is my claim: the ``essence'' of this functionlikeness is their ability to
\emph{compose} and ``chain'', and the \emph{nature} of that composition process.

That is, if you have a morphism \texttt{f} from \texttt{a} to \texttt{b}, and a
morphism \texttt{g} from \texttt{b} to \texttt{c}, then you can ``compose'' or
``chain'' them to get a morphism from \texttt{a} to \texttt{c}.

In Haskell we use the \texttt{(.)} operator for this --- to say more formally:

\begin{Shaded}
\begin{Highlighting}[]
\OtherTok{f     ::}\NormalTok{ morphism a b}
\OtherTok{g     ::}\NormalTok{ morphism b c}
\NormalTok{g }\FunctionTok{.}\OtherTok{ f ::}\NormalTok{ morphism a c}
\end{Highlighting}
\end{Shaded}

Some important aspects of the nature of this composition is that it must
``associate''. That means:

\begin{Shaded}
\begin{Highlighting}[]
\NormalTok{(h }\FunctionTok{.}\NormalTok{ g) }\FunctionTok{.}\NormalTok{ f }\FunctionTok{==}\NormalTok{ h }\FunctionTok{.}\NormalTok{ (g }\FunctionTok{.}\NormalTok{ f)}
\end{Highlighting}
\end{Shaded}

Composing the composition of \texttt{h} and \texttt{g} to \texttt{h} should be
the same as composing \texttt{h} with the composition of \texttt{g} and
\texttt{f}.

The final feature is that there must exist some ``identity'' morphism that
leaves other morphisms unchanged under composition:

\begin{Shaded}
\begin{Highlighting}[]
\NormalTok{id}\OtherTok{ ::} \DataTypeTok{Morphism}\NormalTok{ b b}

\NormalTok{id }\FunctionTok{.}\NormalTok{ f  }\FunctionTok{==}\NormalTok{ f}
\NormalTok{g  }\FunctionTok{.}\NormalTok{ id }\FunctionTok{==}\NormalTok{ g}
\end{Highlighting}
\end{Shaded}

It doesn't really matter what \texttt{id} literally ``does'' --- it only matters
that it leaves morphisms unchanged.

And\ldots{}that's it!

\subsection{\texorpdfstring{Functions are
``morphisms''}{Functions are morphisms}}\label{functions-are-morphisms}

We're just going to take a quick detour verify that normal functions satisfy
this new notion of ``function-likeness''\ldots{}so that we aren't crazy.

In Haskell, our functions are things of type \texttt{a\ -\textgreater{}\ b} ---
a morphism from \texttt{a} to \texttt{b}.

Our composition operator is the familiar \texttt{(.)} from Prelude. You can
prove all of the above laws yourself using the definition of \texttt{(.)}:

\begin{Shaded}
\begin{Highlighting}[]
\OtherTok{(.) ::}\NormalTok{ (b }\OtherTok{->}\NormalTok{ c) }\OtherTok{->}\NormalTok{ (a }\OtherTok{->}\NormalTok{ b) }\OtherTok{->}\NormalTok{ (a }\OtherTok{->}\NormalTok{ c)}
\NormalTok{g }\FunctionTok{.}\NormalTok{ f }\FunctionTok{=}\NormalTok{ \textbackslash{}x }\OtherTok{->}\NormalTok{ g (f x)}
\end{Highlighting}
\end{Shaded}

In practice, you can see associativity

\begin{Shaded}
\begin{Highlighting}[]
\NormalTok{ghci}\FunctionTok{>}\NormalTok{ ((}\FunctionTok{<} \DecValTok{20}\NormalTok{) }\FunctionTok{.}\NormalTok{ (}\FunctionTok{+}\DecValTok{4}\NormalTok{)) }\FunctionTok{.}\NormalTok{ (}\FunctionTok{^}\DecValTok{2}\NormalTok{) }\FunctionTok{$} \DecValTok{4}
\DataTypeTok{True}
\NormalTok{ghci}\FunctionTok{>}\NormalTok{ (}\FunctionTok{<} \DecValTok{20}\NormalTok{) }\FunctionTok{.}\NormalTok{ ((}\FunctionTok{+}\DecValTok{4}\NormalTok{) }\FunctionTok{.}\NormalTok{ (}\FunctionTok{^}\DecValTok{2}\NormalTok{)) }\FunctionTok{$} \DecValTok{4}
\DataTypeTok{True}
\end{Highlighting}
\end{Shaded}

The identity is just Prelude's \texttt{id}:

\begin{Shaded}
\begin{Highlighting}[]
\NormalTok{id}\OtherTok{ ::}\NormalTok{ a }\OtherTok{->}\NormalTok{ a}
\NormalTok{id x }\FunctionTok{=}\NormalTok{ x}
\end{Highlighting}
\end{Shaded}

\begin{Shaded}
\begin{Highlighting}[]
\NormalTok{ghci}\FunctionTok{>}\NormalTok{      (}\FunctionTok{*}\DecValTok{3}\NormalTok{) }\FunctionTok{$} \DecValTok{7}
\DecValTok{21}
\NormalTok{ghci}\FunctionTok{>}\NormalTok{ id }\FunctionTok{.}\NormalTok{ (}\FunctionTok{*}\DecValTok{3}\NormalTok{) }\FunctionTok{$} \DecValTok{7}
\DecValTok{21}
\NormalTok{ghci}\FunctionTok{>}\NormalTok{ (}\FunctionTok{*}\DecValTok{3}\NormalTok{) }\FunctionTok{.}\NormalTok{ id }\FunctionTok{$} \DecValTok{7}
\DecValTok{21}
\end{Highlighting}
\end{Shaded}

\begin{center}\rule{0.5\linewidth}{\linethickness}\end{center}

\textbf{Aside}

I mean it, you can prove it yourself if you are bored some time :) I'll start
you off with one of the identity laws:

\begin{Shaded}
\begin{Highlighting}[]
\NormalTok{g }\FunctionTok{.}\NormalTok{ id }\FunctionTok{=}\NormalTok{ \textbackslash{}x }\OtherTok{->}\NormalTok{ g (id x)     }\CommentTok{-- definition of (.)}
       \FunctionTok{=}\NormalTok{ \textbackslash{}x }\OtherTok{->}\NormalTok{ g x          }\CommentTok{-- definition of id}
       \FunctionTok{=}\NormalTok{ g                  }\CommentTok{-- eta reduction}
\end{Highlighting}
\end{Shaded}

\begin{center}\rule{0.5\linewidth}{\linethickness}\end{center}

So cool\ldots{}this intuition applies to our actual idea of functions, so we are
on a sane track!

\subsection{\texorpdfstring{Autos are
``morphisms''!}{Autos are morphisms!}}\label{autos-are-morphisms}

So we see that functions fit this idea. Let's jump back to what we were trying
to show in the first place --- that Autos fit this ``idea'' of function-like
things, or morphisms.

Let's say I had an \texttt{f\ ::\ Auto\ a\ b} and a \texttt{g\ ::\ Auto\ b\ c}.
I want to ``compose'' them --- \texttt{g\ .\ f}. To get an \texttt{Auto\ a\ c},
somehow. What would that even mean?

Well\ldots{}if we think of \texttt{f} like a stateful function that takes in an
\texttt{a} and pops out a \texttt{b}\ldots{}and \texttt{g} like a stateful
function that takes in a \texttt{b} and pops out a \texttt{c}\ldots{}We can
think of \texttt{g\ .\ f} as a stateful function that takes in an \texttt{a},
feeds it to \texttt{f}, gets the \texttt{b} that \texttt{f} pops out, feeds that
to \texttt{g}, and gets the final \texttt{c} out at the end.

Also, Autos spit out both the result (the \texttt{c}) and the ``updated
Auto''\ldots{}so the updated Auto of the composition can just be the composition
of the updated Autos!

Enough talk, let's code! We'll call our composition operator
\texttt{(\textasciitilde{}.\textasciitilde{})}.

\begin{Shaded}
\begin{Highlighting}[]
\CommentTok{-- source: https://github.com/mstksg/inCode/tree/master/code-samples/machines/Auto2.hs#L67-L70}
\CommentTok{-- interactive: https://www.fpcomplete.com/user/jle/machines}
\OtherTok{(~.~) ::} \DataTypeTok{Auto}\NormalTok{ b c }\OtherTok{->} \DataTypeTok{Auto}\NormalTok{ a b }\OtherTok{->} \DataTypeTok{Auto}\NormalTok{ a c}
\NormalTok{g }\FunctionTok{~.~}\NormalTok{ f }\FunctionTok{=} \DataTypeTok{ACons} \FunctionTok{$}\NormalTok{ \textbackslash{}x }\OtherTok{->} \KeywordTok{let}\NormalTok{ (y, f') }\FunctionTok{=}\NormalTok{ runAuto f x}
\NormalTok{                            (z, g') }\FunctionTok{=}\NormalTok{ runAuto g y}
                        \KeywordTok{in}\NormalTok{  (z, g' }\FunctionTok{~.~}\NormalTok{ f')}
\end{Highlighting}
\end{Shaded}

And\ldots{}that should be it! We run the input through first \texttt{f} then
\texttt{g}, collecting the ``modified \texttt{f} and \texttt{g}'', and returning
both of those at the end, composed.

Let's write a useful helper function so that we have more things to test this
out on:

\begin{Shaded}
\begin{Highlighting}[]
\CommentTok{-- source: https://github.com/mstksg/inCode/tree/master/code-samples/machines/Auto2.hs#L74-L75}
\CommentTok{-- interactive: https://www.fpcomplete.com/user/jle/machines}
\OtherTok{toAuto ::}\NormalTok{ (a }\OtherTok{->}\NormalTok{ b) }\OtherTok{->} \DataTypeTok{Auto}\NormalTok{ a b}
\NormalTok{toAuto f }\FunctionTok{=} \DataTypeTok{ACons} \FunctionTok{$}\NormalTok{ \textbackslash{}x }\OtherTok{->}\NormalTok{ (f x, toAuto f)}
\end{Highlighting}
\end{Shaded}

\texttt{toAuto} basically turns a function \texttt{a\ -\textgreater{}\ b} into a
stateless \texttt{Auto\ a\ b}.

Time to test these out!

\begin{Shaded}
\begin{Highlighting}[]
\NormalTok{ghci}\FunctionTok{>} \KeywordTok{let}\NormalTok{ doubleA  }\FunctionTok{=}\NormalTok{ toAuto (}\FunctionTok{*}\DecValTok{2}\NormalTok{)}\OtherTok{      ::} \DataTypeTok{Auto} \DataTypeTok{Int} \DataTypeTok{Int}
\NormalTok{ghci}\FunctionTok{>} \KeywordTok{let}\NormalTok{ succA    }\FunctionTok{=}\NormalTok{ toAuto (}\FunctionTok{+}\DecValTok{1}\NormalTok{)}\OtherTok{      ::} \DataTypeTok{Auto} \DataTypeTok{Int} \DataTypeTok{Int}

\NormalTok{ghci}\FunctionTok{>}\NormalTok{ testAuto_ doubleA [}\DecValTok{1}\FunctionTok{..}\DecValTok{10}\NormalTok{]}
\NormalTok{[}\DecValTok{2}\NormalTok{,}\DecValTok{4}\NormalTok{,}\DecValTok{6}\NormalTok{,}\DecValTok{8}\NormalTok{,}\DecValTok{10}\NormalTok{,}\DecValTok{12}\NormalTok{,}\DecValTok{14}\NormalTok{,}\DecValTok{16}\NormalTok{,}\DecValTok{18}\NormalTok{,}\DecValTok{20}\NormalTok{]}

\NormalTok{ghci}\FunctionTok{>}\NormalTok{ testAuto_ (succA }\FunctionTok{~.~}\NormalTok{ doubleA) [}\DecValTok{1}\FunctionTok{..}\DecValTok{10}\NormalTok{]}
\NormalTok{[}\DecValTok{3}\NormalTok{,}\DecValTok{5}\NormalTok{,}\DecValTok{7}\NormalTok{,}\DecValTok{9}\NormalTok{,}\DecValTok{11}\NormalTok{,}\DecValTok{13}\NormalTok{,}\DecValTok{15}\NormalTok{,}\DecValTok{17}\NormalTok{,}\DecValTok{19}\NormalTok{,}\DecValTok{21}\NormalTok{]}

\NormalTok{ghci}\FunctionTok{>}\NormalTok{ testAuto_ summer [}\DecValTok{5}\NormalTok{,}\DecValTok{1}\NormalTok{,}\DecValTok{9}\NormalTok{,}\DecValTok{2}\NormalTok{,}\FunctionTok{-}\DecValTok{3}\NormalTok{,}\DecValTok{4}\NormalTok{]}
\NormalTok{[}\DecValTok{5}\NormalTok{,}\DecValTok{6}\NormalTok{,}\DecValTok{15}\NormalTok{,}\DecValTok{17}\NormalTok{,}\DecValTok{14}\NormalTok{,}\DecValTok{18}\NormalTok{]}

\NormalTok{ghci}\FunctionTok{>}\NormalTok{ testAuto_ (doubleA }\FunctionTok{~.~}\NormalTok{ summer) [}\DecValTok{5}\NormalTok{,}\DecValTok{1}\NormalTok{,}\DecValTok{9}\NormalTok{,}\DecValTok{2}\NormalTok{,}\FunctionTok{-}\DecValTok{3}\NormalTok{,}\DecValTok{4}\NormalTok{]}
\NormalTok{[}\DecValTok{10}\NormalTok{,}\DecValTok{12}\NormalTok{,}\DecValTok{30}\NormalTok{,}\DecValTok{34}\NormalTok{,}\DecValTok{28}\NormalTok{,}\DecValTok{39}\NormalTok{]}

\NormalTok{ghci}\FunctionTok{>}\NormalTok{ testAuto_ settableAuto [}\DataTypeTok{Nothing}\NormalTok{,}\DataTypeTok{Nothing}\NormalTok{,}\DataTypeTok{Just}\NormalTok{ (}\FunctionTok{-}\DecValTok{3}\NormalTok{),}\DataTypeTok{Nothing}\NormalTok{,}\DataTypeTok{Nothing}\NormalTok{]}
\NormalTok{[}\DecValTok{1}\NormalTok{,}\DecValTok{2}\NormalTok{,}\FunctionTok{-}\DecValTok{3}\NormalTok{,}\FunctionTok{-}\DecValTok{2}\NormalTok{,}\FunctionTok{-}\DecValTok{1}\NormalTok{]}

\NormalTok{ghci}\FunctionTok{>}\NormalTok{ testAuto_ summer [}\DecValTok{1}\NormalTok{,}\DecValTok{2}\NormalTok{,}\FunctionTok{-}\DecValTok{3}\NormalTok{,}\FunctionTok{-}\DecValTok{2}\NormalTok{,}\FunctionTok{-}\DecValTok{1}\NormalTok{]}
\NormalTok{[}\DecValTok{1}\NormalTok{,}\DecValTok{3}\NormalTok{,}\DecValTok{0}\NormalTok{,}\FunctionTok{-}\DecValTok{2}\NormalTok{,}\FunctionTok{-}\DecValTok{3}\NormalTok{]}

\NormalTok{ghci}\FunctionTok{>}\NormalTok{ testAuto_ (summer }\FunctionTok{~.~}\NormalTok{ settableAuto)}
    \FunctionTok{|}\NormalTok{     [}\DataTypeTok{Nothing}\NormalTok{,}\DataTypeTok{Nothing}\NormalTok{,}\DataTypeTok{Just}\NormalTok{ (}\FunctionTok{-}\DecValTok{3}\NormalTok{),}\DataTypeTok{Nothing}\NormalTok{,}\DataTypeTok{Nothing}\NormalTok{]}
\NormalTok{[}\DecValTok{1}\NormalTok{,}\DecValTok{3}\NormalTok{,}\DecValTok{0}\NormalTok{,}\FunctionTok{-}\DecValTok{2}\NormalTok{,}\FunctionTok{-}\DecValTok{3}\NormalTok{]}
\end{Highlighting}
\end{Shaded}

And it looks like our Autos really can meaningfully compose!

Well, wait. We need one last thing: the identity Auto:

\begin{Shaded}
\begin{Highlighting}[]
\CommentTok{-- source: https://github.com/mstksg/inCode/tree/master/code-samples/machines/Auto2.hs#L78-L79}
\CommentTok{-- interactive: https://www.fpcomplete.com/user/jle/machines}
\OtherTok{idA ::} \DataTypeTok{Auto}\NormalTok{ a a}
\NormalTok{idA }\FunctionTok{=} \DataTypeTok{ACons} \FunctionTok{$}\NormalTok{ \textbackslash{}x }\OtherTok{->}\NormalTok{ (x, idA)}
\end{Highlighting}
\end{Shaded}

\begin{Shaded}
\begin{Highlighting}[]
\NormalTok{ghci}\FunctionTok{>}\NormalTok{ testAuto_ summer [}\DecValTok{5}\NormalTok{,}\DecValTok{1}\NormalTok{,}\DecValTok{9}\NormalTok{,}\DecValTok{2}\NormalTok{,}\FunctionTok{-}\DecValTok{3}\NormalTok{,}\DecValTok{4}\NormalTok{]}
\NormalTok{[}\DecValTok{10}\NormalTok{,}\DecValTok{12}\NormalTok{,}\DecValTok{30}\NormalTok{,}\DecValTok{34}\NormalTok{,}\DecValTok{28}\NormalTok{,}\DecValTok{39}\NormalTok{]}

\NormalTok{ghci}\FunctionTok{>}\NormalTok{ testAuto_ (idA }\FunctionTok{~.~}\NormalTok{ summer) [}\DecValTok{5}\NormalTok{,}\DecValTok{1}\NormalTok{,}\DecValTok{9}\NormalTok{,}\DecValTok{2}\NormalTok{,}\FunctionTok{-}\DecValTok{3}\NormalTok{,}\DecValTok{4}\NormalTok{]}
\NormalTok{[}\DecValTok{10}\NormalTok{,}\DecValTok{12}\NormalTok{,}\DecValTok{30}\NormalTok{,}\DecValTok{34}\NormalTok{,}\DecValTok{28}\NormalTok{,}\DecValTok{39}\NormalTok{]}

\NormalTok{ghci}\FunctionTok{>}\NormalTok{ testAuto_ (summer }\FunctionTok{~.~}\NormalTok{ idA) [}\DecValTok{5}\NormalTok{,}\DecValTok{1}\NormalTok{,}\DecValTok{9}\NormalTok{,}\DecValTok{2}\NormalTok{,}\FunctionTok{-}\DecValTok{3}\NormalTok{,}\DecValTok{4}\NormalTok{]}
\NormalTok{[}\DecValTok{10}\NormalTok{,}\DecValTok{12}\NormalTok{,}\DecValTok{30}\NormalTok{,}\DecValTok{34}\NormalTok{,}\DecValTok{28}\NormalTok{,}\DecValTok{39}\NormalTok{]}
\end{Highlighting}
\end{Shaded}

\subsection{Category}\label{category}

These concepts are actually formalized in the mathematical concept of a
``category'' --- things with objects and morphisms between them, following
certain properties (like the ones I mentioned earlier).

In Haskell, we often consider our objects as Haskell types; our usual morphisms
is the function arrow, \texttt{(-\textgreater{})}\footnote{Remember, we can
  write \texttt{a\ -\textgreater{}\ b} as \texttt{(-\textgreater{})\ a\ b}; like
  other operators, \texttt{(-\textgreater{})} can be used both infix and prefix.}
--- but in this case, it might be interesting to consider a different category
--- the category of Haskell types and morphisms \texttt{Auto\ a\ b}.

In Haskell, we have a typeclass that allows us to do generic operations on all
Categories --- so now we can basically treat \texttt{Auto}s ``as if'' they were
\texttt{(-\textgreater{})}. We can literally ``abstract over'' the idea of a
function. Neat, huh?

\begin{Shaded}
\begin{Highlighting}[]
\KeywordTok{class} \DataTypeTok{Category}\NormalTok{ r }\KeywordTok{where}
\OtherTok{    id  ::}\NormalTok{ r a a}
\OtherTok{    (.) ::}\NormalTok{ r b c }\OtherTok{->}\NormalTok{ r a b }\OtherTok{->}\NormalTok{ r a c}
\end{Highlighting}
\end{Shaded}

Basically, with Category, we can ``abstract over'' function composition and
\texttt{id}. That is, instead of \texttt{(.)} being only for composing normal
functions\ldots{}we can use to compose morphisms in any Category, like Auto! We
can also write ``generic code'' that works on \emph{all} morphisms --- not just
\texttt{(-\textgreater{})}! This is like having functions \texttt{mapM} and
\texttt{sequence} --- which work for \emph{all} Monads, not just IO or Maybe or
something. We can reason about Monads as things on their own, instead of just as
isolated instances.

Just be sure to use the correct imports so you don't have name clashes with the
Prelude operators:

\begin{Shaded}
\begin{Highlighting}[]
\KeywordTok{import }\DataTypeTok{Control.Category}
\KeywordTok{import }\DataTypeTok{Prelude} \KeywordTok{hiding}\NormalTok{ (id, (.))}
\end{Highlighting}
\end{Shaded}

First, let's write the \texttt{(-\textgreater{})} Category instance:

\begin{Shaded}
\begin{Highlighting}[]
\KeywordTok{instance} \DataTypeTok{Category}\NormalTok{ (}\OtherTok{->}\NormalTok{) }\KeywordTok{where}
\NormalTok{    id    }\FunctionTok{=}\NormalTok{ \textbackslash{}x }\OtherTok{->}\NormalTok{ x}
\NormalTok{    g }\FunctionTok{.}\NormalTok{ f }\FunctionTok{=}\NormalTok{ \textbackslash{}x }\OtherTok{->}\NormalTok{ g (f x)}
\end{Highlighting}
\end{Shaded}

And then our \texttt{Auto} Category instance:

\begin{Shaded}
\begin{Highlighting}[]
\CommentTok{-- source: https://github.com/mstksg/inCode/tree/master/code-samples/machines/Auto2.hs#L13-L18}
\CommentTok{-- interactive: https://www.fpcomplete.com/user/jle/machines}
\KeywordTok{instance} \DataTypeTok{Category} \DataTypeTok{Auto} \KeywordTok{where}
\NormalTok{    id    }\FunctionTok{=} \DataTypeTok{ACons} \FunctionTok{$}\NormalTok{ \textbackslash{}x }\OtherTok{->}\NormalTok{ (x, id)}
\NormalTok{    g }\FunctionTok{.}\NormalTok{ f }\FunctionTok{=} \DataTypeTok{ACons} \FunctionTok{$}\NormalTok{ \textbackslash{}x }\OtherTok{->}
              \KeywordTok{let}\NormalTok{ (y, f') }\FunctionTok{=}\NormalTok{ runAuto f x}
\NormalTok{                  (z, g') }\FunctionTok{=}\NormalTok{ runAuto g y}
              \KeywordTok{in}\NormalTok{  (z, g' }\FunctionTok{.}\NormalTok{ f')}
\end{Highlighting}
\end{Shaded}

And now\ldots{} we can work with both \texttt{(-\textgreater{})} and
\texttt{Auto} as if they were the ``same thing'' :)

\begin{Shaded}
\begin{Highlighting}[]
\CommentTok{-- source: https://github.com/mstksg/inCode/tree/master/code-samples/machines/Auto2.hs#L92-L93}
\CommentTok{-- interactive: https://www.fpcomplete.com/user/jle/machines}
\OtherTok{doTwice ::} \DataTypeTok{Category}\NormalTok{ r }\OtherTok{=>}\NormalTok{ r a a }\OtherTok{->}\NormalTok{ r a a}
\NormalTok{doTwice f }\FunctionTok{=}\NormalTok{ f }\FunctionTok{.}\NormalTok{ f}
\end{Highlighting}
\end{Shaded}

\begin{Shaded}
\begin{Highlighting}[]
\NormalTok{ghci}\FunctionTok{>}\NormalTok{ doTwice (}\FunctionTok{*}\DecValTok{2}\NormalTok{) }\DecValTok{5}
\DecValTok{20}
\NormalTok{ghci}\FunctionTok{>}\NormalTok{ testAuto_ (doTwice (toAuto (}\FunctionTok{*}\DecValTok{2}\NormalTok{))) [}\DecValTok{5}\NormalTok{]}
\NormalTok{[}\DecValTok{20}\NormalTok{]}
\NormalTok{ghci}\FunctionTok{>}\NormalTok{ testAuto_ (doTwice summer) [}\DecValTok{5}\NormalTok{,}\DecValTok{1}\NormalTok{,}\DecValTok{9}\NormalTok{,}\DecValTok{2}\NormalTok{,}\FunctionTok{-}\DecValTok{3}\NormalTok{,}\DecValTok{4}\NormalTok{]}
\NormalTok{[}\DecValTok{5}\NormalTok{,}\DecValTok{11}\NormalTok{,}\DecValTok{26}\NormalTok{,}\DecValTok{43}\NormalTok{,}\DecValTok{57}\NormalTok{,}\DecValTok{61}\NormalTok{]}
\NormalTok{ghci}\FunctionTok{>}\NormalTok{ take }\DecValTok{6} \FunctionTok{$}\NormalTok{ testAuto_ (doTwice summer) (repeat }\DecValTok{1}\NormalTok{)}
\NormalTok{[}\DecValTok{1}\NormalTok{,}\DecValTok{3}\NormalTok{,}\DecValTok{6}\NormalTok{,}\DecValTok{10}\NormalTok{,}\DecValTok{15}\NormalTok{,}\DecValTok{21}\NormalTok{]}
\end{Highlighting}
\end{Shaded}

The main cool thing here is that we can now abstract over the ``idea'' of
\texttt{id} and \texttt{(.)}, and now our Autos have basically not only captured
the idea of function-ness, but can now literally act like normal functions in
our code. I mentioned something similar in an earlier post
\href{http://blog.jle.im/entry/the-list-monadplus-practical-fun-with-monads-part}{in
MonadPlus} --- the ability to have a ``common language'' to talk and abstract
over many things is powerful not only for expressiveness but for reasoning and
maintainability.

\section{More Typeclasses!}\label{more-typeclasses}

Anyways, we love typeclasses so much. Let's get more familiar with our Auto type
and see what other useful typeclasses it can be :) Not only are these good
exercises for understanding our type, we'll also be using these instances later!

\subsection{Functor}\label{functor}

Functor is always a fun typeclass! One use case of Functor is as a ``producer
transformer'' --- the \texttt{f\ a} is some ``producer of \texttt{a}''.
\texttt{IO\ a} is a computation that produces an \texttt{a};
\texttt{Reader\ r\ a} produces an \texttt{a} when given an \texttt{r}.

So if you have \texttt{f\ a}, we have a handy function \texttt{fmap}:

\begin{Shaded}
\begin{Highlighting}[]
\NormalTok{fmap}\OtherTok{ ::} \DataTypeTok{Functor}\NormalTok{ f }\OtherTok{=>}\NormalTok{ (a }\OtherTok{->}\NormalTok{ b) }\OtherTok{->}\NormalTok{ f a }\OtherTok{->}\NormalTok{ f b}
\end{Highlighting}
\end{Shaded}

Which says, ``If you have an \texttt{a\ -\textgreater{}\ b}, I can turn a
producer of \texttt{a}'s into a producer of \texttt{b}'s.''

There are some laws associated with fmap --- most importantly that
\texttt{fmap\ id\ =\ id}.

Can we turn \texttt{Auto} into a Functor?

\ldots{}well, no, we can't. Because
\texttt{fmap\ ::\ (a\ -\textgreater{}\ b)\ -\textgreater{}\ Auto\ a\ -\textgreater{}\ Auto\ b}
makes no sense\ldots{}Auto takes two type parameters, not one.

But we \emph{can} think of \texttt{Auto\ r\ a} as a ``producer of \texttt{a}''s,
when used with \texttt{runAuto}. Our Functor is \texttt{Auto\ r}:

\begin{Shaded}
\begin{Highlighting}[]
\NormalTok{fmap}\OtherTok{ ::}\NormalTok{ (a }\OtherTok{->}\NormalTok{ b) }\OtherTok{->} \DataTypeTok{Auto}\NormalTok{ r a }\OtherTok{->} \DataTypeTok{Auto}\NormalTok{ r b}
\end{Highlighting}
\end{Shaded}

Which says, ``Give me any \texttt{a\ -\textgreater{}\ b}, and I'll take an Auto
that takes an \texttt{r} and returns an \texttt{a}\ldots{}and give you an Auto
that takes an \texttt{r} and returns a \texttt{b}''.

How can I do that?

Well\ldots{}for one\ldots{}I can just ``run'' the Auto you give me to get the
\texttt{a}\ldots{}and then apply the function to that \texttt{a} to get the
\texttt{b}!

For example, if I fmapped \texttt{show} onto \texttt{summer} --- if
\texttt{summer} was going to output a 1, it will now output a \texttt{"1"}. It
turns an \texttt{Auto\ Int\ Int} into an \texttt{Auto\ Int\ String}!

\begin{Shaded}
\begin{Highlighting}[]
\CommentTok{-- source: https://github.com/mstksg/inCode/tree/master/code-samples/machines/Auto2.hs#L20-L23}
\CommentTok{-- interactive: https://www.fpcomplete.com/user/jle/machines}
\KeywordTok{instance} \DataTypeTok{Functor}\NormalTok{ (}\DataTypeTok{Auto}\NormalTok{ r) }\KeywordTok{where}
\NormalTok{    fmap f a }\FunctionTok{=} \DataTypeTok{ACons} \FunctionTok{$}\NormalTok{ \textbackslash{}x }\OtherTok{->}
                 \KeywordTok{let}\NormalTok{ (y, a') }\FunctionTok{=}\NormalTok{ runAuto a x}
                 \KeywordTok{in}\NormalTok{  (f y, fmap f a')}
\end{Highlighting}
\end{Shaded}

\begin{Shaded}
\begin{Highlighting}[]
\NormalTok{ghci}\FunctionTok{>}\NormalTok{ testAuto_ (fmap show summer) [}\DecValTok{5}\NormalTok{,}\DecValTok{1}\NormalTok{,}\DecValTok{9}\NormalTok{,}\DecValTok{2}\NormalTok{,}\FunctionTok{-}\DecValTok{3}\NormalTok{,}\DecValTok{4}\NormalTok{]}
\NormalTok{[}\StringTok{"5"}\NormalTok{,}\StringTok{"6"}\NormalTok{,}\StringTok{"15"}\NormalTok{,}\StringTok{"17"}\NormalTok{,}\StringTok{"14"}\NormalTok{,}\StringTok{"18"}\NormalTok{]}
\end{Highlighting}
\end{Shaded}

Functor, check!

What's next?

\begin{center}\rule{0.5\linewidth}{\linethickness}\end{center}

\textbf{Aside}

If you ever have time, try doing some research on the
\href{https://ocharles.org.uk/blog/guest-posts/2013-12-21-24-days-of-hackage-contravariant.html}{Contravariant}
Functors and
\href{https://ocharles.org.uk/blog/guest-posts/2013-12-22-24-days-of-hackage-profunctors.html}{Profunctors}.
Can you make \texttt{Auto} or \texttt{Auto\ r} either one of those? Which ones?
If not all of them, why not?

\begin{center}\rule{0.5\linewidth}{\linethickness}\end{center}

\subsection{Applicative}\label{applicative}

Everyone knows that the ``cool'', \emph{hip} typeclasses are the classic trio,
\href{http://adit.io/posts/2013-04-17-functors,_applicatives,_and_monads_in_pictures.html}{Functor,
Applicative, Monad}. Let's just move on right along and go for Applicative.

If we continue the same sort of pattern that we did with Functor (some Functors
being producers-kinda), Applicative gives you two things: the ability to
``create a new `producer'\,'' producing a given value, and the ability to take
something that produces a function and something that produces a value and
squash it into something that produces the application of the two.

This stuff\ldots{}is really better said in types.

\begin{Shaded}
\begin{Highlighting}[]
\KeywordTok{class} \DataTypeTok{Applicative}\NormalTok{ f }\KeywordTok{where}
\OtherTok{    pure  ::}\NormalTok{ a }\OtherTok{->}\NormalTok{ f a}
\OtherTok{    (<*>) ::}\NormalTok{ f (a }\OtherTok{->}\NormalTok{ b) }\OtherTok{->}\NormalTok{ f a }\OtherTok{->}\NormalTok{ f b}
\end{Highlighting}
\end{Shaded}

In \texttt{pure}, give me an \texttt{a} and I'll give you a ``producer'' of that
very \texttt{a}. In \texttt{(\textless{}*\textgreater{})}, give me a producer of
\texttt{a\ -\textgreater{}\ b} and a producer of \texttt{a} and I'll give you a
producer of \texttt{b}.

We can pretty much use this to write our Applicative instance for
\texttt{Auto\ r}.

\begin{Shaded}
\begin{Highlighting}[]
\CommentTok{-- source: https://github.com/mstksg/inCode/tree/master/code-samples/machines/Auto2.hs#L25-L30}
\CommentTok{-- interactive: https://www.fpcomplete.com/user/jle/machines}
\KeywordTok{instance} \DataTypeTok{Applicative}\NormalTok{ (}\DataTypeTok{Auto}\NormalTok{ r) }\KeywordTok{where}
\NormalTok{    pure y    }\FunctionTok{=} \DataTypeTok{ACons} \FunctionTok{$}\NormalTok{ \textbackslash{}_ }\OtherTok{->}\NormalTok{ (y, pure y)}
\NormalTok{    af }\FunctionTok{<*>}\NormalTok{ ay }\FunctionTok{=} \DataTypeTok{ACons} \FunctionTok{$}\NormalTok{ \textbackslash{}x }\OtherTok{->}
                  \KeywordTok{let}\NormalTok{ (f, af') }\FunctionTok{=}\NormalTok{ runAuto af x}
\NormalTok{                      (y, ay') }\FunctionTok{=}\NormalTok{ runAuto ay x}
                  \KeywordTok{in}\NormalTok{  (f y, af' }\FunctionTok{<*>}\NormalTok{ ay')}
\end{Highlighting}
\end{Shaded}

Note that \texttt{pure} gives us a ``constant Auto'' --- an \texttt{Auto} that
ignores its input and always just produces the same thing.

The useful thing about Applicative is that it gives us \texttt{liftA2}, which
allows us to apply a function ``over Applicative''s.

\begin{Shaded}
\begin{Highlighting}[]
\OtherTok{liftA2 ::}\NormalTok{ (a }\OtherTok{->}\NormalTok{ b }\OtherTok{->}\NormalTok{ c) }\OtherTok{->} \DataTypeTok{Auto}\NormalTok{ r a }\OtherTok{->} \DataTypeTok{Auto}\NormalTok{ r b }\OtherTok{->} \DataTypeTok{Auto}\NormalTok{ r c}
\end{Highlighting}
\end{Shaded}

That is, it ``feeds in'' the \texttt{r} input to \emph{both} the
\texttt{Auto\ r\ a} and the \texttt{Auto\ r\ b}, applies the function to both of
the results, and the returns the result.

\begin{Shaded}
\begin{Highlighting}[]
\NormalTok{ghci}\FunctionTok{>}\NormalTok{ testAuto_ summer [}\DecValTok{5}\NormalTok{,}\DecValTok{1}\NormalTok{,}\DecValTok{9}\NormalTok{,}\DecValTok{2}\NormalTok{,}\FunctionTok{-}\DecValTok{3}\NormalTok{,}\DecValTok{4}\NormalTok{]}
\NormalTok{[}\DecValTok{5}\NormalTok{,}\DecValTok{6}\NormalTok{,}\DecValTok{15}\NormalTok{,}\DecValTok{17}\NormalTok{,}\DecValTok{14}\NormalTok{,}\DecValTok{18}\NormalTok{]}

\NormalTok{ghci}\FunctionTok{>} \KeywordTok{let}\NormalTok{ addSumDoub }\FunctionTok{=}\NormalTok{ liftA2 (}\FunctionTok{+}\NormalTok{) doubleA summer}
\NormalTok{ghci}\FunctionTok{>}\NormalTok{ testAuto_ addSumDoub [}\DecValTok{5}\NormalTok{,}\DecValTok{1}\NormalTok{,}\DecValTok{9}\NormalTok{,}\DecValTok{2}\NormalTok{,}\FunctionTok{-}\DecValTok{3}\NormalTok{,}\DecValTok{4}\NormalTok{]}
\NormalTok{[}\DecValTok{15}\NormalTok{,}\DecValTok{8}\NormalTok{,}\DecValTok{33}\NormalTok{,}\DecValTok{21}\NormalTok{,}\DecValTok{8}\NormalTok{,}\DecValTok{26}\NormalTok{]}
\CommentTok{-- [5 + 10, 6 + 2, 15 + 18, 17 + 4, 14 - 6, 18 + 8]}
\end{Highlighting}
\end{Shaded}

Hopefully by now you've seen enough usage of the \texttt{Auto} type and writing
\texttt{Auto} combinators that do useful things that you are now Auto experts :)
Feel free to press pause here, because we're going to ramp up to some more
unfamiliar abstractions. If you don't understand some of the examples above,
feel free to tinker with them on your own until you are comfortable. And as
always, if you have any questions, feel free to leave a comment or drop by the
freenode \#haskell channel.

Okay, now on to\ldots{}

\subsection{Monad}\label{monad}

Sykes!! We're not going to make a Monad instance :) Even though it is possible,
a Monad instance for \texttt{Auto} is remarkably useless. We won't be using a
monadic interface when working with Auto, so forget about it! What are Monads,
anyway?

Take \emph{that},
\href{http://www.reddit.com/r/programming/comments/25yent/i_like_haskell_because_it_lets_me_live_inside_my/chmj8al}{tomtomtom7}!
:D

\section{Arrow}\label{arrow}

Okay. As it turns out, \texttt{Category} by itself is nice, but for many of the
things we will be playing with function composition, it's just not going to cut
it.

As it turns out, we can actually sort of take advantage of a ``do
notation-like'' syntactical sugar construct to perform complex compositions. But
in order to do that, we first need to be able to ``side chain'' compositions.
That is, split off values, perform different compositions on both forks, and
recombine them. We require sort of basic set of combinators on top of our
Category instance.

The Arrow typeclass was invented for just this --- a grab-bag of combinators
that allow such side-chaining, forking, merging behavior.

\begin{Shaded}
\begin{Highlighting}[]
\KeywordTok{class} \DataTypeTok{Category}\NormalTok{ r }\OtherTok{=>} \DataTypeTok{Arrow}\NormalTok{ r }\KeywordTok{where}
\OtherTok{    arr    ::}\NormalTok{ (a }\OtherTok{->}\NormalTok{ b) }\OtherTok{->}\NormalTok{ r a b}
\OtherTok{    first  ::}\NormalTok{ r a b }\OtherTok{->}\NormalTok{ r (a, c) (b, c)}
\OtherTok{    second ::}\NormalTok{ r a b }\OtherTok{->}\NormalTok{ r (c, a) (c, b)}
\OtherTok{    (***)  ::}\NormalTok{ r a b }\OtherTok{->}\NormalTok{ r c d }\OtherTok{->}\NormalTok{ r (a, c) (b, d)}
\OtherTok{    (&&&)  ::}\NormalTok{ r a b }\OtherTok{->}\NormalTok{ r a c }\OtherTok{->}\NormalTok{ r a (b, c)}
\end{Highlighting}
\end{Shaded}

In our case, \texttt{arr} turns any \texttt{a\ -\textgreater{}\ b} function into
an \texttt{Auto\ a\ b}. \texttt{first} turns an \texttt{Auto\ a\ b} into an
\texttt{Auto\ (a,\ c)\ (b,\ c)} --- an Auto that operates on single values to an
Auto that operates only on the first part of a tuple.

\texttt{(***)} chains Autos side-by-side:
\texttt{Auto\ a\ b\ -\textgreater{}\ Auto\ c\ d\ -\textgreater{}\ Auto\ (a,\ c)\ (b,\ d)}.
It basically has each Auto operate on the tuple ``in parallel''.

\texttt{(\&\&\&)} ``forks''. Give an \texttt{Auto\ a\ b} and an
\texttt{Auto\ a\ c}, and it'll create a ``forking'' \texttt{Auto\ a\ (b,\ c)}.

Writing the instance is straightforward enough:

\begin{Shaded}
\begin{Highlighting}[]
\CommentTok{-- source: https://github.com/mstksg/inCode/tree/master/code-samples/machines/Auto2.hs#L32-L47}
\CommentTok{-- interactive: https://www.fpcomplete.com/user/jle/machines}
\KeywordTok{instance} \DataTypeTok{Arrow} \DataTypeTok{Auto} \KeywordTok{where}
\NormalTok{    arr f     }\FunctionTok{=} \DataTypeTok{ACons} \FunctionTok{$}\NormalTok{ \textbackslash{}x }\OtherTok{->}\NormalTok{ (f x, arr f)}
\NormalTok{    first a   }\FunctionTok{=} \DataTypeTok{ACons} \FunctionTok{$}\NormalTok{ \textbackslash{}(x, z) }\OtherTok{->}
                  \KeywordTok{let}\NormalTok{ (y, a') }\FunctionTok{=}\NormalTok{ runAuto a x}
                  \KeywordTok{in}\NormalTok{  ((y, z), first a')}
\NormalTok{    second a  }\FunctionTok{=} \DataTypeTok{ACons} \FunctionTok{$}\NormalTok{ \textbackslash{}(z, x) }\OtherTok{->}
                  \KeywordTok{let}\NormalTok{ (y, a') }\FunctionTok{=}\NormalTok{ runAuto a x}
                  \KeywordTok{in}\NormalTok{  ((z, y), second a')}
\NormalTok{    a1 }\FunctionTok{***}\NormalTok{ a2 }\FunctionTok{=} \DataTypeTok{ACons} \FunctionTok{$}\NormalTok{ \textbackslash{}(x1, x2) }\OtherTok{->}
                  \KeywordTok{let}\NormalTok{ (y1, a1') }\FunctionTok{=}\NormalTok{ runAuto a1 x1}
\NormalTok{                      (y2, a2') }\FunctionTok{=}\NormalTok{ runAuto a2 x2}
                  \KeywordTok{in}\NormalTok{  ((y1, y2), a1' }\FunctionTok{***}\NormalTok{ a2')}
\NormalTok{    a1 }\FunctionTok{&&&}\NormalTok{ a2 }\FunctionTok{=} \DataTypeTok{ACons} \FunctionTok{$}\NormalTok{ \textbackslash{}x }\OtherTok{->}
                  \KeywordTok{let}\NormalTok{ (y1, a1') }\FunctionTok{=}\NormalTok{ runAuto a1 x}
\NormalTok{                      (y2, a2') }\FunctionTok{=}\NormalTok{ runAuto a2 x}
                  \KeywordTok{in}\NormalTok{  ((y1, y2), a1' }\FunctionTok{&&&}\NormalTok{ a2')}
\end{Highlighting}
\end{Shaded}

\begin{center}\rule{0.5\linewidth}{\linethickness}\end{center}

\textbf{Aside}

We can also just take a shortcut and implement these in terms of combinators we
have already written from different typeclasses:

\begin{Shaded}
\begin{Highlighting}[]
\KeywordTok{instance} \DataTypeTok{Arrow} \DataTypeTok{Auto} \KeywordTok{where}
\NormalTok{    arr f     }\FunctionTok{=}\NormalTok{ fmap f id}
\NormalTok{    first a   }\FunctionTok{=}\NormalTok{ liftA2 (,) (a  }\FunctionTok{.}\NormalTok{ arr fst) (arr snd)}
\NormalTok{    second a  }\FunctionTok{=}\NormalTok{ liftA2 (,) (arr fst)      (a  }\FunctionTok{.}\NormalTok{ arr snd)}
\NormalTok{    a1 }\FunctionTok{***}\NormalTok{ a2 }\FunctionTok{=}\NormalTok{ liftA2 (,) (a1 }\FunctionTok{.}\NormalTok{ arr fst) (a2 }\FunctionTok{.}\NormalTok{ arr snd)}
\NormalTok{    a1 }\FunctionTok{&&&}\NormalTok{ a2 }\FunctionTok{=}\NormalTok{ (a1 }\FunctionTok{***}\NormalTok{ a2) }\FunctionTok{.}\NormalTok{ arr (\textbackslash{}x }\OtherTok{->}\NormalTok{ (x, x))}
\end{Highlighting}
\end{Shaded}

Remember, \texttt{id} is the identity Auto\ldots{} and \texttt{fmap\ f} applies
\texttt{f} ``after'' the identity. So this makes sense.

\texttt{first} is a little tricker; we are using \texttt{liftA2\ (,)} on two
Autos, kind of like we used before. \texttt{liftA2} says ``run these two Autos
in parallel on the same input, and then at the end, \texttt{(,)}-up their
results.''

The first of those two autos is \texttt{a\ .\ arr\ fst} --- get the first thing
in the tuple, and then chain the \texttt{a} auto onto it. The second of those
two autos just simply extracts out the second part of the tuple.

\begin{Shaded}
\begin{Highlighting}[]
\OtherTok{a           ::} \DataTypeTok{Auto}\NormalTok{ a b}
\NormalTok{arr}\OtherTok{ fst     ::} \DataTypeTok{Auto}\NormalTok{ (a, c) a}

\NormalTok{a }\FunctionTok{.}\NormalTok{ arr}\OtherTok{ fst ::} \DataTypeTok{Auto}\NormalTok{ (a, c) b}
\NormalTok{arr}\OtherTok{ snd     ::} \DataTypeTok{Auto}\NormalTok{ (a, c) c}

\NormalTok{liftA2 (,) (a }\FunctionTok{.}\NormalTok{ arr fst) (arr snd)}\OtherTok{ ::} \DataTypeTok{Auto}\NormalTok{ (a, c) (b, c)}
\end{Highlighting}
\end{Shaded}

What does this show? Well, that \texttt{Arrow} really isn't anything too
special\ldots{}it's really just what we already had --- a \texttt{Category} with
\texttt{Applicative}. But we are able to define more efficient instances, and
also sort of look at the problem in a ``different way''.

\begin{center}\rule{0.5\linewidth}{\linethickness}\end{center}

What we have here isn't really anything too mystical. It's just some basic
combinators. And like the aside says, we didn't introduce any ``new power'' ---
anything we could ``do'' with Auto using Arrow, we could already do with
Category and Applicative.

The main point is just that we have these neat combinators to chain things in
more useful and expressive ways --- something very important when we eventually
go into AFRP.

\begin{Shaded}
\begin{Highlighting}[]
\NormalTok{ghci}\FunctionTok{>} \KeywordTok{let}\NormalTok{ sumDoub }\FunctionTok{=}\NormalTok{ summer }\FunctionTok{&&&}\NormalTok{ doubleA}
\NormalTok{ghci}\FunctionTok{>}\NormalTok{ testAuto_ sumDoub [}\DecValTok{5}\NormalTok{,}\DecValTok{1}\NormalTok{,}\DecValTok{9}\NormalTok{,}\DecValTok{2}\NormalTok{,}\FunctionTok{-}\DecValTok{3}\NormalTok{,}\DecValTok{4}\NormalTok{]}
\NormalTok{[(}\DecValTok{5}\NormalTok{, }\DecValTok{10}\NormalTok{), (}\DecValTok{6}\NormalTok{, }\DecValTok{2}\NormalTok{), (}\DecValTok{15}\NormalTok{, }\DecValTok{18}\NormalTok{), (}\DecValTok{17}\NormalTok{, }\DecValTok{4}\NormalTok{), (}\DecValTok{14}\NormalTok{, }\FunctionTok{-}\DecValTok{6}\NormalTok{), (}\DecValTok{18}\NormalTok{, }\DecValTok{8}\NormalTok{)]}
\end{Highlighting}
\end{Shaded}

As we'll see, the \emph{real} benefit of Arrow will be in the syntactical sugar
it provides, analogous to Monad's do-blocks.

\subsubsection{ArrowChoice}\label{arrowchoice}

Another useful set of combinators is the \texttt{ArrowChoice} typeclass, which
provides:

\begin{Shaded}
\begin{Highlighting}[]
\OtherTok{left  ::} \DataTypeTok{Auto}\NormalTok{ a b }\OtherTok{->} \DataTypeTok{Auto}\NormalTok{ (}\DataTypeTok{Either}\NormalTok{ a c) (}\DataTypeTok{Either}\NormalTok{ b c)}
\OtherTok{right ::} \DataTypeTok{Auto}\NormalTok{ a b }\OtherTok{->} \DataTypeTok{Auto}\NormalTok{ (}\DataTypeTok{Either}\NormalTok{ c a) (}\DataTypeTok{Either}\NormalTok{ c b)}
\OtherTok{(+++) ::} \DataTypeTok{Auto}\NormalTok{ a b }\OtherTok{->} \DataTypeTok{Auto}\NormalTok{ c d }\OtherTok{->} \DataTypeTok{Auto}\NormalTok{ (}\DataTypeTok{Either}\NormalTok{ a c) (}\DataTypeTok{Either}\NormalTok{ b d)}
\OtherTok{(|||) ::} \DataTypeTok{Auto}\NormalTok{ a c }\OtherTok{->} \DataTypeTok{Auto}\NormalTok{ b c }\OtherTok{->} \DataTypeTok{Auto}\NormalTok{ (}\DataTypeTok{Either}\NormalTok{ a b) c}
\end{Highlighting}
\end{Shaded}

If you look really closely\ldots{}\texttt{left} is kinda like \texttt{first};
\texttt{right} is kinda like \texttt{second}\ldots{}\texttt{(+++)} is kinda like
\texttt{(***)}, and \texttt{(\textbar{}\textbar{}\textbar{})} is like a
backwards \texttt{(\&\&\&)}.

If \texttt{Arrow} allows computations to be ``side-chained'',
\texttt{ArrowChoice} allows computations to be ``skipped/ignored''.

We'll instance \texttt{left}, which applies the given \texttt{Auto} on every
\texttt{Left} input, and passes any \texttt{Right} input along unchanged; the
\texttt{Auto} isn't stepped or anything. The rest of the methods can be
implemented in terms of \texttt{left} and \texttt{arr}.

\begin{Shaded}
\begin{Highlighting}[]
\CommentTok{-- source: https://github.com/mstksg/inCode/tree/master/code-samples/machines/Auto2.hs#L49-L56}
\CommentTok{-- interactive: https://www.fpcomplete.com/user/jle/machines}
\KeywordTok{instance} \DataTypeTok{ArrowChoice} \DataTypeTok{Auto} \KeywordTok{where}
\NormalTok{    left a }\FunctionTok{=} \DataTypeTok{ACons} \FunctionTok{$}\NormalTok{ \textbackslash{}x }\OtherTok{->}
                 \KeywordTok{case}\NormalTok{ x }\KeywordTok{of}
                   \DataTypeTok{Left}\NormalTok{ l  }\OtherTok{->}
                     \KeywordTok{let}\NormalTok{ (l', a') }\FunctionTok{=}\NormalTok{ runAuto a l}
                     \KeywordTok{in}\NormalTok{  (}\DataTypeTok{Left}\NormalTok{ l', left a')}
                   \DataTypeTok{Right}\NormalTok{ r }\OtherTok{->}
\NormalTok{                     (}\DataTypeTok{Right}\NormalTok{ r, left a)}
\end{Highlighting}
\end{Shaded}

We'll see \texttt{ArrowChoice} used in the upcoming syntactic sugar construct,
enabling for if/then/else's and case statements. Don't worry about it for now if
you don't understand it.

\subsection{Proc Notation}\label{proc-notation}

So finally, here is the \emph{real} reason Arrow is useful. It's actually a
pretty well-kept secret, but\ldots{}just like Monad enables \emph{do notation}
syntactical sugar, Arrow enables \emph{proc notation} syntactical sugar. Which
is probably cooler.

Not gonna lie.

A lot of AFRP and a lot of what we're going to be doing will pretty much rely on
proc notation to be able to express complex compositions\ldots{}rather
elegantly.

Proc notation consists of lines of ``arrows'':

\begin{Shaded}
\begin{Highlighting}[]
\NormalTok{arrow }\FunctionTok{-<}\NormalTok{ x}
\end{Highlighting}
\end{Shaded}

which says ``feed \texttt{x} through the Arrow \texttt{arrow}''.

Like in monadic do-blocks, you can also ``bind'' the result, to be used later in
the block:

\begin{Shaded}
\begin{Highlighting}[]
\NormalTok{y }\OtherTok{<-}\NormalTok{ arrow }\FunctionTok{-<}\NormalTok{ x}
\end{Highlighting}
\end{Shaded}

Which says ``feed \texttt{x} through the Arrow \texttt{arrow}, and name the
result \texttt{y}''.

Hey! It looks like a little ASCII arrow! Cute, huh?

The last line of a proc do block is the ``return''/result of the block, like in
monadic do-blocks.

Let's write our first proc block; one that emulates our
\texttt{liftA2\ (+)\ doubleA\ summer}:

\begin{Shaded}
\begin{Highlighting}[]
\OtherTok{doubSummer ::} \DataTypeTok{Auto} \DataTypeTok{Int} \DataTypeTok{Int}
\NormalTok{doubSummer }\FunctionTok{=}\NormalTok{ proc x }\OtherTok{->} \KeywordTok{do}
\NormalTok{    summed  }\OtherTok{<-}\NormalTok{ summer  }\FunctionTok{-<}\NormalTok{ x}
\NormalTok{    doubled }\OtherTok{<-}\NormalTok{ doubleA }\FunctionTok{-<}\NormalTok{ x}
\NormalTok{    id }\FunctionTok{-<}\NormalTok{ summed }\FunctionTok{+}\NormalTok{ doubled}
\end{Highlighting}
\end{Shaded}

In the last line, we want to ``return'' \texttt{summed\ +\ double}; we have to
put an Arrow command there, so we can just feed \texttt{summed\ +\ double}
through \texttt{id}, to have it pop out at the end.

You can think of \texttt{id} like \texttt{return} in normal do notation.

\subsubsection{Simple useful example}\label{simple-useful-example}

How about an \texttt{Auto\ (Either\ Int\ Int)\ (Int,\ Int)}, which maintains
\emph{two} internal counters. You increment the first one with an input of
\texttt{Left\ x}, and you increment the second one with an input of
\texttt{Right\ x}. The output is the state of both counters.

We could write this ``from scratch'', using explicit recursion:

\begin{Shaded}
\begin{Highlighting}[]
\CommentTok{-- source: https://github.com/mstksg/inCode/tree/master/code-samples/machines/Auto2.hs#L102-L109}
\CommentTok{-- interactive: https://www.fpcomplete.com/user/jle/machines}
\OtherTok{dualCounterR ::} \DataTypeTok{Auto}\NormalTok{ (}\DataTypeTok{Either} \DataTypeTok{Int} \DataTypeTok{Int}\NormalTok{) (}\DataTypeTok{Int}\NormalTok{, }\DataTypeTok{Int}\NormalTok{)}
\NormalTok{dualCounterR }\FunctionTok{=}\NormalTok{ dualCounterWith (}\DecValTok{0}\NormalTok{, }\DecValTok{0}\NormalTok{)}
  \KeywordTok{where}
\NormalTok{    dualCounterWith (x, y) }\FunctionTok{=} \DataTypeTok{ACons} \FunctionTok{$}\NormalTok{ \textbackslash{}inp }\OtherTok{->}
                               \KeywordTok{let}\NormalTok{ newC }\FunctionTok{=} \KeywordTok{case}\NormalTok{ inp }\KeywordTok{of}
                                            \DataTypeTok{Left}\NormalTok{ i  }\OtherTok{->}\NormalTok{ (x }\FunctionTok{+}\NormalTok{ i, y)}
                                            \DataTypeTok{Right}\NormalTok{ i }\OtherTok{->}\NormalTok{ (x, y }\FunctionTok{+} \DecValTok{1}\NormalTok{)}
                               \KeywordTok{in}\NormalTok{  (newC, dualCounterWith newC)}
\end{Highlighting}
\end{Shaded}

But we all know in Haskell that explicit recursion is usually a sign of bad
design and is best avoided whenever possible. So many potential places for bugs!

Let's try writing the same thing using Auto composition:

\begin{Shaded}
\begin{Highlighting}[]
\CommentTok{-- source: https://github.com/mstksg/inCode/tree/master/code-samples/machines/Auto2.hs#L112-L116}
\CommentTok{-- interactive: https://www.fpcomplete.com/user/jle/machines}
\OtherTok{dualCounterC ::} \DataTypeTok{Auto}\NormalTok{ (}\DataTypeTok{Either} \DataTypeTok{Int} \DataTypeTok{Int}\NormalTok{) (}\DataTypeTok{Int}\NormalTok{, }\DataTypeTok{Int}\NormalTok{)}
\NormalTok{dualCounterC }\FunctionTok{=}\NormalTok{ (summer }\FunctionTok{***}\NormalTok{ summer) }\FunctionTok{.}\NormalTok{ arr wrap}
  \KeywordTok{where}
\NormalTok{    wrap (}\DataTypeTok{Left}\NormalTok{ i)  }\FunctionTok{=}\NormalTok{ (i, }\DecValTok{0}\NormalTok{)}
\NormalTok{    wrap (}\DataTypeTok{Right}\NormalTok{ i) }\FunctionTok{=}\NormalTok{ (}\DecValTok{0}\NormalTok{, i)}
\end{Highlighting}
\end{Shaded}

That's a bit more succinct, but I think the proc notation is much nicer!

\begin{Shaded}
\begin{Highlighting}[]
\CommentTok{-- source: https://github.com/mstksg/inCode/tree/master/code-samples/machines/Auto2.hs#L119-L127}
\CommentTok{-- interactive: https://www.fpcomplete.com/user/jle/machines}
\OtherTok{dualCounterP ::} \DataTypeTok{Auto}\NormalTok{ (}\DataTypeTok{Either} \DataTypeTok{Int} \DataTypeTok{Int}\NormalTok{) (}\DataTypeTok{Int}\NormalTok{, }\DataTypeTok{Int}\NormalTok{)}
\NormalTok{dualCounterP }\FunctionTok{=}\NormalTok{ proc inp }\OtherTok{->} \KeywordTok{do}
    \KeywordTok{let}\NormalTok{ (add1, add2) }\FunctionTok{=} \KeywordTok{case}\NormalTok{ inp }\KeywordTok{of} \DataTypeTok{Left}\NormalTok{ i  }\OtherTok{->}\NormalTok{ (i, }\DecValTok{0}\NormalTok{)}
                                   \DataTypeTok{Right}\NormalTok{ i }\OtherTok{->}\NormalTok{ (}\DecValTok{0}\NormalTok{, i)}

\NormalTok{    sum1 }\OtherTok{<-}\NormalTok{ summer }\FunctionTok{-<}\NormalTok{ add1}
\NormalTok{    sum2 }\OtherTok{<-}\NormalTok{ summer }\FunctionTok{-<}\NormalTok{ add2}

\NormalTok{    id }\FunctionTok{-<}\NormalTok{ (sum1, sum2)}
\end{Highlighting}
\end{Shaded}

It's a bit more verbose\ldots{}but I think it's much clearer what's going on,
right?

\begin{Shaded}
\begin{Highlighting}[]
\NormalTok{ghci}\FunctionTok{>}\NormalTok{ testAuto_ dualCounterP [}\DataTypeTok{Right} \DecValTok{1}\NormalTok{, }\DataTypeTok{Left} \DecValTok{2}\NormalTok{, }\DataTypeTok{Right}\NormalTok{ (}\FunctionTok{-}\DecValTok{4}\NormalTok{), }\DataTypeTok{Left} \DecValTok{10}\NormalTok{, }\DataTypeTok{Right} \DecValTok{3}\NormalTok{]}
\NormalTok{[(}\DecValTok{0}\NormalTok{, }\DecValTok{1}\NormalTok{), (}\DecValTok{2}\NormalTok{, }\DecValTok{1}\NormalTok{), (}\DecValTok{2}\NormalTok{, }\FunctionTok{-}\DecValTok{3}\NormalTok{), (}\DecValTok{12}\NormalTok{, }\FunctionTok{-}\DecValTok{3}\NormalTok{), (}\DecValTok{12}\NormalTok{, }\DecValTok{0}\NormalTok{)]}
\end{Highlighting}
\end{Shaded}

\subsubsection{Proc shines}\label{proc-shines}

And let's say we wanted another constraint. Let's say that\ldots{}for the
\texttt{Left} case, every \emph{other} time it's a Left, \emph{ignore the value}
and don't add anything. That is, every second, fourth, sixth \texttt{Left\ i}
input should ignore the \texttt{i} and not add anything.

How would we do this in the explicit recursive case? Why --- well, adding
another component to the ``explicit state'' tuple, and dealing with that when
necessary.

I don't even know how to begin writing it in a readable way using arrow
composition.

But the proc notation? Piece of cake!

\begin{Shaded}
\begin{Highlighting}[]
\CommentTok{-- source: https://github.com/mstksg/inCode/tree/master/code-samples/machines/Auto2.hs#L148-L160}
\CommentTok{-- interactive: https://www.fpcomplete.com/user/jle/machines}
\OtherTok{dualCounterSkipP ::} \DataTypeTok{Auto}\NormalTok{ (}\DataTypeTok{Either} \DataTypeTok{Int} \DataTypeTok{Int}\NormalTok{) (}\DataTypeTok{Int}\NormalTok{, }\DataTypeTok{Int}\NormalTok{)}
\NormalTok{dualCounterSkipP }\FunctionTok{=}\NormalTok{ proc inp }\OtherTok{->} \KeywordTok{do}
\NormalTok{    (add1, add2) }\OtherTok{<-} \KeywordTok{case}\NormalTok{ inp }\KeywordTok{of}
                      \DataTypeTok{Left}\NormalTok{ i }\OtherTok{->} \KeywordTok{do}
\NormalTok{                        count }\OtherTok{<-}\NormalTok{ summer }\FunctionTok{-<} \DecValTok{1}
\NormalTok{                        id }\FunctionTok{-<}\NormalTok{ (}\KeywordTok{if}\NormalTok{ odd count }\KeywordTok{then}\NormalTok{ i }\KeywordTok{else} \DecValTok{0}\NormalTok{, }\DecValTok{0}\NormalTok{)}
                      \DataTypeTok{Right}\NormalTok{ i }\OtherTok{->}
\NormalTok{                        id }\FunctionTok{-<}\NormalTok{ (}\DecValTok{0}\NormalTok{, i)}

\NormalTok{    sum1 }\OtherTok{<-}\NormalTok{ summer }\FunctionTok{-<}\NormalTok{ add1}
\NormalTok{    sum2 }\OtherTok{<-}\NormalTok{ summer }\FunctionTok{-<}\NormalTok{ add2}

\NormalTok{    id }\FunctionTok{-<}\NormalTok{ (sum1, sum2)}
\end{Highlighting}
\end{Shaded}

\begin{Shaded}
\begin{Highlighting}[]
\NormalTok{ghci}\FunctionTok{>}\NormalTok{ testAuto_ dualCounterP [}\DataTypeTok{Right} \DecValTok{1}\NormalTok{, }\DataTypeTok{Left} \DecValTok{2}\NormalTok{, }\DataTypeTok{Right}\NormalTok{ (}\FunctionTok{-}\DecValTok{4}\NormalTok{), }\DataTypeTok{Left} \DecValTok{10}\NormalTok{, }\DataTypeTok{Right} \DecValTok{3}\NormalTok{]}
\NormalTok{[(}\DecValTok{0}\NormalTok{, }\DecValTok{1}\NormalTok{), (}\DecValTok{2}\NormalTok{, }\DecValTok{1}\NormalTok{), (}\DecValTok{2}\NormalTok{, }\FunctionTok{-}\DecValTok{3}\NormalTok{), (}\DecValTok{2}\NormalTok{, }\FunctionTok{-}\DecValTok{3}\NormalTok{), (}\DecValTok{2}\NormalTok{, }\DecValTok{0}\NormalTok{)]}
\end{Highlighting}
\end{Shaded}

And that's something to write home about :)

\subsection{Locally Stateful Composition}\label{locally-stateful-composition}

The last example highlights something very significant about Autos and their
Arrow-based composition: Autos with composition allow you to make \emph{locally
stateful compositions}.

What if we had done the above using some sort of state monad, or doing the
explicit recursion?

We'd have carried the ``entire'' state in the parameter:

\begin{Shaded}
\begin{Highlighting}[]
\CommentTok{-- source: https://github.com/mstksg/inCode/tree/master/code-samples/machines/Auto2.hs#L136-L145}
\CommentTok{-- interactive: https://www.fpcomplete.com/user/jle/machines}
\OtherTok{dualCounterSkipR ::} \DataTypeTok{Auto}\NormalTok{ (}\DataTypeTok{Either} \DataTypeTok{Int} \DataTypeTok{Int}\NormalTok{) (}\DataTypeTok{Int}\NormalTok{, }\DataTypeTok{Int}\NormalTok{)}
\NormalTok{dualCounterSkipR }\FunctionTok{=}\NormalTok{ counterFrom ((}\DecValTok{0}\NormalTok{, }\DecValTok{0}\NormalTok{), }\DecValTok{1}\NormalTok{)}
  \KeywordTok{where}
\NormalTok{    counterFrom ((x, y), s) }\FunctionTok{=}
      \DataTypeTok{ACons} \FunctionTok{$}\NormalTok{ \textbackslash{}inp }\OtherTok{->}
        \KeywordTok{let}\NormalTok{ newCS }\FunctionTok{=} \KeywordTok{case}\NormalTok{ inp }\KeywordTok{of}
                      \DataTypeTok{Left}\NormalTok{ i  }\FunctionTok{|}\NormalTok{ odd s     }\OtherTok{->}\NormalTok{ ((x }\FunctionTok{+}\NormalTok{ i, y), s }\FunctionTok{+} \DecValTok{1}\NormalTok{)}
                              \FunctionTok{|}\NormalTok{ otherwise }\OtherTok{->}\NormalTok{ ((x    , y), s }\FunctionTok{+} \DecValTok{1}\NormalTok{)}
                      \DataTypeTok{Right}\NormalTok{ i             }\OtherTok{->}\NormalTok{ ((x, y }\FunctionTok{+}\NormalTok{ i), s    )}
        \KeywordTok{in}\NormalTok{  (fst newCS, counterFrom newCS)}
\end{Highlighting}
\end{Shaded}

Not only is it a real mess and pain --- and somewhere where bugs are rife to pop
up --- note the entire state is contained in one thing. That means everything
has access to it; all access is sort of haphazard and ad-hoc, as well. Note that
the \texttt{Right} case can do whatever it wants with the \texttt{s}. It has
access to it, can read it, act on it, modify it\ldots{}anything it wants! We
can't really ``enforce'' that the \texttt{Right} case can't touch the
\texttt{s}, without putting in more complicated work/overhead.

In the proc example\ldots{}the \texttt{s} is a \texttt{summer} that is ``locked
inside'' the \texttt{Left} branch. \texttt{Right} branch stuff can't touch it.

In fact, all of the \texttt{summers} keep their own state, independently from
each other. Nothing can really modify their state except for themselves, if they
chose to. Less room for bugs, too, in adding, because you already know that
\texttt{summer} works.

This property --- that every single component maintains its own internal state
--- is, in my opinion, one of the most significant aspects of this whole game
with Auto and Category and Arrow etc. Every component minds its own state.

And also --- if we wanted to ``add a new component'' to the state, like we did,
we don't have to really change anything besides just plopping it on. In the
explicit recursion example, we needed to go in and \emph{change the state type}
to ``make room'' for the new state. We needed to pretty much refactor the entire
thing!

This really demonstrates the core principles of what \emph{composability} and
\emph{modularity} even really \emph{mean}.

\subsubsection{A Quick Gotcha}\label{a-quick-gotcha}

Remember that with proc notation, you are really just composing and building up
a giant \texttt{Auto}. Each individual \texttt{Auto} that you compose has to
already be known at ``composition time''. (That is, before you ever ``run'' it,
the structure of the \texttt{Auto} is known and fixed).

This means that you can't use bindings from \emph{proc} blocks to form the
\texttt{Auto}s that you are composing:

\begin{Shaded}
\begin{Highlighting}[]
\NormalTok{foo }\FunctionTok{=}\NormalTok{ proc x }\OtherTok{->} \KeywordTok{do}
\NormalTok{    y }\OtherTok{<-}\NormalTok{ auto1 }\FunctionTok{-<}\NormalTok{ x}
\NormalTok{    auto2 y }\FunctionTok{-<}\NormalTok{ y}
\end{Highlighting}
\end{Shaded}

This won't work. That's because this is really supposed to be a composition of
\texttt{auto1} and \texttt{auto2\ y}. But what is \texttt{auto2\ y}? \texttt{y}
doesn't even exist when you are making the compositions! \texttt{y} is just a
name we gave to the output of \texttt{auto1}, in the process of our stepping it.
\texttt{y} doesn't exist until we ``step'' \texttt{foo}\ldots{}so can't use
\texttt{auto2\ y} in the process of composing \texttt{foo}.

To see more clearly, see what we'd do if we tried to write \texttt{foo} as a
compositino:\footnote{This was originally a typo but I like the word so much
  that I'm just going to leave it in here.}

\begin{Shaded}
\begin{Highlighting}[]
\NormalTok{foo }\FunctionTok{=}\NormalTok{ auto2 y }\FunctionTok{.}\NormalTok{ auto1}
\end{Highlighting}
\end{Shaded}

Where does the \texttt{y} come from?!

Hopefully from this it is clear to see that it doesn't make sense to use what
you bind/name in \emph{proc} notation to actually ``create'' the \texttt{Arrow}
you are using.

Remember, \emph{proc} notation is
\texttt{result\ \textless{}-\ arrow\ -\textless{}\ input}. The \texttt{arrow}
part has to already be known before everything even starts, so you can't use
things you bind to determine it :)

\section{Moving on}\label{moving-on}

Welp, hopefully by now you are experts at working with the Auto machine, and
understanding it as ``function-like things''. You've gotten deep and intimate by
instancing some common typeclasses.

Then you saw Arrow, and understood how Auto fits into the Arrow abstraction. And
then you learned about proc notation, and how\ldots{}everything just\ldots{}fits
together. And you can declare some nice computations/compositions in a way that
looks a lot like monadic do blocks.

We saw how complex compositions --- and complex recursions --- now look really
pretty and nice in proc-do notation.

And then we saw an extension of the ``opaque state'' concept we learned last
time --- \emph{locally stateful compositions}. Using Auto composition and the
Arrow instance, we can now combine Autos with local state together\ldots{}and
they all maintain and keep track of their own state. No more ``global state'',
like before --- every individual component only minds what it knows, and nothing
else. And that this is really what ``composability'' really is all about.

Up next, we will transition from Auto to the Wire abstraction, which is sort of
like an Auto with more features.

And then we will be on our way! :D

\subsection{Exercises}\label{exercises}

Yeah, I know that a lot of this post was pretty abstract\ldots{}finding ways to
make this post immediately useful with applications was one of the reasons why
it took so long for me to get it out, after the last one.

That being said, there are some things you can try out test your understanding
before Part 3 :)

\begin{enumerate}
\def\labelenumi{\arabic{enumi}.}
\item
  Write the
  \href{https://ocharles.org.uk/blog/guest-posts/2013-12-22-24-days-of-hackage-profunctors.html}{Profunctor}
  instance mentioned above; look at the Functor instance we wrote as a
  reference. And hey, how about \texttt{Strong} and \texttt{Choice}, too?
\item
  Try writing the various Autos we wrote last time at the end using composition
  and proc notation instead of explicit recursion. Feel free to define your own
  ``primitives'' if you find that you must.

  Some of these might be trickier than others!

  Note that some of these can be done with just a straight-up \texttt{autoFold},
  for the entire thing. While this is neat and all, it might be more useful to
  practice the principles of \emph{local statefulness}, and try to break things
  up into as many primitives as possible; always try to avoid keeping every part
  of your state in one giant \texttt{autoFold} parameter.

  \begin{itemize}
  \item
    \emph{Rolling average}: You should be able to do this with just
    \texttt{autoFold} and the right proc block. You can even do it with straight
    up composition, but it's a bit less clean.
  \item
    \emph{onFor}: You should be able to do this with \texttt{settableAuto} (or
    something like that), and some nice proc routing with if/then/elses.
  \item
    \emph{autoMap}: This should also be doable with \texttt{autoFold}; although
    there isn't much state to separate out, so this example isn't as
    interesting. It might be more fun to use this one as a component of a larger
    \texttt{Auto}, and see what you can use it for!
  \end{itemize}
\end{enumerate}

\end{document}
