\documentclass[]{article}
\usepackage{lmodern}
\usepackage{amssymb,amsmath}
\usepackage{ifxetex,ifluatex}
\usepackage{fixltx2e} % provides \textsubscript
\ifnum 0\ifxetex 1\fi\ifluatex 1\fi=0 % if pdftex
  \usepackage[T1]{fontenc}
  \usepackage[utf8]{inputenc}
\else % if luatex or xelatex
  \ifxetex
    \usepackage{mathspec}
    \usepackage{xltxtra,xunicode}
  \else
    \usepackage{fontspec}
  \fi
  \defaultfontfeatures{Mapping=tex-text,Scale=MatchLowercase}
  \newcommand{\euro}{€}
\fi
% use upquote if available, for straight quotes in verbatim environments
\IfFileExists{upquote.sty}{\usepackage{upquote}}{}
% use microtype if available
\IfFileExists{microtype.sty}{\usepackage{microtype}}{}
\usepackage[margin=1in]{geometry}
\ifxetex
  \usepackage[setpagesize=false, % page size defined by xetex
              unicode=false, % unicode breaks when used with xetex
              xetex]{hyperref}
\else
  \usepackage[unicode=true]{hyperref}
\fi
\hypersetup{breaklinks=true,
            bookmarks=true,
            pdfauthor={},
            pdftitle={},
            colorlinks=true,
            citecolor=blue,
            urlcolor=blue,
            linkcolor=magenta,
            pdfborder={0 0 0}}
\urlstyle{same}  % don't use monospace font for urls
% Make links footnotes instead of hotlinks:
\renewcommand{\href}[2]{#2\footnote{\url{#1}}}
\setlength{\parindent}{0pt}
\setlength{\parskip}{6pt plus 2pt minus 1pt}
\setlength{\emergencystretch}{3em}  % prevent overfull lines
\setcounter{secnumdepth}{0}


\begin{document}

\% The Compromiseless Reconciliation of I/O and Purity \% Justin Le \% November
12, 2013

\emph{Originally posted on
\textbf{\href{https://blog.jle.im/entry/the-compromiseless-reconciliation-of-i-o-and-purity.html}{in
Code}}.}

One of the crazy ideals of functional programming is the idea that your program
is simply a list of definitions of mathematical functions. And like real math
functions, FP functions are \textbf{pure}. That means that (1) they cannot
affect any state, and (2) that they must return the same thing every time they
are called with the same arguments.

When you first learn functional programming, this manifests as "your variables
are immutable and you can't do loops; use recursion instead." And if you do
that, everything is "fine".

However, there is an apparent glaring problem with this adherence to purity:
\textbf{I/O}. Input and output are inherently stateful.

\section{The Problem}

For the obvious (input) example, consider \texttt{getchar()} in C. It returns
the character that a user enters. Obviously, if \texttt{getchar()} returned the
same thing every time, you'd have an extraordinarily useless function. Input
\emph{inherently violates} purity, it seems. (Also, consider a
\href{http://xkcd.com/221/}{function generating random numbers})

The idea of output violates purity as well. Consider calling \texttt{printf()}
in C. You're going to change the state of the terminal. A benign example, of
course; but what about a function \texttt{add\_database\_row()} that adds a row
to your database? A call of \texttt{get\_database\_row()} will now return
something different than it would have returned before.
\texttt{get\_database\_row()} now returns two different things when run at two
different times -\/-\/- impure! Blasphemy!

Of course, it should be obvious that not being able to perform IO means that
your program is essentially useless in most real world applications. But purity
is pretty cool, and it gives us guarantees that let us
\href{http://u.jle.im/19JxV5S}{reason with our code} in ways that are impossible
with impure code, and with unprecedented safety. It opens the doors to
previously inaccessible models of parallel, concurrent, and distributed
programming. It allows the compiler to do crazy optimization tricks. It allows
for powerful mathematical analysis of our programs. The full benefits of purity
are beyond the scope of this article, but you can trust me when they say that
they are too much to give up over a technicality.

So how can we reconcile the idea of a pure language with
\textasciitilde{}\textasciitilde{}anything
useful\textasciitilde{}\textasciitilde{} I/O?

\section{A Functional "Program"}

\subsection{Declarations}

Let's look at an almost-typical Haskell program.

\textasciitilde{}\textasciitilde{}\textasciitilde{}haskell -\/- factorial n: n!
factorial :: Int -\textgreater{} Int factorial 0 = 1 factorial n = n * factorial
(n-1)

-\/- fib n: the nth Fibonacci number fib :: Int -\textgreater{} Int fib 0 = 1
fib 1 = 1 fib n = fib (n-2) + fib (n-1)

-\/- first\emph{n}fibs n: a list of the first n Fibonacci numbers
first\emph{n}fibs :: Int -\textgreater{} {[}Int{]} first\emph{n}fibs n = map fib
{[}1..n{]} \textasciitilde{}\textasciitilde{}\textasciitilde{}

One of the first things you should notice is that this looks strikingly similar
to a list of math equations...and almost not like a program.

Notice one important thing about this (at least, in Haskell): there is no
inherent ordering in any of these statements. By this, I mean that
\texttt{factorial}, \texttt{fib}, and \texttt{first\_n\_fibs} can be defined in
any order. When you write declarations of mathematical objects on paper, the
order in which you declare them should have no bearing on what they represent.
These are functions. Immortal, unchanging, ethereal, separate from time and
space. It is simply nonsensical to talk about order in this
context.{[}\^{}strictness{]}

\textless{}!-\/- Thanks to evincarofautum of reddit for pointing out that the
ordering of the -\/-\textgreater{} \textless{}!-\/- pattern matches in this
example actually do matter. -\/-\textgreater{}

Also note that these declarations don't always declare integers/numbers.
\texttt{first\_n\_fibs} actually declares a data structure -\/-\/- a list that
contains integers. Of course this is no big problem...mathematical functions can
map integers to matrices, or matrices to functions, anything you can think of.
We aren't limited to simply defining primitive things. We can also define
structures that contain things.

Of course, this "program" doesn't actually \emph{do} anything. Let's look at
some more programs and see if we can address this.

\subsection{Representing Actions}

There are a lot of data structures/data types that may be expressed in Haskell.
One in particular is called \texttt{IO}. \texttt{IO\ a} represents a computation
that returns something of type \texttt{a}. There are a couple of pre-packaged
computations included in the standard library. Let's write another
almost-typical Haskell program with some.

\textasciitilde{}\textasciitilde{}\textasciitilde{}haskell -\/-
getStringFromStdin: returns a computation that represents the act of -\/-
getting a string from stdin. or rather, a series of instructions on -\/-
interacting with the computer and generating a String. getStringFromStdin :: IO
String getStringFromStdin = getLine

-\/- printFibN: returns a computation that represents the act of printing the
-\/- nth Fibonacci number to stdout and returns () (Nothing). or rather, -\/- a
series of instruction on interacting with the computer to get it to -\/- print a
Fibonacci number and returning nothing. printFibN :: Int -\textgreater{} IO ()
printFibN n = print (fib n) \textasciitilde{}\textasciitilde{}\textasciitilde{}

Let's look at these.

These are simply functions/declarations, just like the ones above. Although
instead of returning an integer or a list data structure, it returns a special
data structure that represents a computation. \texttt{{[}a{]}} represents a list
of \texttt{a}'s. \texttt{IO\ a} represents an abstract computation (or a series
of system instructions) that returns an \texttt{a}.

These declarations and functions are also simply "math" functions. Instead of
returning a set or a matrix or a vector, it returns another type of object.

Note that this has nothing to do with execution. \texttt{printFibN} does
\emph{not} execute a print statement. No more than writing \texttt{printFibN} on
a piece of paper will cause it to magically evaluate. It does not execute
anything: it is simply an abstract data structure representing a computation.

Note again that there is no inherent ordering involved. Whether you define one
or the other first, it does not change what the two names really
\emph{represent}. Just like if you defined two matrices on a piece of paper in a
different order, it does not change the matrices they represent.

Also note that all of these declarations are completely pure.
\texttt{getStringFromStdin} will return the exact same \emph{representation of a
computation} every single time. \texttt{printFibN\ n} will return the exact same
\emph{computation representation} for every \texttt{n} every single time. The
exact same instruction sequence every single time for every \texttt{n}.

And yes, the objects themselves don't actually execute anything. That's like
saying writing down a matrix executes something in the real world.

\subsection{Data Structures, Not Commands}

To illustrate the difference between a data structure representing a computation
and a computation itself, let's look at a possible confusion that might arise
from mixing up the two.

\textasciitilde{}\textasciitilde{}\textasciitilde{}haskell getStringAndPrint ::
IO () getStringAndPrint = print (getStringFromStdin)
\textasciitilde{}\textasciitilde{}\textasciitilde{}

What would you expect to happen here?

Remember, \texttt{print} is an IO instruction prints out what it is passed.
\texttt{getStringFromStdin} is a computation object that gives a string when
executed.

In another language, which deals with computations (and not representations of
them), you would expect it to get a string from stdin and then print it. Sort of
like an echo.

However, this is not the case in Haskell. What is \texttt{getStringFromStdin}?
It is \emph{not} a string -\/-\/- it is a computation object.

What will happen is that \texttt{print} (when executed by a computer) won't
print the result of \texttt{getStringFromStdin}. \texttt{print} will print out
the \textbf{representation of the computation}! It'll print out the \emph{data
structure representing the computation}, or some string "representing" the act
of the computation!

(At least, that's what it's supposed to do. Unfortunately, \texttt{IO} data
structures do not come built-in with a method for their string representation in
vanilla Haskell. But the point remains that \texttt{print} would \emph{try} to
print out the data structure itself somehow, and not the actual result of the
computation)

\section{Instructions as Data Structures}

Let's take a step back and think about what it even means to have a data
structure representing computation.

You can think about it as some kind of list/nested tree (or more accurately, a
graph) of instructions for someone to follow. For the case of \texttt{IO\ Int},
you can see it as, internally, some kind of tree/nested instruction set for
someone to follow in order to produce an \texttt{Int}. In the case of
\texttt{IO}, for GHC, the "someone" is a computer. Or more specifically, a
processor. GHC directly translates any standalone IO object into assembly code
(or even a less optimal C code).

Technically, you \emph{could} "think" of every IO object as a self-contained and
encapsulated little packet of assembly or C code that you can compose and nest
and merge, etc. with other such packets, without worrying about the lower level
code itself. But don't do this, or you risk confusing a possible representation
of an object for the actual abstract object itself. (Think about it like saying
that a mathematical matrix is a series of pencil swirls on a piece of paper.)
But yes, at any time, you can "compile"/make concrete an IO object into
standalone C code with GHC. This is actually a fact, and every IO object can be
said to correspond directly with a chunk of C code.

Really, though, there are many ways to "translate" this data structure into
instructions for anyone to follow.
\href{http://hackage.haskell.org/package/haste-compiler}{Haste}, for example,
takes \texttt{IO} data structure and turns it into something that can be run in
a Javascript interpreter. That is, it takes something like
\texttt{printFibN\ n}, takes the internal tree instruction set, and writes it
out concretely in javascript.

In fact it would not be too hard to imagine a compiler that would take any
arbitrary \texttt{IO} structure and translate it into human-followable (yet very
verbose) instructions on a piece of paper, written in plain English. Or French,
for that matter.

That is because that's all \texttt{IO} \emph{is} -\/-\/- a tree/graph data
structure representing an instruction series, that we assemble/build/compose
using Haskell code. The same way you would assemble/build an array, or a
dictionary, or a linked list in any other language.

\subsection{Other Examples}

It might help to think about similar "instruction-like" data structures.

Take \href{http://hackage.haskell.org/package/persistent}{Persistent}, which (in
some variants) provides the \texttt{SqlPersistM} data structure. This data
structure represents an interaction with an SQL Database. In other words, it
represents a tree of instructions for interacting with one. When you give it to
the Persistent library, it'll translate that \texttt{SqlPersistM} into a series
of \textbf{SQL queries}! Yes, it produces actual SQL query strings, using the
instructions from the data structure, executes them, and returns the result. An
\texttt{SqlPersistM\ Int} is an SQL interaction that returns an Int when run
with the Persistent library.

Then you have \href{http://hackage.haskell.org/package/parsec}{Parsec}, which
provides a \texttt{Parsec} data structure, which are \emph{instructions for
Parsec to parse a string}. A \texttt{Parsec\ Int} structure{[}\^{}parsect{]}
represents instructions for parsing a string into an \texttt{Int}. When you give
a \texttt{Parsec\ Int} and a string to parse to the Parsec library, it will run
the parse specified by the \texttt{Parsec} object and return (hopefully) a
parsed \texttt{Int}. Remember, a \texttt{Parsec\ Int} object does \emph{not}
actually "parse" anything; It is \emph{used by Parsec} to parse a string and
return an \texttt{Int}!

The reason why we use these data structures in Haskell, instead of actually
writing SQL queries and parsing rules from scratch, is because they become
\emph{composable}. SQL queries? Not very composable. Parsing rules? Not exactly,
either. In this way, you can build complex SQL queries without ever touching a
query string by composing simple queries. You can create very complex and
intricate parsing rules without every having to "worry" about actually writing
the parser: you just compose simple, smaler parsers.

And this is really what Haskell "does best" (and possibly what Haskell was
really made for): assembling and composing these possibly complex instruction
data structures in a pure way and "passing them on" to things that can take them
and use them to do great things. An \texttt{SqlPersistM} is used by Persistent,
a \texttt{Parsec} is used by Parsec, and an \texttt{IO} is used by...well, what?
A computer!

\section{The "Main" Point}

So now we see that Haskell has no problems returning a data structure that
represents computer instructions (well, at least, Haskell's standard library
handles all of it for us by giving us useful instruction primitives that we can
build more complex instructions from).

Now we have an instruction object. How do we actually get a computer to use and
execute it?

For this, we rely on convention (or arbitrary specification, however you like to
see it). A Haskell compiler will "understand" your data structures, and it picks
\textbf{one} of them to compile into a binary format for your computer (or
whatever format your executing environment reads best). Out of all of the IO
objects you can return/represent, the Haskell compiler chooses one of them to be
the one it actually compiles into computer-readable code.

And by convention/specification, it is the IO object with the name "main":

\textasciitilde{}\textasciitilde{}\textasciitilde{}haskell -\/- printFibN:
returns a computation that represents the act of printing the -\/- nth Fibonacci
number to stdout and returns () (Nothing). printFibN :: Int -\textgreater{} IO
() printFibN n = print (fib n)

-\/- main: The IO object that we agree that the compiler will actually compile.
main :: IO () main = printFibN 10
\textasciitilde{}\textasciitilde{}\textasciitilde{}

And here we are. A full, executable Haskell program. You can
\href{https://github.com/mstksg/inCode/blob/master/code-samples/io-purity/IO-Purity.hs}{download
and run it yourself}.

As we can see, every function or declaration that makes up our program is
completely pure and side-effectless. In fact, the assembly of \texttt{main}
itself is side-effectless and pure. We assemble the \texttt{IO\ ()} that
\texttt{main} returns in a pure way. \texttt{printFibN\ 10} will return the
exact same computation representation every single time we run it.

\texttt{printFibN\ 10} is \textbf{pure}. Every time we \emph{evaluate}
\texttt{printFibN\ 10}, we get the exact same computation
representation/instruction list.

Therefore, \texttt{main} is pure, as well. Every time we evaluate \texttt{main},
we get the exact same computational data structure.

\subsection{Purity challenged?}

Now consider:

\textasciitilde{}\textasciitilde{}\textasciitilde{}haskell -\/-
getStringFromStdin: returns a computation that represents the act of -\/-
getting a string from stdin getStringFromStdin :: IO String getStringFromStdin =
getLine

-\/- main: The IO object that we agree that the compiler will actually compile.
main :: IO () main = getStringFromStdin \textgreater{}\textgreater{}=
(\textbackslash{}result -\textgreater{} print result)
\textasciitilde{}\textasciitilde{}\textasciitilde{}

(Sample can be
\href{https://github.com/mstksg/inCode/blob/master/code-samples/io-purity/Challenge.hs}{downloaded
and run})

(An aside: \texttt{\textgreater{}\textgreater{}=} here is an operator that takes
the result of the left hand side's computation and passes it as a parameter to
the right hand side. Essentially, it turns two IO computation data structures
and combines/chains them into one big one. The
\texttt{(\textbackslash{}x\ -\textgreater{}\ print\ x)} syntax says "take the
\texttt{x} passed to you and use it in \texttt{print\ x}")

\texttt{main} gets something from the standard input, and then prints it.

Oh wait. This means that if I type something different into standard input, the
program will return something different, right? How is this pure?

Here is the crucial difference between \textbf{evaluation} and
\textbf{execution}:

\texttt{main} will always \textbf{evaluate} to the exact same computation data
structure.

\texttt{main} will always be the \emph{exact} same program, no matter when you
run it. (In particular, the program that gets a string from stdin and prints it)

The computer/processor -\/-\/- which is given a binary representation of the IO
data structure, and is completely separate from the language itself -\/-\/- now
\textbf{executes} this binary/compiled data structure/program. Its execution of
this binary is, of course, potentially unpredictable and in general
non-deterministic, and can depend on things like the temperature, the network
connection, the person at the keyboard, the database contents, etc. The
\emph{instructions/binary} that it follows will be the same every time. The
\emph{result} of those instructions will be different every time (as someone who
has ever attempted to bake a cake can testify).

\texttt{main} is a function that returns/evaluates deterministically to a data
structure representing a computation.

The computation that it represents is not necessarily deterministic.

This distinction between \textbf{evaluation} and \textbf{execution} is what sets
apart this I/O model that permits its purity.

\texttt{main} is a pure value. The instruction data structure \texttt{main}
represesents impure instructions.

And \emph{that} is how we can deal with I/O in Haskell while remaining a pure
language.

\subsection{Illustrating the difference}

To really understand the difference between evaluation and execution, let's look
at this example:

\textasciitilde{}\textasciitilde{}\textasciitilde{}haskell ignoreAndSayHello ::
IO a -\textgreater{} IO () ignoreAndSayHello to\_ignore = print "Hello!"

main :: IO () main = ignoreAndSayHello getStringFromStdin
\textasciitilde{}\textasciitilde{}\textasciitilde{}

What does this program do?

Naively, we expect it to ask for a string from standard input, ignore the
result, and print "Hello!".

Actually, this is \textbf{not} what it does.

This is because \texttt{ignoreAndSayHello\ getStringFromStdin} will evaluate to
\texttt{print\ "Hello"} (remember, it ignores its argument). So \texttt{main}
evaluates to one single IO action: \texttt{print\ "Hello!"}.

So your program returns the simple IO action \texttt{print\ "Hello!"} -\/-\/-
the computation returned by \texttt{main} therefore simply prints "Hello!". This
computation does not represent anything that would ask for input.

The "real" way to do this would be:

\textasciitilde{}\textasciitilde{}\textasciitilde{}haskell ignoreAndSayHello ::
IO a -\textgreater{} IO () ignoreAndSayHello to\emph{ignore = to}ignore
\textgreater{}\textgreater{}= (\textbackslash{}result -\textgreater{} print
"Hello!")

main :: IO () main = ignoreAndSayHello getStringFromStdin
\textasciitilde{}\textasciitilde{}\textasciitilde{}

Remember, \texttt{\textgreater{}\textgreater{}=} "combines" two IO objects into
one. It returns a new IO object that takes the result of the left-hand side and
uses it as an argument to the right hand side. Easy, right?

\section{Ordering}

One major implication that is apparent throughout this entire process is that
statements in Haskell have \textbf{no inherent order}. As we saw, we had a list
of declaration of many different IO actions -\/-\/- none of which were
necessarily evaluated or executed. There is no sense of "this function is
'first', this function is 'second'". Indeed, the idea of ordering makes no sense
when you think of things as mathematical functions.

While there is no "first" or "second", there is a \texttt{main}, which is the
function the compiler/interpreter passes to the runtime environment as the
computation we agree to run. "Order" arrives at this point. We explicity
"create" an IO data structure and specify the ordering implicitly with
\texttt{\textgreater{}\textgreater{}=}. More specifically, \texttt{print}
requires the result of \texttt{getStringFromStdin}, so there arises the first
semblances of "ordering": in the explicit composition of different IO actions
into one big one.

As it turns out, there is a
\href{http://chris-taylor.github.io/blog/2013/02/09/io-is-not-a-side-effect/}{nice
blog post} by Chris Taylor illustrating how this wordering ordering could be
implemented in the internal data structure of IO.

Long story short, \texttt{IO}'s interface provides features to chain and combine
IO actions into one big IO action, as we did before with
\texttt{\textgreater{}\textgreater{}=}. This interface creates dependency trees
in the internal IO data structure that enforces ordering.

But the real story is that outside of the internals of a single \texttt{IO},
there is no inherent ordering -\/-\/- not even between different \texttt{IO}
objects!

\section{Resolution}

In retrospect, the solution seems obvious. A functional program does what it
does best -\/-\/- return an object, purely. This object is the actual
computation itself, which can be pure or impure, deterministic or
nondeterministic -\/-\/- we just pass it off, and the execution environment can
do whatever it wants with it. Not our problem anymore! This is the difference
between evaluation (the pure process) and execution (the impure one).

We have the best of both worlds. Purity and...well, usefulness!

In fact, because of how Haskell is structured...an impure function does not even
make sense. How would one even write a traditional "impure" function in this
language? The language itself is centered around the idea of composing
computation instruction data types. What would an impure function even look
like? Even if it were possible, impurity would be a jarring, unnatural
adjustment to the language that doesn't even really "make sense".

More importantly, however, there isn't really any other way Haskell could handle
this and still feel Haskell. The reason for this is that this is why Haskell
succeeds where other languages struggle: Though we have only seen a glimpse of
this in this in this article, Haskell provides very specialized tools for
assembling and composing complex instruction data structures that make it
extremely simple, expressive, and elegant. Tools for combining two parsing rules
into one. Tools for combining two SQL operations into one. For a language that
handles computational data structures so well, \emph{not} handling IO this way
would be a real shame!

\subsection{Why?}

One might ask about the usefulness of this whole thing. After all, don't most
languages "compile" to the same assembly code every time? Why this game?

The reason is that we can now deal with programs -\/-\/- entire chunks of
assembly code -\/-\/- as first-class objects. You can pass in computational
instruction objects to a function the same way you can pass any normal object.
You can have two separate little "assembly code chunks" in complete
isolation...and then you can combine them if you want, as well. You can easily
introduce parallel forks -\/-\/- you can always pass in an "assembly code
chunk", so why not pass an IO object into a fork function? Every separate IO
object is self-contained and manipulatable. This fact is also true for the other
"instruction set"-like objects mentioned earlier. You can build them up and
compose them and pass them as first-class things.

And like we said before, you get all the benefits of
\href{http://u.jle.im/19JxV5S}{equational reasoning} because you're dealing with
pure "inert" compositions -\/-\/- this is something you could never get if you
dealt with executing the actual functions themselves!

\end{document}
