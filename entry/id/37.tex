\documentclass[]{article}
\usepackage{lmodern}
\usepackage{amssymb,amsmath}
\usepackage{ifxetex,ifluatex}
\usepackage{fixltx2e} % provides \textsubscript
\ifnum 0\ifxetex 1\fi\ifluatex 1\fi=0 % if pdftex
  \usepackage[T1]{fontenc}
  \usepackage[utf8]{inputenc}
\else % if luatex or xelatex
  \ifxetex
    \usepackage{mathspec}
    \usepackage{xltxtra,xunicode}
  \else
    \usepackage{fontspec}
  \fi
  \defaultfontfeatures{Mapping=tex-text,Scale=MatchLowercase}
  \newcommand{\euro}{€}
\fi
% use upquote if available, for straight quotes in verbatim environments
\IfFileExists{upquote.sty}{\usepackage{upquote}}{}
% use microtype if available
\IfFileExists{microtype.sty}{\usepackage{microtype}}{}
\usepackage[margin=1in]{geometry}
\usepackage{color}
\usepackage{fancyvrb}
\newcommand{\VerbBar}{|}
\newcommand{\VERB}{\Verb[commandchars=\\\{\}]}
\DefineVerbatimEnvironment{Highlighting}{Verbatim}{commandchars=\\\{\}}
% Add ',fontsize=\small' for more characters per line
\newenvironment{Shaded}{}{}
\newcommand{\AlertTok}[1]{\textcolor[rgb]{1.00,0.00,0.00}{\textbf{#1}}}
\newcommand{\AnnotationTok}[1]{\textcolor[rgb]{0.38,0.63,0.69}{\textbf{\textit{#1}}}}
\newcommand{\AttributeTok}[1]{\textcolor[rgb]{0.49,0.56,0.16}{#1}}
\newcommand{\BaseNTok}[1]{\textcolor[rgb]{0.25,0.63,0.44}{#1}}
\newcommand{\BuiltInTok}[1]{#1}
\newcommand{\CharTok}[1]{\textcolor[rgb]{0.25,0.44,0.63}{#1}}
\newcommand{\CommentTok}[1]{\textcolor[rgb]{0.38,0.63,0.69}{\textit{#1}}}
\newcommand{\CommentVarTok}[1]{\textcolor[rgb]{0.38,0.63,0.69}{\textbf{\textit{#1}}}}
\newcommand{\ConstantTok}[1]{\textcolor[rgb]{0.53,0.00,0.00}{#1}}
\newcommand{\ControlFlowTok}[1]{\textcolor[rgb]{0.00,0.44,0.13}{\textbf{#1}}}
\newcommand{\DataTypeTok}[1]{\textcolor[rgb]{0.56,0.13,0.00}{#1}}
\newcommand{\DecValTok}[1]{\textcolor[rgb]{0.25,0.63,0.44}{#1}}
\newcommand{\DocumentationTok}[1]{\textcolor[rgb]{0.73,0.13,0.13}{\textit{#1}}}
\newcommand{\ErrorTok}[1]{\textcolor[rgb]{1.00,0.00,0.00}{\textbf{#1}}}
\newcommand{\ExtensionTok}[1]{#1}
\newcommand{\FloatTok}[1]{\textcolor[rgb]{0.25,0.63,0.44}{#1}}
\newcommand{\FunctionTok}[1]{\textcolor[rgb]{0.02,0.16,0.49}{#1}}
\newcommand{\ImportTok}[1]{#1}
\newcommand{\InformationTok}[1]{\textcolor[rgb]{0.38,0.63,0.69}{\textbf{\textit{#1}}}}
\newcommand{\KeywordTok}[1]{\textcolor[rgb]{0.00,0.44,0.13}{\textbf{#1}}}
\newcommand{\NormalTok}[1]{#1}
\newcommand{\OperatorTok}[1]{\textcolor[rgb]{0.40,0.40,0.40}{#1}}
\newcommand{\OtherTok}[1]{\textcolor[rgb]{0.00,0.44,0.13}{#1}}
\newcommand{\PreprocessorTok}[1]{\textcolor[rgb]{0.74,0.48,0.00}{#1}}
\newcommand{\RegionMarkerTok}[1]{#1}
\newcommand{\SpecialCharTok}[1]{\textcolor[rgb]{0.25,0.44,0.63}{#1}}
\newcommand{\SpecialStringTok}[1]{\textcolor[rgb]{0.73,0.40,0.53}{#1}}
\newcommand{\StringTok}[1]{\textcolor[rgb]{0.25,0.44,0.63}{#1}}
\newcommand{\VariableTok}[1]{\textcolor[rgb]{0.10,0.09,0.49}{#1}}
\newcommand{\VerbatimStringTok}[1]{\textcolor[rgb]{0.25,0.44,0.63}{#1}}
\newcommand{\WarningTok}[1]{\textcolor[rgb]{0.38,0.63,0.69}{\textbf{\textit{#1}}}}
\usepackage{graphicx}
\makeatletter
\def\maxwidth{\ifdim\Gin@nat@width>\linewidth\linewidth\else\Gin@nat@width\fi}
\def\maxheight{\ifdim\Gin@nat@height>\textheight\textheight\else\Gin@nat@height\fi}
\makeatother
% Scale images if necessary, so that they will not overflow the page
% margins by default, and it is still possible to overwrite the defaults
% using explicit options in \includegraphics[width, height, ...]{}
\setkeys{Gin}{width=\maxwidth,height=\maxheight,keepaspectratio}
\ifxetex
  \usepackage[setpagesize=false, % page size defined by xetex
              unicode=false, % unicode breaks when used with xetex
              xetex]{hyperref}
\else
  \usepackage[unicode=true]{hyperref}
\fi
\hypersetup{breaklinks=true,
            bookmarks=true,
            pdfauthor={Justin Le},
            pdftitle={Unique sample drawing \& searches with List and StateT --- ``Send more money''},
            colorlinks=true,
            citecolor=blue,
            urlcolor=blue,
            linkcolor=magenta,
            pdfborder={0 0 0}}
\urlstyle{same}  % don't use monospace font for urls
% Make links footnotes instead of hotlinks:
\renewcommand{\href}[2]{#2\footnote{\url{#1}}}
\setlength{\parindent}{0pt}
\setlength{\parskip}{6pt plus 2pt minus 1pt}
\setlength{\emergencystretch}{3em}  % prevent overfull lines
\setcounter{secnumdepth}{0}

\title{Unique sample drawing \& searches with List and StateT --- ``Send more money''}
\author{Justin Le}
\date{April 24, 2015}

\begin{document}
\maketitle

\emph{Originally posted on
\textbf{\href{https://blog.jle.im/entry/unique-sample-drawing-searches-with-list-and-statet.html}{in
Code}}.}

Nothing too crazy today, just a cute (basic/intermediate) haskell trick as a
response to Mark Dominus's excellent \href{http://blog.plover.com}{Universe of
Discourse} post on
\href{http://blog.plover.com/prog/haskell/monad-search.html}{Easy exhaustive
search with the list monad} intended for people new or unfamiliar with haskell
demonstrating the common ``list monad as a constraint solver'' approach that is
standard fare for learning Haskell. I myself have literally done
\href{http://blog.jle.im/entries/series/+monadplus-success-failure-monads}{an
entire series of blog posts} on this usage.

Mark's use case however incorporates a bit of an extra pattern not typically
discussed. The list monad is good for taking ``independent samples'' of things
(looking at different samples from a list):

\begin{Shaded}
\begin{Highlighting}[]
\NormalTok{ghci}\FunctionTok{>} \KeywordTok{do}\NormalTok{ x }\OtherTok{<-} \StringTok{"abc"}
\NormalTok{         y }\OtherTok{<-} \StringTok{"abc"}
\NormalTok{         z }\OtherTok{<-} \StringTok{"abc"}
\NormalTok{         return [x,y,z]}
\NormalTok{[}\StringTok{"aaa"}\NormalTok{,}\StringTok{"aab"}\NormalTok{,}\StringTok{"aac"}\NormalTok{,}\StringTok{"aba"}\NormalTok{,}\StringTok{"abb"} \FunctionTok{...}\NormalTok{ ]}
\end{Highlighting}
\end{Shaded}

However, what if you wanted to instead ``draw'' from a pool, and represent
different drawings? Traditionally, the answer was something like:

\begin{Shaded}
\begin{Highlighting}[]
\NormalTok{ghci}\FunctionTok{>} \KeywordTok{do}\NormalTok{ x }\OtherTok{<-} \StringTok{"abc"}
\NormalTok{         y }\OtherTok{<-}\NormalTok{ filter (}\FunctionTok{/=}\NormalTok{ x) }\StringTok{"abc"}
\NormalTok{         z }\OtherTok{<-}\NormalTok{ filter (}\FunctionTok{/=}\NormalTok{ y) }\FunctionTok{.}\NormalTok{ filter (}\FunctionTok{/=}\NormalTok{ x) }\FunctionTok{$} \StringTok{"abc"}
\NormalTok{         return [x,y,z]}
\StringTok{"abc"}\NormalTok{,}\StringTok{"acb"}\NormalTok{,}\StringTok{"bac"}\NormalTok{,}\StringTok{"bca"}\NormalTok{,}\StringTok{"cab"}\NormalTok{,}\StringTok{"cba"}\NormalTok{]}
\end{Highlighting}
\end{Shaded}

This is a little bit awkward\ldots{}and it definitely gets a lot worse
(\includegraphics{https://latex.codecogs.com/png.latex?O\%28n\%5E2\%29}) when
you have more items. Also, it relies on an \texttt{Eq} constraint --- what if
our thing doesn't have an \texttt{Eq} instance? And this also falls apart when
our list contains duplicate items. If we had used \texttt{"aabc"} instead of
\texttt{"abc"}, the result would be the same --- despite having more
\texttt{\textquotesingle{}a\textquotesingle{}}s to pick from!

\textbf{Important note:} After writing this article, I found out that Douglas
Auclair in \href{https://wiki.haskell.org/wikiupload/6/6a/TMR-Issue11.pdf}{11th
issue of the Monad Reader} solved this exact same problem with pretty much the
exact same approach. (Oops!) If you want to do further reading, check it out! :D

\hypertarget{statet}{%
\section{StateT}\label{statet}}

There's a type in the \emph{transformers} library that provides a very useful
monad instance:

\begin{Shaded}
\begin{Highlighting}[]
\KeywordTok{data} \DataTypeTok{StateT}\NormalTok{ s m a }\FunctionTok{=} \DataTypeTok{StateT}\NormalTok{ (s }\OtherTok{->}\NormalTok{ m (a, s))}
\end{Highlighting}
\end{Shaded}

A \texttt{StateT\ s\ m\ a} is a function that takes an initial state \texttt{s}
and returns a result \texttt{a} with a modified state\ldots{}in the context of
\texttt{m}.

Specialize for \texttt{m\ \textasciitilde{}\ {[}{]}} and we get

\begin{Shaded}
\begin{Highlighting}[]
\KeywordTok{data} \DataTypeTok{StateT}\NormalTok{ s [] a }\FunctionTok{=} \DataTypeTok{StateT}\NormalTok{ (s }\OtherTok{->}\NormalTok{ [(a, s)])}
\end{Highlighting}
\end{Shaded}

Which is basically describing a function from a initial state to a list of
\emph{ways you can modify the state}, and different results from each one. It
returns a list of ``all ways you can mutate this state''.

For example,

\begin{Shaded}
\begin{Highlighting}[]
\OtherTok{foo ::} \DataTypeTok{StateT} \DataTypeTok{Int}\NormalTok{ [] }\DataTypeTok{Bool}
\NormalTok{foo }\FunctionTok{=} \DataTypeTok{StateT} \FunctionTok{$}\NormalTok{ \textbackslash{}x }\OtherTok{->}\NormalTok{ [(even x, x}\FunctionTok{+}\DecValTok{1}\NormalTok{), (odd x, x}\FunctionTok{-}\DecValTok{1}\NormalTok{), (x }\FunctionTok{>} \DecValTok{0}\NormalTok{, negate x)]}
\end{Highlighting}
\end{Shaded}

So \texttt{foo} takes a number, \texttt{x}, and says, ``here are three ways we
might proceed from having this number. We can return whether or not it's even,
in which case the new state is \texttt{x+1}\ldots{}we can return whether or not
it's odd, in which case the new state is \texttt{x-1}\ldots{}.or we can return
whether or not it's positive, in which case the new state is
\texttt{negate\ x}''

What the monad instance does is that it allows you to ``chain'' forks, and go
along different forks, and gather together ``all possible forks'' you could have
taken. At the end, it outputs all possible forks. So if you did
\texttt{foo\ \textgreater{}\textgreater{}\ foo}, there'd be nine results --- one
result for when you took the first route (the \texttt{x+1}) twice, one for when
you took the first and then the second (\texttt{x-1}), one for when you took the
first and the third\ldots{}.and the second and the first\ldots{}etc., etc.

\hypertarget{monadplus}{%
\subsection{MonadPlus}\label{monadplus}}

One other tool we have at our disposal is \texttt{guard}:

\begin{Shaded}
\begin{Highlighting}[]
\OtherTok{guard ::} \DataTypeTok{Bool} \OtherTok{->} \DataTypeTok{StateT} \DataTypeTok{Int}\NormalTok{ [] ()}
\end{Highlighting}
\end{Shaded}

which is a \texttt{StateT} action that says ``kill this current branch if given
\texttt{False}, or go on if given \texttt{True}''

\hypertarget{the-problem}{%
\section{The Problem}\label{the-problem}}

The problem, as stated, was to find distinct digits for each letter to solve:

\begin{verbatim}
    S E N D
+   M O R E
-----------
  M O N E Y
\end{verbatim}

So \texttt{SEND} is a four-digit number, \texttt{MORE} is a four-digit number,
and \texttt{MONEY} is a five-digit number that is the sum of the two. The first
digit of \texttt{MONEY} has to be the first digit of \texttt{MORE}, the last
digit of \texttt{MORE} has to be the second digit of \texttt{SEND}, etc.

The previous approach was done using the entire ``pick from all
possibilities\ldots{}except for the ones already chosen'', using \texttt{(/=)}
and filtering over all of the things seen vs all of the things to pick from.

However, we can abstract over ``picking dependently from a sample'' by defining
a function called \texttt{select}, which really should be in the base libraries
but isn't for some reason:

\begin{Shaded}
\begin{Highlighting}[]
\CommentTok{-- source: https://github.com/mstksg/inCode/tree/master/code-samples/misc/send-more-money.hs#L7-L9}

\OtherTok{select ::}\NormalTok{ [a] }\OtherTok{->}\NormalTok{ [(a, [a])]}
\NormalTok{select []     }\FunctionTok{=}\NormalTok{ []}
\NormalTok{select (x}\FunctionTok{:}\NormalTok{xs) }\FunctionTok{=}\NormalTok{ (x,xs) }\FunctionTok{:}\NormalTok{ [(y,x}\FunctionTok{:}\NormalTok{ys) }\FunctionTok{|}\NormalTok{ (y,ys) }\OtherTok{<-}\NormalTok{ select xs]}
\end{Highlighting}
\end{Shaded}

(Implementation thanks to Cale, who has fought valiantly yet fruitlessly to get
this into base for many years.)

\texttt{select} will take a list \texttt{{[}a{]}} and return a list of different
``selected'' \texttt{a}s, with the rest of the list, too:

\begin{Shaded}
\begin{Highlighting}[]
\NormalTok{ghci}\FunctionTok{>}\NormalTok{ select }\StringTok{"abcd"}
\NormalTok{[(}\CharTok{'a'}\NormalTok{,}\StringTok{"bcd"}\NormalTok{),(}\CharTok{'b'}\NormalTok{,}\StringTok{"acd"}\NormalTok{),(}\CharTok{'c'}\NormalTok{,}\StringTok{"abd"}\NormalTok{),(}\CharTok{'d'}\NormalTok{,}\StringTok{"abc"}\NormalTok{)]}
\end{Highlighting}
\end{Shaded}

But, hey\ldots{}does the type signature of \texttt{select} look like anything
familiar?

It looks \emph{exactly} like something that \texttt{StateT} is supposed to
describe! Give an initial state (\texttt{{[}a{]}}), and returns a list of all
possible ways to ``mutate'' that state (by removing one element from the state),
and a result from each mutation (the removed element).

\begin{Shaded}
\begin{Highlighting}[]
\DataTypeTok{StateT}\OtherTok{ select ::} \DataTypeTok{StateT}\NormalTok{ [a] [] a}
\end{Highlighting}
\end{Shaded}

And armed with this\ldots{}we have all we need

\begin{Shaded}
\begin{Highlighting}[]
\CommentTok{-- source: https://github.com/mstksg/inCode/tree/master/code-samples/misc/send-more-money.hs#L3-L35}

\KeywordTok{import} \DataTypeTok{Control.Monad}\NormalTok{             (guard, mfilter)}
\KeywordTok{import} \DataTypeTok{Control.Monad.Trans.State}
\KeywordTok{import} \DataTypeTok{Data.List}\NormalTok{                 (foldl')}

\OtherTok{asNumber ::}\NormalTok{ [}\DataTypeTok{Int}\NormalTok{] }\OtherTok{->} \DataTypeTok{Int}
\NormalTok{asNumber }\FunctionTok{=}\NormalTok{ foldl' (\textbackslash{}t o }\OtherTok{->}\NormalTok{ t}\FunctionTok{*}\DecValTok{10} \FunctionTok{+}\NormalTok{ o) }\DecValTok{0}

\OtherTok{main ::} \DataTypeTok{IO}\NormalTok{ ()}
\NormalTok{main }\FunctionTok{=}\NormalTok{ print }\FunctionTok{.}\NormalTok{ flip evalStateT [}\DecValTok{0}\FunctionTok{..}\DecValTok{9}\NormalTok{] }\FunctionTok{$} \KeywordTok{do}
\NormalTok{    s }\OtherTok{<-} \DataTypeTok{StateT}\NormalTok{ select}
\NormalTok{    e }\OtherTok{<-} \DataTypeTok{StateT}\NormalTok{ select}
\NormalTok{    n }\OtherTok{<-} \DataTypeTok{StateT}\NormalTok{ select}
\NormalTok{    d }\OtherTok{<-} \DataTypeTok{StateT}\NormalTok{ select}
\NormalTok{    m }\OtherTok{<-} \DataTypeTok{StateT}\NormalTok{ select}
\NormalTok{    o }\OtherTok{<-} \DataTypeTok{StateT}\NormalTok{ select}
\NormalTok{    r }\OtherTok{<-} \DataTypeTok{StateT}\NormalTok{ select}
\NormalTok{    y }\OtherTok{<-} \DataTypeTok{StateT}\NormalTok{ select}
\NormalTok{    guard }\FunctionTok{$}\NormalTok{ s }\FunctionTok{/=} \DecValTok{0} \FunctionTok{&&}\NormalTok{ m }\FunctionTok{/=} \DecValTok{0}
    \KeywordTok{let}\NormalTok{ send  }\FunctionTok{=}\NormalTok{ asNumber [s,e,n,d]}
\NormalTok{        more  }\FunctionTok{=}\NormalTok{ asNumber [m,o,r,e]}
\NormalTok{        money }\FunctionTok{=}\NormalTok{ asNumber [m,o,n,e,y]}
\NormalTok{    guard }\FunctionTok{$}\NormalTok{ send }\FunctionTok{+}\NormalTok{ more }\FunctionTok{==}\NormalTok{ money}
\NormalTok{    return (send, more, money)}
\end{Highlighting}
\end{Shaded}

Remember, \texttt{StateT} here operates with an underlying state of
\texttt{{[}Int{]}}, a list of numbers not yet picked. \texttt{StateT\ select}
picks one of these numbers, and modifies the state to now only include the items
that were not picked. So every time you sequence \texttt{StateT\ select},
\texttt{select} draws from a smaller and smaller pool of numbers, and makes the
state list smaller and smaller. What sequencing \texttt{StateT} does is allow us
to explore \emph{all} of the possible ways we could pick and modify state, all
at once. Using \texttt{guard}, we then ``close off'' and kill off the paths that
don't end up how we'd like.

\texttt{asNumber} takes a list like \texttt{{[}1,2,3{]}} and turns it into the
number \texttt{123}; credit to the source blog.

And, to test it out\ldots{}

\begin{Shaded}
\begin{Highlighting}[]
\NormalTok{$ }\ExtensionTok{ghc}\NormalTok{ -O2 send-more-money.hs}
\NormalTok{$ }\ExtensionTok{./send-more-money}
\CommentTok{# [(9567,1085,10652)]}
\end{Highlighting}
\end{Shaded}

It returns the one and only solution, \texttt{SEND\ =\ 9567},
\texttt{MORE\ =\ 1085}, and \texttt{MONEY\ =\ 10652}.\footnote{For some reason
  this runs pretty slowly if you use \texttt{runghc}/\texttt{runHaskell}, but it
  runs in the blink of an eye when you actually compile it (and especially with
  optimizations on). The difference is pretty striking\ldots{}and I don't really
  know what's going on here, to be honest. If anyone does know a good
  explanation, I'd love to hear it :)}

\begin{center}\rule{0.5\linewidth}{\linethickness}\end{center}

\textbf{Aside}

We can make things a little bit more efficient with minimal cost in
expressiveness. But not that it matters\ldots{}the original version runs fast
already.

\begin{Shaded}
\begin{Highlighting}[]
\CommentTok{-- source: https://github.com/mstksg/inCode/tree/master/code-samples/misc/send-more-money.hs#L38-L59}

\OtherTok{select' ::}\NormalTok{ [a] }\OtherTok{->}\NormalTok{ [(a,[a])]}
\NormalTok{select' }\FunctionTok{=}\NormalTok{ go []}
  \KeywordTok{where}
\NormalTok{   go xs [] }\FunctionTok{=}\NormalTok{ []}
\NormalTok{   go xs (y}\FunctionTok{:}\NormalTok{ys) }\FunctionTok{=}\NormalTok{ (y,xs}\FunctionTok{++}\NormalTok{ys) }\FunctionTok{:}\NormalTok{ go (y}\FunctionTok{:}\NormalTok{xs) ys}

\OtherTok{main' ::} \DataTypeTok{IO}\NormalTok{ ()}
\NormalTok{main' }\FunctionTok{=}\NormalTok{ print }\FunctionTok{.}\NormalTok{ flip evalStateT [}\DecValTok{0}\FunctionTok{..}\DecValTok{9}\NormalTok{] }\FunctionTok{$} \KeywordTok{do}
\NormalTok{    s }\OtherTok{<-}\NormalTok{ mfilter (}\FunctionTok{/=} \DecValTok{0}\NormalTok{) }\FunctionTok{$} \DataTypeTok{StateT}\NormalTok{ select'}
\NormalTok{    m }\OtherTok{<-}\NormalTok{ mfilter (}\FunctionTok{/=} \DecValTok{0}\NormalTok{) }\FunctionTok{$} \DataTypeTok{StateT}\NormalTok{ select'}
\NormalTok{    e }\OtherTok{<-} \DataTypeTok{StateT}\NormalTok{ select'}
\NormalTok{    n }\OtherTok{<-} \DataTypeTok{StateT}\NormalTok{ select'}
\NormalTok{    d }\OtherTok{<-} \DataTypeTok{StateT}\NormalTok{ select'}
\NormalTok{    o }\OtherTok{<-} \DataTypeTok{StateT}\NormalTok{ select'}
\NormalTok{    r }\OtherTok{<-} \DataTypeTok{StateT}\NormalTok{ select'}
\NormalTok{    y }\OtherTok{<-} \DataTypeTok{StateT}\NormalTok{ select'}
    \KeywordTok{let}\NormalTok{ send  }\FunctionTok{=}\NormalTok{ asNumber [s,e,n,d]}
\NormalTok{        more  }\FunctionTok{=}\NormalTok{ asNumber [m,o,r,e]}
\NormalTok{        money }\FunctionTok{=}\NormalTok{ asNumber [m,o,n,e,y]}
\NormalTok{    guard }\FunctionTok{$}\NormalTok{ send }\FunctionTok{+}\NormalTok{ more }\FunctionTok{==}\NormalTok{ money}
\NormalTok{    return (send, more, money)}
\end{Highlighting}
\end{Shaded}

This is a more performant version of \texttt{select}
\href{http://chimera.labs.oreilly.com/books/1230000000929/pr01.html}{courtesy of
Simon Marlow} that doesn't preserve the order of the ``rest of the elements''.

Also, we use \texttt{mfilter} to ``eliminate bad \texttt{s} and \texttt{m}s''
right off the bat, before having to pick any more things. \texttt{mfilter} can
be thought of as ``killing the fork immediately'' if the action doesn't satisfy
the predicate. If the \texttt{s} picked doesn't match \texttt{(/=\ 0)}, then the
entire branch/fork is immediately ruled invalid.

\begin{center}\rule{0.5\linewidth}{\linethickness}\end{center}

By the way, isn't it neat that it does all of this in ``constant space''? It
just keeps track of the output list, but the actual search processes is in
constant space. You don't need to keep track of all \texttt{10!} combinations in
memory at once. Hooray laziness!

\hypertarget{other-applications}{%
\section{Other Applications}\label{other-applications}}

Using \texttt{select} and \texttt{StateT}, we can do a lot of things involving
picking from a sample, or permutations. Anything that you used to awkwardly do
by using filter not-equal-to's can work now. You can do things like drawing from
a deck:

\begin{Shaded}
\begin{Highlighting}[]
\OtherTok{pokerGame ::}\NormalTok{ [}\DataTypeTok{Ordering}\NormalTok{]}
\NormalTok{pokerGame }\FunctionTok{=}\NormalTok{ flip evalStateT [}\DecValTok{0}\FunctionTok{..}\DecValTok{51}\NormalTok{] }\FunctionTok{$} \KeywordTok{do}
\NormalTok{    p2Hand }\OtherTok{<-}\NormalTok{ replicateM }\DecValTok{5}\NormalTok{ (}\DataTypeTok{StateT}\NormalTok{ select)}
\NormalTok{    p1Hand }\OtherTok{<-}\NormalTok{ replicateM }\DecValTok{5}\NormalTok{ (}\DataTypeTok{StateT}\NormalTok{ select)}
\NormalTok{    return }\FunctionTok{$}\NormalTok{ pokerCompare p1Hand p2Hand}
\end{Highlighting}
\end{Shaded}

Which would draw five distinct cards from a deck of \texttt{{[}0..51{]}}, and
return who won for each draw (assuming you had a suitable
\texttt{pokerCompare\ ::\ {[}Card{]}\ -\textgreater{}\ {[}Card{]}\ -\textgreater{}\ Ordering}).
Note that if you use \texttt{runStateT}, you'd get the results (the winner),
\emph{as well as} the leftover cards in the deck for each path!

You can even combine the two sorts of drawings --- sampling independently (like
rolling dice) using \texttt{lift}, and drawing from an underlying deck. For
example, you might encode some game logic from a board game like monopoly:

\begin{Shaded}
\begin{Highlighting}[]
\NormalTok{combo }\FunctionTok{=}\NormalTok{ flip evalStateT initialDeck }\FunctionTok{$} \KeywordTok{do}
\NormalTok{    roll }\OtherTok{<-}\NormalTok{ lift [}\DecValTok{1}\FunctionTok{..}\DecValTok{6}\NormalTok{]}
\NormalTok{    draw }\OtherTok{<-} \DataTypeTok{StateT}\NormalTok{ select}
    \FunctionTok{...}
\end{Highlighting}
\end{Shaded}

Whenever you want a dice roll, use \texttt{lift\ {[}1..6{]}}\ldots{}and whenever
you want to draw from the deck, use \texttt{StateT\ select}.

What you get in the end, remember, is a list of ``all possible paths''. You'll
get a list of every possible result from all of your different rolling and
drawing choices.

Happy Haskelling!

\hypertarget{signoff}{%
\section{Signoff}\label{signoff}}

Hi, thanks for reading! You can reach me via email at
\href{mailto:justin@jle.im}{\nolinkurl{justin@jle.im}}, or at twitter at
\href{https://twitter.com/mstk}{@mstk}! This post and all others are published
under the \href{https://creativecommons.org/licenses/by-nc-nd/3.0/}{CC-BY-NC-ND
3.0} license. Corrections and edits via pull request are welcome and encouraged
at \href{https://github.com/mstksg/inCode}{the source repository}.

If you feel inclined, or this post was particularly helpful for you, why not
consider \href{https://www.patreon.com/justinle/overview}{supporting me on
Patreon}, or a \href{bitcoin:3D7rmAYgbDnp4gp4rf22THsGt74fNucPDU}{BTC donation}?
:)

\end{document}
