\documentclass[]{article}
\usepackage{lmodern}
\usepackage{amssymb,amsmath}
\usepackage{ifxetex,ifluatex}
\usepackage{fixltx2e} % provides \textsubscript
\ifnum 0\ifxetex 1\fi\ifluatex 1\fi=0 % if pdftex
  \usepackage[T1]{fontenc}
  \usepackage[utf8]{inputenc}
\else % if luatex or xelatex
  \ifxetex
    \usepackage{mathspec}
    \usepackage{xltxtra,xunicode}
  \else
    \usepackage{fontspec}
  \fi
  \defaultfontfeatures{Mapping=tex-text,Scale=MatchLowercase}
  \newcommand{\euro}{€}
\fi
% use upquote if available, for straight quotes in verbatim environments
\IfFileExists{upquote.sty}{\usepackage{upquote}}{}
% use microtype if available
\IfFileExists{microtype.sty}{\usepackage{microtype}}{}
\usepackage[margin=1in]{geometry}
\usepackage{color}
\usepackage{fancyvrb}
\newcommand{\VerbBar}{|}
\newcommand{\VERB}{\Verb[commandchars=\\\{\}]}
\DefineVerbatimEnvironment{Highlighting}{Verbatim}{commandchars=\\\{\}}
% Add ',fontsize=\small' for more characters per line
\newenvironment{Shaded}{}{}
\newcommand{\AlertTok}[1]{\textcolor[rgb]{1.00,0.00,0.00}{\textbf{#1}}}
\newcommand{\AnnotationTok}[1]{\textcolor[rgb]{0.38,0.63,0.69}{\textbf{\textit{#1}}}}
\newcommand{\AttributeTok}[1]{\textcolor[rgb]{0.49,0.56,0.16}{#1}}
\newcommand{\BaseNTok}[1]{\textcolor[rgb]{0.25,0.63,0.44}{#1}}
\newcommand{\BuiltInTok}[1]{#1}
\newcommand{\CharTok}[1]{\textcolor[rgb]{0.25,0.44,0.63}{#1}}
\newcommand{\CommentTok}[1]{\textcolor[rgb]{0.38,0.63,0.69}{\textit{#1}}}
\newcommand{\CommentVarTok}[1]{\textcolor[rgb]{0.38,0.63,0.69}{\textbf{\textit{#1}}}}
\newcommand{\ConstantTok}[1]{\textcolor[rgb]{0.53,0.00,0.00}{#1}}
\newcommand{\ControlFlowTok}[1]{\textcolor[rgb]{0.00,0.44,0.13}{\textbf{#1}}}
\newcommand{\DataTypeTok}[1]{\textcolor[rgb]{0.56,0.13,0.00}{#1}}
\newcommand{\DecValTok}[1]{\textcolor[rgb]{0.25,0.63,0.44}{#1}}
\newcommand{\DocumentationTok}[1]{\textcolor[rgb]{0.73,0.13,0.13}{\textit{#1}}}
\newcommand{\ErrorTok}[1]{\textcolor[rgb]{1.00,0.00,0.00}{\textbf{#1}}}
\newcommand{\ExtensionTok}[1]{#1}
\newcommand{\FloatTok}[1]{\textcolor[rgb]{0.25,0.63,0.44}{#1}}
\newcommand{\FunctionTok}[1]{\textcolor[rgb]{0.02,0.16,0.49}{#1}}
\newcommand{\ImportTok}[1]{#1}
\newcommand{\InformationTok}[1]{\textcolor[rgb]{0.38,0.63,0.69}{\textbf{\textit{#1}}}}
\newcommand{\KeywordTok}[1]{\textcolor[rgb]{0.00,0.44,0.13}{\textbf{#1}}}
\newcommand{\NormalTok}[1]{#1}
\newcommand{\OperatorTok}[1]{\textcolor[rgb]{0.40,0.40,0.40}{#1}}
\newcommand{\OtherTok}[1]{\textcolor[rgb]{0.00,0.44,0.13}{#1}}
\newcommand{\PreprocessorTok}[1]{\textcolor[rgb]{0.74,0.48,0.00}{#1}}
\newcommand{\RegionMarkerTok}[1]{#1}
\newcommand{\SpecialCharTok}[1]{\textcolor[rgb]{0.25,0.44,0.63}{#1}}
\newcommand{\SpecialStringTok}[1]{\textcolor[rgb]{0.73,0.40,0.53}{#1}}
\newcommand{\StringTok}[1]{\textcolor[rgb]{0.25,0.44,0.63}{#1}}
\newcommand{\VariableTok}[1]{\textcolor[rgb]{0.10,0.09,0.49}{#1}}
\newcommand{\VerbatimStringTok}[1]{\textcolor[rgb]{0.25,0.44,0.63}{#1}}
\newcommand{\WarningTok}[1]{\textcolor[rgb]{0.38,0.63,0.69}{\textbf{\textit{#1}}}}
\usepackage{graphicx}
\makeatletter
\def\maxwidth{\ifdim\Gin@nat@width>\linewidth\linewidth\else\Gin@nat@width\fi}
\def\maxheight{\ifdim\Gin@nat@height>\textheight\textheight\else\Gin@nat@height\fi}
\makeatother
% Scale images if necessary, so that they will not overflow the page
% margins by default, and it is still possible to overwrite the defaults
% using explicit options in \includegraphics[width, height, ...]{}
\setkeys{Gin}{width=\maxwidth,height=\maxheight,keepaspectratio}
\ifxetex
  \usepackage[setpagesize=false, % page size defined by xetex
              unicode=false, % unicode breaks when used with xetex
              xetex]{hyperref}
\else
  \usepackage[unicode=true]{hyperref}
\fi
\hypersetup{breaklinks=true,
            bookmarks=true,
            pdfauthor={Justin Le},
            pdftitle={A (Dead End?) Arrowized Dataflow Parallelism Interface Attempt},
            colorlinks=true,
            citecolor=blue,
            urlcolor=blue,
            linkcolor=magenta,
            pdfborder={0 0 0}}
\urlstyle{same}  % don't use monospace font for urls
% Make links footnotes instead of hotlinks:
\renewcommand{\href}[2]{#2\footnote{\url{#1}}}
\setlength{\parindent}{0pt}
\setlength{\parskip}{6pt plus 2pt minus 1pt}
\setlength{\emergencystretch}{3em}  % prevent overfull lines
\setcounter{secnumdepth}{0}

\title{A (Dead End?) Arrowized Dataflow Parallelism Interface Attempt}
\author{Justin Le}
\date{April 3, 2014}

\begin{document}
\maketitle

\emph{Originally posted on
\textbf{\href{https://blog.jle.im/entry/a-dead-end-arrowized-dataflow-parallelism-interface-attempt.html}{in
Code}}.}

So I've been having several `dead end' projects in Haskell recently that I've
sort of just scrapped and move from, but I decided that it might be nice to
document some of them :) For reading back on it later, for people to possibly
help/offer suggestions, for people to peruse and possibly learn something, or
for people to laugh at me. Here is my most recent semi-failure --- implicit
dataflow parallelism through an Arrow interface.

tl;dr:

\begin{enumerate}
\def\labelenumi{\arabic{enumi}.}
\tightlist
\item
  Compose parallelizable computations using expressive proc notation.
\item
  Consolidate and join forks to maintain maximum parallelization.
\item
  All data dependencies implicit; allows for nice succinct direct translations
  of normal functions.
\item
  All ``parallelizable'' functions can also trivially be typechecked and run as
  normal functions, due to arrow polymorphism.
\end{enumerate}

The main problem:

\begin{itemize}
\tightlist
\item
  Consider \texttt{ParArrow\ a\ c}, \texttt{ParArrow\ b\ d},
  \texttt{ParArrow\ (c,d)\ (e,f)}, \texttt{ParArrow\ e\ g}, and
  \texttt{ParArrow\ f\ h}. We execute the first two in parallel, apply the
  third, and execute the second two in parallel. Basically, we want two
  independent \texttt{ParArrow\ a\ g} and \texttt{ParArrow\ c\ h} that we can
  fork. And this is possible, as long as the ``middle'' arrow does not
  ``cross-talk'' --- that is, it can't be something like
  \texttt{arr\ (\textbackslash{}(x,y)\ -\textgreater{}\ (y,x))}.
\end{itemize}

\hypertarget{the-vision}{%
\section{The Vision}\label{the-vision}}

So what do I mean?

\hypertarget{dataflow-parallelism}{%
\subsection{Dataflow Parallelism}\label{dataflow-parallelism}}

By ``dataflow parallelism'', I refer to structuring parallel computations by
``what depends on what''. If two values \emph{can} be computed in parallel, then
that is taken advantage of. Consider something like \texttt{map\ f\ xs}.
Normally, this would: one by one step over \texttt{xs} and apply \texttt{f} to
each one, building a new list as you go along one at a time.

But note that there are some easy places to parallelize this --- because none of
the results the mapped list depend on eachother, you can apply \texttt{f} to
every element in parallel, and re-collect everything back at the end. And this
is a big deal if \texttt{f} takes a long time. This is an example of something
commonly refered to as ``embarassingly parallel''.

\hypertarget{arrows}{%
\subsection{Arrows}\label{arrows}}

So what kind of Arrow interface am I imagining with this?

Haskell has some nice syntax for composing ``functions'' (\texttt{f},
\texttt{g}, and \texttt{h}):

\begin{Shaded}
\begin{Highlighting}[]
\NormalTok{proc x }\OtherTok{->} \KeywordTok{do}
\NormalTok{    y }\OtherTok{<-}\NormalTok{ f }\FunctionTok{-<}\NormalTok{ x}
\NormalTok{    z }\OtherTok{<-}\NormalTok{ g }\FunctionTok{-<}\NormalTok{ x}
\NormalTok{    q }\OtherTok{<-}\NormalTok{ h }\FunctionTok{-<}\NormalTok{ y}
\NormalTok{    returnA }\FunctionTok{-<}\NormalTok{ y }\FunctionTok{*}\NormalTok{ z }\FunctionTok{+}\NormalTok{ q}
\end{Highlighting}
\end{Shaded}

A \texttt{proc} statement is a fancy lambda, which takes an input \texttt{x} and
``funnels'' \texttt{x} through several different ``functions'' --- in our case,
\texttt{f}, \texttt{g}, and \texttt{h} --- and lets you name the results so that
you can use them later.

\begin{figure}
\centering
\includegraphics{/img/entries/pararrow/proc1.png}
\caption{The above proc statement, diagrammed.}
\end{figure}

While this looks like you are `performing' \texttt{f}, then \texttt{g}, then
\texttt{h}, what is actually happening is that you are \emph{composing} and
synthesizing a \emph{new function}. You are ``assembling'' a new function that,
when given an \texttt{x}, collects the results of \texttt{x} run through
\texttt{f}, \texttt{g}, and \texttt{h}, and pops out a function of what comes
out of those functions.

Except\ldots{}\texttt{f}, \texttt{g}, and \texttt{h} don't have to be normal
functions. They are ``generalized'' functions; functions that could perhaps even
have side-effects, or trigger special things, or be evaluated in special ways.
They are instances of the \texttt{Arrow} typeclass.

An \texttt{Arrow\ a\ b} just represents, abstractly, a way to get some
\texttt{b} from some \texttt{a}, equipped with combinators that allow you to
compose them in neat ways. Proc notation allows us to assemble a giant new
arrow, from sequencing and composing smaller arrows.

\hypertarget{forking-arrows}{%
\subsection{Forking Arrows}\label{forking-arrows}}

Look at the proc statement and tell me that that doesn't scream ``data
parallelism'' to you. Because every arrow \texttt{f}, \texttt{g}, and \texttt{h}
can potentially do side-effecty, stateful, IO things, depending on how we
implemented the arrow\ldots{}what if \texttt{f}, \texttt{g}, and \texttt{h}
represented ``a way to get a \texttt{b} from an \texttt{a}\ldots{}in its own
separate thread''?

So if I were to ``run'' this special arrow, a \texttt{ParArrow\ a\ b}, I would
do

\begin{Shaded}
\begin{Highlighting}[]
\OtherTok{runPar ::} \DataTypeTok{ParArrow}\NormalTok{ a b }\OtherTok{->}\NormalTok{ a }\OtherTok{->} \DataTypeTok{IO}\NormalTok{ b}
\end{Highlighting}
\end{Shaded}

Where if i gave \texttt{runPar} a \texttt{ParArrow\ a\ b}, and an \texttt{a}, It
would fork itself into its own thread and give you an \texttt{IO\ b} in response
to your \texttt{a}.

Because of Arrow's ability to ``separate out'' and ``side-chain'' compositions
(note that \texttt{q} in the previous example does not depend on \texttt{z} at
all, and can clearly be launched in parallel alongside the calculation of
\texttt{z}), it looks like from a \texttt{proc} notation statement, we can
easily write arrows that all `fork themselves' under composition.

Using this, in the above proc example with the fancy diagram, we should be able
to see that \texttt{z} is completely independent of \texttt{y} and \texttt{q},
so the \texttt{g} arrow could really compute itself ``in parallel'', forked-off,
from the \texttt{f} and \texttt{h} arrows.

You should also be able to ``join together'' parallel computations. That is, if
you have an \texttt{a\ -\textgreater{}\ c} and a \texttt{b\ -\textgreater{}\ d},
you could make a ``parallel'' \texttt{(a,b)\ -\textgreater{}\ (c,d)}. But what
if I also had a \texttt{c\ -\textgreater{}\ e} and a
\texttt{d\ -\textgreater{}\ f}? I could chain the entire
\texttt{a}-\texttt{c}-\texttt{e} chain and the \texttt{b}-\texttt{d}-\texttt{f}
chain, and perform both chains in parallel and re-collect things at the end.
That is, a \texttt{(a,b)\ -\textgreater{}\ (c,d)} and a
\texttt{(c,d)\ -\textgreater{}\ (e,f)} should meaningfully compose into a
\texttt{(a,b)\ -\textgreater{}\ (e,f)}, where the left and right sides (the
\texttt{a\ -\textgreater{}\ e} and the \texttt{b\ -\textgreater{}\ f}) are
performed ``in parallel'' from eachother.

With that in mind, we could even do something like \texttt{parMap}:

\begin{Shaded}
\begin{Highlighting}[]
\OtherTok{parMap ::} \DataTypeTok{ParArrow}\NormalTok{ a b }\OtherTok{->} \DataTypeTok{ParArrow}\NormalTok{ [a] [b]}
\NormalTok{parMap f }\FunctionTok{=}\NormalTok{ proc input }\OtherTok{->} \KeywordTok{do}
    \KeywordTok{case}\NormalTok{ input }\KeywordTok{of}
\NormalTok{      []     }\OtherTok{->}
\NormalTok{          returnA        }\FunctionTok{-<}\NormalTok{ []}
\NormalTok{      (x}\FunctionTok{:}\NormalTok{xs) }\OtherTok{->} \KeywordTok{do}
\NormalTok{          y  }\OtherTok{<-}\NormalTok{ f        }\FunctionTok{-<}\NormalTok{ x}
\NormalTok{          ys }\OtherTok{<-}\NormalTok{ parMap f }\FunctionTok{-<}\NormalTok{ xs}
\NormalTok{          returnA        }\FunctionTok{-<}\NormalTok{ y}\FunctionTok{:}\NormalTok{ys}
\end{Highlighting}
\end{Shaded}

And because ``what depends on what'' is so \emph{obviously clear} from proc/do
notation --- you know exactly what depends on what, and the graph is already
laid out there for you --- and because \texttt{f} is actaully a ``smart''
function, with ``smart'' semantics which can do things like fork threads to
solve itself\ldots{}this should be great way to structure programs and take
advantage of implicit data parallelism.

\hypertarget{the-coolest-thing}{%
\subsubsection{The coolest thing}\label{the-coolest-thing}}

Also notice something cool -- if leave our proc blocks polymorphic:

\begin{Shaded}
\begin{Highlighting}[]
\OtherTok{map' ::} \DataTypeTok{ArrowChoice}\NormalTok{ r }\OtherTok{=>}\NormalTok{ r a b }\OtherTok{->}\NormalTok{ r [a] [b]}
\NormalTok{map' f }\FunctionTok{=}\NormalTok{ proc input }\OtherTok{->} \KeywordTok{do}
    \KeywordTok{case}\NormalTok{ input }\KeywordTok{of}
\NormalTok{      []     }\OtherTok{->}
\NormalTok{          returnA        }\FunctionTok{-<}\NormalTok{ []}
\NormalTok{      (x}\FunctionTok{:}\NormalTok{xs) }\OtherTok{->} \KeywordTok{do}
\NormalTok{          y  }\OtherTok{<-}\NormalTok{ f        }\FunctionTok{-<}\NormalTok{ x}
\NormalTok{          ys }\OtherTok{<-}\NormalTok{ map' f   }\FunctionTok{-<}\NormalTok{ xs}
\NormalTok{          returnA        }\FunctionTok{-<}\NormalTok{ y}\FunctionTok{:}\NormalTok{ys}
\end{Highlighting}
\end{Shaded}

We can now use \texttt{map\textquotesingle{}} as \emph{both} a normal,
sequentual function \emph{and} a parallel, forked computation!

\begin{Shaded}
\begin{Highlighting}[]
\NormalTok{λ}\FunctionTok{:}\NormalTok{          map' (arr (}\FunctionTok{*}\DecValTok{2}\NormalTok{)) [}\DecValTok{1}\FunctionTok{..}\DecValTok{5}\NormalTok{]}
\NormalTok{[}\DecValTok{2}\NormalTok{,}\DecValTok{4}\NormalTok{,}\DecValTok{6}\NormalTok{,}\DecValTok{8}\NormalTok{,}\DecValTok{10}\NormalTok{]}
\NormalTok{λ}\FunctionTok{:}\NormalTok{ runPar }\FunctionTok{$}\NormalTok{ map' (arr (}\FunctionTok{*}\DecValTok{2}\NormalTok{)) [}\DecValTok{1}\FunctionTok{..}\DecValTok{5}\NormalTok{]}
\NormalTok{[}\DecValTok{2}\NormalTok{,}\DecValTok{4}\NormalTok{,}\DecValTok{6}\NormalTok{,}\DecValTok{8}\NormalTok{,}\DecValTok{10}\NormalTok{]}
\end{Highlighting}
\end{Shaded}

Yup!

Let's try implementing it, and let's see where things go wrong.

\hypertarget{pararrow}{%
\section{ParArrow}\label{pararrow}}

\hypertarget{data-and-instances}{%
\subsection{Data and Instances}\label{data-and-instances}}

Let's start out with our arrow data type:

\begin{Shaded}
\begin{Highlighting}[]
\CommentTok{-- source: https://github.com/mstksg/inCode/tree/master/code-samples/pararrow/ParArrow.hs#L12-L18}

\KeywordTok{data} \DataTypeTok{ParArrow}\NormalTok{ a b }\FunctionTok{=}                     \DataTypeTok{Pure}\NormalTok{  (a }\OtherTok{->}\NormalTok{ b)}
                  \FunctionTok{|}\NormalTok{ forall z}\FunctionTok{.}           \DataTypeTok{Seq}\NormalTok{   (}\DataTypeTok{ParArrow}\NormalTok{ a z)}
\NormalTok{                                              (}\DataTypeTok{ParArrow}\NormalTok{ z b)}
                  \FunctionTok{|}\NormalTok{ forall a1 a2 b1 b2}\FunctionTok{.} \DataTypeTok{Par}\NormalTok{   (a }\OtherTok{->}\NormalTok{ (a1, a2))}
\NormalTok{                                              (}\DataTypeTok{ParArrow}\NormalTok{ a1 b1)}
\NormalTok{                                              (}\DataTypeTok{ParArrow}\NormalTok{ a2 b2)}
\NormalTok{                                              ((b1, b2) }\OtherTok{->}\NormalTok{ b)}
\end{Highlighting}
\end{Shaded}

So a \texttt{ParArrow\ a\ b} represents a (pure) paralleizable, forkable
computation that returns a \texttt{b} (as \texttt{IO\ b}) when given an
\texttt{a}.\footnote{Technically, all \texttt{ParArrow} computations are pure,
  so you might not loose too much by just returning a \texttt{b} instead of an
  \texttt{IO\ b} with \texttt{unsafePerformIO}, but\ldots{}}

\begin{itemize}
\item
  \texttt{Pure\ f} wraps a pure function in a \texttt{ParArrow} that computes
  that function in a fork when necessary.
\item
  \texttt{Seq\ f\ g} sequences a \texttt{ParArrow\ a\ z} and a
  \texttt{ParArrow\ z\ b} into a big \texttt{ParArrow\ a\ b}. It reprensents
  composing two forkable functions into one big forkable function, sequentially.
\item
  \texttt{Par\ l\ f\ g\ r} takes two \texttt{ParArrow}s \texttt{f} and
  \texttt{g} of different types and represents the idea of performing them in
  parallel. Of forking them off from eachother and computing them independently,
  and collecting it all together.

  \texttt{l} and \texttt{r} are supposed to be functions that turn the tupled
  inputs/outputs of the parallel computations and makes them fit
  \texttt{ParArrow\ a\ b}. \texttt{r} is kind of supposed to be \texttt{id}, and
  \texttt{l} is supposed to be \texttt{id} (to continue a parallel action) or
  \texttt{\textbackslash{}x\ -\textgreater{}\ (x,x)} (to begin a fork).

  It's a little hacky, and there might be a better way with GADT's and all sorts
  of type/kind-level magic, but it was the way I found that I understood the
  most.

  The main purpose of \texttt{l} and \texttt{r} is to be able to meaningfully
  refer to the two parallel \texttt{ParArrow}s in terms of the \texttt{a} and
  \texttt{b} of the ``combined'' \texttt{ParArrow}. Otherwise, the two inputs of
  the two parallel \texttt{ParArrow}s don't have anything to do with the input
  type \texttt{a} of the combined \texttt{ParArrow}, and same for output.
\end{itemize}

Okay, let's define a Category instance, that lets us compose \texttt{ParArrow}s:

\begin{Shaded}
\begin{Highlighting}[]
\CommentTok{-- source: https://github.com/mstksg/inCode/tree/master/code-samples/pararrow/ParArrow.hs#L20-L22}

\KeywordTok{instance} \DataTypeTok{Category} \DataTypeTok{ParArrow} \KeywordTok{where}
\NormalTok{    id    }\FunctionTok{=} \DataTypeTok{Pure}\NormalTok{ id}
\NormalTok{    f }\FunctionTok{.}\NormalTok{ g }\FunctionTok{=} \DataTypeTok{Seq}\NormalTok{ g f}
\end{Highlighting}
\end{Shaded}

No surprises there, hopefully! Now an Arrow instance:

\begin{Shaded}
\begin{Highlighting}[]
\CommentTok{-- source: https://github.com/mstksg/inCode/tree/master/code-samples/pararrow/ParArrow.hs#L24-L29}

\KeywordTok{instance} \DataTypeTok{Arrow} \DataTypeTok{ParArrow} \KeywordTok{where}
\NormalTok{    arr      }\FunctionTok{=} \DataTypeTok{Pure}
\NormalTok{    first f  }\FunctionTok{=}\NormalTok{ f  }\FunctionTok{***}\NormalTok{ id}
\NormalTok{    second g }\FunctionTok{=}\NormalTok{ id }\FunctionTok{***}\NormalTok{ g}
\NormalTok{    f }\FunctionTok{&&&}\NormalTok{ g  }\FunctionTok{=} \DataTypeTok{Par}\NormalTok{ (id }\FunctionTok{&&&}\NormalTok{ id) f g id}
\NormalTok{    f }\FunctionTok{***}\NormalTok{ g  }\FunctionTok{=} \DataTypeTok{Par}\NormalTok{ id          f g id}
\end{Highlighting}
\end{Shaded}

Also simple enough. Note that \texttt{first} and \texttt{second} are defined in
terms of \texttt{(***)}, instead of the typical way of defining \texttt{second},
\texttt{(\&\&\&)}, and \texttt{(***)} in terms of \texttt{arr} and
\texttt{first}.

\hypertarget{the-magic}{%
\subsection{The Magic}\label{the-magic}}

Now, for the magic --- consolidating a big composition of fragmented
\texttt{ParArrow}s into a streamlined simple-as-possible graph:

\begin{Shaded}
\begin{Highlighting}[]
\CommentTok{-- source: https://github.com/mstksg/inCode/tree/master/code-samples/pararrow/ParArrow.hs#L31-L51}

\OtherTok{collapse ::} \DataTypeTok{ParArrow}\NormalTok{ a b }\OtherTok{->} \DataTypeTok{ParArrow}\NormalTok{ a b}
\NormalTok{collapse (}\DataTypeTok{Seq}\NormalTok{ f g)       }\FunctionTok{=}
    \KeywordTok{case}\NormalTok{ (collapse f, collapse g) }\KeywordTok{of}
\NormalTok{      (}\DataTypeTok{Pure}\NormalTok{ p1, }\DataTypeTok{Pure}\NormalTok{ p2)      }\OtherTok{->} \DataTypeTok{Pure}\NormalTok{ (p1 }\FunctionTok{>>>}\NormalTok{ p2)}
\NormalTok{      (}\DataTypeTok{Seq}\NormalTok{ s1 s2, _)          }\OtherTok{->} \DataTypeTok{Seq}\NormalTok{ (collapse s1)}
\NormalTok{                                     (collapse (}\DataTypeTok{Seq}\NormalTok{ s2 g))}
\NormalTok{      (_, }\DataTypeTok{Seq}\NormalTok{ s1 s2)          }\OtherTok{->} \DataTypeTok{Seq}\NormalTok{ (collapse (}\DataTypeTok{Seq}\NormalTok{ f s1))}
\NormalTok{                                     (collapse s2)}
\NormalTok{      (}\DataTypeTok{Pure}\NormalTok{ p, }\DataTypeTok{Par}\NormalTok{ l p1 p2 r) }\OtherTok{->} \DataTypeTok{Par}\NormalTok{ (p }\FunctionTok{>>>}\NormalTok{ l)}
\NormalTok{                                     (collapse p1) (collapse p2)}
\NormalTok{                                     r}
\NormalTok{      (}\DataTypeTok{Par}\NormalTok{ l p1 p2 r, }\DataTypeTok{Pure}\NormalTok{ p) }\OtherTok{->} \DataTypeTok{Par}\NormalTok{ l}
\NormalTok{                                     (collapse p1) (collapse p2)}
\NormalTok{                                     (r }\FunctionTok{>>>}\NormalTok{ p)}
\NormalTok{      (}\DataTypeTok{Par}\NormalTok{ l p1 p2 r,}
       \DataTypeTok{Par}\NormalTok{ l' p1' p2' r')     }\OtherTok{->} \KeywordTok{let}\NormalTok{ p1f x }\FunctionTok{=}\NormalTok{ fst }\FunctionTok{.}\NormalTok{ l' }\FunctionTok{.}\NormalTok{ r }\FunctionTok{$}\NormalTok{ (x, undefined)}
\NormalTok{                                     p2f x }\FunctionTok{=}\NormalTok{ snd }\FunctionTok{.}\NormalTok{ l' }\FunctionTok{.}\NormalTok{ r }\FunctionTok{$}\NormalTok{ (undefined, x)}
\NormalTok{                                     pp1 }\FunctionTok{=}\NormalTok{ collapse (p1 }\FunctionTok{>>>}\NormalTok{ arr p1f }\FunctionTok{>>>}\NormalTok{ p1')}
\NormalTok{                                     pp2 }\FunctionTok{=}\NormalTok{ collapse (p2 }\FunctionTok{>>>}\NormalTok{ arr p2f }\FunctionTok{>>>}\NormalTok{ p2')}
                                 \KeywordTok{in}  \DataTypeTok{Par}\NormalTok{ l pp1 pp2 r'}
\NormalTok{collapse p }\FunctionTok{=}\NormalTok{ p}
\end{Highlighting}
\end{Shaded}

There are probably a couple of redundant calls to \texttt{collapse} in there,
but the picture should still be evident:

\begin{itemize}
\item
  Collapsing two sequenced \texttt{Pure}s should just be a single \texttt{Pure}
  with their pure functions composed.
\item
  Collapsing a \texttt{Seq} sequenced with anything else should re-associate the
  \texttt{Seq}s to the left, and collapse the \texttt{ParArrow}s inside as well.
\item
  Collapsing a \texttt{Pure} and a \texttt{Par} should just involve moving the
  function inside the \texttt{Pure} to the wrapping/unwrapping functions around
  the \texttt{Par}.
\item
  Collapsing two \texttt{Par}s is where the fun happens!

  We ``fuse'' the parallel branches of the fork together. We do that by running
  the export functions and the extract functions on each side, ``ignoring'' the
  other half of the tuple. This should work if the export/extract functions are
  all either \texttt{id} or \texttt{id\ \&\&\&\ id}.
\end{itemize}

And\ldots{}here we have a highly condensed parallelism graph.

\hypertarget{inspecting-pararrow-structures}{%
\subsection{\texorpdfstring{Inspecting \texttt{ParArrow}
structures}{Inspecting ParArrow structures}}\label{inspecting-pararrow-structures}}

It might be useful to get a peek at the internal structures of a collapsed
\texttt{ParArrow}. I used a helper data type, \texttt{Graph}.

\begin{Shaded}
\begin{Highlighting}[]
\CommentTok{-- source: https://github.com/mstksg/inCode/tree/master/code-samples/pararrow/ParArrow.hs#L76-L79}

\KeywordTok{data} \DataTypeTok{Graph} \FunctionTok{=} \DataTypeTok{GPure}                  \CommentTok{-- Pure function}
           \FunctionTok{|} \DataTypeTok{Graph} \FunctionTok{:->:} \DataTypeTok{Graph}       \CommentTok{-- Sequenced arrows}
           \FunctionTok{|} \DataTypeTok{Graph} \FunctionTok{:/:} \DataTypeTok{Graph}        \CommentTok{-- Parallel arrows}
           \KeywordTok{deriving} \DataTypeTok{Show}
\end{Highlighting}
\end{Shaded}

And we can convert a given \texttt{ParArrow} into its internal graph:

\begin{Shaded}
\begin{Highlighting}[]
\CommentTok{-- source: https://github.com/mstksg/inCode/tree/master/code-samples/pararrow/ParArrow.hs#L81-L87}

\OtherTok{analyze' ::} \DataTypeTok{ParArrow}\NormalTok{ a b }\OtherTok{->} \DataTypeTok{Graph}
\NormalTok{analyze' (}\DataTypeTok{Pure}\NormalTok{ _) }\FunctionTok{=} \DataTypeTok{GPure}
\NormalTok{analyze' (}\DataTypeTok{Seq}\NormalTok{ f g) }\FunctionTok{=}\NormalTok{ analyze' f }\FunctionTok{:->:}\NormalTok{ analyze' g}
\NormalTok{analyze' (}\DataTypeTok{Par}\NormalTok{ _ f g _) }\FunctionTok{=}\NormalTok{ analyze' f }\FunctionTok{:/:}\NormalTok{ analyze' g}

\OtherTok{analyze ::} \DataTypeTok{ParArrow}\NormalTok{ a b }\OtherTok{->} \DataTypeTok{Graph}
\NormalTok{analyze }\FunctionTok{=}\NormalTok{ analyze' }\FunctionTok{.}\NormalTok{ collapse}
\end{Highlighting}
\end{Shaded}

\hypertarget{sample-pararrows}{%
\subsection{Sample ParArrows}\label{sample-pararrows}}

Let's try examining it with some simple \texttt{Arrow}s, like the one we
mentioned before:

\begin{Shaded}
\begin{Highlighting}[]
\NormalTok{λ}\FunctionTok{:} \KeywordTok{let}\NormalTok{ test1 }\FunctionTok{=}
 \FunctionTok{|}\NormalTok{       proc x }\OtherTok{->} \KeywordTok{do}
 \FunctionTok{|}\NormalTok{       y }\OtherTok{<-}\NormalTok{ arr (}\FunctionTok{*}\DecValTok{2}\NormalTok{) }\FunctionTok{-<}\NormalTok{ x}
 \FunctionTok{|}\NormalTok{       z }\OtherTok{<-}\NormalTok{ arr (}\FunctionTok{+}\DecValTok{3}\NormalTok{) }\FunctionTok{-<}\NormalTok{ x}
 \FunctionTok{|}\NormalTok{       q }\OtherTok{<-}\NormalTok{ arr (}\FunctionTok{^}\DecValTok{2}\NormalTok{) }\FunctionTok{-<}\NormalTok{ y}
 \FunctionTok{|}\NormalTok{       returnA }\FunctionTok{-<}\NormalTok{ y }\FunctionTok{*}\NormalTok{ z }\FunctionTok{+}\NormalTok{ q}
\NormalTok{λ}\FunctionTok{:} \FunctionTok{:}\NormalTok{t test1}
\OtherTok{test1 ::}\NormalTok{ (}\DataTypeTok{Arrow}\NormalTok{ r, }\DataTypeTok{Num}\NormalTok{ t) }\OtherTok{=>}\NormalTok{ r t t}
\NormalTok{λ}\FunctionTok{:}\NormalTok{ test1 }\DecValTok{5}
\DecValTok{180}
\NormalTok{λ}\FunctionTok{:}\NormalTok{ analyze test1}
\DataTypeTok{GPure} \FunctionTok{:/:} \DataTypeTok{GPure}
\end{Highlighting}
\end{Shaded}

This is what we would expect. From looking at the diagram above, we can see that
there are two completely parallel forks; so in the collapsed arrow, there are
indeed only two parallel forks of pure functions.

How about a much simpler one that we unroll ourselves:

\begin{Shaded}
\begin{Highlighting}[]
\NormalTok{λ}\FunctionTok{:} \KeywordTok{let}\NormalTok{ test2 }\FunctionTok{=}\NormalTok{ arr (uncurry (}\FunctionTok{+}\NormalTok{))}
 \FunctionTok{|}           \FunctionTok{.}\NormalTok{ (arr (}\FunctionTok{*}\DecValTok{2}\NormalTok{) }\FunctionTok{***}\NormalTok{ arr (}\FunctionTok{+}\DecValTok{3}\NormalTok{))}
 \FunctionTok{|}           \FunctionTok{.}\NormalTok{ (id }\FunctionTok{&&&}\NormalTok{ id)}
\NormalTok{λ}\FunctionTok{:} \FunctionTok{:}\NormalTok{t test2}
\OtherTok{test2 ::}\NormalTok{ (}\DataTypeTok{Arrow}\NormalTok{ r, }\DataTypeTok{Num}\NormalTok{ t) }\OtherTok{=>}\NormalTok{ r t t}
\NormalTok{λ}\FunctionTok{:}\NormalTok{ test2 }\DecValTok{5}
\DecValTok{18}
\NormalTok{λ}\FunctionTok{:}\NormalTok{ analyze' test2}
\NormalTok{((}\DataTypeTok{GPure} \FunctionTok{:/:} \DataTypeTok{GPure}\NormalTok{) }\FunctionTok{:->:}\NormalTok{ (}\DataTypeTok{GPure} \FunctionTok{:/:} \DataTypeTok{GPure}\NormalTok{)) }\FunctionTok{:->:} \DataTypeTok{GPure}
\NormalTok{λ}\FunctionTok{:}\NormalTok{ analyze test2}
\DataTypeTok{GPure} \FunctionTok{:/:} \DataTypeTok{GPure}
\end{Highlighting}
\end{Shaded}

So as we can see, the ``uncollapsed'' \texttt{test2} is actually three sequenced
functions (as we would expect): Two parallel pure arrows (the
\texttt{id\ \&\&\&\ id} and \texttt{(arr\ (*2)\ ***\ arr\ (+3))}) and then one
sequential arrow (the \texttt{arr\ (uncurry\ (+))}).

However, we can see that that is just a single fork-and-recombine, so when we
collapse it, we get \texttt{GPure\ :/:\ GPure}, as we would expect.

\hypertarget{running-pararrows}{%
\section{Running ParArrows}\label{running-pararrows}}

Now we just need a way to run a \texttt{ParArrow}, and do the proper forking.
This actually isn't too bad at all, because of what we did in \texttt{collapse}.

\begin{Shaded}
\begin{Highlighting}[]
\CommentTok{-- source: https://github.com/mstksg/inCode/tree/master/code-samples/pararrow/ParArrow.hs#L92-L113}

\OtherTok{runPar' ::} \DataTypeTok{ParArrow}\NormalTok{ a b }\OtherTok{->}\NormalTok{ (a }\OtherTok{->} \DataTypeTok{IO}\NormalTok{ b)}
\NormalTok{runPar' }\FunctionTok{=}\NormalTok{ go}
  \KeywordTok{where}
\OtherTok{    go ::} \DataTypeTok{ParArrow}\NormalTok{ a b }\OtherTok{->}\NormalTok{ (a }\OtherTok{->} \DataTypeTok{IO}\NormalTok{ b)}
\NormalTok{    go (}\DataTypeTok{Pure}\NormalTok{ f)      }\FunctionTok{=}\NormalTok{ \textbackslash{}x }\OtherTok{->}\NormalTok{ putStrLn }\StringTok{"P"} \FunctionTok{>>}\NormalTok{ return (f x)}
\NormalTok{    go (}\DataTypeTok{Seq}\NormalTok{ f g)     }\FunctionTok{=}\NormalTok{ go f }\FunctionTok{>=>}\NormalTok{ go g}
\NormalTok{    go (}\DataTypeTok{Par}\NormalTok{ l f g r) }\FunctionTok{=}\NormalTok{ \textbackslash{}x }\OtherTok{->} \KeywordTok{do}
\NormalTok{      putStrLn }\StringTok{"F"}

\NormalTok{      fres }\OtherTok{<-}\NormalTok{ newEmptyMVar}
\NormalTok{      gres }\OtherTok{<-}\NormalTok{ newEmptyMVar}

      \KeywordTok{let}\NormalTok{ (fin,gin) }\FunctionTok{=}\NormalTok{ l x}
\NormalTok{      forkIO }\FunctionTok{$}\NormalTok{ runPar' f fin }\FunctionTok{>>=}\NormalTok{ putMVar fres}
\NormalTok{      forkIO }\FunctionTok{$}\NormalTok{ runPar' g gin }\FunctionTok{>>=}\NormalTok{ putMVar gres}

\NormalTok{      reses }\OtherTok{<-}\NormalTok{ (,) }\FunctionTok{<$>}\NormalTok{ takeMVar fres }\FunctionTok{<*>}\NormalTok{ takeMVar gres}
\NormalTok{      return (r reses)}

\OtherTok{runPar ::} \DataTypeTok{ParArrow}\NormalTok{ a b }\OtherTok{->}\NormalTok{ (a }\OtherTok{->} \DataTypeTok{IO}\NormalTok{ b)}
\NormalTok{runPar }\FunctionTok{=}\NormalTok{ runPar' }\FunctionTok{.}\NormalTok{ collapse}
\end{Highlighting}
\end{Shaded}

(Note that I left in debug traces)

\hypertarget{testing}{%
\subsection{Testing}\label{testing}}

Sweet, now let's run it!

\begin{Shaded}
\begin{Highlighting}[]
\NormalTok{λ}\FunctionTok{:}\NormalTok{ test2 }\DecValTok{5}
\DecValTok{18}
\NormalTok{λ}\FunctionTok{:}\NormalTok{ runPar test2 }\DecValTok{5}
\DataTypeTok{F}
\DataTypeTok{P}
\DataTypeTok{P}
\DecValTok{18}
\end{Highlighting}
\end{Shaded}

That works as expected!

We can see from the debug trace that first things are forked, and then two pure
functions are run. A final value of 18 is returned, which is the same as for the
\texttt{(-\textgreater{})} version. (Note how we can use \texttt{test2} as both,
due to what we mentioned above)

Okay, so it looks like this does exactly what we want. It intelligently
``knows'' when to fork, when to unfork, when to ``sequence'' forks. Let's try it
with \texttt{test1}, which was written in \texttt{proc} notation.

\begin{Shaded}
\begin{Highlighting}[]
\NormalTok{λ}\FunctionTok{:}\NormalTok{ test1 }\DecValTok{5}
\DecValTok{180}
\NormalTok{λ}\FunctionTok{:}\NormalTok{ runPar test1 }\DecValTok{5}
\DataTypeTok{F}
\DataTypeTok{P}
\DataTypeTok{P}
\FunctionTok{***} \DataTypeTok{Exception}\FunctionTok{:}\NormalTok{ Prelude.undefined}
\end{Highlighting}
\end{Shaded}

What! :/

\hypertarget{what-went-wrong}{%
\section{What went wrong}\label{what-went-wrong}}

Let's dig into actual desguaring. According to the proc notation specs:

\begin{Shaded}
\begin{Highlighting}[]
\NormalTok{test3 }\FunctionTok{=}\NormalTok{ proc x }\OtherTok{->} \KeywordTok{do}
\NormalTok{    y }\OtherTok{<-}\NormalTok{ arr (}\FunctionTok{*}\DecValTok{2}\NormalTok{) }\FunctionTok{-<}\NormalTok{ x}
\NormalTok{    z }\OtherTok{<-}\NormalTok{ arr (}\FunctionTok{+}\DecValTok{3}\NormalTok{) }\FunctionTok{-<}\NormalTok{ x}
\NormalTok{    returnA }\FunctionTok{-<}\NormalTok{ y }\FunctionTok{+}\NormalTok{ z}

\CommentTok{-- desugared:}
\NormalTok{test3' }\FunctionTok{=}\NormalTok{ arr (\textbackslash{}(x,y) }\OtherTok{->}\NormalTok{ x }\FunctionTok{+}\NormalTok{ y)     }\CommentTok{-- add}
       \FunctionTok{.}\NormalTok{ arr (\textbackslash{}(x,y) }\OtherTok{->}\NormalTok{ (y,x))     }\CommentTok{-- flip}
       \FunctionTok{.}\NormalTok{ first (arr (}\FunctionTok{+}\DecValTok{3}\NormalTok{))          }\CommentTok{-- z}
       \FunctionTok{.}\NormalTok{ arr (\textbackslash{}(x,y) }\OtherTok{->}\NormalTok{ (y,x))     }\CommentTok{-- flip}
       \FunctionTok{.}\NormalTok{ first (arr (}\FunctionTok{*}\DecValTok{2}\NormalTok{))          }\CommentTok{-- y}
       \FunctionTok{.}\NormalTok{ arr (\textbackslash{}x }\OtherTok{->}\NormalTok{ (x,x))         }\CommentTok{-- split}
\end{Highlighting}
\end{Shaded}

Ah. Everything is in terms of \texttt{arr} and \texttt{first}, and it never uses
\texttt{second}, \texttt{(***)}, or \texttt{(\&\&\&)}. (These should be
equivalent, due to the Arrow laws, of course; my instance is obviously unlawful,
oops)

I'm going to cut right to the chase here. The main problem is our collapsing
sequenced \texttt{Pure} and \texttt{Par}s.

Basically, the collapsing rules say that if we have:

\begin{Shaded}
\begin{Highlighting}[]
\DataTypeTok{Par}\NormalTok{ l p1 p2 r }\OtherTok{`Seq`} \DataTypeTok{Pure}\NormalTok{ f }\OtherTok{`Seq`} \DataTypeTok{Par}\NormalTok{ l' p1' p2' r'}
\end{Highlighting}
\end{Shaded}

It should be the same as one giant \texttt{Par}, where \texttt{f} is
``injected'' between \texttt{p1} and \texttt{p1\textquotesingle{}}, \texttt{p2}
and \texttt{p2\textquotesingle{}}.

The bridge is basically a tuple, and we take advantage of laziness to basically
pop the results of \texttt{p1} into a tuple using \texttt{r}, apply \texttt{f}
to the tuple, and extract it using \texttt{l}, and run it through
\texttt{p1\textquotesingle{}}.

So \texttt{f} has to be some sort of function
\texttt{(a,b)\ -\textgreater{}\ (c,d)}, where \texttt{c}'s value can only depend
on \texttt{a}'s value, and \texttt{d}'s value can only depend on \texttt{b}'s
value. Basically, it has to be derived from functions
\texttt{a\ -\textgreater{}\ c} and \texttt{b\ -\textgreater{}\ d}. A
``parallel'' function.

As long as this is true, this will work.

However, we see in the desugaring of \texttt{test3} that \texttt{f} is not
always that. \texttt{f} can be \emph{any} function, actually, and we can't
really control what happens to it. In \texttt{test3}, we actaully use
\texttt{f\ =\ \textbackslash{}(x,y)\ -\textgreater{}\ (y,x)}\ldots{}definitely
not a ``parallel'' function!

Actually, this doesn't even make any sense in terms of our parallel computation
model! How can we ``combine'' two parallel forks\ldots{}when halfway in between
the two forks, they must exchange information? Then it's no longer fully
parallel!

We can ``fix'' this. We can make \texttt{collapse} not collapse the
\texttt{Pure}-\texttt{Par} cases:

\begin{Shaded}
\begin{Highlighting}[]
\CommentTok{-- source: https://github.com/mstksg/inCode/tree/master/code-samples/pararrow/ParArrow.hs#L53-L116}

\OtherTok{collapse_ ::} \DataTypeTok{ParArrow}\NormalTok{ a b }\OtherTok{->} \DataTypeTok{ParArrow}\NormalTok{ a b}
\NormalTok{collapse_ (}\DataTypeTok{Seq}\NormalTok{ f g)       }\FunctionTok{=}
    \KeywordTok{case}\NormalTok{ (collapse_ f, collapse_ g) }\KeywordTok{of}
\NormalTok{      (}\DataTypeTok{Pure}\NormalTok{ p1, }\DataTypeTok{Pure}\NormalTok{ p2)      }\OtherTok{->} \DataTypeTok{Pure}\NormalTok{ (p1 }\FunctionTok{>>>}\NormalTok{ p2)}
\NormalTok{      (}\DataTypeTok{Seq}\NormalTok{ s1 s2, _)          }\OtherTok{->} \DataTypeTok{Seq}\NormalTok{ (collapse_ s1)}
\NormalTok{                                     (collapse_ (}\DataTypeTok{Seq}\NormalTok{ s2 g))}
\NormalTok{      (_, }\DataTypeTok{Seq}\NormalTok{ s1 s2)          }\OtherTok{->} \DataTypeTok{Seq}\NormalTok{ (collapse_ (}\DataTypeTok{Seq}\NormalTok{ f s1))}
\NormalTok{                                     (collapse_ s2)}
      \CommentTok{-- (Pure p, Par l p1 p2 r) -> Par (p >>> l)}
      \CommentTok{--                                (collapse_ p1) (collapse_ p2)}
      \CommentTok{--                                r}
      \CommentTok{-- (Par l p1 p2 r, Pure p) -> Par l}
      \CommentTok{--                                (collapse_ p1) (collapse_ p2)}
      \CommentTok{--                                (r >>> p)}
\NormalTok{      (}\DataTypeTok{Par}\NormalTok{ l p1 p2 r,}
       \DataTypeTok{Par}\NormalTok{ l' p1' p2' r')     }\OtherTok{->} \KeywordTok{let}\NormalTok{ p1f x }\FunctionTok{=}\NormalTok{ fst }\FunctionTok{.}\NormalTok{ l' }\FunctionTok{.}\NormalTok{ r }\FunctionTok{$}\NormalTok{ (x, undefined)}
\NormalTok{                                     p2f x }\FunctionTok{=}\NormalTok{ snd }\FunctionTok{.}\NormalTok{ l' }\FunctionTok{.}\NormalTok{ r }\FunctionTok{$}\NormalTok{ (undefined, x)}
\NormalTok{                                     pp1 }\FunctionTok{=}\NormalTok{ collapse_ (p1 }\FunctionTok{>>>}\NormalTok{ arr p1f }\FunctionTok{>>>}\NormalTok{ p1')}
\NormalTok{                                     pp2 }\FunctionTok{=}\NormalTok{ collapse_ (p2 }\FunctionTok{>>>}\NormalTok{ arr p2f }\FunctionTok{>>>}\NormalTok{ p2')}
                                 \KeywordTok{in}  \DataTypeTok{Par}\NormalTok{ l pp1 pp2 r'}
\NormalTok{      (f,g)                   }\OtherTok{->} \DataTypeTok{Seq}\NormalTok{ f g}
\NormalTok{collapse_ p }\FunctionTok{=}\NormalTok{ p}

\OtherTok{analyze_ ::} \DataTypeTok{ParArrow}\NormalTok{ a b }\OtherTok{->} \DataTypeTok{Graph}
\NormalTok{analyze_ }\FunctionTok{=}\NormalTok{ analyze' }\FunctionTok{.}\NormalTok{ collapse_}

\OtherTok{runPar_ ::} \DataTypeTok{ParArrow}\NormalTok{ a b }\OtherTok{->}\NormalTok{ (a }\OtherTok{->} \DataTypeTok{IO}\NormalTok{ b)}
\NormalTok{runPar_ }\FunctionTok{=}\NormalTok{ runPar' }\FunctionTok{.}\NormalTok{ collapse_}
\end{Highlighting}
\end{Shaded}

Then we have:

\begin{Shaded}
\begin{Highlighting}[]
\NormalTok{λ}\FunctionTok{:}\NormalTok{ analyze_ test1}
\NormalTok{(}
  \DataTypeTok{GPure} \FunctionTok{:->:}\NormalTok{ ( ( }\DataTypeTok{GPure} \FunctionTok{:->:} \DataTypeTok{GPure}\NormalTok{ ) }\FunctionTok{:/:} \DataTypeTok{GPure}\NormalTok{ )}
\NormalTok{) }\FunctionTok{:->:}\NormalTok{ ((}
  \DataTypeTok{GPure} \FunctionTok{:->:}\NormalTok{ ( ( }\DataTypeTok{GPure} \FunctionTok{:->:} \DataTypeTok{GPure}\NormalTok{ ) }\FunctionTok{:/:} \DataTypeTok{GPure}\NormalTok{ )}
\NormalTok{) }\FunctionTok{:->:}\NormalTok{ ((}
  \DataTypeTok{GPure} \FunctionTok{:->:}\NormalTok{ ( ( }\DataTypeTok{GPure} \FunctionTok{:->:} \DataTypeTok{GPure}\NormalTok{ ) }\FunctionTok{:/:} \DataTypeTok{GPure}\NormalTok{ )}
\NormalTok{) }\FunctionTok{:->:}
  \DataTypeTok{GPure}
\NormalTok{))}
\end{Highlighting}
\end{Shaded}

We basically have three
\texttt{GPure\ :-\textgreater{}\ ((GPure\ :-\textgreater{}:\ GPure)\ :/:\ GPure)}'s
in a row. A pure function followed by parallel functions. This sort of makes
sense, and if we sort of imagined manually unrolling \texttt{test3}, this is
what we'd imagine we'd get, sorta. now we don't ``collapse'' the three parallel
forks together.

This runs without error:

\begin{Shaded}
\begin{Highlighting}[]
\NormalTok{λ}\FunctionTok{:}\NormalTok{ runPar_ test1 }\DecValTok{5}
\DataTypeTok{P}
\DataTypeTok{F}
\DataTypeTok{P}
\DataTypeTok{P}
\DataTypeTok{P}
\DataTypeTok{P}
\DataTypeTok{F}
\DataTypeTok{P}
\DataTypeTok{P}
\DataTypeTok{P}
\DataTypeTok{P}
\DecValTok{18}
\end{Highlighting}
\end{Shaded}

And the trace shows that it is ``forking'' two times. The structural analysis
would actaully suggest that we forked three times, but\ldots{}I'm not totally
sure what's going on here heh. Still two is much more than what should ideally
be required (one).

\hypertarget{oh.}{%
\section{Oh.}\label{oh.}}

So now we can no longer fully ``collapse'' the two parallel forks, and it
involves forking twice. Which makes complete sense, because we have to swap in
the middle.

And without the collapsing\ldots{}there are a lot of unecessary
reforks/recominbations that would basically kill any useful parallelization
unless you pre-compose all of your forks\ldots{}which kind of defeats the
purpose of the implicit dataflow parallization in the first place.

Anyways, this is all rather annoying, because the analogous manual
\texttt{(\&\&\&)} / \texttt{(***)} / \texttt{second}-based \texttt{test1} should
not ever fail, because we never fork. So if the proc block had desugared to
using those combinators and never using
\texttt{arr\ (\textbackslash{}(x,y)\ -\textgreater{}\ (y,x))}, everything would
work out fine!

But hey, if you write out the arrow computation manually by composing
\texttt{(\&\&\&)}, \texttt{(***)}, and \texttt{second}\ldots{}this will
\emph{all actually work}! I mean serious! Isn't that crazy! (Provided, all of
your \texttt{Pure}'s are sufficiently ``parallel'').

But the whole point in the first place was to use proc/do notation, so this
becomes a lot less useful than before.

Also, it's inherently pretty fragile, as you can no longer rely on the type
system to enforce ``sufficiently parallel'' \texttt{Pure}`s. You can't even
check against something like
\texttt{arr\ (\textbackslash{}(x,y)\ -\textgreater{}\ (x,x))}, which makes no
sense again in 'isolated parallel' computations.

(Interestingly enough, you \emph{can} use the type system to enforce against
things like \texttt{arr\ (\textbackslash{}(x,y)\ -\textgreater{}\ x)} or
\texttt{arr\ (\textbackslash{}(x,y)\ -\textgreater{}\ 5)}; you can't collapse
tuples)

Basically, \emph{it mostly works} for almost all
\texttt{ParArrow\ (a,b)\ (c,d)}\ldots{}\emph{except} for when they have
cross-talk.

So, well\ldots{}back to the drawing board I guess.

\hypertarget{what-can-be-done}{%
\section{What can be done?}\label{what-can-be-done}}

So I'm open to seeing different avenues that this can be approached by, and also
if anyone else has tried doing this and had more success than me.

In particular, I do not have much experience with type-/kind-level stuff
involving those fun extensions, so if there is something that can be done there,
I would be happy to learn :)

\hypertarget{other-avenues}{%
\subsection{Other avenues}\label{other-avenues}}

I have tried other things ``in addition to'' the things mentioned in this post,
but most of them have also been dead ends. Among one of the attempts that I
tried involve throwing exceptions from one thread to another containing the
``missing half''. If an
\texttt{arr\ (\textbackslash{}(x,y)\ -\textgreater{}\ (y,x))}-like function is
used, then each thread will know, and ``wait'' on the other to throw an
exception to the other containing the missing data.

I couldn't get this to work, exactly, because I couldn't get it to work without
adding a \texttt{Typeable} constraint to the parameters\ldots{}and even when
using things like the
\href{http://hsenag.livejournal.com/11803.html}{constrained monads technique}, I
couldn't get the ``unwrap'' functions to work because I couldn't show that
\texttt{z}, \texttt{a1}, \texttt{b1}, etc. were Typeable.

Perhaps without the exception method, I could use \texttt{MVar}s to sort of have
a branch ``wait'' on the other if they find out that they have been given an
\texttt{arr} that has cross-talk.

Another path is just giving up \texttt{Arrow} completely and using non-typeclass
\ldots{} but I don't think that offers much advantages over the current system
(using \texttt{(***)} etc.), and also it gives up the entire point --- using
proc notation, and also the neat ability to use them as if they were regular
functions.

For now, though, I am calling this a ``dead end''\footnote{Actually, this is
  technically not true; while I was writing this article another idea came to me
  by using some sort of state machine/automation arrow to wait on the results
  and pass them on, but that's still in the first stages of being thought
  through :)}; if anyone has any suggestions, I'd be happy to hear them :) I
just thought it'd be worth putting up my thought process up in written form
somewhere so that I could look back on them, or so that people can see what
doesn't work and/or possibly learn :) And of course for entertainment in case I
am hilariously awful.

\hypertarget{signoff}{%
\section{Signoff}\label{signoff}}

Hi, thanks for reading! You can reach me via email at
\href{mailto:justin@jle.im}{\nolinkurl{justin@jle.im}}, or at twitter at
\href{https://twitter.com/mstk}{@mstk}! This post and all others are published
under the \href{https://creativecommons.org/licenses/by-nc-nd/3.0/}{CC-BY-NC-ND
3.0} license. Corrections and edits via pull request are welcome and encouraged
at \href{https://github.com/mstksg/inCode}{the source repository}.

If you feel inclined, or this post was particularly helpful for you, why not
consider \href{https://www.patreon.com/justinle/overview}{supporting me on
Patreon}, or a \href{bitcoin:3D7rmAYgbDnp4gp4rf22THsGt74fNucPDU}{BTC donation}?
:)

\end{document}
