\documentclass[]{article}
\usepackage{lmodern}
\usepackage{amssymb,amsmath}
\usepackage{ifxetex,ifluatex}
\usepackage{fixltx2e} % provides \textsubscript
\ifnum 0\ifxetex 1\fi\ifluatex 1\fi=0 % if pdftex
  \usepackage[T1]{fontenc}
  \usepackage[utf8]{inputenc}
\else % if luatex or xelatex
  \ifxetex
    \usepackage{mathspec}
    \usepackage{xltxtra,xunicode}
  \else
    \usepackage{fontspec}
  \fi
  \defaultfontfeatures{Mapping=tex-text,Scale=MatchLowercase}
  \newcommand{\euro}{€}
\fi
% use upquote if available, for straight quotes in verbatim environments
\IfFileExists{upquote.sty}{\usepackage{upquote}}{}
% use microtype if available
\IfFileExists{microtype.sty}{\usepackage{microtype}}{}
\usepackage[margin=1in]{geometry}
\usepackage{color}
\usepackage{fancyvrb}
\newcommand{\VerbBar}{|}
\newcommand{\VERB}{\Verb[commandchars=\\\{\}]}
\DefineVerbatimEnvironment{Highlighting}{Verbatim}{commandchars=\\\{\}}
% Add ',fontsize=\small' for more characters per line
\newenvironment{Shaded}{}{}
\newcommand{\AlertTok}[1]{\textcolor[rgb]{1.00,0.00,0.00}{\textbf{#1}}}
\newcommand{\AnnotationTok}[1]{\textcolor[rgb]{0.38,0.63,0.69}{\textbf{\textit{#1}}}}
\newcommand{\AttributeTok}[1]{\textcolor[rgb]{0.49,0.56,0.16}{#1}}
\newcommand{\BaseNTok}[1]{\textcolor[rgb]{0.25,0.63,0.44}{#1}}
\newcommand{\BuiltInTok}[1]{#1}
\newcommand{\CharTok}[1]{\textcolor[rgb]{0.25,0.44,0.63}{#1}}
\newcommand{\CommentTok}[1]{\textcolor[rgb]{0.38,0.63,0.69}{\textit{#1}}}
\newcommand{\CommentVarTok}[1]{\textcolor[rgb]{0.38,0.63,0.69}{\textbf{\textit{#1}}}}
\newcommand{\ConstantTok}[1]{\textcolor[rgb]{0.53,0.00,0.00}{#1}}
\newcommand{\ControlFlowTok}[1]{\textcolor[rgb]{0.00,0.44,0.13}{\textbf{#1}}}
\newcommand{\DataTypeTok}[1]{\textcolor[rgb]{0.56,0.13,0.00}{#1}}
\newcommand{\DecValTok}[1]{\textcolor[rgb]{0.25,0.63,0.44}{#1}}
\newcommand{\DocumentationTok}[1]{\textcolor[rgb]{0.73,0.13,0.13}{\textit{#1}}}
\newcommand{\ErrorTok}[1]{\textcolor[rgb]{1.00,0.00,0.00}{\textbf{#1}}}
\newcommand{\ExtensionTok}[1]{#1}
\newcommand{\FloatTok}[1]{\textcolor[rgb]{0.25,0.63,0.44}{#1}}
\newcommand{\FunctionTok}[1]{\textcolor[rgb]{0.02,0.16,0.49}{#1}}
\newcommand{\ImportTok}[1]{#1}
\newcommand{\InformationTok}[1]{\textcolor[rgb]{0.38,0.63,0.69}{\textbf{\textit{#1}}}}
\newcommand{\KeywordTok}[1]{\textcolor[rgb]{0.00,0.44,0.13}{\textbf{#1}}}
\newcommand{\NormalTok}[1]{#1}
\newcommand{\OperatorTok}[1]{\textcolor[rgb]{0.40,0.40,0.40}{#1}}
\newcommand{\OtherTok}[1]{\textcolor[rgb]{0.00,0.44,0.13}{#1}}
\newcommand{\PreprocessorTok}[1]{\textcolor[rgb]{0.74,0.48,0.00}{#1}}
\newcommand{\RegionMarkerTok}[1]{#1}
\newcommand{\SpecialCharTok}[1]{\textcolor[rgb]{0.25,0.44,0.63}{#1}}
\newcommand{\SpecialStringTok}[1]{\textcolor[rgb]{0.73,0.40,0.53}{#1}}
\newcommand{\StringTok}[1]{\textcolor[rgb]{0.25,0.44,0.63}{#1}}
\newcommand{\VariableTok}[1]{\textcolor[rgb]{0.10,0.09,0.49}{#1}}
\newcommand{\VerbatimStringTok}[1]{\textcolor[rgb]{0.25,0.44,0.63}{#1}}
\newcommand{\WarningTok}[1]{\textcolor[rgb]{0.38,0.63,0.69}{\textbf{\textit{#1}}}}
\ifxetex
  \usepackage[setpagesize=false, % page size defined by xetex
              unicode=false, % unicode breaks when used with xetex
              xetex]{hyperref}
\else
  \usepackage[unicode=true]{hyperref}
\fi
\hypersetup{breaklinks=true,
            bookmarks=true,
            pdfauthor={Justin Le},
            pdftitle={Streaming Huffman Compression in Haskell (Part 1: Trees and State)},
            colorlinks=true,
            citecolor=blue,
            urlcolor=blue,
            linkcolor=magenta,
            pdfborder={0 0 0}}
\urlstyle{same}  % don't use monospace font for urls
% Make links footnotes instead of hotlinks:
\renewcommand{\href}[2]{#2\footnote{\url{#1}}}
\setlength{\parindent}{0pt}
\setlength{\parskip}{6pt plus 2pt minus 1pt}
\setlength{\emergencystretch}{3em}  % prevent overfull lines
\setcounter{secnumdepth}{0}

\title{Streaming Huffman Compression in Haskell (Part 1: Trees and State)}
\author{Justin Le}
\date{March 31, 2014}

\begin{document}
\maketitle

\emph{Originally posted on
\textbf{\href{https://blog.jle.im/entry/streaming-huffman-compression-in-haskell-part-1-trees.html}{in
Code}}.}

So you're learning Haskell and are looking for some projects that aren't super
trivial, are familiar enough that you can use what you already know, and are
difficult enough to maybe help you learn new things. Hey, maybe this is for you
:)

Let's take a go at \href{http://en.wikipedia.org/wiki/Huffman_coding}{Huffman
encoding} in Haskell. We will look at two types of binary trees, which we use to
implement immutable/persistent priority queues and prefix trees. We'll play
around with the State monad a bit, explore some useful typeclasses, learn how to
serialize, marshal, and unmarshal data structures using the
\href{http://hackage.haskell.org/package/binary}{binary} library, and also look
at how to load data from a file and write to another in a pure way, avoiding
lazy IO using the ever-more-popular
\emph{\href{http://hackage.haskell.org/package/pipes}{pipes}} library. And
hopefully we learn some neat Haskell idioms!

We're going to be assuming some basic Haskell knowledge, like algebraic data
types, higher order functions, basic monad usage, and some basic familiarity
with the functions in Prelude/base, the standard library. If you have any
questions, feel free to leave a comment, drop by on \#haskell on freenode, throw
me a \href{https://twitter.com/mstk}{tweet}, or give the great
\href{http://learnyouahaskell.com}{Learn You A Haskell} a quick read!

\hypertarget{prefix-trees-huffman-coding}{%
\section{Prefix trees \& Huffman coding}\label{prefix-trees-huffman-coding}}

You might have encountered this in the past, but Huffman encoding solves the
problem of finding the optimal binary prefix code to encode a string.

I'll leave you to read \href{http://en.wikipedia.org/wiki/Huffman_coding}{the
wikipedia article}, which explains it much better than I could. Basically,
binary prefix codes are nice because you don't have to encode any ``stop''
symbol --- as soon as you reach a leaf of the tree, you know that you have found
a letter, and can move on.

Huffman trees are built from the bottom-up using priority queues. The two
lowest-frequency nodes are continually ``popped'' from the queue, combined into
a new node, and placed back in the queue.

Our first challenge --- representing a Huffman tree as a data structure!

\hypertarget{the-tree}{%
\subsection{The Tree}\label{the-tree}}

(All the code in this section on is
\href{https://github.com/mstksg/inCode/tree/master/code-samples/huffman/PreTree.hs}{available
for download} for you to try it out yourself!)

So some properties about prefix trees that might be useful to us --- all data is
stored in the leaves, and all internal nodes have exactly two children. This
sounds like the perfect candidate for an Algebraic Data Structure.

\begin{Shaded}
\begin{Highlighting}[]
\CommentTok{-- source: https://github.com/mstksg/inCode/tree/master/code-samples/huffman/PreTree.hs#L19-L21}
\CommentTok{-- interactive: https://www.fpcomplete.com/user/jle/huffman-encoding}

\KeywordTok{data} \DataTypeTok{PreTree}\NormalTok{ a }\FunctionTok{=} \DataTypeTok{PTLeaf}\NormalTok{ a}
               \FunctionTok{|} \DataTypeTok{PTNode}\NormalTok{ (}\DataTypeTok{PreTree}\NormalTok{ a) (}\DataTypeTok{PreTree}\NormalTok{ a)}
               \KeywordTok{deriving}\NormalTok{ (}\DataTypeTok{Show}\NormalTok{, }\DataTypeTok{Eq}\NormalTok{, }\DataTypeTok{Generic}\NormalTok{)}
\end{Highlighting}
\end{Shaded}

We leave the type parameterized on \texttt{a} (which is like a template/generic
in C++/Java) so we can decide what to put into it later.

\hypertarget{pretree-operations}{%
\subsubsection{\texorpdfstring{\texttt{PreTree}
operations}{PreTree operations}}\label{pretree-operations}}

So, what sort of things are we going to want to do with our \texttt{PreTree}?

Well..first of all, we might want a way to put something into an empty tree ---
create a leaf containing just that data.

That function is sort of embarrassingly easy:

\begin{Shaded}
\begin{Highlighting}[]
\OtherTok{makePT' ::}\NormalTok{ a }\OtherTok{->} \DataTypeTok{PreTree}\NormalTok{ a}
\NormalTok{makePT' x }\FunctionTok{=} \DataTypeTok{PTLeaf}\NormalTok{ x}
\end{Highlighting}
\end{Shaded}

Remember, that's \texttt{PTLeaf} is a data constructor that ``creates'' a
\texttt{PreTree} when you use \texttt{PTLeaf\ x}.

However, something like this is just begging to be eta-reduced, and we can
simplify it as:

\begin{Shaded}
\begin{Highlighting}[]
\CommentTok{-- source: https://github.com/mstksg/inCode/tree/master/code-samples/huffman/PreTree.hs#L46-L47}
\CommentTok{-- interactive: https://www.fpcomplete.com/user/jle/huffman-encoding}

\OtherTok{makePT ::}\NormalTok{ a }\OtherTok{->} \DataTypeTok{PreTree}\NormalTok{ a}
\NormalTok{makePT }\FunctionTok{=} \DataTypeTok{PTLeaf}
\end{Highlighting}
\end{Shaded}

Which does the same thing. Basically, \texttt{PTLeaf} is already a function
\texttt{a\ -\textgreater{}\ PreTree\ a}\ldots{}so \texttt{makePT} is literally
just \texttt{PTLeaf}.

\begin{Shaded}
\begin{Highlighting}[]
\NormalTok{ghci}\FunctionTok{>} \KeywordTok{let}\NormalTok{ pt }\FunctionTok{=}\NormalTok{ makePT }\CharTok{'c'}
\NormalTok{ghci}\FunctionTok{>} \FunctionTok{:}\NormalTok{t pt}
\DataTypeTok{PreTree} \DataTypeTok{Char}
\NormalTok{ghci}\FunctionTok{>}\NormalTok{ pt}
\DataTypeTok{PTLeaf} \CharTok{'c'}
\end{Highlighting}
\end{Shaded}

Now, we might also want a way to ``merge'' two \texttt{PreTree\ a}'s. This is at
the heart of building the tree in the first place\ldots{}successively merge two
trees until everything is in one giant tree.

This isn't too bad either:

\begin{Shaded}
\begin{Highlighting}[]
\OtherTok{mergePT' ::} \DataTypeTok{PreTree}\NormalTok{ a }\OtherTok{->} \DataTypeTok{PreTree}\NormalTok{ a }\OtherTok{->} \DataTypeTok{PreTree}\NormalTok{ a}
\NormalTok{mergePT' t1 t2 }\FunctionTok{=} \DataTypeTok{PTNode}\NormalTok{ t1 t2}
\end{Highlighting}
\end{Shaded}

Which, from what we saw before, can just be written as:

\begin{Shaded}
\begin{Highlighting}[]
\CommentTok{-- source: https://github.com/mstksg/inCode/tree/master/code-samples/huffman/PreTree.hs#L50-L51}
\CommentTok{-- interactive: https://www.fpcomplete.com/user/jle/huffman-encoding}

\OtherTok{mergePT ::} \DataTypeTok{PreTree}\NormalTok{ a }\OtherTok{->} \DataTypeTok{PreTree}\NormalTok{ a }\OtherTok{->} \DataTypeTok{PreTree}\NormalTok{ a}
\NormalTok{mergePT }\FunctionTok{=} \DataTypeTok{PTNode}
\end{Highlighting}
\end{Shaded}

\begin{Shaded}
\begin{Highlighting}[]
\NormalTok{ghci}\FunctionTok{>} \KeywordTok{let}\NormalTok{ pt1 }\FunctionTok{=}\NormalTok{ makePT }\CharTok{'c'}
\NormalTok{ghci}\FunctionTok{>} \KeywordTok{let}\NormalTok{ pt2 }\FunctionTok{=}\NormalTok{ makePT }\CharTok{'t'}
\NormalTok{ghci}\FunctionTok{>} \KeywordTok{let}\NormalTok{ pt3 }\FunctionTok{=}\NormalTok{ pt1 }\OtherTok{`mergePT`}\NormalTok{ pt2}
\NormalTok{ghci}\FunctionTok{>} \FunctionTok{:}\NormalTok{t pt3}
\DataTypeTok{PreTree} \DataTypeTok{Char}
\NormalTok{ghci}\FunctionTok{>}\NormalTok{ pt3}
\DataTypeTok{PTNode}\NormalTok{ (}\DataTypeTok{PTLeaf} \CharTok{'c'}\NormalTok{) (}\DataTypeTok{PTLeaf} \CharTok{'t'}\NormalTok{)}
\end{Highlighting}
\end{Shaded}

Hm. Maybe that's a bit too easy. Feels a little unsettling, isn't it?

Welcome to Haskell!

\hypertarget{weighting-things}{%
\subsection{Weighting things}\label{weighting-things}}

We're going to need some way of comparing the weights/priorities of two
\texttt{PreTree}s when we are assembling the tree. Let's introduce a data type
that includes both a \texttt{PreTree} and an (integer) weight.

\begin{Shaded}
\begin{Highlighting}[]
\CommentTok{-- source: https://github.com/mstksg/inCode/tree/master/code-samples/huffman/Weighted.hs#L13-L15}
\CommentTok{-- interactive: https://www.fpcomplete.com/user/jle/huffman-encoding}

\KeywordTok{data} \DataTypeTok{Weighted}\NormalTok{ a }\FunctionTok{=} \DataTypeTok{WPair}\NormalTok{ \{}\OtherTok{ _wWeight ::} \DataTypeTok{Int}
\NormalTok{                        ,}\OtherTok{ _wItem   ::}\NormalTok{ a}
\NormalTok{                        \} }\KeywordTok{deriving}\NormalTok{ (}\DataTypeTok{Show}\NormalTok{, }\DataTypeTok{Functor}\NormalTok{)}
\end{Highlighting}
\end{Shaded}

(Code for the Weighted module is
\href{https://github.com/mstksg/inCode/tree/master/code-samples/huffman/Weighted.hs}{available
for download})

We will say that a \texttt{Weighted\ a} is some \texttt{a} associated with an
integer weight.

We can create, say, a \texttt{PreTree} containing the character `a', weighted
with integer 1:

\begin{Shaded}
\begin{Highlighting}[]
\NormalTok{ghci}\FunctionTok{>} \DataTypeTok{WPair} \DecValTok{1}\NormalTok{ (makePTLeaf }\CharTok{'a'}\NormalTok{)}
\DataTypeTok{WPair} \DecValTok{1}\NormalTok{ (makePTLeaf }\CharTok{'a'}\NormalTok{)}\OtherTok{ ::} \DataTypeTok{Weighted}\NormalTok{ (}\DataTypeTok{PreTree} \DataTypeTok{Char}\NormalTok{)}
\end{Highlighting}
\end{Shaded}

This weighted \texttt{PreTree} is pretty useful, let's give it an alias/typedef:

\begin{Shaded}
\begin{Highlighting}[]
\CommentTok{-- source: https://github.com/mstksg/inCode/tree/master/code-samples/huffman/PreTree.hs#L54-L54}
\CommentTok{-- interactive: https://www.fpcomplete.com/user/jle/huffman-encoding}

\KeywordTok{type} \DataTypeTok{WeightedPT}\NormalTok{ a }\FunctionTok{=} \DataTypeTok{Weighted}\NormalTok{ (}\DataTypeTok{PreTree}\NormalTok{ a)}
\end{Highlighting}
\end{Shaded}

Let's make the same functions for \texttt{WeightedPT} as we did for
\texttt{PreTree}:

\begin{Shaded}
\begin{Highlighting}[]
\CommentTok{-- source: https://github.com/mstksg/inCode/tree/master/code-samples/huffman/PreTree.hs#L58-L59}
\CommentTok{-- interactive: https://www.fpcomplete.com/user/jle/huffman-encoding}

\OtherTok{makeWPT ::} \DataTypeTok{Int} \OtherTok{->}\NormalTok{ a }\OtherTok{->} \DataTypeTok{WeightedPT}\NormalTok{ a}
\NormalTok{makeWPT w }\FunctionTok{=} \DataTypeTok{WPair}\NormalTok{ w }\FunctionTok{.}\NormalTok{ makePT}
\end{Highlighting}
\end{Shaded}

The above basically says "to make a \texttt{WeightedPT} with weight \texttt{w},
first \texttt{makePT} it, and then add that result it to a \texttt{WPair\ w}.

\begin{Shaded}
\begin{Highlighting}[]
\NormalTok{ghci}\FunctionTok{>} \KeywordTok{let}\NormalTok{ pt }\FunctionTok{=}\NormalTok{ makeWPT }\DecValTok{1} \CharTok{'w'}
\NormalTok{ghci}\FunctionTok{>} \FunctionTok{:}\NormalTok{t pt}
\DataTypeTok{WeightedPT} \DataTypeTok{Char}
\NormalTok{ghci}\FunctionTok{>}\NormalTok{ pt}
\DataTypeTok{WPair} \DecValTok{1}\NormalTok{ (}\DataTypeTok{PTLeaf} \CharTok{'w'}\NormalTok{)}
\end{Highlighting}
\end{Shaded}

We will also want to merge two \texttt{WeightedPT}s:

\begin{Shaded}
\begin{Highlighting}[]
\CommentTok{-- source: https://github.com/mstksg/inCode/tree/master/code-samples/huffman/PreTree.hs#L62-L64}
\CommentTok{-- interactive: https://www.fpcomplete.com/user/jle/huffman-encoding}

\OtherTok{mergeWPT ::} \DataTypeTok{WeightedPT}\NormalTok{ a }\OtherTok{->} \DataTypeTok{WeightedPT}\NormalTok{ a }\OtherTok{->} \DataTypeTok{WeightedPT}\NormalTok{ a}
\NormalTok{mergeWPT (}\DataTypeTok{WPair}\NormalTok{ w1 pt1) (}\DataTypeTok{WPair}\NormalTok{ w2 pt2)}
    \FunctionTok{=} \DataTypeTok{WPair}\NormalTok{ (w1 }\FunctionTok{+}\NormalTok{ w2) (mergePT pt1 pt2)}
\end{Highlighting}
\end{Shaded}

so that the total weight is the sum of the weights of the two subtrees.

Finally, the entire point of having weighted things is so that we can compare
them and impose some total ordering. Haskell has a typeclass that abstracts
these comparing operations, \texttt{Ord}:

\begin{Shaded}
\begin{Highlighting}[]
\CommentTok{-- source: https://github.com/mstksg/inCode/tree/master/code-samples/huffman/Weighted.hs#L17-L21}
\CommentTok{-- interactive: https://www.fpcomplete.com/user/jle/huffman-encoding}

\KeywordTok{instance} \DataTypeTok{Eq}\NormalTok{ (}\DataTypeTok{Weighted}\NormalTok{ a) }\KeywordTok{where}
    \DataTypeTok{WPair}\NormalTok{ w1 _ }\FunctionTok{==} \DataTypeTok{WPair}\NormalTok{ w2 _ }\FunctionTok{=}\NormalTok{ w1 }\FunctionTok{==}\NormalTok{ w2}

\KeywordTok{instance} \DataTypeTok{Ord}\NormalTok{ (}\DataTypeTok{Weighted}\NormalTok{ a) }\KeywordTok{where}
\NormalTok{    compare (}\DataTypeTok{WPair}\NormalTok{ w1 _) (}\DataTypeTok{WPair}\NormalTok{ w2 _) }\FunctionTok{=}\NormalTok{ compare w1 w2}
\end{Highlighting}
\end{Shaded}

Which says that \texttt{Weighted\ a} is an \texttt{Ord} (is
orderable/comparable), and to compare two \texttt{WPair\ w\ x}'s, you compare
the \texttt{w}'s.

\begin{Shaded}
\begin{Highlighting}[]
\NormalTok{ghci}\FunctionTok{>}\NormalTok{ makeWPT }\DecValTok{2} \CharTok{'a'} \FunctionTok{>}\NormalTok{ makeWPT }\DecValTok{3} \CharTok{'b'}
\DataTypeTok{False}
\NormalTok{ghci}\FunctionTok{>}\NormalTok{ makeWPT }\DecValTok{4} \CharTok{'t'} \FunctionTok{==}\NormalTok{ makeWPT }\DecValTok{4} \CharTok{'k'}
\DataTypeTok{True}
\end{Highlighting}
\end{Shaded}

\hypertarget{priority-queues}{%
\section{Priority Queues}\label{priority-queues}}

There are some great priority queue libraries on Hackage, like
\href{http://hackage.haskell.org/package/PSQueue}{PSQueue}. However, for fun,
we're going to be making our own! Yay!

Our Priority Queue code module is
\href{https://github.com/mstksg/inCode/tree/master/code-samples/huffman/PQueue.hs}{available
for download} to try out!

\hypertarget{skew-heaps}{%
\subsection{Skew heaps}\label{skew-heaps}}

A traditional approach to making efficient priority queues is to use a
\href{http://en.wikipedia.org/wiki/Heap_(data_structure)}{heap}, a tree with
insertion algorithms that make sure the root of the tree is the most prioritized
element and that the tree stays balanced. Heaps make heavy use of stateful
mutation to do this, and while it's not so hard to do this in Haskell, we might
consider a `pure' version of a heap: a
\href{http://en.wikipedia.org/wiki/Skew_heap}{skew heap}.

A skew heap is a heap that doesn't explicitly maintain its balance, but
maintains ``heap ordering'' (parents are always higher priority than their
children).

I'll leave it to the wikipedia article to do most of the explaining because they
have pretty pictures, but here is the gist of it --- skew heaps have only three
operations: making new (singleton) one, merging two skew heaps, and popping off
the root. Traditional ``insert'' is done by making a new skew heap with one
element, and merging it with the main heap.

Merging is simple enough: The higher-priority root becomes the new root, and the
lower-priority root is merged with the child tree of that new root. (Some
left-right flipping of branches is done to make sure things tend to stay
balanced. Pictures provided in the wikipedia article.)

Popping the root is simple too; just take the root, and merge its two sub-trees
to make the new tree.

This is a new type of binary tree, so let's define a new data type:

\begin{Shaded}
\begin{Highlighting}[]
\CommentTok{-- source: https://github.com/mstksg/inCode/tree/master/code-samples/huffman/PQueue.hs#L20-L22}
\CommentTok{-- interactive: https://www.fpcomplete.com/user/jle/huffman-encoding}

\KeywordTok{data} \DataTypeTok{SkewHeap}\NormalTok{ a }\FunctionTok{=} \DataTypeTok{SEmpty}
                \FunctionTok{|} \DataTypeTok{SNode}\NormalTok{ a (}\DataTypeTok{SkewHeap}\NormalTok{ a) (}\DataTypeTok{SkewHeap}\NormalTok{ a)}
                \KeywordTok{deriving}\NormalTok{ (}\DataTypeTok{Show}\NormalTok{, }\DataTypeTok{Eq}\NormalTok{, }\DataTypeTok{Foldable}\NormalTok{)}
\end{Highlighting}
\end{Shaded}

Creating a new \texttt{SkewHeap} with one item:

\begin{Shaded}
\begin{Highlighting}[]
\CommentTok{-- source: https://github.com/mstksg/inCode/tree/master/code-samples/huffman/PQueue.hs#L25-L26}
\CommentTok{-- interactive: https://www.fpcomplete.com/user/jle/huffman-encoding}

\OtherTok{makeSH ::}\NormalTok{ a }\OtherTok{->} \DataTypeTok{SkewHeap}\NormalTok{ a}
\NormalTok{makeSH x }\FunctionTok{=} \DataTypeTok{SNode}\NormalTok{ x }\DataTypeTok{SEmpty} \DataTypeTok{SEmpty}
\end{Highlighting}
\end{Shaded}

Popping the root off of a skew tree:

\begin{Shaded}
\begin{Highlighting}[]
\CommentTok{-- source: https://github.com/mstksg/inCode/tree/master/code-samples/huffman/PQueue.hs#L31-L33}
\CommentTok{-- interactive: https://www.fpcomplete.com/user/jle/huffman-encoding}

\OtherTok{popSH ::} \DataTypeTok{Ord}\NormalTok{ a }\OtherTok{=>} \DataTypeTok{SkewHeap}\NormalTok{ a }\OtherTok{->}\NormalTok{ (}\DataTypeTok{Maybe}\NormalTok{ a, }\DataTypeTok{SkewHeap}\NormalTok{ a)}
\NormalTok{popSH }\DataTypeTok{SEmpty}          \FunctionTok{=}\NormalTok{ (}\DataTypeTok{Nothing}\NormalTok{, }\DataTypeTok{SEmpty}\NormalTok{)}
\NormalTok{popSH (}\DataTypeTok{SNode}\NormalTok{ r h1 h2) }\FunctionTok{=}\NormalTok{ (}\DataTypeTok{Just}\NormalTok{ r , mergeSH h1 h2)}
\end{Highlighting}
\end{Shaded}

We make it return a potential result (\texttt{Maybe\ a}), and the resulting new
popped tree. The result is \texttt{Maybe\ a} because we might potentially not be
able to pop anything! We also require an \texttt{Ord} constraint because in
order to merge two skew heaps, the data must be comparable.

Finally, the hardest piece of code so far: merging two skew heaps:

\begin{Shaded}
\begin{Highlighting}[]
\CommentTok{-- source: https://github.com/mstksg/inCode/tree/master/code-samples/huffman/PQueue.hs#L37-L42}
\CommentTok{-- interactive: https://www.fpcomplete.com/user/jle/huffman-encoding}

\OtherTok{mergeSH ::} \DataTypeTok{Ord}\NormalTok{ a }\OtherTok{=>} \DataTypeTok{SkewHeap}\NormalTok{ a }\OtherTok{->} \DataTypeTok{SkewHeap}\NormalTok{ a }\OtherTok{->} \DataTypeTok{SkewHeap}\NormalTok{ a}
\NormalTok{mergeSH }\DataTypeTok{SEmpty}\NormalTok{ h }\FunctionTok{=}\NormalTok{ h}
\NormalTok{mergeSH h }\DataTypeTok{SEmpty} \FunctionTok{=}\NormalTok{ h}
\NormalTok{mergeSH hA}\FunctionTok{@}\NormalTok{(}\DataTypeTok{SNode}\NormalTok{ xA lA rA) hB}\FunctionTok{@}\NormalTok{(}\DataTypeTok{SNode}\NormalTok{ xB lB rB)}
    \FunctionTok{|}\NormalTok{ xA }\FunctionTok{<}\NormalTok{ xB    }\FunctionTok{=} \DataTypeTok{SNode}\NormalTok{ xA (mergeSH rA hB) lA}
    \FunctionTok{|}\NormalTok{ otherwise  }\FunctionTok{=} \DataTypeTok{SNode}\NormalTok{ xB (mergeSH rB hA) lB}
\end{Highlighting}
\end{Shaded}

Hopefully this is very pleasing to read --- it reads a lot like a specification,
or a math formula:

\begin{enumerate}
\def\labelenumi{\arabic{enumi}.}
\tightlist
\item
  Merging any skew heap with an empty heap is that same skew heap.
\item
  When merging two heaps, the new heap is an \texttt{SNode} with the smaller
  root, whose children are the merge of the smaller tree and the original
  children. (Admittedly, the math/code is a bit more expressive than English in
  this case)
\end{enumerate}

(Remember that in our case, the \emph{lower} value/weight is the \emph{higher}
priority.)

We require an \texttt{Ord} constraint because we compare the node element on the
third case.

\hypertarget{priority-queue-interface}{%
\subsection{Priority Queue interface}\label{priority-queue-interface}}

Ok, neat!

Let's wrap this up in a tidy interface/API for a \texttt{PQueue} type:

\begin{Shaded}
\begin{Highlighting}[]
\CommentTok{-- source: https://github.com/mstksg/inCode/tree/master/code-samples/huffman/PQueue.hs#L48-L71}
\CommentTok{-- interactive: https://www.fpcomplete.com/user/jle/huffman-encoding}

\KeywordTok{newtype} \DataTypeTok{PQueue}\NormalTok{ a }\FunctionTok{=} \DataTypeTok{PQ}\NormalTok{ (}\DataTypeTok{SkewHeap}\NormalTok{ a) }\KeywordTok{deriving} \DataTypeTok{Show}

\OtherTok{emptyPQ ::} \DataTypeTok{PQueue}\NormalTok{ a}
\NormalTok{emptyPQ }\FunctionTok{=} \DataTypeTok{PQ} \DataTypeTok{SEmpty}

\OtherTok{insertPQ ::} \DataTypeTok{Ord}\NormalTok{ a }\OtherTok{=>}\NormalTok{ a }\OtherTok{->} \DataTypeTok{PQueue}\NormalTok{ a }\OtherTok{->} \DataTypeTok{PQueue}\NormalTok{ a}
\NormalTok{insertPQ x (}\DataTypeTok{PQ}\NormalTok{ h) }\FunctionTok{=} \DataTypeTok{PQ}\NormalTok{ (mergeSH h (makeSH x))}

\OtherTok{popPQ ::} \DataTypeTok{Ord}\NormalTok{ a }\OtherTok{=>} \DataTypeTok{PQueue}\NormalTok{ a }\OtherTok{->}\NormalTok{ (}\DataTypeTok{Maybe}\NormalTok{ a, }\DataTypeTok{PQueue}\NormalTok{ a)}
\NormalTok{popPQ (}\DataTypeTok{PQ}\NormalTok{ h) }\FunctionTok{=}\NormalTok{ (res, }\DataTypeTok{PQ}\NormalTok{ h')}
  \KeywordTok{where}
\NormalTok{    (res, h') }\FunctionTok{=}\NormalTok{ popSH h}

\OtherTok{sizePQ ::} \DataTypeTok{PQueue}\NormalTok{ a }\OtherTok{->} \DataTypeTok{Int}
\NormalTok{sizePQ (}\DataTypeTok{PQ}\NormalTok{ h) }\FunctionTok{=}\NormalTok{ length (toList h)}
\end{Highlighting}
\end{Shaded}

(Notice \texttt{toList}, from the
\href{http://hackage.haskell.org/package/base-4.6.0.1/docs/Data-Foldable.html}{Foldable}
module; we derived \texttt{Foldable} so that we can use \texttt{toList} on our
\texttt{SkewHeap}s. If your Haskell implementation cannot derive foldable (if
you are not using GHC, for example) --- and even if your implementation can ---
it might be fun to think about how to implement \texttt{sizePQ} without it!)

We do this so that we hide our low-level skew heap implementation over a
``high-level'' priority queue interface. We do not export the \texttt{PQ}
constructor, so users cannot ever directly access the underlying skew heap. In
this case, the high level isn't much higher of a level, but it's good practice
to hide away the implementation details when you can in Haskell, a language
whose power lies so much in abstraction.

\hypertarget{building-our-huffman-encoding-tree}{%
\section{Building our Huffman encoding
tree}\label{building-our-huffman-encoding-tree}}

Now that we have what we need in place, let's get to doing building our tree.
(Again, all available
\href{https://github.com/mstksg/inCode/tree/master/code-samples/huffman/Huffman.hs}{for
download}.)

\hypertarget{frequency-tables}{%
\subsection{Frequency Tables}\label{frequency-tables}}

First, we need to have some sort of frequency table. We will use
\texttt{Data.Map.Strict}'s \texttt{Map} type:

\begin{Shaded}
\begin{Highlighting}[]
\CommentTok{-- source: https://github.com/mstksg/inCode/tree/master/code-samples/huffman/Huffman.hs#L19-L19}
\CommentTok{-- interactive: https://www.fpcomplete.com/user/jle/huffman-encoding}

\KeywordTok{type} \DataTypeTok{FreqTable}\NormalTok{ a }\FunctionTok{=} \DataTypeTok{Map}\NormalTok{ a }\DataTypeTok{Int}
\end{Highlighting}
\end{Shaded}

and we'll import the operations from \texttt{Data.Map.Strict} qualified:

\begin{Shaded}
\begin{Highlighting}[]
\KeywordTok{import} \KeywordTok{qualified} \DataTypeTok{Data.Map.Strict} \KeywordTok{as} \DataTypeTok{M}
\end{Highlighting}
\end{Shaded}

Just to work with things now, let's make a way to generate a \texttt{FreqTable}
from an arbitrary string:

\begin{Shaded}
\begin{Highlighting}[]
\CommentTok{-- source: https://github.com/mstksg/inCode/tree/master/code-samples/huffman/Huffman.hs#L22-L25}
\CommentTok{-- interactive: https://www.fpcomplete.com/user/jle/huffman-encoding}

\OtherTok{listFreq ::} \DataTypeTok{Ord}\NormalTok{ a }\OtherTok{=>}\NormalTok{ [a] }\OtherTok{->} \DataTypeTok{FreqTable}\NormalTok{ a}
\NormalTok{listFreq }\FunctionTok{=}\NormalTok{ foldr f M.empty}
  \KeywordTok{where}
\NormalTok{    f x m }\FunctionTok{=}\NormalTok{ M.insertWith (}\FunctionTok{+}\NormalTok{) x }\DecValTok{1}\NormalTok{ m}
\end{Highlighting}
\end{Shaded}

This says that \texttt{listFreq} is a fold, where you start with
\texttt{M.empty} (an empty \texttt{FreqTable}) and for every element, you insert
it into the map as a key with value 1. If the key already exists, add one to its
current value instead.

\begin{Shaded}
\begin{Highlighting}[]
\NormalTok{ghci}\FunctionTok{>}\NormalTok{ listFreq }\StringTok{"hello world"}
\NormalTok{fromList [(}\CharTok{' '}\NormalTok{,}\DecValTok{1}\NormalTok{),(}\CharTok{'d'}\NormalTok{,}\DecValTok{1}\NormalTok{),(}\CharTok{'e'}\NormalTok{,}\DecValTok{1}\NormalTok{),(}\CharTok{'h'}\NormalTok{,}\DecValTok{1}\NormalTok{),(}\CharTok{'l'}\NormalTok{,}\DecValTok{3}\NormalTok{),(}\CharTok{'o'}\NormalTok{,}\DecValTok{2}\NormalTok{),(}\CharTok{'r'}\NormalTok{,}\DecValTok{1}\NormalTok{),(}\CharTok{'w'}\NormalTok{,}\DecValTok{1}\NormalTok{)]}
\end{Highlighting}
\end{Shaded}

\hypertarget{building-the-queue}{%
\subsection{Building the queue}\label{building-the-queue}}

Next, we would like to create Huffman leaves out of all of these elements, with
associated weights, and insert them all into a \texttt{PQueue}. We can do this
by using \texttt{M.foldrWithKey}, which is a \texttt{foldr} over the map, giving
the folding function both the key and the value.

\begin{Shaded}
\begin{Highlighting}[]
\CommentTok{-- source: https://github.com/mstksg/inCode/tree/master/code-samples/huffman/Huffman.hs#L43-L46}
\CommentTok{-- interactive: https://www.fpcomplete.com/user/jle/huffman-encoding}

\OtherTok{listQueue ::} \DataTypeTok{Ord}\NormalTok{ a }\OtherTok{=>}\NormalTok{ [a] }\OtherTok{->} \DataTypeTok{PQueue}\NormalTok{ (}\DataTypeTok{Weighted}\NormalTok{ a)}
\NormalTok{listQueue }\FunctionTok{=}\NormalTok{ M.foldrWithKey f emptyPQ }\FunctionTok{.}\NormalTok{ listFreq}
  \KeywordTok{where}
\NormalTok{    f k v pq }\FunctionTok{=}\NormalTok{ insertPQ (}\DataTypeTok{WPair}\NormalTok{ v k) pq}
\end{Highlighting}
\end{Shaded}

\begin{Shaded}
\begin{Highlighting}[]
\NormalTok{ghci}\FunctionTok{>} \KeywordTok{let}\NormalTok{ pq }\FunctionTok{=}\NormalTok{ listQueue }\StringTok{"hello world"}
\NormalTok{ghci}\FunctionTok{>} \FunctionTok{:}\NormalTok{t pq}
\OtherTok{pq ::} \DataTypeTok{PQueue}\NormalTok{ (}\DataTypeTok{WPair} \DataTypeTok{Int}\NormalTok{ (}\DataTypeTok{PreTree} \DataTypeTok{Char}\NormalTok{))}
\NormalTok{ghci}\FunctionTok{>}\NormalTok{ sizePQ pq}
\DecValTok{8}
\NormalTok{ghci}\FunctionTok{>} \KeywordTok{let}\NormalTok{ (popped1, pq') }\FunctionTok{=}\NormalTok{ popPQ pq}
\NormalTok{ghci}\FunctionTok{>}\NormalTok{ popped1}
\DataTypeTok{Just}\NormalTok{ (}\DataTypeTok{WPair} \DecValTok{1} \CharTok{' '}\NormalTok{)}
\NormalTok{ghci}\FunctionTok{>} \KeywordTok{let}\NormalTok{ (popped2, pq'') }\FunctionTok{=}\NormalTok{ popPQ pq'}
\NormalTok{ghci}\FunctionTok{>}\NormalTok{ popped2}
\DataTypeTok{Just}\NormalTok{ (}\DataTypeTok{WPair} \DecValTok{1} \CharTok{'d'}\NormalTok{)}
\NormalTok{ghci}\FunctionTok{>}\NormalTok{ sizePQ pq''}
\DecValTok{6}
\end{Highlighting}
\end{Shaded}

\hypertarget{building-the-tree}{%
\subsection{Building the tree}\label{building-the-tree}}

Building the tree is going to be a bit harder than a simple fold over the queue,
because we have to ``branch'' based on the state of the queue. Depending on the
state of the queue, we make decisions on ``control flow''.

The experienced Haskelleur will recognize that this language is very evocative
of the Monad design pattern.

\hypertarget{the-state-monad}{%
\subsubsection{The State monad}\label{the-state-monad}}

In particular, we will be using the State monad, which is basically a plain ol'
newtype wrapper around functions \texttt{s\ -\textgreater{}\ (a,\ s)}.
Basically, functions that act on a state and return a value with a modified
state.

These functions are actually surprisingly useful, and as it turns out, all
stateful computations can be described as ``compositions'' of these functions.

What do I mean by ``compositions''?

Let's say I have two functions:

\begin{Shaded}
\begin{Highlighting}[]
\OtherTok{f1 ::}\NormalTok{ s }\OtherTok{->}\NormalTok{ (a, s)}
\OtherTok{f2 ::}\NormalTok{ s }\OtherTok{->}\NormalTok{ (b, s)}
\end{Highlighting}
\end{Shaded}

And I wanted to sequence them:

\begin{Shaded}
\begin{Highlighting}[]
\NormalTok{f1 }\OtherTok{`andThen`}\NormalTok{ f2}
\end{Highlighting}
\end{Shaded}

What would that even ``look like''?

Well, I expect that sequencing two state functions will return a new, ``giant''
state function that does both functions ``one after the other''. That is:

\begin{Shaded}
\begin{Highlighting}[]
\NormalTok{f1 }\OtherTok{`andThen` f2 ::}\NormalTok{ s }\OtherTok{->}\NormalTok{ (b, s)}
\end{Highlighting}
\end{Shaded}

This new function will first run the input state on \texttt{f1}, and take that
resulting state and pass it into \texttt{f2}, and then return the result of
\texttt{f2} and the resulting modified state of \texttt{f2}.

So we have something like

\begin{Shaded}
\begin{Highlighting}[]
\OtherTok{andThen ::}\NormalTok{ (s }\OtherTok{->}\NormalTok{ (a,s)) }\OtherTok{->}\NormalTok{ (s }\OtherTok{->}\NormalTok{ (b,s)) }\OtherTok{->}\NormalTok{ (s }\OtherTok{->}\NormalTok{ (b,s))}
\NormalTok{andThen f1 f2 }\FunctionTok{=}\NormalTok{ \textbackslash{}st }\OtherTok{->} \KeywordTok{let}\NormalTok{ (_,st') }\FunctionTok{=}\NormalTok{ f1 s}
                       \KeywordTok{in}\NormalTok{  f2 st'}
\end{Highlighting}
\end{Shaded}

Think of \texttt{andThen} like a semicolon, of sorts.

Notice that we ``lose'' the result of \texttt{f1} with \texttt{andThen}. What if
we wanted to use it? We might write a combinator:

\begin{verbatim}
andThenWith ::       (s -> (a, s))
            -> (a -> (s -> (b, s)))
            ->       (s -> (b, s))
\end{verbatim}

Which you would use like

\begin{Shaded}
\begin{Highlighting}[]
\NormalTok{f1 }\OtherTok{`andThenWith`}\NormalTok{ (\textbackslash{}x }\OtherTok{->}\NormalTok{ f2 x)}
\end{Highlighting}
\end{Shaded}

where \texttt{f2} is a function that takes an \texttt{a} and returns a
\texttt{s\ -\textgreater{}\ (a,s)}.

Basically, it would be exactly the same as \texttt{andThen}, except the second
argument gets access to the result of the first. Writing it is almost as simple,
actually ---

\begin{Shaded}
\begin{Highlighting}[]
\OtherTok{andThenWith ::}\NormalTok{ (s }\OtherTok{->}\NormalTok{ (a,s)) }\OtherTok{->}\NormalTok{ (a }\OtherTok{->}\NormalTok{ (s }\OtherTok{->}\NormalTok{ (b, s))) }\OtherTok{->}\NormalTok{ (s }\OtherTok{->}\NormalTok{ (b, s))}
\NormalTok{andThenWith f1 f2 }\FunctionTok{=}\NormalTok{ \textbackslash{}st }\OtherTok{->} \KeywordTok{let}\NormalTok{ (x,st') }\FunctionTok{=}\NormalTok{ f1 s}
                           \KeywordTok{in}\NormalTok{  (f2 x) st'}
\end{Highlighting}
\end{Shaded}

As it turns out\ldots{}if you squint hard enough, the type signature
\texttt{andThenWith} looks a lot like the type signature for
\texttt{(\textgreater{}\textgreater{}=)}:

\begin{Shaded}
\begin{Highlighting}[]
\OtherTok{(>>=) ::} \DataTypeTok{Monad}\NormalTok{ m }\OtherTok{=>}\NormalTok{ m a }\OtherTok{->}\NormalTok{ (a }\OtherTok{->}\NormalTok{ m b) }\OtherTok{->}\NormalTok{ m b}
\end{Highlighting}
\end{Shaded}

Hm. Let's create a type synonym for our \texttt{s\ -\textgreater{}\ (a,\ s)}, to
make things more clear.

\begin{Shaded}
\begin{Highlighting}[]
\KeywordTok{type} \DataTypeTok{State}\NormalTok{ s a }\FunctionTok{=}\NormalTok{ s }\OtherTok{->}\NormalTok{ (a, s)}
\end{Highlighting}
\end{Shaded}

So now our \texttt{andThenWith} looks like:

\begin{Shaded}
\begin{Highlighting}[]
\OtherTok{andThenWith ::} \DataTypeTok{State}\NormalTok{ s a }\OtherTok{->}\NormalTok{ (a }\OtherTok{->} \DataTypeTok{State}\NormalTok{ s b) }\OtherTok{->} \DataTypeTok{State}\NormalTok{ s b}
\end{Highlighting}
\end{Shaded}

If we let \texttt{m\ \textasciitilde{}\ State\ s}:

\begin{Shaded}
\begin{Highlighting}[]
\OtherTok{andThenWith ::}\NormalTok{ m a }\OtherTok{->}\NormalTok{ (a }\OtherTok{->}\NormalTok{ m b) }\OtherTok{->}\NormalTok{ m b}
\end{Highlighting}
\end{Shaded}

Neat!

As it turns out, we can turn our state functions into a Monad, which
encapsulates ``sequencing'' state functions one after another.

We just need \texttt{return}:

\begin{Shaded}
\begin{Highlighting}[]
\OtherTok{returnState ::}\NormalTok{ a }\OtherTok{->} \DataTypeTok{State}\NormalTok{ s a}
\NormalTok{returnState x }\FunctionTok{=}\NormalTok{ \textbackslash{}st }\OtherTok{->}\NormalTok{ (x, st)}
\end{Highlighting}
\end{Shaded}

And \texttt{return} is \texttt{returnState},
\texttt{(\textgreater{}\textgreater{})} is \texttt{andThen}, and
\texttt{(\textgreater{}\textgreater{}=)} is \texttt{andThenWith}.

In real life, we can't define typeclass instances on type synonyms, so we
actually use a \texttt{newtype}. The standard implementation comes from the
\href{http://hackage.haskell.org/package/transformers}{transformers} library.
Because \texttt{State\ s} is a member of the \texttt{Monad} typeclass, we can
use normal monad combinators, operators, and do notation. The transformers
implementation comes with a few useful primitives:

\begin{Shaded}
\begin{Highlighting}[]
\CommentTok{-- wrap a normal state function into the State wrapper}
\OtherTok{state ::}\NormalTok{ (s }\OtherTok{->}\NormalTok{ (a, s)) }\OtherTok{->} \DataTypeTok{State}\NormalTok{ s a}

\CommentTok{-- get grabs the state as the result.}
\OtherTok{get ::} \DataTypeTok{State}\NormalTok{ s s}
\NormalTok{get }\FunctionTok{=}\NormalTok{ state (\textbackslash{}st }\OtherTok{->}\NormalTok{ (st, st))}

\CommentTok{-- put sets the state to the input}
\OtherTok{put ::}\NormalTok{ s }\OtherTok{->} \DataTypeTok{State}\NormalTok{ s ()}
\NormalTok{put s }\FunctionTok{=}\NormalTok{ state (\textbackslash{}_ }\OtherTok{->}\NormalTok{ ((), st))}

\CommentTok{-- modifies the state with the given function}
\OtherTok{modify ::}\NormalTok{ (s }\OtherTok{->}\NormalTok{ s) }\OtherTok{->} \DataTypeTok{State}\NormalTok{ s ()}
\NormalTok{modify f }\FunctionTok{=}\NormalTok{ state (\textbackslash{}st }\OtherTok{->}\NormalTok{ ((), f st))}

\CommentTok{-- alternative implementation of `modify`}
\NormalTok{modify'}\OtherTok{ f ::}\NormalTok{ (s }\OtherTok{->}\NormalTok{ s) }\OtherTok{->} \DataTypeTok{State}\NormalTok{ s ()}
\NormalTok{modify' f }\FunctionTok{=} \KeywordTok{do}
\NormalTok{    st }\OtherTok{<-}\NormalTok{ get}
\NormalTok{    put (f st)}
\end{Highlighting}
\end{Shaded}

If you're still lost, check out Brandon Simmon's
\href{http://brandon.si/code/the-state-monad-a-tutorial-for-the-confused/}{state
monad tutorial}, which was the article that eventually cleared it all up for
myself. And feel free to ask questions!

The \emph{big} usefulness for this ``composing stateful functions'' business,
instead of manually unwrapping and re-wrapping the state, is that now
\texttt{State} actions are first-class, and you can freely compose them and pass
them around as objects, and you can write individual ``sub-routines'', are
little packets of commands that modify state, and then ``call them'' and compose
them from other stateful computations.

\hypertarget{why-monads}{%
\subsubsection{Why monads?}\label{why-monads}}

One might pause to wonder why we would want to instance our
\texttt{s\ -\textgreater{}\ (a,\ s)} functions as a Monad. Why can't we just
always sequence our state functions using \texttt{andThen} and
\texttt{andThenWith}?

\begin{enumerate}
\def\labelenumi{\arabic{enumi}.}
\item
  Using monads, we can now use \texttt{do} notation, which is pretty nice sugar.
\item
  We now have access to the wide library of useful Haskell
  \href{http://hackage.haskell.org/package/base-4.6.0.1/docs/Control-Monad.html}{monad
  combinators}. And boy are there a lot --- \texttt{sequence}, \texttt{mapM},
  \texttt{when}, \texttt{filterM}, etc.
\item
  We also get an Applicative instance for free, so we can do arbitrary-arity
  lifting with things like
  \texttt{f\ \textless{}\$\textgreater{}\ x\ \textless{}*\textgreater{}\ y},
  where \texttt{f} is a pure function like \texttt{(+)} and \texttt{x} and
  \texttt{y} are stateful functions. We also get a free Functor instance as
  well, so we can \texttt{fmap}.
\item
  We can now reason with our stateful functions with all of the powerful
  equational reasoning tools that the monad laws offer.
\end{enumerate}

As you can see, monads are not just a curiosity --- they are a powerful and
expressive tool!

\hypertarget{a-quick-look-back}{%
\subsubsection{A quick look back}\label{a-quick-look-back}}

This is a bit of an unrelated aside\ldots{}but notice that we could have
actually done our previous \texttt{fold}s as state monad operations; like
\texttt{listFreq}:

\begin{Shaded}
\begin{Highlighting}[]
\CommentTok{-- source: https://github.com/mstksg/inCode/tree/master/code-samples/huffman/Huffman.hs#L29-L36}
\CommentTok{-- interactive: https://www.fpcomplete.com/user/jle/huffman-encoding}

\OtherTok{runListFreq ::}\NormalTok{ forall a}\FunctionTok{.} \DataTypeTok{Ord}\NormalTok{ a }\OtherTok{=>}\NormalTok{ [a] }\OtherTok{->} \DataTypeTok{FreqTable}\NormalTok{ a}
\NormalTok{runListFreq xs }\FunctionTok{=}\NormalTok{ execState listFreqState M.empty}
  \KeywordTok{where}
\OtherTok{    listFreqState ::} \DataTypeTok{State}\NormalTok{ (}\DataTypeTok{FreqTable}\NormalTok{ a) ()}
\NormalTok{    listFreqState }\FunctionTok{=}\NormalTok{ mapM_ addFreq xs}

\OtherTok{    addFreq ::}\NormalTok{ a }\OtherTok{->} \DataTypeTok{State}\NormalTok{ (}\DataTypeTok{FreqTable}\NormalTok{ a) ()}
\NormalTok{    addFreq x }\FunctionTok{=}\NormalTok{ modify (M.insertWith (}\FunctionTok{+}\NormalTok{) x }\DecValTok{1}\NormalTok{)}
\end{Highlighting}
\end{Shaded}

\texttt{execState} runs the given \texttt{State} computation with the given
initial state, and returns the final state \texttt{s} at the end of it all.
Basically, it takes an \texttt{s\ -\textgreater{}\ (a,\ s)} (the
\texttt{State\ s\ a}), an \texttt{s}, applies the function to it, and returns
just the \texttt{s} in the tuple.

Remember that the best way to read \texttt{State\ s\ a} is just ``a type synonym
for \texttt{s\ -\textgreater{}\ (a,s)}''. So when we say
\texttt{listFreqState\ ::\ State\ (FreqTable\ a)\ ()}, we mean that
\texttt{listFreqState} is a function from a \texttt{FreqTable\ a} to
\texttt{((),\ FreqTable\ a)}.

How about \texttt{listQueue}? We could do it with the state monad too, if we
wanted to.

\begin{Shaded}
\begin{Highlighting}[]
\CommentTok{-- source: https://github.com/mstksg/inCode/tree/master/code-samples/huffman/Huffman.hs#L50-L59}
\CommentTok{-- interactive: https://www.fpcomplete.com/user/jle/huffman-encoding}

\OtherTok{listQueueState ::} \DataTypeTok{Ord}\NormalTok{ a }\OtherTok{=>}\NormalTok{ [a] }\OtherTok{->} \DataTypeTok{State}\NormalTok{ (}\DataTypeTok{PQueue}\NormalTok{ (}\DataTypeTok{WeightedPT}\NormalTok{ a)) ()}
\NormalTok{listQueueState xs }\FunctionTok{=}\NormalTok{ M.traverseWithKey addNode (listFreq xs) }\FunctionTok{>>}\NormalTok{ return ()}
  \KeywordTok{where}
\OtherTok{    addNode ::}\NormalTok{ a }\OtherTok{->} \DataTypeTok{Int} \OtherTok{->} \DataTypeTok{State}\NormalTok{ (}\DataTypeTok{PQueue}\NormalTok{ (}\DataTypeTok{WeightedPT}\NormalTok{ a)) ()}
\NormalTok{    addNode x i }\FunctionTok{=}\NormalTok{ modify (insertPQ (}\DataTypeTok{WPair}\NormalTok{ i (makePT x)))}

\OtherTok{runListQueue ::} \DataTypeTok{Ord}\NormalTok{ a }\OtherTok{=>}\NormalTok{ [a] }\OtherTok{->} \DataTypeTok{PQueue}\NormalTok{ (}\DataTypeTok{WeightedPT}\NormalTok{ a)}
\NormalTok{runListQueue xs }\FunctionTok{=}\NormalTok{ execState (listQueueState xs) emptyPQ}
\end{Highlighting}
\end{Shaded}

In these cases, the monadic usage isn't quite necessary or useful on its own. A
fold would have probably been more expressive and easier to read. The above
examples were just for demonstrations/exercises.

But when do we ``need'' the state monad? (Or rather, when is a fold not powerful
enough or much messier?)

It's when we want to make decisions or ``branch'' based on the current state, or
the results of our state actions. ``Fold for three items; if the next list item
is even then do this fold afterwards, otherwise do that fold''. This is when the
state monad shines as a monad.

Another case where we might want to use a state monad over a fold is if we
forsee us wanting to ``compose'' our folds into bigger stateful computations.
For example, in \texttt{listQueueState}, we ``process'' a state, and leave it
modified for \emph{another state monad action} to use.

For example:

\begin{Shaded}
\begin{Highlighting}[]
\OtherTok{prepareQueue ::} \DataTypeTok{State}\NormalTok{ (}\DataTypeTok{PQueue}\NormalTok{ (}\DataTypeTok{WeightedPT}\NormalTok{ a)) ()}
\OtherTok{useQueue     ::} \DataTypeTok{State}\NormalTok{ (}\DataTypeTok{PQueue}\NormalTok{ (}\DataTypeTok{WeightedPT}\NormalTok{ a)) a}

\OtherTok{doAllTogether ::} \DataTypeTok{Ord}\NormalTok{ a }\OtherTok{=>}\NormalTok{ [a] }\OtherTok{->} \DataTypeTok{State}\NormalTok{ (}\DataTypeTok{PQueue}\NormalTok{ (}\DataTypeTok{WeightedPT}\NormalTok{ a)) a}
\NormalTok{doAllTogether xs }\FunctionTok{=}\NormalTok{ prepareQueue }\FunctionTok{>>}\NormalTok{ listQueueState xs }\FunctionTok{>>}\NormalTok{ useQueue}

\CommentTok{-- alternatively, the same thing but in do notation}
\OtherTok{doAllTogether' ::} \DataTypeTok{Ord}\NormalTok{ a }\OtherTok{=>}\NormalTok{ [a] }\OtherTok{->} \DataTypeTok{State}\NormalTok{ (}\DataTypeTok{PQueue}\NormalTok{ (}\DataTypeTok{WeightedPT}\NormalTok{ a)) a}
\NormalTok{doAllTogether' xs }\FunctionTok{=} \KeywordTok{do}
\NormalTok{    prepareQueue}
\NormalTok{    listQueueState xs}
\NormalTok{    useQueue}

\OtherTok{runDoAllTogether ::} \DataTypeTok{Ord}\NormalTok{ a }\OtherTok{=>}\NormalTok{ [a] }\OtherTok{->}\NormalTok{ a}
\NormalTok{runDoAllTogether xs }\FunctionTok{=}\NormalTok{ evalState (doAllTogether xs) emptyPQ}
\end{Highlighting}
\end{Shaded}

(Remember that \texttt{(\textgreater{}\textgreater{})} is just our
\texttt{andThen}, and when we sequence using
\texttt{(\textgreater{}\textgreater{})} we mean ``combine these two actions into
one big action that feeds the resulting state of the left side into the
beginning state of the right side.'')

Anyways, see that we can just plop a call to \texttt{listQueueState} inside a
sequence of stateful actions, and it'll just process the queue and leave it for
the next action to use.

If we had used \texttt{listQueue} as a ``pure'' fold\ldots{}this is a bit harder
to do. You'd have to rewrite \texttt{listQueue} to take in any arbitrary
``starting queue''\ldots{}extract the starting queue using \texttt{get} after
\texttt{prepareQueue}, use a \texttt{let} to bind it as a pure function, then
use \texttt{put} to pop the result back into the state for \texttt{useQueue} to
use. Or use \texttt{modify} in a just-as-convoluted way.

Moving on, we actually won't be using \texttt{runListFreq} in the future (it was
mostly for fun), but (spoilers) we might want to hold onto
\texttt{listQueueState} :)

\hypertarget{building-with-state}{%
\subsection{Building with State}\label{building-with-state}}

So let's remember how the building process works:

\begin{enumerate}
\def\labelenumi{\arabic{enumi}.}
\tightlist
\item
  Pop an item from the queue.
\item
  Pop another item. If the queue was actually empty, and nothing was poppable,
  you are done; return the result of step 1.
\item
  Merge the two popped items, and push them back into the queue. Go back to step
  1.
\end{enumerate}

Sounds simple enough. We should take into account that we would fail to build a
tree if the queue was empty to begin with, by returning a
\texttt{Maybe\ (PreTree\ a)} instead of a \texttt{PreTree\ a}.

\begin{Shaded}
\begin{Highlighting}[]
\CommentTok{-- source: https://github.com/mstksg/inCode/tree/master/code-samples/huffman/Huffman.hs#L75-L98}
\CommentTok{-- interactive: https://www.fpcomplete.com/user/jle/huffman-encoding}

\OtherTok{buildTree ::} \DataTypeTok{State}\NormalTok{ (}\DataTypeTok{PQueue}\NormalTok{ (}\DataTypeTok{WeightedPT}\NormalTok{ a)) (}\DataTypeTok{Maybe}\NormalTok{ (}\DataTypeTok{PreTree}\NormalTok{ a))}
\NormalTok{buildTree }\FunctionTok{=} \KeywordTok{do}
\NormalTok{    t1' }\OtherTok{<-}\NormalTok{ state popPQ}
    \KeywordTok{case}\NormalTok{ t1' }\KeywordTok{of}
      \DataTypeTok{Nothing} \OtherTok{->}
        \CommentTok{-- queue was empty to begin with, so this fails.}
\NormalTok{        return }\DataTypeTok{Nothing}
      \DataTypeTok{Just}\NormalTok{ t1 }\OtherTok{->} \KeywordTok{do}
\NormalTok{        t2' }\OtherTok{<-}\NormalTok{ state popPQ}
        \KeywordTok{case}\NormalTok{ t2' }\KeywordTok{of}
          \DataTypeTok{Nothing}  \OtherTok{->}
            \CommentTok{-- We're done, there was only one item!  Return a `Just` to}
            \CommentTok{-- indicate success.}
\NormalTok{            return (}\DataTypeTok{Just}\NormalTok{ (_wItem t1))     }\CommentTok{-- break out of the loop}
          \DataTypeTok{Just}\NormalTok{ t2 }\OtherTok{->} \KeywordTok{do}
            \CommentTok{-- merge and push}
            \KeywordTok{let}\NormalTok{ combined }\FunctionTok{=}\NormalTok{ mergeWPT t1 t2}
\NormalTok{            modify (insertPQ combined)}
\NormalTok{            buildTree                     }\CommentTok{-- recursive call}

\OtherTok{runBuildTree ::} \DataTypeTok{Ord}\NormalTok{ a }\OtherTok{=>}\NormalTok{ [a] }\OtherTok{->}\NormalTok{ (}\DataTypeTok{Maybe}\NormalTok{ (}\DataTypeTok{PreTree}\NormalTok{ a))}
\NormalTok{runBuildTree xs }\FunctionTok{=}\NormalTok{ evalState (listQueueState xs }\FunctionTok{>>}\NormalTok{ buildTree) emptyPQ}
\end{Highlighting}
\end{Shaded}

Note that due to our uncanny foresight,
\texttt{popPQ\ ::\ PQueue\ a\ -\textgreater{}\ (Maybe\ a,\ PQueue\ a)} is
already a state function \texttt{s\ -\textgreater{}\ (a,\ s)}, where the state
is \texttt{PSQueue\ a} and the return value is \texttt{Maybe\ a}. So all we need
to do is say \texttt{state\ popPQ} to wrap it in the \texttt{State\ s\ a}
newtype wrapper/container, and it becomes an ``official''
\texttt{State\ (PQueue\ a)\ (Maybe\ a)}.

Remember that \texttt{State\ s\ a} is \emph{just a thin wrapper/container} over
a function \texttt{s\ -\textgreater{}\ (a,\ s)}, anyway, so the two should be
somewhat equivalent in your mind; the requirement to wrap it in \texttt{State}
using \texttt{state} is only because of Haskell's own language limitations
(namely, that you can't define a Monad instance for
\texttt{s\ -\textgreater{}\ (a,\ s)} in a clean way). When you read
\texttt{State\ s\ a}, you \emph{should really read}
\texttt{s\ -\textgreater{}\ (a,\ s)}, because they are for the most part
\emph{completely equivalent}.

Again, \texttt{(\textgreater{}\textgreater{})} is Monad-speak for our
\texttt{andThen} function we defined earlier, so for \texttt{buildTree}, we do
``\texttt{listQueueState\ xs} \emph{and then} \texttt{buildTree}''.
\texttt{(\textgreater{}\textgreater{})} joins two
\texttt{s\ -\textgreater{}\ (a,\ s)} functions into one giant
\texttt{s\ -\textgreater{}\ (a,s)}, by feeding the resulting state of the first
action into the next one. \texttt{listQueueState} takes an empty priority queue
and `fills' it with nodes generated from \texttt{xs}, leaving a filled priority
queue. \texttt{buildTree} then takes that filled queue and performs our building
operations on it, modifying it as it goes along, and ends up with an empty queue
as a state and returning the finished tree as a result.

\texttt{evalState} is like partner of \texttt{execState} --- it runs the state
operation on the given starting state, and outputs the final result (instead of
the final state). It takes an \texttt{s\ -\textgreater{}\ (a,\ s)}, an
\texttt{s}, and applies the function to it and gives the resulting \texttt{a} of
the tuple.

\hypertarget{putting-it-all-together}{%
\subsection{Putting it all together}\label{putting-it-all-together}}

Let's try it out, shall we?

\begin{Shaded}
\begin{Highlighting}[]
\NormalTok{ghci}\FunctionTok{>}\NormalTok{ fromJust }\FunctionTok{$}\NormalTok{ runBuildTree }\StringTok{"hello world"}
\DataTypeTok{PTNode}\NormalTok{ (}\DataTypeTok{PTNode}\NormalTok{ (}\DataTypeTok{PTNode}\NormalTok{ (}\DataTypeTok{PTLeaf} \CharTok{'h'}\NormalTok{)}
\NormalTok{                       (}\DataTypeTok{PTLeaf} \CharTok{'e'}\NormalTok{)}
\NormalTok{               )}
\NormalTok{               (}\DataTypeTok{PTNode}\NormalTok{ (}\DataTypeTok{PTLeaf} \CharTok{'w'}\NormalTok{)}
\NormalTok{                       (}\DataTypeTok{PTLeaf} \CharTok{'r'}\NormalTok{)}
\NormalTok{               )}
\NormalTok{       )}
\NormalTok{       (}\DataTypeTok{PTNode}\NormalTok{ (}\DataTypeTok{PTLeaf} \CharTok{'l'}\NormalTok{)}
\NormalTok{               (}\DataTypeTok{PTNode}\NormalTok{ (}\DataTypeTok{PTNode}\NormalTok{ (}\DataTypeTok{PTLeaf} \CharTok{'d'}\NormalTok{)}
\NormalTok{                               (}\DataTypeTok{PTLeaf} \CharTok{' '}\NormalTok{)}
\NormalTok{                       )}
\NormalTok{                       (}\DataTypeTok{PTLeaf} \CharTok{'o'}\NormalTok{)}
\NormalTok{               )}
\NormalTok{       )}
\end{Highlighting}
\end{Shaded}

Congrats, we built a Huffman encoding tree! Notice that the most commonly used
letter (\texttt{\textquotesingle{}l\textquotesingle{}}, occurring 3 times) is
only at depth 2 (and is most accessible), while the others are at depths 3 and
4.

\hypertarget{next-steps}{%
\section{Next steps}\label{next-steps}}

That's it for this post, it's already long enough!

In the next posts we will look at how we would use this Huffman tree to encode
and decode text, and general bytes (\texttt{Word8}s), and then hook it all up to
make a ``streaming'' compressor and uncompressor that reads a file byte-by-byte
and outputs a compressed file as it goes. We'll then figure out how to store
this huffman tree in a compact, serialized binary way, and load it cleanly.

In the mean time, try
\href{https://github.com/mstksg/inCode/tree/master/code-samples/huffman}{downloading
the source}, or
\href{https://www.fpcomplete.com/user/jle/huffman-encoding}{playing with it
online} on \href{http://www.fpcomplete.com}{fpcomplete}!

Hi, thanks for reading! You can reach me via email at
\href{mailto:justin@jle.im}{\nolinkurl{justin@jle.im}}, or at twitter at
\href{https://twitter.com/mstk}{@mstk}! This post and all others are published
under the \href{https://creativecommons.org/licenses/by-nc-nd/3.0/}{CC-BY-NC-ND
3.0} license. Corrections and edits via pull request are welcome and encouraged
at \href{https://github.com/mstksg/inCode}{the source repository}.

If you would like to donate, I am currently accepting bitcoin donations at
\emph{\href{bitcoin:3D7rmAYgbDnp4gp4rf22THsGt74fNucPDU}{3D7rmAYgbDnp4gp4rf22THsGt74fNucPDU}}!

\end{document}
