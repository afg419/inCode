\documentclass[]{article}
\usepackage{lmodern}
\usepackage{amssymb,amsmath}
\usepackage{ifxetex,ifluatex}
\usepackage{fixltx2e} % provides \textsubscript
\ifnum 0\ifxetex 1\fi\ifluatex 1\fi=0 % if pdftex
  \usepackage[T1]{fontenc}
  \usepackage[utf8]{inputenc}
\else % if luatex or xelatex
  \ifxetex
    \usepackage{mathspec}
    \usepackage{xltxtra,xunicode}
  \else
    \usepackage{fontspec}
  \fi
  \defaultfontfeatures{Mapping=tex-text,Scale=MatchLowercase}
  \newcommand{\euro}{€}
\fi
% use upquote if available, for straight quotes in verbatim environments
\IfFileExists{upquote.sty}{\usepackage{upquote}}{}
% use microtype if available
\IfFileExists{microtype.sty}{\usepackage{microtype}}{}
\usepackage[margin=1in]{geometry}
\ifxetex
  \usepackage[setpagesize=false, % page size defined by xetex
              unicode=false, % unicode breaks when used with xetex
              xetex]{hyperref}
\else
  \usepackage[unicode=true]{hyperref}
\fi
\hypersetup{breaklinks=true,
            bookmarks=true,
            pdfauthor={},
            pdftitle={},
            colorlinks=true,
            citecolor=blue,
            urlcolor=blue,
            linkcolor=magenta,
            pdfborder={0 0 0}}
\urlstyle{same}  % don't use monospace font for urls
% Make links footnotes instead of hotlinks:
\renewcommand{\href}[2]{#2\footnote{\url{#1}}}
\setlength{\parindent}{0pt}
\setlength{\parskip}{6pt plus 2pt minus 1pt}
\setlength{\emergencystretch}{3em}  % prevent overfull lines
\setcounter{secnumdepth}{0}


\begin{document}

\% Functional Programming is Awesome: Parser Combinators \% Justin Le

\emph{Originally posted on
\textbf{\href{https://blog.jle.im/entry/functional-programming-is-awesome-parser-combinators.html}{in
Code}}.}

If you've spent your entire life programming imperatively, you could not even
begin to imagine the vast worlds of thinking that are inaccessible because of
restrictions in your thinking. Not "better", or "harder" ways of thinking
-\/-\/- possibly just as easy, but...different.

Don't believe me? Here is an example of an amazingly awesome and amazing and
incomprehensibly mind-opeining design pattern that shines in functional
programming: applicative combinators. In specific, we are going to write a small
parser library! Yay!

Now, if you are a sane person, you should know that \emph{nobody} enjoys writing
parsers...imperatively. The whole deal is a mess; you have to keep track of
multiple layers of state, work out back tracking and edge cases, and to top it
off, parsers tend to be very uncomposable, and parsing two different things
usually requires either source code surgery in the best case or a complete
rewrite in the worst.

But what if I told you that combinator-based parsers are so easy to work with
and composable and intuitive that they are...well, fun? I'll tease you:

\textasciitilde{}\textasciitilde{}\textasciitilde{}haskell -\/- a parser for a
special variable name, which can optionally start with an -\/- \texttt{@}
symbol, and then an alphabet character, followed by many alphanumeric -\/-
characters variableName :: Parser String variableName = (optional '@')
\textless{}:\textgreater{} alpha \textless{}:\textgreater{} (many alphaNum)

-\/- usage: -\/- λ: parse variableName "9one" -\/- Nothing -\/- λ: parse
variableName "var01" -\/- Just ("var01", "") -\/- λ: parse variableName "@var2"
-\/- Just ("@var2", "") -\/- λ: parse variableName "@9one" -\/- Nothing
\textasciitilde{}\textasciitilde{}\textasciitilde{}

Let's take a look, shall we? We will first write our library, and then see how
we build complex parsers out of smaller, simpler ones. And maybe learn a little
bit of Haskell along the way -\/-\/- in particular, the magic of Functors and
Applicative Functors (and maybe some Monads if we're lucky).

(Much of this post is taken from a University of Göteborg
\href{http://www.cse.chalmers.se/edu/course/TDA452/FPLectures/Vid/}{lecture
series} available online; it has been slightly rephrased to be more
applicative-centric than monad-centric)

\section{Our Parser Type}

So; our goal here is to write a library for parsing that is powerful to be
completely general and adaptable for any purpose. And also be very intuitive to
use. Easy, right? Let's start with the type of our Parser!

\subsection{Our Parser type}

So our Parser type is going to be an object of type \texttt{Parser\ a} that
contains one thing: a parsing function. This parsing function will take in a
string and return the parsed value of type \texttt{a}, together with the
"leftover" string that is unparsed.

In Haskell, we represent this object as:

\textasciitilde{}\textasciitilde{}\textasciitilde{}haskell data Parser a = P
(String -\textgreater{} (a, String))
\textasciitilde{}\textasciitilde{}\textasciitilde{}

which reads "The type \texttt{Parser\ a} is an object that you specify by saying
'\texttt{P\ f}', where \texttt{f} is a function that takes in a string and
returns an \texttt{a} with another string.

\textless{}div class="note"\textgreater{} \textbf{Aside}

Hi! These asides are for people unfamiliar with Haskell syntax. Feel free to
skip them if you already are comfortable.

If you've ever used an object-oriented language like Java or C++,
\texttt{Parser\ a} is a template, or generic, and would be written in those
languages as the class \texttt{Parser\textless{}a\textgreater{}}, with only one
instance variable: a function taking a string and returning something of type
\texttt{a} with a leftover string. \textless{}/div\textgreater{}

And we write a function \texttt{parse} that will take any \texttt{Parser\ a}
object and any string, and return the parsed string:

\textasciitilde{}\textasciitilde{}\textasciitilde{}haskell parse :: Parser a
-\textgreater{} String -\textgreater{} (a, String) -\/- 1 parse (P f) str = f
str -\/- 2 \textasciitilde{}\textasciitilde{}\textasciitilde{}

\textless{}div class="note"\textgreater{} \textbf{Aside}

A bunch of Haskell syntax features here!

\begin{enumerate}
\item
  This first line says "the function \texttt{parse} is a function that takes a
  \texttt{Parser\ a} and a \texttt{String} and return something of the value
  \texttt{(a,\ String)}"
\item
  Now to define the function. It has two arguments; the first one is of the form
  \texttt{P\ f}, and the second is just a string \texttt{str}. Remember,
  \texttt{P\ f} is how you specify a \texttt{Parser\ a}; the \texttt{f} is the
  function inside the object. We then call \texttt{f} with \texttt{str}, and
  that's what we want.
\end{enumerate}

\textless{}/div\textgreater{}

Let's say we had a parser \texttt{integerParser} pre-made, that parses a string
into an integer. Here is how we would use it in the interactive
\href{http://en.wikipedia.org/wiki/Read-eval-print_loop}{repl} prompt:

\textasciitilde{}\textasciitilde{}\textasciitilde{}haskell λ: :type
integerParser integerParser :: Parser Int λ: parse integerParser "12" (12, "")
-\/- returns the integer 12, with nothing leftover λ: parse integerParser "-82
zero" (82, " zero") -\/- returns the integer -82, with some leftovers λ: parse
integerParser "hello" ?????? -\/- ??????
\textasciitilde{}\textasciitilde{}\textasciitilde{}

\textless{}div class="note"\textgreater{} \textbf{Aside}

In this article, code that begins with \texttt{λ:} means stuff entered at the
interactive prompt, ghci. You enter an expression, and it is evaluated, its
result printed.

\texttt{:type} is a ghci command that returns the type of the thing in question.
\textless{}/div\textgreater{}

Hm. There's a problem. Sometimes, the parse will fail. How can we indicate that
a parse is failable?

In most languages, we might return a \texttt{nil} or \texttt{null} value. But
this is not cool in Haskell because we like having type safety -\/-\/- having
things be \texttt{nil} or \texttt{null} willy-nilly means that your compiler
can't help you, and you open up the door to runtime errors.

In Haskell, we have an object of type \texttt{Maybe\ a}, which can either be
\texttt{Just\ a} (the object contains a value \texttt{a}), or \texttt{Nothing}
(the object contains nothing). Let's change our function so that it will return
\texttt{Just\ a} if our parse succeeds and \texttt{Nothing} if it fails.

\textless{}div class="note"\textgreater{} \textbf{Aside}

Again keeping with the analogies to Object-Oriented Programming, you can think
of \texttt{Maybe\ a} as a superclass, \texttt{Maybe\textless{}a\textgreater{}},
with two subclasses: \texttt{Just\ a} and \texttt{Nothing}. \texttt{Just\ a}
contains one instance variable of type \texttt{a}, and \texttt{Nothing}
contains...nothing. \textless{}/div\textgreater{}

\textasciitilde{}\textasciitilde{}\textasciitilde{}haskell data Parser a = P
(String -\textgreater{} Maybe (a, String))

parse :: Parser a -\textgreater{} String -\textgreater{} Maybe (a, String) parse
(P f) str = f str \textasciitilde{}\textasciitilde{}\textasciitilde{}

\textasciitilde{}\textasciitilde{}\textasciitilde{}haskell λ: parse
integerParser "12" Just (12, "") -\/- succesfully returns the integer 12, inside
a -\/- Maybe object λ: parse integerParser "-82 zero" Just (-82, " zero") -\/-
succesfully returns the integer -82 λ: parse integerParser "hello" Nothing -\/-
fails \textasciitilde{}\textasciitilde{}\textasciitilde{}

Okay, cool. Let's get started.

\subsection{Our base parsers}

So remember, the whole point is to be able to somehow build complex parsers out
of very simple ones. So let's start with the simplest parsers of all: a parser
that always fails.

\textasciitilde{}\textasciitilde{}\textasciitilde{}haskell failure :: Parser a
failure = P returnNothing -\/- 1 where returnNothing \_ = Nothing -\/- 2
\textasciitilde{}\textasciitilde{}\textasciitilde{}

\textasciitilde{}\textasciitilde{}\textasciitilde{}haskell λ: parse failure "12"
Nothing λ: parse failure "anything" Nothing
\textasciitilde{}\textasciitilde{}\textasciitilde{}

\textless{}div class="note"\textgreater{} \textbf{Aside}

\begin{enumerate}
\tightlist
\item
  Remember, we specify/construct parsers as \texttt{P\ f}, where \texttt{f} is
  the parsing function.
\item
  \texttt{\_} is a wildcard in Haskell; \texttt{returnNothing} is a function
  that takes \emph{anything} and returns \texttt{Nothing}.
\end{enumerate}

\textless{}/div\textgreater{}

Easy enough. How about some parsers that always succeeds with a pre-defined
answer, and does not consume any of the string?

\textasciitilde{}\textasciitilde{}\textasciitilde{}haskell successfulTrue ::
Parser Bool successfulTrue = P returnTrue where returnTrue str = Just (True,
str)

succesfulZero :: Parser Int succesfulZero = P returnZero where returnZero str =
Just (0, str) \textasciitilde{}\textasciitilde{}\textasciitilde{}

\textasciitilde{}\textasciitilde{}\textasciitilde{}haskell λ: parse
succesfulTrue "12" Just (True, "12") λ: parse succesfulZero "anything" Just (0,
"anything") \textasciitilde{}\textasciitilde{}\textasciitilde{}

We have two parsers here, with two different parse functions. If you notice,
they pretty much do the same thing, and have very similar structure. This is a
key flag that we can probably abstract out some part of it. Let's write a new
parse function that covers both of them:

\textasciitilde{}\textasciitilde{}\textasciitilde{}haskell successful :: a
-\textgreater{} Parser a successful val = P returnVal where returnVal str = Just
(val, str)

succesfulTrue :: Parser Bool succesfulTrue = succesful True

succesfulZero :: Parser Int succesfulZero = succesful 0
\textasciitilde{}\textasciitilde{}\textasciitilde{}

So \texttt{successful\ val} is a function that takes any value and gives us a
\texttt{Parser\ a} with the function \texttt{returnVal} -\/-\/-
\texttt{returnVal\ str} yields a \texttt{Just} with the \texttt{val} we gave to
\texttt{successful}, with the string untouched. Let's try it out.

\textasciitilde{}\textasciitilde{}\textasciitilde{}haskell λ: parse (successful
5) "hello" Just (5, "hello") λ: parse (successful 8.26) "hello" Just (8.26,
"hello") λ: :type (successful True) succesful True :: Parser Bool
\textasciitilde{}\textasciitilde{}\textasciitilde{}

Note that the type of \texttt{succesful\ True} is exactly the same type as our
\texttt{succesfulTrue} function. Which is what we would expect. Our
\texttt{succesful\ val} function takes a value and returns a parser that parses
anything into that given value. So if we pass in \texttt{True}, it will return a
parser/function that parses anything into \texttt{True}.

Okay, so our two base parses are nice, but it doesn't sound like they are useful
at all for any real parsing work. Let's write our first
somewhat-useful-kinda-maybe parser: \texttt{anyChar}

\textasciitilde{}\textasciitilde{}\textasciitilde{}haskell anyChar :: Parser
Char anyChar = P getFirst where getFirst (x:xs) = Just (x, xs) -\/- 1 getFirst
{[}{]} = Nothing -\/- 2 \textasciitilde{}\textasciitilde{}\textasciitilde{}

\textless{}div class="note"\textgreater{} \textbf{Aside}

\begin{enumerate}
\tightlist
\item
  If you call \texttt{getFirst} on a non-empty list, it will set the \texttt{x}
  to the first element and \texttt{xs} to the rest of the list. Kind of like in
  C, a string is just an alias for a list of characters. So
  \texttt{getFirst\ "hello"} will return
  \texttt{Just\ (\textquotesingle{}h\textquotesingle{},\ "ello")}.
\item
  If you call \texttt{getFirst} on an empty list, it fails.
\end{enumerate}

\textless{}/div\textgreater{}

Let's see it at work.

\textasciitilde{}\textasciitilde{}\textasciitilde{}haskell λ: parse anyChar "h"
Just ('h', "") λ: parse anyChar "hello" Just ('h', "ello") λ: parse anyChar ""
Nothing \textasciitilde{}\textasciitilde{}\textasciitilde{}

So \texttt{anyChar} parses a string into a...character. That's...kind of boring,
admittedly. I'm sure you could have written something that parses a one-letter
string into a character in any language without a problem...one that would parse
\texttt{"h"}, the string, into \texttt{\textquotesingle{}h\textquotesingle{}},
the character.

Hm. Let's try something else. Let's try a parser that always fails \emph{unless}
the taken character satisfies a certain condition:

\textasciitilde{}\textasciitilde{}\textasciitilde{}haskell z :: Parser Char z =
P isA\emph{z where isA}z (x:xs) = if x == 'z' then Just (x, xs) else Nothing
isA\_z {[}{]} = Nothing

lowercase :: Parser Char lowercase = P isLowercase where isLowercase (x:xs) = if
isLower x then Just (x, xs) else Nothing isLowercase {[}{]} = Nothing
\textasciitilde{}\textasciitilde{}\textasciitilde{}

Remember that Haskell's built-in \texttt{Data.Char} module provides the function
\texttt{isLower\ ::\ Char\ -\textgreater{}\ Bool}, which is true if the given
character is a lowercase letter and false otherwise.

\textasciitilde{}\textasciitilde{}\textasciitilde{}haskell λ: parse z "not z"
Nothing λ: parse lowercase "A" Nothing λ: parse lowercase "a" Just ('a', "")
\textasciitilde{}\textasciitilde{}\textasciitilde{}

But wait! Like in the case for \texttt{succesful}, we notice a definite
pattern...and we can abstract this out.

\textasciitilde{}\textasciitilde{}\textasciitilde{}haskell satisfies :: (Char
-\textgreater{} Bool) -\textgreater{} Parser Char satisfies p = P
passesPredicate where passesPredicate (x:xs) = if p z then Just (x, xs) else
Nothing passesPredicate {[}{]} = Nothing

z :: Parser Char z = satisfies (== 'z')

lowercase :: Parser Char lowercase = satisfies isLower
\textasciitilde{}\textasciitilde{}\textasciitilde{}

\textless{}div class="note"\textgreater{} \textbf{Aside}

Similar in structure to \texttt{(*2)},
\texttt{(==\ \textquotesingle{}z\textquotesingle{})} is a function that returns
true if the input is equal to \texttt{\textquotesingle{}z\textquotesingle{}} and
false otherwise. That is,
\texttt{(==\ \textquotesingle{}z\textquotesingle{})\ \textquotesingle{}a\textquotesingle{}}
is false, while
\texttt{(==\ \textquotesingle{}z\textquotesingle{})\ \textquotesingle{}z\textquotesingle{}}
is true. \textless{}/div\textgreater{}

Remember, just like in the case with \texttt{succesful}: \texttt{satisfies\ p}
returns a \texttt{Parser\ Char} that fails unless the first character satisfies
the given predicate \texttt{p}; \texttt{p} has to be a function from
\texttt{Char} to \texttt{Bool} -\/-\/- a predicate on \texttt{Char}s. So then
\texttt{satisfies\ isLower} returns a new parser with the predicate
\texttt{isLower} as the "checker".

Let's try this out with some more built-in predicates from \texttt{Data.Char}:

\textasciitilde{}\textasciitilde{}\textasciitilde{}haskell λ: parse (satisfies
isAlpha) "hello" Just ('h', "ello") λ: parse (satisfies isDigit) "123abc" Just
('1', "23abc") λ: parse (== 'a') "abc" Just ('a', "bc")
\textasciitilde{}\textasciitilde{}\textasciitilde{}

In fact, some of these parsers are so useful, let's name some:

\textasciitilde{}\textasciitilde{}\textasciitilde{}haskell digit :: Parser Char
digit = satisfies isDigit

alphaNum :: Parser Char alphaNum = satisfies isAlphaNum

alpha :: Parser Char alphaNum = satisfies isAlpha

char :: Char -\textgreater{} Parser Char char c = satisfies (== c)
\textasciitilde{}\textasciitilde{}\textasciitilde{}

The last one is particularly neat -\/-\/- it says "accept only the given
character" -\/-\/- \texttt{char\ \textquotesingle{}a\textquotesingle{}}
generates a parser that succeeds only on the character \texttt{a}.

Note that we even can rewrite \texttt{anyChar} in terms of \texttt{satisfies}:

\textasciitilde{}\textasciitilde{}\textasciitilde{}haskell anyChar :: Parser
Char anyChar = satisfies (const True)
\textasciitilde{}\textasciitilde{}\textasciitilde{}

where \texttt{const\ True\ ::\ a\ -\textgreater{}\ Bool} is a function that
ignores all its input and always returns true no matter what. \texttt{anyChar}
is a parser that will take any character, no matter what.

\subsection{Now we're cooking}

Okay, so we are making baby steps. Here are our "base" parsers. They can't
really do much on their own; you can either always fail, always succeed, and
parse any one-character string into a character. But the last one...the ability
to \emph{selectively} parse a string into a character...that might be
interesting. It's not totally obvious yet how we can make it useful, but...I
think we have enough base parsers to start building our \emph{combinators}
-\/-\/- functions that combine and modify parsers.

\section{Functors}

You have probably worked with functors before in your programming life with or
without knowing it. Functors represent objects or data structures that "contain"
data that can be "mapped over" -\/-\/- that is, functors are objects where you
can apply functions to the values inside. The most famous functor in any
language is the list, \texttt{{[}{]}}; another neat one is \texttt{Maybe}.

\textasciitilde{}\textasciitilde{}\textasciitilde{}haskell λ: (\emph{2)
\textless{}\$\textgreater{} {[}1,2,3{]} {[}2,4,6{]} λ: (}2)
\textless{}\$\textgreater{} Just 3 Just 6
\textasciitilde{}\textasciitilde{}\textasciitilde{}

\textless{}div class="note"\textgreater{} \textbf{Aside}

First of all, to clarify, \texttt{(*2)} is a function that doubles whatever is
passed to it -\/-\/- \texttt{(*2)\ 3} is 6.

Next, in haskell, we can use the \texttt{\textless{}\$\textgreater{}} operator
to say "apply this...'inside' of the structure". This is analogous to the
more-often-used \texttt{\$} operator:

\textasciitilde{}\textasciitilde{}\textasciitilde{}haskell λ: (*2) \$ 3 6
\textasciitilde{}\textasciitilde{}\textasciitilde{}

Which says "apply this (\texttt{(*2)})...to this (\texttt{3})".

Compare that to:

\textasciitilde{}\textasciitilde{}\textasciitilde{}haskell λ: (\emph{2)
\textless{}\$\textgreater{} {[}3{]} {[}6{]} λ: (}2) \textless{}\$\textgreater{}
Just 3 6 \textasciitilde{}\textasciitilde{}\textasciitilde{}

As you can see, \texttt{\textless{}\$\textgreater{}} is the "container" version
of \texttt{\$}. It is like \texttt{\$}, but "inside".

For more information, refer to adit's
\href{http://adit.io/posts/2013-04-17-functors,_applicatives,_and_monads_in_pictures.html}{amazing
article} on the topic.

\textless{}/div\textgreater{}

You can probably see where I'm going with this.

\subsection{Turning our Parser into a Functor}

\subsubsection{What does that even mean?}

Hopefully you saw this coming. Functors seem to be very cool. But does it even
make sense that our Parser object is a functor? The \texttt{Parser\ Int} object
doesn't "contain" any \texttt{Int}s...does it?

Well...it kind of "does"...in two ways, sorta kinda maybe.

\begin{enumerate}
\tightlist
\item
  You could say that a \texttt{Parser\ Int} "contains" the to-be-parsed integer.
\item
  You could say that a \texttt{Parser\ Int} "contains" the \textbf{idea} of an
  integer...abstractly. Like, whoa, man.
\end{enumerate}

So...does it make sense to apply a function "inside" a \texttt{Parser\ a}? The
answer is yes!

\begin{enumerate}
\tightlist
\item
  If a \texttt{Parser\ Int} contains a to-be-parsed integer, then applying a
  function inside a \texttt{Parser\ Int} would be like applying the function to
  the to-be-parsed integer!
\item
  If a \texttt{Parser\ Int} contains the \emph{idea} of an integer, then
  applying a function inside a \texttt{Parser\ Int} would be applying the
  function to the idea of the integer.
\end{enumerate}

Huh. This is kinda cool, isn't it? So let's say that when we apply a function
"inside" our Parser...it means that we apply a function to the finished result
of a succesful parse.

Basically, we want something like this:

\textasciitilde{}\textasciitilde{}\textasciitilde{}haskell λ: let succesfulSix =
(*2) \textless{}\$\textgreater{} (succesful 3) λ: :type succesfulSix
succesfulSix :: Parser Int λ: parse succesfulSix "something" Just 6
\textasciitilde{}\textasciitilde{}\textasciitilde{}

Okay, cool. So instead of \texttt{succesful\ 3} always succeeding with a 3, it
succeeds with \texttt{(*2)\ 3}, which is 6. Still kinda boring but w/e.

Let's try something else...we're going to take advantage of the
\texttt{Data.Char} method \texttt{digitToInt\ ::\ Char\ -\textgreater{}\ Int},
which takes a \texttt{Char} and returns the digit that it represents (for
example, \texttt{digitToInt\ \textquotesingle{}5\textquotesingle{}} is 5, the
integer):

\textasciitilde{}\textasciitilde{}\textasciitilde{}haskell λ: let digitInt =
digitToInt \textless{}\$\textgreater{} digit -\/- 1 λ: :type digitInt digitInt
:: Parser Int λ: parse digitInt "2" Just (2, "") λ: parse digitInt "a" Nothing
\textasciitilde{}\textasciitilde{}\textasciitilde{}

\begin{enumerate}
\tightlist
\item
  Remember, we defined the parser \texttt{digit} earlier, as a parser that only
  succeeds if the first character is a digit.
\end{enumerate}

Did you just see that? That sounds genuinely useful! If we manage to implement
\texttt{\textless{}\$\textgreater{}} properly, we just have a way to selectively
parse any digit character into a digit, and fail otherwise.

Let's try chaining \texttt{\textless{}\$\textgreater{}}'s:

\textasciitilde{}\textasciitilde{}\textasciitilde{}haskell λ: let
digitIntDoubled = (*2) \textless{}\$\textgreater{} (digitToInt
\textless{}\$\textgreater{} digit) λ: :type digitIntDoubled digitIntDoubled ::
Parser Int λ: parse digitIntDoubled "2" Just (4, "") λ: parse digitIntDoubled
"a" Nothing \textasciitilde{}\textasciitilde{}\textasciitilde{}

What just happened here?

\begin{enumerate}
\tightlist
\item
  \texttt{digit} is a parser that succeeds only on digit characters.
\item
  \texttt{digitToInt\ \textless{}\$\textgreater{}\ digit} is the result of
  applying \texttt{digitToInt} on the value of a succesful parse. Basically,
  \texttt{digitToInt\ \textless{}\$\textgreater{}\ digit} is a \emph{new parser}
  that succeeds on digit characters, returning an integer.
\item
  \texttt{(*2)\ \textless{}\$\textgreater{}\ (digitToInt\ \textless{}\$\textgreater{}\ digit)}
  is the result of applying \texttt{(*2)} (the doubling function) to
  \texttt{digitToInt\ \textless{}\$\textgreater{}\ digit}, the
  return-digit-as-int parser. The \texttt{(*2)\ \textless{}\$\textgreater{}}
  doubles the result of the return-digit-as-int parser. Essentially, the entire
  expression is a \emph{new parser} that succeeds on digit character, returning
  the doubled integer as the parse result.
\end{enumerate}

*Phew*. That was a conceptual doozy. But this seems like a good path. Do you see
what we did? We turned old, boring parsers into new, interesting parsers by
"transforming" them. This is the kind of thing we were looking for in the first
place, right?

\subsubsection{Implementing the Functor instance}

That's all fine and dandy. Now we have to actually implement
\texttt{\textless{}\$\textgreater{}}.

Luckily in Haskell, there are mechanisms for uniting functor interfaces. That
is, if you tell it that \texttt{Parser\ a} is a functor, then you open up all
sorts of powerful functor tools that were written for things like lists and
maybe's -\/-\/- now we can use them all on \texttt{Parser\ a}, too.

To do this, we "instance" the Functor \emph{typeclass}. A typeclass in Haskell
is like an interface in Object-Oriented Programming. Basically, we say
"\texttt{Parser\ a} is a functor...treat it like one!" But in order to make it a
valid functor, we must tell it how to \texttt{fmap}. \texttt{fmap} is basically
\texttt{\textless{}\$\textgreater{}} -\/-\/- it takes a function, a functor
object, and returns a new functor object with that function applied "inside".
Once we define \texttt{fmap}, then we get \texttt{\textless{}\$\textgreater{}}
for free. Or rather, Haskell knows that our parser is a functor, and knows how
to use \texttt{\textless{}\$\textgreater{}} on functors.

Here's how we do it. Hold on tight!

\textasciitilde{}\textasciitilde{}\textasciitilde{}haskell instance Functor
Parser where -\/- 1 fmap f parser = P mapped\emph{parser}function -\/- 2 where

\begin{verbatim}
        unmapped_result :: Maybe (a, String)
        unmapped_result = parse parser str                      -- 3

        mapped_parser_function :: String -> Maybe (b, String)
        mapped_parser_function str =
            case unmapped_result of                             -- 4
                Just (result, rest) -> Just (f result, rest)    -- 5
                Nothing             -> Nothing                  -- 6
\end{verbatim}

\textasciitilde{}\textasciitilde{}\textasciitilde{}

\textless{}div class="note"\textgreater{} \textbf{Aside}

\begin{enumerate}
\tightlist
\item
  This line is syntax for "we are declaring that \texttt{Parser} is a functor!",
  so that Haskell can treat \texttt{Parser} like a functor (and have
  \texttt{\textless{}\$\textgreater{}} work).
\item
  In order to declare something as a functor, we must define the function
  \texttt{fmap\ ::\ (a\ -\textgreater{}\ b)\ -\textgreater{}\ Parser\ a\ -\textgreater{}\ Parser\ b},
  which takes a function from \texttt{a} to \texttt{b}, a \texttt{Parser\ a},
  and returns the mapped-over parser/parser with the function applied "inside".
\item
  Grab the result of the parse with the unmapped parser, first.
\item
  Some more Haskell syntax. Basically, the result of
  \texttt{mapped\_parser\_function} is dependent on what
  \texttt{unmapped\_result} is.
\item
  If \texttt{unmapped\_result} is a \texttt{Just}, then return the success, but
  apply \texttt{f} to the result first.
\item
  If \texttt{unmapped\_result} is \texttt{Nothing}...well, you can't apply
  anything to the result if the result is a failure (a \texttt{Nothing}). Pass
  on the failure.
\end{enumerate}

\textless{}/div\textgreater{}

And now we can use our \texttt{Parser\ a} as a functor; all of our use cases
above will now work.

\subsection{Fun with functors}

Let's take a brief moment to see what kind of things we can do with only
functors, and nothing else. It's probably not going to be too much more useful
than what you can write imperatively, but it is interesting what you can do by
only 'modifying' simpler parsers...which is what the whole point of combinators
are.

With that in mind, let's try out some neat use cases.

What if you want to parse a digit and look it up in a list of keywords, and
return that keyword?

\textasciitilde{}\textasciitilde{}\textasciitilde{}haskell keywords ::
{[}String{]} keywords = {[}"debug","normal","warning","error"{]}

digit :: Parser Char digit = satisfies isDigit

digitInt :: Parser Int digitInt = digitToInt \textless{}\$\textgreater{} digit

keyword :: Parser String keyword = (!! keywords) \textless{}\$\textgreater{}
digitInt \textasciitilde{}\textasciitilde{}\textasciitilde{}

\textless{}div class="note"\textgreater{} \textbf{Aside}

\texttt{(!!\ keywords)} is a function that looks up the given index in the given
list. In our case, \texttt{(!!\ keywords)\ 2} would return what in other
languages is \texttt{keywords{[}2{]}}, which is the string "warning".
\textless{}/div\textgreater{}

\textasciitilde{}\textasciitilde{}\textasciitilde{}haskell λ: parse keyword "1"
Just ("normal", "") λ: parse keyword "hello" Nothing
\textasciitilde{}\textasciitilde{}\textasciitilde{}

\textless{}!-\/- \textasciitilde{}\textasciitilde{}\textasciitilde{}haskell
-\/-\textgreater{} \textless{}!-\/- λ: parse keyword "1" -\/-\textgreater{}
\textless{}!-\/- Just ("normal", "") -\/-\textgreater{} \textless{}!-\/- λ:
parse keyword "hello" -\/-\textgreater{} \textless{}!-\/- Nothing
-\/-\textgreater{} \textless{}!-\/- λ: parse keyword "67" -\/-\textgreater{}
\textless{}!-\/- *** Exception: (!!): index too large -\/-\textgreater{}
\textless{}!-\/- \textasciitilde{}\textasciitilde{}\textasciitilde{}
-\/-\textgreater{}

\textless{}!-\/- Oops! We forgot that \texttt{!!} is an unsafe/partial function
(it is undefined for -\/-\textgreater{} \textless{}!-\/- some inputs)...it
should have been obvious becuase \texttt{!!} is a pretty -\/-\textgreater{}
\textless{}!-\/- unsafe-looking name for a function. It tried to find index 6 of
the -\/-\textgreater{} \textless{}!-\/- list...but that index doesn't exist.
-\/-\textgreater{}

\textless{}!-\/- There are ways to take care of that kind of sloppily, but what
we want is a -\/-\textgreater{} \textless{}!-\/- function that takes a parser, a
condition (the number has to be less than -\/-\textgreater{} \textless{}!-\/-
four, for instance), and returns a new parser with that condition built-in.
-\/-\textgreater{} \textless{}!-\/- Some spoilers -\/-\/- an object that has
this ability is known as a "monad". Let's -\/-\textgreater{} \textless{}!-\/-
keep this in mind as we move forward with our applicative parsers.
-\/-\textgreater{}

\end{document}
