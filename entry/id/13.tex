\documentclass[]{article}
\usepackage{lmodern}
\usepackage{amssymb,amsmath}
\usepackage{ifxetex,ifluatex}
\usepackage{fixltx2e} % provides \textsubscript
\ifnum 0\ifxetex 1\fi\ifluatex 1\fi=0 % if pdftex
  \usepackage[T1]{fontenc}
  \usepackage[utf8]{inputenc}
\else % if luatex or xelatex
  \ifxetex
    \usepackage{mathspec}
    \usepackage{xltxtra,xunicode}
  \else
    \usepackage{fontspec}
  \fi
  \defaultfontfeatures{Mapping=tex-text,Scale=MatchLowercase}
  \newcommand{\euro}{€}
\fi
% use upquote if available, for straight quotes in verbatim environments
\IfFileExists{upquote.sty}{\usepackage{upquote}}{}
% use microtype if available
\IfFileExists{microtype.sty}{\usepackage{microtype}}{}
\usepackage[margin=1in]{geometry}
\ifxetex
  \usepackage[setpagesize=false, % page size defined by xetex
              unicode=false, % unicode breaks when used with xetex
              xetex]{hyperref}
\else
  \usepackage[unicode=true]{hyperref}
\fi
\hypersetup{breaklinks=true,
            bookmarks=true,
            pdfauthor={},
            pdftitle={},
            colorlinks=true,
            citecolor=blue,
            urlcolor=blue,
            linkcolor=magenta,
            pdfborder={0 0 0}}
\urlstyle{same}  % don't use monospace font for urls
% Make links footnotes instead of hotlinks:
\renewcommand{\href}[2]{#2\footnote{\url{#1}}}
\setlength{\parindent}{0pt}
\setlength{\parskip}{6pt plus 2pt minus 1pt}
\setlength{\emergencystretch}{3em}  % prevent overfull lines
\setcounter{secnumdepth}{0}


\begin{document}

\% Wolf, Goat, Cabbage: The List MonadPlus \& Logic Problems \% Justin Le \%
December 26, 2013

\emph{Originally posted on
\textbf{\href{https://blog.jle.im/entry/wolf-goat-cabbage-the-list-monadplus-logic-problems.html}{in
Code}}.}

Today we're going to learn to solve the classic and ageless logic problems
without any data structures besides List's monadic properties as a MonadPlus!

We are going to be solving this old-as-time logic puzzle, which
\href{http://en.wikipedia.org/wiki/Fox,_goose_and_bag_of_beans_puzzle}{Wikipedia}
claims dates back to the 9th century:

\begin{quote}
A farmer has a wolf, a goat, and a cabbage that he wishes to transport across a
river. Unfortunately, his boat can carry only one thing at a time with him. He
can't leave the wolf alone with the goat, or the wolf will eat the goat. He
can't leave the goat alone with the cabbage, or the goat will eat the cabbage.
How can he properly transport his belongings to the other side one at a time,
without any disasters?
\end{quote}

We're going to assume a somewhat basic familiarity with functional programming
concepts and a basic understanding of monads (if you don't know that much, check
out
\href{http://adit.io/posts/2013-04-17-functors,_applicatives,_and_monads_in_pictures.html}{adit's}
great concice guide). If you aren't familiar with MonadPlus/Alternative (and how
they work as monads) check out
\href{http://blog.jle.im/entry/practical-fun-with-monads-introducing-monadplus}{Part
1} and
\href{http://blog.jle.im/entry/the-list-monadplus-practical-fun-with-monads-part}{Part
2}, which should provide all the background and most of the syntax. Most Haskell
syntax is either be explained here as we get to it or in the previous parts.
Still, if you have any questions, feel free to leave a comment, give
\href{http://learnyouahaskell.com}{Learn You A Haskell} a quick read, or stop by
freenode's friendly \#haskell!

\subsection{A MonadPlus Review}

The usefulness of a monad depends on how you define the characteristic "bind" or
"chaining" behavior. For this article, MonadPlus refers to the design pattern
(and Haskell typeclass) where you model this "chaining" as a "success/fail"
process{[}\^{}plus{]}.

There is a common language with to talk about this process: \texttt{mzero} means
"fail here" and \texttt{return\ x} means "succeed with a result of the value
\texttt{x} here". So chaining is implemented such that chaining anything to a
failure will propagate that failure forward. That is,
\texttt{mzero\ \textgreater{}\textgreater{}\ return\ x} = \texttt{mzero}.

\section{Our Approach}

So, armed with what we learned in
\href{http://blog.jle.im/entry/the-list-monadplus-practical-fun-with-monads-part}{Part
2}, let's formulate a general plan for finding all solutions in \texttt{n}
moves.

Now, in the List monad, we can think of things as "journeys" or stories: subject
your value to a long and arduous journey, specifying at every step of the way
what choices it has to continue. Then specify where journeys fail and end. At
the end of it all, the result is a list of the finishing values of all trails
that have completed the journey.

With the List monad, we say "Here is the description of \emph{a} (single)
journey. What journeys following this description succeed?"

So what could this journey be for us? Well, we think of a journey in this
situation as the accumulation of moves to a plan. We start out with a blank plan
("Do nothing"). The next step, we add one move to our plan ("Just move the fox",
for example). Then the next step, we add another move ("First move the fox, then
move the farmer").

\begin{enumerate}
\tightlist
\item
  Start with a blank plan; a tabula rasa.
\item
  Add a legal and safe move to it.
\item
  Repeat Step 2 \texttt{n} times
\item
  Fail if you aren't a solution; succeed if you are.
\end{enumerate}

Simple, right? We just laid out \emph{the path of a single plan}, from its birth
to its eventual death or ascension.

This is the most significant thing about this approach: it allows you to
describe \textbf{one journey}, in general terms, and List will "automatically"
find out all successful journeys that fit your mold. You don't ever have to
worry about the ensemble or manually deal with explicit branching or filtering.
Cognitively, all you have to do is \emph{write \textbf{one} story}. Just
\emph{one}. That is the power of the List Monad abstraction.

\section{Our Types}

The first thing we do when writing any Haskell program: define our types!

\textasciitilde{}\textasciitilde{}\textasciitilde{}haskell data Character =
Farmer \textbar{} Wolf \textbar{} Goat \textbar{} Cabbage -\/- 1 deriving (Show,
Eq, Enum)

newtype Move = MoveThe Character -\/- 2 deriving (Eq)

instance Show Move where -\/- 3 show (MoveThe Farmer) = "F" show (MoveThe Wolf)
= "W" show (MoveThe Goat) = "G" show (MoveThe Cabbage) = "C"

type Plan = {[}Move{]} -\/- 4

data Position = West \textbar{} East -\/- 5 deriving (Show, Eq)
\textasciitilde{}\textasciitilde{}\textasciitilde{}

\begin{enumerate}
\tightlist
\item
  First, we define the enumerated type \texttt{Character} all the characters we
  will be working with: the farmer, the wolf, the goat, and the cabbage.
\item
  Next, we define a simple \texttt{Move} container, which just contains a
  character. A \texttt{MoveThe\ Farmer} will represent a movement of only the
  farmer, a \texttt{MoveThe\ Wolf} will represent the movement of both the
  farmer and the wolf, etc.
\item
  For the purposes of easy debugging, we're going to define our own instance of
  \texttt{Show} for moves so that we can use \texttt{print} on them.
\item
  A simple type synonym; a \texttt{Plan} is just a list of \texttt{Move}s. Note
  that we are not using this list as a MonadPlus -\/-\/- it's just a plain dumb
  list of moves in our plan.
\item
  A \texttt{Position} type: either on the west bank or on the east bank of the
  river. Everyone starts out on the west bank, and we want them all to end up on
  the east bank.
\end{enumerate}

\textless{}div class="note"\textgreater{} \textbf{Welcome to Haskell!}

Hi! These "Welcome to Haskell" asides are for people unfamiliar with Haskell,
mostly for Haskell syntax stuff. If you already feel comfortable, feel free to
skip them.

There's a lot of Haskell syntax and concepts here; mostly, all we are doing is
declaring new types.

\begin{enumerate}
\item
  We declare that \texttt{Character} is "either" a \texttt{Farmer},
  \texttt{Wolf}, \texttt{Goat}, or \texttt{Cabbage}. This is like saying that a
  \texttt{Bool} is either a \texttt{False} or a \texttt{True}: in fact, you
  could define your own \texttt{Bool} with something like this: (or even your
  own \texttt{Int})

  \textasciitilde{}\textasciitilde{}\textasciitilde{}haskell data Bool = False
  \textbar{} True data Int = -536870912 ... \textbar{} -1 \textbar{} 0
  \textbar{} 1 \textbar{} 2 \textbar{} ... 536870911
  \textasciitilde{}\textasciitilde{}\textasciitilde{}

  The \texttt{deriving} syntax tells the compiler to automatically derive
  functions for printing the type (Show), testing for equality (Eq), and
  enumerating through them (Enum)
\item
  We declare a new type \texttt{Move} which is just a wrapper around a
  \texttt{Character}. We can create a new \texttt{Move} by using
  \texttt{MoveThe}:

  \textasciitilde{}\textasciitilde{}\textasciitilde{}haskell ghci\textgreater{}
  :t MoveThe MoveThe :: Character -\textgreater{} Move ghci\textgreater{} :t
  MoveThe Wolf MoveThe Wolf :: Move
  \textasciitilde{}\textasciitilde{}\textasciitilde{}

  (\texttt{ghci\textgreater{}} represents a command at the interactive prompt
  ghci, and \texttt{:t} asks for the type of whatever comes after it)
\item
  Here we define custom functions for printing out a \texttt{Move}
\item
  Here is a type synonym \texttt{Plan}. Every time we use \texttt{Plan} as a
  type, we really mean \texttt{{[}Move{]}}, and the compiler treats the two
  things as the same.
\item
  \texttt{Position}: same deal as \texttt{Character}.
\end{enumerate}

\textless{}/div\textgreater{}

\section{Implementation}

\subsection{The Final Step}

We're going to skip to the end and write our final step and what it is supposed
to be, and then fill in the functions that are necessary to make it happen.

The last stage of our journey is after we have made all \texttt{n} moves, we end
the journey if it is not a solution.

\textasciitilde{}\textasciitilde{}\textasciitilde{}haskell makeNMoves :: Int
-\textgreater{} {[}Plan{]} -\/- 1 isSolution :: Plan -\textgreater{} Bool

findSolutions :: Int -\textgreater{} {[}Plan{]} -\/- 2 findSolutions n = do p
\textless{}- makeNMoves n -\/- 3 guard \$ isSolution p -\/- 4 return p -\/- 5
\textasciitilde{}\textasciitilde{}\textasciitilde{}

\begin{enumerate}
\tightlist
\item
  The type signatures of the helper functions we will be using.
\item
  \texttt{findSolutions\ n} is going to be the all successful plans after
  \texttt{n} moves.
\item
  Let \texttt{p} be a plan after \texttt{n} moves have been added to it. Note
  that \texttt{makeNMoves} is itself a journey -\/-\/- a sub-journey. So
  \texttt{p} is a single plan that has \emph{already gone through} the
  \texttt{makeNMoves} journey. We are continuing that journey.
\item
  End the journey unless \texttt{p} is a solution (all characters are on the
  east side)
\item
  Succeed with \texttt{p} if the journey has not yet ended.
\end{enumerate}

Hm. Sounds good! We're done!

So now we only need to implement \texttt{makeNMoves} and \texttt{isSolution}!

\textless{}div class="note"\textgreater{} \textbf{Welcome to Haskell!}

Haskell is a functional language...but that "do" block sure looks very
imperative to me. What gives?

As explained in
\href{http://blog.jle.im/entry/practical-fun-with-monads-introducing-monadplus}{Part
1}, all do blocks are just syntactical sugar for repeated applications of
\texttt{\textgreater{}\textgreater{}=}:

\textasciitilde{}\textasciitilde{}\textasciitilde{}haskell findSolutions :: Int
-\textgreater{} {[}Plan{]} findSolutions = makeNMoves n
\textgreater{}\textgreater{}= (\textbackslash{}p -\textgreater{} guard
(isSolution p) \textgreater{}\textgreater{} return p)
\textasciitilde{}\textasciitilde{}\textasciitilde{}

And \texttt{\textgreater{}\textgreater{}=} is just the (hopefully) familiar
bind. Again, look at
\href{http://blog.jle.im/entry/practical-fun-with-monads-introducing-monadplus}{Part
1} or
\href{http://adit.io/posts/2013-04-17-functors,_applicatives,_and_monads_in_pictures.html}{adit's}
tutorial for a fuller explanation. \textless{}/div\textgreater{}

\subsection{makeNMoves}

\texttt{makeNMoves} is going to be the main logic of our program. We want it to
be a journey, itself -\/-\/- a journey of a single plan going through \texttt{n}
additions of moves.

That means we want something like:

\textasciitilde{}\textasciitilde{}\textasciitilde{}haskell makeMove :: Plan
-\textgreater{} {[}Plan{]}

startingPlan :: Plan startingPlan = {[}{]}

makeNMoves :: Int -\textgreater{} {[}Plan{]} makeNMoves n = do m1 \textless{}-
makeMove startingPlan m2 \textless{}- makeMove m1 m3 \textless{}- makeMove m2
-\/- ... (n times) mn \textless{}- makeMove mx return mn
\textasciitilde{}\textasciitilde{}\textasciitilde{}

Which says "The journey of \texttt{makeNMoves} is repeatedly making a move
\texttt{n} times."

Of course we have seen that particular type of \texttt{do} block before, it is
simply:

\textasciitilde{}\textasciitilde{}\textasciitilde{}haskell makeNMoves :: Int
-\textgreater{} {[}Plan{]} makeNMoves n = makeMove startingPlan
\textgreater{}\textgreater{}= makeMove \textgreater{}\textgreater{}= makeMove
\textgreater{}\textgreater{}= makeMove -\/- ... \textgreater{}\textgreater{}=
makeMove -\/- (n times) \textasciitilde{}\textasciitilde{}\textasciitilde{}

Luckily there is a function in the standard library that allows us to repeatedly
apply a function \texttt{n} times:
\texttt{iterate\ ::\ (a\ -\textgreater{}\ a)\ -\textgreater{}\ a\ -\textgreater{}\ {[}a{]}}.
\texttt{iterate\ f\ x} takes a function \texttt{f\ ::\ a\ -\textgreater{}\ a}
and repeatedly applies it to a starting value \texttt{x\ ::\ a} and yields the
results as a list:

\textasciitilde{}\textasciitilde{}\textasciitilde{}haskell iterate f x = {[} x,
f x, f (f x), f (f (f x)) ... {]}
\textasciitilde{}\textasciitilde{}\textasciitilde{}

And so to get the \texttt{n}th application, we use \texttt{iterate\ f\ x\ !!\ n}
(\texttt{!!} being the indexing function, getting the \texttt{n}th element of
the list)

So now we can define \texttt{makeNMoves}:

\textasciitilde{}\textasciitilde{}\textasciitilde{}haskell makeNMoves :: Int
-\textgreater{} {[}Plan{]} makeNMoves n = iterate (\textgreater{}\textgreater{}=
makeMove) (return startingPlan) !! n
\textasciitilde{}\textasciitilde{}\textasciitilde{}

We say "apply \texttt{(\textgreater{}\textgreater{}=\ makeMove)} \texttt{n}
times, starting the single starting plan".

\textless{}div class="note"\textgreater{} \textbf{Welcome to Haskell!}

Remember that \texttt{return\ x\ \textgreater{}\textgreater{}=\ f} is the same
as \texttt{f\ x}. You can see this here:

\textasciitilde{}\textasciitilde{}\textasciitilde{}haskell foo1 = do y
\textless{}- return x f y

-\/- identical foo2 = f x \textasciitilde{}\textasciitilde{}\textasciitilde{}

Where \texttt{return\ x} says "succeed with the value \texttt{x}", and
\texttt{y\ \textless{}-} says "set \texttt{y} to the value of that success". Of
course, \texttt{y} is just going to be \texttt{x}, because we had just said
"succeed with the value of \texttt{x}. That means that \texttt{f\ y} is the same
as \texttt{f\ x}. \textless{}/div\textgreater{}

Even though the syntax is not the cleanest, it is important to remember here
that what we are doing is simply defining the journey \texttt{makeNMoves} as the
result of taking \texttt{n} \texttt{makeMove} journeys one after the other. The
same as that first do block.

\subsection{isSolution}

Let's define our helper function
\texttt{isSolution\ ::\ Plan\ -\textgreater{}\ Bool}. Basically, we want to
check if the positions of all of the characters are \texttt{East}.

First, we need a way to get the position of a farmer/animal after a given plan
has been executed.

\subsubsection{positionOf}

Our function
\texttt{positionOf\ ::\ Plan\ -\textgreater{}\ Character\ -\textgreater{}\ Position}
is going to take a \texttt{Plan} and a \texttt{Character}, and report what side
of the river the character is on.

Because every single move swaps the position of the farmer, the final position
of the farmer depends only on the even-/odd-ness of the number of total moves.
If it is even, then the farmer is on the west bank still (consider 0 moves, two
moves, etc.). If it is odd, then the farmer is on the east bank.

\textasciitilde{}\textasciitilde{}\textasciitilde{}haskell positionOf :: Plan
-\textgreater{} Character -\textgreater{} Position positionOf p c = case c of
Farmer -\textgreater{} positionFromCount \$ length p \_ -\textgreater{}
undefined where positionFromCount n \textbar{} even n = West \textbar{}
othherwise = East \textasciitilde{}\textasciitilde{}\textasciitilde{}

Now, what if we want to know about non-farmers?

Instead of finding the total number of moves, we only need to find the number of
moves involving that given animal.

Let's first filter the Plan \texttt{p} by moves involving the character
\texttt{c}:

\textasciitilde{}\textasciitilde{}\textasciitilde{}haskell filter (== MoveThe c)
p \textasciitilde{}\textasciitilde{}\textasciitilde{}

This will return a new Plan, but with only the moves involving the character
\texttt{c}. We can then use the length of \emph{that}.

\textless{}div class="note"\textgreater{} \textbf{Welcome to Haskell!}

\texttt{filter\ ::\ (a\ -\textgreater{}\ Bool)\ -\textgreater{}\ {[}a{]}\ -\textgreater{}\ {[}a{]}}
is a common function that takes a predicate \texttt{a\ -\textgreater{}\ Bool}
and a list, and returns a new list with only the items for which the predicate
returns true.

\texttt{(==\ MoveThe\ c)} is a function that returns true if the move is equal
to \texttt{MoveThe\ c}. \textless{}/div\textgreater{}

Putting it all together:

\textasciitilde{}\textasciitilde{}\textasciitilde{}haskell positionOf :: Plan
-\textgreater{} Character -\textgreater{} Position positionOf p c = case c of
Farmer -\textgreater{} positionFromCount . length \$ p c -\textgreater{}
positionFromCount . length \$ filter (== MoveThe c) p where positionFromCount n
\textbar{} even n = West \textbar{} othherwise = East
\textasciitilde{}\textasciitilde{}\textasciitilde{}

\textless{}div class="note"\textgreater{} \textbf{Welcome to Haskell!}

What is \texttt{positionFromCount\ .\ length\ \$\ p}?

In Haskell, the \texttt{(.)} operator represents function composition.
\texttt{(f\ .\ g)\ x} is equivalent to \texttt{f\ (g\ x)}. "Apply \texttt{g}
first, then apply \texttt{f}".

Also recall that you can think of \texttt{\$} as adding an implicit parentheses
around both sides of it. You visualize it like the spine of a butterfly -\/-\/-
the "wings" are wrapped parentheses around either side of it. In that sense,
\texttt{f\ .\ g\ \$\ x} is the same as \texttt{(f\ .\ g)\ (x)} (A rather
lopsided butterfly).

So, altogether, \texttt{positionFromCount\ .\ length\ \$\ p} is the same as
\texttt{(positionFromCount\ .\ length)\ p}, which says "first, find the length
of \texttt{p}, then turn that length into a position."

In the same way,
\texttt{positionFromCount\ .\ length\ \$\ filter\ (==\ MoveThe\ c)\ p} is
\texttt{(positionFromCount\ .\ length)\ (filter\ (==\ MoveThe\ c)\ p)} -\/-\/-
find the length of the filtered list, then turn that length into a position. We
use \texttt{\$} mostly because we don't like writing parentheses everywhere when
we don't have to. \textless{}/div\textgreater{}

Does this actually work? Let's try out some examples.

\textasciitilde{}\textasciitilde{}\textasciitilde{}haskell ghci\textgreater{}
let p = {[}MoveThe Goat, MoveThe Farmer, MoveThe Wolf, MoveThe Goat{]}
ghci\textgreater{} positionOf p Goat West ghci\textgreater{} positionOf p Wolf
East ghci\textgreater{} positionOf p Farmer West
\textasciitilde{}\textasciitilde{}\textasciitilde{}

It works! By the way, as an unrelated note, isn't it cool that our \texttt{Plan}
literal reads a lot like English? MoveThe Goat, MoveThe Farmer, MoveThe Wolf...

\subsubsection{Checking the Path}

Now we have to check that the plan is a solution.

Simple -\/-\/- that means that all \texttt{Characters} are on the east side.

We can check this manually:

\textasciitilde{}\textasciitilde{}\textasciitilde{}haskell isSolution :: Plan
-\textgreater{} Bool isSolution p = positionOf p Farmer == East \&\& positionOf
p Wolf == East \&\& positionOf p Goat == East \&\& positionOf p Cabbage == East
\textasciitilde{}\textasciitilde{}\textasciitilde{}

Hm. Rather ugly.

We see a common pattern that we need \texttt{positionOf\ p\ c} for all
\texttt{c}s. That looks like a map!

We also compare all of them to \texttt{East}. That sounds like a job for the
prelude function
\texttt{all\ ::\ (a\ -\textgreater{}\ Bool)\ -\textgreater{}\ {[}a{]}\ -\textgreater{}\ Bool},
which takes a predicate and a list and returns true if all items in the list
satisfy the predicate.

Let's piece it all together:

\textasciitilde{}\textasciitilde{}\textasciitilde{}haskell isSolution p = all
(== East) positions where positions = map (positionOf p) {[}Farmer ..{]}
\textasciitilde{}\textasciitilde{}\textasciitilde{}

\textless{}div class="note"\textgreater{} \textbf{Welcome to Haskell!}

\texttt{map} is probably the most ubiquitous concept in functional programming
-\/-\/- it takes a function and a list and returns a new list with the function
applied to every item.

For example, \texttt{map\ f\ {[}x,y,z{]}} = \texttt{{[}f\ x,\ f\ y,\ f\ z{]}}.
If we wanted to find the lengths of a list of strings, we'd do:

\textasciitilde{}\textasciitilde{}\textasciitilde{}haskell map length
{[}"alice","bob"{]} = {[}length "alice", length "bob"{]} = {[}5,3{]}
\textasciitilde{}\textasciitilde{}\textasciitilde{}

So in our case:

\textasciitilde{}\textasciitilde{}\textasciitilde{}haskell map (positionOf p)
{[}Farmer, Wolf, Goat, Cabbage{]} = {[} positionOf p Farmer -\/- Position of the
farmer , positionOf p Wolf -\/- Position of the wolf , positionOf p Goat -\/-
Position of the goat , positionOf p Cabbage -\/- Position of the cabbage {]}
\textasciitilde{}\textasciitilde{}\textasciitilde{}
\textless{}/div\textgreater{}

We use \texttt{{[}Farmer\ ..{]}} as shorthand for
\texttt{{[}Farmer,\ Wolf,\ Goat,\ Cabbage{]}} -\/-\/- this is because
\texttt{Character} is an Enum, so it can be enumerated using enumeration syntax.
It basically means "\texttt{Farmer}, etc."

\subsection{makeMove}

So let's get down to the meat of our journey. How do we make a move?

\textasciitilde{}\textasciitilde{}\textasciitilde{}haskell makeMove :: Plan
-\textgreater{} {[}Plan{]} \textasciitilde{}\textasciitilde{}\textasciitilde{}

\texttt{makeMove} will be a function that takes a plan and returns all the
successful ways you can add a move to that plan. It takes a plan and takes it
through a journey of adding a move, and returns the results of all of the
successful ways it can fulfill this journey. This is similar to our old
\texttt{halveOrDouble\ ::\ Int\ -\textgreater{}\ {[}Int{]}}, which takes an int
and returns the successful paths our int could have taken (it could have been
halved...or doubled).

What does a plan have to "go through" in its journey in adding a move?

\begin{enumerate}
\tightlist
\item
  First, we get the move we want to add. We could pick a
  \texttt{MoveThe\ Farmer}, a \texttt{MoveThe\ Goat}, or anything!
\item
  Then, we fail/end the journey if we pick a move that isn't legal. For example,
  we can't move the goat if the farmer is not on the same side of the river that
  the goat is on.
\item
  Now, we add that move that we got to the plan.
\item
  Then, we fail/end the journey if that new plan is "unsafe" -\/-\/- if it
  leaves either the Wolf and Goat alone on a riverbank or the Goat and Cabbage.
\item
  At the end of it all, we succeed with the new plan.
\end{enumerate}

Let's try this out:

\textasciitilde{}\textasciitilde{}\textasciitilde{}haskell moveLegal :: Plan
-\textgreater{} Move -\textgreater{} Bool -\/- 1 safePlan :: Plan
-\textgreater{} Bool

makeMove :: Plan -\textgreater{} {[}Plan{]} makeMove p = do next \textless{}-
MoveThe \textless{}\$\textgreater{} {[}Farmer .. Cabbage{]} -\/- 2 guard \$
moveLegal p next -\/- 3 let p' = p ++ {[}next{]} -\/- 4 guard \$ safePlan p'
-\/- 5 return p' -\/- 6 \textasciitilde{}\textasciitilde{}\textasciitilde{}

\begin{enumerate}
\tightlist
\item
  Here are the types of the helper functions we will be using.
\item
  In this context, \texttt{MoveThe\ \textless{}\$\textgreater{}} means to apply
  \texttt{MoveThe} to whatever we choose out of
  \texttt{{[}Farmer\ ..\ Cabbage{]}}. Kind of an "intercept it on the way out,
  and turn it into a Move". So \texttt{next} is \texttt{MoveThe\ Farmer} or
  \texttt{MoveThe\ Wolf}, etc.; \texttt{next} is \emph{one} of those. For every
  journey, we pick \emph{one} of the possible moves.
\item
  We insta-fail if the move is not legal with the given plan. By this, we mean
  that we can't possibly move an animal unless the farmer is on the same side as
  the animal.
\item
  Let's let \texttt{p\textquotesingle{}} be \texttt{next} appended to the
  original plan \texttt{p}.
\item
  We insta-fail unless the new plan is safe.
\item
  If we haven't failed yet, then we succeed with the new plan as the result.
\end{enumerate}

\textless{}div class="note"\textgreater{} \textbf{Welcome to Haskell!}

Okay, so I was slightly hand-wavey with \texttt{\textless{}\$\textgreater{}}.
But it is true that something like:

\textasciitilde{}\textasciitilde{}\textasciitilde{}haskell x \textless{}- (*2)
\textless{}\$\textgreater{} Just 3
\textasciitilde{}\textasciitilde{}\textasciitilde{}

will put 6 (\texttt{3\ *\ 2}) into \texttt{x} -\/-\/- it'll take out the 3 and
then apply \texttt{(*2)} to it before storing it in \texttt{x}.

What's going on under the hood is actually less magical.
\texttt{\textless{}\$\textgreater{}} basically says "apply inside". It is like
\texttt{\$}, but "inside". Remember how we can do:

\textasciitilde{}\textasciitilde{}\textasciitilde{}haskell ghci\textgreater{}
(*2) \$ 3 6 \textasciitilde{}\textasciitilde{}\textasciitilde{}

to apply \texttt{(*2)} to 3? We can then also do:

\textasciitilde{}\textasciitilde{}\textasciitilde{}haskell ghci\textgreater{}
(\emph{2) \$ 3 6 ghci\textgreater{} (}2) \textless{}\$\textgreater{} Just 3 Just
6 ghci\textgreater{} (*2) \textless{}\$\textgreater{} {[}3{]} {[}6{]}
\textasciitilde{}\textasciitilde{}\textasciitilde{}

Now, if we think of a List like a list of possible successes, then applying a
function "inside" means applying the function to all of the possible successes:

\textasciitilde{}\textasciitilde{}\textasciitilde{}haskell ghci\textgreater{}
(*2) \textless{}\$\textgreater{} {[}3,4,5{]} {[}6,8,10{]}

ghci\textgreater{} MoveThe \$ Farmer MoveThe Farmer ghci\textgreater{} MoveThe
\textless{}\$\textgreater{} {[}Farmer, Wolf, Goat, Cabbage{]} {[}MoveThe Farmer,
MoveThe Wolf, MoveThe Goat, MoveThe Cabbage{]}
\textasciitilde{}\textasciitilde{}\textasciitilde{}

So when I say

\textasciitilde{}\textasciitilde{}\textasciitilde{}haskell next \textless{}-
MoveThe \textless{}\$\textgreater{} {[}Farmer, Wolf, Goat, Cabbage{]}
\textasciitilde{}\textasciitilde{}\textasciitilde{}

I really mean

\textasciitilde{}\textasciitilde{}\textasciitilde{}haskell next \textless{}-
{[}MoveThe Farmer, MoveThe Wolf, MoveThe Goat, MoveThe Cabbage{]}
\textasciitilde{}\textasciitilde{}\textasciitilde{}

But still, it sometimes is cool to think of it as "Get the item inside, and then
apply this function to it before you bind it to your variable", if only for
funsies. \textless{}/div\textgreater{}

\subsubsection{Thought experiment}

So let's say our plan is, currently,
\texttt{{[}MoveThe\ Goat,\ MoveThe\ Farmer,\ MoveThe\ Wolf{]}}. At the end of it
all, our goat, wolf, and farmer are on the east bank, and the cabbage is on the
west bank.

What happens on a typical journey of \texttt{makeMove}?

\begin{enumerate}
\tightlist
\item
  First, we pick something to move. Let's say \texttt{next} is
  \texttt{MoveThe\ Farmer}.
\item
  This move is legal (moving the farmer is always legal).
\item
  Our new plan is
  \texttt{{[}MoveThe\ Goat,\ MoveThe\ Farmer,\ MoveThe\ Wolf,\ MoveThe\ Farmer{]}}
\item
  This plan is not safe. If we move the farmer, the goat and the wolf will be
  alone, and that is bad news for the goat. We fail at the second guard.
\item
  We don't return anything, because this journey is a total and utter failure.
\end{enumerate}

Huh. How unfortunate. Let's try again with another pick for \texttt{next}:

\begin{enumerate}
\tightlist
\item
  Let's pick \texttt{MoveThe\ Cabbage} this time for \texttt{next}.
\item
  This move isn't even legal! The cabbage is on the west bank but the farmer is
  on the east. Failure!
\end{enumerate}

Well, that's kind of depressing. Let's try another:

\begin{enumerate}
\tightlist
\item
  We pick \texttt{MoveThe\ Goat} for \texttt{next}.
\item
  This move is legal; both the goat and the farmer are on the east bank.
\item
  Our new plan is
  \texttt{{[}MoveThe\ Goat,\ MoveThe\ Farmer,\ MoveThe\ Wolf,\ MoveThe\ Goat{]}}.
\item
  This plan is indeed safe. The goat and the cabbage are now on the west bank,
  but so is the farmer.
\item
  Because all is well, we return our new plan!
\end{enumerate}

Hooray!

As an exercise, see how the journey fares if we had picked
\texttt{MoveThe\ Wolf} for \texttt{next}.

Anyways, at the end of it all, \texttt{makeMove} will return all new plans from
the successful journeys. So it won't be returning the plans with
\texttt{MoveThe\ Farmer} and \texttt{MoveThe\ Cabbage} added to it, but will
likely be retuning the plans with \texttt{MoveThe\ Goat} and
\texttt{MoveThe\ Wolf} added to it. And it'll return those two together in a
List structure.

We're almost there! Now to just define our helper predicates \texttt{moveLegal}
and \texttt{safePlan}.

\subsubsection{moveLegal}

What makes a move legal? Well, the farmer has to be on the same side as whatever
is being moved.

We can re-use our
\texttt{positionOf\ ::\ Plan\ -\textgreater{}\ Character\ -\textgreater{}\ Position}
function here.

\textasciitilde{}\textasciitilde{}\textasciitilde{}haskell moveLegal :: Plan
-\textgreater{} Move -\textgreater{} Bool moveLegal p (MoveThe Farmer) = True
moveLegal p (MoveThe c) = positionOf p c == positionOf p Farmer
\textasciitilde{}\textasciitilde{}\textasciitilde{}

\subsubsection{safePlan}

One last piece. How can we tell if a plan is safe or not?

The plan is safe if nothing can eat anything else. That means if the wolf and
goat or goat and cabbage sit on the same bank, so too must the farmer. Some
boolean arithmetic will show that this is the same as if either the farmer is on
the same side as the goat or the goat and cabbage are both "safe" (not on the
side of their predators).

\textasciitilde{}\textasciitilde{}\textasciitilde{}haskell safePlan :: Plan
-\textgreater{} Bool safePlan p = goatPos == farmerPos \textbar{}\textbar{}
safeGoat \&\& safeCabbage where goatPos = positionOf p Goat farmerPos =
positionOf p Farmer safeGoat = goatPos /= positionOf p Wolf safeCabbage =
positionOf p Cabbage /= goatPos
\textasciitilde{}\textasciitilde{}\textasciitilde{}

And...that's it! We finished!

\subsubsection{Exercise}

Notice that sometimes we are going to make "redundant moves". For example, we
could move the farmer or goat twice in a row. How can we add another guard to
check if the move isn't redundant? That is, that the move we are adding isn't
identical to the last move of the plan?

The implementation is in the final solution later on, but think about how you
would do it and compare the final solution to yours!

\section{Wrapping Up}

The final code for this project is available
\href{https://github.com/mstksg/inCode/blob/master/code-samples/monad-plus/WolfGoatCabbage.hs}{on
Github} so you can follow along yourself. You can also
\href{https://www.fpcomplete.com/user/jle/wolf-goat-cabbage}{load it
interactively online} on FPComplete, a great online Haskell IDE where you can
test your code right there in the browser.

So...let's test it!

\subsection{Tests}

First, let's load it up on ghci:

\textasciitilde{}\textasciitilde{}\textasciitilde{}haskell ghci\textgreater{} :l
WolfGoatCabbage.hs Ok, modules loaded: Main.
\textasciitilde{}\textasciitilde{}\textasciitilde{}

Let's try a few plan lengths and see when we get one that has a valid solution:

\textasciitilde{}\textasciitilde{}\textasciitilde{}haskell ghci\textgreater{}
findSolutions 5 {[}{]} ghci\textgreater{} findSolutions 6 {[}{]}
ghci\textgreater{} findSolutions 7 {[}{[}G,F,W,G,C,F,G{]},{[}G,F,C,G,W,F,G{]}{]}
\textasciitilde{}\textasciitilde{}\textasciitilde{}

Great, we have two solutions of length 7. If we try them out, it seems like they
both work! Notice that, interestingly enough, the two solutions are their own
reverses. This makes sense, because any solution of getting from the west bank
to the east bank must also be, backwards, a valid solution of getting from the
east bank to the west bank.

It turns out that the solutions of length 9 and 11 are both identical to the
solutions for length 7, just with some redundant moves thrown in (moving the
farmer twice in a row, moving the goat twice in a row, etc.). Also, note that
all possible solutions are of odd lengths, because for even lengths, the farmer
ends up on the west bank.

If we add the filter on redundant moves mentioned earlier, the next valid
solutions with no direct redundancies come at length 13, and then at 19:

\textasciitilde{}\textasciitilde{}\textasciitilde{}haskell ghci\textgreater{}
findSolutions 13 {[}{[}G,F,W,G,C,W,G,C,W,G,C,F,G{]}
,{[}G,F,C,G,W,C,G,W,C,G,W,F,G{]}{]} ghci\textgreater{} findSolutions 19
{[}{[}G,F,W,G,C,W,G,C,W,G,C,W,G,C,W,G,C,F,G{]}
,{[}G,F,C,G,W,C,G,W,C,G,W,C,G,W,C,G,W,F,G{]}{]}
\textasciitilde{}\textasciitilde{}\textasciitilde{}

Again note that both of these solutions come in pairs, with one being the
reverse of the other. Also curious is the fact that they are actually identical
to the length 7 solutions, just with cycles of \texttt{W,G,C} (or
\texttt{C,G,W}) over and over again in the middle.

\subsection{Reflections}

We have solved the classic logic puzzle without using any control flow other
than the List's MonadPlus instance. The solution isn't necessarily optimal, but
it is interesting that we can model something like this simply as saying: "Here
is the description of a journey. What journeys following this description
succeed?"

With the List MonadPlus, you can solve any problem that can be described as the
result of a nondeterministic journey with choices and pitfalls along the way.

In this particular puzzle, you could have done something similar from the start
using only maps and filters. However, sometimes it is more useful or more
insightful to, instead of using maps and filters, use abstractions that help you
frame the problem in a more meaningful way.

Hopefully as a result of this three part series and through playing around with
the source code, you can appreciate the wonders of Succeed/Fail and MonadPlus!

\subsection{The future}

Where to go from here? You might want to take a look at the
\href{http://hackage.haskell.org/package/base/docs/Control-Applicative.html}{Alternative}
typeclass/design pattern, which also deals with the concept of success/failure
-\/-\/- just not with their consecutive chaining, like MonadPlus. It deals with
their parallel choices, actually, as the name implies. This functionality is
redundantly implemented in MonadPlus in Haskell today (2013), and the
parallel-choice operator \texttt{\textless{}\textbar{}\textgreater{}} for
Alternative is \texttt{mplus} for MonadPlus. I might write something on the
matter some day. Anyways, learning about Alternative will help you see more
about the usefulness of the success/fail design pattern, and it might help you
gain the perspective which much of the early Haskell implementors apparently
lacked: not everything is a monad!

\end{document}
