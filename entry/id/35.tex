\documentclass[]{article}
\usepackage{lmodern}
\usepackage{amssymb,amsmath}
\usepackage{ifxetex,ifluatex}
\usepackage{fixltx2e} % provides \textsubscript
\ifnum 0\ifxetex 1\fi\ifluatex 1\fi=0 % if pdftex
  \usepackage[T1]{fontenc}
  \usepackage[utf8]{inputenc}
\else % if luatex or xelatex
  \ifxetex
    \usepackage{mathspec}
    \usepackage{xltxtra,xunicode}
  \else
    \usepackage{fontspec}
  \fi
  \defaultfontfeatures{Mapping=tex-text,Scale=MatchLowercase}
  \newcommand{\euro}{€}
\fi
% use upquote if available, for straight quotes in verbatim environments
\IfFileExists{upquote.sty}{\usepackage{upquote}}{}
% use microtype if available
\IfFileExists{microtype.sty}{\usepackage{microtype}}{}
\usepackage[margin=1in]{geometry}
\ifxetex
  \usepackage[setpagesize=false, % page size defined by xetex
              unicode=false, % unicode breaks when used with xetex
              xetex]{hyperref}
\else
  \usepackage[unicode=true]{hyperref}
\fi
\hypersetup{breaklinks=true,
            bookmarks=true,
            pdfauthor={},
            pdftitle={},
            colorlinks=true,
            citecolor=blue,
            urlcolor=blue,
            linkcolor=magenta,
            pdfborder={0 0 0}}
\urlstyle{same}  % don't use monospace font for urls
% Make links footnotes instead of hotlinks:
\renewcommand{\href}[2]{#2\footnote{\url{#1}}}
\setlength{\parindent}{0pt}
\setlength{\parskip}{6pt plus 2pt minus 1pt}
\setlength{\emergencystretch}{3em}  % prevent overfull lines
\setcounter{secnumdepth}{0}


\begin{document}

\% Auto: Building a Declarative Chatbot with Implicit Serialization \% Justin Le
\% March 25, 2015

\emph{Originally posted on
\textbf{\href{https://blog.jle.im/entry/auto-building-a-declarative-chatbot-with-implicit-serialization.html}{in
Code}}.}

Today we're going to continue along with the
\href{http://blog.jle.im/entries/series/+all-about-auto}{All About Auto}
introduction series and look at building a declarative chatbot using the
denotational components from the
\href{http://hackage.haskell.org/package/auto}{auto} library that is modular and
has implicit serialization. Most importantly, we'll look at the "design
process", and principles of architecture that you can apply to your own
projects.

This post assumes \emph{some} concepts from the
\href{https://github.com/mstksg/auto/blob/master/tutorial/tutorial.md}{tutorial},
or at least my \href{http://blog.jle.im/entry/introducing-the-auto-library}{last
post} or the
\href{https://github.com/mstksg/auto/blob/master/README.md}{README}. If some of
these ideas seem completely new, than looking through the
\href{https://github.com/mstksg/auto/blob/master/tutorial/tutorial.md}{tutorial}
or the \href{http://hackage.haskell.org/package/auto}{docs} might refresh your
mind...feel free to also leave a comment, stop by \emph{\#haskell-auto} on
freenode where I go by \emph{jle`}, or \href{https://twitter.com/mstk}{tweet me}

All of the code in this tutorial can be
\href{https://github.com/mstksg/inCode/tree/master/code-samples/auto/chatbot.hs}{downloaded
and run} using \texttt{runghc} (with the appropriate dependencies installed).
Feel free to play along!

\section{Overall Layout}

\emph{auto} is a library that at the highest level gives you a stream
transformer on streams of values. Transform a stream of input values to a stream
of output values. So when we approach a chat bot, we have to think -\/-\/- what
are the inputs, and what are the outputs?

The choice should be pretty straightforward -\/- our input stream is a stream of
input messages from the irc server, and our output stream is a stream of
messages to send to the server. In haskell we like types, so let's make some
types.

\textasciitilde{}\textasciitilde{}\textasciitilde{}haskell -\/- first, our
imports -\/- source:
https://github.com/mstksg/inCode/tree/master/code-samples/auto/chatbot.hs\#L25-43
import Control.Auto import Control.Auto.Blip import Control.Auto.Collection
(mux) import Control.Auto.Run (runOnChanM) import Control.Auto.Serialize
(serializing') import Control.Auto.Switch (resetOn) import Control.Concurrent
(Chan, newChan, writeChan, forkIO, threadDelay) import Control.Monad (void,
forever) import Control.Monad.IO.Class import Data.Foldable (forM\_) import
Data.Map (Map) import Data.Serialize import Data.Text hiding (words, unwords,
map) import Data.Text.Encoding import Data.Text.Encoding.Error import Data.Time
import Network.SimpleIRC import Prelude hiding ((.), id) -\/- we use (.) and id
from \texttt{Control.Category} import qualified Data.Map as M
\textasciitilde{}\textasciitilde{}\textasciitilde{}

\textasciitilde{}\textasciitilde{}\textasciitilde{}haskell -\/- source:
https://github.com/mstksg/inCode/tree/master/code-samples/auto/chatbot.hs\#L51-66
type Nick = String type Channel = String type Message = String

data InMessage = InMessage \{ \emph{inMessageNick :: Nick , }inMessageBody ::
Message , \emph{inMessageSource :: Channel , }inMessageTime :: UTCTime \}
deriving Show

newtype OutMessages = OutMessages (Map Channel {[}Message{]}) deriving Show

instance Monoid OutMessages where mempty = OutMessages M.empty mappend
(OutMessages m1) (OutMessages m2) = OutMessages (M.unionWith (++) m1 m2)
\textasciitilde{}\textasciitilde{}\textasciitilde{}

We make some type aliases to make things a bit clearer. Our inputs are going to
be a data type/"struct" with a nick, a body, a source, and a time. Our outputs
are going to be a \texttt{Data.Map.Map} from \emph{containers} associating
channels with messages to send. I'm just adding here a \texttt{Monoid} instance
in case we want to combine \texttt{OutMessages} maps.

The type for a chat bot over a monad \texttt{m} would then be:

\textasciitilde{}\textasciitilde{}\textasciitilde{}haskell -\/- source:
https://github.com/mstksg/inCode/tree/master/code-samples/auto/chatbot.hs\#L68-68
type ChatBot m = Auto m InMessage OutMessages
\textasciitilde{}\textasciitilde{}\textasciitilde{}

A \texttt{ChatBot} takes a stream of \texttt{InMessage}s and returns a stream of
\texttt{OutMessages}s...and might have effects in \texttt{m} as it does so.

Note that we get a free instance of \texttt{Monoid} on \texttt{ChatBot\ m}:

\textasciitilde{}\textasciitilde{}\textasciitilde{}haskell mappend :: ChatBot m
-\textgreater{} ChatBot m -\textgreater{} ChatBot m
\textasciitilde{}\textasciitilde{}\textasciitilde{}

That takes two \texttt{ChatBot}s and creates a new \texttt{ChatBot} that forks
the input stream (sends all \texttt{InMessage}s) to both of the original ones,
and \texttt{mappend}s the results. So the new \texttt{ChatBot} will send message
to both original ones and return a "combined" \texttt{OutMessages}.

However, not all modules really have to "care" about the room of the
outputs...they might just always reply directly to the room they received the
message on. So it'll help us to also make another sort of \texttt{Auto}:

\textasciitilde{}\textasciitilde{}\textasciitilde{}haskell -\/- source:
https://github.com/mstksg/inCode/tree/master/code-samples/auto/chatbot.hs\#L69-69
type RoomBot m = Auto m InMessage (Blip {[}Message{]})
\textasciitilde{}\textasciitilde{}\textasciitilde{}

A \texttt{RoomBot} doesn't care where its messages go...it just replies to the
same room it got its input from. It outputs a blip stream of message lists; when
it doesn't want to send messages out, it doesn't emit. When it does, it
\emph{does} emit, with the list of messages.

(Remember, a \emph{blip stream} is just like a normal stream of values, except
it only actually \emph{has} a value every once in a while, when it "emits". A
\texttt{Blip\ Bool} is a stream that sometimes, occasionally emits with a
\texttt{Bool}. We work with them using combinators and \texttt{Auto}s from
\href{http://hackage.haskell.org/package/auto/docs/Control-Auto-Blip.html}{\texttt{Control.Auto.Blip}})

\subsection{Converting}

We can write a quick helper function to convert a \texttt{RoomBot} into a
full-on \texttt{ChatBot}, so we can merge them together with
\texttt{mappend}/\texttt{(\textless{}\textgreater{})}:

\textasciitilde{}\textasciitilde{}\textasciitilde{}haskell -\/- source:
https://github.com/mstksg/inCode/tree/master/code-samples/auto/chatbot.hs\#L72-75
perRoom :: Monad m =\textgreater{} RoomBot m -\textgreater{} ChatBot m perRoom
rb = proc inp@(InMessage \_ \_ src \_) -\textgreater{} do messages \textless{}-
fromBlips {[}{]} . rb -\textless{} inp id -\textless{} OutMessages \$
M.singleton src messages \textasciitilde{}\textasciitilde{}\textasciitilde{}

(This example uses proc notation; see this
\href{https://github.com/mstksg/auto/blob/master/tutorial/tutorial.md\#brief-primer-on-proc-notation}{proc
notation primer} for a quick run-down of the relevant aspects)

We say that \texttt{messages} is just the output of \texttt{rb} fed with the
input, except it "collapses" the blip stream into a normal stream by
substituting in \texttt{{[}{]}} whenever the stream doesn't emit. So
\texttt{messages} is \texttt{{[}{]}} when \texttt{rb} doesn't emit (it doesn't
want to send anything), and \texttt{messages} is
\texttt{{[}message1,\ message2\ ...{]}}, with the emitted contents, when it
\emph{does}.

The "output" will be a singleton map with the source of the input and the
messages to send to that source.

So now if we have a \texttt{RoomBot\ m}, we can convert it up into a
\texttt{ChatBot\ m}, and combine it/merge it with other \texttt{ChatBot\ m}s.

\subsection{The whole deal}

We have enough now then to imagine our entire program architecture:

\begin{itemize}
\tightlist
\item
  Write a bunch of separate modules, as \texttt{ChatBot\ m}s or
  \texttt{RoomBot\ m}s, which ever one is more convenient. The beauty is that we
  can merge them all together in the end with our promoter.
\item
  Combine all of our modules with \texttt{mconcat} -\/-\/- that is, something
  like
  \texttt{chatBot\ =\ mconcat\ {[}module1,\ module2,\ module3,\ module\ 4{]}}.
  And that's it, that's our entire chat bot!
\item
  Having an overall \texttt{chatBot\ ::\ ChatBot\ m}, we can use something like
  \texttt{runOnChan} from \texttt{Control.Auto.Run} to have it exist on a
  concurrent thread and watch a channel for input, and perform an action on
  output.
\item
  Find an out-of-the-box irc library that can trigger adding something to a
  concurrent queue when it receives a message, and where you can send messages
  to rooms.
\end{itemize}

And...that's it. Program logic in our \texttt{ChatBot\ m}s, and handling the
"view"/input with our backend.

\subsubsection{Free Serialization}

Remember that \emph{auto} gives us the ability to serialize and resume our
\texttt{Auto}s for free...so we can at any time save the state of our chat bot
to disk, and resume it when we re-load. We don't have to worry about manually
gathering our state between each \texttt{Auto} and writing serialization code.

There's a "convenience combinator" called \texttt{serializing\textquotesingle{}}
in \texttt{Control.Auto.Serialize} (it's one of many different ones that can do
something like this;
\href{http://hackage.haskell.org/package/auto/docs/Control-Auto-Serialize.html}{check
out the module} to see other ways of varying disciplined-ness!). It'll take any
\texttt{Auto} and turn it into an \texttt{Auto} that "self-serializes" -\/-\/-
when you begin running it, it automatically loads its previous state if it
exists, and as you run it, it automatically maintains an updated "resume state"
on disk.

\textasciitilde{}\textasciitilde{}\textasciitilde{}haskell serializing' ::
MonadIO m =\textgreater{} FilePath -\textgreater{} ChatBot m -\textgreater{}
ChatBot m \textasciitilde{}\textasciitilde{}\textasciitilde{}

Note that
\texttt{serializing\textquotesingle{}\ fp\ ::\ MonadIO\ m\ =\textgreater{}\ ChatBot\ m\ -\textgreater{}\ ChatBot\ m}.
It looks a lot like an "identity-ish" sort of function, right? That's because it
is meant to behave \emph{like} \texttt{id}...the returned \texttt{ChatBot}
behaves identical to the previous one...except it splices in the serializing
action in-between. (We are in \texttt{MonadIO} now, because the \texttt{Auto}
has to access \texttt{IO} in order to serialize itself between steps).

So, instead of

\textasciitilde{}\textasciitilde{}\textasciitilde{}haskell chatBot :: Monad m
=\textgreater{} ChatBot m chatBot = mconcat {[}module1, module2, module3{]}
\textasciitilde{}\textasciitilde{}\textasciitilde{}

We can do:

\textasciitilde{}\textasciitilde{}\textasciitilde{}haskell chatBot :: MonadIO m
=\textgreater{} ChatBot m chatBot = serializing' "state.dat" \$ mconcat
{[}module1, module2, module3{]}
\textasciitilde{}\textasciitilde{}\textasciitilde{}

And now our \texttt{chatBot} will automatically resume itself on program
startup, and keep its state backed up on disk at \texttt{state.dat}. We get this
for free, without doing anything extra in the composition of our modules.

Note that in practice, with a bot you are actively developing, this might not be
the best idea. \texttt{serializing\textquotesingle{}} \emph{analyzes} your
\texttt{Auto}s to determine a serialization and reloading strategy, and applies
that to do its job. However, if you, for example, add a new module to your chat
bot...the serialization strategy will change, and your new bot won't be able to
resume old save files.

One solution at this point is just to serialize individual modules that you do
not see yourself changing...or even just serializing parts of the modules you
don't see yourself changing. Then you can change each portion separately and not
worry about migration issues.

\textasciitilde{}\textasciitilde{}\textasciitilde{}haskell chatBot :: MonadIO m
=\textgreater{} ChatBot m chatBot = mconcat {[} serializing' "m1.dat" module1 ,
module2 , serializing' "m3.dat" module3 {]}
\textasciitilde{}\textasciitilde{}\textasciitilde{}

We're not all-or-nothing now here, either! So, \texttt{module1} gets serialized
and auto-resumed from \texttt{m1.dat}, \texttt{module2} is not serialized at
all, and \texttt{module3} now gets serialized and auto-resumed from
\texttt{m3.dat}.

\section{IRC Backend (the ugly part)}

Before we get started on our actual modules, let's just write out the
backend/interface between our \texttt{ChatBot} and irc to get it out of the way.
This will vary based on what library you use; I'm going to use the
\href{http://hackage.haskell.org/package/simpleirc-0.3.0}{simpleirc-0.3.0}, but
feel free to use any interface/library you want.

\textasciitilde{}\textasciitilde{}\textasciitilde{}haskell -\/- source:
https://github.com/mstksg/inCode/tree/master/code-samples/auto/chatbot.hs\#L45-228
withIrcConf :: IrcConfig -\textgreater{} ChatBot IO -\textgreater{} IO ()
withIrcConf ircconf chatbot = do

\begin{verbatim}
-- chan to receive `InMessage`s
inputChan <- newChan :: IO (Chan InMessage)

-- configuring IRC
let events   = cEvents ircconf ++ [ Privmsg (onMessage inputChan) ]
    ircconf' = ircconf { cEvents = events }

-- connect; simplified for demonstration purposes
Right server <- connect ircconf' True True

-- run `chatbot` on `inputChan`
void . forkIO . void $
    runOnChanM id (processOutput server) inputChan chatbot
\end{verbatim}

where -\/- what to do when \texttt{chatBot} outputs processOutput :: MIrc
-\textgreater{} OutMessages -\textgreater{} IO Bool processOutput server
(OutMessages outs) = do print outs \_ \textless{}- flip M.traverseWithKey outs
\$ \textbackslash{}channel messages -\textgreater{} do let channel' = encodeUtf8
. pack \$ channel forM\_ messages \$ \textbackslash{}message -\textgreater{} do
let message' = encodeUtf8 . pack \$ message sendMsg server channel' message'
return True -\/- "yes, continue on"

\begin{verbatim}
-- what to do when you get a new message
onMessage :: Chan InMessage -> EventFunc
onMessage inputChan = \_ message -> do
  case (mNick message, mOrigin message) of
    (Just nick, Just src) -> do
      time <- getCurrentTime
      writeChan inputChan $ InMessage (unpack (decodeUtf8 nick))
                                      (unpack (decodeUtf8 (mMsg message)))
                                      (unpack (decodeUtf8 src))
                                      time
\end{verbatim}

channels :: {[}Channel{]} channels = {[}"\#testchan1", "\#testchan2"{]}

conf :: IrcConfig conf = (mkDefaultConfig "myserver" "mynick") \{ cChannels =
channels \}

main :: IO () main = do withIrcConf conf chatBot forever (threadDelay
1000000000) \textasciitilde{}\textasciitilde{}\textasciitilde{}

That should be it...don't worry if you don't understand all of it, most of it is
just implementation details from \texttt{simpleirc}. The overall loop is
\texttt{runOnChanM} waits on a separate thread for \texttt{inputChan}...when it
gets input, it runs it through \texttt{ChatBot} and sends the outputs through
\emph{simpleirc}'s interface. Meanwhile, \texttt{onMessage} is triggered
whenever \emph{simpleirc} receives a message, where it prepares an
\texttt{InMessage} and drops it off at \texttt{inputChan}.

\textasciitilde{}\textasciitilde{}\textasciitilde{}haskell runOnChanM :: Monad m
=\textgreater{} (forall c. m c -\textgreater{} IO c) -\/- convert \texttt{m} to
\texttt{IO} -\textgreater{} (b -\textgreater{} IO Bool) -\/- handle output
-\textgreater{} Chan a -\/- chan to await input on -\textgreater{} Auto m a b
-\/- \texttt{Auto} to run -\textgreater{} IO (Auto m a b)
\textasciitilde{}\textasciitilde{}\textasciitilde{}

\texttt{runOnChanM} runs any \texttt{Auto\ m\ a\ b}, as long as there's a way to
convert it to \texttt{Auto\ IO\ a\ b} (we can use a \texttt{ChatBot\ IO}, so we
just put \texttt{id} there). You give it a "handler"
\texttt{b\ -\textgreater{}\ IO\ Bool} that it run whenever it outputs; if the
handler returns \texttt{False}, then the whole thing stops. You give it the
\texttt{Chan\ a} to await for input \texttt{a}s on, and it takes care of the
rest. It blocks until the handler returns \texttt{False}, where it'll return the
"updated" \texttt{Auto\ m\ a\ b} with updated state after running through all of
those inputs.

Phew. With that out of the way, let's get right on to the fun part -\/-\/-
building our chat bot modules.

\section{The Modules}

\subsection{seenBot}

What's a common module? Well, we can write a module that keeps track of the last
time any user was "seen" (sent a message), and then respond when there is a
query.

There are two components here...the part that keeps track of the last seen time,
and the part that responds to queries.

Keeping track of our last seen time sounds like a job that takes in a stream of
\texttt{(Nick,\ UTCTime)} pairs and outputs a stream of
\texttt{Map\ Nick\ UTCTime}, where we could look up the last seen time for a
nick by looking up the nick in the map.

Logically, this is pretty straightforward, and anything other than
\texttt{accum} (which is like \texttt{foldl\textquotesingle{}}) would really be
a bit overkill; every input would just update the output map.

\textasciitilde{}\textasciitilde{}\textasciitilde{}haskell trackSeens :: Monad m
=\textgreater{} Auto m (Nick, UTCTime) (Map Nick UTCTime) trackSeens = accum
(\textbackslash{}mp (nick, time) -\textgreater{} M.insert nick time mp) M.empty
\textasciitilde{}\textasciitilde{}\textasciitilde{}

\texttt{accum} takes the same thing that \texttt{foldl} takes:

\textasciitilde{}\textasciitilde{}\textasciitilde{}haskell foldl :: (b
-\textgreater{} a -\textgreater{} b) -\textgreater{} b -\textgreater{} {[}a{]}
-\textgreater{} b accum :: Monad m =\textgreater{} (b -\textgreater{} a
-\textgreater{} b) -\textgreater{} b -\textgreater{} Auto m a b
\textasciitilde{}\textasciitilde{}\textasciitilde{}

So it basically "folds up" the entire history of inputs, with a starting value.
Every time an input comes, the output is the new folded history of inputs. You
can sort of think of it as it applying the function to any incoming values to an
internal accumulator and updating it at every step.

By the way, because \texttt{trackSeens} is self-serializing, we need a
\texttt{Serialize} instance for \texttt{UTCTime}...just for the sake of
demonstration, let's make one now. Let's also write a \texttt{Serialize}
instance for \texttt{Day} (which represents a date) too, while we're at it.

\textasciitilde{}\textasciitilde{}\textasciitilde{}haskell -\/- source:
https://github.com/mstksg/inCode/tree/master/code-samples/auto/chatbot.hs\#L230-236
instance Serialize UTCTime where get = read \textless{}\$\textgreater{} get -\/-
haha don't do this in real life. put = put . show

instance Serialize Day where get = ModifiedJulianDay \textless{}\$\textgreater{}
get put = put . toModifiedJulianDay
\textasciitilde{}\textasciitilde{}\textasciitilde{}

The next component is just to respond to requests. We want to do something on
some "triggering" input. Every once in a while, some input will come that will
"trigger" some special response. This is a sign that we can use \emph{blip
streams}.

\textasciitilde{}\textasciitilde{}\textasciitilde{}haskell queryBlips :: Auto m
Message (Blip Nick) queryBlips = emitJusts (getRequest . words) where getRequest
("@seen":nick:\emph{) = Just nick getRequest } = Nothing
\textasciitilde{}\textasciitilde{}\textasciitilde{}

\texttt{queryBlips} takes an input stream of strings and turns it into an output
\emph{blip stream} that emits with a \texttt{Nick} whenever the input stream
contains a request in the form of \texttt{"@seen\ {[}nick{]}"}.

With these simple blocks, we can build our \texttt{seenBot}:

\textasciitilde{}\textasciitilde{}\textasciitilde{}haskell -\/- seenBot :: Monad
m =\textgreater{} Auto m InMessage (Blip {[}Message{]}) -\/- source:
https://github.com/mstksg/inCode/tree/master/code-samples/auto/chatbot.hs\#L96-115
seenBot :: Monad m =\textgreater{} RoomBot m seenBot = proc (InMessage nick msg
\_ time) -\textgreater{} do seens \textless{}- trackSeens -\textless{} (nick,
time)

\begin{verbatim}
queryB <- queryBlips -< msg

let respond :: Nick -> [Message]
    respond qry = case M.lookup qry seens of
                    Just t  -> [qry ++ " last seen at " ++ show t ++ "."]
                    Nothing -> ["No record of " ++ qry ++ "."]

id -< respond <$> queryB
\end{verbatim}

where trackSeens :: Monad m =\textgreater{} Auto m (Nick, UTCTime) (Map Nick
UTCTime) trackSeens = accum (\textbackslash{}mp (nick, time) -\textgreater{}
M.insert nick time mp) M.empty queryBlips :: Auto m Message (Blip Nick)
queryBlips = emitJusts (getRequest . words) where getRequest
("@seen":nick:\emph{) = Just nick getRequest } = Nothing
\textasciitilde{}\textasciitilde{}\textasciitilde{}

Here we define \texttt{respond} as a function that takes a \texttt{Nick} and
returns the output \texttt{{[}Message{]}}. We could have also defined it outside
as a helper function
\texttt{respond\ ::\ Map\ Nick\ UTCTime\ -\textgreater{}\ Nick\ -\textgreater{}\ {[}Message{]}}...but
\texttt{seens} is already in scope, so we might as well just do it there.

For our output, we use the \texttt{Functor} instance of blip streams.
\texttt{respond\ \textless{}\$\textgreater{}\ queryB} is a blip stream that
emits whenever \texttt{queryB} emits (so, whenever there is a query input), but
replaces the emitted value with the result of the function on the value. So
whenever \texttt{queryB} emits, this whole thing emits with \texttt{respond}
applied to whatever \texttt{Nick} was emitted -\/-\/- in this case, our
\texttt{{[}Message{]}}.

Short, sweet, simple. In fact, \texttt{trackSeens} and \texttt{queryBlips} are
small enough that their definition could really have been inlined. Breaking them
down just allowed us to look at them individually for this tutorial.

So that's it for that; also, if we wanted \texttt{seenBot} to serialize and
persist across sessions, all we have to do is use:

\textasciitilde{}\textasciitilde{}\textasciitilde{}haskell serializing'
"seenbot.dat" seenBot :: MonadIO m =\textgreater{} RoomBot m
\textasciitilde{}\textasciitilde{}\textasciitilde{}

Neat, right?

If we forsee ourselves adding more features to \texttt{seenBot}, we can
future-proof our \texttt{seenBot} for now by only serializing
\texttt{trackSeens}, meaning replacing that line with:

\textasciitilde{}\textasciitilde{}\textasciitilde{}haskell seens \textless{}-
serializing' "seen.dat" trackSeens -\textless{} (nick, time)
\textasciitilde{}\textasciitilde{}\textasciitilde{}

Remember, \texttt{serializing\textquotesingle{}\ fp} acts as a sort of
"identity", so you can drop it in anywhere and you'd expect it to behave the
same.

\subsection{repBot}

Another common bot is a "reputation bot", which allows users to increment or
decrement another user's reputation scores, and look up a user's total score.

Again there are two components -\/-\/- keeping track of the scores of all of the
users, and responding to requests.

This time though, our "score updates" only happen every once in a while,
triggered by certain words in the message. Again, this pattern calls for a blip
stream:

\textasciitilde{}\textasciitilde{}\textasciitilde{}haskell updateBlips :: Auto m
(Nick, Message) (Blip (Nick, Int)) updateBlips = emitJusts getUpdateCommand
where -\/- updater is the person triggering the update blip getUpdateCommand
(updater, msg) = case words msg of "@addRep":nick:\_ \textbar{} nick /= updater
-\textgreater{} Just (nick, 1) "@subRep":nick:\_ -\textgreater{} Just (nick, -1)
\_ -\textgreater{} Nothing \textasciitilde{}\textasciitilde{}\textasciitilde{}

\texttt{updateBlips} takes in a stream of \texttt{(Nick,\ Message)}, with the
person who is sending the message and their message, and outputs a blip stream
that and emits with a \texttt{(Nick,\ Int)} whenever the message is a command.
The emitted \texttt{(Nick,\ Int)} has the person to adjust, and the amount to
adjust by. Note that we ignore commands where the person is trying to increase
their own reputation because that's just lame.

We probably want to keep track of the scores as a \texttt{Map\ Nick\ Int}, so we
can do that with something like \texttt{accum} again. However, \texttt{accum}
takes a stream of normal values, but we have a \emph{blip stream}, so we can use
\texttt{scanB} instead. \texttt{scanB} is pretty much the same thing, but it
collapses a blip stream into a value stream by holding the "current result" of
the fold.{[}\^{}accumB{]}

\textasciitilde{}\textasciitilde{}\textasciitilde{}haskell trackReps :: Monad m
=\textgreater{} Auto m (Blip (Nick, Int)) (Map Nick Int) trackReps = scanB
(\textbackslash{}mp (nick, change) -\textgreater{} M.insertWith (+) nick change
mp) M.empty \textasciitilde{}\textasciitilde{}\textasciitilde{}

And finally, the "response" portion -\/-\/- we want to be able to respond to
commands and look up the result. We basically had this identical pattern for
\texttt{seenBot}:

\textasciitilde{}\textasciitilde{}\textasciitilde{}haskell queryBlips :: Auto m
Message (Blip Nick) queryBlips = emitJusts (getRequest . words) where getRequest
("@rep":nick:\emph{) = Just nick getRequest } = Nothing
\textasciitilde{}\textasciitilde{}\textasciitilde{}

And...now we can wrap it all together with a nice proc block:

\textasciitilde{}\textasciitilde{}\textasciitilde{}haskell -\/- repBot :: Monad
m =\textgreater{} Auto m InMessage (Blip {[}Message{]}) -\/- source:
https://github.com/mstksg/inCode/tree/master/code-samples/auto/chatbot.hs\#L117-147
repBot :: Monad m =\textgreater{} RoomBot m repBot = proc (InMessage nick msg \_
\_) -\textgreater{} do updateB \textless{}- updateBlips -\textless{} (nick, msg)

\begin{verbatim}
reps    <- trackReps   -< updateB

queryB  <- queryBlips  -< msg

let lookupRep :: Nick -> [Message]
    lookupRep nick = [nick ++ " has a reputation of " ++ show rep ++ "."]
      where
        rep = M.findWithDefault 0 nick reps

id -< lookupRep <$> queryB
\end{verbatim}

where updateBlips :: Auto m (Nick, Message) (Blip (Nick, Int)) updateBlips =
emitJusts getUpdateCommand where -\/- updater is the person triggering the
update blip getUpdateCommand (updater, msg) = case words msg of
"@addRep":nick:\_ \textbar{} nick /= updater -\textgreater{} Just (nick, 1)
"@subRep":nick:\_ -\textgreater{} Just (nick, -1) \_ -\textgreater{} Nothing
trackReps :: Monad m =\textgreater{} Auto m (Blip (Nick, Int)) (Map Nick Int)
trackReps = scanB (\textbackslash{}mp (nick, change) -\textgreater{}
M.insertWith (+) nick change mp) M.empty queryBlips :: Auto m Message (Blip
Nick) queryBlips = emitJusts (getRequest . words) where getRequest
("@rep":nick:\emph{) = Just nick getRequest } = Nothing
\textasciitilde{}\textasciitilde{}\textasciitilde{}

Again note that we take advantage of the \texttt{Functor} instance of blip
streams to create a new blip stream
(\texttt{lookupRep\ \textless{}\$\textgreater{}\ queryB}) that emits whenever
\texttt{queryB} emits, but replaces the value with \texttt{lookupRep} applied to
whatever \texttt{Nick} was in the query blip. We also take advantage that
\texttt{reps} is in scope and define \texttt{lookupRep} right there in the
block.

\subsection{announceBot}

Let's just go over one more module...and I think you'll be able to use your
imagination to think of and implement your own from here.

Let's make an "announceBot", that listens for "announcement" messages from
anyone (even in private messages) and broadcasts them to all of the channels in
the provided list. It rate-limits the announcements, though, so that a user is
only limited to three announcements per day.

We can start with our typical "blip stream that emits on a certain command" to
start off everything:

\textasciitilde{}\textasciitilde{}\textasciitilde{}haskell announceBlips ::
Monad m =\textgreater{} Auto m (Nick, Message) (Blip {[}Message{]})
announceBlips = emitJusts getAnnounces where getAnnounces (nick, msg) = case
words msg of "@ann":ann -\textgreater{} Just {[}nick ++ " says "" ++ unwords ann
++ ""."{]} \_ -\textgreater{} Nothing
\textasciitilde{}\textasciitilde{}\textasciitilde{}

\texttt{announceBlips} takes in a nick-message pair and emits an announcement
\texttt{{[}Message{]}} whenever the incoming message is an announcement command.

Next, we'd like to keep track of how many times a user has made an announcement
today. This is pretty much just \texttt{scanB} again like with \texttt{repBot}:

\textasciitilde{}\textasciitilde{}\textasciitilde{}haskell trackAnns :: Monad m
=\textgreater{} Auto m (Blip Nick) (Map Nick Int) trackAnns = scanB
(\textbackslash{}mp nick -\textgreater{} M.insertWith (+) nick 1 mp) M.empty
\textasciitilde{}\textasciitilde{}\textasciitilde{}

However, we'd like to be able to "reset" this map whenever a new day arrives.
For that, we can use \texttt{resetOn} from
\href{http://hackage.haskell.org/package/auto/docs/Control-Auto-Switch.html}{\texttt{Control.Auto.Switch}},
which takes an \texttt{Auto} and gives it a "reset channel" input blip stream,
that resets the whole thing whenever the blip stream emits:

\textasciitilde{}\textasciitilde{}\textasciitilde{}haskell resetOn :: Monad m
=\textgreater{} Auto m a b -\textgreater{} Auto m (a , Blip c) b resetOn
trackAnns :: Monad m =\textgreater{} Auto m (Blip Nick, Blip c) (Map Nick Int)
\textasciitilde{}\textasciitilde{}\textasciitilde{}

(It doesn't care about the actual value emitted, so we can leave it as a type
variable \texttt{c} conceptually.)

Now the only thing we need is a blip stream that emits whenever there is a new
day. For that, we can use \texttt{onChange} from
\href{http://hackage.haskell.org/package/auto/docs/Control-Auto-Blip.html}{\texttt{Control.Auto.Blip}}:

\textasciitilde{}\textasciitilde{}\textasciitilde{}haskell newDayBlips :: Monad
m =\textgreater{} Auto m Day (Blip Day) newDayBlips = onChange
\textasciitilde{}\textasciitilde{}\textasciitilde{}

\texttt{newDayBlips} takes in a stream of \texttt{Day}s (from
\texttt{Data.Time}) that we get from the \texttt{InMessage} and outputs a blip
stream that emits whenever the day changes. It emits with the new
\texttt{Day}...but we don't really care about the emitted value, we're just
using it to trigger \texttt{resetOn\ trackAnns}.

Finally, let's wrap it all together!

Remember, \texttt{announceBot} is a full on \texttt{ChatBot\ m}, and not a
\texttt{RoomBot\ m} anymore, so it has to say where it wants to send its
messages.

\textasciitilde{}\textasciitilde{}\textasciitilde{}haskell -\/- announceBot ::
Monad m =\textgreater{} {[}Channel{]} -\textgreater{} Auto m InMessage
OutMessages -\/- source:
https://github.com/mstksg/inCode/tree/master/code-samples/auto/chatbot.hs\#L149-183
announceBot :: Monad m =\textgreater{} {[}Channel{]} -\textgreater{} ChatBot m
announceBot chans = proc (InMessage nick msg src time) -\textgreater{} do
announceB \textless{}- announceBlips -\textless{} (nick, msg)

\begin{verbatim}
newDayB   <- newDayBlips       -< utctDay time

annCounts <- resetOn trackAnns -< (nick <$ announceB, newDayB)

let hasFlooded  = M.findWithDefault 0 nick annCounts > 3

    targetChans :: [Channel]
    targetChans | hasFlooded = [src]
                | otherwise  = chans

    outB        :: Blip [Message]
    outB        | hasFlooded = [nick ++ ": No flooding!"] <$ announceB
                | otherwise  = announceB

    outMsgsB    :: Blip OutMessages
    outMsgsB    = (\out -> OutMessages (M.fromList (map (,out) targetChans)))
              <$> outB

fromBlips mempty -< outMsgsB
\end{verbatim}

where announceBlips :: Monad m =\textgreater{} Auto m (Nick, Message) (Blip
{[}Message{]}) announceBlips = emitJusts getAnnounces where getAnnounces (nick,
msg) = case words msg of "@ann":ann -\textgreater{} Just {[}nick ++ " says "" ++
unwords ann ++ ""."{]} \_ -\textgreater{} Nothing newDayBlips :: Monad m
=\textgreater{} Auto m Day (Blip Day) newDayBlips = onChange trackAnns :: Monad
m =\textgreater{} Auto m (Blip Nick) (Map Nick Int) trackAnns = scanB
(\textbackslash{}mp nick -\textgreater{} M.insertWith (+) nick 1 mp) M.empty
\textasciitilde{}\textasciitilde{}\textasciitilde{}

Only slightly more involved, but still pretty readable, right? We find out if
things have flooded, and our target channels will be just the original source if
true (a message as a reprimand); otherwise, all the channels in \texttt{chans}.
If they have flooded, then our \texttt{outB} (blip stream of
\texttt{{[}Message{]}} to send to each room) will just be
\texttt{{[}"No\ flooding!"{]}} if yes, or the actual announcement otherwise.

Finally, our \texttt{Blip\ OutMessages} will be the \texttt{OutMessage} formed
by associating all of the channels in \texttt{targetChans} with the message in
\texttt{outB}...emitting whenever \texttt{outB} emits.

Note here that we use \texttt{(\textless{}\$)} from the \texttt{Functor}
instance of blip streams. \texttt{x\ \textless{}\$\ fooB} is a new blip stream
that emits whenever \texttt{fooB} emits...but instead \emph{replaces the emitted
value}. So for \texttt{4\ \textless{}\$\ fooB}, if \texttt{fooB} emits with
\texttt{"hello"}, \texttt{4\ \textless{}\$\ fooB} emits with \texttt{4}. "Emit
at the same time, but pry out the value and put in your own."

Finally we use \texttt{fromBlips}, which we met before in the definition of
\texttt{perRoom}: the output is the \texttt{OutMessage} in \texttt{outMsgsB}
whenever \texttt{outMsgsB} \emph{does} emit...or it's \texttt{mempty} (the empty
map) when it doesn't.

\section{Wrapping it all up}

We have three nice modules now. Now let's wrap it all together.

\textasciitilde{}\textasciitilde{}\textasciitilde{}haskell -\/- source:
https://github.com/mstksg/inCode/tree/master/code-samples/auto/chatbot.hs\#L83-88
chatBot :: MonadIO m =\textgreater{} ChatBot m chatBot = serializing'
"chatbot.dat" . mconcat \$ {[} perRoom seenBot , perRoom repBot , announceBot
channels {]} \textasciitilde{}\textasciitilde{}\textasciitilde{}

Or, to future-proof, in case we foresee adding new modules:

\textasciitilde{}\textasciitilde{}\textasciitilde{}haskell -\/- source:
https://github.com/mstksg/inCode/tree/master/code-samples/auto/chatbot.hs\#L90-94
chatBot' :: MonadIO m =\textgreater{} ChatBot m chatBot' = mconcat {[} perRoom .
serializing' "seens.dat" \$ seenBot , perRoom . serializing' "reps.dat" \$
repBot , serializing' "anns.dat" \$ announceBot channels {]}
\textasciitilde{}\textasciitilde{}\textasciitilde{}

And...that's it!

Feel free to
\href{https://github.com/mstksg/inCode/tree/master/code-samples/auto/chatbot.hs}{download
and run this all yourself} using \texttt{runghc}! (provided you have the
appropriate libraries installed)

\textless{}div class="note"\textgreater{}

\textbf{Aside}

This is a quick diversion! It's slightly more advanced, so don't worry if you
don't understand it immediately.

Note that the \texttt{perRoom} upgrade has the same \texttt{RoomBot\ m} watch
\emph{all} of the channels and send any replies back to the channel that it just
received from. Every channel is really interacting with the \emph{same}
\texttt{RoomBot} instance, with one shared state. So \texttt{perRoom\ repBot}
keeps track of reputations between rooms -\/-\/- asking for someone's reputation
in one room will be the same as asking for it in another room.

Another way we could "upgrade" a \texttt{RoomBot} is to give each channel its
own little copy, with separate state. We can do this using \texttt{mux}:

\textasciitilde{}\textasciitilde{}\textasciitilde{}haskell -\/- source:
https://github.com/mstksg/inCode/tree/master/code-samples/auto/chatbot.hs\#L78-81
isolatedRooms :: Monad m =\textgreater{} RoomBot m -\textgreater{} ChatBot m
isolatedRooms rb = proc inp@(InMessage \_ \_ src \_) -\textgreater{} do messages
\textless{}- fromBlips {[}{]} . mux (const rb) -\textless{} (src, inp) id
-\textless{} OutMessages \$ M.singleton src messages
\textasciitilde{}\textasciitilde{}\textasciitilde{}

\texttt{mux} is an "\texttt{Auto} multiplexer":

\textasciitilde{}\textasciitilde{}\textasciitilde{}haskell mux :: (k
-\textgreater{} Auto m a b) -\textgreater{} Auto m (k, a) b
\textasciitilde{}\textasciitilde{}\textasciitilde{}

\texttt{mux\ f} associates a separate/different \texttt{Auto}, with its own
isolated state, with every key \texttt{k}. It takes in a stream of key-input
pairs \texttt{(k,\ a)} and feeds the \texttt{a} into the \texttt{Auto} it has
associated with that key \texttt{k}. The function \texttt{f} is what
\texttt{Auto} initialize if the \texttt{k} has not yet been seen before.

So we feed it a \texttt{(Channel,\ InMessage)}, and it feeds in that
\texttt{InMessage} to the \texttt{RoomBot\ m} associated with that
\texttt{Channel}...and the output is the \texttt{Blip\ {[}Message{]}} blip
stream that the \texttt{RoomBot} at that \texttt{Channel} popped out.

Our "auto initialization function" is \texttt{const\ rb}, because no matter what
channel we're in, we always want to initialize with the same \texttt{rb}.

So, for example, if we had \texttt{isolatedRooms\ repBot}, if a message came
from channel \emph{\#foo} saying \texttt{"@rep\ john"}, \emph{only the
\texttt{repBot} associated with \#foo} would get the message, and only that
\texttt{repBot}'s output will be displayed. If here is not yet a \texttt{repBot}
instance associated with \emph{\#foo}, then a new one will be created by calling
\texttt{const\ repBot} on \texttt{"\#foo"}...initializing a new \texttt{repBot}
that only knows about \emph{\#foo} messages.

So now every channel has its own \texttt{repBot}, and maintains its own
independent reputation database.

\textless{}/div\textgreater{}

\section{Fin}

Hopefully, going over this project, you're starting to see some common and
powerful idioms and tools. I hope that a clear picture of how to approach and
finish a program with the \emph{auto} library looks...and how beneficial the
platform and what it offers is to streamlining the development process.

Also, hopefully the "declarative" nature of everything is apparent. Especially
for \emph{proc} blocks...everything just "looks like" a graph of relationships.
This quantity is related to this quantity in this way, this quantity is related
to that in that way, etc. It looks like you're just specifying a graph of
relationships, which is really what the core of \emph{auto} is all about. We
assemble complex relationships by putting together small, simple relationships.

Another interesting thing is that we never really explicitly managed any sort of
state or state type. All of our \texttt{Auto}s handled their state on their own.
We didn't need to make a giant massive aggregate bulky "global state" type...and
we can add new "aspects" of state (new modules) without ever having to change
any data type. The state all manages itself!

And yeah, we didn't just implement "easy" modules/components...these are actual
working components that you might see in real bots, and not just toy ones.

Where can we go from here? Well, you might actually want to maybe write
"subscription" \texttt{Auto}s that are updated every minute or so:

\textasciitilde{}\textasciitilde{}\textasciitilde{}haskell type ChronBot m =
Auto m UTCTime OutMessages \textasciitilde{}\textasciitilde{}\textasciitilde{}

You feed them inputs every minute with the time, and it's allowed to react with
the time and output an \texttt{OutMessages}. You can use this bot to implement
things like rss feed watchers/subscribers, for instance.

So, instead of using an input channel waiting for \texttt{InMessage}, you might
wait for \texttt{Either\ InMessage\ UTCTime}...and drop in \texttt{Left\ im}
whenever you get a message, and \texttt{Right\ time} from a thread that just
waits a minute and repeatedly throws in times.

We can do this with minimal extra work by using the
\texttt{(\textbar{}\textbar{}\textbar{})} combinator from
\texttt{Control.Arrow}:

\textasciitilde{}\textasciitilde{}\textasciitilde{}haskell
(\textbar{}\textbar{}\textbar{}) :: Auto m a c -\textgreater{} Auto m b c
-\textgreater{} Auto m (Either a b ) c (\textbar{}\textbar{}\textbar{}) ::
ChatBot m -\textgreater{} ChronBot m -\textgreater{} Auto m (Either InMessage
UTCTime) OutMessages \textasciitilde{}\textasciitilde{}\textasciitilde{}

And...you get it all for free! No extra work. Now both the \texttt{ChatBot} and
the \texttt{ChronBot} will wait on the input stream, and the \texttt{Left}s will
be fed to the \texttt{ChatBot} and the \texttt{Right}s will be fed to the
\texttt{ChronBot}.

Anyway, this post is long enough. Have fun exploring \emph{auto} on your own;
expect more tutorials soon as I continue the
\href{http://blog.jle.im/entries/series/+all-about-auto}{All About Auto} series,
too. I'm always happy to hear about any project you might be working on! You can
find me on twitter as \href{https://twitter.com/mstk}{mstk}. If you have any
questions or comments/suggestions, feel free to leave a comment down below or
drop by freenode's \emph{\#haskell-auto} or \emph{\#haskell-game}, where I go by
\emph{jle`}! And, as always, happy Haskelling!

\end{document}
