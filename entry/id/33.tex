\documentclass[]{article}
\usepackage{lmodern}
\usepackage{amssymb,amsmath}
\usepackage{ifxetex,ifluatex}
\usepackage{fixltx2e} % provides \textsubscript
\ifnum 0\ifxetex 1\fi\ifluatex 1\fi=0 % if pdftex
  \usepackage[T1]{fontenc}
  \usepackage[utf8]{inputenc}
\else % if luatex or xelatex
  \ifxetex
    \usepackage{mathspec}
    \usepackage{xltxtra,xunicode}
  \else
    \usepackage{fontspec}
  \fi
  \defaultfontfeatures{Mapping=tex-text,Scale=MatchLowercase}
  \newcommand{\euro}{€}
\fi
% use upquote if available, for straight quotes in verbatim environments
\IfFileExists{upquote.sty}{\usepackage{upquote}}{}
% use microtype if available
\IfFileExists{microtype.sty}{\usepackage{microtype}}{}
\usepackage[margin=1in]{geometry}
\usepackage{color}
\usepackage{fancyvrb}
\newcommand{\VerbBar}{|}
\newcommand{\VERB}{\Verb[commandchars=\\\{\}]}
\DefineVerbatimEnvironment{Highlighting}{Verbatim}{commandchars=\\\{\}}
% Add ',fontsize=\small' for more characters per line
\newenvironment{Shaded}{}{}
\newcommand{\AlertTok}[1]{\textcolor[rgb]{1.00,0.00,0.00}{\textbf{#1}}}
\newcommand{\AnnotationTok}[1]{\textcolor[rgb]{0.38,0.63,0.69}{\textbf{\textit{#1}}}}
\newcommand{\AttributeTok}[1]{\textcolor[rgb]{0.49,0.56,0.16}{#1}}
\newcommand{\BaseNTok}[1]{\textcolor[rgb]{0.25,0.63,0.44}{#1}}
\newcommand{\BuiltInTok}[1]{#1}
\newcommand{\CharTok}[1]{\textcolor[rgb]{0.25,0.44,0.63}{#1}}
\newcommand{\CommentTok}[1]{\textcolor[rgb]{0.38,0.63,0.69}{\textit{#1}}}
\newcommand{\CommentVarTok}[1]{\textcolor[rgb]{0.38,0.63,0.69}{\textbf{\textit{#1}}}}
\newcommand{\ConstantTok}[1]{\textcolor[rgb]{0.53,0.00,0.00}{#1}}
\newcommand{\ControlFlowTok}[1]{\textcolor[rgb]{0.00,0.44,0.13}{\textbf{#1}}}
\newcommand{\DataTypeTok}[1]{\textcolor[rgb]{0.56,0.13,0.00}{#1}}
\newcommand{\DecValTok}[1]{\textcolor[rgb]{0.25,0.63,0.44}{#1}}
\newcommand{\DocumentationTok}[1]{\textcolor[rgb]{0.73,0.13,0.13}{\textit{#1}}}
\newcommand{\ErrorTok}[1]{\textcolor[rgb]{1.00,0.00,0.00}{\textbf{#1}}}
\newcommand{\ExtensionTok}[1]{#1}
\newcommand{\FloatTok}[1]{\textcolor[rgb]{0.25,0.63,0.44}{#1}}
\newcommand{\FunctionTok}[1]{\textcolor[rgb]{0.02,0.16,0.49}{#1}}
\newcommand{\ImportTok}[1]{#1}
\newcommand{\InformationTok}[1]{\textcolor[rgb]{0.38,0.63,0.69}{\textbf{\textit{#1}}}}
\newcommand{\KeywordTok}[1]{\textcolor[rgb]{0.00,0.44,0.13}{\textbf{#1}}}
\newcommand{\NormalTok}[1]{#1}
\newcommand{\OperatorTok}[1]{\textcolor[rgb]{0.40,0.40,0.40}{#1}}
\newcommand{\OtherTok}[1]{\textcolor[rgb]{0.00,0.44,0.13}{#1}}
\newcommand{\PreprocessorTok}[1]{\textcolor[rgb]{0.74,0.48,0.00}{#1}}
\newcommand{\RegionMarkerTok}[1]{#1}
\newcommand{\SpecialCharTok}[1]{\textcolor[rgb]{0.25,0.44,0.63}{#1}}
\newcommand{\SpecialStringTok}[1]{\textcolor[rgb]{0.73,0.40,0.53}{#1}}
\newcommand{\StringTok}[1]{\textcolor[rgb]{0.25,0.44,0.63}{#1}}
\newcommand{\VariableTok}[1]{\textcolor[rgb]{0.10,0.09,0.49}{#1}}
\newcommand{\VerbatimStringTok}[1]{\textcolor[rgb]{0.25,0.44,0.63}{#1}}
\newcommand{\WarningTok}[1]{\textcolor[rgb]{0.38,0.63,0.69}{\textbf{\textit{#1}}}}
\ifxetex
  \usepackage[setpagesize=false, % page size defined by xetex
              unicode=false, % unicode breaks when used with xetex
              xetex]{hyperref}
\else
  \usepackage[unicode=true]{hyperref}
\fi
\hypersetup{breaklinks=true,
            bookmarks=true,
            pdfauthor={Justin Le},
            pdftitle={Abstracting over Sequential Random Algorithms with Free},
            colorlinks=true,
            citecolor=blue,
            urlcolor=blue,
            linkcolor=magenta,
            pdfborder={0 0 0}}
\urlstyle{same}  % don't use monospace font for urls
% Make links footnotes instead of hotlinks:
\renewcommand{\href}[2]{#2\footnote{\url{#1}}}
\setlength{\parindent}{0pt}
\setlength{\parskip}{6pt plus 2pt minus 1pt}
\setlength{\emergencystretch}{3em}  % prevent overfull lines
\setcounter{secnumdepth}{0}

\title{Abstracting over Sequential Random Algorithms with Free}
\author{Justin Le}

\begin{document}
\maketitle

\emph{Originally posted on
\textbf{\href{https://blog.jle.im/entry/abstracting-over-sequential-random-algorithms-with-free.html}{in
Code}}.}

It's fair enough to say that I'm a little late to the free monad party, but I
still think their power is greatly either misunderstood or underrated or obscure
in Haskell, and this article will be my attempt to throw more examples of its
usage.

Here we're going to construct a type that can represent sequential computations
using randomness or entropy, and we're going to abstract over our entropy source
--- not by swapping out the pseudorandom generator, but by really abstracting
over what our computation really \emph{is} to the barest essentials. We're going
to abstract over what an sequential random algorithm even is.

The man plain is to first create a type that represents just a value produced
from a random number\ldots{}and then figure out how to chain them.

\begin{Shaded}
\begin{Highlighting}[]
\CommentTok{-- source: https://github.com/mstksg/inCode/tree/master/code-samples/free-random/Rand.hs#L18-L22}

\KeywordTok{data} \DataTypeTok{RandF}\NormalTok{ a }\KeywordTok{where}
    \DataTypeTok{FromRandom}\OtherTok{   ::} \DataTypeTok{Random}\NormalTok{ r }\OtherTok{=>}\NormalTok{           ( r  }\OtherTok{->}\NormalTok{ a) }\OtherTok{->} \DataTypeTok{RandF}\NormalTok{ a}
    \DataTypeTok{FromRandomR}\OtherTok{  ::} \DataTypeTok{Random}\NormalTok{ r }\OtherTok{=>}\NormalTok{ r }\OtherTok{->}\NormalTok{ r }\OtherTok{->}\NormalTok{ ( r  }\OtherTok{->}\NormalTok{ a) }\OtherTok{->} \DataTypeTok{RandF}\NormalTok{ a}
    \DataTypeTok{FromRandoms}\OtherTok{  ::} \DataTypeTok{Random}\NormalTok{ r }\OtherTok{=>}\NormalTok{           ([r] }\OtherTok{->}\NormalTok{ a) }\OtherTok{->} \DataTypeTok{RandF}\NormalTok{ a}
    \DataTypeTok{FromRandomRs}\OtherTok{ ::} \DataTypeTok{Random}\NormalTok{ r }\OtherTok{=>}\NormalTok{ r }\OtherTok{->}\NormalTok{ r }\OtherTok{->}\NormalTok{ ([r] }\OtherTok{->}\NormalTok{ a) }\OtherTok{->} \DataTypeTok{RandF}\NormalTok{ a}
\end{Highlighting}
\end{Shaded}

For those of you unfamiliar with GADT syntax, this is just basically specifying
a type by the type of its contructors; \texttt{RandF} has four constructors ---
one of which, \texttt{FromRandom}, takes a function
\texttt{(r\ -\textgreater{}\ a)}, and returns a \texttt{RandF\ a}\ldots{}as long
as the \texttt{r} is an instance of the \texttt{Random} typeclass (from the
\emph{random} package).

For comparison, \texttt{Maybe} could have been written like:

\begin{Shaded}
\begin{Highlighting}[]
\KeywordTok{data} \DataTypeTok{Maybe}\NormalTok{ a }\KeywordTok{where}
    \DataTypeTok{Just}\OtherTok{    ::}\NormalTok{ a }\OtherTok{->} \DataTypeTok{Maybe}\NormalTok{ a}
    \DataTypeTok{Nothing}\OtherTok{ ::} \DataTypeTok{Maybe}\NormalTok{ a}
\end{Highlighting}
\end{Shaded}

Anyways, every constructor is basically made with a function, ``\emph{If} I had
a random number, what would I do with it?'' \texttt{FromRandom} basically
contains a function \texttt{r\ -\textgreater{}\ a}\ldots{}if I had a random
number, how would I get my \texttt{a}? Note that this can really contain a
function from \emph{any} \texttt{r} you want, as long as it is a part of the
\texttt{Random} typeclass. So we can have something that takes any random
\texttt{Bool} to a \texttt{String}:

\begin{Shaded}
\begin{Highlighting}[]
\DataTypeTok{FromRandom}\NormalTok{ (\textbackslash{}r }\OtherTok{->}\NormalTok{ show (}\OtherTok{r ::} \DataTypeTok{Bool}\NormalTok{))}\OtherTok{ ::} \DataTypeTok{RandF} \DataTypeTok{String}
\end{Highlighting}
\end{Shaded}

Which says, ``if I had a random \texttt{Bool}, then I'd \texttt{show} it, to get
a \texttt{String}''.

Or maybe something that takes any random \texttt{Double} to an \texttt{Int},
like:

\begin{Shaded}
\begin{Highlighting}[]
\DataTypeTok{FromRandom}\NormalTok{ (\textbackslash{}r }\OtherTok{->}\NormalTok{ round (}\DecValTok{100} \FunctionTok{*}\NormalTok{ r }\FunctionTok{*}\OtherTok{ r ::} \DataTypeTok{Double}\NormalTok{))}\OtherTok{ ::} \DataTypeTok{RandF} \DataTypeTok{Int}
\end{Highlighting}
\end{Shaded}

``If I had a random \texttt{Double}, I'd square it and divide it by three, then
round it, to get my \texttt{Int}''.

Or something as simple as just getting a random \texttt{Double}:

\begin{Shaded}
\begin{Highlighting}[]
\DataTypeTok{FromRandom}\OtherTok{ id ::} \DataTypeTok{RandF} \DataTypeTok{Double}
\end{Highlighting}
\end{Shaded}

``If I had a random \texttt{Double}\ldots{}well, that's what I want in the
end.''

\texttt{FromRandomR} takes a range first before giving the function;
\texttt{FromRandoms} asks, ``if I had an infinite list of random items, what
would I do?''

\begin{Shaded}
\begin{Highlighting}[]
\DataTypeTok{FromRandomR} \DecValTok{0} \DecValTok{10}\NormalTok{ (\textbackslash{}r }\OtherTok{->} \DecValTok{1} \FunctionTok{/}\NormalTok{ sqrt r)}\OtherTok{ ::} \DataTypeTok{RandF} \DataTypeTok{Double}
\DataTypeTok{FromRandoms}\NormalTok{ (\textbackslash{}rs }\OtherTok{->}\NormalTok{ sum (take }\DecValTok{10}\NormalTok{ rs))}\OtherTok{ ::} \DataTypeTok{RandF} \DataTypeTok{Double}
\end{Highlighting}
\end{Shaded}

We can ``evaluate'' the random value in a \texttt{RandF} by just generating a
random value of the type desired and then applying the function to it. One
typical way of doing this is to ask for a random generator/seed from the user:

\begin{Shaded}
\begin{Highlighting}[]
\CommentTok{-- source: https://github.com/mstksg/inCode/tree/master/code-samples/free-random/Rand.hs#L31-L35}

\OtherTok{runRandomF ::} \DataTypeTok{RandomGen}\NormalTok{ g }\OtherTok{=>} \DataTypeTok{RandF}\NormalTok{ a }\OtherTok{->}\NormalTok{ g }\OtherTok{->}\NormalTok{ a}
\NormalTok{runRandomF (}\DataTypeTok{FromRandom}\NormalTok{ f)         }\FunctionTok{=}\NormalTok{ f }\FunctionTok{.}\NormalTok{ fst }\FunctionTok{.}\NormalTok{ random}
\NormalTok{runRandomF (}\DataTypeTok{FromRandomR}\NormalTok{ r0 r1 f)  }\FunctionTok{=}\NormalTok{ f }\FunctionTok{.}\NormalTok{ fst }\FunctionTok{.}\NormalTok{ randomR (r0, r1)}
\NormalTok{runRandomF (}\DataTypeTok{FromRandoms}\NormalTok{ f)        }\FunctionTok{=}\NormalTok{ f }\FunctionTok{.}\NormalTok{ randoms}
\NormalTok{runRandomF (}\DataTypeTok{FromRandomRs}\NormalTok{ r0 r1 f) }\FunctionTok{=}\NormalTok{ f }\FunctionTok{.}\NormalTok{ randomRs (r0, r1)}
\end{Highlighting}
\end{Shaded}

We can try some of these out:

\begin{Shaded}
\begin{Highlighting}[]
\NormalTok{ghci}\FunctionTok{>} \KeywordTok{let}\NormalTok{ s }\FunctionTok{=}\NormalTok{ mkStdGen }\DecValTok{192837465}
\NormalTok{ghci}\FunctionTok{>}\NormalTok{ runRandomF (}\DataTypeTok{FromRandom}\NormalTok{ (\textbackslash{}r }\OtherTok{->}\NormalTok{ show (}\OtherTok{r ::} \DataTypeTok{Bool}\NormalTok{))) s}
\StringTok{"False"}
\NormalTok{ghci}\FunctionTok{>}\NormalTok{ runRandomF (}\DataTypeTok{FromRandom}\NormalTok{ (\textbackslash{}r }\OtherTok{->}\NormalTok{ round (}\DecValTok{100} \FunctionTok{*}\NormalTok{ r }\FunctionTok{*}\NormalTok{ r))) s}
\DecValTok{32}
\NormalTok{ghci}\FunctionTok{>}\NormalTok{ runRandomF (}\DataTypeTok{FromRandom}\OtherTok{ id ::} \DataTypeTok{RandF} \DataTypeTok{Double}\NormalTok{) s}
\FloatTok{0.5631451666688826}
\NormalTok{ghci}\FunctionTok{>}\NormalTok{ runRandomF (}\DataTypeTok{FromRandomR} \DecValTok{0} \DecValTok{10}\NormalTok{ (\textbackslash{}r }\OtherTok{->} \DecValTok{1} \FunctionTok{/}\NormalTok{ sqrt r)) s}
\FloatTok{0.4213954281350406}
\NormalTok{ghci}\FunctionTok{>}\NormalTok{ runRandomF (}\DataTypeTok{FromRandoms}\NormalTok{ (sum }\FunctionTok{.}\NormalTok{ take }\DecValTok{10}\NormalTok{)}\OtherTok{ ::} \DataTypeTok{RandF} \DataTypeTok{Double}\NormalTok{) s}
\FloatTok{9.434604856390711}
\end{Highlighting}
\end{Shaded}

We might also realize that we can make a \texttt{Functor} instance on
\texttt{RandF}, where \texttt{fmap} is like applying the fmapping function to
the outbound value. For example, if we had
\texttt{FromRandom\ (\textbackslash{}r\ -\textgreater{}\ r\ *\ 2)}, if we
\texttt{fmap\ show}, we would want
\texttt{FromRandom\ (\textbackslash{}r\ -\textgreater{}\ show\ (r\ *\ 2))}, so
whenever we ``ran'' the \texttt{RandF}\ldots{}if it was ``meant'' to make a
\texttt{10} originally, it would now make a \texttt{"10"}.

\begin{Shaded}
\begin{Highlighting}[]
\CommentTok{-- source: https://github.com/mstksg/inCode/tree/master/code-samples/free-random/Rand.hs#L24-L29}

\KeywordTok{instance} \DataTypeTok{Functor} \DataTypeTok{RandF} \KeywordTok{where}
\NormalTok{    fmap h rnd }\FunctionTok{=} \KeywordTok{case}\NormalTok{ rnd }\KeywordTok{of}
        \DataTypeTok{FromRandom}\NormalTok{         f }\OtherTok{->} \DataTypeTok{FromRandom}\NormalTok{         (h }\FunctionTok{.}\NormalTok{ f)}
        \DataTypeTok{FromRandomR}\NormalTok{ r0 r1  f }\OtherTok{->} \DataTypeTok{FromRandomR}\NormalTok{ r0 r1  (h }\FunctionTok{.}\NormalTok{ f)}
        \DataTypeTok{FromRandoms}\NormalTok{        f }\OtherTok{->} \DataTypeTok{FromRandoms}\NormalTok{        (h }\FunctionTok{.}\NormalTok{ f)}
        \DataTypeTok{FromRandomRs}\NormalTok{ r0 r1 f }\OtherTok{->} \DataTypeTok{FromRandomRs}\NormalTok{ r0 r1 (h }\FunctionTok{.}\NormalTok{ f)}
\end{Highlighting}
\end{Shaded}

\begin{Shaded}
\begin{Highlighting}[]
\NormalTok{ghci}\FunctionTok{>} \KeywordTok{let}\NormalTok{ s }\FunctionTok{=}\NormalTok{ mkStdGen }\DecValTok{192837465}
\NormalTok{ghci}\FunctionTok{>} \KeywordTok{let}\NormalTok{ r }\FunctionTok{=} \DataTypeTok{FromRandom}\NormalTok{ (\textbackslash{}r }\OtherTok{->}\NormalTok{ round (}\DecValTok{100} \FunctionTok{*}\NormalTok{ r }\FunctionTok{*}\NormalTok{ r))}
\NormalTok{ghci}\FunctionTok{>}\NormalTok{ runRandomF r s}
\DecValTok{32}
\NormalTok{ghci}\FunctionTok{>}\NormalTok{ runRandomF (fmap show r) s}
\StringTok{"32"}
\NormalTok{ghci}\FunctionTok{>}\NormalTok{ runRandomF (fmap negate r) s}
\FunctionTok{-}\DecValTok{32}
\end{Highlighting}
\end{Shaded}

Put in a rather abstract way, if you think of a \texttt{RandF\ Double} as
something that ``contains'' a random \texttt{Double}, waiting to be
computed\ldots{}then \texttt{fmap\ show} applies \texttt{show} to the
``contained'' random \texttt{Double}.

Hi, thanks for reading! You can reach me via email at
\href{mailto:justin@jle.im}{\nolinkurl{justin@jle.im}}, or at twitter at
\href{https://twitter.com/mstk}{@mstk}! This post and all others are published
under the \href{https://creativecommons.org/licenses/by-nc-nd/3.0/}{CC-BY-NC-ND
3.0} license. Corrections and edits via pull request are welcome and encouraged
at \href{https://github.com/mstksg/inCode}{the source repository}.

If you feel inclined, or this post was particularly helpful for you, why not
consider \href{https://www.patreon.com/justinle/overview}{supporting me on
Patreon}, or a \href{bitcoin:3D7rmAYgbDnp4gp4rf22THsGt74fNucPDU}{BTC donation}?
:)

\end{document}
