\documentclass[]{article}
\usepackage{lmodern}
\usepackage{amssymb,amsmath}
\usepackage{ifxetex,ifluatex}
\usepackage{fixltx2e} % provides \textsubscript
\ifnum 0\ifxetex 1\fi\ifluatex 1\fi=0 % if pdftex
  \usepackage[T1]{fontenc}
  \usepackage[utf8]{inputenc}
\else % if luatex or xelatex
  \ifxetex
    \usepackage{mathspec}
    \usepackage{xltxtra,xunicode}
  \else
    \usepackage{fontspec}
  \fi
  \defaultfontfeatures{Mapping=tex-text,Scale=MatchLowercase}
  \newcommand{\euro}{€}
\fi
% use upquote if available, for straight quotes in verbatim environments
\IfFileExists{upquote.sty}{\usepackage{upquote}}{}
% use microtype if available
\IfFileExists{microtype.sty}{\usepackage{microtype}}{}
\usepackage[margin=1in]{geometry}
\ifxetex
  \usepackage[setpagesize=false, % page size defined by xetex
              unicode=false, % unicode breaks when used with xetex
              xetex]{hyperref}
\else
  \usepackage[unicode=true]{hyperref}
\fi
\hypersetup{breaklinks=true,
            bookmarks=true,
            pdfauthor={},
            pdftitle={},
            colorlinks=true,
            citecolor=blue,
            urlcolor=blue,
            linkcolor=magenta,
            pdfborder={0 0 0}}
\urlstyle{same}  % don't use monospace font for urls
% Make links footnotes instead of hotlinks:
\renewcommand{\href}[2]{#2\footnote{\url{#1}}}
\setlength{\parindent}{0pt}
\setlength{\parskip}{6pt plus 2pt minus 1pt}
\setlength{\emergencystretch}{3em}  % prevent overfull lines
\setcounter{secnumdepth}{0}


\begin{document}

\% Inside My World (Ode to Functor and Monad) \% Justin Le \% May 19, 2014

\emph{Originally posted on
\textbf{\href{https://blog.jle.im/entry/inside-my-world-ode-to-functor-and-monad.html}{in
Code}}.}

I like Haskell because it lets me live inside my world.

There are a lot of special worlds out there! And Haskell lets me stay in those
worlds, and use all of the tools I normally have when I'm not there. I get to
transform normal tools into tools that work in my world.

(This post is meant to be approachable by people unfamiliar with Haskell! That
being said, if there is a concept you don't understand, feel free to leave a
comment, \href{https://twitter.com/mstk}{tweet} me, stop by on irc at freenode's
\#haskell, or give \href{http://learnyouahaskell.com/}{Learn You a Haskell} a
quick read!)

\section{Stuck in Maybe}

(Feel free to play along with the code in this section by
\href{https://github.com/mstksg/inCode/tree/master/code-samples/inside/maybe.hs}{loading
it into ghci}, the Haskell interpreter!)

In Haskell, we have a type called \texttt{Maybe\ a}:

\textasciitilde{}\textasciitilde{}\textasciitilde{}haskell data Maybe a =
Nothing \textbar{} Just a \textasciitilde{}\textasciitilde{}\textasciitilde{}

This says that \texttt{Maybe\ a} is like an Enumerable type of sorts; The
\texttt{\textbar{}} reads like "\emph{or}".

This is like saying

\textasciitilde{}\textasciitilde{}\textasciitilde{}haskell data Bool = False
\textbar{} True \textasciitilde{}\textasciitilde{}\textasciitilde{}

to define a \texttt{Bool} data type. If I have something of type \texttt{Bool},
it can be (literally) \texttt{False} or \texttt{True}. If I have something of
type \texttt{Maybe\ a}, it can be \texttt{Nothing} (nothing is there, it's
empty) or \texttt{Just\ x} (it contains a value \texttt{x}).

If you are used to an OOP language with templates or generics, this is similar
to saying \texttt{Maybe\textless{}a\textgreater{}} -\/-
\texttt{Maybe\textless{}a\textgreater{}} is a parameterized type over some
\texttt{a}.

This type is useful for functions that might fail:

\textasciitilde{}\textasciitilde{}\textasciitilde{}haskell -\/- source:
https://github.com/mstksg/inCode/tree/master/code-samples/inside/maybe.hs\#L23-L41
-\/- interactive: https://www.fpcomplete.com/user/jle/inside-my-world -\/-
divideMaybe: Takes two integers and returns -\/- possibly -\/- their integer
-\/- quotient. It succeeds if the denominator is not zero, and fails if -\/- it
is. divideMaybe :: Int -\textgreater{} Int -\textgreater{} Maybe Int divideMaybe
\_ 0 = Nothing divideMaybe x y = Just (x \texttt{div} y)

-\/- headMaybe: Takes a list and returns -\/- possibly -\/- its first element.
-\/- Fails if the list is empty, and succeeds with the first element -\/-
otherwise. headMaybe :: {[}a{]} -\textgreater{} Maybe a headMaybe {[}{]} =
Nothing headMaybe (x:\_) = Just x

-\/- halveMaybe: Takes an integer and returns -\/- possibly -\/- its half. Fails
-\/- if it is an odd number. halveMaybe :: Int -\textgreater{} Maybe Int
halveMaybe x \textbar{} x \texttt{mod} 2 == 0 = Just (x \texttt{div} 2)
\textbar{} otherwise = Nothing
\textasciitilde{}\textasciitilde{}\textasciitilde{}

\textless{}div class="note"\textgreater{} \textbf{Aside}

Oh hi!

For people new to Haskell:

\textasciitilde{}\textasciitilde{}\textasciitilde{}haskell foo :: Int
-\textgreater{} Bool foo x = ...
\textasciitilde{}\textasciitilde{}\textasciitilde{}

declares a function named \texttt{foo} of type
\texttt{Int\ -\textgreater{}\ Bool} -\/-\/- we use \texttt{::} to specify type
signatures. \texttt{Int\ -\textgreater{}\ Bool} means that it takes an
\texttt{Int} (named \texttt{x}) and returns a \texttt{Bool}.

I'll often just say \texttt{bar\ ::\ Bool} to say "the value \texttt{bar} (of
type \texttt{Bool})"; you could just read \texttt{::} as "type of".

So
\texttt{divideMaybe\ ::\ Int\ -\textgreater{}\ Int\ -\textgreater{}\ Maybe\ Int}
means that \texttt{divideMaybe} takes two \texttt{Int}s and returns something of
type \texttt{Maybe\ Int}.

You might have also noticed the pattern matching construct,
\texttt{headMaybe\ (x:\_)}. This matches the first element in the list to the
name \texttt{x}, and the rest of the list to the wildcard, \texttt{\_}.
\textless{}/div\textgreater{}

When you want to return a value of type \texttt{Maybe\ a}, you can either return
\texttt{Just\ x} or \texttt{Nothing} (where \texttt{x\ ::\ a}) -\/-\/- they both
are members of type \texttt{Maybe\ a}. That's what \texttt{Maybe\ Int} means
-\/-\/- an \texttt{Int} that might or might not be there!

If I had something of type \texttt{Maybe\ Int}, would you know for sure if that
\texttt{Int} was there or not (from just the type)? You wouldn't! You are living
in the world of uncertainties.

Welcome to the world of uncertainty.{[}\^{}dundundun{]}

\subsection{The Problem}

Okay, well, I have a \texttt{Maybe\ Int}. Which is nice and all...but...I want
to do normal inty-things with it.

That is...I have all these functions that work only on \texttt{Int}!

\textasciitilde{}\textasciitilde{}\textasciitilde{}haskell -\/- source:
https://github.com/mstksg/inCode/tree/master/code-samples/inside/maybe.hs\#L43-L50
-\/- interactive: https://www.fpcomplete.com/user/jle/inside-my-world addThree
:: Int -\textgreater{} Int addThree = (+ 3)

square :: Int -\textgreater{} Int square = (\^{} 2)

showInt :: Int -\textgreater{} String showInt = show
\textasciitilde{}\textasciitilde{}\textasciitilde{}

But...I can't do these things on \texttt{Maybe\ Int}!

\textasciitilde{}\textasciitilde{}\textasciitilde{}haskell ghci\textgreater{}
addThree (Just 5) *** SCARY ERROR! *** addThree takes an Int but you gave it a
Maybe Int. *** What are you trying to do anyway, wise guy.
\textasciitilde{}\textasciitilde{}\textasciitilde{}

\textless{}div class="note"\textgreater{} \textbf{Aside}

In this post, commands at the interactive Haskell interpreter (REPL) ghci are
prefaced with the prompt \texttt{ghci\textgreater{}}. If you see
\texttt{ghci\textgreater{}}, it means that this is something you'd enter at
ghci. If not, it is normal Haskell source code!

In \texttt{ghci}, we also have this command \texttt{:t} that you'll be seeing
often that lets you find the type of something:

\textasciitilde{}\textasciitilde{}\textasciitilde{}haskell ghci\textgreater{} :t
True True :: Bool \textasciitilde{}\textasciitilde{}\textasciitilde{}
\textless{}/div\textgreater{}

In most other languages, to get around this, you would "exit" your uncertain
world. That is, you would turn your uncertain 5 into either a certain 5 or an
error. Or you would turn your uncertain 5 into either a certain 5 or some
"default" value.

That is, you would use functions like these to exit your world:{[}\^{}fjfm{]}

\textasciitilde{}\textasciitilde{}\textasciitilde{}haskell -\/- source:
https://github.com/mstksg/inCode/tree/master/code-samples/inside/maybe.hs\#L76-L82
-\/- interactive: https://www.fpcomplete.com/user/jle/inside-my-world
certaintify :: Maybe a -\textgreater{} a certaintify (Just x) = x certaintify
Nothing = error "Nothing was there, you fool!"

certaintifyWithDefault :: a -\textgreater{} Maybe a -\textgreater{} a
certaintifyWithDefault \_ (Just x) = x certaintifyWithDefault d Nothing = d
\textasciitilde{}\textasciitilde{}\textasciitilde{}

And then you can just willy-nilly use your normal
\texttt{Int\ -\textgreater{}\ Int} functions on what you pull out...using
various "error handling" mechanisms if it was \texttt{Nothing}.

\textasciitilde{}\textasciitilde{}\textasciitilde{}haskell ghci\textgreater{}
addThree (certaintify (headMaybe {[}1,2,3{]})) 4 ghci\textgreater{} square
(certaintify (halveMaybe 7)) *** Exception: Nothing was there, you fool!
ghci\textgreater{} square (certaintifyWithDefault 0 (halveMaybe 7)) 0
\textasciitilde{}\textasciitilde{}\textasciitilde{}

But...work with me here. Let's say I want to live in my uncertain world.

There are a lot of reasons why one might want to do that.

Let's say you had a function that looked up a person from a database given their
ID number. But not all ID numbers have a person attached, so the function might
fail and not lookup anyone.

\textasciitilde{}\textasciitilde{}\textasciitilde{}haskell personFromId :: ID
-\textgreater{} Maybe Person \textasciitilde{}\textasciitilde{}\textasciitilde{}

And you also had a function that returned the age of a given person:

\textasciitilde{}\textasciitilde{}\textasciitilde{}haskell age :: Person
-\textgreater{} Int \textasciitilde{}\textasciitilde{}\textasciitilde{}

What if you wanted to write a function that looked up \emph{the age of the
person in that database with that ID}. The result is going to also be in a
\texttt{Maybe}, because the given ID might not correspond to anyone to have an
age for.

\textasciitilde{}\textasciitilde{}\textasciitilde{}haskell ageFromId :: ID
-\textgreater{} Maybe Int \textasciitilde{}\textasciitilde{}\textasciitilde{}

In this case, it would make no sense to "exit" the world of uncertainty as soon
as we get a \texttt{Maybe\ Person}, and then "re-enter" it somehow when you
return the \texttt{Maybe\ Int}. Our entire answer is shrouded in uncertainty, so
we need to \emph{stay inside this world} the entire time. We want to find a way
to deal with values inside a world \emph{without leaving it}.

So we have a function \texttt{Person\ -\textgreater{}\ Int}, and a
\texttt{Maybe\ Person}...darnit. How do we use our \texttt{age} function,
without leaving \texttt{Maybe}? We certainly want to re-use the same function
somehow, and not write it again from scratch!

\subsection{Can I have a lift?}

So the problem: I have a function \texttt{a\ -\textgreater{}\ b} that I want to
be able to use on a \texttt{Maybe\ a}...I want to stay in my \texttt{Maybe}
world and use that function on the uncertain value.

If you look at this carefully, we want some sort of "function transformer". Give
our transformer an \texttt{a\ -\textgreater{}\ b}, it'll output a new function
\texttt{Maybe\ a\ -\textgreater{}\ Maybe\ b}. The new function takes an
\texttt{a} that may or may not be there, and outputs a \texttt{b} that may or
not be there.

We want a function of type
\texttt{(a\ -\textgreater{}\ b)\ -\textgreater{}\ (Maybe\ a\ -\textgreater{}\ Maybe\ b)}

Let's make one! It'll apply the function to the value inside a \texttt{Just},
and leave a \texttt{Nothing} alone.

\textasciitilde{}\textasciitilde{}\textasciitilde{}haskell -\/- source:
https://github.com/mstksg/inCode/tree/master/code-samples/inside/maybe.hs\#L84-L88
-\/- interactive: https://www.fpcomplete.com/user/jle/inside-my-world inMaybe ::
(a -\textgreater{} b) -\textgreater{} (Maybe a -\textgreater{} Maybe b) inMaybe
f = liftedF where liftedF (Just x) = Just (f x) liftedF Nothing = Nothing
\textasciitilde{}\textasciitilde{}\textasciitilde{}

What can we do with it?

\textasciitilde{}\textasciitilde{}\textasciitilde{}haskell ghci\textgreater{}
let addThreeInMaybe = inMaybe addThree ghci\textgreater{} addThreeInMaybe (Just
7) Just 10 ghci\textgreater{} addThreeInMaybe Nothing Nothing ghci\textgreater{}
(inMaybe square) (Just 9) Just 81 ghci\textgreater{} (inMaybe showInt) Nothing
Nothing ghci\textgreater{} (inMaybe showInt) (Just 8) Just "8"
\textasciitilde{}\textasciitilde{}\textasciitilde{}

Wow! We can now use normal functions and still stay inside my uncertain world.
We could even write our \texttt{ageFromId}:

\textasciitilde{}\textasciitilde{}\textasciitilde{}haskell -\/- source:
https://github.com/mstksg/inCode/tree/master/code-samples/inside/maybe.hs\#L68-L69
-\/- interactive: https://www.fpcomplete.com/user/jle/inside-my-world ageFromId
:: ID -\textgreater{} Maybe Int ageFromId i = (inMaybe age) (personFromId i)
\textasciitilde{}\textasciitilde{}\textasciitilde{}

Now we are no longer afraid of dealing with uncertainty. It's a scary realm, but
as long as we have \texttt{inMaybe}...all of our normal tools apply!

\textasciitilde{}\textasciitilde{}\textasciitilde{}haskell ghci\textgreater{}
let x = headMaybe {[}2,3,4{]} -\/- x = Just 2 ghci\textgreater{} let y =
(inMaybe square) x -\/- y = Just 4 ghci\textgreater{} let z = (inMaybe addThree)
y -\/- z = Just 7 ghci\textgreater{} (inMaybe (\textgreater{} 5)) z Just True
ghci\textgreater{} let x' = halveMaybe 7 -\/- x' = Nothing ghci\textgreater{}
let y' = (inMaybe square) x' -\/- y' = Nothing ghci\textgreater{} let z' =
(inMaybe addThree) y' -\/- z' = Nothing ghci\textgreater{} (inMaybe
(\textgreater{} 5)) z' Nothing
\textasciitilde{}\textasciitilde{}\textasciitilde{}

\subsection{Functor}

This concept of "bringing functions into worlds" is actually a useful and
generalizable concept. In fact, in the standard libraries, there's a typeclass
(which is like an interface, sorta, for you Java/OOP people) that provides a
common API/interface for "worlds that you can bring functions into."

We call it \texttt{Functor}, and this "bring into world" function is called
\texttt{fmap}.

It should come as no surprise that \texttt{Maybe} is a Functor, so \texttt{fmap}
\emph{does} take any function \texttt{a\ -\textgreater{}\ b} and "lifts" it into
the \texttt{Maybe} world, turning it into a
\texttt{Maybe\ a\ -\textgreater{}\ Maybe\ b}.

\texttt{fmap} for \texttt{Maybe} is incidentally exactly our \texttt{inMaybe}.

\textasciitilde{}\textasciitilde{}\textasciitilde{}haskell ghci\textgreater{}
(fmap square) (headMaybe {[}4,5,6{]}) Just 16 ghci\textgreater{} (fmap square)
(halveMaybe 7) Nothing \textasciitilde{}\textasciitilde{}\textasciitilde{}

\textless{}div class="note"\textgreater{} \textbf{Aside}

Any "legitimate" instance of \texttt{Functor} must satisfy a couple of
properties -\/-\/- "laws", so to speak. These laws basically ensure that
whatever instance you define is useful and sensible, and follow what sort of
meaning \texttt{fmap} is supposed to convey.

\begin{enumerate}
\tightlist
\item
  \texttt{fmap\ (f\ .\ g)} should equal \texttt{fmap\ f\ .\ fmap\ g}; that is,
  lifting composed functions be the same as composing lifted functions.
  (\texttt{(.)} is the function composition operator)
\item
  \texttt{fmap\ id\ thing} should leave \texttt{thing} unchanged.
\end{enumerate}

\textless{}/div\textgreater{}

Some notes before we move on!

First of all, even though we have been writing things like
\texttt{(fmap\ f)\ x}, the parentheses are actually unnecessary due to the way
Haskell associates function calls. So \texttt{(fmap\ f)\ x} is the same as
\texttt{fmap\ f\ x}, and we'll be writing it that way from now on.

Secondly, an infix operator alias for \texttt{fmap} exists:
\texttt{(\textless{}\$\textgreater{})}. That way, you can write
\texttt{fmap\ f\ x} as \texttt{f\ \textless{}\$\textgreater{}\ x}, which is
meant to look similar to \texttt{f\ \$\ x}:

\textasciitilde{}\textasciitilde{}\textasciitilde{}haskell ghci\textgreater{}
addThree \$ 7 10 ghci\textgreater{} addThree \textless{}\$\textgreater{} Just 7
Just 10 ghci\textgreater{} addThree \textless{}\$\textgreater{} Nothing Nothing
\textasciitilde{}\textasciitilde{}\textasciitilde{}

(For those unfamiliar, \texttt{f\ \$\ x} = \texttt{f\ x})

\subsection{Sort of a big deal}

Let's pause and reflect to see that this is sort of a big deal, and see what
problem \texttt{Functor} just solved.

In another language, you might somehow have a
\texttt{Maybe\textless{}Int\textgreater{}} (using generics syntax). And you have
lots and lots and lots of functions that take \texttt{Int}s. Heck, why would you
even ever have a function take a \texttt{Maybe\textless{}Int\textgreater{}}? A
function would be like:

\textasciitilde{}\textasciitilde{}\textasciitilde{}java class Monster \{ void
deal\_damage(int damage) \{\}; \}
\textasciitilde{}\textasciitilde{}\textasciitilde{}

where your \texttt{deal\_damage} function would take an integer. So
\texttt{Maybe\ Int} is useless! You either have to re-write
\texttt{deal\_damage} to take a \texttt{Maybe\ Int}, and have \emph{two
versions} of it, or you turn your \texttt{Maybe\ Int} into an \texttt{Int}
somehow.

In this light, \texttt{Maybe} is a huge nuisance. It is a big, annoying thing to
deal with and it probably results in a lot of boilerplate, making you either
duplicate functions or extract \texttt{Maybe} values every time you get one.

But now...\emph{now}, \texttt{Maybe} is not a nuisance, and there is \emph{no
boilerplate}. All your functions now...\emph{just work}, as they are!

And this is a big deal.

\section{"Pre-lifting"}

Okay, so we now can turn \texttt{a\ -\textgreater{}\ b} into
\texttt{Maybe\ a\ -\textgreater{}\ Maybe\ b}.

That might be nice, but if you scroll up just a bit, you might see that there
are other functions that might be interesting to apply on a \texttt{Maybe\ a}.

What about \texttt{halveMaybe\ ::\ Int\ -\textgreater{}\ Maybe\ Int}? Can I use
\texttt{halveMaybe} on a \texttt{Maybe\ Int}?

\textasciitilde{}\textasciitilde{}\textasciitilde{}haskell ghci\textgreater{}
let x = divideMaybe 12 3 -\/- x = Just 4 :: Maybe Int ghci\textgreater{}
halveMaybe x *** SCARY ERROR! *** halveMaybe takes an Int but you gave it *** a
Maybe Int. Please think about your life.
\textasciitilde{}\textasciitilde{}\textasciitilde{}

Oh no! Maybe we can't really stay inside our \texttt{Maybe} world after all!

This might be important! Let's imagine this trip down our world of uncertainty
-\/-\/- let's say we wanted a function \texttt{halfOfAge}

\textasciitilde{}\textasciitilde{}\textasciitilde{}haskell halfOfAge :: ID
-\textgreater{} Maybe Int \textasciitilde{}\textasciitilde{}\textasciitilde{}

That returns (possibly), half of the age of the person corresponding to that ID
(and \texttt{Nothing} if the person looked up has an odd age. Because odd ages
don't have halves, of course.). Well, we already have
\texttt{ageFromId\ ::\ ID\ -\textgreater{}\ Maybe\ Int}, but we want to apply
\texttt{halveMaybe} to that \texttt{Maybe\ Int}. But we can't! Because
\texttt{halveMaybe} only works on \texttt{Int}!

We can't even use \texttt{fmap}, because:

\textasciitilde{}\textasciitilde{}\textasciitilde{}haskell ghci\textgreater{} :t
fmap halveMaybe fmap halveMaybe :: Maybe Int -\textgreater{} Maybe (Maybe Int)
\textasciitilde{}\textasciitilde{}\textasciitilde{}

Wrong wrong wrong! We don't want a
\texttt{Maybe\ Int\ -\textgreater{}\ Maybe\ (Maybe\ Int)}, we want a
\texttt{Maybe\ Int\ -\textgreater{}\ Maybe\ Int}! \texttt{fmap} lifts "both
sides" of the function, but we only want, in this case, to "lift" the input.

This is a disaster!

But wait, calm down. We have overcome similar things before. With our recent
journey to Functor enlightenment in mind, let's try to look for a similar path.

We had an \texttt{a\ -\textgreater{}\ b} that we wanted to apply to a
\texttt{Maybe\ a}, we used \texttt{fmap} to turn it into a
\texttt{Maybe\ a\ -\textgreater{}\ Maybe\ b}. So we have a
\texttt{a\ -\textgreater{}\ Maybe\ b} here that we want to apply to a
\texttt{Maybe\ a}. The plan is simple! We turn an
\texttt{a\ -\textgreater{}\ Maybe\ b} into a
\texttt{Maybe\ a\ -\textgreater{}\ Maybe\ b}. Let's pretend we had such a
function.

\textasciitilde{}\textasciitilde{}\textasciitilde{}haskell liftInput :: (a
-\textgreater{} Maybe b) -\textgreater{} (Maybe a -\textgreater{} Maybe b)
\textasciitilde{}\textasciitilde{}\textasciitilde{}

How should we expect this to behave?

Well, let's think this through case-by-case.

If we want to apply \texttt{halveMaybe} to a number that isn't there...well...it
should also return a number that isn't there. It should propagate the
not-thereness.

If we want to apply \texttt{halveMaybe} to a number that \emph{is} there...well,
just apply it to that number, and take that result! If the result is there, then
you have a result there. If the result is not there, then you don't.

We have enough to write this out ourselves:

\textasciitilde{}\textasciitilde{}\textasciitilde{}haskell -\/- source:
https://github.com/mstksg/inCode/tree/master/code-samples/inside/maybe.hs\#L90-L94
-\/- interactive: https://www.fpcomplete.com/user/jle/inside-my-world liftInput
:: (a -\textgreater{} Maybe b) -\textgreater{} (Maybe a -\textgreater{} Maybe b)
liftInput f = liftedF where liftedF Nothing = Nothing liftedF (Just x) = f x
\textasciitilde{}\textasciitilde{}\textasciitilde{}

\textasciitilde{}\textasciitilde{}\textasciitilde{}haskell ghci\textgreater{} :t
liftInput halveMaybe Maybe Int -\textgreater{} Maybe Int ghci\textgreater{} let
x = divideMaybe 12 3 -\/- x = Just 4 :: Maybe Int ghci\textgreater{} (liftInput
halveMaybe) x Just 2 ghci\textgreater{} let y = divideMaybe 12 0 -\/- y =
Nothing :: Maybe Int ghci\textgreater{} (liftInput halveMaybe) y Nothing
\textasciitilde{}\textasciitilde{}\textasciitilde{}

Neat! Now we don't have to fear \texttt{a\ -\textgreater{}\ Maybe\ b}'s...we can
use them and \emph{still stay in our world}, without leaving our world of
uncertainty!

\textasciitilde{}\textasciitilde{}\textasciitilde{}haskell -\/- source:
https://github.com/mstksg/inCode/tree/master/code-samples/inside/maybe.hs\#L71-L72
-\/- interactive: https://www.fpcomplete.com/user/jle/inside-my-world halfOfAge
:: ID -\textgreater{} Maybe Int halfOfAge i = (liftInput halveMaybe) (ageFromId
i) \textasciitilde{}\textasciitilde{}\textasciitilde{}

\subsection{Monad}

Like with Functor and \texttt{fmap}, this general pattern of turning an
\texttt{a\ -\textgreater{}\ f\ b} into an \texttt{f\ a\ -\textgreater{}\ f\ b}
is also useful to generalize.

In general, you can think functions \texttt{a\ -\textgreater{}\ world\ b} as
functions that "bring you into your world". We would like to turn it, in
general, into \texttt{world\ a\ -\textgreater{}\ world\ b}. Lifting the input
only, so to speak.

We say that if a world has such a way of lifting the input of such a function
(plus some other requirements), it implements the \texttt{Monad}
typeclass{[}\^{}othermonad{]}.

Monad is a typeclass (which is kinda like an interface), so that means that if
\texttt{Maybe} is a Monad, it "implements" that way to turn a
\texttt{a\ -\textgreater{}\ Maybe\ b} into a
\texttt{Maybe\ a\ -\textgreater{}\ Maybe\ b}.

We call this
\texttt{(a\ -\textgreater{}\ Maybe\ b)\ -\textgreater{}\ (Maybe\ a\ -\textgreater{}\ Maybe\ b)}
function "bind".

Now, embarrassingly enough, "bind" actually isn't called \texttt{bind} in the
standard library...it actually only exists as an operator,
\texttt{(=\textless{}\textless{})}.

(Remember how there was an operator form of \texttt{fmap}? We have both
\texttt{fmap} and \texttt{(\textless{}\$\textgreater{})}? Well, in this case, we
\emph{only} have the operator form of \texttt{bind},
\texttt{(=\textless{}\textless{})}. Yeah, I know. But we live with it just
fine!).

\texttt{(=\textless{}\textless{})} is exactly our \texttt{liftInput} for
\texttt{Maybe}. Let's try it out:

\textasciitilde{}\textasciitilde{}\textasciitilde{}haskell ghci\textgreater{} :t
(=\textless{}\textless{}) halveMaybe Maybe Int -\textgreater{} Maybe Int
ghci\textgreater{} let x = divideMaybe 12 3 -\/- x = Just 4 :: Maybe Int

-\/- use it as a prefix function ghci\textgreater{} (=\textless{}\textless{})
halveMaybe x Just 2 ghci\textgreater{} let y = divideMaybe 12 0 -\/- y = Nothing
:: Maybe Int

-\/- use it as an infix operator ghci\textgreater{} halveMaybe
=\textless{}\textless{} y Nothing
\textasciitilde{}\textasciitilde{}\textasciitilde{}

And now maybe we can finally rest easy knowing that we can "stay inside
\texttt{Maybe}" and never have to leave it.

\textless{}div class="note"\textgreater{} \textbf{Aside}

The "other thing" that Monad has to have (the other thing that the "interface"
demands, besides \texttt{(=\textless{}\textless{})}) is a way to "bring a value
into your world".

This function is called \texttt{return}.

For example, for \texttt{Maybe}, we need a way to take a normal value like an
\texttt{Int} and "bring it into" our world of uncertainty -\/-\/- an
\texttt{Int\ -\textgreater{}\ Maybe\ Int}. For \texttt{Maybe}, semantically, to
bring something like \texttt{7} into the world of uncertainty...well, we already
know the \texttt{7} is there. So to bring a \texttt{7} into \texttt{Maybe}, it's
just \texttt{Just\ 7}

For an instance of \texttt{Monad} to be considered legitimate, there are a few
rules/laws that \texttt{return} and \texttt{(=\textless{}\textless{})} must obey
when used together in order to be useful (just like for \texttt{Functor}). If
you define nonsensical \texttt{return} and \texttt{(=\textless{}\textless{})},
of course, your instance won't be very useful anyway, and you wouldn't be able
to reason with how they work together. The laws sort of are some way of ensuring
that your instance is useful and sensible, and that
\texttt{(=\textless{}\textless{})} and \texttt{return} make sense at all.
\textless{}/div\textgreater{}

\textless{}div class="note"\textgreater{} \textbf{Aside}

Now, for some strange reason, it is actually much more popular to use
\texttt{(\textgreater{}\textgreater{}=)} over
\texttt{(=\textless{}\textless{})}; \texttt{(\textgreater{}\textgreater{}=)} is
just \texttt{(=\textless{}\textless{})} backwards:

\textasciitilde{}\textasciitilde{}\textasciitilde{}haskell ghci\textgreater{}
halveMaybe =\textless{}\textless{} Just 8 Just 4 ghci\textgreater{} Just 8
\textgreater{}\textgreater{}= halveMaybe Just 4
\textasciitilde{}\textasciitilde{}\textasciitilde{}

This is really weird! I mean...really \emph{really} weird! Why would you ever
put the function you are applying \emph{after} the value you are applying it to?
That's like having \texttt{x\ ::\ a} and \texttt{f\ ::\ a\ -\textgreater{}\ b},
and doing \texttt{x\ f} or something!

Why is this style the norm? Who knows!{[}\^{}whoknows{]} People are just weird!

For the rest of this article, we will be using
\texttt{(=\textless{}\textless{})}; just be aware that you might see
\texttt{(\textgreater{}\textgreater{}=)} out in the wild more often!
\textless{}/div\textgreater{}

\section{Recap}

Thanks to Functor and Monad, we now have a way to confidently stay in our world
of uncertainty and still use normal functions on our uncertain values -\/-\/- we
only have to use the right "lifters".

If you have an \texttt{x\ ::\ Maybe\ a} and you have a:

\begin{itemize}
\item
  \texttt{f\ ::\ a\ -\textgreater{}\ b}, then use \texttt{fmap} or
  \texttt{(\textless{}\$\textgreater{})} -\/-\/- \texttt{fmap\ f\ x} or
  \texttt{f\ \textless{}\$\textgreater{}\ x}
\item
  \texttt{f\ ::\ a\ -\textgreater{}\ Maybe\ b}, then use
  \texttt{(=\textless{}\textless{})} -\/-\/-
  \texttt{f\ =\textless{}\textless{}\ x}
\end{itemize}

Armed with these two, you can comfortably stay in \texttt{Maybe} without ever
having to "get out of it"!

\subsection{A big picture}

Again, the big picture is this: sometimes we get values inside contexts, or
worlds. But we have functions like \texttt{a\ -\textgreater{}\ world\ b} that
\emph{produce values inside your world}.

Which normally would leave you "high and dry", because you can't, say, apply
that same function twice. You either have to write a new
\texttt{world\ a\ -\textgreater{}\ world\ b} version or some other boilerplate.

With Functor, we can make normal functions treat our world values like normal
values; with Monad, we can do the same with functions that "bring us into"
worlds. With these two together...maybe contexted values aren't so bad after
all!

\section{Other Worlds}

\subsection{on Worlds}

You might have noticed that up until now I have used the word "world" pretty
vaguely.

When I say that a value is "inside" a "world", I mean that it "lives" inside the
context of what that world represents. A \texttt{Maybe\ a} is an \texttt{a}
living in the \texttt{Maybe} "world" -\/-\/- it is an \texttt{a} that can exist
or not exist. \texttt{Maybe} represents a context of
existing-or-not-existing.{[}\^{}worlds{]}

But there are other worlds, and other contexts too. And though I have shown you
what Functor and Monad look like for \texttt{Maybe}...you probably need to see a
few more examples to be really convinced that these are general design patterns
that you can apply to multiple "values in contexts".

It's important to remember that \texttt{fmap} and
\texttt{(=\textless{}\textless{})} don't really have any inherent semantic
meaning...and their usefulness and "meaning" come from just the specific
instance. We saw what they "did" for \texttt{Maybe}, but their meaning came from
\texttt{Maybe} itself. For other worlds, as we will see, we can make them mean
completely different things.

\textless{}div class="note"\textgreater{} \textbf{Aside}

There are some important nuances that might trip you up! Though useful worlds
are instances of Monad, it is improper to say that "Monads are worlds/values in
contexts". That's not what Monads \emph{are}. Monads are just Monads (the two
functions and their laws), no more and no less.

In our usage here, Functor and Monad mean only "these things implement some sort
of \texttt{fmap} and \texttt{(=\textless{}\textless{})}, etc., and those two are
useful." That is, the interface offered by Functor and Monad are useful for our
specific world. But there are plenty of Functors and Monads that are not
"worlds". \textless{}/div\textgreater{}

Anyways, here is a whirlwind tour of different worlds, to help you realize how
often you'll actually want to live in these worlds in Haskell, and why having
\texttt{fmap} and \texttt{(=\textless{}\textless{})} are so useful!

\subsection{The world of future values}

(Play along with this section too by
\href{https://github.com/mstksg/inCode/tree/master/code-samples/inside/reader.hs}{loading
the source}!)

In Haskell, we have a \texttt{Reader\ r} world. You can think of
\texttt{(Reader\ r)\ a} as a little machine that "waits" for something of type
\texttt{r}, then \emph{uses} it to (purely) make an \texttt{a}. The \texttt{a}
doesn't exist yet; it's a future \texttt{a} that will exist as soon as you give
it an \texttt{r}.

\textasciitilde{}\textasciitilde{}\textasciitilde{}haskell -\/- source:
https://github.com/mstksg/inCode/tree/master/code-samples/inside/reader.hs\#L17-L27
-\/- interactive: https://www.fpcomplete.com/user/jle/inside-my-world -\/-
futureLength: A future \texttt{Int} that will be the length of whatever the -\/-
list it is waiting for will be. futureLength :: (Reader {[}a{]}) Int

-\/- futureHead: An future \texttt{a} that will be the first element of whatever
the -\/- list it is waiting for will be. futureHead :: (Reader {[}a{]}) a

-\/- futureOdd: A future \texttt{Bool} that will be whether the \texttt{Int} it
is waiting -\/- for is odd or not. futureOdd :: (Reader Int) Bool
\textasciitilde{}\textasciitilde{}\textasciitilde{}

\texttt{futureLength} is a "future \texttt{Int}"; an \texttt{Int} waiting (for
an \texttt{{[}a{]}}) to be realized. \texttt{futureHead} is a "future
\texttt{a}", waiting for an \texttt{{[}a{]}}.

We use the function \texttt{runReader} to "force" the \texttt{a} out of the
\texttt{(Reader\ r)\ a}:

\textasciitilde{}\textasciitilde{}\textasciitilde{}haskell -\/- given a
\texttt{(Reader\ r)\ a} and an \texttt{r}, uses that \texttt{r} to finally get
the \texttt{a}: runReader :: (Reader r) a -\textgreater{} r -\textgreater{} a
\textasciitilde{}\textasciitilde{}\textasciitilde{}

\textasciitilde{}\textasciitilde{}\textasciitilde{}haskell ghci\textgreater{}
runReader futureLength {[}1,2,3{]} 3 ghci\textgreater{} runReader futureHead
{[}1,2,3{]} 1 ghci\textgreater{} runReader futureOdd 6 False ghci\textgreater{}
runReader futureOdd 5 True \textasciitilde{}\textasciitilde{}\textasciitilde{}

Welcome to the world of future values.

\textless{}div class="note"\textgreater{} \textbf{Aside}

It is important to note here that \texttt{(Reader\ Int)\ Bool} and
\texttt{(Reader\ {[}Int{]})\ Bool} \emph{do not exist} in the same world. One
lives in a \texttt{Reader\ Int} world -\/-\/- a world of future values awaiting
an \texttt{Int}. The other lives in a \texttt{Reader\ {[}Int{]}} world -\/-\/- a
world of future values awaiting an \texttt{{[}Int{]}}.
\textless{}/div\textgreater{}

Let's say I have a future \texttt{Int}. Say, \texttt{futureLength}, waiting on
an \texttt{{[}a{]}}. And I have a function
\texttt{(\textless{}\ 5)\ ::\ Int\ -\textgreater{}\ Bool}. Can I apply
\texttt{(\textless{}\ 5)} to my future \texttt{Int}, in order to get a future
\texttt{Bool}?

At first, no! This future \texttt{Int} is useless! I can't even use it in
\emph{any} of my normal functions! Time to reach for the exit button?

Oh -\/-\/- but, because \texttt{Reader\ {[}a{]}} is a Functor, I can use
\texttt{fmap} to turn \texttt{(\textless{}\ 5)\ ::\ Int\ -\textgreater{}\ Bool}
into
\texttt{fmap\ (\textless{}\ 5)\ ::\ (Reader\ {[}a{]})\ Int\ -\textgreater{}\ (Reader\ {[}a{]})\ Bool}!

\textasciitilde{}\textasciitilde{}\textasciitilde{}haskell -\/- source:
https://github.com/mstksg/inCode/tree/master/code-samples/inside/reader.hs\#L34-L38
-\/- interactive: https://www.fpcomplete.com/user/jle/inside-my-world
futureShorterThan :: Int -\textgreater{} (Reader {[}a{]}) Bool futureShorterThan
n = fmap (\textless{} n) futureLength

futureShorterThan5 :: (Reader {[}a{]}) Bool futureShorterThan5 =
futureShorterThan 5 \textasciitilde{}\textasciitilde{}\textasciitilde{}

\textasciitilde{}\textasciitilde{}\textasciitilde{}haskell ghci\textgreater{}
runReader futureShorterThan5 {[}1,2,3{]} True ghci\textgreater{} runReader
(futureShorterThan 3) {[}1,2,3,4{]} False
\textasciitilde{}\textasciitilde{}\textasciitilde{}

And voilà, we have a future \texttt{Bool}. We turned an
\texttt{Int\ -\textgreater{}\ Bool} into a function that takes a future
\texttt{Int} and returns a future \texttt{Bool}. We \emph{applied
\texttt{(\textless{}\ 5)} to our future length}, to get a future \texttt{Bool}
telling us if that length is less than 5.

So \texttt{futureShorterThan} is a function that takes an \texttt{Int} and turns
it into a future \texttt{Bool}. Let's go...deeper. What if I wanted to apply
\texttt{futureShorterThan} to a \emph{future} \texttt{Int}? To \emph{still} get
a future \texttt{Bool}?

I can't apply \texttt{futureShorterThan} to a future \texttt{Int} straight-up,
because it only takes \texttt{Int}. Boo! But, wait -\/-\/-
\texttt{Reader\ {[}Int{]}} is a Monad, so that means I can take the
\texttt{Int\ -\textgreater{}\ (Reader\ {[}a{]})\ Bool} and turn it into a
\texttt{(Reader\ {[}a{]})\ Int\ -\textgreater{}\ (Reader\ {[}a{]})\ Bool} using
\texttt{(=\textless{}\textless{})}!

Using \texttt{(=\textless{}\textless{})}, we turned a function from \texttt{Int}
to a future \texttt{Bool} to a function from a future \texttt{Int} to a future
\texttt{Bool}.

\textasciitilde{}\textasciitilde{}\textasciitilde{}haskell futureShorterThan ::
Int -\textgreater{} (Reader {[}a{]}) Bool (=\textless{}\textless{})
futureShorterThan :: (Reader {[}a{]}) Int -\textgreater{} (Reader {[}a{]}) Bool
\textasciitilde{}\textasciitilde{}\textasciitilde{}

Hm. Let's try this out on a future \texttt{Int} we have...we can use
\texttt{futureHead\ ::\ (Reader\ {[}Int{]})\ Int}.

\textasciitilde{}\textasciitilde{}\textasciitilde{}haskell -\/- source:
https://github.com/mstksg/inCode/tree/master/code-samples/inside/reader.hs\#L40-L41
-\/- interactive: https://www.fpcomplete.com/user/jle/inside-my-world
futureShorterThanHead :: (Reader {[}Int{]}) Bool futureShorterThanHead =
futureShorterThan =\textless{}\textless{} futureHead
\textasciitilde{}\textasciitilde{}\textasciitilde{}

So, we are applying \texttt{futureShorterThan} to the \texttt{Int} we got from
\texttt{futureHead}. And so we get a future \texttt{Bool} that tells us if that
future \texttt{Int} we got from the input list is shorter than the input list.

\textasciitilde{}\textasciitilde{}\textasciitilde{}haskell ghci\textgreater{}
runReader futureShorterThanHead {[}1,2,3{]} False ghci\textgreater{} runReader
futureShorterThanHead {[}5,2,3{]} True
\textasciitilde{}\textasciitilde{}\textasciitilde{}

Neat!

Now, we can live in a world of "future values", and now use all of our "normal"
functions on future values!

In another language, we might have had to do this complicated dance of "forcing"
futures to get the value (to "exit" the world of future values), applying
functions to that value, and going "back into" the future.

But now, we \emph{don't have to be scared of future values}. We can work with
them just as well as if they were normal values, and "leave" them as futures the
entire time, without ever forcing them until when we really need to, and only
force them \emph{once}.

Who said futures were complicated, anyway?

\subsection{The world of "IO"}

(The source code for this section is
\href{https://github.com/mstksg/inCode/tree/master/code-samples/inside/io.hs}{also
available online} for you to play with!)

And now we go into the most infamous of Haskell worlds, \texttt{IO}.

\texttt{IO} is "kind of like" our \texttt{Reader\ r} world -\/-\/- an
\texttt{IO\ Int} is an \texttt{Int} that \emph{doesn't yet exist}...but that
will be computed by a CPU/computer \emph{when a CPU executes it}.

In this sense, an \texttt{IO\ Int} is kind of like a little packet of Assembly
or C code -\/-\/- it contains instructions (assembly commands, machine language)
for a computer to do this and that and eventually produce an \texttt{Int}. An
\texttt{IO\ String} could, for example, be a little packet of C code that reads
a file and outputs the contents of that file. The \texttt{String} doesn't exist
yet -\/-\/- but it will, once the computer executes those commands.

If you've ever used a Unix operating system, there is a shell command
\texttt{ls} that lists the contents of a directory. The actual \texttt{ls}
program is kind of like an \texttt{IO\ {[}FilePath{]}}. The
\texttt{{[}FilePath{]}} does not "exist" inside" \texttt{ls} -\/-\/- rather,
\texttt{ls} is a program that promises a list of \texttt{FilePath}s when it is
executed by the computer or interpreter.

So an \texttt{IO\ String} doesn't "contain" a \texttt{String} -\/-\/- it is a
program that \emph{promises} a \texttt{String} in the future, when a computer
eventually executes it,{[}\^{}exitio{]}.

One common IO object we are given is \texttt{getLine\ ::\ IO\ String}.
\texttt{getLine} is kind of like the Unix program \texttt{cat} -\/-\/- it
promises a \texttt{String}, and it gets that \texttt{String} by taking in from
standard input. That is, it is a program that, when executed by a computer,
pulls a line from stdin, and returns that as the \texttt{String} it promises.
\texttt{getLine} contains instructions for a computer to get a \texttt{String}
from stdin. A future/promised \texttt{String}.

We want to apply \texttt{length\ ::\ String\ -\textgreater{}\ Int} to that
future/promised \texttt{String}, to get us a future/promised \texttt{Int}.
Again, we can't apply \texttt{length} to \texttt{getLine} directly -\/-\/- but
because \texttt{IO} is a Functor, we can use
\texttt{fmap\ length\ ::\ IO\ String\ -\textgreater{}\ IO\ Int}.

\textasciitilde{}\textasciitilde{}\textasciitilde{}haskell getLine :: IO String
length :: String -\textgreater{} Int fmap length :: IO String -\textgreater{} IO
Int fmap length getLine :: IO Int
\textasciitilde{}\textasciitilde{}\textasciitilde{}

Neat!

We had a function that only worked on \texttt{String}, but we made it work the
"future/promised" \texttt{String} of \texttt{IO\ String}

Let's look at a function returning an IO action \texttt{wc}, which takes a
filename and returns a program that, when executed, promises an \texttt{Int}
-\/-\/- the number of lines in that file.

\textasciitilde{}\textasciitilde{}\textasciitilde{}haskell -\/- source:
https://github.com/mstksg/inCode/tree/master/code-samples/inside/io.hs\#L19-L19
-\/- interactive: https://www.fpcomplete.com/user/jle/inside-my-world wc ::
String -\textgreater{} IO Int
\textasciitilde{}\textasciitilde{}\textasciitilde{}

So \texttt{wc\ "file.txt"} would evaluate to a computation that, when executed
by a computer, produces an \texttt{Int} (by loading the file from disk using
system calls, reading it, and counting the lines).

\texttt{wc} is a function that takes a (non-future, normal) \texttt{String}.

But what if we wanted to apply \texttt{wc} to \texttt{getLine}, the
\texttt{IO\ String} we had? We want to apply \texttt{wc} to that "future
\texttt{String}". We can't apply it directly. We want to turn our
\texttt{String\ -\textgreater{}\ IO\ Int} into an
\texttt{IO\ String\ -\textgreater{}\ IO\ Int}.

Luckily, \texttt{IO} is a Monad, so we have \texttt{(=\textless{}\textless{})}
at our disposal.

\textasciitilde{}\textasciitilde{}\textasciitilde{}haskell getLine :: IO String
wc :: String -\textgreater{} IO Int (=\textless{}\textless{}) wc :: IO String
-\textgreater{} IO Int wc =\textless{}\textless{} getLine :: IO Int
\textasciitilde{}\textasciitilde{}\textasciitilde{}

Neat!

What does \texttt{wc\ =\textless{}\textless{}\ getLine} do, as a program? How
does it compute that \texttt{Int}?

Conceptually, it all sort of "makes sense" if you look at it from a high level
view. \texttt{getLine} is an \texttt{IO\ String} -\/-\/- a future
\texttt{String}. \texttt{wc} takes a \texttt{String} and returns a future
\texttt{Int}. If we "applied \texttt{wc} to \texttt{getLine}", we would be
applying \texttt{wc} to that future \texttt{String}, to get a future
\texttt{Int}.

And so there we have it -\/-\/- we \emph{don't ever have to} actually work
"directly" with computed values \texttt{a} that are received from IO. All we
ever have to do is work with \texttt{IO\ a}, and we can use \emph{all of our
normal functions} on that \texttt{IO\ a}, as if they were normal \texttt{a}s.

In that way, we don't have to be scared of working with "future computable
values" -\/-\/- we can use all of our normal tools on them!

\section{Even more}

There are lots of other worlds besides just \texttt{Maybe}, \texttt{Reader\ r},
and \texttt{IO}. Each one comes with their own unique
semantics/meanings/contexts, and their own answer for what \texttt{fmap} and
\texttt{(=\textless{}\textless{})} are supposed to "mean".

Here are some others -\/-\/- with brief descriptions.

\begin{enumerate}
\item
  The world of \texttt{Either\ e}, which is like \texttt{Maybe} in which things
  may or may not be there. But in \texttt{Either\ e}, when things aren't there,
  they come with a reason why they are not (of type \texttt{e}).
\item
  The world of \texttt{{[}{]}}, where things are the results of computations
  that have ambiguous answers. I wrote a
  \href{http://blog.jle.im/entries/series/+monadplus-success-failure-monads}{series
  of posts} on this :)
\item
  The world of \texttt{State\ s}, which is a world of future things awaiting an
  \texttt{s}, which modify the \texttt{s} in the process.
\item
  The world of \texttt{Parser}, which is a world of things that an input string
  will be parsed into.
\end{enumerate}

There are many more! The great thing about Haskell is that with \texttt{fmap}
and \texttt{(=\textless{}\textless{})}, it is easy to work with values inside
these worlds with all of your normal functions, without any real extra effort.
Normal functions, normal values, values inside worlds...we have them all at our
disposal.

Haskell lets me stay \emph{inside my world}!

\section{Final Notes}

For some further reading, Gabriel Gonzalez's
\href{http://www.haskellforall.com/2012/09/the-functor-design-pattern.html}{"Functor
Design Pattern"} post covers a similar concept for people more familiar with
Haskell and explains it more elegantly than I ever could have.

Don't forget as you're reading and moving on that it's not correct to say
"Functors are worlds", or "Monads are worlds". As I mentioned before in an
aside, Monads aren't "anything" other than the functions and the laws. Rather,
if we look at \texttt{Maybe}, etc. as a "world", then \emph{having a Monad
interface/instance} allows us to do cool things with that world.

Feel free to again
\href{https://github.com/mstksg/inCode/tree/master/code-samples/inside}{play
around with} the code used here and load it in ghci yourself!

Experienced readers might have noted an unconventional omission of "Applicative
Functors", which (since 2008-ish) traditionally goes somewhere in between the
section on Functor and the section on Monad. Applicative Functors, in this
context, are handy in that they let you combine two values in worlds together;
that is, if you have a \texttt{Maybe\ a} and a \texttt{Maybe\ b}, it allows you
to use an \texttt{a\ -\textgreater{}\ b\ -\textgreater{}\ c} to "squash" them
into a \texttt{Maybe\ c}. This squashing action is called \texttt{liftA2}.

As always, if you have any questions, leave them in the comments, or come find
me on freenode's \#haskell -\/-\/- I go by \emph{jle`} :)

(Special thanks to c\_wraith and rhaps0dy for their time reviewing this post)

\end{document}
