\documentclass[]{article}
\usepackage{lmodern}
\usepackage{amssymb,amsmath}
\usepackage{ifxetex,ifluatex}
\usepackage{fixltx2e} % provides \textsubscript
\ifnum 0\ifxetex 1\fi\ifluatex 1\fi=0 % if pdftex
  \usepackage[T1]{fontenc}
  \usepackage[utf8]{inputenc}
\else % if luatex or xelatex
  \ifxetex
    \usepackage{mathspec}
    \usepackage{xltxtra,xunicode}
  \else
    \usepackage{fontspec}
  \fi
  \defaultfontfeatures{Mapping=tex-text,Scale=MatchLowercase}
  \newcommand{\euro}{€}
\fi
% use upquote if available, for straight quotes in verbatim environments
\IfFileExists{upquote.sty}{\usepackage{upquote}}{}
% use microtype if available
\IfFileExists{microtype.sty}{\usepackage{microtype}}{}
\usepackage[margin=1in]{geometry}
\usepackage{color}
\usepackage{fancyvrb}
\newcommand{\VerbBar}{|}
\newcommand{\VERB}{\Verb[commandchars=\\\{\}]}
\DefineVerbatimEnvironment{Highlighting}{Verbatim}{commandchars=\\\{\}}
% Add ',fontsize=\small' for more characters per line
\newenvironment{Shaded}{}{}
\newcommand{\AlertTok}[1]{\textcolor[rgb]{1.00,0.00,0.00}{\textbf{#1}}}
\newcommand{\AnnotationTok}[1]{\textcolor[rgb]{0.38,0.63,0.69}{\textbf{\textit{#1}}}}
\newcommand{\AttributeTok}[1]{\textcolor[rgb]{0.49,0.56,0.16}{#1}}
\newcommand{\BaseNTok}[1]{\textcolor[rgb]{0.25,0.63,0.44}{#1}}
\newcommand{\BuiltInTok}[1]{#1}
\newcommand{\CharTok}[1]{\textcolor[rgb]{0.25,0.44,0.63}{#1}}
\newcommand{\CommentTok}[1]{\textcolor[rgb]{0.38,0.63,0.69}{\textit{#1}}}
\newcommand{\CommentVarTok}[1]{\textcolor[rgb]{0.38,0.63,0.69}{\textbf{\textit{#1}}}}
\newcommand{\ConstantTok}[1]{\textcolor[rgb]{0.53,0.00,0.00}{#1}}
\newcommand{\ControlFlowTok}[1]{\textcolor[rgb]{0.00,0.44,0.13}{\textbf{#1}}}
\newcommand{\DataTypeTok}[1]{\textcolor[rgb]{0.56,0.13,0.00}{#1}}
\newcommand{\DecValTok}[1]{\textcolor[rgb]{0.25,0.63,0.44}{#1}}
\newcommand{\DocumentationTok}[1]{\textcolor[rgb]{0.73,0.13,0.13}{\textit{#1}}}
\newcommand{\ErrorTok}[1]{\textcolor[rgb]{1.00,0.00,0.00}{\textbf{#1}}}
\newcommand{\ExtensionTok}[1]{#1}
\newcommand{\FloatTok}[1]{\textcolor[rgb]{0.25,0.63,0.44}{#1}}
\newcommand{\FunctionTok}[1]{\textcolor[rgb]{0.02,0.16,0.49}{#1}}
\newcommand{\ImportTok}[1]{#1}
\newcommand{\InformationTok}[1]{\textcolor[rgb]{0.38,0.63,0.69}{\textbf{\textit{#1}}}}
\newcommand{\KeywordTok}[1]{\textcolor[rgb]{0.00,0.44,0.13}{\textbf{#1}}}
\newcommand{\NormalTok}[1]{#1}
\newcommand{\OperatorTok}[1]{\textcolor[rgb]{0.40,0.40,0.40}{#1}}
\newcommand{\OtherTok}[1]{\textcolor[rgb]{0.00,0.44,0.13}{#1}}
\newcommand{\PreprocessorTok}[1]{\textcolor[rgb]{0.74,0.48,0.00}{#1}}
\newcommand{\RegionMarkerTok}[1]{#1}
\newcommand{\SpecialCharTok}[1]{\textcolor[rgb]{0.25,0.44,0.63}{#1}}
\newcommand{\SpecialStringTok}[1]{\textcolor[rgb]{0.73,0.40,0.53}{#1}}
\newcommand{\StringTok}[1]{\textcolor[rgb]{0.25,0.44,0.63}{#1}}
\newcommand{\VariableTok}[1]{\textcolor[rgb]{0.10,0.09,0.49}{#1}}
\newcommand{\VerbatimStringTok}[1]{\textcolor[rgb]{0.25,0.44,0.63}{#1}}
\newcommand{\WarningTok}[1]{\textcolor[rgb]{0.38,0.63,0.69}{\textbf{\textit{#1}}}}
\ifxetex
  \usepackage[setpagesize=false, % page size defined by xetex
              unicode=false, % unicode breaks when used with xetex
              xetex]{hyperref}
\else
  \usepackage[unicode=true]{hyperref}
\fi
\hypersetup{breaklinks=true,
            bookmarks=true,
            pdfauthor={Justin Le},
            pdftitle={Abstract Validating Forms with Free Applicative/Alternative},
            colorlinks=true,
            citecolor=blue,
            urlcolor=blue,
            linkcolor=magenta,
            pdfborder={0 0 0}}
\urlstyle{same}  % don't use monospace font for urls
% Make links footnotes instead of hotlinks:
\renewcommand{\href}[2]{#2\footnote{\url{#1}}}
\setlength{\parindent}{0pt}
\setlength{\parskip}{6pt plus 2pt minus 1pt}
\setlength{\emergencystretch}{3em}  % prevent overfull lines
\setcounter{secnumdepth}{0}

\title{Abstract Validating Forms with Free Applicative/Alternative}
\author{Justin Le}

\begin{document}
\maketitle

\emph{Originally posted on
\textbf{\href{https://blog.jle.im/entry/forms-with-free-applicative-alternative.html}{in
Code}}.}

One tool I've been finding myself using a lot recently is the \emph{Free
Applicative} (and \emph{Free Alternative}), from the
\emph{\href{https://hackage.haskell.org/package/free}{free}} package.

Free Monads are great, and they're often used to implement the ``interpreter
pattern'' (although I personally prefer
\emph{\href{https://hackage.haskell.org/package/operational}{operational}}, as I
wrote about in a
\href{https://blog.jle.im/entry/interpreters-a-la-carte-duet.html}{previous blog
post}, for that design pattern). However, Free Applicatives are really a
completely different type of thing, and the use cases for each are pretty
disjoint.

If I had to make a general statement, I'll say that free monads are especially
good at representing the idea of abstract \emph{sequential} generators
(sequences that are chained dependently one after the other), and that free
applicatives are especially good at representing the idea of abstract
\emph{parallel} generators (things operating in parallel without any
interconnected data dependences).

For this post, I'll be talking about using the Free Applicative \texttt{Ap} (and
the Free Alternative, \texttt{Alt}) with an abstract representation of a form
element in order to generate an abstract representation of a validating form,
and leveraging this representation to realize these forms in terminal IO,
JSON/YAML, PDF documents, and even on the browser using \emph{ghcjs} and
\emph{\href{https://hackage.haskell.org/package/miso}{miso}}.

\hypertarget{overview}{%
\section{Overview}\label{overview}}

The general approach to utilizing the Free Applicative for this type of
application is to start with some Functor \texttt{F} (\texttt{F\ a} represents
the act of generating a value of type \texttt{a}). Once you throw \texttt{F}
into \texttt{Ap} to get \texttt{Ap\ F}, you now are able to \emph{combine
\texttt{F}s in parallel} with \texttt{\textless{}\$\textgreater{}},
\texttt{\textless{}*\textgreater{}}, \texttt{liftA2}, \texttt{sequence},
\texttt{traverse}, etc., even though \texttt{F} normally could not support such
combinations. Then, finally, you have the ability to provide a concrete
generator function
\texttt{forall\ a.\ Applicative\ f\ =\textgreater{}\ F\ a\ -\textgreater{}\ f\ a}
(given \texttt{F\ a}, return an actual generator of \texttt{a}s in some
\texttt{Applicative}), and the magic of the Free Applicative will go in and
actually run all of your combined \texttt{F} actions ``in parallel''. The trick
is that, with the same value of \texttt{Ap\ F\ a}, you can \emph{run multiple
different concrete generators} on it, so you can realize \texttt{Ap\ F} in
multiple different contexts and situations, adapting it for whatever you need.

So, in our case, we're going to be making a Functor representing a form element:

\begin{Shaded}
\begin{Highlighting}[]
\KeywordTok{data} \DataTypeTok{FormElem}\NormalTok{ a}
\end{Highlighting}
\end{Shaded}

Where a \texttt{FormElem\ a} represents a \emph{single form element} producing
an \texttt{a}. A \texttt{FormElem\ Int}, for instance, will represent a single
form element producing an \texttt{Int}.

Then, we can create, using
\href{https://hackage.haskell.org/package/free/docs/Control-Applicative-Free.html}{Ap}:

\begin{Shaded}
\begin{Highlighting}[]
\KeywordTok{type} \DataTypeTok{Form} \FunctionTok{=} \DataTypeTok{Ap} \DataTypeTok{FormElem}
\end{Highlighting}
\end{Shaded}

And now we have a type where \texttt{Form\ a} is \emph{whole form with multiple
elements} that all work together to produce a value of type \texttt{a}!

For example, if we had \texttt{intElem\ ::\ FormElem\ Int}, then
\texttt{liftAp\ intElem\ ::\ Form\ Int}, a single-item form that makes an
\texttt{Int}:

\begin{Shaded}
\begin{Highlighting}[]
\OtherTok{intElem ::} \DataTypeTok{FormElem} \DataTypeTok{Int}

\OtherTok{intForm ::} \DataTypeTok{Form} \DataTypeTok{Int}
\NormalTok{intForm }\FunctionTok{=}\NormalTok{ liftAp intElem}
\end{Highlighting}
\end{Shaded}

We can now do all of our Applicativey stuff with it, to generate, for instance,
a form with two items that produces their sum:

\begin{Shaded}
\begin{Highlighting}[]
\CommentTok{-- | A form with two elements, whose overal result is the sum of the two}
\CommentTok{-- element's inputs}
\OtherTok{addingForm ::} \DataTypeTok{Form} \DataTypeTok{Int}
\NormalTok{addingForm }\FunctionTok{=}\NormalTok{ (}\FunctionTok{+}\NormalTok{) }\FunctionTok{<$>}\NormalTok{ intForm }\FunctionTok{<*>}\NormalTok{ intForm}
\end{Highlighting}
\end{Shaded}

Or, we can even generate a form with many `Int' elements, and produce a list of
all of their items:

\begin{Shaded}
\begin{Highlighting}[]
\CommentTok{-- | A form with five elements, whose result is a list of all of their inputs}
\OtherTok{bunchaInts ::} \DataTypeTok{Form}\NormalTok{ [}\DataTypeTok{Int}\NormalTok{]}
\NormalTok{bunchaInts }\FunctionTok{=}\NormalTok{ replicateM }\DecValTok{5}\NormalTok{ intForm}
\end{Highlighting}
\end{Shaded}

Of course, in real life, forms usually have \emph{Alternative} instances, which
allows you to use \texttt{\textless{}\textbar{}\textgreater{}} to ``chose'' a
result between potentially invalid entries, and also create ``optional'' entries
for free using \texttt{optional}. To do that, we actually use the \emph{Free
Alternative},
\emph{\href{https://hackage.haskell.org/package/free/docs/Control-Alternative-Free-Final.html}{Alt}},
instead:

\begin{Shaded}
\begin{Highlighting}[]
\KeywordTok{type} \DataTypeTok{Form} \FunctionTok{=} \DataTypeTok{Alt} \DataTypeTok{FormElem}

\OtherTok{intForm ::} \DataTypeTok{Form} \DataTypeTok{Int}
\NormalTok{intForm }\FunctionTok{=}\NormalTok{ liftAlt intElem}
\end{Highlighting}
\end{Shaded}

And now we get the ability to represent forms with multiple options with
\texttt{\textless{}\textbar{}\textgreater{}}:

\begin{Shaded}
\begin{Highlighting}[]
\OtherTok{boolForm ::} \DataTypeTok{Form} \DataTypeTok{Bool}

\CommentTok{-- | A form with two elements (one producing an 'Int' and one producing a}
\CommentTok{-- 'Bool'), where the result is 'Either Int Bool' -- a "the first or the}
\CommentTok{-- second".}
\OtherTok{eitherInt ::} \DataTypeTok{Form}\NormalTok{ (}\DataTypeTok{Either} \DataTypeTok{Int} \DataTypeTok{Bool}\NormalTok{)}
\NormalTok{eitherInt }\FunctionTok{=}\NormalTok{ (}\DataTypeTok{Left} \FunctionTok{<$>}\NormalTok{ intForm) }\FunctionTok{<|>}\NormalTok{ (}\DataTypeTok{Right} \FunctionTok{<$>}\NormalTok{ boolForm)}
\end{Highlighting}
\end{Shaded}

From multiple forms, with \texttt{choice}:

\begin{Shaded}
\begin{Highlighting}[]
\OtherTok{stringForm ::} \DataTypeTok{Form} \DataTypeTok{String}

\CommentTok{-- | A form with five elements -- an 'Int' element, a 'Bool' element, and three}
\CommentTok{-- 'String' elements.  The result is the first of the four options to succeed.}
\OtherTok{oneOfMany ::} \DataTypeTok{Form} \DataTypeTok{String}
\NormalTok{oneOfMany }\FunctionTok{=}\NormalTok{ choice [ show }\FunctionTok{<$>}\NormalTok{ intForm}
\NormalTok{                   , show }\FunctionTok{<$>}\NormalTok{ boolForm}
\NormalTok{                   , stringForm}
\NormalTok{                   , (}\FunctionTok{++}\NormalTok{) }\FunctionTok{<$>}\NormalTok{ stringForm }\FunctionTok{<*>}\NormalTok{ stringForm}
\NormalTok{                   ]}
\end{Highlighting}
\end{Shaded}

And create ``optional'' entries:

\begin{Shaded}
\begin{Highlighting}[]
\CommentTok{-- | A form with a single 'Int' element that returns a 'Maybe Int', because}
\CommentTok{-- it's optional.}
\OtherTok{optionalInt ::} \DataTypeTok{Form}\NormalTok{ (}\DataTypeTok{Maybe} \DataTypeTok{Int}\NormalTok{)}
\NormalTok{optionalInt }\FunctionTok{=}\NormalTok{ optional intForm}
\end{Highlighting}
\end{Shaded}

We get all of these capabilities \emph{for free}! All we did was \emph{define a
single form element}. Our form element type doesn't have have any concept of
combining with other elements or of being able to choose between different
elements or of having optional results. Then, \texttt{Alt} gives us the ability
to combine them with
\texttt{\textless{}*\textgreater{}}/\texttt{\textless{}\$\textgreater{}}, create
optional form items with \texttt{optional}, and create multiple form options
with \texttt{\textless{}\textbar{}\textgreater{}}!

\hypertarget{our-types}{%
\section{Our Types}\label{our-types}}

\hypertarget{form-element}{%
\subsection{Form Element}\label{form-element}}

Our form elements will all have monomorphic base element paired with a
``parser'', a description, and an identifier.

To start off, we'll make a GADT representing a concrete element, as well as what
is required to actually represent it for the user to interact with:

\begin{Shaded}
\begin{Highlighting}[]
\CommentTok{-- source: https://github.com/mstksg/inCode/tree/master/code-samples/free-applicative-forms/Form.hs#L19-L27}

\KeywordTok{data} \DataTypeTok{Elem}\OtherTok{ ::} \DataTypeTok{Type} \OtherTok{->} \DataTypeTok{Type} \KeywordTok{where}
    \CommentTok{-- | Text field}
    \DataTypeTok{EText}\OtherTok{   ::} \DataTypeTok{Elem} \DataTypeTok{String}
    \CommentTok{-- | Numberic field}
    \DataTypeTok{ENumber}\OtherTok{ ::} \DataTypeTok{Elem} \DataTypeTok{Scientific}
    \CommentTok{-- | Select box, with list of options to display}
    \DataTypeTok{ESelect}\OtherTok{ ::}\NormalTok{ [}\DataTypeTok{String}\NormalTok{] }\OtherTok{->} \DataTypeTok{Elem}\NormalTok{ (}\DataTypeTok{Maybe} \DataTypeTok{Int}\NormalTok{)}
    \CommentTok{-- | Check box, with the labels to attach to on/off states}
    \DataTypeTok{ECheck}\OtherTok{  ::} \DataTypeTok{String} \OtherTok{->} \DataTypeTok{String} \OtherTok{->} \DataTypeTok{Elem} \DataTypeTok{Bool}
\end{Highlighting}
\end{Shaded}

We tag the \texttt{Elem} with a type representing the element's \emph{native}
output. So an \texttt{Elem\ String} is an element that natively/naively outputs
a \texttt{String} (like a text input), an \texttt{Elem\ Bool} is an element that
natively/naively outputs a \texttt{Bool} (like check box), etc.

Each native \texttt{Elem} contains the information necessary to render it -- so
we have:

\begin{itemize}
\tightlist
\item
  \texttt{EText}, natively outputting a \texttt{String}.
\item
  \texttt{ENumber}, natively outputting a number (\texttt{Scientific}).
\item
  \texttt{ESelect}, a drop-down menu, a list of strings to show as items,
  natively outputting \texttt{Maybe\ Int} (``maybe'' a selected index)
\item
  \texttt{ECheck}, a check box containing labels for its on and off positions,
  natively outputting a \texttt{Bool}.
\end{itemize}

And then we our representation of a generator for a single form element output:

\begin{Shaded}
\begin{Highlighting}[]
\CommentTok{-- source: https://github.com/mstksg/inCode/tree/master/code-samples/free-applicative-forms/Form.hs#L29-L35}

\KeywordTok{data} \DataTypeTok{FormElem}\OtherTok{ ::} \DataTypeTok{Type} \OtherTok{->} \DataTypeTok{Type} \KeywordTok{where}
    \DataTypeTok{FE}\OtherTok{ ::}\NormalTok{ \{}\OtherTok{ feElem  ::} \DataTypeTok{Elem}\NormalTok{ b}
\NormalTok{          ,}\OtherTok{ feParse ::}\NormalTok{ b }\OtherTok{->} \DataTypeTok{Either} \DataTypeTok{String}\NormalTok{ a}
\NormalTok{          ,}\OtherTok{ feDesc  ::} \DataTypeTok{String}
\NormalTok{          ,}\OtherTok{ feIdent ::} \DataTypeTok{String}
\NormalTok{          \}}
        \OtherTok{->} \DataTypeTok{FormElem}\NormalTok{ a}
\end{Highlighting}
\end{Shaded}

A \texttt{FormElem\ a} will contain:

\begin{itemize}
\tightlist
\item
  The \texttt{Elem\ b} representing the actual native form element, with
  information required to render it.
\item
  A function \texttt{b\ -\textgreater{}\ Either\ String\ a}, which lets you
  \emph{parse} the element's native output into an \texttt{a} (what the
  \texttt{FormElem} outputs), with a potential \texttt{String} error message
\item
  A display name and identifier, which we will use to describe the element and
  as a part of the JSON schema for our JSON backend.
\end{itemize}

\hypertarget{form}{%
\subsection{Form}\label{form}}

We now have a functor, \texttt{FormElem}, that is a single form element and all
of its decorations. Now, for the magic --- to make a full \texttt{Form} type,
it's just:

\begin{Shaded}
\begin{Highlighting}[]
\CommentTok{-- source: https://github.com/mstksg/inCode/tree/master/code-samples/free-applicative-forms/Form.hs#L39-L39}

\KeywordTok{type} \DataTypeTok{Form} \FunctionTok{=} \DataTypeTok{Alt} \DataTypeTok{FormElem}
\end{Highlighting}
\end{Shaded}

Ta dah!

And just like that, we now have functions like:

\begin{Shaded}
\begin{Highlighting}[]
\CommentTok{-- | Map over the results of a Form}
\NormalTok{fmap}\OtherTok{   ::}\NormalTok{ (a }\OtherTok{->}\NormalTok{ b) }\OtherTok{->} \DataTypeTok{Form}\NormalTok{ a }\OtherTok{->} \DataTypeTok{Form}\NormalTok{ b}

\CommentTok{-- | Combine two forms together, and merge their results with a combining}
\CommentTok{-- function}
\OtherTok{liftA2 ::}\NormalTok{ (a }\OtherTok{->}\NormalTok{ b }\OtherTok{->}\NormalTok{ c) }\OtherTok{->} \DataTypeTok{Form}\NormalTok{ a }\OtherTok{->} \DataTypeTok{Form}\NormalTok{ b }\OtherTok{->} \DataTypeTok{Form}\NormalTok{ c}

\CommentTok{-- | Combine all of the forms in a list to create a single form, whose result}
\CommentTok{-- is the collection of all of their results}
\NormalTok{sequence}\OtherTok{ ::}\NormalTok{ [}\DataTypeTok{Form}\NormalTok{ a] }\OtherTok{->} \DataTypeTok{Form}\NormalTok{ [a]}

\CommentTok{-- | Duplicate a form multiple times, creating a new form whose reuslt is a}
\CommentTok{-- collection of all of their results}
\OtherTok{replicateM ::} \DataTypeTok{Int} \OtherTok{->} \DataTypeTok{Form}\NormalTok{ a }\OtherTok{->} \DataTypeTok{Form}\NormalTok{ [a]}

\CommentTok{-- | Combine two forms together to create a final form whose result picks from}
\CommentTok{-- the two input forms}
\OtherTok{(<|>) ::} \DataTypeTok{Form}\NormalTok{ a }\OtherTok{->} \DataTypeTok{Form}\NormalTok{ a }\OtherTok{->} \DataTypeTok{Form}\NormalTok{ a}

\CommentTok{-- | Combine several forms together to create a final form whose result picks}
\NormalTok{from one }\KeywordTok{of}\NormalTok{ the several options}
\OtherTok{choice ::}\NormalTok{ [}\DataTypeTok{Form}\NormalTok{ a] }\OtherTok{->} \DataTypeTok{Form}\NormalTok{ a}

\CommentTok{-- | Turn a form into an optional form}
\OtherTok{optional ::} \DataTypeTok{Form}\NormalTok{ a }\OtherTok{->} \DataTypeTok{Form}\NormalTok{ (}\DataTypeTok{Maybe}\NormalTok{ a)}
\end{Highlighting}
\end{Shaded}

All that for free. Neat!

Note -- it is very important to use \texttt{Alt} from
\emph{Control.Alternative.Free.Final}, and \textbf{not} the one from
\emph{Control.Alternative.Free}. The \emph{Control.Alternative.Free}
\texttt{Alt} is broken for our use case, because it normalizes against
\href{https://stackoverflow.com/q/45647253/292731}{an extra Alternative law}
that forms do not obey.

\hypertarget{making-real-forms}{%
\section{Making Real Forms}\label{making-real-forms}}

\hypertarget{primitive-forms}{%
\subsection{Primitive Forms}\label{primitive-forms}}

Let's create a set of primitive \texttt{Form}s, which we will use to build up
our more complex ones.

First, a \texttt{Form\ String} with a single text box:

\begin{Shaded}
\begin{Highlighting}[]
\CommentTok{-- source: https://github.com/mstksg/inCode/tree/master/code-samples/free-applicative-forms/Form.hs#L41-L50}

\NormalTok{stringInput}
\OtherTok{    ::} \DataTypeTok{String}           \CommentTok{-- ^ description}
    \OtherTok{->} \DataTypeTok{String}           \CommentTok{-- ^ identifier}
    \OtherTok{->} \DataTypeTok{Form} \DataTypeTok{String}
\NormalTok{stringInput desc ident }\FunctionTok{=}\NormalTok{ liftAlt }\FunctionTok{$}
    \DataTypeTok{FE}\NormalTok{ \{ feElem  }\FunctionTok{=} \DataTypeTok{EText}
\NormalTok{       , feParse }\FunctionTok{=}\NormalTok{ mfilter (not }\FunctionTok{.}\NormalTok{ null) }\FunctionTok{.} \DataTypeTok{Right}
\NormalTok{       , feDesc  }\FunctionTok{=}\NormalTok{ desc}
\NormalTok{       , feIdent }\FunctionTok{=}\NormalTok{ ident}
\NormalTok{       \}}
\end{Highlighting}
\end{Shaded}

Note that its \texttt{String\ -\textgreater{}\ Either\ String\ String} function
is just \texttt{Right}, since it is always a ``successful'' parse.

And maybe one that is based on a text box, but instead of just outputting the
output \texttt{String}, it parses it into a Haskell value using \texttt{Read}:

\begin{Shaded}
\begin{Highlighting}[]
\CommentTok{-- source: https://github.com/mstksg/inCode/tree/master/code-samples/free-applicative-forms/Form.hs#L52-L63}

\NormalTok{readInput}
\OtherTok{    ::}\NormalTok{ (}\DataTypeTok{Read}\NormalTok{ a, }\DataTypeTok{Show}\NormalTok{ a)}
    \OtherTok{=>} \DataTypeTok{String}           \CommentTok{-- ^ description}
    \OtherTok{->} \DataTypeTok{String}           \CommentTok{-- ^ identifier}
    \OtherTok{->} \DataTypeTok{Form}\NormalTok{ a}
\NormalTok{readInput desc ident }\FunctionTok{=}\NormalTok{ liftAlt }\FunctionTok{$}
    \DataTypeTok{FE}\NormalTok{ \{ feElem  }\FunctionTok{=} \DataTypeTok{EText}
\NormalTok{       , feParse }\FunctionTok{=}\NormalTok{ maybe (}\DataTypeTok{Left}\NormalTok{ (}\StringTok{"Could not parse "} \FunctionTok{++}\NormalTok{ ident)) }\DataTypeTok{Right}
                 \FunctionTok{.}\NormalTok{ readMaybe}
\NormalTok{       , feDesc  }\FunctionTok{=}\NormalTok{ desc}
\NormalTok{       , feIdent }\FunctionTok{=}\NormalTok{ ident}
\NormalTok{       \}}
\end{Highlighting}
\end{Shaded}

Now two ``numerical'' inputs -- one expecting \texttt{Integral} values, and
another expecting floating-point values.

\begin{Shaded}
\begin{Highlighting}[]
\CommentTok{-- source: https://github.com/mstksg/inCode/tree/master/code-samples/free-applicative-forms/Form.hs#L65-L88}

\NormalTok{intInput}
\OtherTok{    ::} \DataTypeTok{Integral}\NormalTok{ a}
    \OtherTok{=>} \DataTypeTok{String}     \CommentTok{-- ^ description}
    \OtherTok{->} \DataTypeTok{String}     \CommentTok{-- ^ identifier}
    \OtherTok{->} \DataTypeTok{Form}\NormalTok{ a}
\NormalTok{intInput desc ident }\FunctionTok{=}\NormalTok{ liftAlt }\FunctionTok{$}
    \DataTypeTok{FE}\NormalTok{ \{ feElem  }\FunctionTok{=} \DataTypeTok{ENumber}
\NormalTok{       , feParse }\FunctionTok{=}\NormalTok{ either (\textbackslash{}_ }\OtherTok{->} \DataTypeTok{Left}\NormalTok{ (ident }\FunctionTok{++} \StringTok{" should be integer"}\NormalTok{)) }\DataTypeTok{Right}
                 \FunctionTok{.}\NormalTok{ floatingOrInteger }\FunctionTok{@}\DataTypeTok{Double}
\NormalTok{       , feDesc  }\FunctionTok{=}\NormalTok{ desc}
\NormalTok{       , feIdent }\FunctionTok{=}\NormalTok{ ident}
\NormalTok{       \}}

\NormalTok{floatInput}
\OtherTok{    ::} \DataTypeTok{RealFloat}\NormalTok{ a}
    \OtherTok{=>} \DataTypeTok{String}           \CommentTok{-- ^ description}
    \OtherTok{->} \DataTypeTok{String}           \CommentTok{-- ^ identifier}
    \OtherTok{->} \DataTypeTok{Form}\NormalTok{ a}
\NormalTok{floatInput desc ident }\FunctionTok{=}\NormalTok{ liftAlt }\FunctionTok{$}
    \DataTypeTok{FE}\NormalTok{ \{ feElem  }\FunctionTok{=} \DataTypeTok{ENumber}
\NormalTok{       , feParse }\FunctionTok{=} \DataTypeTok{Right} \FunctionTok{.}\NormalTok{ realToFrac}
\NormalTok{       , feDesc  }\FunctionTok{=}\NormalTok{ desc}
\NormalTok{       , feIdent }\FunctionTok{=}\NormalTok{ ident}
\NormalTok{       \}}
\end{Highlighting}
\end{Shaded}

A simple drop-down menu element that lets the user pick from a list of items:

\begin{Shaded}
\begin{Highlighting}[]
\CommentTok{-- source: https://github.com/mstksg/inCode/tree/master/code-samples/free-applicative-forms/Form.hs#L90-L99}

\NormalTok{selectInput}
\OtherTok{    ::} \DataTypeTok{Show}\NormalTok{ a}
    \OtherTok{=>} \DataTypeTok{String}           \CommentTok{-- ^ description}
    \OtherTok{->} \DataTypeTok{String}           \CommentTok{-- ^ identifier}
    \OtherTok{->}\NormalTok{ [a]              }\CommentTok{-- ^ options}
    \OtherTok{->} \DataTypeTok{Form}\NormalTok{ a}
\NormalTok{selectInput desc ident opts }\FunctionTok{=}\NormalTok{ liftAlt }\FunctionTok{$}
    \DataTypeTok{FE}\NormalTok{ (}\DataTypeTok{ESelect}\NormalTok{ (show }\FunctionTok{<$>}\NormalTok{ opts)) p desc ident}
  \KeywordTok{where}
\NormalTok{    p }\FunctionTok{=}\NormalTok{ maybe (}\DataTypeTok{Left}\NormalTok{ (}\StringTok{"No selection for "} \FunctionTok{++}\NormalTok{ ident)) (}\DataTypeTok{Right} \FunctionTok{.}\NormalTok{ (opts }\FunctionTok{!!}\NormalTok{))}
\end{Highlighting}
\end{Shaded}

And a simple check box, which outputs one of two items:

\begin{Shaded}
\begin{Highlighting}[]
\CommentTok{-- source: https://github.com/mstksg/inCode/tree/master/code-samples/free-applicative-forms/Form.hs#L101-L114}

\NormalTok{checkInput}
\OtherTok{    ::} \DataTypeTok{Show}\NormalTok{ a}
    \OtherTok{=>} \DataTypeTok{String}           \CommentTok{-- ^ description}
    \OtherTok{->} \DataTypeTok{String}           \CommentTok{-- ^ identifier}
    \OtherTok{->}\NormalTok{ a                }\CommentTok{-- ^ unchecked option}
    \OtherTok{->}\NormalTok{ a                }\CommentTok{-- ^ checked option}
    \OtherTok{->} \DataTypeTok{Form}\NormalTok{ a}
\NormalTok{checkInput desc ident x y }\FunctionTok{=}\NormalTok{ liftAlt }\FunctionTok{$}
    \DataTypeTok{FE}\NormalTok{ (}\DataTypeTok{ECheck}\NormalTok{ (show x) (show y)) p desc ident}
  \KeywordTok{where}
\NormalTok{    p }\FunctionTok{=} \DataTypeTok{Right} \FunctionTok{.}\NormalTok{ bool x y}

\OtherTok{boolInput ::} \DataTypeTok{String} \OtherTok{->} \DataTypeTok{String} \OtherTok{->} \DataTypeTok{Form} \DataTypeTok{Bool}
\NormalTok{boolInput desc ident }\FunctionTok{=}\NormalTok{ checkInput desc ident }\DataTypeTok{False} \DataTypeTok{True}
\end{Highlighting}
\end{Shaded}

\hypertarget{sample-form}{%
\subsection{Sample Form}\label{sample-form}}

To explore this type, let's make a sample form which we will be re-using for the
rest of this post!

We will be making a registration form, a form producing an account, which
contains:

\begin{enumerate}
\def\labelenumi{\arabic{enumi}.}
\tightlist
\item
  A name
\item
  An optional age
\item
  A favorite color (from one of the pre-defined color, or a custom one)
\item
  An account type (normal account, or premium account?)
\end{enumerate}

\begin{Shaded}
\begin{Highlighting}[]
\CommentTok{-- source: https://github.com/mstksg/inCode/tree/master/code-samples/free-applicative-forms/Form.hs#L116-L127}

\KeywordTok{data} \DataTypeTok{AccountType} \FunctionTok{=} \DataTypeTok{Normal} \FunctionTok{|} \DataTypeTok{Premium}
    \KeywordTok{deriving} \DataTypeTok{Show}

\KeywordTok{data} \DataTypeTok{Color} \FunctionTok{=} \DataTypeTok{Red} \FunctionTok{|} \DataTypeTok{Blue} \FunctionTok{|} \DataTypeTok{Orange} \FunctionTok{|} \DataTypeTok{Yellow}
    \KeywordTok{deriving} \DataTypeTok{Show}

\KeywordTok{data} \DataTypeTok{Account} \FunctionTok{=} \DataTypeTok{Acc}\NormalTok{ \{}\OtherTok{ accName     ::} \DataTypeTok{String}
\NormalTok{                   ,}\OtherTok{ accAge      ::} \DataTypeTok{Maybe} \DataTypeTok{Int}
\NormalTok{                   ,}\OtherTok{ accFavColor ::} \DataTypeTok{Either} \DataTypeTok{Color} \DataTypeTok{String}
\NormalTok{                   ,}\OtherTok{ accPremium  ::} \DataTypeTok{AccountType}
\NormalTok{                   \}}
    \KeywordTok{deriving} \DataTypeTok{Show}
\end{Highlighting}
\end{Shaded}

And we'll make the form using Applicative style:

\begin{Shaded}
\begin{Highlighting}[]
\CommentTok{-- source: https://github.com/mstksg/inCode/tree/master/code-samples/free-applicative-forms/Form.hs#L129-L138}

\OtherTok{accountForm ::} \DataTypeTok{Form} \DataTypeTok{Account}
\NormalTok{accountForm }\FunctionTok{=}
    \DataTypeTok{Acc} \FunctionTok{<$>}\NormalTok{ stringInput }\StringTok{"Name"} \StringTok{"name"}
        \FunctionTok{<*>}\NormalTok{ optional (intInput }\StringTok{"Age"} \StringTok{"age"}\NormalTok{)}
        \FunctionTok{<*>}\NormalTok{ (}\DataTypeTok{Left} \FunctionTok{<$>}\NormalTok{ favColor }\FunctionTok{<|>} \DataTypeTok{Right} \FunctionTok{<$>}\NormalTok{ customColor)}
        \FunctionTok{<*>}\NormalTok{ checkInput }\StringTok{"Premium Account"} \StringTok{"premium"} \DataTypeTok{Normal} \DataTypeTok{Premium} 
  \KeywordTok{where}
\NormalTok{    favColor    }\FunctionTok{=}\NormalTok{ selectInput }\StringTok{"Favorite Color"} \StringTok{"fav-color"}
\NormalTok{                    [}\DataTypeTok{Red}\NormalTok{, }\DataTypeTok{Blue}\NormalTok{, }\DataTypeTok{Orange}\NormalTok{, }\DataTypeTok{Yellow}\NormalTok{]}
\NormalTok{    customColor }\FunctionTok{=}\NormalTok{ stringInput }\StringTok{"Custom Color"} \StringTok{"custum-color"}
\end{Highlighting}
\end{Shaded}

If you're feeling fancy, you can also make it using ``Applicative Do'' style,
which makes it a little more flexible if you want to re-arrange the items in the
form (put the ``age'' field before the ``name'' field, etc.)

\begin{Shaded}
\begin{Highlighting}[]
\CommentTok{-- source: https://github.com/mstksg/inCode/tree/master/code-samples/free-applicative-forms/Form.hs#L140-L148}

\OtherTok{accountFormAdo ::} \DataTypeTok{Form} \DataTypeTok{Account}
\NormalTok{accountFormAdo }\FunctionTok{=} \KeywordTok{do}
\NormalTok{    nam }\OtherTok{<-}\NormalTok{ stringInput }\StringTok{"Name"} \StringTok{"name"}
\NormalTok{    age }\OtherTok{<-}\NormalTok{ optional }\FunctionTok{$}\NormalTok{ intInput }\StringTok{"Age"} \StringTok{"age"}
\NormalTok{    col }\OtherTok{<-} \DataTypeTok{Left}  \FunctionTok{<$>}\NormalTok{ selectInput }\StringTok{"Favorite Color"} \StringTok{"fav-color"}
\NormalTok{                       [}\DataTypeTok{Red}\NormalTok{, }\DataTypeTok{Blue}\NormalTok{, }\DataTypeTok{Orange}\NormalTok{, }\DataTypeTok{Yellow}\NormalTok{]}
       \FunctionTok{<|>} \DataTypeTok{Right} \FunctionTok{<$>}\NormalTok{ stringInput }\StringTok{"Custom Color"} \StringTok{"custom-color"}
\NormalTok{    typ }\OtherTok{<-}\NormalTok{ checkInput }\StringTok{"Premium Account"} \StringTok{"premium"} \DataTypeTok{Normal} \DataTypeTok{Premium} 
\NormalTok{    pure (}\DataTypeTok{Acc}\NormalTok{ nam age col typ)}
\end{Highlighting}
\end{Shaded}

Hi, thanks for reading! You can reach me via email at
\href{mailto:justin@jle.im}{\nolinkurl{justin@jle.im}}, or at twitter at
\href{https://twitter.com/mstk}{@mstk}! This post and all others are published
under the \href{https://creativecommons.org/licenses/by-nc-nd/3.0/}{CC-BY-NC-ND
3.0} license. Corrections and edits via pull request are welcome and encouraged
at \href{https://github.com/mstksg/inCode}{the source repository}.

If you feel inclined, or this post was particularly helpful for you, why not
consider \href{https://www.patreon.com/justinle}{supporting me on Patreon}, or a
BTC donation at
\emph{\href{bitcoin:3D7rmAYgbDnp4gp4rf22THsGt74fNucPDU}{3D7rmAYgbDnp4gp4rf22THsGt74fNucPDU}}?
:)

\end{document}
