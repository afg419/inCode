\documentclass[]{article}
\usepackage{lmodern}
\usepackage{amssymb,amsmath}
\usepackage{ifxetex,ifluatex}
\usepackage{fixltx2e} % provides \textsubscript
\ifnum 0\ifxetex 1\fi\ifluatex 1\fi=0 % if pdftex
  \usepackage[T1]{fontenc}
  \usepackage[utf8]{inputenc}
\else % if luatex or xelatex
  \ifxetex
    \usepackage{mathspec}
    \usepackage{xltxtra,xunicode}
  \else
    \usepackage{fontspec}
  \fi
  \defaultfontfeatures{Mapping=tex-text,Scale=MatchLowercase}
  \newcommand{\euro}{€}
\fi
% use upquote if available, for straight quotes in verbatim environments
\IfFileExists{upquote.sty}{\usepackage{upquote}}{}
% use microtype if available
\IfFileExists{microtype.sty}{\usepackage{microtype}}{}
\usepackage[margin=1in]{geometry}
\ifxetex
  \usepackage[setpagesize=false, % page size defined by xetex
              unicode=false, % unicode breaks when used with xetex
              xetex]{hyperref}
\else
  \usepackage[unicode=true]{hyperref}
\fi
\hypersetup{breaklinks=true,
            bookmarks=true,
            pdfauthor={},
            pdftitle={},
            colorlinks=true,
            citecolor=blue,
            urlcolor=blue,
            linkcolor=magenta,
            pdfborder={0 0 0}}
\urlstyle{same}  % don't use monospace font for urls
% Make links footnotes instead of hotlinks:
\renewcommand{\href}[2]{#2\footnote{\url{#1}}}
\setlength{\parindent}{0pt}
\setlength{\parskip}{6pt plus 2pt minus 1pt}
\setlength{\emergencystretch}{3em}  % prevent overfull lines
\setcounter{secnumdepth}{0}


\begin{document}

\% Effectful, Recursive, Real-World Autos: Intro to Machine/Auto Part 3 \%
Justin Le \% February 9, 2015

\emph{Originally posted on
\textbf{\href{https://blog.jle.im/entry/effectful-recursive-real-world-autos-intro-to-machine.html}{in
Code}}.}

Hi! I have to apologize a bit for the long delay; starting grad school and
things like that have made me have to scramble to adjust to the new life. But a
couple of people have asked me to finish up and wrap up this series, and I think
I owe it to them then :) Welcome to the final chapter.

In the
\href{http://blog.jle.im/entry/auto-as-category-applicative-arrow-intro-to-machines}{last
post}, we looked deeper into the Auto type, played around with instancing it as
familiar typeclasses, saw it as a member of the powerful \emph{Category} and
\emph{Arrow} typeclasses, and took advantage of this by composing Autos both
manually and using proc/do notation, and were freed from the murk and mire of
explicit recursion. We observed the special nature of this composition, and saw
some neat properties, like local statefulness.

At this point I consider most of the important concepts about working with
\texttt{Auto} covered, but now, we are going to push this abstraction further,
to the limits of real-world industrial usage. We're going to be exploring
mechanisms for adding effects and, making the plain ol' \texttt{Auto} into
something more rich and featureful. We'll see how to express denotative and
declarative compositions using recursively binded \texttt{Auto}s, and what that
even means. It'll be a trip down several avenues to motivate and see practical
Auto usage.{[}\^{}nofrp{]} Basically, it'll be a "final hurrah".

A fair bit of warning -\/-\/- if the last post is not fresh in your mind, or you
still have some holes, I recommend going back and reading through them again.
This one is going to hit hard and fast :) (Also, it's admittedly kind of long
for a single post, but I didn't want to break things up into two really short
parts.)

As always, feel free to leave a comment if you have any questions, drop by
freenode's \emph{\#haskell}, or find me on
\href{https://twitter.com/mstk}{twitter} :)

All of the code in this post is
\href{http://blog.jle.im/source/code-samples/machines}{available for download}
and to load up into ghci for playing along!

\section{Effectful Stepping}

Recall our original definition of \texttt{Auto\ a\ b} as a newtype wrapper over
a function:

\textasciitilde{}\textasciitilde{}\textasciitilde{}haskell a -\textgreater{} (b,
Auto a b) \textasciitilde{}\textasciitilde{}\textasciitilde{}

This can be read as saying, "feed the \texttt{Auto} an \texttt{a}, and (purely)
get a resulting \texttt{b}, and a 'next stepper'" -\/-\/- the \texttt{b} is the
result, and the \texttt{Auto\ a\ b} contains the information on how to proceed
from then on.

If you've been doing Haskell for any decent amount of time, you can probably
guess what's going to happen next!

Instead of "purely" creating a naked result and a "next step"...we're going to
return it in a context.

\textasciitilde{}\textasciitilde{}\textasciitilde{}haskell a -\textgreater{} f
(b, Auto a b) \textasciitilde{}\textasciitilde{}\textasciitilde{}

What, you say? What good does that do?

Well, what does returning things in a context \emph{ever} let you do?

In Haskell, contexts like these are usually meant to be able to defer the
process of "getting the value" until the end, after you've built up your
contextual computation. This process can be complicated, or simple, or trivial.

For example, a function like:

\textasciitilde{}\textasciitilde{}\textasciitilde{}haskell a -\textgreater{} b
\textasciitilde{}\textasciitilde{}\textasciitilde{}

means that it simply creates a \texttt{b} from an \texttt{a}. But a function
like:

\textasciitilde{}\textasciitilde{}\textasciitilde{}haskell a -\textgreater{}
State s b \textasciitilde{}\textasciitilde{}\textasciitilde{}

Means that, given an \texttt{a}, you get a state machine that can \emph{create a
\texttt{b}} using a stateful process, once given an initial state. The
\texttt{b} doesn't "exist" yet; all you've given is instructions for creating
that \texttt{b}...and the \texttt{b} that is eventually created will in general
depend on whatever initial \texttt{s} you give the state machine.

A function like:

\textasciitilde{}\textasciitilde{}\textasciitilde{}haskell a -\textgreater{} IO
b \textasciitilde{}\textasciitilde{}\textasciitilde{}

Means that, given an \texttt{a}, you're given \emph{a computer program} that,
when executed by a computer, will generate a \texttt{b}. The \texttt{b} doesn't
"exist" yet; depending on how the world is and how IO processes interact, how
you are feeling that day...the \texttt{b} generated will be different. The
process of IO execution has the ability to \emph{choose} the \texttt{b}.

So how about something like:

\textasciitilde{}\textasciitilde{}\textasciitilde{}haskell a -\textgreater{}
State s (b, Auto a b) \textasciitilde{}\textasciitilde{}\textasciitilde{}

This means that, given \texttt{a}, "running" the \texttt{Auto} with an
\texttt{a} will give you \emph{a state machine} that gives you, using a stateful
process, both the \emph{result} and the \emph{next step}. The crazy thing is
that now you are given the state machine \emph{the ability to decide the next
\texttt{Auto}}, the next "step".

Something like:

\textasciitilde{}\textasciitilde{}\textasciitilde{}haskell a -\textgreater{} IO
(b, Auto a b) \textasciitilde{}\textasciitilde{}\textasciitilde{}

means that your new \texttt{Auto}-running function will give you a result and a
"next step" that is going to be dependent on IO actions.

Let's jump straight to abstracting over this and explore a new type, shall we?

\subsection{Monadic Auto}

\textasciitilde{}\textasciitilde{}\textasciitilde{}haskell -\/- source:
https://github.com/mstksg/inCode/tree/master/code-samples/machines/Auto3.hs\#L27-27
newtype AutoM m a b = AConsM \{ runAutoM :: a -\textgreater{} m (b, AutoM m a b)
\} \textasciitilde{}\textasciitilde{}\textasciitilde{}

We already explained earlier the new power of this type. Let's see if we can
write our favorite instances with it. First of all, what would a
\texttt{Category} instance even do?

Recall that the previous \texttt{Category} instance "ticked" each \texttt{Auto}
one after the other and gave the final results, and then the "next Auto" was the
compositions of the ticked autos.

In our new type, the "ticking" happens \emph{in a context}. And we need to tick
twice; and the second one is dependent on the result of the first. This means
that your context has to be \emph{monadic} in order to allow you to do this.

So we sequence two "ticks" inside the monadic context, and then return the
result afterwards, with the new composed autos.

The neat thing is that Haskell's built-in syntax for handling monadic sequencing
is nice, so you might be surprised when you write the \texttt{Category}
instance:

\textasciitilde{}\textasciitilde{}\textasciitilde{}haskell -\/- source:
https://github.com/mstksg/inCode/tree/master/code-samples/machines/Auto3.hs\#L43-48
instance Monad m =\textgreater{} Category (AutoM m) where id = AConsM \$
\textbackslash{}x -\textgreater{} return (x, id) g . f = AConsM \$
\textbackslash{}x -\textgreater{} do (y, f') \textless{}- runAutoM f x (z, g')
\textless{}- runAutoM g y return (z, g' . f')
\textasciitilde{}\textasciitilde{}\textasciitilde{}

Does it look familiar?

It should! Remember the logic from the \texttt{Auto} Category instance?

\textasciitilde{}\textasciitilde{}\textasciitilde{}haskell -\/- source:
https://github.com/mstksg/inCode/tree/master/code-samples/machines/Auto2.hs\#L13-18
instance Category Auto where id = ACons \$ \textbackslash{}x -\textgreater{} (x,
id) g . f = ACons \$ \textbackslash{}x -\textgreater{} let (y, f') = runAuto f x
(z, g') = runAuto g y in (z, g' . f')
\textasciitilde{}\textasciitilde{}\textasciitilde{}

It's basically \emph{identical} and exactly the same :O The only difference is
that instead of \texttt{let}, we have \texttt{do}...instead of \texttt{=} we
have \texttt{\textless{}-}, and instead of \texttt{in} we have \texttt{return}.
:O

The takeaway here is that when you have monadic functions, their sequencing and
application and composition can really be abstracted away to look pretty much
like application and composition of normal values. And Haskell is one of the few
languages that gives you language features and a culture to be able to fully
realize the symmetry and similarities.

Check out the \texttt{Functor} and \texttt{Arrow} instances, too -\/-\/- they're
exactly the same!

\textasciitilde{}\textasciitilde{}\textasciitilde{}haskell -\/- source:
https://github.com/mstksg/inCode/tree/master/code-samples/machines/Auto2.hs\#L20-47
instance Functor (Auto r) where fmap f a = ACons \$ \textbackslash{}x
-\textgreater{} let (y, a') = runAuto a x in (f y, fmap f a')

instance Arrow Auto where arr f = ACons \$ \textbackslash{}x -\textgreater{} (f
x, arr f) first a = ACons \$ (x, z) -\textgreater{} let (y, a') = runAuto a x in
((y, z), first a') second a = ACons \$ (z, x) -\textgreater{} let (y, a') =
runAuto a x in ((z, y), second a') a1 *** a2 = ACons \$ (x1, x2) -\textgreater{}
let (y1, a1') = runAuto a1 x1 (y2, a2') = runAuto a2 x2 in ((y1, y2), a1' ***
a2') a1 \&\&\& a2 = ACons \$ \textbackslash{}x -\textgreater{} let (y1, a1') =
runAuto a1 x (y2, a2') = runAuto a2 x in ((y1, y2), a1' \&\&\& a2')
\textasciitilde{}\textasciitilde{}\textasciitilde{}

\textasciitilde{}\textasciitilde{}\textasciitilde{}haskell -\/- source:
https://github.com/mstksg/inCode/tree/master/code-samples/machines/Auto3.hs\#L50-77
instance Monad m =\textgreater{} Functor (AutoM m r) where fmap f a = AConsM \$
\textbackslash{}x -\textgreater{} do (y, a') \textless{}- runAutoM a x return (f
y, fmap f a')

instance Monad m =\textgreater{} Arrow (AutoM m) where arr f = AConsM \$
\textbackslash{}x -\textgreater{} return (f x, arr f) first a = AConsM \$ (x, z)
-\textgreater{} do (y, a') \textless{}- runAutoM a x return ((y, z), first a')
second a = AConsM \$ (z, x) -\textgreater{} do (y, a') \textless{}- runAutoM a x
return ((z, y), second a') a1 *** a2 = AConsM \$ (x1, x2) -\textgreater{} do
(y1, a1') \textless{}- runAutoM a1 x1 (y2, a2') \textless{}- runAutoM a2 x2
return ((y1, y2), a1' *** a2') a1 \&\&\& a2 = AConsM \$ \textbackslash{}x
-\textgreater{} do (y1, a1') \textless{}- runAutoM a1 x (y2, a2') \textless{}-
runAutoM a2 x return ((y1, y2), a1' \&\&\& a2')
\textasciitilde{}\textasciitilde{}\textasciitilde{}

(I've left the rest of the instances from the previous part as an exercise; the
solutions are available in the downloadable.)

Neat, huh? Instead of having to learn over again the logic of \texttt{Functor},
\texttt{Applicative}, \texttt{Arrow}, \texttt{ArrowPlus}, etc., you can directly
use the intuition that you gained from the past part and apply it to here, if
you abstract away function application and composition to application and
composition in a context.

Our previous instances were then just a "specialized" version of \texttt{AutoM},
one where we used naked application and composition,{[}\^{}compapl{]}

\textless{}div class="note"\textgreater{} \textbf{Aside}

If you look at the instances we wrote out, you might see that for some of them,
\texttt{Monad} is a bit overkill. For example, for the \texttt{Functor}
instance,

\textasciitilde{}\textasciitilde{}\textasciitilde{}haskell instance Functor m
=\textgreater{} Functor (AutoM m r) where fmap f a = AConsM \$ (f *** fmap f) .
runAutoM a \textasciitilde{}\textasciitilde{}\textasciitilde{}

is just fine. We only need \texttt{Functor} to make \texttt{AutoM\ m\ r} a
\texttt{Functor}. Cool, right?

If you try, how much can we "generalize" our other instances to? Which ones can
be generalized to \texttt{Functor}, which ones \texttt{Applicative}...and which
ones can't? \textless{}/div\textgreater{}

By the way, it might be worth noting that our original \texttt{Auto} type is
identical to \texttt{AutoM\ Identity} -\/-\/- all of the instances do the exact
same thing.

\subsection{Putting it to use}

Now let's try using these!

First some utility functions just for playing around: \texttt{autoM}, which
upgrades an \texttt{Auto\ a\ b} to an \texttt{AutoM\ m\ a\ b} for any
\texttt{Monad} \texttt{m}{[}\^{}autom{]}, and \texttt{arrM}, which is like
\texttt{arr}, but instead of turning an \texttt{a\ -\textgreater{}\ b} into an
\texttt{Auto\ a\ b}, it turns an \texttt{a\ -\textgreater{}\ m\ b} into an
\texttt{AutoM\ m\ a\ b}:

\textasciitilde{}\textasciitilde{}\textasciitilde{}haskell -\/- source:
https://github.com/mstksg/inCode/tree/master/code-samples/machines/Auto3.hs\#L97-107
autoM :: Monad m =\textgreater{} Auto a b -\textgreater{} AutoM m a b autoM a =
AConsM \$ \textbackslash{}x -\textgreater{} let (y, a') = runAuto a x in return
(y, autoM a')

arrM :: Monad m =\textgreater{} (a -\textgreater{} m b) -\textgreater{} AutoM m
a b arrM f = AConsM \$ \textbackslash{}x -\textgreater{} do y \textless{}- f x
return (y, arrM f) \textasciitilde{}\textasciitilde{}\textasciitilde{}

We will need to of course re-write our trusty
\href{https://github.com/mstksg/inCode/tree/master/code-samples/machines/Auto.hs\#L17-25}{\texttt{testAuto}}
functions from the first entry, which is again a direct translation of the
original ones:

\textasciitilde{}\textasciitilde{}\textasciitilde{}haskell -\/- source:
https://github.com/mstksg/inCode/tree/master/code-samples/machines/Auto3.hs\#L31-39
testAutoM :: Monad m =\textgreater{} AutoM m a b -\textgreater{} {[}a{]}
-\textgreater{} m ({[}b{]}, AutoM m a b) testAutoM a {[}{]} = return ({[}{]}, a)
testAutoM a (x:xs) = do (y , a' ) \textless{}- runAutoM a x (ys, a'')
\textless{}- testAutoM a' xs return (y:ys, a'')

testAutoM\_ :: Monad m =\textgreater{} AutoM m a b -\textgreater{} {[}a{]}
-\textgreater{} m {[}b{]} testAutoM\_ a = liftM fst . testAutoM a
\textasciitilde{}\textasciitilde{}\textasciitilde{}

First, let's test \texttt{arrM} -\/-\/-

\textasciitilde{}\textasciitilde{}\textasciitilde{}haskell ghci\textgreater{} :t
arrM putStrLn arrM putStrLn :: AutoM IO String () ghci\textgreater{} res
\textless{}- testAutoM\_ (arrM putStrLn) {[}"hello", "world"{]} "hello" "world"
ghci\textgreater{} res {[}(), (){]}
\textasciitilde{}\textasciitilde{}\textasciitilde{}

\texttt{arrM\ putStrLn} is, like \texttt{arr\ show}, just an \texttt{Auto} with
no internal state. It outputs \texttt{()} for every single input string, except,
in the process of getting the "next Auto" (and producing the \texttt{()}), it
emits a side-effect -\/-\/- in our case, printing the string.

\subsubsection{in IO}

We can sort of abuse this to get an \texttt{Auto} with "two input streams": one
from the normal input, and the other from \texttt{IO}:

\textasciitilde{}\textasciitilde{}\textasciitilde{}haskell -\/- source:
https://github.com/mstksg/inCode/tree/master/code-samples/machines/Auto3.hs\#L119-123
replicateGets :: AutoM IO Int String replicateGets = proc n -\textgreater{} do
ioString \textless{}- arrM (\_ -\textgreater{} getLine) -\textless{} () let
inpStr = concat (replicate n ioString) autoM monoidAccum -\textless{} inpStr
\textasciitilde{}\textasciitilde{}\textasciitilde{}

So, \texttt{replicateGets} uses
\href{https://github.com/mstksg/inCode/tree/master/code-samples/machines/Auto.hs\#L106-107}{\texttt{monoidAccum}}
(or, an \texttt{AutoM} version) to accumulate a string. At every step, it adds
\texttt{inpStr} to the running accumulated string. \texttt{inpStr} is the result
of repeating the the string that \texttt{getLine} returns replicated \texttt{n}
times -\/-\/- \texttt{n} being the official "input" to the \texttt{AutoM} when
we eventually run it.

\textasciitilde{}\textasciitilde{}\textasciitilde{}haskell ghci\textgreater{}
testAutoM\_ replicateGets {[}3,1,5{]}

\begin{quote}
hello world bye {[} "hellohellohello" -\/- added "hello" three times ,
"hellohellohelloworld" -\/- added "world" once ,
"hellohellohelloworldbyebyebyebyebye" -\/- added "bye" five times {]}
\textasciitilde{}\textasciitilde{}\textasciitilde{}
\end{quote}

Here, we used \texttt{IO} to get a "side channel input". The main input is the
number of times we repeat the string, and the side input is what we get from
sequencing the \texttt{getLine} effect.

You can also use this to "tack on" effects into your pipeline.

\textasciitilde{}\textasciitilde{}\textasciitilde{}haskell -\/- source:
https://github.com/mstksg/inCode/tree/master/code-samples/machines/Auto3.hs\#L127-131
logging :: Show b =\textgreater{} Auto a b -\textgreater{} AutoM IO a b logging
a = proc x -\textgreater{} do y \textless{}- autoM a -\textless{} x arrM
(appendFile "log.txt") -\textless{} show y ++ "\textbackslash{}n" id
-\textless{} y \textasciitilde{}\textasciitilde{}\textasciitilde{}

Here, \texttt{logging\ a} will "run" \texttt{a} with the input like normal (no
side-channel inputs), but then also log the results line-by-line to
\emph{log.txt}.

\textasciitilde{}\textasciitilde{}\textasciitilde{}haskell ghci\textgreater{}
testAutoM\_ (logging summer) {[}6,2,3,4,1{]} {[}6,8,11,15,16{]}
ghci\textgreater{} putStrLn =\textless{}\textless{} readFile "log.txt" 6 8 11 15
16 \textasciitilde{}\textasciitilde{}\textasciitilde{}

(By the way, as a side note,
\texttt{logging\ ::\ Auto\ a\ b\ -\textgreater{}\ AutoM\ IO\ a\ b} here can be
looked at as an "\texttt{Auto} transformer". It takes a normal \texttt{Auto} and
transforms it into an otherwise identical \texttt{Auto}, yet which logs its
results as it ticks on.)

\subsubsection{Motivations}

At this point, hopefully you are either excited about the possibilities that
monadic \texttt{Auto} composition/ticking offers, or are horribly revolted at
how we mixed IO and unconstrained effects and "implicit side channel" inputs. Or
both!

After all, if all we were doing in \texttt{replicateGets} was having two inputs,
we could have just used:

\textasciitilde{}\textasciitilde{}\textasciitilde{}haskell replicateGets' ::
Auto (String, Int) String \textasciitilde{}\textasciitilde{}\textasciitilde{}

And have the user "get" the string before they run the \texttt{Auto}.

And hey, if all we were doing in \texttt{logging} was having an extra logging
channel, we could have just manually logged all of the outputs as they popped
out.

All valid suggestions. Separate the pure from the impure. We went out of our way
to avoid \emph{global} states and side-effects, so why bother to bring it all
back?

Superficially, it might seem like just moving the burden from one place to the
other. Instead of having the user having to worry about getting the string, or
writing the log, the \texttt{Auto} can just handle it itself internally without
the "running" code having to worry.

The real, deep advantage in \texttt{AutoM}, however, is -\/-\/- like in
\texttt{Auto} -\/-\/- its (literal) \emph{composability}.

Imagine \texttt{replicateGets\textquotesingle{}} was not our "final
\texttt{Auto}" that we run...imagine it was in fact an \texttt{Auto} used in a
composition inside the definition of an \texttt{Auto} used several times inside
a composition inside the definition of another \texttt{Auto}. All of a sudden,
having to "manually thread" the extra channel of input in is a real nightmare.
In addition, you can't even statically guarantee that the \texttt{String}
\texttt{replicateGets} eventually was the same \texttt{String} that the user
originally passed in. When composing/calling it, who knows if the Auto that
composes \texttt{replicateGets\textquotesingle{}} passes in the same initially
gotten String?{[}\^{}readernotio{]}

Imagine that the \texttt{Auto} whose results we wanted to log actually was not
the final output of the entire \texttt{Auto} we run (maybe we want to log a
small internal portion of a big \texttt{Auto}). Again, now you have to manually
thread the \emph{output}. And if you're logging several things through several
layers -\/-\/- it gets ugly very fast.

And now, all of your other \texttt{Auto}s in the composition get to (and
\emph{have} to) see the values of the log! So much for "locally stateful"!

As you can see, there is a trade-off in either decision we make. But these
monadic compositions really just give us another tool in our toolset that we can
(judiciously) use.

\subsubsection{Other contexts}

It's fun to imagine what sort of implications the different popular monads in
Haskell can provide. \texttt{Writer} gives you a running log that all
\texttt{Auto}s can append to, for example. \texttt{Reader} gives you every
composed \texttt{Auto} the ability to access a shared global environment...and
has an advantage over manual "passing in" of parameters because every composed
\texttt{Auto} is guaranteed to "see" the same global environment per tick.

\texttt{State} gives every composed \texttt{Auto} the ability to access and
modify a globally shared state. We talk a lot about every \texttt{Auto} having
their own local, internal state; usually, it is impossible for two composed
\texttt{Auto}s to directly access each other's state (except by communicating
through output and input). With \texttt{State}, we now give the opportunity for
every \texttt{Auto} to share and modify a collective and global state, which
they can use to determine how to proceed, etc.

Good? Bad? Uncontrollable, unpredictable? Perhaps. You now bring in all of the
problems of shared state and reasoning with shared mutable state...and avoiding
these problems was one of the things that originally motivated the usage of
\texttt{Auto} in the first place! But, we can make sound and judicious decisions
without resorting to "never do this" dogma.{[}\^{}dogma{]} Remember, these are
just tools we can possibly explore. Whether or not they work in the real world
-\/-\/- or whether or not they are self-defeating -\/-\/- is a complex story!

\subsubsection{in State}

Here is a toy state example to demonstrate different autos talking to each
other; here, the state is a measure of "fuel"; we can take any
\texttt{Auto\ a\ b} and give it a "cost" using the \texttt{limit} function
defined here. Here, every \texttt{Auto} consumes fuel from the same pool, given
at the initial \texttt{runState} running.

\textasciitilde{}\textasciitilde{}\textasciitilde{}haskell -\/- source:
https://github.com/mstksg/inCode/tree/master/code-samples/machines/Auto3.hs\#L139-175
limit :: Int -\textgreater{} Auto a b -\textgreater{} AutoM (State Int) a (Maybe
b) limit cost a = proc x -\textgreater{} do fuel \textless{}- arrM (\_
-\textgreater{} get) -\textless{} () if fuel \textgreater{}= cost then do arrM
(\_ -\textgreater{} modify (subtract cost)) -\textless{} () y \textless{}- autoM
a -\textless{} x id -\textless{} Just y else id -\textless{} Nothing

sumSqDiff :: AutoM (State Int) Int Int sumSqDiff = proc x -\textgreater{} do
sums \textless{}- fromMaybe 0 \textless{}\$\textgreater{} limit 3 summer
-\textless{} x sumSqs \textless{}- fromMaybe 0 \textless{}\$\textgreater{} limit
1 summer -\textless{} x\^{}2 id -\textless{} sumSqs - sums

stuff :: AutoM (State Int) Int (Maybe Int, Maybe Int, Int) stuff = proc x
-\textgreater{} do doubled \textless{}- limit 1 id -\textless{} x * 2 tripled
\textless{}- if even x then limit 2 id -\textless{} x * 3 else id -\textless{}
Just (x * 3) sumSqD \textless{}- sumSqDiff -\textless{} x id -\textless{}
(doubled, tripled, sumSqD) \textasciitilde{}\textasciitilde{}\textasciitilde{}

\textasciitilde{}\textasciitilde{}\textasciitilde{}haskell -\/- a State machine
returning the result and the next Auto ghci\textgreater{} let stuffState =
runAutoM stuff 4 -\/- a State machine returning the result ghci\textgreater{}
let stuffState\_ = fst \textless{}\$\textgreater{} stuffState ghci\textgreater{}
:t stuffState\emph{ stuffState} :: State Int (Maybe Int, Maybe Int, Int) -\/-
start with 10 fuel ghci\textgreater{} runState stuffState\_ 10 ((Just 8, Just
12, 12), 3) -\/- end up with 3 fuel left -\/- start with 2 fuel
ghci\textgreater{} runState stuffState\_ 2 ((Just 8, Nothing, 16), 0) -\/- poop
out halfway \textasciitilde{}\textasciitilde{}\textasciitilde{}

You can see that an initial round with an even number should cost you seven
fuel...if you can get to the end. In the case where we only started with two
fuel, we only were able to get to the "doubled" part before running out of fuel.

Let's see what happens if we run it several times:

\textasciitilde{}\textasciitilde{}\textasciitilde{}hasell ghci\textgreater{} let
stuffStateMany = testAutoM\_ stuff {[}3..6{]} ghci\textgreater{} :t
stuffStateMany stuffStateMany :: State Int {[}(Maybe Int, Maybe Int, Int){]}
ghci\textgreater{} runState stuffStateMany 9 ( {[} (Just 6 , Just 9 , 6 ) ,
(Just 8 , Just 12, 25) , (Nothing, Just 15, 0 ) , (Nothing, Nothing, 0 ) {]} , 0
) \textasciitilde{}\textasciitilde{}\textasciitilde{}

So starting with nine fuel, we seem to run out halfway through the second step.
The third field should be the sum of the squares so far, minus the sum so
far...at \texttt{25}, it's probably just the sum of the squares so far. So it
couldn't even subtract out the sum so far. Note that the \texttt{Just\ 15} on
the third step goes through because for \emph{odd} inputs (5, in this case), the
second field doesn't require any fuel.

Anyways, imagine having to thread this global state through by hand. Try it.
It'd be a disaster! Everything would have to take an extra parameter and get and
extra parameter...it really is quite a headache. Imagine the source for
\texttt{stuff} being written out in \texttt{Auto} with manual state threading.

But hey, if your program needs global state, then it's probably a good sign that
you might have had a design flaw somewhere along the way, right?

\subsubsection{in Reader}

Here we use \texttt{Reader} to basically give a "second argument" to an
\texttt{Auto} when we eventually run it, but we use the fact that every composed
\texttt{Auto} gets the \emph{exact same} input to great effect:

\textasciitilde{}\textasciitilde{}\textasciitilde{}haskell -\/- source:
https://github.com/mstksg/inCode/tree/master/code-samples/machines/Auto3.hs\#L199-234
integral :: Double -\textgreater{} AutoM (Reader Double) Double Double integral
x0 = AConsM \$ \textbackslash{}dx -\textgreater{} do dt \textless{}- ask let x1
= x0 + dx * dt return (x1, integral x1)

derivative :: AutoM (Reader Double) Double (Maybe Double) derivative = AConsM \$
\textbackslash{}x -\textgreater{} return (Nothing, derivative' x) where -\/- x0
is the "previous input" derivative' x0 = AConsM \$ \textbackslash{}x1
-\textgreater{} do let dx = x1 - x0 dt \textless{}- ask return (Just (dx/dt),
derivative' x1)

fancyCalculus :: AutoM (Reader Double) Double (Double, Double) fancyCalculus =
proc x -\textgreater{} do deriv \textless{}- fromMaybe 0
\textless{}\$\textgreater{} derivative -\textless{} x deriv2 \textless{}-
fromMaybe 0 \textless{}\$\textgreater{} derivative -\textless{} deriv intdev
\textless{}- integral 0 -\textless{} deriv id -\textless{} (deriv2, intdev)
\textasciitilde{}\textasciitilde{}\textasciitilde{}

Now, we are treating our input stream as time-varying values, and the "Reader
environment" contains the "time passed since the last tick" -\/-\/- The time
step or sampling rate, so to speak, of the input stream. We have two stateful
\texttt{Auto}s ("locally stateful", internal state) that compute the time
integral and time derivative of the input stream of numbers...but in order to do
so, it needs the time step. We get it using \texttt{ask}. (Note that the time
step doesn't have to be the same between every different tick ...
\texttt{integral} and \texttt{derivative} should work just fine with a new
timestep every tick.) (Also note that \texttt{derivative} is \texttt{Nothing} on
its first step, because there is not yet any meaningful derivative on the first
input)

In \texttt{fancyCalculus}, we calculate the integral, the derivative, the second
derivative, and the integral of the derivative, and return the second derivative
and the integral of the derivative.

In order for us to even \emph{meaningfully say} "the second derivative" or "the
integral of the derivative", the double derivative has to be calculated with the
same time step, and the integral and the derivative have to be calculated with
the same time step. If they are fed different time steps, then we aren't really
calculating a real second derivative or a real integral of a derivative anymore.
We're just calculating arbitrary numbers.

Anyways, if you have taken any introduction to calculus course, you'll know that
the integral of a derivative is the original function -\/-\/- so the integral of
the derivative, if we pick the right \texttt{x0}, should just be an "id"
function:

\textasciitilde{}\textasciitilde{}\textasciitilde{}haskell integral x0 .
derivative == id -\/- or off by a constant difference
\textasciitilde{}\textasciitilde{}\textasciitilde{}

Let's try this out with some input streams where we know what the second
derivative should be, too.

We'll try it first with \texttt{x\^{}2}, where we know the second derivative
will just be 2, the entire time:

\textasciitilde{}\textasciitilde{}\textasciitilde{}haskell ghci\textgreater{}
let x2s = map (\^{}2) {[}0,0.05..1{]} ghci\textgreater{} let x2Reader =
testAutoM\_ fancyCalculus x2s ghci\textgreater{} :t x2Reader x2Reader :: Reader
Double {[}(Double, Double){]} ghci\textgreater{} map fst (runReader x2Reader
0.05) {[} ... 2.0, 2.0 ... {]} -\/- with a couple of "stabilizing" first terms
ghci\textgreater{} map snd (runReader x2Reader 0.05) {[} 0.0, 0.0025, 0.01,
0.0225, 0.04 ...{]} ghci\textgreater{} x2s {[} 0.0, 0.0025, 0.01, 0.0225, 0.04
...{]} \textasciitilde{}\textasciitilde{}\textasciitilde{}

Perfect! The second derivative we expected (all 2's) showed up, and the integral
of the derivative is pretty much exactly the original function.

For fun, try running it with a \texttt{sin} function. The second derivative of
\texttt{sin} is \texttt{netage\ .\ sin}. Does it end up as expected?

The alternative to using \texttt{AutoM} and \texttt{Reader} here would be to
have each composed Auto be manually "passed" the \texttt{dt} timestep. But then
we really don't have any "guarantees", besides checking ourselves, that every
\texttt{Auto} down the road, down every composition, will have the same
\texttt{dt}. We can't say that we really are calculating integrals or
derivatives. And plus, it's pretty messy when literally every one of your
composed \texttt{Auto} needs \texttt{dt}.

\subsubsection{Mixing Worlds}

We talked about a huge drawback of \texttt{State\ s} -\/-\/- global mutable
state is really something that we originally looked to \texttt{Auto} to avoid in
the first place. But some portions of logic are much more convenient to write
with autos that all have access to a global state.

What if we wanted the best of both worlds? What would that look like?

In Haskell, one common technique we like to use, eloquently stated by Gabriel
Gonzalez in his post
\href{http://www.haskellforall.com/2012/09/the-functor-design-pattern.html}{the
Functor design pattern}, is to pick a "common denominator" type, and push all of
our other types into it.

We have two fundamentally different options here. We can pick our "main type" to
be \texttt{AutoM\ (State\ s)} and have global state, and "push" all of our
non-global-state Autos into it, or we can pick our "main type" to be
\texttt{Auto}, and "seal" our global-state-Autos into non-global-state ones.

For the former, we'd use \texttt{autoM}'s whenever we want to bring our
\texttt{Auto}s into \texttt{AutoM\ (State\ s)}...or we can always write
\texttt{AutoM}'s parameterized over \texttt{m}:

\textasciitilde{}\textasciitilde{}\textasciitilde{}haskell summer :: (Monad m,
Num a) =\textgreater{} AutoM m a a
\textasciitilde{}\textasciitilde{}\textasciitilde{}

It is statically guaranteed that \texttt{summer} \emph{cannot touch any global
state}.

For the latter option, we take \texttt{AutoM\ (State\ s)}'s that operate on
global state and then basically "seal off" their access to be just within their
local worlds, as we turn them into \texttt{Auto}'s.

\textasciitilde{}\textasciitilde{}\textasciitilde{}haskell -\/- source:
https://github.com/mstksg/inCode/tree/master/code-samples/machines/Auto3.hs\#L182-192
sealStateAuto :: AutoM (State s) a b -\textgreater{} s -\textgreater{} Auto a b
sealStateAuto a s0 = ACons \$ \textbackslash{}x -\textgreater{} let ((y, a'),
s1) = runState (runAutoM a x) s0 in (y, sealStateAuto a' s1)

runStateAuto :: AutoM (State s) a b -\textgreater{} Auto (a, s) (b, s)
runStateAuto a = ACons \$ (x, s) -\textgreater{} let ((y, a'), s') = runState
(runAutoM a x) s in ((y, s'), runStateAuto a')
\textasciitilde{}\textasciitilde{}\textasciitilde{}

\texttt{sealStateAuto} does exactly this. Give it an initial state, and the
\texttt{Auto} will just continuously feed in its output state at every tick back
in as the input state. Every \texttt{Auto} inside now has access to a local
state, untouchable from the outside.

\texttt{runStateAuto} is a way to do this were you can pass in a new initial
state every time you "step" the \texttt{Auto}, and observe how it changes
-\/-\/- also another useful use case.

\textless{}div class="note"\textgreater{} \textbf{Aside}

We can even pull this trick to turn any \texttt{AutoM\ (StateT\ s\ m)} into an
\texttt{AutoM\ m}. See if you can write it :)

\textasciitilde{}\textasciitilde{}\textasciitilde{}haskell sealStateAutoM ::
AutoM (StateT s m) a b -\textgreater{} s -\textgreater{} AutoM m a b
sealStateAutoM = ...

runStateAutoM :: AutoM (StateT s m) a b -\textgreater{} AutoM m (a, s) (b, s)
runStateAutoM = ... \textasciitilde{}\textasciitilde{}\textasciitilde{}
\textless{}/div\textgreater{}

In both of these methods, what is the real win? The big deal is that you can now
chose to "work only in the world of non-global-state", combining non-global
\texttt{Auto}s like we did in part 1 and part 2 to create non-global algorithms.
And then you can also chose to "work in the world of global state", combining
global \texttt{Auto}s like we did in the previous section, where having
\texttt{State\ s} made everything more clear.

We're allowed to live and compose (using \texttt{Category}, proc notation, etc.)
in whatever world we like -\/-\/- create as complex compositions as we could
even imagine -\/-\/- and at the end of it all, we take the final complex product
and "glue it on" to our big overall type so everything can work together.

This discussion is about \texttt{State}, but the ramifications work with almost
any \texttt{Auto} or type of \texttt{Auto} or underlying monad we talk about.

We can simulate an "immutable local environment", for example:

\textasciitilde{}\textasciitilde{}\textasciitilde{}haskell -\/- source:
https://github.com/mstksg/inCode/tree/master/code-samples/machines/Auto3.hs\#L248-251
runReaderAuto :: AutoM (Reader r) a b -\textgreater{} Auto (a, r) b
runReaderAuto a = ACons \$ (x, e) -\textgreater{} let (y, a') = runReader
(runAutoM a x) e in (y, runReaderAuto a')
\textasciitilde{}\textasciitilde{}\textasciitilde{}

Now you can use a \texttt{Reader} -\/-\/- composed with "global environment"
semantics -\/-\/- inside a normal \texttt{Auto}! Just give it the new
environment very step! (Can you write a \texttt{sealReaderAuto} that just takes
an initial \texttt{r} and feeds it back in forever?)

\section{Recursive Auto}

Let's move back to our normal \texttt{Auto} for now, and imagine a very common
use case that might come up.

What if you wanted two chained \texttt{Auto}s to "talk to each other" -\/-\/-
for their inputs to depend on the other's outputs?

Here's a common example -\/-\/- in control theory, you often have to have adjust
an input to a system to get it to "respond" to a certain desired output (a
control).

One way is to start with a test input, at every step, observe the resulting
response and adjust it up or down until we get the response we want. We call the
difference between the response and the control the "error".

How do you think you would calculate the adjustment? Well...if the error is big,
we probably want a big adjustment. And, the longer we are away from the error,
we also might want to make a bigger adjustment accordingly, too.

In other words, we might want our adjustment to have a term \emph{proportional}
to the error, and a term that is \emph{the sum of all} errors so far.

This system is known as \href{http://en.wikipedia.org/wiki/PID_controller}{PI},
and is actually used in many industrial control systems today, for controlling
things like lasers and other super important stuff. Congrats, you are now a
control theorist!

Let's see how we might write this using our \texttt{Auto}s:

\textasciitilde{}\textasciitilde{}\textasciitilde{}haskell piTargeter :: Auto
Double Double piTargeter = proc control -\textgreater{} do let err = control -
response errSums \textless{}- summer -\textless{} err

\begin{verbatim}
input    <- summer         -< 0.2 * err + 0.01 * errSums
response <- blackBoxSystem -< input

id -< response
\end{verbatim}

where blackBoxSystem = id -\/- to simplify things :)
\textasciitilde{}\textasciitilde{}\textasciitilde{}

So this is an \texttt{Auto} that takes in a \texttt{Double} -\/-\/- the control
-\/-\/- and outputs a \texttt{Double} -\/-\/- the response. The goal is to get
the response to "match" control, by running a value, \texttt{input}, through a
"black box system" (To simplify here, we're only running \texttt{input} through
\texttt{id}).

Here is the "logic", or the relationships between the values:

\begin{enumerate}
\tightlist
\item
  The error value \texttt{err} is the difference between the control and the
  response.
\item
  The sum of errors \texttt{errSums} is the cumulative sum of all of the error
  values so far.
\item
  The input \texttt{input} is the cumulative sum of all of the correction terms:
  a multiple of \texttt{err} and a multiple of \texttt{errSums}.
\item
  The response \texttt{response} is the result of running the input through the
  black box system (here, just \texttt{id}).
\item
  The output is the response!
\end{enumerate}

Look at what we wrote. Isn't it just...beautifully declarative? Elegant? All we
stated were \emph{relationships between terms}...we didn't worry about state,
loops, variables, iterations...there is no concept of "how to update",
everything is just "how things are". It basically popped up exactly as how we
"said" it. I don't know about you, but this demonstration always leaves me
amazed, and was one of the things that sold me on this abstraction in the first
place.

But, do you see the problem? To calculate \texttt{err}, we used \texttt{resp}.
But to get \texttt{resp}, we need \texttt{err}!

We need to be able to define "recursive bindings". Have Autos recursively depend
on each other.

In another language, this would be hopeless. We'd have to have to resort to
keeping explicit state and using a loop. However, with Haskell...and the world
of laziness, recursive bindings, and tying knots...I think that we're going to
have a \emph{real win} if we can make something like what we wrote work.

\subsection{ArrowLoop}

There is actually a construct in \emph{proc} notation that lets you do just
that. I'm going to cut to the chase and show you how it looks, and how you use
it. I'll explain the drawbacks and caveats. And then I'll explain how it works
in an aside -\/-\/- it's slightly heavy, but some people like to understand.

Without further ado -\/-\/-

\textasciitilde{}\textasciitilde{}\textasciitilde{}haskell -\/- source:
https://github.com/mstksg/inCode/tree/master/code-samples/machines/Auto3.hs\#L266-276
piTargeter :: Auto Double Double piTargeter = proc control -\textgreater{} do
rec let err = control - response errSums \textless{}- summer -\textless{} err

\begin{verbatim}
    input    <- laggingSummer  -< 0.2 * err + 0.01 * errSums
    response <- blackBoxSystem -< input

id -< response
\end{verbatim}

where blackBoxSystem = id -\/- to simplify things :)
\textasciitilde{}\textasciitilde{}\textasciitilde{}

The key here is the \emph{rec} keyword. Basically, we require that we write an
instance of \texttt{ArrowLoop} for our \texttt{Auto}...and now things can refer
to each other, and it all works out like magic! Now our solution works...the
feedback loop is closed with the usage of \texttt{rec}. Now, our algorithm looks
\emph{exactly} like how we would "declare" the relationship of all the
variables. We "declare" that \texttt{err} is the difference between the control
and the response. We "declare" that \texttt{errSums} is the cumulative sum of
the error values. We "declare" that our \texttt{input} is the cumulative sum of
all of the adjustment terms. And we "declare" that our response is just the
result of feeding our input through our black box.

No loops. No iteration. No mutable variables. Just...a declaration of
relationships.

\textasciitilde{}\textasciitilde{}\textasciitilde{}haskell ghci\textgreater{}
testAuto\_ piTargeter {[}5,5.01..6{]} -\/- vary our desired target slowly {[} 0,
1.05, 1.93, 2.67, 3.28 ... -\/- "seeking"/tracking to 5 , 5.96, 5.97, 5.98,
5.99, 6.00 -\/- properly tracking {]}
\textasciitilde{}\textasciitilde{}\textasciitilde{}

Perfect!

Wait wait wait hold on...but how does this even work? Is this magic? Can we just
throw \emph{anything} into a recursive binding, and expect it to magically
figure out what we mean?

Kinda, yes, no. This works based on Haskell's laziness. It's the reason
something like \texttt{fix} works:

\textasciitilde{}\textasciitilde{}\textasciitilde{}haskell fix :: (a
-\textgreater{} a) -\textgreater{} a fix f = f (fix f)
\textasciitilde{}\textasciitilde{}\textasciitilde{}

Infinite loop, right?

\textasciitilde{}\textasciitilde{}\textasciitilde{}haskell ghci\textgreater{}
head (fix (1:)) 1 \textasciitilde{}\textasciitilde{}\textasciitilde{}

What?

\texttt{fix\ (1:)} is basically an infinite lists of ones. But remember that
\texttt{head} only requires the first element to be evaluated:

\textasciitilde{}\textasciitilde{}\textasciitilde{}haskell head (fix (1:)) head
(1 : fix (1:)) -\/- head (x:\_) = x 1
\textasciitilde{}\textasciitilde{}\textasciitilde{}

So that's the key. If what we \emph{want} doesn't require the entire result of
the infinite loop...then we can safely reason about infinite recursion in
haskell.

The MVP here really is this function that I sneakily introduced,
\texttt{laggingSummer}:

\textasciitilde{}\textasciitilde{}\textasciitilde{}haskell -\/- source:
https://github.com/mstksg/inCode/tree/master/code-samples/machines/Auto3.hs\#L258-262
laggingSummer :: Num a =\textgreater{} Auto a a laggingSummer = sumFrom 0 where
sumFrom :: Num a =\textgreater{} a -\textgreater{} Auto a a sumFrom x0 = ACons
\$ \textbackslash{}x -\textgreater{} (x0, sumFrom (x0 + x))
\textasciitilde{}\textasciitilde{}\textasciitilde{}

\texttt{laggingSummer} is like \texttt{summer}, except all of the sums are
delayed. Every step, it adds the input to the accumulator...but returns the
accumulator \emph{before} the addition. Sort of like \texttt{x++} instead of
\texttt{++x} in C. If the accumulator is at 10, and it receives a 2, it
\emph{outputs 10}, and \emph{updates the accumulator to 12}. The key is that it
\emph{doesn't need the input} to \emph{immediately return that step's output}.

\textasciitilde{}\textasciitilde{}\textasciitilde{}haskell ghci\textgreater{}
testAuto\_ laggingSummer {[}5..10{]} {[}0, 5, 11, 18, 26, 35{]}
\textasciitilde{}\textasciitilde{}\textasciitilde{}

The accumulator starts off at 0, and receives a 5...it then outputs 0 and
updates the accumulator to 5. The accumulator then has 5 and receives a 6...it
outputs 5 and then updates the accumulator to 11. Etc. The next step it would
output 45 \emph{no matter what input it gets}.

Look at the definition of \texttt{piTargeter} again. How would it get its "first
value"?

\begin{enumerate}
\tightlist
\item
  The first output is just \texttt{response}.
\item
  The first response is just the first \texttt{input}
\item
  The first \texttt{input} is just the result of \texttt{laggingSummer}.
\item
  The first result of \texttt{laggingSummer} is 0.
\end{enumerate}

And that's it! Loop closed! The first result is zero...no infinite recursion
here.

Now that we know that the first result of \texttt{response} is 0, we can also
find the first values of \texttt{err} and \texttt{errSums}: The first
\texttt{err} is the first control (input to the \texttt{Auto}) minus 0 (the
first response), and the first \texttt{errSums} is a cumulative sum of
\texttt{errs}, so it too starts off as the first control minus zero.

So now, we have all of the first values of \emph{all} of our Autos. Check! Now
the next step is the same thing!

Recursive bindings have a lot of power in that they allow us to directly
translate natural language and (cyclic) graph-like "relationships" (here,
between the different values of a control system) and model them \emph{as
relationships}. Not as loops and updates and state modifications. But \emph{as
relationships}. Something we can \emph{declare}, at a high level.

And that's definitely something I would write home about.

The only caveat is, of course, that we have to make sure our loop can produce a
"first value" without worrying about its input. Autos like
\texttt{laggingSummer} give this to us.

In the following aside, I detail the exact mechanics of how this works :)

\textless{}div class="note"\textgreater{} \textbf{Aside}

Ah, so you're curious? Or maybe you are just one of those people who really
wants to know how things work?

The \texttt{rec} keyword in proc/do blocks desugars to applications of a
function called \texttt{loop}:

\textasciitilde{}\textasciitilde{}\textasciitilde{}haskell class Arrow r
=\textgreater{} ArrowLoop r where loop :: r (a, c) (b, c) -\textgreater{} r a b
\textasciitilde{}\textasciitilde{}\textasciitilde{}

The type signature seems a bit funny. Loop takes a morphism from
\texttt{(a,\ c)} to \texttt{(b,\ c)} and turns it into a morphism from
\texttt{a} to \texttt{b}. But...how does it do that?

I'll point you to \href{https://wiki.haskell.org/Circular_programming}{a whole
article about the \texttt{(-\textgreater{})} instance of \texttt{ArrowLoop}} and
how it is useful, if you're interested. But we're looking at \texttt{Auto} for
now.

We can write an \texttt{ArrowLoop} instance for \texttt{Auto}:

\textasciitilde{}\textasciitilde{}\textasciitilde{}haskell -\/- source:
https://github.com/mstksg/inCode/tree/master/code-samples/machines/Auto2.hs\#L58-61
instance ArrowLoop Auto where loop a = ACons \$ \textbackslash{}x
-\textgreater{} let ((y, d), a') = runAuto a (x, d) in (y, loop a')
\textasciitilde{}\textasciitilde{}\textasciitilde{}

So what does this mean? When will we be able to "get a \texttt{y}"?

We will be able to get a \texttt{y} in the case that the \texttt{Auto} can just
"pop out" your \texttt{y} without ever evaluating its arguments...or only using
\texttt{x}.

The evaluation of \texttt{a\textquotesingle{}} is then deferred until
later...and through this, everything kinda makes sense. The loop is closed. See
the article linked above for more information on how \texttt{loop} really works.

The actual desugaring of a \texttt{rec} block is a little tricky, but we can
trust that if we have a properly defined \texttt{loop} (that typechecks and has
the circular dependencies that loop demands), then \texttt{ArrowLoop} will do
what it is supposed to do.

In any case, we can actually understand \emph{how to work with rec blocks}
pretty well -\/-\/- as long as we can have an \texttt{Auto} in the pipeline that
can pop something out immediately ignoring its input, then we can rest assured
that our knot will be closed.

By the way, this trick works with \texttt{ArrowM} too -\/-\/- provided that the
\texttt{Monad} is an instance of \texttt{MonadFix}, which is basically a
generalization of the recursive \texttt{let} bindings we used above:

\textasciitilde{}\textasciitilde{}\textasciitilde{}haskell -\/- source:
https://github.com/mstksg/inCode/tree/master/code-samples/machines/Auto3.hs\#L88-91
instance MonadFix m =\textgreater{} ArrowLoop (AutoM m) where loop a = AConsM \$
\textbackslash{}x -\textgreater{} do rec ((y, d), a') \textless{}- runAutoM a
(x, d) return (y, loop a') \textasciitilde{}\textasciitilde{}\textasciitilde{}
\textless{}/div\textgreater{}

\section{Going Kleisli}

This is going to be our last "modification" to the \texttt{Auto} type -\/-\/-
one more common \texttt{Auto} variation/trick that is used in real life usages
of \texttt{Auto}.

\subsection{Inhibition}

It might some times be convenient to imagine the \emph{results} of the
\texttt{Auto}s coming in contexts -\/-\/- for example, \texttt{Maybe}:

\textasciitilde{}\textasciitilde{}\textasciitilde{}haskell Auto a (Maybe b)
\textasciitilde{}\textasciitilde{}\textasciitilde{}

How can we interpret/use this? In many domains, this is used to model "on/off"
behavior of \texttt{Auto}s. The \texttt{Auto} is "on" if the output is
\texttt{Just}, and "off" if the output is \texttt{Nothing}.

We can imagine "baking this in" to our Auto type:

\textasciitilde{}\textasciitilde{}\textasciitilde{}haskell -\/- source:
https://github.com/mstksg/inCode/tree/master/code-samples/machines/AutoOn.hs\#L19-19
newtype AutoOn a b = AConsOn \{ runAutoOn :: a -\textgreater{} (Maybe b, AutoOn
a b) \} \textasciitilde{}\textasciitilde{}\textasciitilde{}

Where the semantics of composition are: if you get a \texttt{Nothing} as an
input, just don't tick anything and pop out a \texttt{Nothing}; if you get a
\texttt{Just\ x} as an input run the auto on the \texttt{x}:

\textasciitilde{}\textasciitilde{}\textasciitilde{}haskell -\/- source:
https://github.com/mstksg/inCode/tree/master/code-samples/machines/AutoOn.hs\#L22-29
instance Category AutoOn where id = AConsOn \$ \textbackslash{}x -\textgreater{}
(Just x, id) g . f = AConsOn \$ \textbackslash{}x -\textgreater{} let (y, f') =
runAutoOn f x (z, g') = case y of Just \emph{y -\textgreater{} runAutoOn g }y
Nothing -\textgreater{} (Nothing, g) in (z, g' . f')
\textasciitilde{}\textasciitilde{}\textasciitilde{}

The other instances are on the file linked above, but I won't post them here, so
you can write them as an exercise. Have fun on the \texttt{ArrowLoop}
instance!{[}\^{}autoonnt{]}

\textless{}div class="note"\textgreater{} \textbf{Aside}

This aside contains category-theoretic justification for what we just did. You
can feel free to skip it if you aren't really too familiar with the bare basics
of Category Theory (What an endofunctor is, for example)... but, if you are,
this might be a fun perspective :)

What we've really done here is taken a category with objects as Haskell types
and morphisms are \texttt{Auto\ a\ b}, and turned it into a category with
objects as Haskell types and whose morphisms are \texttt{Auto\ a\ (m\ b)}, where
\texttt{m} is a Monad.

The act of forming this second category from the first is called forming the
\emph{Kleisli category} on a category. We took \texttt{Auto} and are now looking
at the Kleisli category on \texttt{Auto} formed by \texttt{Maybe}.

By the way, a "Monad" here is actually different from the normal \texttt{Monad}
typeclass found in standard Haskell. A Monad is an endofunctor on a category
with two associated natural transformations -\/-\/- unit and join.

Because we're not dealing with the typical Haskell category anymore (on
\texttt{(-\textgreater{})}), we have to rethink what we actually "have".

For any Haskell Monad, we get for free our natural transformations:

\textasciitilde{}\textasciitilde{}\textasciitilde{}haskell unitA :: Monad m
=\textgreater{} Auto a (m a) unitA = arr return

joinA :: Monad m =\textgreater{} Auto (m (m a)) (m a) joinA = arr join
\textasciitilde{}\textasciitilde{}\textasciitilde{}

But what we \emph{don't get}, necessarily, the \emph{endofunctor}. An
endofunctor must map both objects and morphisms. A type constructor like
\texttt{Maybe} can map objects fine -\/-\/- we have the same objects in
\texttt{Auto} as we do in \texttt{(-\textgreater{})} (haskell types). But we
also need the ability to map \emph{morphisms}:

\textasciitilde{}\textasciitilde{}\textasciitilde{}haskell class FunctorA f
where fmapA :: Auto a b -\textgreater{} Auto (f a) (f b) -\/- fmapA id = id -\/-
fmapA g . fmapA f = fmapA (g . f)
\textasciitilde{}\textasciitilde{}\textasciitilde{}

So, if this function exists for a type constructor, following the usual
\texttt{fmap} laws, then that type is an endofunctor in our \texttt{Auto}
category. And if it's also a Monad in \texttt{(-\textgreater{})}, then it's also
then a Monad in \texttt{Auto}.

We can write such an \texttt{fmapA} for \texttt{Maybe}:

\textasciitilde{}\textasciitilde{}\textasciitilde{}haskell instance FunctorA
Maybe where fmapA a = ACons \$ \textbackslash{}x -\textgreater{} case x of Just
\_x -\textgreater{} let (y, a') = runAuto a x in (Just y, fmapA a') Nothing
-\textgreater{} (Nothing, fmapA a)
\textasciitilde{}\textasciitilde{}\textasciitilde{}

And, it is a fact that if we have a Monad, we can write the composition of its
Kleisli category for free:

\textasciitilde{}\textasciitilde{}\textasciitilde{}haskell
(\textless{}\textasciitilde{}=\textless{}) :: (FunctorA f, Monad f)
=\textgreater{} Auto a (f c) -\textgreater{} Auto a (f b) -\textgreater{} Auto a
(f c) g \textless{}\textasciitilde{}=\textless{} f = joinA . fmapA g . f
\textasciitilde{}\textasciitilde{}\textasciitilde{}

In fact, for \texttt{f\ \textasciitilde{}\ Maybe}, this definition is identical
to the one for the \texttt{Category} instance we wrote above for
\texttt{AutoOn}.

And, if the \texttt{FunctorA} is a real functor and the \texttt{Monad} is a real
monad, then we have for free the associativity of this super-fish operator:

\textasciitilde{}\textasciitilde{}\textasciitilde{}haskell (h
\textless{}\textasciitilde{}=\textless{} g)
\textless{}\textasciitilde{}=\textless{} f == h
\textless{}\textasciitilde{}=\textless{} (g
\textless{}\textasciitilde{}=\textless{} f) f
\textless{}\textasciitilde{}=\textless{} unitA == unitA
\textless{}\textasciitilde{}=\textless{} f == f
\textasciitilde{}\textasciitilde{}\textasciitilde{}

Category theory is neat!

By the way, definitely not all endofunctors on \texttt{(-\textgreater{})} are
endofunctors on \texttt{Auto}. We see that \texttt{Maybe} is one. Can you think
of any others? Any others where we could write an instance of \texttt{FunctorA}
that follows the laws? Think about it, and post some in the comments!

One immediate example is \texttt{Either\ e}, which is used for great effect in
many FRP libraries! It's "inhibit, with a \emph{value}". As an exercise, see if
you can write its \texttt{FunctorA} instance, or re-write the \texttt{AutoOn} in
this section to work with \texttt{Either\ e} (you might need to impose a
typeclass constraint on the \texttt{e}) instaed of \texttt{Maybe}!
\textless{}/div\textgreater{}

I'm not going to spend too much time on this, other than saying that it is
useful to imagine how it might be useful to have an "off" Auto "shut down" every
next Auto in the chain.

One neat thing is that \texttt{AutoOn} admits a handy \texttt{Alternative}
instance; \texttt{a1\ \textless{}\textbar{}\textgreater{}\ a2} will create a new
\texttt{AutoOn} that feeds in its input to \emph{both} \texttt{a1} and
\texttt{a2}, and the result is the first \texttt{Just}.

\textasciitilde{}\textasciitilde{}\textasciitilde{}haskell -\/- source:
https://github.com/mstksg/inCode/tree/master/code-samples/machines/AutoOn.hs\#L80-86
instance Alternative (AutoOn a) where empty = AConsOn \$ \_ -\textgreater{}
(Nothing, empty) -\/- (\textless{}\textbar{}\textgreater{}) :: AutoOn a b
-\textgreater{} AutoOn a b -\textgreater{} AutoOn a b a1
\textless{}\textbar{}\textgreater{} a2 = AConsOn \$ \textbackslash{}x
-\textgreater{} let (y1, a1') = runAutoOn a1 x (y2, a2') = runAutoOn a2 x in (y1
\textless{}\textbar{}\textgreater{} y2, a1' \textless{}\textbar{}\textgreater{}
a2') \textasciitilde{}\textasciitilde{}\textasciitilde{}

Unexpectedly, we also get the handy \texttt{empty}, which is a "always off"
\texttt{AutoOn}. Feed anything through \texttt{empty} and it'll produce a
\texttt{Nothing} no matter what. You can use this to provide an "always fail",
"short-circuit here" kind of composition, like \texttt{Nothing} in the
\texttt{Maybe} monad.

You also get this an interesting and useful concept called "switching" that
comes from this; the ability to switch from running one Auto or the other by
looking if the result is on or off -\/-\/- here is a common switch that behaves
like the first \texttt{AutoOn} until it is off, and then behaves like the second
forever after:

\textasciitilde{}\textasciitilde{}\textasciitilde{}haskell -\/- source:
https://github.com/mstksg/inCode/tree/master/code-samples/machines/AutoOn.hs\#L115-121
(-\/-\textgreater{}) :: AutoOn a b -\textgreater{} AutoOn a b -\textgreater{}
AutoOn a b a1 -\/-\textgreater{} a2 = AConsOn \$ \textbackslash{}x
-\textgreater{} let (y1, a1') = runAutoOn a1 x in case y1 of Just \_
-\textgreater{} (y1, a1' -\/-\textgreater{} a2) Nothing -\textgreater{}
runAutoOn a2 x infixr 1 -\/-\textgreater{}
\textasciitilde{}\textasciitilde{}\textasciitilde{}

\subsection{Usages}

Let's test this out; first, some helper functions (the same ones we wrote for
\texttt{AutoM})

\textasciitilde{}\textasciitilde{}\textasciitilde{}haskell -\/- source:
https://github.com/mstksg/inCode/tree/master/code-samples/machines/AutoOn.hs\#L91-107
autoOn :: Auto a b -\textgreater{} AutoOn a b autoOn a = AConsOn \$
\textbackslash{}x -\textgreater{} let (y, a') = runAuto a x in (Just y, autoOn
a')

arrOn :: (a -\textgreater{} Maybe b) -\textgreater{} AutoOn a b arrOn f =
AConsOn \$ \textbackslash{}x -\textgreater{} (f x, arrOn f)

fromAutoOn :: AutoOn a b -\textgreater{} Auto a (Maybe b) fromAutoOn a = ACons
\$ \textbackslash{}x -\textgreater{} let (y, a') = runAutoOn a x in (y,
fromAutoOn a') \textasciitilde{}\textasciitilde{}\textasciitilde{}

\texttt{autoOn} turns an \texttt{Auto\ a\ b} into an \texttt{AutoOn\ a\ b},
where the result is always \texttt{Just}. \texttt{arrOn} is like \texttt{arr}
and \texttt{arrM}...it takes an \texttt{a\ -\textgreater{}\ Maybe\ b} and turns
it into an \texttt{AutoOn\ a\ b}. \texttt{fromAutoOn} turns an
\texttt{AutoOn\ a\ b} into a normal \texttt{Auto\ a\ (Maybe\ b)}, just so that
we can leverage our existing test functions on normal \texttt{Auto}s.

Let's play around with some test \texttt{AutoOn}s!

\textasciitilde{}\textasciitilde{}\textasciitilde{}haskell -\/- source:
https://github.com/mstksg/inCode/tree/master/code-samples/machines/AutoOn.hs\#L131-152
onFor :: Int -\textgreater{} AutoOn a a onFor n = proc x -\textgreater{} do i
\textless{}- autoOn summer -\textless{} 1 if i \textless{}= n then id
-\textless{} x -\/- succeed else empty -\textless{} x -\/- fail -\/-
alternatively, using explit recursion: -\/- onFor 0 = empty -\/- onFor n =
AConsOn \$ \textbackslash{}x -\textgreater{} (Just x, onFor' (n-1))

filterA :: (a -\textgreater{} Bool) -\textgreater{} AutoOn a a filterA p = arrOn
(\textbackslash{}x -\textgreater{} x \textless{}\$ guard (p x))

untilA :: (a -\textgreater{} Bool) -\textgreater{} AutoOn a a untilA p = proc x
-\textgreater{} do stopped \textless{}- autoOn (autoFold (\textbar{}\textbar{})
False) -\textless{} p x if stopped then empty -\textless{} x -\/- fail else id
-\textless{} x -\/- succeed -\/- alternatively, using explicit recursion: -\/-
untilA p = AConsOn \$ \textbackslash{}x -\textgreater{} -\/- if p x -\/- then
(Just x , untilA p) -\/- else (Nothing, empty )
\textasciitilde{}\textasciitilde{}\textasciitilde{}

One immediate usage is that we can use these to "short circuit" our proc blocks,
just like with monadic \texttt{Maybe} and do blocks:

\textasciitilde{}\textasciitilde{}\textasciitilde{}haskell -\/- source:
https://github.com/mstksg/inCode/tree/master/code-samples/machines/AutoOn.hs\#L163-173
shortCircuit1 :: AutoOn Int Int shortCircuit1 = proc x -\textgreater{} do
filterA even -\textless{} x onFor 3 -\textless{} () id -\textless{} x * 10

shortCircuit2 :: AutoOn Int Int shortCircuit2 = proc x -\textgreater{} do onFor
3 -\textless{} () filterA even -\textless{} x id -\textless{} x * 10
\textasciitilde{}\textasciitilde{}\textasciitilde{}

If either the \texttt{filterA} or the \texttt{onFor} are off, then the whole
thing is off. How do you think the two differ?

\textasciitilde{}\textasciitilde{}\textasciitilde{}haskell ghci\textgreater{}
testAuto (fromAutoOn shortCircuit1) {[}1..12{]} {[} Nothing, Just 20, Nothing,
Just 40, Nothing, Just 60 , Nothing, Nothing, Nothing, Nothing, Nothing,
Nothing{]} ghci\textgreater{} testAuto (fromAutoOn shortCircuit2) {[}1..12{]}
{[} Nothing, Just 20, Nothing, Nothing, Nothing, Nothing , Nothing, Nothing,
Nothing, Nothing, Nothing, Nothing{]}
\textasciitilde{}\textasciitilde{}\textasciitilde{}

Ah. For \texttt{shortCircuit1}, as soon as the \texttt{filterA} fails, it jumps
\emph{straight to the end}, short-circuiting; it doesn't bother "ticking along"
the \texttt{onFor} and updating its state!

The arguably more interesting usage, and the one that is more used in real
life{[}\^{}schead{]}, is the powerful usage of the switching combinator
\texttt{(-\/-\textgreater{})} in order to be able to combine multiple
\texttt{Auto}'s that simulate "stages"...an \texttt{Auto} can "do what it
wants", and then choose to "hand it off" when it is ready.

\textasciitilde{}\textasciitilde{}\textasciitilde{}haskell -\/- source:
https://github.com/mstksg/inCode/tree/master/code-samples/machines/AutoOn.hs\#L178-183
stages :: AutoOn Int Int stages = stage1 -\/-\textgreater{} stage2
-\/-\textgreater{} stage3 -\/-\textgreater{} stages where stage1 = onFor 2 . arr
negate stage2 = untilA (\textgreater{} 15) . autoOn summer stage3 = onFor 3 .
(pure 100 . filterA even \textless{}\textbar{}\textgreater{} pure 200)
\textasciitilde{}\textasciitilde{}\textasciitilde{}

\textasciitilde{}\textasciitilde{}\textasciitilde{}haskell ghci\textgreater{}
testAuto\_ (fromAutoOn stages) {[}1..15{]} {[} Just (-1), Just (-2) -\/- stage 1
, Just 3, Just 7, Just 12 -\/- stage 2 , Just 100, Just 200, Just 100 -\/- stage
3 , Just (-9), Just (-10) -\/- stage 1 , Just 11 -\/- stage 2 , Just 100, Just
200, Just 100 -\/- stage 3 , Just (-15), Just (-16) -\/- stage 1 {]}
\textasciitilde{}\textasciitilde{}\textasciitilde{}

Note that the stages continually "loop around", as our recursive definition
seems to imply. Neat!

\textless{}div class="note"\textgreater{} \textbf{Aside}

You might note that sometimes, to model on/off behavior, it might be nice to
really be able to "keep on counting" even when receiving a \texttt{Nothing} in a
composition. For example, you might want both versions of \texttt{shortCircuit}
to be the same -\/-\/- let \texttt{onFor} still "keep on counting" even when it
has been inhibited upstream.

If this is the behavior you want to model (and this is actually the behavior
modeled in some FRP libraries), then the type above isn't powerful enough;
you'll have to go deeper:

\textasciitilde{}\textasciitilde{}\textasciitilde{}haskell -\/- source:
https://github.com/mstksg/inCode/tree/master/code-samples/machines/AutoOn2.hs\#L10-10
newtype AutoOn2 a b = ACons2 \{ runAutoOn2 :: Maybe a -\textgreater{} (Maybe b,
AutoOn2 a b) \} \textasciitilde{}\textasciitilde{}\textasciitilde{}

So now, you can write something like \texttt{onFor}, which keeps on "ticking on"
even if it receives a \texttt{Nothing} from upstream:

\textasciitilde{}\textasciitilde{}\textasciitilde{}haskell -\/- source:
https://github.com/mstksg/inCode/tree/master/code-samples/machines/AutoOn2.hs\#L16-18
onFor :: Int -\textgreater{} AutoOn2 a a onFor 0 = ACons2 \$ \_ -\textgreater{}
(Nothing, onFor 0) onFor n = ACons2 \$ \textbackslash{}x -\textgreater{} (x,
onFor (n - 1)) \textasciitilde{}\textasciitilde{}\textasciitilde{}

You can of course translate all of your \texttt{AutoOn}s into this new type:

\textasciitilde{}\textasciitilde{}\textasciitilde{}haskell -\/- source:
https://github.com/mstksg/inCode/tree/master/code-samples/machines/AutoOn2.hs\#L24-31
autoOn :: AutoOn a b -\textgreater{} AutoOn2 a b autoOn a = ACons2 \$
\textbackslash{}x -\textgreater{} case x of Just \emph{x -\textgreater{} let (y,
a') = runAutoOn a }x in (y, autoOn a') Nothing -\textgreater{} (Nothing, autoOn
a) \textasciitilde{}\textasciitilde{}\textasciitilde{}

Or you can use the smart constructor method detailed immediately following.
\textless{}/div\textgreater{}

\section{Working all together}

Of course, we can always literally throw everything we can add together into our
\texttt{Auto} type:

\textasciitilde{}\textasciitilde{}\textasciitilde{}haskell -\/- source:
https://github.com/mstksg/inCode/tree/master/code-samples/machines/AutoX.hs\#L18-18
newtype AutoX m a b = AConsX \{ runAutoX :: a -\textgreater{} m (Maybe b, AutoX
m a b) \} \textasciitilde{}\textasciitilde{}\textasciitilde{}

(Again, instances are in the source file, but not here in the post directly)

Here is a same of a big conglomerate type where we throw in a bunch of things.

The benefit? Well, we could work and compose "normal" \texttt{Auto}s, selecting
for features that we only need to work with. And then, when we need to, we can
just "convert it up" to our "lowest common denominator" type.

This is the common theme, the "functor design pattern". Pick your common
unifying type, and just pop everything into it. You can compose, etc. with the
semantics of the other type when convenient, and then have all the parts work
together in the end.

This pattern is awesome, if only we didn't have so many types to convert in
between manually.

Well, we're in luck. There's actually a great trick, that makes \emph{all of
this} even more streamlined: we can replace the "normal constructors" like
\texttt{ACons}, \texttt{AConsM}, and \texttt{AConsOn}, with \emph{smart
constructors} \texttt{aCons}, \texttt{aConsM}, \texttt{aConsOn}, that work
\emph{exactly the same way}:

\textasciitilde{}\textasciitilde{}\textasciitilde{}haskell -\/- source:
https://github.com/mstksg/inCode/tree/master/code-samples/machines/AutoX.hs\#L85-100
aCons :: Monad m =\textgreater{} (a -\textgreater{} (b, AutoX m a b))
-\textgreater{} AutoX m a b aCons a = AConsX \$ \textbackslash{}x
-\textgreater{} let (y, aX) = a x in return (Just y, aX)

aConsM :: Monad m =\textgreater{} (a -\textgreater{} m (b, AutoX m a b))
-\textgreater{} AutoX m a b aConsM a = AConsX \$ \textbackslash{}x
-\textgreater{} do (y, aX) \textless{}- a x return (Just y, aX)

aConsOn :: Monad m =\textgreater{} (a -\textgreater{} (Maybe b, AutoX m a b))
-\textgreater{} AutoX m a b aConsOn a = AConsX \$ \textbackslash{}x
-\textgreater{} let (y, aX) = a x in return (y, aX)
\textasciitilde{}\textasciitilde{}\textasciitilde{}

Compare these definitions of \texttt{summer}, \texttt{arrM}, and \texttt{untilA}
from their "specific type" "real constructor" versions to their
\texttt{AutoX}-generic "smart constructor" versions:

\textasciitilde{}\textasciitilde{}\textasciitilde{}haskell -\/- source:
https://github.com/mstksg/inCode/tree/master/code-samples/machines/Auto.hs\#L67-73
summer :: Num a =\textgreater{} Auto a a summer = sumFrom 0 where sumFrom :: Num
a =\textgreater{} a -\textgreater{} Auto a a sumFrom n = ACons \$
\textbackslash{}input -\textgreater{} let s = n + input in ( s , sumFrom s )
-\/- source:
https://github.com/mstksg/inCode/tree/master/code-samples/machines/Auto3.hs\#L104-107
arrM :: Monad m =\textgreater{} (a -\textgreater{} m b) -\textgreater{} AutoM m
a b arrM f = AConsM \$ \textbackslash{}x -\textgreater{} do y \textless{}- f x
return (y, arrM f) -\/- source:
https://github.com/mstksg/inCode/tree/master/code-samples/machines/AutoOn.hs\#L154-158
untilA' :: (a -\textgreater{} Bool) -\textgreater{} AutoOn a a untilA' p =
AConsOn \$ \textbackslash{}x -\textgreater{} if p x then (Just x , untilA p)
else (Nothing, empty ) \textasciitilde{}\textasciitilde{}\textasciitilde{}

\textasciitilde{}\textasciitilde{}\textasciitilde{}haskell -\/- source:
https://github.com/mstksg/inCode/tree/master/code-samples/machines/AutoX.hs\#L106-128
summer :: (Monad m, Num a) =\textgreater{} AutoX m a a summer = sumFrom 0 where
sumFrom n = aCons \$ \textbackslash{}input -\textgreater{} let s = n + input in
( s , sumFrom s )

arrM :: Monad m =\textgreater{} (a -\textgreater{} m b) -\textgreater{} AutoX m
a b arrM f = aConsM \$ \textbackslash{}x -\textgreater{} do y \textless{}- f x
return (y, arrM f)

untilA :: Monad m =\textgreater{} (a -\textgreater{} Bool) -\textgreater{} AutoX
m a a untilA p = aConsOn \$ \textbackslash{}x -\textgreater{} if p x then (Just
x , untilA p) else (Nothing, empty )
\textasciitilde{}\textasciitilde{}\textasciitilde{}

They are literally exactly the same...we just change the constructor to the
smart constructor!

You might also note that we can express a "pure, non-Monadic" \texttt{Auto} in
\texttt{AutoM} and \texttt{AutoX} by making the type polymorphic over all
monads:

\textasciitilde{}\textasciitilde{}\textasciitilde{}haskell -\/- source:
https://github.com/mstksg/inCode/tree/master/code-samples/machines/AutoX.hs\#L106-106
summer :: (Monad m, Num a) =\textgreater{} AutoX m a a
\textasciitilde{}\textasciitilde{}\textasciitilde{}

An \texttt{Auto} with a type like this says, "I cannot perform any effects
during stepping" -\/-\/- and we know that \texttt{summer} definitely does not.
\texttt{summer} is statically guaranteed not to affect any state or IO, and it's
reflected in its type.

The takeaway? You don't even have to mungle around multiple types to make this
strategy work -\/-\/- just make all your \texttt{Auto}s from the start using
these smart constructors, and they all compose together! One type from the start
-\/-\/- we just expose different constructors to expose the different "subtypes
of power" we want to offer.

Now it's all just to chose your "greatest common denominator". If you don't want
inhibition-based semantics, just only use \texttt{AutoM}, for example!

By the way, here's a "smart constructor" for \texttt{AutoM}.

\textasciitilde{}\textasciitilde{}\textasciitilde{}haskell -\/- source:
https://github.com/mstksg/inCode/tree/master/code-samples/machines/Auto3.hs\#L112-113
aCons :: Monad m =\textgreater{} (a -\textgreater{} (b, AutoM m a b))
-\textgreater{} AutoM m a b aCons f = AConsM \$ \textbackslash{}x
-\textgreater{} return (f x) \textasciitilde{}\textasciitilde{}\textasciitilde{}

\section{Closing Remarks}

That was a doozy, wasn't it? For those of you who have been waiting, thank you
for being patient. I hope most if not all of you are still with me.

Hopefully after going through all of these examples, you can take away some
things:

\begin{itemize}
\item
  From the previous parts, you've recognized the power of local statefulness and
  the declarative style offered by proc notation.
\item
  From here, you've seen that the Auto type can be equipped in many ways to give
  it many features which have practical applications in the real world.
\item
  You've learned how to handle those features and use and manage them together
  in sane ways, and their limitations.
\item
  You also know that you can really "program", "compose", or "think" in any sort
  of Auto or composition semantics that you want, for any small part of the
  problem. And then at the end, just push them all into your greatest common
  denominator type. So, you aren't afraid to play with different effect types
  even in the same program!
\item
  You've seen the power of recursive bindings to make complete the promise of
  declarative programming -\/-\/- being able to extend the realm of what we can
  express "declaratively", and what we can \emph{denote}.
\item
  You are ready to really understand anything you encounter involving
  \texttt{Auto} and \texttt{Auto}-like entities.
\end{itemize}

So, what's next?

\begin{itemize}
\item
  \href{http://blog.jle.im/source/code-samples/machines}{Download the files of
  this post}, play along with the examples in this post, CTRL+F this page for
  "exercise" to find exercises, and try writing your own examples!
\item
  Feel ready to be able to have a grasp of the situation you see \texttt{Auto}
  in the real world, such as in the popular FRP library
  \href{http://hackage.haskell.org/package/netwire}{netwire}!
\item
  Apply it to the real world and your real world problems!
\item
  Well, a bit of self-promotion, my upcoming library
  \href{https://github.com/mstksg/auto.}{auto} is basically supposed to be
  almost all of these concepts (except for implicit on/off behavior) implemented
  as a finely tuned and optimized performant library, attached with semantic
  tools for working with real-world problems with these concepts of local
  statefulness, composition, and declarative style. You can really apply what
  you learned here to start building projects right away!

  Well, sorta. Unfortunately, as of Feburary 2015, it is not yet ready for real
  usage, and the API is still being finalized. But now that this post is
  finished, I will be posting more examples and hype posts in the upcoming weeks
  and months leading up to its official release.

  I am open to pull requests and help on the final stages of documentation :)

  If you're interested, or are curious, stop by \emph{\#haskell-auto} on
  freenode or send me a message!
\item
  Look forward to an actual series on Arrowized FRP, coming up soon! We'll be
  using the concepts in this series to \emph{implement FRP}.
\end{itemize}

Happy Haskelling!

\end{document}
